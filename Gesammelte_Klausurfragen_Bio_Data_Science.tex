\documentclass[a4paper, 9pt]{scrartcl}\usepackage[]{graphicx}\usepackage[]{xcolor}
% maxwidth is the original width if it is less than linewidth
% otherwise use linewidth (to make sure the graphics do not exceed the margin)
\makeatletter
\def\maxwidth{ %
  \ifdim\Gin@nat@width>\linewidth
    \linewidth
  \else
    \Gin@nat@width
  \fi
}
\makeatother

\definecolor{fgcolor}{rgb}{0.345, 0.345, 0.345}
\newcommand{\hlnum}[1]{\textcolor[rgb]{0.686,0.059,0.569}{#1}}%
\newcommand{\hlstr}[1]{\textcolor[rgb]{0.192,0.494,0.8}{#1}}%
\newcommand{\hlcom}[1]{\textcolor[rgb]{0.678,0.584,0.686}{\textit{#1}}}%
\newcommand{\hlopt}[1]{\textcolor[rgb]{0,0,0}{#1}}%
\newcommand{\hlstd}[1]{\textcolor[rgb]{0.345,0.345,0.345}{#1}}%
\newcommand{\hlkwa}[1]{\textcolor[rgb]{0.161,0.373,0.58}{\textbf{#1}}}%
\newcommand{\hlkwb}[1]{\textcolor[rgb]{0.69,0.353,0.396}{#1}}%
\newcommand{\hlkwc}[1]{\textcolor[rgb]{0.333,0.667,0.333}{#1}}%
\newcommand{\hlkwd}[1]{\textcolor[rgb]{0.737,0.353,0.396}{\textbf{#1}}}%
\let\hlipl\hlkwb

\usepackage{framed}
\makeatletter
\newenvironment{kframe}{%
 \def\at@end@of@kframe{}%
 \ifinner\ifhmode%
  \def\at@end@of@kframe{\end{minipage}}%
  \begin{minipage}{\columnwidth}%
 \fi\fi%
 \def\FrameCommand##1{\hskip\@totalleftmargin \hskip-\fboxsep
 \colorbox{shadecolor}{##1}\hskip-\fboxsep
     % There is no \\@totalrightmargin, so:
     \hskip-\linewidth \hskip-\@totalleftmargin \hskip\columnwidth}%
 \MakeFramed {\advance\hsize-\width
   \@totalleftmargin\z@ \linewidth\hsize
   \@setminipage}}%
 {\par\unskip\endMakeFramed%
 \at@end@of@kframe}
\makeatother

\definecolor{shadecolor}{rgb}{.97, .97, .97}
\definecolor{messagecolor}{rgb}{0, 0, 0}
\definecolor{warningcolor}{rgb}{1, 0, 1}
\definecolor{errorcolor}{rgb}{1, 0, 0}
\newenvironment{knitrout}{}{} % an empty environment to be redefined in TeX

\usepackage{alltt}
\usepackage[ngerman]{babel}

% -----------------------------------------------------------------------

% -----------------------------------------------------------------------
%% ------------------------------------------------------------
%% by J.Kruppa on Friday, February 11, 2022 (11:31)
%% \def\mainDir{\Sexpr{exam_path}}
\def\source{/Users/jokruppa/source/tex}
\usepackage[margin=2cm, includefoot]{geometry}
\setlength{\parindent}{0cm}
\usepackage{booktabs}
\usepackage{amsmath}
\usepackage{scalerel,amssymb}
\usepackage{setspace}
\def\csquare{{\Large $\boxtimes$}}
\def\msquare{{\Large $\square$}}
\usepackage[normalem]{ulem}
\usepackage{array}
\usepackage{xcolor}
\usepackage{float}
\usepackage{currfile}
\usepackage{tikz}
\usepackage[nomessages]{fp}

%% beamer defs
\def\lecture{Klausurfragen der Bio Data Science}

%% exam defs
\def\examtitle{\lecture}
\def\exammodule{
\vspace{-1.75cm}  
\begin{graybox}{}
\vspace{2Ex}
\textbf{\large Name:} \rule[0ex]{16.75em}{.4pt}
\hfill \textnormal{\textit{Nicht bestanden:}} \msquare \\[2.5Ex]
\textbf{\large Vorname:} \rule[0ex]{15em}{.4pt} \\[2.5Ex]
\textbf{\large Matrikelnummer:} \rule[0ex]{10.8em}{.4pt}
\hfill Endnote: \rule[0ex]{7em}{.4pt} 
\end{graybox}
\vspace{3Ex}
\phantom{text}
}
\def\examsemester{Sommersemester \& Wintersemester}
\def\examdate{\today}
%% ------------------------------------------------------------
\definecolor{darkblue}{rgb}{0,0,.5}
\definecolor{darkpurple}{rgb}{0.4117, 0.2, 0.4117}
\definecolor{uni}{rgb}{0,0.3137,0.6078}
\definecolor{gray}{gray}{0.7}

\usepackage{tcolorbox}
\definecolor{logo1}{RGB}{0, 158, 227}
\definecolor{gray5}{RGB}{247, 247, 247}
\definecolor{gray2}{RGB}{102, 102, 102}

\newtcolorbox{graybox}[1]{
  colback=gray5,%%red!5!white,
  colframe=gray2,%%red!75!black,
  fonttitle=\bfseries\Large,
  %%valign=center,
  fontupper=\large,
  before skip=10pt plus 2pt,
  after skip=20pt plus 4pt,
  title=#1}

\newtcolorbox{takehomebox}[1]{
  colback=gray5,%%red!5!white,
  colframe=logo1,%%red!75!black,
  fonttitle=\bfseries\Large,
  %%valign=center,
  fontupper=\large,
  before skip=10pt plus 2pt,
  after skip=10pt plus 2pt,
  title=#1}

\def\Rlogo{\includegraphics[width = 0.5cm]{\string~/Documents/GitHub/exam/img/Rlogo}\;}

\usepackage[scaled=.90]{helvet} 
\usepackage{fancyhdr}
\usepackage{lastpage}
\usepackage{hyperref}
\hypersetup{
    colorlinks=true,       % false: boxed links; true: colored links
    linkcolor=black,          % color of internal links 
    urlcolor=magenta           % color of external links
}
\renewcommand{\familydefault}{\sfdefault}

\title{
\large \exammodule \\[5Ex]
\Huge \examtitle \\[2Ex] 
\Large Hochschule Osnabr{\"u}ck
}
\author{Pr{\"u}fer: Prof. Dr. Jochen Kruppa \\
Fakult{\"a}t f{\"u}r Agrarwissenschaften und Landschaftsarchitektur \\ 
j.kruppa@hs-osnabrueck.de}
\date{Version vom \examdate}

%% ------------------------------------------------------------
%% by J.Kruppa on Tuesday, September 23, 2014 (12:50)
%% Header
\renewcommand{\headrulewidth}{0pt}
\renewcommand{\footrulewidth}{0pt}
\pagestyle{fancy}

\fancyhf{}
\fancyhead[L]{}
\fancyhead[R]{}
\fancyfoot[R]{\thepage}
\fancyfoot[L]{\footnotesize \examtitle}

\fancypagestyle{empty}{
 \fancyhf{}
 \fancyhead[L]{}
 \fancyhead[R]{}
 \fancyfoot[R]{\thepage}
 \fancyfoot[L]{\footnotesize \examtitle}
}

\usepackage{arevtext,arevmath}

\newcommand\Tstrut{\rule{0pt}{2.6ex}}         % = `top' strut
\newcommand\Bstrut{\rule[-0.9ex]{0pt}{0pt}}   % = `bottom' strut
\def\strut{\Tstrut\Bstrut}

% -----------------------------------------------------------------------
\IfFileExists{upquote.sty}{\usepackage{upquote}}{}
\begin{document}
\date{Wintersemester 2024/25 
\vfill
\begin{center}
\includegraphics[width = 1.9cm]{avatare/Alex}\hspace{-8mm}
\includegraphics[width = 1.9cm]{avatare/Jessica}\hspace{-8mm}
\includegraphics[width = 1.9cm]{avatare/Jonas}\hspace{-8mm}
\includegraphics[width = 1.9cm]{avatare/Mark}\hspace{-8mm}
\includegraphics[width = 1.9cm]{avatare/Nilufar}\hspace{-8mm}
\includegraphics[width = 1.9cm]{avatare/Paula}\hspace{-8mm}
\includegraphics[width = 1.9cm]{avatare/Steffen}\hspace{-8mm}
\includegraphics[width = 1.9cm]{avatare/Tina}\hspace{-8mm}
\includegraphics[width = 1.9cm]{avatare/Yuki}\\
\small
\vspace{1.5Ex}
\textit{"`The test of a student is not how much he knows,\\ but how much he wants to know."'\\ --- Alice W. Rollins}
\end{center}}
% -----------------------------------------------------------------------
\maketitle
\fancypagestyle{empty}{
  \fancyfoot[L]{\tiny $\blacksquare\!\blacksquare\!\square\!\square\!\blacksquare\!\square\!\square\!\square\!\blacksquare\!\square\!\square\!\square\!\blacksquare\!\square\!\square\!\square\!\blacksquare\!\blacksquare\!\square\!\blacksquare$}
}
\thispagestyle{empty}
\clearpage
% -----------------------------------------------------------------------

\begin{graybox}{Erlaubte Hilfsmittel}
  \vspace{1Ex}
  \begin{itemize}
  \item Normaler Taschenrechner ohne Möglichkeit der Kommunikation mit anderen
    Geräten! Ausdrücklich kein Handy!
  \item Eine DIN A4-Seite als beidseitig, selbstgeschriebene,
    handschriftliche Formelsammlung. Keine digitalen Ausdrucke! 
  \item \textbf{\textcolor{red}{Die Verwendung eines roten Farbstiftes ist nicht gestattet! Korrekturfarbe!}}
  \item \textit{You can answer the questions in English without any consequences.}  
  \end{itemize}
\end{graybox}
\vfill

\begin{graybox}{Endnote}
  \vspace{1Ex}
  \begin{itemize}
  \item[] \rule[0ex]{3em}{.4pt}\, von 20\, Punkten sind aus den Multiple
    Choice Aufgaben erreicht.
  \item[] \rule[0ex]{3em}{.4pt}\, von 80 Punkten sind aus den Rechen- und
    Textaufgaben erreicht. 
  \item[] \rule[0ex]{3em}{.4pt}\, von 100 Punkten in Summe.
  \item[] Es wird folgender Notenschlüssel angewendet.   
  \end{itemize}
  \vspace{1ex}
\begin{center}
  \begin{tabular}[c]{cc}
    \toprule
    \textbf{Punkte}	&	\textbf{Note}	\\
    \midrule
    95.5 - 10.0	&	1,0	\\
    90.5 - 95.0	&	1,3	\\
    85.5 - 90.0	&	1,7	\\
    80.5 - 85.0	&	2,0	\\
    75.5 - 80.0	&	2,3	\\
    70.5 - 75.0	&	2,7	\\
    65.5 - 70.0	&	3,0	\\
    60.5 - 65.0	&	3,3	\\
    55.5 - 60.0	&	3,7	\\
    50.0 - 55.0	&	4,0	\\
    \bottomrule
  \end{tabular}
\end{center}
  \vspace{1ex}
\begin{itemize}
\item[] Es ergibt sich eine Endnote von \rule[0ex]{4em}{.4pt}.
\end{itemize}
  \vspace{1Ex}
\end{graybox}

% -----------------------------------------------------------------------
\newpage
% -----------------------------------------------------------------------

\begin{graybox}{Multiple Choice Aufgaben}
  \begin{itemize}
  \item Pro Multipe Choice Frage ist \emph{genau} eine Antwort richtig.
  \item \textbf{Übertragen Sie Ihre Kreuze in die Tabelle auf
      dieser Seite.}
  \end{itemize}

\begin{center}
  \large
  \begin{tabular}{|l|c|c|c|c|c?c|}
    \hline
    & \textbf{A} & \textbf{B} & \textbf{C} & \textbf{D} & \textbf{E} & $\boldsymbol{\checkmark}$\strut\\
    \hline
    \textbf{Aufgabe 1} &   &   &   &   &   & \strut\\
    \hline
    \textbf{Aufgabe 2} &   &   &   &   &   & \strut\\
    \hline
    \textbf{Aufgabe 3} &   &   &   &   &   & \strut\\
    \hline
    \textbf{Aufgabe 4} &   &   &   &   &   & \strut\\
    \hline
    \textbf{Aufgabe 5} &   &   &   &   &   & \strut\\
    \hline
    \textbf{Aufgabe 6} &   &   &   &   &   & \strut\\
    \hline
    \textbf{Aufgabe 7} &   &   &   &   &   & \strut\\
    \hline
    \textbf{Aufgabe 8} &   &   &   &   &   & \strut\\
    \hline
    \textbf{Aufgabe 9} &   &   &   &   &   & \strut\\
    \hline
    \textbf{Aufgabe 10} &   &   &   &   &   & \strut\\
    \hline
  \end{tabular}
\end{center}

\begin{itemize}
\item Es sind \rule[0ex]{2em}{.4pt}\, von 20 Punkten erreicht worden.
\end{itemize}
\end{graybox}

\vfill

\begin{graybox}{Rechen- und Textaufgaben}
  \begin{center}
    \large
    \begin{tabular}{|l|c|c|c|c|c|c|c|}
      \hline
      \textbf{Aufgabe} & \textbf{11} & \textbf{12} & \textbf{13} & \textbf{14} & \textbf{15} & \textbf{16} & \textbf{17} \strut\\
      \hline
      \textbf{Punkte} & 
      \hspace{1Ex}\Large\textcolor{gray!70}{8}\hspace{1Ex}  & 
      \hspace{1Ex}\Large\textcolor{gray!70}{20}\hspace{1Ex}  & 
      \hspace{1Ex}\Large\textcolor{gray!70}{9}\hspace{1Ex}  & 
      \hspace{1Ex}\Large\textcolor{gray!70}{12}\hspace{1Ex}  & 
      \hspace{1Ex}\Large\textcolor{gray!70}{9}\hspace{1Ex}  & 
      \hspace{1Ex}\Large\textcolor{gray!70}{10}\hspace{1Ex}  & 
      \hspace{1Ex}\Large\textcolor{gray!70}{12}\hspace{1Ex} \strut\\
      \hline
  \end{tabular}
\end{center}
\begin{itemize}
\item Es sind \rule[0ex]{2em}{.4pt}\, von 80 Punkten erreicht worden.
\end{itemize}
\end{graybox}

% -----------------------------------------------------------------------
\clearpage
% -----------------------------------------------------------------------
\begin{graybox}{Multiple Choice Aufgaben}
Die Multiple Choice Aufgaben \textcolor{red}{unterliegen dem Zufall}. Die Reihenfolge der Antworten ist zufällig. Die Fragen und Antworten sind semantisch zufällig und haben somit \textcolor{red}{verschiedene Textvarianten}. Insbesondere die reinen Textaufgaben haben verschiedene Textvarianten. Die Semeantik mag sich unterscheiden, die Inhalte sind aber gleich.
\end{graybox}
\section*{ANOVA}

\section{Aufgabe \hfill (2 Punkte)}





Nach der Berechnung einer einfaktoriellen ANOVA ergibt sich ein $\eta^2 = 0.12$. Welche Aussage ist richtig?



\begin{enumerate}
\item [\textbf{A} \msquare] Die Berechnung von $\eta^2$ ist ein Wert f{"u}r die Interaktion.
\item [\textbf{B} \msquare] Das $\eta^2$ beschreibt den Anteil der Varianz, der von den Behandlungsbedingungen nicht erkl{"a}rt wird. Somit der Rest an nicht erkl{"a}rbarer Varianz.
\item [\textbf{C} \msquare] Das $\eta^2$ ist die Korrelation der ANOVA. Mit der Ausnahme, dass 0 der beste Wert ist.
\item [\textbf{D} \msquare] Das $\eta^2$ ist damit mit dem $R^2$ aus der linearen Regression zu vergleichen und beschreibt den Anteil der Varianz, der von den Behandlungsbedingungen erkl{"a}rt wird.
\item [\textbf{E} \msquare] Das $\eta^2$ ist ein Wert f{"u}r die G{"u}te der ANOVA. Je kleiner desto besser. Ein $\eta^2$ von 0 bedeutet ein perfektes Modell mit keiner Abweichung. Die Varianz ist null.
\end{enumerate} 

\section{Aufgabe \hfill (2 Punkte)}



Sie führen ein Feldexperiment durch um das Gewicht von Maiss zu
steigern. Die Pflanzen wachsen unter einer Kontrolle und zwei verschiedenen
Behandlungsbedingungen. Nach der Berechnung einer einfaktoriellen ANOVA
ergibt sich ein $\eta^2 = 0.2$. Welche Aussage ist richtig?



\begin{enumerate}
\item [\textbf{A} \msquare] Mit dem $\eta^2$ lässt sich auf die Qualität der Randomisierung und damit der Strukturgleichheit zwischen der Grundgesamtheit und der Stichprobe schließen. Es gilt dabei die Regel, dass ein $\eta^2$-Wert von 1 zu bevorzugen ist.
\item [\textbf{B} \msquare] Das $\eta^2$ beschreibt den Anteil der Varianz, der durch den Forschenden entsteht. Es gilt die Regel, dass ca. 70\% der Varianz eines Versuches durch die Versuchsdurchführung entstehen sollen.
\item [\textbf{C} \msquare] Es werden 80\% der Varianz durch die Behandlung erklärt. Das $\eta^2$ beschreibt den Anteil der Varianz, der von den unterschiedlichen Behandlungsbedingungen nicht erklärt wird.
\item [\textbf{D} \msquare] Das $\eta^2$ beschreibt den Anteil der Varianz, der von den Umweltbedingungen erklärt wird. Daher werden 20\% der Varianz durch die Umweltbedingungen erklärt. Der Anteil der Varianz durch die Behandlungsgruppen ist dann 80\%.
\item [\textbf{E} \msquare] Das $\eta^2$ beschreibt den Anteil der Varianz, der von den Behandlungsbedingungen erklärt wird. Daher werden 20\% der Varianz durch die Behandlungsgruppen erklärt.
\end{enumerate} 

\section{Aufgabe \hfill (2 Punkte)}



Sie rechnen eine einfaktorielle ANOVA und erhalten eine Teststatistik. Nun müssen Sie diese Teststatistik interpretieren. Welche Aussage ist richtig?



\begin{enumerate}
\item [\textbf{A} \msquare] Wenn die F-Statistik kleiner als der kritische Wert ist kann die Nullhypothese nicht abgelehnt werden. Die F-Statistik ist der Quotient der MS der Behandlung durch die MS des Fehlers.
\item [\textbf{B} \msquare] Wenn die F-Statistik höher ist als der kritische Wert kann die Nullhypothese nicht abgelehnt werden. Die F-Statistik ist die Differenz der MS der Behandlung durch die MS des Fehlers.
\item [\textbf{C} \msquare] Die ANOVA berechnet die F-Statistik indem die MS des Fehlers durch die MS der Behandlung geteilt werden. Wenn die F-Statistik sich der 0 ann{"a}hert kann die Nullhypothese abgelehnt werden.
\item [\textbf{D} \msquare] Die ANOVA berechnet die F-Statistik indem die MS des Fehlers durch die MS der Behandlung geteilt werden. Wenn die F-Statistik sich der 1 ann{"a}hert kann die Nullhypothese nicht abgelehnt werden.
\item [\textbf{E} \msquare] Die ANOVA berechnet die T-Statistik aus der Multiplikation der MS Behandlung mit der MS der Fehler. Wenn die F-Statistik genau 0 ist, kann die Nullhypothese nicht abgelehnt werden.
\end{enumerate} 

\section{Aufgabe \hfill (2 Punkte)}




Viele statistische Verfahren nutzen eine Teststatistik um eine Aussage über den Zusammenhang zwischen der Grundgesamthat und der Stichprobe abzubilden. Ein statistisches Testwerkzeug ist hierbei die ANOVA. Die ANOVA rechnet dabei...



\begin{enumerate}
\item [\textbf{A} \msquare] ... den Unterschied zwischen der Varianz über alle Behandlungsgruppen oder der Varianz aus verschiedenen Behandlungsguppen. Wenn die ANOVA signifikant ist, muss sich zwischen einem der beiden Varianzquellen entschieden werden.
\item [\textbf{B} \msquare] ... den Unterschied zwischen der Varianz aus verschiedenen Behandlungsguppen und der Varianz über alle Behandlungsgruppen. Wenn die ANOVA signifikant ist, muss über einen Posthoc-Test nachgedacht werden um den signifikanten Unterschied in den Gruppen exakt zu bestimmen.
\item [\textbf{C} \msquare] ... den Unterschied zwischen der Varianz in den verschiedenen Behandlungsguppen und der Varianz in einer der Behandlungsgruppen. Wenn die ANOVA signifikant ist, muss über einen Posthoc-Test nachgedacht werden um den signifikanten Unterschied in einer der Gruppen exakt zu bestimmen.
\item [\textbf{D} \msquare] ... den Unterschied zwischen mehreren Varianzen aus verschiedenen Behandlungsguppen. Wenn die ANOVA signifikant ist, ist nicht bekannt welcher Vergleich konkret unterschiedlich ist.
\item [\textbf{E} \msquare] ... den Unterschied zwischen der Varianz ausgelöst durch alle Behandlungsgruppen und der Varianz aus globalen Behandlungsguppen der Kontrollen. Wenn die ANOVA nicht signifikant ist, muss ein Posthoc-Test ausgeschlossen werden.
\end{enumerate} 

\section{Aufgabe \hfill (2 Punkte)}



Die folgende Abbildung enthält die Daten aus einer Studie zur Bewertung der Wirkung von Vitamin C auf das Zahnwachstum bei Hasen. Der Versuch wurde an 61 Tieren durchgeführt, wobei jedes Tier eine von  drei Vitamin-C-Dosen (0.5, 1 und 1.5 mg/Tag) über eine von zwei Verabreichungsmethoden erhielt. Welche Aussage ist richtig im Bezug auf eine zweifaktorielle ANOVA?



{\centering \includegraphics[width=\maxwidth]{img/mc-anova-02-a-1} 

}







\begin{enumerate}
\item [\textbf{A} \msquare] Die Koeffizienten sind negativ $(\beta_0 < 0; \beta_1 < 0)$.
\item [\textbf{B} \msquare] Eine mittlere bis starke Interaktion liegt vor $(p \leq 0.05)$
\item [\textbf{C} \msquare] Eine positive Interaktion liegt vor $(\rho \leq -0.5)$ 
\item [\textbf{D} \msquare] Das Bestimmtheitsmaß $R^2$ ist klein.
\item [\textbf{E} \msquare] Keine Interaktion liegt vor $(p \leq 0.05)$.
\end{enumerate} 
\section*{Deskriptive Statistik \& Explorative Datenanalyse}

\section{Aufgabe \hfill (2 Punkte)}




Berechnen Sie den Mittelwert und Standardabweichung von $y$ mit 12, 8, 10, 11 und 13.



\begin{enumerate}
\item [\textbf{A} \msquare] Es berechnet sich 11.8 +/- 3.7
\item [\textbf{B} \msquare] Es berechnet sich 10.8 +/- 1.92
\item [\textbf{C} \msquare] Es berechnet sich 10.8 +/- 3.7
\item [\textbf{D} \msquare] Es ergibt sich 9.8 +/- 1.85
\item [\textbf{E} \msquare] Sie erhalten 10.8 +/- 1.39
\end{enumerate} 

\section{Aufgabe \hfill (2 Punkte)}




Berechnen Sie den Median, das $1^{st}$ Quartile sowie das $3^{rd}$ Quartile von $y$ mit 23, 14, 24, 33, 30, 25 und 63.




\begin{enumerate}
\item [\textbf{A} \msquare] Es ergibt sich 25 +/- 23
\item [\textbf{B} \msquare] Sie erhalten 25 [21; 31]
\item [\textbf{C} \msquare] Es berechnet sich 30 [24; 34]
\item [\textbf{D} \msquare] Es ergibt sich 25 [23; 33]
\item [\textbf{E} \msquare] Es berechnet sich 26 [24; 32]
\end{enumerate} 

\section{Aufgabe \hfill (2 Punkte)}



Die empfohlene Mindestanzahl an Beobachtungen für die Visualisierung mit einem Boxplot sind...



\begin{enumerate}
\item [\textbf{A} \msquare] 10 Beobachtungen.
\item [\textbf{B} \msquare] Die Mindestanzahl liegt bei fünf Beobachtungen.
\item [\textbf{C} \msquare] Wir sollten zwei bis fünf Beobachtungen mindestens pro Gruppe vorliegen haben.
\item [\textbf{D} \msquare] Die untere Grenze liegt bei einer Beobachtung.
\item [\textbf{E} \msquare] 1 Beobachtung.
\end{enumerate}

\section{Aufgabe \hfill (2 Punkte)}



Die Standardabweichung ist eine bedeutende deskriptive Statistik für die Analyse von Daten. Wie müssen Sie vorgehen um die Standardabweichung zu berechnen?



\begin{enumerate}
\item [\textbf{A} \msquare] Als erstes berechnen wir den Mittelwert. Dann bilden wir die Summe der quadratischen Abstände zu dem Mittelwert. Abschließend subtrahieren wir die Fallzahl.
\item [\textbf{B} \msquare] Als erstes berechnen wir den Mittelwert. Dann bilden wir die Summe der quadratischen Abstände zu dem Mittelwert. Abschließend teilen wir durch die Fallzahl.
\item [\textbf{C} \msquare] Wir berechnen erst den Mittelwert und dann die quadratischen Abstände zu dem Mittelwert. Diese quadratischen Abstände summieren wir auf und teilen am Ende durch die Fallzahl. Als letzten Schritt ziehen wir die quadratische Wurzel.
\item [\textbf{D} \msquare] Wir berechnen erst den Mittelwert und dann die absoluten Abstände zu dem Mittelwert. Diese quadratischen Abstände summieren wir auf und teilen am Ende durch die Fallzahl.
\item [\textbf{E} \msquare] Den Mittelwert berechen, dann die absoluten Abstände zum Mittelwert aufsummieren
\end{enumerate} 

\section{Aufgabe \hfill (2 Punkte)}



In Ihrer Abschlußarbeit wolllen Sie Ihre Daten für den Ertrag in einem Barplot darstellen. Sie nutzen den Barplot auch, da der Barplot zu den meist genutzten Visualiserungen von Daten gehört. Welche statistischen Maßzahlen stellt der Barplot dar?

 



\begin{enumerate}
\item [\textbf{A} \msquare] Durch die Abbildung des Barplot erhalten wir die Informationen über den Median und die Quartile.
\item [\textbf{B} \msquare] Der Barplot stellt den Median und die Streuung dar.
\item [\textbf{C} \msquare] Durch die Abbildung des Barplot erhalten wir die Informationen über die Mittelwerte und die Varianz.
\item [\textbf{D} \msquare] Den Mittelwert und die Varianz.
\item [\textbf{E} \msquare] Der Barplot stellt die Mittelwerte und die Standardabweichung dar.
\end{enumerate}

\section{Aufgabe \hfill (2 Punkte)}



Der Mittelwert $\bar{y}$ und der Median $\tilde{y}$ unterscheiden sich in Ihren Feldexperiment zu Leistungssteigerung von Erbsen.  Welche Aussage ist richtig?



\begin{enumerate}
\item [\textbf{A} \msquare] Da sich der Mittelwert und der Median unterscheiden, liegen vermutlich keine Outlier in den Daten vor. Wir verweden den Datensatz so wie er ist.
\item [\textbf{B} \msquare] Da sich der Mittelwert und der Median unterscheiden, ist der Datensatz nicht zu verwenden. Mittelwert und Median müssen gleich sein.
\item [\textbf{C} \msquare] Wenn sich der Mittelwert und der Median unterscheiden, liegen vermutlich Outlier in den Daten vor.
\item [\textbf{D} \msquare] Wenn sich der Mittelwert und der Median unterscheiden, liegen vermutlich keine Outlier in den Daten vor.
\item [\textbf{E} \msquare] Der Mittelwert und der Median sollten sich unterscheiden sein, wenn Outlier in den Daten vorliegen. 
\end{enumerate}

\section{Aufgabe \hfill (2 Punkte)}



Ihre Betreuung der Abschlussarbeit fragt überraschend in der letzten Besprechung, ob Ihre Messwerte einer Normalverteilung genügen. Sonst könnten Sie ja gar nicht einen t-Test rechnen. Da Ihnen die Zeit wegrennt, entscheiden Sie sich für eine schnelle Visualisierung im Anhang. Welche Visualisierung nutzen Sie und welche Regel kommt zur Abschätzung einer Normalverteilung zur Anwendung?



\begin{enumerate}
\item [\textbf{A} \msquare] Wir erstellen uns für jede Behandlung einen Dotplot und schauen, ob die Dots und damit die Varianz für jede Behandlung gleich groß sind.
\item [\textbf{B} \msquare] In einer explorativen Datanalyse nutzen wir den Violinplot. Dabei sollte der Bauch am Rand liegen. Dann können wir von einer Normalverteilung ausgehen.
\item [\textbf{C} \msquare] Einen Violinplot. Der Bauch der Violine muss hierbei einen höhren Wert annehmen als der Steg der Violine. Dann kann die Annahme einer Normalverteilung angenommen werden.
\item [\textbf{D} \msquare] In einer explorativen Datanalyse nutzen wir den Boxplot. Dabei sollte der Median als dicke Linie in der Mitte der Box liegen. Dann können wir von einer Normalverteilung ausgehen.
\item [\textbf{E} \msquare] Nach dem Einlesen der Daten nutzen wir einen Boxplot um zu schauen, ob alle Boxen über alle Behandlungen in etwa gleich groß sind. Damit ist dann auch das IQR in allen Behandlungen in etwa gleich.
\end{enumerate}

\section{Aufgabe \hfill (2 Punkte)}




Nach der Durchführung Ihres Feldexperiments wollen Sie eine ANOVA rechnen. Dafür muss aber Ihr Messwert zumindestens approximativ einer Normalverteilung folgen. Welche der drei Abbildungen erlaubt Ihnen abzuschätzen, ob Sie eine Normalverteilung in Ihrem Endpunkt vorliegen haben?





\begin{enumerate}
\item [\textbf{A} \msquare] Scatterplot, Densityplot, Barplot
\item [\textbf{B} \msquare] Barplot, Mosaicplot, Violinplot
\item [\textbf{C} \msquare] Boxplot, Violinplot, Mosaicplot
\item [\textbf{D} \msquare] Scatterplot, Mosaicplot, Boxplot
\item [\textbf{E} \msquare] Boxplot, Densityplot, Violinplot
\end{enumerate} 

\section{Aufgabe \hfill (2 Punkte)}



In dem folgenden Histogramm von $n = 196$ Pflanzen ist welche Verteilung abgebildet?



{\centering \includegraphics[width=\maxwidth]{img/mc-distribution-02-a-1} 

}







\begin{enumerate}
\item [\textbf{A} \msquare] Es handelt sich um eine Normalverteilung.
\item [\textbf{B} \msquare] Wir haben eine Gammaverteilung vorliegen.
\item [\textbf{C} \msquare] Eine Standardnormalverteilung.
\item [\textbf{D} \msquare] Wir haben eine Poisson-Verteilung vorliegen.
\item [\textbf{E} \msquare] Eine multivariate Normalverteilung.
\end{enumerate} 
\section*{Lineare Regression \& Korrelation}

\section{Aufgabe \hfill (2 Punkte)}



Im Allgemeinen gibt es zwei mögliche Ziele für ein Regressionsmodell. Wir können eine Vorhersagemodell oder ein kausales Modell rechnen. Welche Aussage ist für ein kausales Modell richtig?



\begin{enumerate}
\item [\textbf{A} \msquare] Wir modellieren den Zusammenhang zwischen $X$ und $Y$ wenn ein kausales Modell rerechnet wird. Dabei kann der gesamte Datensatz genutzt werden. Eine Aufteilung wie in einem prädiktiven Modell ist nicht notwendig.
\item [\textbf{B} \msquare] Es wird ein Trainingsdatensatz zum Modellieren des Trainingsmodells benötigt. Der Testdatensatz dient rein zur Visualisierung. Dies gilt vor allem für ein kausales Modell.
\item [\textbf{C} \msquare] Ein kausales Modell basiert auf einem Traingsdatensatz und einem Testdatensatz. Auf dem Trainingsdatensatz wird das Modell trainiert und auf dem Testdatensatz validiert.
\item [\textbf{D} \msquare] Ein kausales Modell schliesst grundsätzlich lineare Modell aus. Es muss ein Graph gefunden werden, der alle Punkte beinhaltet. Erst dann kann das $R^2$ berechnet werden.
\item [\textbf{E} \msquare] Ein kausales Modell wird auf einem Trainingsdatensatz trainiert und anschliessend über eine explorative Datenanalyse validiert. Signifikanzen über $\beta_i$ können hier nicht festgestellt werden.
\end{enumerate}

\section{Aufgabe \hfill (2 Punkte)}



Nach einer Regressions sollten die Residuen normalverteilt sein. Was bei einer simplen Regression noch relativ einfach visuell in einem Scatterplot zu überprüfen ist. Für komplexere Modell liefert der QQ-Plot die notwendigen Informationen über die Normalverteilung. Welche Aussage ist richtig?



{\centering \includegraphics[width=\maxwidth]{img/mc-regression-05-a-1} 

}







\begin{enumerate}
\item [\textbf{A} \msquare] Wir betrachten insbesondere die beiden Enden der Gerade. Der Rest ist mehr oder minder egal, dann ist die Annahme an die Normalverteilung der Residuen erfüllt.
\item [\textbf{B} \msquare] Wir betrachten die Gerade, die durch die einzelnen Punkte laufen sollte. Wenn die 95\% der Punkte von der Geraden getroffen werden, dann gehen wir von normalverteilten Residuen aus.
\item [\textbf{C} \msquare] Die Annahme der normalverteilten Residuen ist erfüllt. Die Punkte liegen zum überwiegenden Teil nicht auf der Geraden und Korrelation ist negativ.
\item [\textbf{D} \msquare] Die Annahme der normalverteilten Residuen ist nicht erfüllt. Die Punkte liegen zum überwiegenden Teil auf der Geraden.
\item [\textbf{E} \msquare] Wir betrachten die Punkte auf der Geraden. Wenn die Punkte einigermaßen auf der Geraden liegen, dann gehen wir von normalverteilten Residuen aus. Wir können hier von normalverteilten Residuen ausgehen.
\end{enumerate}

\section{Aufgabe \hfill (2 Punkte)}



Sie rechnen eine linearen Regression und erhalten folgende Abbildung der Residuen (\texttt{.resid}). Welche Aussage ist richtig?



{\centering \includegraphics[width=\maxwidth]{img/mc-regression-06-a-1} 

}







\begin{enumerate}
\item [\textbf{A} \msquare] Wenn wir die Nulllinie betrachten so liegen die Punkte nicht gleichmäßig über und unter der Nulllinie. Unser Modell erfüllt nicht die Annahme von normalverteilten Residuen mit einem Mittelwert von 0 und einer Streuung von $s^2$.
\item [\textbf{B} \msquare] Die Annahme der normalverteilten Residuen ist erfüllt. Kein Muster ist zu erkennen und keine Outlier zu beobachten.
\item [\textbf{C} \msquare] Wenn wir die Nulllinie betrachten so müssen die Punkte gleichmäßig über der Nulllinie liegen. Unser Modell erfüllt somit nicht die Annahme von normalverteilten Residuen mit einem Mittelwert von $>0$ und einer Streuung von $s$.
\item [\textbf{D} \msquare] Die Annahme der normalverteilten Residuen ist nicht erfüllt. Es ist kein Muster zu erkennen.
\item [\textbf{E} \msquare] Die Annahme der normalverteilten Residuen ist erfüllt. Es ist ein Muster zu erkennen und wir können damit auf die Signifkanz von $x_1, ..., x_p$ schließen.
\end{enumerate}

\section{Aufgabe \hfill (2 Punkte)}




In den Humanwissenschaften wird der Korrelationskoeffizienten $\rho$ sehr häufig verwendet. Daher ist es auch wichtig für andere Forschende den Korrelationskoeffizienten $\rho$ zu verstehen. Welche Aussazu zu dem Korrelationskoeffizienten $\rho$ ist richtig?




\begin{enumerate}
\item [\textbf{A} \msquare] Der Korrelationskoeffizienten $\rho$ zeigt keinen Zusammenhang zwischen zwei Variablen $x$ und $y$ bei einem Wert von 0. Einen negativen Zusammenhang Richtung -1 und somit auch einen positiven Zusammenhang Richtung 1. Je größer die Zahl allgemein, desto stärker der Effekt.
\item [\textbf{B} \msquare] Der Korrelationskoeffizienten $\rho$ wird wie das $\eta^2$ aus der ANOVA interpretiert. Der Korrelationskoeffizienten $\rho$ beschreibt den Anteil an erklärter Varianz durch die Regression. Dabei gibt er jedoch eine Richtung an und kann auch negativ werden.
\item [\textbf{C} \msquare] Der Korrelationskoeffizienten $\rho$ zeigt keinen Zusammenhang zwischen zwei Variablen $x$ und $y$ bei einem Wert von 0. Einen maximalen negativen Zusammenhang bei -1 und somit auch einen maximalen positiven Zusammenhang bei 1. Korrelationskoeffizienten $\rho$ ist einheitslos.
\item [\textbf{D} \msquare] Der Korrelationskoeffizienten $\rho$ ist eine veraltete Darstellungsform von Effekten in der linearen Regression und wird wie das $\eta^2$ aus der ANOVA interpretiert. Der Korrelationskoeffizienten $\rho$ beschreibt den Anteil an erklärter Varianz durch die Regression.
\item [\textbf{E} \msquare] Der Korrelationskoeffizienten $\rho$ liegt zwischen -1 und 1. Darüber hinaus ist der Korrelationskoeffizienten $\rho$ als standardisierte Steigung zu verstehen, wenn eine Standardisierung durchgeführt wurde. Diese Adjustierung nach Fischer muss am Anschluß der Berechnung der Korrelation durchgeführt werden.
\end{enumerate}

\section{Aufgabe \hfill (2 Punkte)}



Nach einer simplen linearen Regression zur Untersuchung vom Einfluss der $Fe_3O_4$-Konzentration in [$\mu g$] im Wasser auf das Wachstum von Erbsen in [$kg$] erhalten Sie einen $\beta_{Fe_3O_4}$ Koeffizienten von $2.3\times 10^{-9}$ und einen hoch signifikanten $p$-Wert mit $2e-04$. Warum sehen Sie so einen kleinen Effekt bei einer so deutlichen Signifikanz?




\begin{enumerate}
\item [\textbf{A} \msquare] Manchmal ist die Einheit der Einflussvariable $X$ zu groß gewählt, so dass der Ansteig von 1 Einheit in $X$ zu einer zu großen Änderung in $y$ führt. Daher kann der Effekt $\beta_{Fe_3O_4}$ sehr klein wirken, da der p-Wert wird auf einer einheitslosen Teststatistik bestimmt wird.
\item [\textbf{B} \msquare] Die Einheit der $Fe_3O_4$-Konzentration ist zu klein gewählt. Dadurch sehen wir den sehr kleinen $p$-Wert. Der $p$-Wert und die Einheit von der $Fe_3O_4$-Konzentration hängen antiproportional zusammen.
\item [\textbf{C} \msquare] Die Fallzahl ist zu hoch angesetzt. Je höher die Fallzahl ist, desto kleiner ist die Teststatistik und damit ist dann auch der $p$-Wert sehr klein. Es sollte über eine Reduzierung der Fallzahl nachgedacht werden. Dann sollte der Effekt zum p-Wert passen.
\item [\textbf{D} \msquare] Wenn der Effekt $\beta_{Fe_3O_4}$ sehr klein ist, dann kann es an einer falsch gewählten Einheit liegen. Der Anstieg von einer Einheit in $X$ führt ja zu einer Änderung von $\beta_{Fe_3O_4}$ in $y$. Daher ist hier mit einer anderen Einheit in den Daten zu rechnen, so dass wir hier einen besser formatierten Effekt sehen. Der p-Wert stammt aus einer einheitslosen Teststatistik.
\item [\textbf{E} \msquare] Das Gewicht und die $Fe_3O_4$-Konzentration korrelieren sehr stark, deshalb wird der $\beta_{Fe_3O_4}$ Koeffizient sehr klein. Mit einer ANOVA kann für die Korrelation korrigiert werden und der Effektschätzer passt dann zum p-Wert.
\end{enumerate}

\section{Aufgabe \hfill (2 Punkte)}



Nachdem Sie Ihr Experiment abgeschlossen haben, stehen Sie vor der Frage wie Sie Ihre Daten modellieren sollen. In der Beispielauswertung von Ihrem Betreuenden finden Sie die Funktion \texttt{lm()} in \Rlogo. Welche Aussage ist richtig?





\begin{enumerate}
\item [\textbf{A} \msquare] Die Funktion \texttt{lm()} in \Rlogo ist der letzte Schritt für einen Gruppenvergleich. Vorher kann eine ANOVA oder aber ein multipler Vergleich in \{emmeans\} gerechnet werden. In der Funktion  \texttt{lm()} werden die Gruppenvarianzen bestimmt.
\item [\textbf{B} \msquare] Die Funktion \texttt{lm()} in \Rlogo wird klassischerweise für die nicht-lineare Regression genutzt. Ist die Einflussvariable $X$ numerisch so werden die Gruppenmittelwerte geschätzt.
\item [\textbf{C} \msquare] Die Funktion \texttt{lm()} in \Rlogo ist der erste Schritt für einen Gruppenvergleich. Danach kann eine ANOVA oder aber ein multipler Vergleich in \{emmeans\} gerechnet werden. In der Funktion  \texttt{lm()} werden die Gruppenmittelwerte bestimmt.
\item [\textbf{D} \msquare] Ist die Einflussvariable $X$ ein Faktor so werden die Gruppenmittelwerte geschätzt und eine anschließende ANOVA sowie multipler Gruppenvergleich mit \{emmeans\} ist möglich. Die Funktion \texttt{lm()} kann dabei eigentlich weggelassen werden, wird aber traditionell gerechnet.
\item [\textbf{E} \msquare] Neben der klassichen Verwendung der Funktion \texttt{lm()} in der linearen Regression kann auch ein Gruppenvergleich gerechnet werden. Dafür müssen aber alle Faktoren aus den Daten entfernt und numerishc umgewandelt werden. Dann kann das R Paket \{emmeans\} genutzt werden um die Korrelation zu berechnen. Eine Adjustierung ist dann nicht mehr notwendig.
\end{enumerate}

\section{Aufgabe \hfill (2 Punkte)}



In Ihrer Abschlussarbeit haben Sie neben den klassischen normalverteilten Endpunkte, wie Trockgewicht und Wuchshöhe noch den Infektionsstatus und Zähldaten erhoben. Um diese nicht normalverteilten Endpunkte auszuwerten nutzen Sie das \textit{generalisierte lineare Modell (GLM)}. Welche Aussage ist richtig?




\begin{enumerate}
\item [\textbf{A} \msquare] Dank dem \textit{generalisierten linearen Modell (GLM)} können auch andere Verteilungsfamilien -- außer die Normalverteilung -- mit einer linearen Regression modelliert werden. Dafür werden alle Verteilungen in eine Normalverteilung überführt und anschließend standardisiert.
\item [\textbf{B} \msquare] Das GLM ist ein faktisch maschineller Lernalgorithmus, der selstständig die Verteilungsfamilie für Y wählt.
\item [\textbf{C} \msquare] In \Rlogo ist mit dem \textit{generalisierten linearen Modell (GLM)} eine Modellierung implementiert, die die Poissonverteilung für Zähldaten oder die Binomialverteilung für 0/1-Daten modellieren kann. Weitere Modellierungen sind in \Rlogo auch mit zusätzlich geladenen Paketen nicht möglich.
\item [\textbf{D} \msquare] Dank dem \textit{generalisierten linearen Modell (GLM)} können auch andere Verteilungsfamilien als die Normalverteilung mit einer linearen Regression modelliert werden.
\item [\textbf{E} \msquare] Das GLM erlaubt auch nicht normalverteilte Residuen in der Schätzung der Regressionsgrade.
\end{enumerate}
\section*{Vermischte Themen}  

\section{Aufgabe \hfill (2 Punkte)}

Die Randomisierung von Beobachtungen zu den Versuchseinheiten
ist bedeutend in der Versuchsplanung. Welche der folgenden Aussagen ist richtig?



\begin{enumerate}
\item [\textbf{A} \msquare] Randomisierung erlaubt erst die Mittelwerte zu schätzen. Ohne Randomisierung keine Mittelwerte. Ohne Mittelwerte keine Varianz und somit auch kein statistischer Test.
\item [\textbf{B} \msquare] Randomisierung sorgt für Strukturgleichheit und erlaubt erst von der Stichprobe auf die Grundgesamtheit zurückzuschliessen.
\item [\textbf{C} \msquare] Randomisierung erlaubt erst die Varianzen zu schätzen. Ohne eine Randomisierung ist die Berechnung von Mittelwerten und Varianzen nicht möglich. Dadurch lässt sich erst ein Experiment auswerten.
\item [\textbf{D} \msquare] Durch eine Randomisierung können wir nicht von Strukturgleichheit zwischen der Stichprobe und der Grundgesamtheit ausgehen.
\item [\textbf{E} \msquare] Strukturgleichheit ist durch Randomisierung gegeben. Leider hilft die Randomisierung noch nicht um von der Stichprobe auf die Grundgesamtheit zu schließen. Deshalb wurde das Falsifikationsprinzip entwickelt.
\end{enumerate}

\section{Aufgabe \hfill (2 Punkte)}



Sie wollen Ihren Datensatz in \Rlogo einlesen und stehen nun vor einem Problem. Sie stellen fest, dass die Hilfeseiten alle in englischer Sprache verfasst sind. Warum mag die Nutzung von Deutsch problematisch sein?



\begin{enumerate}
\item [\textbf{A} \msquare] Es gibt keinen Grund nicht auch deutsche Wörter zu verwenden. Es ist ein Stilmittel.
\item [\textbf{B} \msquare] Programmiersprachen können nur englische Begriffe verarbeiten. Zusätzliche Pakete können zwar geladen werden, aber meist funktionieren diese Pakete nicht richtig. Deutsch ist International nicht bedeutend genug.
\item [\textbf{C} \msquare] \Rlogo Pakete sind nur in englischer Sprache verfasst. Es macht keinen Sinn \Rlogo daher in Deutsch zu bedienen.
\item [\textbf{D} \msquare] Im Allgemeinen haben Programmiersprachen Probleme mit Umlauten und Sonderzeichen, die in der deutschen Sprache vorkommen. Eine Nutzung der englischen Sprache umgeht dieses Problem auf einfache Art.
\item [\textbf{E} \msquare] Die Spracherkennung von \Rlogo ist nicht in der Lage Deutsch zu verstehen.
\end{enumerate}

\section{Aufgabe \hfill (2 Punkte)}



Bei der explorativen Datenanalyse (EDA) in \Rlogo gibt es eine richtige Abfolge von Prozessschritten, auch 	extit{Circle of life} genannt. Wie lautet die richtige Reihenfolge für die Erstellung einer EDA?



\begin{enumerate}
\item [\textbf{A} \msquare] Wir lesen als erstes die Daten über \texttt{read\_excel()} ein, transformieren die Spalten über \texttt{mutate()} in die richtige Form und können dann  über \text{ggplot()} uns die Abbildungen erstellen lassen. Wichtig ist, dass wir keine Faktoren sondern nur numerische Variablen vorliegen haben.
\item [\textbf{B} \msquare] Wir lesen die Daten über eine generische Funktion \texttt{read()} ein und müssen dann die Funktion \texttt{ggplot()} nur noch installieren. Dann haben wir die Abbildungen als \texttt{*.png} vorliegen.
\item [\textbf{C} \msquare] Wir lesen als erstes die Daten über \texttt{read\_excel()} ein, transformieren die Spalten über \texttt{mutate()} in die richtige Form und können dann über \text{ggplot()} uns die Abbildungen erstellen lassen.
\item [\textbf{D} \msquare] Die Funktionsreihenfolge ist wie folgt: \texttt{read\_excel()} ->  \texttt{mutate()} -> \text{ggplot()}. Dabei ist bei der Transformation der Daten darauf zu achten, dass keine Faktoren erstellt werden.
\item [\textbf{E} \msquare] Wir transformieren die Spalten über \texttt{mutate()} in ein \texttt{tibble} und können dann über \text{ggplot()} uns die Abbildungen erstellen lassen. Dabei beachten wir das wir keine Faktoren in den Daten haben.
\end{enumerate}

\section{Aufgabe \hfill (2 Punkte)}



Sie haben das abstrakte Modell $Y \sim X$ mit $X$ als Faktor mit zwei Leveln vorliegen. Welche Aussage über $s^2_1 \neq s^2_2$ ist richtig?



\begin{enumerate}
\item [\textbf{A} \msquare] Es handelt sich um unabhängige Beobachtungen.
\item [\textbf{B} \msquare] Es handelt sich um ein balanciertes Design.
\item [\textbf{C} \msquare] Es liegt Varianzhetrogenität vor.
\item [\textbf{D} \msquare] Es handelt sich um ein unbalanciertes Design.
\item [\textbf{E} \msquare] Es liegt Varianzhomogenität vor.
\end{enumerate}

\section{Aufgabe \hfill (2 Punkte)}



In einem Zuchtexperiment messen wir die Ferkel verschiedener Sauen. Die Ferkel einer Muttersau sind daher im statistischen Sinne...



\begin{enumerate}
\item [\textbf{A} \msquare] Je nach Stallanlage kommt eine andere Analyse in Betracht. Eine allgemeine Aussage über Ferkel und Sauen lässt sich statistisch nicht treffen.
\item [\textbf{B} \msquare] Untereinander unabhängig. Die Ferkel sind eigenständig und benötigen keine zusätzliche Behandlung.
\item [\textbf{C} \msquare] Untereinander unabhängig. Sollten die Mütter verwandt sein, so ist die Varianzstruktur ähnlich und muss modelliert werden.
\item [\textbf{D} \msquare] Untereinander abhängig. Die Ferkel stammen von einem Muttertier und haben vermutliche eine ähnliche Varianzstruktur.
\item [\textbf{E} \msquare] Abhängig von der Stallanlage und des Experiments können die Ferkel abhängig oder unabhängig sein. Allgmein gilt, dass Ferkel von unterschiedlichen Sauen näher miteinander verwandt sind als Ferkel von gleichen Sauen. Das Fisher-Axiom.
\end{enumerate}

\section{Aufgabe \hfill (2 Punkte)}



In einer Studie wollen Sie den Effektschätzer Odds ratio berechnen. Sie finden in Ihrem Experiment zur Behandlung von Klaueninfektionen bei Kühen in 6 Tieren Erkrankung der Klauen vor. 8 Tiere sind gesund. Welche Aussage ist richtig?



\begin{enumerate}
\item [\textbf{A} \msquare] Da es sich um ein Chancenverhältnis handelt ergibt sich ein Odds ratio von 2.33.
\item [\textbf{B} \msquare] Das Verhältnis der Anteile Odds ratio ergibt ein Anteilsverhältnis von 0.43. Wir sind am Anteil der Kranken interessiert.
\item [\textbf{C} \msquare] Der Anteil der Gesunden wird berechnet. Da es sich um ein Anteil handelt ergibt sich ein Odds ratio von 0.43.
\item [\textbf{D} \msquare] Es ergibt sich ein Odds ratio von 0.43, da es sich um eine Chancenverhältnis handelt.
\item [\textbf{E} \msquare] Das Verhältnis von Chancen Odds ratio ergibt ein Chancenverhältnis von 0.75.
\end{enumerate}

\section{Aufgabe \hfill (2 Punkte)}



Historisch gesehen ergibt sich ein Problem, wenn Sie mit sehr großen Datensätzen, wie in der Bio Data Sience üblich, rechnen. Warum ist es ein Problem, wenn Ihre Datensätze sehr groß werden hinsichtlich der Bewertung anhand der Signifikanz?




\begin{enumerate}
\item [\textbf{A} \msquare] Big Data ist ein Problem der parametrischen Statistik. Parameter lassen sich nur auf kleinen Datensätzen berechnen, da es sich sonst nicht mehr um eine Stichprobe im engen Sinne der Statistik handelt.
\item [\textbf{B} \msquare] Eine große Fallzahl führt zu mehr signifikanten Ergebnissen auch bei kleinen Effekten. Daher werden fast alle Vergleich esignifikant, wenn die Fallzahl nur groß genug wird.
\item [\textbf{C} \msquare] Aktuell werden immer größere Datensätze erhoben. Dadurch wird auch die Varianz immer höher was automatisch zu mehr signifikanten Ergebnissen führt.
\item [\textbf{D} \msquare] Aktuell werden zu grosse Datensätze für die gänigige Statistik gemessen. Daher wendet man maschinelle Lernverfahren für kausale Modelle an. Hier ist die Relevanz gleich Signifikanz.
\item [\textbf{E} \msquare] Riesige Datensätz haben mehr Fallzahl was zur $\alpha$-Inflation führt. Durch eine Adjustoerung kann dem Problem entgegengewirkt werden.
\end{enumerate}
\section*{Multiple Gruppenvergleiche}    

\section{Aufgabe \hfill (2 Punkte)}



Sie haben folgende unadjustierten p-Werte gegeben: 0.34, 0.02, 0.01, 0.89, 0.42 und 0.03. Sie adjustieren die p-Werte nach
Bonferroni. Welche Aussage ist richtig?



\begin{enumerate}
\item [\textbf{A} \msquare] Nach der Bonferroni-Adjustierung ergeben sich die adjustierten p-Werte von 0.0567, 0.0033, 0.0017, 0.1483, 0.07 und 0.005. Die adjustierten p-Werte werden zu einem $\alpha$-Niveau von 5\% verglichen.
\item [\textbf{B} \msquare] Nach der Bonferroni-Adjustierung ergeben sich die adjustierten p-Werte von 1, 0.12, 0.06, 1, 1 und 0.18. Die adjustierten p-Werte werden zu einem $\alpha$-Niveau von 5\% verglichen.
\item [\textbf{C} \msquare] Nach der Bonferroni-Adjustierung ergeben sich die adjustierten p-Werte von 2.04, 0.12, 0.06, 5.34, 2.52 und 0.18. Die adjustierten p-Werte werden zu einem $\alpha$-Niveau von 5\% verglichen.
\item [\textbf{D} \msquare] Nach der Bonferroni-Adjustierung ergeben sich die adjustierten p-Werte von 0.0567, 0.0033, 0.0017, 0.1483, 0.07 und 0.005. Die adjustierten p-Werte werden zu einem $\alpha$-Niveau von 0.83\% verglichen.
\item [\textbf{E} \msquare] Nach der Bonferroni-Adjustierung ergeben sich die adjustierten p-Werte von 1, 0.12, 0.06, 1, 1 und 0.18. Die adjustierten p-Werte werden zu einem $\alpha$-Niveau von 0.83\% verglichen.
\end{enumerate}

\section{Aufgabe \hfill (2 Punkte)}



Sie rechnen einen PostHoc-Test. Nun sollen Sie ein \textit{CLD} erstellen. Was bedeutet dieser Fachbegriff und welche folgende Beschreibung der Interpretation ist korrekt?



\begin{enumerate}
\item [\textbf{A} \msquare] Compact letter display. Das CLD ist umstritten, da es die Gleichheit der Behandlungen durch gleiche Buchstaben darstellt. Dadurch ist das CLD nicht mehr sauber auf einer Linie mit dem statistischen Testen. Wir lehnen die Nullhypothese ab und zeigen keine Gleichheit im statistischen Testen.
\item [\textbf{B} \msquare] Compact letter display. Gleiche Buchstaben bedeuten, dass sich die Behandlungen unterscheiden. Daher ist das CLD sehr unintuitiv. Es wäre besser, wenn gleiche Buchstaben Gleichheit anzeigen würden. Dies ist aber leider in der statistischen Testtheorie nicht möglich.
\item [\textbf{C} \msquare] Compact letter display. Gleiche Buchstaben zeigen Gleichheit in den Behandlungen. Die Interpretation ist deshalb sehr intuitiv und einfach. Darüber hinaus ist damit das CLD auch auf einer Linie mit der Testtheorie, da wir ja auch dort die Gültigkeit der Nullhypothese nachweisen. Wir suchen ja Gleichheit.
\item [\textbf{D} \msquare] Contrast letter display. Unterschiede in den Behandlungen werden durch den gleichen Buchstaben oder Symbol dargestellt. Die Interpretation des CLD führt häufig in die Irre.
\item [\textbf{E} \msquare] Compact letter detection. Gleichheit in den Behandlungen wird durch den gleichen Buchstaben oder Symbol dargestellt.
\end{enumerate}

\section{Aufgabe \hfill (2 Punkte)}




Der multiple Vergleich als Posthoc-Test nach einer ANOVA ist in den Agrarwissenschaften heutzutage Standard. Welches R Paket wird häufig für den multiplen Vergleich genutzt? Welche Beschreibung der Eigenschaften ist korrekt?



\begin{enumerate}
\item [\textbf{A} \msquare] Das R Paket \{hmisc\} erlaubt die Durchführung eines multiplen Gruppenvergleichs aus verschiedenen Modellen heraus. Aus einem hmisc Objekt lässt sich recht einfach das CLD erstellen und so über Barplots eine schnelle Interpration der statistischen Auswertung durchführen.
\item [\textbf{B} \msquare] Das R Paket \{lm\}. Das Paket \{lm\} erstellt selbstständig Konfidenzintervalle und entsprechende p-Werte. Da wir in dem Paket nicht adjustieren müssen, ist es bei Anwendern sehr beliebt.
\item [\textbf{C} \msquare] Das R Paket \{emmeans\} erlaubt die Durchführung eines multiplen Gruppenvergleichs. Aus einem emmeans Objekt lässt sich leider kein CLD erstellen. Dennoch ist das Paket einfach zu bedienen und wird deshalb genutzt. Die Interpretation der statistischen Auswertung wird über einen Barplot abgebildet.
\item [\textbf{D} \msquare] Das R Paket \{emmeans\} erlaubt die Durchführung eines multiplen Gruppenvergleichs. Aus einem \{emmeans\} Objekt lässt sich recht einfach das CLD erstellen und so über Barplots eine schnelle Interpration der statistischen Auswertung durchführen.
\item [\textbf{E} \msquare] Das R Paket \{ggplot\}. Wir erhalten hier sofort eine Visualisierung der Daten. Anhand der Visualisierung lässt sich eine explorative Datenanalyse durchführen, die gleichwertig zu einem Posthoc-Test ist.
\end{enumerate}

\section{Aufgabe \hfill (2 Punkte)}



Bei einem multiplen Vergleich oder Posthoc Test kann es zu einer Besonderheit beim statistischen Testen kommen. Wie nennt man diese Besonderheit beim statistischen Testen und wie kann man mit ihr umgehen?



\begin{enumerate}
\item [\textbf{A} \msquare] Das globale Signifikanzniveau explodiert und erreicht Werte größer als Eins. Es kommt zu einer $\alpha$-Inflation. Dagegen kann mit der Adjustierung der $\alpha$-Werte nach Bonferroni vorgegangen werden.
\item [\textbf{B} \msquare] Beim multiplen Testen kann es zu einer $\alpha$-Inflation kommen. Das globale Signifikanzniveau liegt nicht mehr bei $5\%$ sondern weit darunter. Daher müssen die p-Werte entsprechend adjustiert werden. Hierfür gibt es verschiedene Verfahren, wobei das Verfahren zur Adjustierung der p-Werte nach Welch das bekanneste Verfahren ist.
\item [\textbf{C} \msquare] Die Adjustierung der p-Werte nach Bonferroni erlaubt es gegen die $\beta$-Inflation vorzugehen, die häufig beim multiplen Testen auftritt. Das globale Powerniveau liegt nicht mehr bei $80\%$ sondern sehr viel niedriger.
\item [\textbf{D} \msquare] Die Adjustierung der p-Werte nach Bonferroni erlaubt es gegen die $\alpha$-Inflation vorzugehen, die häufig beim multiplen Testen auftritt. Das globale Signifikanzniveau liegt nicht mehr bei $5\%$ sondern sehr viel höher. Das ist der Grund warum die p-Werte entsprechend adjustiert werden müssen.
\item [\textbf{E} \msquare] Beim multiplen Testen kann es zu einer $\beta$-Inflation kommen. Das globale Signifikanzniveau liegt nicht mehr bei $20\%$. Daher müssen die p-Werte entsprechend adjustiert werden. Hierfür gibt es verschiedene Verfahren, wobei das Verfahren zur Adjustierung der p-Werte nach Bonferroni das bekanneste Verfahren ist.
\end{enumerate}

\section{Aufgabe \hfill (2 Punkte)}




In Ihrer Bachelorarbeit werten Sie einen einfaktoriellen Versuch aus. Dafür rechnen Sie in \Rlogo zunächst eine ANOVA und schließen dann dann einen multiplen vergleich mit t-Tests an. Welche Aussage über die Effekte in Ihrem versuch ist richtig?



\begin{enumerate}
\item [\textbf{A} \msquare] Wenn ein multipler Test gerechnet wird, dann muss der Effekt $\Delta$ nach Bonferroni adjustiert werden. Dafür wird der Effekt mit der Anzahl an Vergleichen $k$ multipliziert. Dies geschiet analog zu den p-Werten.
\item [\textbf{B} \msquare] Beim multiplen Testen kann es zu einer $\Delta$-Inflation kommen. Das globale Effektniveau liegt nicht mehr bei $20\%$. Daher müssen die Effekte entsprechend adjustiert werden. Hierfür gibt es verschiedene Verfahren, wobei das Verfahren zur Adjustierung der Effekte nach Bonferroni das bekanneste Verfahren ist.
\item [\textbf{C} \msquare] Wenn ein multipler Test gerechnet wird, dann muss der Effekt $\Delta$ nicht adjustiert werden im Gegensatz zu den p-Werten.
\item [\textbf{D} \msquare] Beim multiplen Testen kann es zu einer Effektüberschätzung ($\Delta$-Inflation) kommen. Daher müssen die Effekte angepasst werden. Dies geschieht nicht händisch sondern intern in den angewendeten Algorithmen.
\item [\textbf{E} \msquare] Beim multiplen Testen kann es zu einer $\Delta$-Deflation kommen. Das globale Relevanzniveau liegt nicht mehr bei $5\%$ sondern weit darunter. Daher müssen die $\Delta$-Werte entsprechend adjustiert werden. Hierfür gibt es verschiedene Verfahren, wobei das Verfahren zur Adjustierung der $\Delta$-Werte nach Bonferroni das bekanneste Verfahren ist. Die $\Delta$-Werte werden durch die Anzahl an Vergleichen geteilt.
\end{enumerate}
\section*{Statistische Testtheorie}  

\section{Aufgabe \hfill (2 Punkte)}




Sie haben den mathematischen Ausdruck $Pr(D|H_0)$ vorliegen, welche Aussage ist richtig?



\begin{enumerate}
\item [\textbf{A} \msquare] Die Wahrscheinlichkeit der Daten unter der Nullhypothese in der Grundgesamtheit.
\item [\textbf{B} \msquare] Die Inverse der Wahrscheinlichkeit unter der die Nullhypothese nicht mehr die Alternativehypothese überdeckt.
\item [\textbf{C} \msquare] $Pr(D|H_0)$ ist die Wahrscheinlichkeit der Alternativehypothese und somit $1 - Pr(H_A)$
\item [\textbf{D} \msquare] $Pr(D|H_0)$ stellt die Wahrscheinlichkeit die Daten $D$ und somit die Teststatistik $T_D$ zu beobachten dar, wenn die Nullhypothese wahr ist.
\item [\textbf{E} \msquare] $Pr(D|H_0)$ ist die Wahrscheinlichkeit nicht die Daten $D$ zu beobachten sondern die Nullhypothese, wenn diese wahr ist.
\end{enumerate}

\section{Aufgabe \hfill (2 Punkte)}



Das statistische Testen basiert auf dem Falsifikationsprinzip. Es besagt,



\begin{enumerate}
\item [\textbf{A} \msquare] ... dass Fehlerterme in statistischen Modellen nicht verifiziert werden können.
\item [\textbf{B} \msquare] ... dass in der Wissenschaft immer etwas falsch sein muss. Sonst gebe es keinen Fortschritt.
\item [\textbf{C} \msquare] ... dass ein schlechtes Modell durch ein schlechteres Modell ersetzt wird. Die Wissenschaft lehnt ab und verifiziert nicht.
\item [\textbf{D} \msquare] ... dass ein schlechtes Modell durch ein weniger schlechtes Modell ersetzt wird. Die Wissenschaft lehnt ab und verifiziert nicht.
\item [\textbf{E} \msquare] ... dass ein schlechtes Modell durch das Falsifikationsprinzip durch ein noch schlechteres Modell ersetzt wird. Die Wissenschaft lehnt ab und verifiziert nicht.
\end{enumerate}

\section{Aufgabe \hfill (2 Punkte)}



In fast allen wissenschaftlichen Disziplinen liegt der Grenzwert für das Signifikanzniveau $\alpha$ bei 5\%. Wieso wurde dieser Konsens über die Signifikanzschwelle in dieser Form getroffen?



\begin{enumerate}
\item [\textbf{A} \msquare] Der Wert ergab sich aus einer Auswertung von 1042 wissenschaftlichen Veröffentlichungen zwischen 1914 und 1948. Der Wert $5\%$ wurde in $28\%$ der Veröffentlichungen genutzt. Daher legte man sich auf diese Zahl fest.
\item [\textbf{B} \msquare] Im Rahmen eines langen Disputs zwischen Neyman und Fischer wurde $\alpha = 5\%$ festgelegt. Leider werden die Randbedingungen und Voraussetzungen an statistsiche Modelle heute immer wieder ignoriert.
\item [\textbf{C} \msquare] Als Kulturkonstante hat $\alpha = 5\%$ den Rang einer Naturkonstante und wurde nach langer Diskussion in der UN im Jahre 1983 festgesetzt. Damals auch schon mit der Zustimmung der UdSSR.
\item [\textbf{D} \msquare] Der Begründer der modernen Statistik, R. Fischer, hat die Grenze simuliert und berechnet. Dadurch ergibt sich dieser optimale Cut-Off.
\item [\textbf{E} \msquare] Da Wissenschaftler eine Schwelle für die statistische Testentscheidung benötigen wurde $\alpha$ historisch gewählt. Damit ist $\alpha = 5\%$ eine Kulturkonstante.
\end{enumerate}

\section{Aufgabe \hfill (2 Punkte)}

Betrachten wir die Teststatistik aus einem abstrakteren Blickwinkel. Beim
statistischen Testen wird das \textit{"`signal"'} mit dem
\textit{"`noise"'} aus den Daten $D$ zu einer Teststatistik $T_D$ verrechnet. Welche der Formel
berechnet korrekt die Teststatistik $T_D$?



\begin{enumerate}
\item [\textbf{A} \msquare] Es gilt $T_D = \cfrac{noise}{signal}$
\item [\textbf{B} \msquare] Es gilt $T_D = signal \cdot noise$
\item [\textbf{C} \msquare] Es gilt $T_D = \cfrac{signal}{noise}$
\item [\textbf{D} \msquare] Es gilt $T_D = (signal \cdot noise)^2$
\item [\textbf{E} \msquare] Es gilt $T_D = \cfrac{signal}{noise^2}$
\end{enumerate}

%% ------------------------------------------------------------

\section{Aufgabe \hfill (2 Punkte)}



Sie versuchen folgende Aussage richtig in die Analogie der statistischen Testtheorie zu setzen. Welche Analogie ist richtig?

\begin{center}
\textit{$H_0$ ablehnen obwohl die $H_0$ gilt}
\end{center}



\begin{enumerate}
\item [\textbf{A} \msquare] In die Analogie eines Rauchmelders: \textit{Alarm without fire police}, dem $\alpha$-Fehler.
\item [\textbf{B} \msquare] In die Analogie eines Feuerwehrautos: \textit{Car without noise}.
\item [\textbf{C} \msquare] Dem $\beta$-Fehler mit der Analogie eines Rauchmelders: \textit{Fire without alarm}.
\item [\textbf{D} \msquare] In die Analogie eines Rauchmelders: \textit{Alarm without fire}, dem $\alpha$-Fehler.
\item [\textbf{E} \msquare] In die Analogie eines Rauchmelders: \textit{Alarm with fire}.
\end{enumerate}

\section{Aufgabe \hfill (2 Punkte)}



Sie lesen eine wissenschaftliche Arbeit, die damit wirbt, dass Effekte und Signifikanz nicht separat dargestellt sind, sondern in einer statistischen Maßzahl zusammen. Welche Aussage ist richtig?



\begin{enumerate}
\item [\textbf{A} \msquare] Über das Konfidenzintervall. Das Konfidenzinterval inkludiert eine Entscheidung über die Relevanz und zusätzlich kann über die Visualizierung des Konfidenzintervals eine Signifikanzschwelle vom Forschenden definiert werden.
\item [\textbf{B} \msquare] Der p-Wert. Durch den Vergleich mit $\alpha$ lässt sich über die Signifikanz entscheiden und der $\beta$-Fehler erlaubt über die Power eine Einschätzung der Relevanz.
\item [\textbf{C} \msquare] Einem Konfidenzintervall. Das Konfidenzinterval bringt durch eine Visualisierung und zwei Intervallgrenzen die Möglichkeit mit, eine Relevanzschwelle neben der definierten Signifikanzschwelle zu definieren.
\item [\textbf{D} \msquare] Das $\Delta$. Durch die Effektstärke haben wir einen Wert für die Relevanz, die vom Anwender bewertet werden muss. Da $\Delta$ antiproportional zum p-Wert ist, bedeutet auch ein hohes $\Delta$ ein sehr kleinen p-Wert.
\item [\textbf{E} \msquare] Das OR. Als Chancenverhältnis gibt es das Verhältnis von Relevanz und Signifikanz wieder.
\end{enumerate}

\section{Aufgabe \hfill (2 Punkte)}



Ein statistischer Test produziert für einen Gruppenvergleich einen $p$-Wert. Welche Aussage zusammen mit dem Signifikanzniveau $\alpha$ gleich 5\% stimmt?



\begin{enumerate}
\item [\textbf{A} \msquare] Wir vergleichen mit dem $p$-Wert und dem Signifikanzniveau $\alpha$ absolute Werte auf einem Zahlenstrahl und damit den Unterschied der Teststatistiken, wenn die $H_0$ gilt.
\item [\textbf{B} \msquare] Wir machen eine Aussage über die indivduelle Wahrscheinlichkeit des Eintretens der Nullhypothese $H_0$. Der $p$-Wert wird mit dem Signifikanzniveau verglichen und bewertet.
\item [\textbf{C} \msquare] Wir vergleichen mit dem $p$-Wert und dem Signifikanzniveau $\alpha$ Wahrscheinlichkeiten und damit die absoluten Werte auf einem Zahlenstrahl, wenn die $H_0$ gilt.
\item [\textbf{D} \msquare] Wir machen ein Aussage über die Flächen und der Kurve der Teststatistik, wenn die $H_0$ gilt. Dabei werden Wahrscheinlichkeiten vergleichen, die durch die Flächen unter der Kurve repräsentiert werden.
\item [\textbf{E} \msquare] Wir vergleichen die Effekte des $p$-Wertes mit den Effekten der Signifikanzschwelle unter der Annahme der Nullhypothese. Dabei gilt, dass wir die Nullhypothese nur ablehnen können anhand des Falsifikationsprinzips.
\end{enumerate}

\section{Aufgabe \hfill (2 Punkte)}



Um die Ergebnisse eines statistischen Tests und die damit verbundene Theorie besser zu verstehen, kann eine Analogie zur Wettervorhersage genutzt werden. Welche Analogie zu der Testtheorie trifft am meisten zu?



\begin{enumerate}
\item [\textbf{A} \msquare] In der Analogie der Sonnenscheindauer: Wie lange kann mit einem entsprechenden Effekt gerechnet werden? Die Wahrscheinlichkeit für den Effekt gibt der statistische Test wieder.
\item [\textbf{B} \msquare] In der Analogie der Regenwahrscheinlichkeit in einem bestimmten Gebiet: ein statistischer Test gibt die Wahrscheinlichkeit für ein Ereignis in einem Experiment mit den Daten $D$ wieder und lässt sich kaum verallgemeinern.
\item [\textbf{C} \msquare] In der Analogie der Maximaltemperatur: Was ist der maximale Unterschied zwischen zwei Gruppen. Wir erhalten hier eine Aussage über die Spannweite und den maximalen Effekt.
\item [\textbf{D} \msquare] In der Analogie der Wahrscheinlichkeit für Regen: ein statistischer Test erlaubt die Wahrscheinlichkeit für ein Ereignis abzuschätzen. Die Stärke des Effektes können wir nicht bestimmen.
\item [\textbf{E} \msquare] In der Analogie des Niederschlags oder Regenmenge: ein statistischer Test gibt die Stärke eines Effektes wieder. Zum Beispiel, wie hoch ist der Mittelwertsunterschied.
\end{enumerate}

\section{Aufgabe \hfill (2 Punkte)}



In Ihrer Abschlussarbeit wollen Sie eine Aussage über die untersuchte Population treffen. Dazu nutzen Sie einen statistischen Test. Können Sie eine valide Aussage treffen?



\begin{enumerate}
\item [\textbf{A} \msquare] Ja, wir erhalten nur eine Aussage zu zwei Individuen. Ein statistischer Test liefert Informationen zu einem Individuum im Vergleich zu einem anderen Individuum.
\item [\textbf{B} \msquare] Nein, die untersuchte Population können wir mit einem statistischen Test nicht auswerten. Wir erhalten keine Aussage zur Population.
\item [\textbf{C} \msquare] Ja, wir können die untersuchte Population mit einer ANOVA auswerten. Wir erhalten eine Aussage zur Population.
\item [\textbf{D} \msquare] Nein, wir können die untersuchte Population nicht mit einem t-Test auswerten. Wir erhalten keine Aussage zur Population. Wir können aber den Effekt als Quelle der Relevanz nutzen.
\item [\textbf{E} \msquare] Weder eine Ausssage über die Population noch über das Individuum ist mit einem statistischen Test möglich. Wir erhalten eine Aussage über ein Experiment.
\end{enumerate}

\section{Aufgabe \hfill (2 Punkte)}



Sie haben die \textit{Power} berechnet. Was sagt Ihnen dieser statistische Begriff aus?



\begin{enumerate}
\item [\textbf{A} \msquare] Es gilt $\alpha + \beta = 1$ und somit liegt $\beta$ meist bei 95\%.
\item [\textbf{B} \msquare] Die Power $1-\beta$ wird auf 80\% gesetzt. Alle statistischen Tests sind so konstruiert, dass die $H_A$ mit 80\% \textit{bewiesen wird}.
\item [\textbf{C} \msquare] Alle statistischen Tests sind so konstruiert, dass die $H_A$ mit 20\% \textit{bewiesen wird}. Die Power ist $1-\beta$ mit $\beta$ gleich 80\% gesetzt.
\item [\textbf{D} \msquare] Die Power ist nicht in der aktuellen Testthorie mehr vertreten. Wir rechnen nur noch mit dem Fehler 1. Art.
\item [\textbf{E} \msquare] Die Power beschreibt die Wahrscheinlichkeit die $H_A$ abzulehnen. Wir testen die Power jedoch nicht.
\end{enumerate}

\section{Aufgabe \hfill (2 Punkte)}



In Ihrer Abschlussarbeit sollen Sie neben den p-Werten auch die Effekte mit angeben. Welche Aussage ist richtig?



\begin{enumerate}
\item [\textbf{A} \msquare] Der Effekt eines statistischen Tests beschreibt die mathematisch interpretierbare Ausgabe eines Tests. Damit ist der Effekt direkt mit dem Begriff der Signifikanz verbunden. Die Entscheidung über die Signifikanz trifft der Forschende unabhängig von der Relevanz eines statistsichen Tests.
\item [\textbf{B} \msquare] Der Effekt eines statistischen Tests beschreibt die biologisch interpretierbare Ausgabe eines Tests. Moderen Algorithmen liefern keine Effekte mehr sondern nur noch bedingte Wahrscheinlichkeiten. Der Effekt spielt in der modernen Statistik keine Rollen mehr.
\item [\textbf{C} \msquare] Der Effekt eines statistischen Tests beschreibt den Output oder die Wiedergabe eines Tests in einem Computer.
\item [\textbf{D} \msquare] Der Effekt eines statistischen Tests beschreibt die biologisch interpretierbare Ausgabe eines Tests. Damit ist der Effekt direkt mit dem Begriff der Signifikanz verbunden. Die Entscheidung über die Signifikanz trifft der Forschende unabhängig von der Relevanz eines statistsichen Tests.
\item [\textbf{E} \msquare] Der Forschende muss am Ende wissen, ob das Eregbnis eines Experiments relevant für seine Forschung ist. Dafür kann der Effekt eines statistischen Tests genutzt werden. Damit beschreibt der Effekt den biologischen interpretierbaren Teil einer Ausgabe eines Tests. Zum Beispiel der Unterschied zwischen zwei Anteilen.
\end{enumerate}

\section{Aufgabe \hfill (2 Punkte)}



Welche Aussage über die Entscheidung anhand des 95\%-Konfidenzintervalls gegen die
Nullhypothese ist richtig?



\begin{enumerate}
\item [\textbf{A} \msquare] Anhand des 95\%-Konfidenzintervalls lässt sich wie folgt eine Entscheidung treffen. Liegt der Wert über oder gleich dem Signifikanzniveau $\alpha$ dann kann die Nullhypothese abgelehnt werden.
\item [\textbf{B} \msquare] Ist $T_{D}$ h{"o}her als der kritische Wert $T_{\alpha = 5\%}$ dann wird die Nullhypothese $H_0$ abgelehnt.
\item [\textbf{C} \msquare] Ist in dem 95\%-Konfidenzintervall nicht die Null enthalten dann wird die Nullhypothese $H_0$ abgelehnt.
\item [\textbf{D} \msquare] Ist $Pr(D|H_0)$ kleiner als das Signifikanzniveau $\alpha$ gleich $5\%$ dann wird die Nullhypothese $H_0$ abgelehnt.
\item [\textbf{E} \msquare] Anhand des 95\%-Konfidenzintervalls lässt sich wie folgt eine Entscheidung treffen. Liegt der Wert in dem Signifikanzniveauintervall $\alpha$ dann kann die Nullhypothese abgelehnt werden.
\end{enumerate}

\section{Aufgabe \hfill (2 Punkte)}



In Ihrer Abschlussarbeit müssen Sie für die statistischen Tests im Anhang Ihrer Arbeit die Hypothesen $H$ formulieren. Welche Aussage über Hypothesen $H$ ist richtig



\begin{enumerate}
\item [\textbf{A} \msquare] Die Hypothesen $H_0$ und $H_A$ sind rein prosarischer Natur und bilden keinen mathematischen Hintergrund ab. In der Statistik wird die wissenschaftliche Fragestellung getestet. Daher stehen auch die verständlichen Hypothesen im Mittelpunkt der biologischen Interpretation.
\item [\textbf{B} \msquare] Mit der Nullhypothese $H_A$ und der Alternativehypothese $H_0$ gibt es zwei Hypothesen, die aber selten genutzt werden.
\item [\textbf{C} \msquare] Es gibt ein statistisches Hypothesenpaar mit der Hypothese für und gegen die wissenschaftliche Fragestellung. Die Hypothesen werden $H_{pro}$ und $H_{contra}$ bezeichnet.
\item [\textbf{D} \msquare] Es gibt - bedingt durch das das Falsifikationsprinzip - ein Set von $k$ Nullhypothesen, die iterative gegen $k-1$ Alternativhypothesen getestet werden.
\item [\textbf{E} \msquare] Es gibt ein statistisches Hypothesenpaar mit der Nullhypothese $H_0$ und der Alternativehypothese $H_A$ oder $H_1$.
\end{enumerate}
\section*{Statistische Tests für Gruppenvergleiche} 

\section{Aufgabe \hfill (2 Punkte)}



In Ihrer Abschlussarbeit rechnen Sie einen Student t-Test. Welche Aussage ist auch für den Welch t-Test richtig?



\begin{enumerate}
\item [\textbf{A} \msquare] Der t-Test vergleicht zwei Gruppen indem die Mittelwerte miteinander verglichen werden.
\item [\textbf{B} \msquare] Der t-Test testet generell zu einem erhöhten $\alpha$-Niveau von 20\%.
\item [\textbf{C} \msquare] Der t-Test vergleicht die Varianzen von mindestens zwei oder mehr Gruppen
\item [\textbf{D} \msquare] Der t-Test vergleicht zwei oder mehr Gruppen indem die Mittelwerte miteinander verglichen werden.
\item [\textbf{E} \msquare] Der t-Test berechnet die Differenz von zwei Mittelwerten als Effekt und gibt eine Entscheidung, ob sich die beiden Mittelwerte \textit{jeweils} von Null unterscheiden.
\end{enumerate}

\section{Aufgabe \hfill (2 Punkte)}



Ein Versuch wurde in 7 Parzellen pro Gruppe durchgeführt. Die folgende Abbildung enthält die Daten aus diesem Versuch zur Bewertung der Wirkung des Mikronährstoff Eisen auf den Ertrag in t/ha von Mango im Vergleich zu einer Kontrolle.  Welche Aussage ist richtig, wenn Sie einen t-Test rechnen?



{\centering \includegraphics[width=\maxwidth]{img/mc-testing-ttest-02-1} 

}







\begin{enumerate}
\item [\textbf{A} \msquare] Der Effekt und die Signifikanz lassen sich nicht aus Barplots abschätzen. Höchtens der Effekt als relativer Unterschied zwischen der Höhe der Barplots. Standard ist der mediane Unterschied aus Boxplots.
\item [\textbf{B} \msquare] Es liegt ein signifikanter Unterschied vor. Der Effekt liegt bei -0.7.
\item [\textbf{C} \msquare] Nach Betrachtung des Barplots liegt kein signifikanter Unterschied vor. Der Effekt kann nicht bei einem t-Test aus Barplots bestimmt werden.
\item [\textbf{D} \msquare] Die Barplots deuten auf keinen signifikanten Unterschied. Der Effekt liegt vermutlich bei -7 unter einer groben Abschätzung.
\item [\textbf{E} \msquare] Es liegt ein signifikanter Unterschied vor. Der Effekt liegt bei -7.
\end{enumerate}

\section{Aufgabe \hfill (2 Punkte)}




Welche Aussage über den gepaarten t-Test für verbundene Stichproben ist richtig?



\begin{enumerate}
\item [\textbf{A} \msquare] Der gepaarte t-Test wird gerechnet, wenn die Beobachtungen nicht unabhängig voneinander sind. Wir messen wiederholt an dem gleichen Probanden oder Tier oder Pflanze. Wir bilden die Differenzen um den gepaarten t-Test rechnen zu können.
\item [\textbf{B} \msquare] Abhängige Beobachtungen müssen gesondert in einem gepaarten t-Test modelliert werden. Wenn wiederholt an dem gleichen Tier oder Pflanze gemessen wird, dann bilden wir den Quotienten zwischen den beiden Zeitpunkten. Auf den Quotienten rechnen wir den gepaarten t-Test.
\item [\textbf{C} \msquare] Wenn die Beobachtungen unabhängig voneinander sind, rechnen wir einen gepaarten t-Test. Messen wir wiederholt an dem gleichen Tier oder Pflanze dann bilden wir das Produkt zwischen den zwei Messpunkten.
\item [\textbf{D} \msquare] Der gepaarte t-Test wird genutzt, wenn die Differenzen der Beobachtungen verbunden sind und wir dadurch die Unabhäängigkeit nicht mehr vorliegen haben.
\item [\textbf{E} \msquare] Der gepaarte t-Test wird gerechnet, wenn die Beobachtungen abhängig voneinander sind. Wir messen jede Beobachtung nur einmal und berechnen dann die Differenz zu dem Mittel der anderen Beobachtungen.
\end{enumerate}

\section{Aufgabe \hfill (2 Punkte)}



Sie führen paarweise t-Tests für alle Vergleiche der verschiedenen Rapssorten in Ihrem Experiment durch. Nach der Adjustierung für multiples Testen ist kein p-Wert unter der $\alpha$-Schwelle. Ihr Experiment beinhaltet vier Rapssorten und eine ANOVA ergibt $p = 0.045$ für den Ertrag. Sie schauen sich auch die rohen, unadjustierten p-Werte an und finden hier als niedrigsten p-Wert $p_{3-2} = 0.051$. Welche Aussage ist richtig?




\begin{enumerate}
\item [\textbf{A} \msquare] Es gibt einen Fehler in der Varianzstruktur. Daher kann die ANOVA nicht richtig sein und paarweise t-Tests liefern das richtige Ergebnis.
\item [\textbf{B} \msquare] Hier kommt der Effekt der stiegenden Fallzahl auf die Anzahl an signifikante Ergebnisse zu tragen. Da die ANOVA auf weniger Fallzahl testet als die paarweisen t-Tests, kann die ANOVA schwerer einen signifikanten Unterscheid nachweisen.
\item [\textbf{C} \msquare] Das ist kein Wunder. Die ANOVA testet nicht auf der gesamten Fallzahl und die paarweisen t-Tests gewinnen immer eine oder mehr Gruppen als Fallzahl dazu. Mit steigender Fallzahl sind mehr signifikante Unterschiede zu erwarten. Die p-Werte unterscheiden sich numerisch auch kaum.
\item [\textbf{D} \msquare] Hier kommt der Effekt der stiegenden Fallzahl auf die Anzahl an signifikante Ergebnisse zu tragen. Da die ANOVA auf mehr Fallzahl testet als die einzelnen paarweisen t-Tests, kann die ANOVA leichter einen signifikanten Unterscheid nachweisen. Die p-Werte sind immer etwas kleiner als bei den t-Tests.
\item [\textbf{E} \msquare] Die adjustierten p-Werte deuten in die richtige Richtung. Zusammen mit den nicht signifikanten rohen p-Werten ist von einem Fehler in der ANOVA auszugehen.
\end{enumerate}
    
% -----------------------------------------------------------------------
\clearpage
% -----------------------------------------------------------------------
\part{Deskriptive Statistik \& Explorative Datenanalyse}
% -----------------------------------------------------------------------

\section{Aufgabe \hfill (8 Punkte)}

\textit{Geben Sie grundsätzlich Formeln und Rechenweg zur Lösung der Teilaufgaben mit an!} \\[1Ex]
 

 
%% --------------------------------------------------------------------
\begin{minipage}[t]{0.5\textwidth}
\includegraphics[width = 1.3cm]{/Users/kruppajo/work/GitHub/exam/avatare/Jessica.png}
\end{minipage}
\begin{minipage}[t]{0.5\textwidth}
\hfill
\href{https://youtu.be/t0WYa_LVc5o}{\includegraphics[width = 2cm]{img/youtube}}\\[1Ex]
\end{minipage}
\vspace{1ex}
%% --------------------------------------------------------------------



Jessica steht vor einem ersten Problem, denn wenn es nach ihrer Betreuerin geht, soll sie in einem einer Klimakammer Lauch auswertet. Soweit eigentlich alles passend. Besser wäre was anderes gewesen. Jessica liebt Warhammer. Darin kann sie sich wirklich verlieren und immer wieder neu begeistern. Das heißt erstmal überlegen für Jessica. Aus den Boxen wummert David Bowie und ihr Mund ist verklebt von Schokobons. 'Herrlich', denkt Jessica. Die Behandlung werden verschiedene Substrattypen ($torf$, $40p60n$ und $70p30n$) sein. In ihrer Exceldatei wird sie den Outcome ($Y$) \textit{Trockengewicht} als \textit{drymatter} aufnehmen. Vorab soll Jessica aber eimal die folgenden Barplots ihrer Betreuerin nachbauen, damit sie den \Rlogo Code schonmal für später vorliegen hat. Damit geht das Problem schon los. Eine echte Herausforderung für sie war schon immer der Mangel gewesen. Ein leidiges Lied.



{\centering \includegraphics[width=\maxwidth]{img/barplot-02-1} 

}




Leider kennt sich Jessica mit der Erstellung von Barplots in \Rlogo nicht aus. Deshalb braucht sie bei der Visualisierung Ihre Hilfe!

\begin{enumerate}
\item Formulieren Sie die wissenschaftliche Fragestellung! \textbf{(1 Punkt)}
\item Erstellen Sie eine Tabelle mit den statistischen Maßzahlen aus der obigen Abbildung der drei Barplots! \textit{Beachten Sie die korrekte Darstellungsform der statistischen Maßzahlen!} \textbf{(3 Punkte)}
\item Erstellen Sie einen beispielhaften Datensatz, aus dem die drei Barplots \textit{möglicherweise} erstellt wurden, im \Rlogo üblichen Format! \textbf{(2 Punkte)}
\item Kann Jessica einen Unterschied zwischen den Behandlungen erwarten? Begründen Sie Ihre Antwort! \textbf{(2 Punkte)}
\end{enumerate} 
\clearpage
% -----------------------------------------------------------------------

\section{Aufgabe \hfill (8 Punkte)}

\textit{Geben Sie grundsätzlich Formeln und Rechenweg zur Lösung der Teilaufgaben mit an!} \\[1Ex]
 

 
%% --------------------------------------------------------------------
\begin{minipage}[t]{0.5\textwidth}
\includegraphics[width = 1.3cm]{/Users/kruppajo/work/GitHub/exam/avatare/Mark.png}
\end{minipage}
\begin{minipage}[t]{0.5\textwidth}
\hfill
\href{https://youtu.be/vXnLttRL_VI}{\includegraphics[width = 2cm]{img/youtube}}\\[1Ex]
\end{minipage}
\vspace{1ex}
%% --------------------------------------------------------------------



Mark steht vor einem ersten Problem, denn wenn es nach seinem Betreuer geht, soll er in einem einem Freilandversuch Erbsen auswertet. Soweit eigentlich alles passend. Besser wäre was anderes gewesen. Mark liebt Geocaching. Darin kann er sich wirklich verlieren und immer wieder neu begeistern. Die Behandlung waren verschiedene Bewässerungstypen ($low$, $mid$ und $high$). In seiner Exceldatei hat er den Messwert ($Y$) \textit{Frischegewicht} als \textit{freshmatter} aufgenommen. Nun soll Mark die Daten eimal als Barplots in einer Präsentation visualisieren, damit seinem Betreuer wieder klar wird, was er eigentlich nochmal gemacht hat und was für ein Ergbnis in einem statistischen Test zu erwarten wäre. Wäre da nicht noch etwas. Eine echte Herausforderung für ihn war schon immer die Unsicherheit gewesen. Ein leidiges Lied. Aber egal. Einfach mal raus um zu Reiten. Ohne Ziel und Uhr. Das ist was für Mark.

\begin{table}[!h]
\centering
\begin{tabular}{cc}
\toprule
treatment & freshmatter\\
\midrule
mid & 43.4\\
high & 33.0\\
mid & 44.5\\
low & 46.2\\
high & 28.2\\
\addlinespace
low & 38.7\\
low & 47.2\\
high & 37.6\\
low & 35.6\\
mid & 47.4\\
\bottomrule
\end{tabular}
\end{table}



Leider kennt sich Mark mit der Erstellung von Barplots nicht aus. Deshalb braucht er bei der Visualisierung Ihre Hilfe!

\begin{enumerate}
\item Formulieren Sie die wissenschaftliche Fragestellung! \textbf{(1 Punkt)}
\item Zeichnen Sie in \textit{einer} Abbildung die Barplots für die Behandlung von Erbsen! Beschriften Sie die Achsen entsprechend!\textbf{(4 Punkte)}
\item Beschriften Sie \textit{einen} Barplot mit den gängigen statistischen Maßzahlen! \textbf{(2 Punkte)}
\item Wenn Mark \textit{keinen Effekt} zwischen den Behandlungen von Erbsen erwarten würde, wie sehen dann die Barplots aus? \textit{Antworten Sie mit einer Skizze der Barplots!}
  \textbf{(1 Punkt)}
\end{enumerate} 
\clearpage
% -----------------------------------------------------------------------

\section{Aufgabe \hfill (9 Punkte)}

\textit{Geben Sie grundsätzlich Formeln und Rechenweg zur Lösung der Teilaufgaben mit an!} \\[1Ex]
 

 
%% --------------------------------------------------------------------
\begin{minipage}[t]{0.5\textwidth}
\includegraphics[width = 1.3cm]{/Users/kruppajo/work/GitHub/exam/avatare/Yuki.png}
\end{minipage}
\begin{minipage}[t]{0.5\textwidth}
\hfill
\href{https://youtu.be/Xf0yE-o7bEU}{\includegraphics[width = 2cm]{img/youtube}}\\[1Ex]
\end{minipage}
\vspace{1ex}
%% --------------------------------------------------------------------



Eine echte Herausforderung für sie war schon immer die Faulheit gewesen. Ein leidiges Lied. Deshalb gilt anschauen, was andere vor einem gemacht haben. Für Yuki ist es eine Möglichkeit schneller ans Ziel zu gelangen. Yuki soll in ihrem Projektbericht Erdbeeren untersuchen. Die Behandlung in ihrem Projektbericht werden verschiedene Genotypen ($AA$, $AB$ und $BB$) sein. Erheben wird Yuki als Messwert ($Y$) \textit{Frischegewicht} benannt als \textit{freshmatter} in ihrer Exceldatei. Von ihrem Betreuer erhält sie nun folgende Abbildung von Boxplots, die sie erstmal zur Übung nachbauen soll, bevor sie mit dem eigentlichen Versuch beginnt. Aber nur in passender Atmospäre! Hm, lecker Reese's Peanut Butter Cups und dazu dann im Hintergrund Matrix laufen lassen.



{\centering \includegraphics[width=\maxwidth]{img/boxplot-02-zer-1} 

}




Leider kennt sich Yuki mit der Erstellung von Boxplots in \Rlogo nicht aus. Deshalb braucht sie bei der Visualisierung Ihre Hilfe!

\begin{enumerate}
\item Erstellen Sie eine Tabelle mit den statistischen Maßzahlen aus der obigen Abbildung der drei Boxplots! \textit{Beachten Sie die korrekte Darstellungsform der statistischen Maßzahlen!} \textbf{(3 Punkte)}
\item Beschriften Sie \textit{einen} der Boxplots mit den gängigen statistischen Maßzahlen! \textbf{(2 Punkte)}
\item Erstellen Sie einen beispielhaften Datensatz, aus dem die drei Boxplots \textit{möglicherweise} erstellt wurden, im \Rlogo üblichen Format! \textbf{(2 Punkte)}
\item Kann Yuki einen Unterschied zwischen den Behandlungen erwarten? Begründen Sie Ihre Antwort! \textbf{(2 Punkte)}
\end{enumerate} 
\clearpage
% -----------------------------------------------------------------------

\section{Aufgabe \hfill (9 Punkte)}

\textit{Geben Sie grundsätzlich Formeln und Rechenweg zur Lösung der Teilaufgaben mit an!} \\[1Ex]
 

 
%% --------------------------------------------------------------------
\begin{minipage}[t]{0.5\textwidth}
\includegraphics[width = 1.3cm]{/Users/kruppajo/work/GitHub/exam/avatare/Jonas.png}
\end{minipage}
\begin{minipage}[t]{0.5\textwidth}
\hfill
\href{https://youtu.be/0xc0jIPeiyw}{\includegraphics[width = 2cm]{img/youtube}}\\[1Ex]
\end{minipage}
\vspace{1ex}
%% --------------------------------------------------------------------



Boxplots sind bedeutend in der Darstellung von wissenschaftlichen Ergebnissen. Leider hat sich Jonas nicht gemerkt, welche statistischen Maßzahlen für einen Boxplot erhoben werden müssen. Besser wäre was anderes gewesen. Jonas liebt Stricken. Darin kann er sich wirklich verlieren und immer wieder neu begeistern. Das ist in soweit doof, da nach seiner Betreuerin nun Boxplots aus seinen Daten gebaut werden sollen, bevor es mit dem statistischen Testen weitergeht. Anhand von Boxplots lässt sich eine Aussage über die Normalverteilung von $Y$ treffen. Die Behandlung für Erdbeeren waren verschiedene Genotypen ($AA$ und $BB$). Erfasst wurde von Jonas als Endpunkt ($Y$) \textit{Trockengewicht}. Jonas hat dann \textit{drymatter} in seiner Exceldatei eintragen. Aber nur in passender Atmospäre! Hm, lecker Snickers und dazu dann im Hintergrund Mission Impossible laufen lassen.

\begin{table}[!h]
\centering
\begin{tabular}{cc}
\toprule
treatment & drymatter\\
\midrule
BB & 39.2\\
BB & 35.5\\
AA & 46.7\\
AA & 45.6\\
AA & 43.4\\
\addlinespace
AA & 46.7\\
BB & 38.6\\
AA & 35.5\\
BB & 26.4\\
BB & 24.8\\
\addlinespace
AA & 49.5\\
BB & 21.0\\
AA & 50.0\\
BB & 35.8\\
BB & 39.0\\
\addlinespace
BB & 36.4\\
BB & 39.2\\
\bottomrule
\end{tabular}
\end{table}



Leider kennt sich Jonas mit der Erstellung von Boxplots nicht aus. Deshalb braucht er bei der Visualisierung Ihre Hilfe!

\begin{enumerate}
\item Zeichnen Sie in \textit{einer} Abbildung die beiden Boxplots für die zwei Behandlungen von Erdbeeren! Beschriften Sie die Achsen entsprechend! \textbf{(5 Punkte)} 
\item Wie ist Ihr Vorgehen, wenn Sie eine \textit{gerade} Anzahl an
  Beobachtungen pro Gruppe haben? \textbf{(1 Punkt)}
\item Beschriften Sie \textit{einen} der beiden Boxplots mit den gängigen
  statistischen Maßzahlen! \textbf{(2 Punkte)}
\item Wenn Sie \textit{keinen Effekt} zwischen den Behandlungen von
  Erdbeeren erwarten würden, wie sehen dann die beiden Boxplots aus?
  \textit{Antworten Sie mit einer Skizze der Boxplots!}
  \textbf{(1 Punkt)}
\end{enumerate} 
\clearpage
% -----------------------------------------------------------------------

\section{Aufgabe \hfill (8 Punkte)}

\textit{Geben Sie grundsätzlich Formeln und Rechenweg zur Lösung der Teilaufgaben mit an!} \\[1Ex]
 

 
%% --------------------------------------------------------------------
\begin{minipage}[t]{0.5\textwidth}
\includegraphics[width = 1.3cm]{/Users/kruppajo/work/GitHub/exam/avatare/Yuki.png}
\end{minipage}
\begin{minipage}[t]{0.5\textwidth}
\hfill
\href{https://youtu.be/aXvxGC4YLqk}{\includegraphics[width = 2cm]{img/youtube}}\\[1Ex]
\end{minipage}
\vspace{1ex}
%% --------------------------------------------------------------------



In seinem Projektbericht möchte Yuki gerne die Daten aus einem Leistungssteigerungsversuch mit Hühnern in einem Histogramm darstellen. Das Histogramm erlaubt ihm dabei Rückschlüsse auf die Verteilung über den Messwert ($Y$) zu treffen. 'Hm...', Reese's Peanut Butter Cups und London Grammar. Das ist und bleibt die beste Kombination zum Nachdenken für Yuki. In seinem Experiment hat Yuki die Anzahl an weißen Blutkörperchen gezählt. Es wäre einfacher, wenn da nicht noch was wäre. Wenn die Faulheit nicht wäre, ja dann wäre wohl vieles möglich für Yuki! Aber so.. Wenn London Grammar ertönt, dann sucht das Minischwein schleunigst Schutz unter dem Sofa. Yuki schüttelt den Kopf.

\begin{center}
Die Anzahl an weißen Blutkörperchen: 9, 4, 5, 3, 6, 5, 8, 7, 7, 4, 2, 8, 1, 5, 2, 6, 1, 3, 6, 3, 2, 5, 3, 6, 5, 5, 5, 5, 3, 2, 5, 6, 2
\end{center}

Leider kennt sich Yuki mit der Erstellung von Histogrammen überhaupt nicht aus. Deshalb braucht er bei der Erstellung Ihre Hilfe!

\begin{enumerate}
\item Zeichen Sie ein Histogramm um die Verteilung der Daten zu visualisieren! (\textbf{3 Punkte})
\item Beschriften Sie die Achsen der Abbildung! (\textbf{2 Punkte})
\item Ergänzen Sie die absoluten und relativen Häufigkeiten in der
  Abbildung! \textbf{(1 Punkt)}
\item Berechnen Sie aus den Daten die \textit{Wahrscheinlichkeit}
  mehr als die Anzahl 6 zu beobachten! \textbf{(1
    Punkt)}
\item Berechnen Sie aus den Daten die \textit{Chance} mehr
  als die Anzahl 6 zu beobachten! \textbf{(1 Punkt)}
\end{enumerate}

 
\clearpage
% -----------------------------------------------------------------------

\section{Aufgabe \hfill (8 Punkte)}

\textit{Geben Sie grundsätzlich Formeln und Rechenweg zur Lösung der Teilaufgaben mit an!} \\[1Ex]
 

 
%% --------------------------------------------------------------------
\begin{minipage}[t]{0.5\textwidth}
\includegraphics[width = 1.3cm]{/Users/kruppajo/work/GitHub/exam/avatare/Alex.png}
\end{minipage}
\begin{minipage}[t]{0.5\textwidth}
\hfill
\href{https://youtu.be/ORHSPTCdfeY}{\includegraphics[width = 2cm]{img/youtube}}\\[1Ex]
\end{minipage}
\vspace{1ex}
%% --------------------------------------------------------------------



Aus den Boxen wummert Abba und sein Mund ist verklebt von Gummibärchen. 'Herrlich', denkt Alex. Alex betrachtet die folgenden Daten nach einem Leistungssteigerungsversuch mit Hühnern. In dem Experiment wurden die mittleren auffälligen Hautflecken gezählt. Nach der Meinung seiner Betreuerin muss als erstes geschaut werden, wie diese verteilt sind. Also welcher statistischen Verteilung die mittleren auffälligen Hautflecken folgen. Dazu soll Alex ein Histogramm verwenden. Dann hätte man auch einen guten Überblick über den Endpunkt ($Y$). Es wäre einfacher, wenn da nicht noch was wäre. Alex und die Gefälligkeit, eine unendliche Geschichte mit kniffeligen Wendungen. Alex streichelt liebevoll die Katze. Der Kopf ist in seinem Schloß vergraben um den Klang von Abba zu dämpfen.

\begin{center}
Die mittleren auffälligen Hautflecken: 12.8, 11.2, 8.2, 7.4, 9.7, 10.4, 7.8, 7.6, 7.9, 7.2, 13.2, 9.7, 11.2, 9.4, 11.1, 9.6, 9.6, 12.2, 11.4, 9.8, 5.9, 10, 9.2, 9.4, 8.7
\end{center}

Leider kennt sich Alex mit der Erstellung von Histogrammen überhaupt nicht aus. Deshalb braucht er bei der Erstellung Ihre Hilfe!

\begin{enumerate}
\item Zeichen Sie ein Histogramm um die Verteilung der Daten zu visualisieren! (\textbf{3 Punkte})
 \item Erläutern Sie Ihr Vorgehen um ein Histogramm für kontinuierliche Daten zu zeichnen!  (\textbf{2 Punkte})
\item Beschriften Sie die Achsen der Abbildung! (\textbf{2 Punkte})
\item Ergänzen Sie die relativen Häufigkeiten in der Abbildung! \textbf{(1 Punkt)}  
\end{enumerate}

 
\clearpage
% -----------------------------------------------------------------------

\section{Aufgabe \hfill (10 Punkte)}

\textit{Geben Sie grundsätzlich Formeln und Rechenweg zur Lösung der Teilaufgaben mit an!} \\[1Ex]
 

 
%% --------------------------------------------------------------------
\begin{minipage}[t]{0.5\textwidth}
\includegraphics[width = 1.3cm]{/Users/kruppajo/work/GitHub/exam/avatare/Steffen.png}
\end{minipage}
\begin{minipage}[t]{0.5\textwidth}
\hfill
\href{https://youtu.be/VAqiUdV4WQ0}{\includegraphics[width = 2cm]{img/youtube}}\\[1Ex]
\end{minipage}
\vspace{1ex}
%% --------------------------------------------------------------------




Steffen liest laut: 'Wenn zwei kontinuierliche Variablen vorliegen, können diese in einer exploartiven Datenanalyse...'. Steffen stoppt. Aus den Boxen wummert Taylor Swift und sein Mund ist verklebt von Oreos. 'Herrlich', denkt Steffen. Was waren noch gleich kontinuierliche Variablen? In seiner Abschlussarbeit hatte er ein Kreuzungsexperiment in der Uckermark durchgeführt. Dabei ging es um den Zusammenhang zwischen Gewichtszuwachs in der 1LW und durchschnittlicher Tagestemperatur [C/d] im groben Kontext von Hühnern. Nun stellt sich die Frage für ihn, ob es überhaupt einen Zusammenhang zwischen den gemessenen Variablen gibt. Dafür war eine explorative Datenanalyse gut! Wenn die Romantik nicht wäre, ja dann wäre wohl vieles möglich für Steffen! Aber so.. Dann was anderes. Das Verrückte ist, dass der Leguan Harry Potter wirklich liebt. Das ist Steffen sehr recht, denn er braucht Entspannung.

\begin{table}[!h]
\centering
\begin{tabular}{cc}
\toprule
Gewichtszuwachs in der 1LW & Durchschnittlicher Tagestemperatur [C/d]\\
\midrule
19.3 & 31.9\\
18.5 & 34.5\\
22.0 & 35.9\\
20.6 & 32.8\\
16.8 & 29.8\\
\addlinespace
20.6 & 35.2\\
21.8 & 36.7\\
22.3 & 37.9\\
18.2 & 34.4\\
19.2 & 34.0\\
\addlinespace
20.9 & 37.0\\
\bottomrule
\end{tabular}
\end{table}



Leider kennt sich Steffen mit der Erstellung einer explorativen Datenanalyse für kontinuierliche Daten überhaupt nicht aus. Deshalb braucht er bei der Erstellung Ihre Hilfe!

\begin{enumerate}
\item Erstellen Sie eine Visualisierung für die Datentabelle. Beschriften Sie
  die Achsen entsprechend! \textbf{(4 Punkte)}
\item Schätzen Sie eine Gerade durch die Punkte! \textbf{(1 Punkt)}
\item Beschriften Sie die Gerade mit den gängigen statistischen Maßzahlen! Geben Sie die numerischen Zahlenwerte mit an! \textbf{(3 Punkte)}
\item Wenn \textit{kein} Effekt von $x$ auf $y$ vorhanden wäre, wie würde die Gerade verlaufen und welche Werte würden die statistischen Maßzahlen annehmen? \textbf{(2 Punkt)}
\end{enumerate} 
\clearpage
% -----------------------------------------------------------------------

\section{Aufgabe \hfill (10 Punkte)}

\textit{Geben Sie grundsätzlich Formeln und Rechenweg zur Lösung der Teilaufgaben mit an!} \\[1Ex]
 

 
%% --------------------------------------------------------------------
\begin{minipage}[t]{0.5\textwidth}
\includegraphics[width = 1.3cm]{/Users/kruppajo/work/GitHub/exam/avatare/Jonas.png}
\end{minipage}
\begin{minipage}[t]{0.5\textwidth}
\hfill
\href{https://youtu.be/t_1KL77mfmg}{\includegraphics[width = 2cm]{img/youtube}}\\[1Ex]
\end{minipage}
\vspace{1ex}
%% --------------------------------------------------------------------



Wenn Mission Impossible läuft, dann ist das Meerschweinchen nicht mehr da. Aber jetzt braucht er mal Entspannung! Aber Ablenkung hilft nur begrenzt. 'Uff!', denkt sich Jonas. Jetzt hat er doch tatsächlich zwei kategoriale Variablen in seiner Hausarbeit gemessen. Zum einen die Behandlung Außenklimakontakt [ja/nein] und zum anderen die Messung Protein/Fettrate im Zielbereich [ja/nein] im Kontext von Hühnern. Hierfür hat er einen Leistungssteigerungsversuch im Wendland durchgeführt. Jetzt möchte Jonas die Daten einmal in einer explorativen Datenanalyse darstellen. Danach kann er dann über den passenden statistischen Test nachdenken. Dabei unterstützt seine Betreuerin diesen Ansatz bevor es in der Datenanalyse weiter geht. So schön wie so gut. Eine echte Herausforderung für ihn war schon immer die Erschöpfung gewesen. Ein leidiges Lied.



\vspace{1Ex}

\begin{center}
\begin{minipage}[t]{0.45\textwidth}
%\small
\begin{center}

\begin{tabular}{p{2.5cm}p{2.5cm}p{2.5cm}p{2.5cm}}
\toprule
Protein/Fettrate im Zielbereich & Außenklimakontakt\\
\midrule
ja & nein\\
nein & ja\\
ja & nein\\
nein & nein\\
ja & nein\\
\addlinespace
nein & ja\\
nein & nein\\
nein & ja\\
nein & ja\\
ja & nein\\
\addlinespace
ja & nein\\
ja & ja\\
nein & nein\\
nein & ja\\
nein & ja\\
\addlinespace
nein & ja\\
\bottomrule
\end{tabular}


\end{center}
\end{minipage}
\begin{minipage}[t]{0.45\textwidth}
%\small
\begin{center}

\begin{tabular}{p{2.5cm}p{2.5cm}p{2.5cm}p{2.5cm}}
\toprule
Protein/Fettrate im Zielbereich & Außenklimakontakt\\
\midrule
ja & nein\\
nein & nein\\
ja & nein\\
ja & nein\\
nein & ja\\
\addlinespace
ja & nein\\
nein & nein\\
nein & ja\\
nein & nein\\
nein & ja\\
\addlinespace
nein & ja\\
ja & ja\\
nein & nein\\
ja & nein\\
ja & nein\\
\addlinespace
ja & ja\\
\bottomrule
\end{tabular}


\end{center}
\end{minipage}
\end{center}

\vspace{2Ex}

Leider kennt sich Jonas mit der Erstellung einer explorativen Datenanalyse für kategoriale Daten überhaupt nicht aus. Deshalb braucht er bei der Erstellung Ihre Hilfe!

\begin{enumerate}
\item Stellen Sie den Zusammenhang zwischen den beiden kategorialen Variablen in einer zusammenfassenden Tabelle dar! \textbf{(3 Punkte)}
\item Visualisieren Sie den Zusammenhang zwischen den beiden kategorialen Variablen! \textbf{(3 Punkte)}
\item Berechnen Sie die Verhältnisse in der Visualisierung! Welche Annahme haben Sie getroffen? \textbf{(2 Punkte)}
\item Wenn \textit{ein} Effekt von der Behandlung vorliegen würde, wie würde die Tabelle und die Visualisierung aussehen? \textbf{(2 Punkt)}
\end{enumerate} 
\clearpage
% -----------------------------------------------------------------------

\section{Aufgabe \hfill (10 Punkte)}

\textit{Geben Sie grundsätzlich Formeln und Rechenweg zur Lösung der Teilaufgaben mit an!} \\[1Ex]
 

 
%% --------------------------------------------------------------------
\begin{minipage}[t]{0.5\textwidth}
\includegraphics[width = 1.3cm]{/Users/kruppajo/work/GitHub/exam/avatare/Nilufar.png}\hspace{-4mm}\includegraphics[width = 1.3cm]{/Users/kruppajo/work/GitHub/exam/avatare/Tina.png}
\end{minipage}
\begin{minipage}[t]{0.5\textwidth}
\hfill
\href{https://youtu.be/Op-gjzASH9I}{\includegraphics[width = 2cm]{img/youtube}}
\end{minipage}
\vspace{1ex}
%% --------------------------------------------------------------------



'Was soll das denn jetzt schon wieder sein? Drei Boxplot, die auf der Seite liegen?', entfährt es Nilufar und schaut dabei Tina an. 'Keine Ahnung. Es ist bestimmt wieder so ein Lernziel mit der Verteilung und so.', meint Tina sichtlich genervt und mampft noch ein paar Katjes. 'Du weißt doch wie es heißt, \textit{Frei ist, wer missfallen kann.}\footnote{Oschmann, A. (2024) Mädchen stärken: Stärken fördern, Selbstwert erhöhen und liebevoll durch Krisen begleiten. Goldegg Verlag}', merkt Nilufar nickend an. Die beiden schauen angestrengt auf die drei Boxplots. Das Ziel ist es zu verstehen, wie eine Verteilung anhand eines Boxplots bewertet werden kann. Tina und die Erwartung machen die Sache nicht einfacher.



{\centering \includegraphics[width=\maxwidth]{img/desc-stat-11-1} 

}




Jetzt brauchen Nilufar und Tina Ihre Hilfe bei der Abschätzung einer Verteilung anhand von Boxplots um ihre Arbeit dann in diesem Semester noch abschließen zu können.

\begin{enumerate}
\item Zeichnen Sie über die Boxplots die entsprechende zugehörige Verteilung! \textbf{(3 Punkte)} 
\item Zeichnen Sie unter die Boxplots die entsprechende zugehörige Beobachtungen als Stiche! \textbf{(3 Punkte)}
\item Wie viel Prozent der Beobachtungen fallen in das IQR? Ergänzen Sie die Abbildung entsprechend um den Bereich! \textbf{(2 Punkte)}
\item Wie viel Prozent der Beobachtungen fallen in $\bar{y} \pm 1s$ und $\bar{y} \pm 2s$  unter der Annahme einer Normalverteilung? \textbf{(2 Punkte)}
\end{enumerate} 
\clearpage
% -----------------------------------------------------------------------

\section{Aufgabe \hfill (10 Punkte)}

\textit{Geben Sie grundsätzlich Formeln und Rechenweg zur Lösung der Teilaufgaben mit an!} \\[1Ex]
 

 
%% --------------------------------------------------------------------
\begin{minipage}[t]{0.5\textwidth}
\includegraphics[width = 1.3cm]{/Users/kruppajo/work/GitHub/exam/avatare/Nilufar.png}\hspace{-4mm}\includegraphics[width = 1.3cm]{/Users/kruppajo/work/GitHub/exam/avatare/Paula.png}
\end{minipage}
\begin{minipage}[t]{0.5\textwidth}
\hfill
\href{https://youtu.be/ZrJhn2wPbq4}{\includegraphics[width = 2cm]{img/youtube}}
\end{minipage}
\vspace{1ex}
%% --------------------------------------------------------------------



Nilufar und die Erwartung machen die Sache mit dem Studium nicht einfacher. Immerhin ist noch Paula zur Hilfe mit dabei. Paula hat Takis Blue Heat mitgebracht und Deichkind aufgedreht. Das ist immerhin eine Ablenkung. Nicht so gut wie Blasmusik, aber immerhin etwas. Jetzt sollen die beiden diese komische Aufgabe lösen. Es geht um verschiedene Normalverteilungen. Anscheinend hängen Normalverteilungen vom Mittelwert $\bar{y}$ und der Standardabweichung $s$ ab. 'Wozu brauchen wir nochmal Normalverteilungen?', entfährt es Nilufar. Durch das Mampfen von Paula versteht sie kein Wort der Antwort. Paula lächelt.\\



Jetzt brauchen Nilufar und Paula Ihre Hilfe bei der Abschätzung einer Verteilung um ihre Arbeit dann in diesem Semester noch abschließen zu können.

\begin{enumerate}
\item Skizzieren Sie zwei Normalverteilungen mit $\bar{y}_1 \neq \bar{y}_2$ und $s_1 \neq s_2$! \textbf{(3 Punkte)}
\item Beschriften Sie die Normalverteilungen mit den statistischen Maßzahlen! \textbf{(2 Punkte)}
\item Liegt Varianzhomogenität oder Varianzheterogenität vor? Begründen Sie Ihre Antwort! \textbf{(2 Punkte)}
\item In welchen Bereich fallen 68\% bzw. 95\% der Beobachtungen in einer Normalverteilung? Ergänzen Sie die Bereiche in \underline{einer} Normalverteilung! \textbf{(2 Punkte)}
\item Ergänzen Sie unter \underline{einer} der Normalverteilungen den entsprechenden Boxplot! \textbf{(1 Punkt)}
\end{enumerate}

 
\clearpage
% -----------------------------------------------------------------------

\section{Aufgabe \hfill (10 Punkte)}

\textit{Geben Sie grundsätzlich Formeln und Rechenweg zur Lösung der Teilaufgaben mit an!} \\[1Ex]
 

 
%% --------------------------------------------------------------------
\begin{minipage}[t]{0.5\textwidth}
\includegraphics[width = 1.3cm]{/Users/kruppajo/work/GitHub/exam/avatare/Nilufar.png}\hspace{-4mm}\includegraphics[width = 1.3cm]{/Users/kruppajo/work/GitHub/exam/avatare/Tina.png}
\end{minipage}
\begin{minipage}[t]{0.5\textwidth}
\hfill
\href{https://youtu.be/MiD42k4l5Ag}{\includegraphics[width = 2cm]{img/youtube}}
\end{minipage}
\vspace{1ex}
%% --------------------------------------------------------------------



'Was sollen wir hier dann noch zeichnen?!', entfährt es Tina und schaut dabei Nilufar an. 'Wir sollen eine Normalverteilung mit einem Mittelwert von $\bar{y}_1 = 2$ und einer Standardabweichung von $s_1 = 4$ zeichnen. Sowie eine weitere Normalverteilung mit einem Mittelwert von $\bar{y}_2 = 0$ und einer Standardabweichung von $s_2 = 4$. Keine Ahnung wie das geht. Darunter sollen dann noch eine Poissonverteilung mit einem Mittelwert von $\lambda_1 = 1$ sowie einer weiteren Poissonverteilung mit einem Mittelwert von $\lambda_2 = 25$ gezeichnet werden.', meint Nilufar sichtlich genervt und mampft noch ein paar Takis Blue Heat. Im Hintergrund spielt leise Deichkind. 'Wirre Geschichte...', merkt Tina nickend an. Die beiden schauen angestrengt auf die leeren Flächen für die Abbildungen. Nilufar und die Wut machen die Suche nach der Lösung nicht einfacher.\\




{\centering \includegraphics[width=\maxwidth]{img/histogram-01-1} 

}




Jetzt brauchen Tina und Nilufar Ihre Hilfe bei der Abschätzung einer Verteilung um ihre Arbeit dann in diesem Semester noch abschließen zu können.


\begin{enumerate}
\item Skizzieren Sie die zwei Normalverteilungen und zwei Poissonverteilungen! \textbf{(4 Punkte)}
\item Achten Sie auf die entsprechende Skalierung in den jeweiligen Abbildungen! \textbf{(2 Punkte)}
\item Ergänzen Sie unter \underline{einer} Normalverteilung den entsprechenden Boxplot! \textbf{(1 Punkt)}
\item Ergänzen Sie unter \underline{einer} Poissonverteilung den entsprechenden Boxplot! \textbf{(1 Punkt)}
\item Geben Sie ein Beispiel für ein Outcome $y$, welches einer Normalverteilung folgt! \textbf{(1 Punkt)}
\item Geben Sie ein Beispiel für ein Outcome $y$, welches einer Poissonverteilung folgt! \textbf{(1 Punkt)}
\end{enumerate} 
\clearpage
% -----------------------------------------------------------------------
\part{Statistisches Testen \& statistische Testtheorie}
% -----------------------------------------------------------------------  

\section{Aufgabe \hfill (9 Punkte)}

%% --------------------------------------------------------------------
\hfill\href{https://youtu.be/aHVYuFKTqZs}{\includegraphics[width =
  2cm]{img/youtube}}\\[1Ex]
%% --------------------------------------------------------------------

Grundlage des statistischen Testen ist das Verst{\"a}ndnis von der
Grundgesamtheit (eng. \textit{population} oder \textit{ground truth}) und
der experimentellen Stichprobe (eng. \textit{sample}). 

\begin{enumerate}
\item Nennen Sie das statistische Verfahren und zwei konkrete Beispiele zur
  Durchf{\"u}hrung um von einer Grundgesamtheit auf eine Stichprobe zu
  gelangen! \textbf{(3 Punkte)}
\item Erkl{\"a}ren Sie den Zusammenhang zwischen Stichprobe und Grundgesamtheit
  an einem Schaubild! Beschriften Sie das Schaubild entsprechend!
  \textit{Nutzen Sie hierf{\"u}r als Veranschaulichung die K{\"o}rpergr{\"o}{\ss}e von
    M{\"a}nnern oder Frauen aus den Gummib{\"a}rchendaten!}  \textbf{(3 Punkte)}
\item Erweitern Sie das Schaubild um die Entstehung von $Pr(D|H_0)$!
  \textit{Nutzen Sie hierf{\"u}r als Veranschaulichung zus{\"a}tzlich die
    Gruppierungsvariable "`Modul"' aus den Gummib{\"a}rchendaten!}  \textbf{(3
    Punkte)}
\end{enumerate} 
\clearpage
% -----------------------------------------------------------------------

\section{Aufgabe \hfill (9 Punkte)}

%% --------------------------------------------------------------------
\hfill\href{https://youtu.be/Ric8ne39DtI}{\includegraphics[width =
  2cm]{img/youtube}}\\[1Ex]
%% --------------------------------------------------------------------




F{\"u}r ein besseres Verst{\"a}ndnis der statistischen Testtheorie, auch
Null-Ritual genannt, kann eine Visualisierung als Kreuztabelle genutzt werden.  

\begin{enumerate}
\item Tragen Sie folgende statistische Fachbegriffe zur statistischen
  Testtheorie korrekt eine selbst erstellte Kreuztabelle ein! \textbf{(3
    Punkte)}
  \begin{center}
  \begin{tabular}{cccc}
  $\beta$-Fehler & 5\% & Testentscheidung & 20\% \\
  \end{tabular}
  \end{center}
\item Erg{\"a}nzen Sie Ihre erstellte Kreuztabelle um vier weitere, passende
  Fachbegriffe zur statistischen Testtheorie! \textbf{(2 Punkte)}
\end{enumerate}

Die Entscheidungsfindung durch einen statistischen Test kann auch durch die
Analogie zu einem Feuermelder abgebildet werden. Dabei symbolisiert der
Feuermelder den statistischen Test und es soll getestet werden, ob ein Feuer
ausgebrochen ist.

\begin{enumerate}
  \setcounter{enumi}{2}    
\item In der Analogie des Feuermelders, wie lautet der $\alpha$-Fehler? \textbf{(1 Punkt)}
\item In der Analogie des Feuermelders, wie lautet der $\beta$-Fehler? \textbf{(1 Punkt)}
\item Wenn der Feuermelder einmal pro Tag messen w{\"u}rde, wie oft w{\"u}rde der
  Feuermelder mit einem $\alpha$ von 5\% in einem Jahr Alarm schlagen?
  Begr{\"u}nden Sie Ihre Antwort! \textbf{(2 Punkte)}
\end{enumerate}



 
\clearpage
% -----------------------------------------------------------------------

\section{Aufgabe \hfill (9 Punkte)}

\textit{Geben Sie grunds{\"a}tzlich Formeln und Rechenweg zur L{\"o}sung der
  Teilaufgaben mit an!} \\[1Ex]

%% --------------------------------------------------------------------
\hfill\href{https://youtu.be/32JjH1eyuTU}{\includegraphics[width =
  2cm]{img/youtube}}\\[1Ex]
%% --------------------------------------------------------------------



Abgebildet ist die t-Verteilung unter der Anahme der G{\"u}ltigkeit der
Nullhypothese. \textit{Beachten Sie, dass im Folgenden keine
  numerisch korrekte Darstellung verlangt wird! Es gilt Erkennbarkeit vor
  Genauigkeit!}

\begin{enumerate}
\item Erg{\"a}nzen Sie eine beschriftete $x$-Achse! \textbf{(1 Punkt)}
\item Erg{\"a}nzen Sie "`$\bar{y}_1 = \bar{y}_2$"'! \textbf{(1 Punkt)} 
\item Erg{\"a}nzen Sie "`$A = 95\%$"'! \textbf{(1 Punkt)}
\item Zeichnen Sie $T_{\alpha=5\%}$ in die Abbildung! \textbf{(1 Punkt)} 
\item Zeichnen Sie das Signifikanzniveau $\alpha$ in die Abbildung! Begr{\"u}nden
  Sie Ihre Antwort! \textbf{(2 Punkte)} 
\item Zeichnen Sie $-T_{D}$ in die Abbildung! \textbf{(1
    Punkt)}
\item Zeichnen Sie einen signifikant p-Wert in die Abbildung! Begr{\"u}nden
  Sie Ihre Antwort! \textbf{(2 Punkte)}   
\end{enumerate}



{\centering \includegraphics[width=\maxwidth]{img/statistisches-testen-3-1} 

}


 
\clearpage
% -----------------------------------------------------------------------

\section{Aufgabe \hfill (10 Punkte)}

%% --------------------------------------------------------------------
\hfill\href{https://youtu.be/CN_O4fYPbhs}{\includegraphics[width =
  2cm]{img/youtube}}\\[1Ex]
%% --------------------------------------------------------------------



Sie rechnen einen t-Test f{\"u}r Gruppenvergleiche der Mittelwerte. Sie
sch{\"a}tzen den Unterschied zwischen dem mittleren Befall mit Parasiten zu einer unbehandelten
Kontrolle.

\begin{enumerate}
\item Beschriften Sie die untenstehende Abbildung mit der
  Signifikanzschwelle! Begr{\"u}nden Sie Ihre Antwort! \textbf{(2 Punkte)}
\item Erg{\"a}nzen Sie eine \textit{in den Kontext passende} Relevanzschwelle!
  Begr{\"u}nden Sie Ihre Antwort! \textbf{(2 Punkte)} 
\item Skizieren Sie in die
  untenstehende Abbildung sechs einzelne Konfidenzintervalle (a-f) mit den
  jeweiligen Eigenschaften! \textbf{(6 Punkte)}
  \begin{itemize}
  \item[(a)] Ein nicht signifikantes, nicht relevantes 95\% Konfidenzintervall 	
  \item[(b)] Ein 95\% Konfidenzintervall mit niedriger Fallzahl $n$ in der Stichprobe als der Rest 95\% der Konfidenzintervalle 	
  \item[(c)] Ein signifikantes, relevantes 99\% Konfidenzintervall. 	
  \item[(d)] Ein signifikantes, nicht relevantes 95\% Konfidenzintervall 
  \item[(e)] Ein 95\% Konfidenzintervall mit h{"o}herer Fallzahl $n$ in der Stichprobe als der Rest der 95\% Konfidenzintervalle
  \item[(f)] Ein signifikantes, relevantes 95\% Konfidenzintervall
  \end{itemize}
\end{enumerate}

\begin{center}
  \includegraphics[height = 12cm]{/Users/kruppajo/work/GitHub/exam/question/img/statistisches-testen-04}
\end{center}


 
\clearpage
% -----------------------------------------------------------------------

\section{Aufgabe \hfill (10 Punkte)}

\textit{Geben Sie grunds{\"a}tzlich Formeln und Rechenweg zur L{\"o}sung der
  Teilaufgaben mit an!} \\[1Ex]

%% --------------------------------------------------------------------
\hfill\href{https://youtu.be/FgZmpnEWDag}{\includegraphics[width =
  2cm]{img/youtube}}\\[1Ex]
%% --------------------------------------------------------------------



Beim statistischen Testen gibt es einen Zusammenhang zwischen dem Effekt,
der Streuung sowie der Fallzahl. Gegeben sei die Formel f{\"u}r den Student
t-Test auf den die folgenden {\"U}berlegungen basieren sollen. Welche
Auswirkung hat die {\"A}nderungen der jeweiligen statistischen Ma{\ss}zahl des
Effekts $\Delta$, der Streuung $s$ und der Fallzahl $n$ auf die Teststistik
$T_{D}$, den p-Wert $Pr(D|H_0)$ sowie dem Konfidenzintervall
$KI_{1-\alpha}$?

\begin{enumerate}
\item Visualisieren Sie den Zusammenhang zwischen der Teststatiatik
  $T_{D}$ und dem p-Wert $Pr(D|H_0)$ f{\"u}r sich ver{\"a}ndernde $T_{D}$-Werte!
  \textit{Geben Sie daf{\"u}r ein numerisches Beispiel in dem Sie drei
    $T_{D}$-Werte und deren Einfluss auf den p-Wert vergleichen!}
  \textbf{(3 Punkte)}  
\item  F{\"u}llen Sie die untenstehende Tabelle aus in dem Sie die {\"A}nderung der
  statistischen Ma{\ss}zahlen auf die Teststatistik, den p-Wert sowie das
  Konfidenzintervall in \textit{einem} Wort oder Symbol beschreiben! \textbf{(4 Punkte)}
\begin{center}
  \large
  \begin{tabular}[c]{l|c|c|c|l|c|c|c}
    & $T_{D}$ & $Pr(D|H_0)$ & $KI_{1-\alpha}$ & & $T_{D}$ & $Pr(D|H_0)$ & $KI_{1-\alpha}$\strut\\ 
    \hline
    \textbf{$\Delta\; \uparrow$} & \hspace{1.8cm} & \hspace{1.8cm}  & \hspace{1.8cm} & \textbf{
                                                          $\Delta\; \downarrow$} &
                                                                          \hspace{1.8cm} & \hspace{1.8cm}  & \hspace{1.8cm}\strut\\
    \hline
        \textbf{$s\; \uparrow$} & \hspace{1.8cm} & \hspace{1.8cm}  & \hspace{1.8cm} & \textbf{
                                                          $s\; \downarrow$} &
                                                                          \hspace{1.8cm}
                                                & \hspace{1.8cm}  & \hspace{1.8cm}\strut\\
    \hline
        \textbf{$n\; \uparrow$} & \hspace{1.8cm} & \hspace{1.8cm}  & \hspace{1.8cm} & \textbf{
                                                          $n\; \downarrow$} &
                                                                          \hspace{1.8cm}
                                                & \hspace{1.8cm}  & \hspace{1.8cm}\strut\\
    \hline
  \end{tabular}
\end{center}
\item Visualisieren Sie ein 95\%-iges Konfidenzintervall im Vergleich
  zu einem 99\%-igen Konfidenzintervall! Begr{\"u}nden Sie Ihre Visualisierung anhand der Formel
  des Konfidenzintervalls des t-Tests mathematisch! \textbf{(3 Punkte)} 
\end{enumerate} 
\clearpage
% -----------------------------------------------------------------------
\part{Der Student t-Test, Welch t-Test \& gepaarter t-Test}
% -----------------------------------------------------------------------

\section{Aufgabe \hfill (9 Punkte)}

\textit{Geben Sie grundsätzlich Formeln und Rechenweg zur Lösung der Teilaufgaben mit an!} \\[1Ex]
 

 
%% --------------------------------------------------------------------
\begin{minipage}[t]{0.5\textwidth}
\includegraphics[width = 1.3cm]{/Users/kruppajo/work/GitHub/exam/avatare/Tina.png}
\end{minipage}
\begin{minipage}[t]{0.5\textwidth}
\hfill
\href{https://youtu.be/eejS2uG4o-M}{\includegraphics[width = 2cm]{img/youtube}}\\[1Ex]
\end{minipage}
\vspace{1ex}
%% --------------------------------------------------------------------



Der t-Test. Tina erschaudert. Eine echte Herausforderung für sie war schon immer die Wut gewesen. Ein leidiges Lied. Ein mächtiges Werkzeug ist der t-Test in den Händen desjenigen, der ein normalverteiltes Outcome ($Y$) hat. Aber erstmal überhaupt den t-Test rechnen können. Wie sah das Experiment von Tina überhaupt aus? Aus den Boxen wummert Tocotronic und ihr Mund ist verklebt von Katjes. 'Herrlich', denkt Tina. Tina hat ein Stallexperiment mit Hühnern durchgeführt um eine neue technische Versuchsanlage zu testen. Bei dem Pilotexperiment mit sehr geringer Fallzahl $(n_1 = n_2 = 3)$ wurde die Behandlung Ernährungszusatz ($ctrl$ und $fedX$) an den Hühnern getestet und dabei wurde geschaut, ob der Versuch überhaupt technisch klappen könnte. Gemessen hat Tina dann als Messwert Schlachtgewicht [kg]. Warum der Versuch im Oldenburger Land für ihrer Hausarbeit stattfinden musste, ist ihr bis heute ein Rätsel. Egal. Gibt es jetzt einen Zusammenhang zwischen der Behandlung und Schlachtgewicht [kg]?

\begin{table}[!h]
\centering
\begin{tabular}{cc}
\toprule
treatment & weight\\
\midrule
ctrl & 16.5\\
dose & 23.4\\
dose & 15.6\\
ctrl & 17.1\\
dose & 18.6\\
\addlinespace
ctrl & 13.8\\
\bottomrule
\end{tabular}
\end{table}



Leider kennt sich Tina mit der Berechnung eines t-Tests überhaupt nicht aus. Deshalb braucht sie bei der Berechnung Ihre Hilfe!

\begin{enumerate}
  \item Formulieren Sie das statistische Hypothesenpaar! \textbf{(1 Punkt)}
  \item Bestimmen Sie die Teststatistik $T_{D}$ eines Welch t-Tests! \textbf{(3 Punkte)}
  \item Treffen Sie mit $T_{\alpha = 5\%} = 2.68$ eine Aussage zur Nullhypothese! Begründen Sie Ihre Antwort! \textbf{(2 Punkte)}
  \item Berechnen Sie den Effekt des Welch t-Tests! \textbf{(1 Punkt)}
  \item Formulieren Sie eine Antwort an Tina über das Ergebnis Ihrer statistischen Analyse! \textbf{(2 Punkte)}
\end{enumerate} 
\clearpage
% -----------------------------------------------------------------------

\section{Aufgabe \hfill (12 Punkte)}

\textit{Geben Sie grundsätzlich Formeln und Rechenweg zur Lösung der Teilaufgaben mit an!} \\[1Ex]
 

 
%% --------------------------------------------------------------------
\begin{minipage}[t]{0.5\textwidth}
\includegraphics[width = 1.3cm]{/Users/kruppajo/work/GitHub/exam/avatare/Paula.png}
\end{minipage}
\begin{minipage}[t]{0.5\textwidth}
\hfill
\href{https://youtu.be/Cq_rF_z4xOk}{\includegraphics[width = 2cm]{img/youtube}}\\[1Ex]
\end{minipage}
\vspace{1ex}
%% --------------------------------------------------------------------



Der Teuteburgerwald, unendliche Weiten. Wir schreiben das Jahr 2024. Dies sind die Abenteuer von Paula, die mit ihrer 1 Frau starken Besatzung 12 Wochen lang unterwegs ist, um neue Welten zu erforschen, neues Leben und neue Zivilisationen. 'Oder nennen wir es Ödnis und Verzweiflung', denkt Paula. Für ihrer Hausarbeit ist Paula ins Nichts gezogen. Eine echte Herausforderung für sie war schon immer der Perfektionismus gewesen. Ein leidiges Lied. Was macht sie nun? Paula hat einen Leistungssteigerungsversuch mit Hühnern durchgeführt. Die Behandlung Flüssignahrung ($ctrl$ und $flOw$) wurde an Hühnern getestet. Gemessen hat sie dann als ein normalverteiltes Outcome ($Y$) Schlachtgewicht [kg]. Jetzt soll sie ihrer Betreuerin nach testen, ob die Behandlung Flüssignahrung ($ctrl$ und $flOw$) ein signifikantes Ergebnis liefert. Hm..., was entspannendes wäre gut. Paula schmeißt noch eine Handvoll Smarties in ihren Rachen. Im Hintergrund klirrt leise der Spiegel zum Sound von White Lies.

\begin{table}[!h]
\centering
\begin{tabular}{cc}
\toprule
Flüssignahrung & Schlachtgewicht\\
\midrule
ctrl & 41.0\\
ctrl & 42.1\\
flOw & 36.1\\
ctrl & 37.5\\
flOw & 53.8\\
\addlinespace
ctrl & 42.5\\
flOw & 9.7\\
flOw & 31.2\\
flOw & 19.6\\
flOw & 25.8\\
\addlinespace
flOw & 41.6\\
ctrl & 42.5\\
flOw & 29.0\\
flOw & 6.2\\
ctrl & 35.4\\
\addlinespace
ctrl & 46.9\\
flOw & 20.5\\
flOw & 18.1\\
\bottomrule
\end{tabular}
\end{table}



Leider kennt sich Paula mit der Berechnung eines t-Tests überhaupt nicht aus. Deshalb braucht sie bei der Berechnung Ihre Hilfe!

\begin{enumerate}
  \item Formulieren Sie die wissenschaftliche Fragestellung! \textbf{(1 Punkt)}
  \item Formulieren Sie das statistische Hypothesenpaar! \textbf{(1 Punkt)}
  \item Bestimmen Sie die Teststatistik $T_{D}$ eines Student t-Tests! \textbf{(3 Punkte)}
\item Treffen Sie mit $T_{\alpha = 5\%} = 2.68$ eine Aussage zur Nullhypothese! Begründen Sie Ihre Antwort! \textbf{(2 Punkte)}
\item Berechnen Sie den Effekt des Student t-Tests! \textbf{(1 Punkt)}
\item Wenn Sie \textit{keinen} Unterschied zwischen den Behandlungsgruppen erwarten würden, wie groß wäre dann der Effekt? Begründen Sie Ihre Antwort! \textbf{(2 Punkte)}
\item Formulieren Sie eine Antwort an Paula über das Ergebnis Ihrer statistischen Analyse! \textbf{(2 Punkte)}
\end{enumerate} 
\clearpage
% -----------------------------------------------------------------------

\section{Aufgabe \hfill (12 Punkte)}

\textit{Geben Sie grundsätzlich Formeln und Rechenweg zur Lösung der Teilaufgaben mit an!} \\[1Ex]
 

 
%% --------------------------------------------------------------------
\begin{minipage}[t]{0.5\textwidth}
\includegraphics[width = 1.3cm]{/Users/kruppajo/work/GitHub/exam/avatare/Yuki.png}
\end{minipage}
\begin{minipage}[t]{0.5\textwidth}
\hfill
\href{https://youtu.be/TbSEOMCQYl4}{\includegraphics[width = 2cm]{img/youtube}}\\[1Ex]
\end{minipage}
\vspace{1ex}
%% --------------------------------------------------------------------



Das Emsland, unendliche Weiten. Wir schreiben das Jahr 2024. Dies sind die Abenteuer von Yuki, die mit ihrer 1 Frau starken Besatzung 12 Wochen lang unterwegs ist, um neue Welten zu erforschen, neues Leben und neue Zivilisationen. 'Oder nennen wir es Ödnis und Verzweiflung', denkt Yuki. Für ihren Projektbericht ist Yuki ins Nichts gezogen. Wenn die Faulheit nicht wäre, ja dann wäre wohl vieles möglich für Yuki! Aber so.. Was macht sie nun? Yuki hat ein Stallexperiment mit Hühnern durchgeführt. Die Behandlung Genotypen ($AA$ und $BB$) wurde an Hühnern getestet. Gemessen hat sie dann als ein normalverteiltes Outcome ($Y$) Protein/Fettrate [\%/kg]. Jetzt soll sie ihrer Betreuerin nach testen, ob die Behandlung Genotypen ($AA$ und $BB$) ein signifikantes Ergebnis liefert. Hm..., was entspannendes wäre gut. Yuki schmeißt noch eine Handvoll Reese's Peanut Butter Cups in ihren Rachen. Im Hintergrund klirrt leise der Spiegel zum Sound von London Grammar.

\begin{table}[!h]
\centering
\begin{tabular}{cc}
\toprule
Genotypen & Protein/Fettrate\\
\midrule
AA & 46.8\\
AA & 40.4\\
AA & 39.1\\
AA & 36.6\\
AA & 44.0\\
\addlinespace
BB & 35.7\\
AA & 28.8\\
AA & 53.1\\
AA & 36.9\\
BB & 35.5\\
\addlinespace
BB & 34.2\\
BB & 40.2\\
BB & 39.0\\
AA & 47.7\\
BB & 48.2\\
\addlinespace
AA & 50.9\\
AA & 37.7\\
BB & 39.1\\
BB & 41.2\\
\bottomrule
\end{tabular}
\end{table}



Leider kennt sich Yuki mit der Berechnung eines t-Tests überhaupt nicht aus. Deshalb braucht sie bei der Berechnung Ihre Hilfe!

\begin{enumerate}
  \item Formulieren Sie die wissenschaftliche Fragestellung! \textbf{(1 Punkt)}
  \item Formulieren Sie das statistische Hypothesenpaar! \textbf{(1 Punkt)}
  \item Bestimmen Sie die Teststatistik $T_{D}$ eines  Welch t-Tests! \textbf{(3 Punkte)}
  \item Treffen Sie mit $T_{\alpha = 5\%} = 2.68$ eine Aussage zur Nullhypothese! Begründen Sie Ihre Antwort! \textbf{(2 Punkte)}
\item Berechnen Sie das 99\% Konfidenzintervall. Welche Annahmen haben Sie getroffen? \textbf{(2 Punkte)}
\item Nennen Sie den statistischen Grund, warum Sie sich zwischen einem Student t-Test und einem Welch t-Test entscheiden müssen! \textbf{(1 Punkt)}
\item Formulieren Sie eine Antwort an Yuki über das Ergebnis Ihrer statistischen Analyse! \textbf{(2 Punkte)}
\end{enumerate} 
\clearpage
% -----------------------------------------------------------------------

\section{Aufgabe \hfill (11 Punkte)}

\textit{Geben Sie grundsätzlich Formeln und Rechenweg zur Lösung der Teilaufgaben mit an!} \\[1Ex]
 

 
%% --------------------------------------------------------------------
\begin{minipage}[t]{0.5\textwidth}
\includegraphics[width = 1.3cm]{/Users/kruppajo/work/GitHub/exam/avatare/Nilufar.png}\hspace{-4mm}\includegraphics[width = 1.3cm]{/Users/kruppajo/work/GitHub/exam/avatare/Paula.png}
\end{minipage}
\begin{minipage}[t]{0.5\textwidth}
\hfill
\href{https://youtu.be/QR90zyn0Iz8}{\includegraphics[width = 2cm]{img/youtube}}
\end{minipage}
\vspace{1ex}
%% --------------------------------------------------------------------



Alles voll mit Hühnern. Das haben Nilufar und Paula gemeinsam in einer Hausarbeit gemacht! Worum ging es aber konkret? Beide haben als ein normalverteiltes Outcome ($Y$) Fettgehalt [\%/kg] von Hühnern bestimmt. Die Daten haben beide zusammen in einem Stallexperiment erhoben. In dem Experiment ging es um eine vorher/nachher Untersuchung an den gleichen Hühnern. Als Behandlung wurde Ernährungszusatz ($ohne$ und $14d$) eingesetzt. Nach der Meinung des Betreuers muss hier ein gepaarter t-Test gerechnet werden. Nilufar schaut nachdenklich zu Paula. Wenn die Erwartung nicht wäre, ja dann wäre wohl vieles möglich für Nilufar! Aber so... Steffen denkt derweil angestrengt an Smarties. Im Hintergrund wummert Deichkind.

\begin{table}[!h]
\centering
\begin{tabular}{ccc}
\toprule
ID & treatment & freshmatter\\
\midrule
8 & 14d & 16.0\\
7 & ohne & 26.7\\
9 & ohne & 27.2\\
2 & 14d & 20.7\\
5 & 14d & 26.9\\
\addlinespace
1 & 14d & 29.7\\
4 & 14d & 19.2\\
8 & ohne & 42.4\\
1 & ohne & 34.1\\
10 & ohne & 40.1\\
\addlinespace
2 & ohne & 41.7\\
5 & ohne & 29.6\\
6 & 14d & 25.9\\
3 & ohne & 33.1\\
11 & ohne & 35.6\\
\addlinespace
3 & 14d & 16.1\\
4 & ohne & 32.3\\
6 & ohne & 41.9\\
7 & 14d & 24.3\\
\bottomrule
\end{tabular}
\end{table}



Leider kennen sich Nilufar und Paula mit der Berechnung eines gepaarten t-Tests überhaupt nicht aus. Deshalb brauchen sie beide bei der Berechnung Ihre Hilfe!

\begin{enumerate}
  \item Formulieren Sie die wissenschaftliche Fragestellung! \textbf{(1 Punkt)}
  \item Formulieren Sie das statistische Hypothesenpaar! \textbf{(1 Punkt)}
  \item Bestimmen Sie die Teststatistik $T_{D}$ eines gepaarten t-Tests! \textbf{(3 Punkte)}
  \item Treffen Sie mit $T_{\alpha = 5\%} = 1.84$ eine Aussage zur Nullhypothese! Begründen Sie Ihre Antwort! \textbf{(2 Punkte)}
\item Schätzen Sie den $p$-Wert des gepaarten t-Tests ab! Begründen Sie Ihre Antwort mit einer Skizze! \textbf{(2 Punkte)}
\item Formulieren Sie eine Antwort an Nilufar über das Ergebnis Ihrer statistischen Analyse! \textbf{(2 Punkte)}
\end{enumerate}


 
\clearpage
% -----------------------------------------------------------------------

\section{Aufgabe \hfill (10 Punkte)}

\textit{Geben Sie grundsätzlich Formeln und Rechenweg zur Lösung der Teilaufgaben mit an!} \\[1Ex]
 

 
%% --------------------------------------------------------------------
\begin{minipage}[t]{0.5\textwidth}
\includegraphics[width = 1.3cm]{/Users/kruppajo/work/GitHub/exam/avatare/Alex.png}\hspace{-4mm}\includegraphics[width = 1.3cm]{/Users/kruppajo/work/GitHub/exam/avatare/Paula.png}\hspace{-4mm}\includegraphics[width = 1.3cm]{/Users/kruppajo/work/GitHub/exam/avatare/Steffen.png}
\end{minipage}
\begin{minipage}[t]{0.5\textwidth}
\hfill
\href{https://youtu.be/exDo7AyHl4Q}{\includegraphics[width = 2cm]{img/youtube}}
\end{minipage}
\vspace{1ex}
%% --------------------------------------------------------------------



\Rlogo ist schon ein tolles Programm, wenn man mit dem Ding umgehen kann. Super umgehen kann damit Steffen. Eine echte Herausforderung für Steffen ist die Romantik, aber das ist noch eine andere Sache. Deshalb sind aber Alex und Paula nicht bei ihm. Sondern um sich bei einem gemeinsamen Projekt helfen zu lassen. Im Hintergrund wummert Taylor Swift. Beide arbeiten gemeinsam an einer Hausarbeit. In dem zu beschreibenden Versuch geht es im Oldenburger Land um einem Kreuzungsexperiment mit Hühnern. Dabei ging darum herauszufinden, ob es einen Zusammenhang zwischen der Behandlung Genotypen ($AA$ und $BB$) und dem Messwert Schlachtgewicht [kg] gibt. Da der Messwert Schlachtgewicht [kg] normalverteilt ist kann ein t-Test gerechnet werden. Alex möchte dann später noch mehr über Steffens Hobby Klemmbausteine erfahren.

\begin{knitrout}
\definecolor{shadecolor}{rgb}{0.969, 0.969, 0.969}\color{fgcolor}\begin{kframe}
\begin{verbatim}
## 
## 	Two Sample t-test
## 
## data:  Schlachtgewicht by Genotypen
## t = -6.5175, df = 20, p-value = 2.366e-06
## alternative hypothesis: true  is not equal to [condensed]
## 95 percent confidence interval:
##  -12.084508  -6.224583
## sample estimates:
## mean in group AA mean in group BB 
##         25.06364         34.21818
\end{verbatim}
\end{kframe}
\end{knitrout}

Helfen Sie Steffen bei der Interpretation des t-Tests! Sonst geht es auch für Alex und Paula nicht weiter.
  
\begin{enumerate}
  \item Formulieren Sie die wissenschaftliche Fragestellung! \textbf{(1 Punkt)}
  \item Formulieren Sie das statistische Hypothesenpaar! \textbf{(1 Punkt)}
\item Liegt ein signifikanter Unterschied zwischen den Gruppen vor? Begründen Sie Ihre Antwort! \textbf{(2 Punkte)}
\item Skizzieren Sie eine Abbildung in der Sie $T_{D}$, $Pr(D|H_0)$, $A=0.95$, sowie $T_{\alpha=5\%} = |2.09|$ einzeichnen! \textbf{(4 Punkte)}
\item Beschriften Sie die Abbildung! \textbf{(1 Punkt)}  
\item Berechnen Sie den Effekt des t-Tests! \textbf{(1 Punkt)}
\end{enumerate} 
\clearpage
% -----------------------------------------------------------------------

\section{Aufgabe \hfill (10 Punkte)}

\textit{Geben Sie grundsätzlich Formeln und Rechenweg zur Lösung der Teilaufgaben mit an!} \\[1Ex]
 

 
%% --------------------------------------------------------------------
\begin{minipage}[t]{0.5\textwidth}
\includegraphics[width = 1.3cm]{/Users/kruppajo/work/GitHub/exam/avatare/Jonas.png}\hspace{-4mm}\includegraphics[width = 1.3cm]{/Users/kruppajo/work/GitHub/exam/avatare/Nilufar.png}\hspace{-4mm}\includegraphics[width = 1.3cm]{/Users/kruppajo/work/GitHub/exam/avatare/Tina.png}
\end{minipage}
\begin{minipage}[t]{0.5\textwidth}
\hfill
\href{https://youtu.be/wJqsNV1hOW8}{\includegraphics[width = 2cm]{img/youtube}}
\end{minipage}
\vspace{1ex}
%% --------------------------------------------------------------------



'Mit dem R Paket \texttt{\{emmeans\}} können wir gleich die Gruppenvergleiche rechnen und uns das \textit{compact letter displac}' ausgeben lassen!', verkündet Nilufar sichtlich stolz. Ein paar Mal hat sie schon die Erwartung gehindert weiterzumachen. 'Nach Meinung des Betreuers soll es aber nur erstmal ein t-Test sein. Und die Ausgabe ist schon wirr genug.', merkt Tina an. Tina und Jonas sind bei Nilufar um sich in \Rlogo helfen zu lassen. Im Hintergrund wummert Deichkind. Jonas streichelt zur Beruhigung das Huhn von Nilufar. Die beiden waren 3 Monate im Wendland um einen Versuch mit Hühnern in einem Kreuzungsexperiment durchzuführen. Ziel war es das Outcome ($Y$) Fettgehalt [\%/kg] zu bestimmen. Nilufar überlegt, ob sie die beiden nicht noch auf den Film \textit{Star Trek} einlädt oder dann doch lieber raus geht um zu Kicken? Vielleicht will ja Jonas mit. Besser als der Film.

\begin{knitrout}
\definecolor{shadecolor}{rgb}{0.969, 0.969, 0.969}\color{fgcolor}\begin{kframe}
\begin{verbatim}
## 
## 	Two Sample t-test
## 
## data:  Fettgehalt by Genotypen
## t = -0.46185, df = 16, p-value = 0.6504
## alternative hypothesis: true  is not equal to [condensed]
## 95 percent confidence interval:
##  -9.321401  5.986401
## sample estimates:
## mean in group AA mean in group BB 
##          38.2700          39.9375
\end{verbatim}
\end{kframe}
\end{knitrout}

Helfen Sie Nilufar bei der Interpretation des t-Tests! Sonst geht es auch für Tina und Jonas nicht weiter.

\begin{enumerate}
  \item Formulieren Sie die wissenschaftliche Fragestellung! \textbf{(1 Punkt)}
  \item Formulieren Sie das statistische Hypothesenpaar! \textbf{(1 Punkt)}
\item Liegt ein signifikanter Unterschied zwischen den Gruppen vor? Begründen Sie Ihre Antwort! \textbf{(2 Punkte)}
\item Skizieren Sie das sich ergebende 95\% Konifidenzintervall! \textbf{(2 Punkte)}
\item Beschriften Sie die Abbildung und das 95\% Konfidenzintervall entsprechend! \textbf{(2 Punkte)}  
\item Interpretieren Sie den Effekt des 95\% Konifidenzintervalls! \textbf{(2 Punkte)}
\end{enumerate} 
\clearpage
% -----------------------------------------------------------------------

\section{Aufgabe \hfill (9 Punkte)}

\textit{Geben Sie grundsätzlich Formeln und Rechenweg zur Lösung der Teilaufgaben mit an!} \\[1Ex]
 

 
%% --------------------------------------------------------------------
\begin{minipage}[t]{0.5\textwidth}
\includegraphics[width = 1.3cm]{/Users/kruppajo/work/GitHub/exam/avatare/Alex.png}\hspace{-4mm}\includegraphics[width = 1.3cm]{/Users/kruppajo/work/GitHub/exam/avatare/Tina.png}\hspace{-4mm}\includegraphics[width = 1.3cm]{/Users/kruppajo/work/GitHub/exam/avatare/Yuki.png}
\end{minipage}
\begin{minipage}[t]{0.5\textwidth}
\hfill
\href{https://youtu.be/w62HJlbN28U}{\includegraphics[width = 2cm]{img/youtube}}
\end{minipage}
\vspace{1ex}
%% --------------------------------------------------------------------



'Wir waren im Teuteburgerwald um Hühnern in einem Kreuzungsexperiment zu messen.', Alex legt das Dokument auf den Tisch und schaut Yuki und Tina fragend an. Beide schauen fragend zurück. Gäbe es die Wut nicht, dann wäre es für Tina irgendwie einfacher hier zu helfen. Echt unangenehm. Die beiden sind zu Alex gekommen, da sie sich nicht mit \Rlogo auskennen und daher Hilfe bei der Interpretation des t-Tests brauchen. Im Hintergrund wummert Abba und leere Gummibärchen Packungen stappeln sich auf dem Boden. 'Kein Problem', sagt Alex und streichelt langsam die Katze. 'Aber worum es in dem Versuch geht, lässt sich nur aus dem Text in seiner Hand erahnen.' merkt er an. Vielleicht hilft da ja die Ausgabe des t-Tests in R weiter. Draußen geht blutrot die Sonne unter.

\begin{knitrout}
\definecolor{shadecolor}{rgb}{0.969, 0.969, 0.969}\color{fgcolor}\begin{kframe}
\begin{verbatim}
## 
## 	Two Sample t-test
## 
## data:  Protein/Fettrate by Ernährungszusatz
## t = 4.0484, df = 16, p-value = 0.0009322
## alternative hypothesis: true  is not equal to [condensed]
## 95 percent confidence interval:
##   8.458844 27.055442
## sample estimates:
## mean in group ctrl mean in group fedX 
##           44.35714           26.60000
\end{verbatim}
\end{kframe}
\end{knitrout}

Helfen Sie Alex bei der Interpretation des t-Tests! Sonst geht es auch für Yuki und Tina nicht weiter.
  
\begin{enumerate}
  \item Formulieren Sie die wissenschaftliche Fragestellung! \textbf{(1 Punkt)}
  \item Formulieren Sie das statistische Hypothesenpaar! \textbf{(1 Punkt)}
\item Liegt ein signifikanter Unterschied zwischen den Gruppen vor?
  Begründen Sie Ihre Antwort! \textbf{(2 Punkte)}
\item Skizieren Sie die sich ergebenden Boxplot!
  Welche Annahmen an die Daten haben Sie getroffen? Begründen Sie Ihre
  Antwort! \textbf{(2 Punkte)} 
\item Skizieren Sie die sich ergebenden Barplots! \textbf{(2 Punkte)}
\item Berechnen Sie den Effekt des t-Tests! \textbf{(1 Punkt)}
\end{enumerate}
 
\clearpage
% -----------------------------------------------------------------------

\section{Aufgabe \hfill (10 Punkte)}

\textit{Geben Sie grundsätzlich Formeln und Rechenweg zur Lösung der Teilaufgaben mit an!} \\[1Ex]
 

 
%% --------------------------------------------------------------------
\begin{minipage}[t]{0.5\textwidth}
\includegraphics[width = 1.3cm]{/Users/kruppajo/work/GitHub/exam/avatare/Mark.png}\hspace{-4mm}\includegraphics[width = 1.3cm]{/Users/kruppajo/work/GitHub/exam/avatare/Yuki.png}
\end{minipage}
\begin{minipage}[t]{0.5\textwidth}
\hfill
\href{https://youtu.be/kHmfEmU6lrk}{\includegraphics[width = 2cm]{img/youtube}}
\end{minipage}
\vspace{1ex}
%% --------------------------------------------------------------------



Mark und Yuki haben sich dazu entschieden zusammenzuarbeiten. Das sollte alles etwas einfacher machen. Jeder hat zwar ein getrenntes Themenfeld aber den Hauptversuch machen beide gemeinsam. Das hat sich schonmal als gut Idee soweit herausgestellt. In einer Abschlussarbeit sollen beide herausfinden, ob es einen Zusammenhang zwischen Bestandsdichte ($hoch$ und $niedrig$) und Protein/Fettrate [\%/kg] gibt. Die Besonderheit ist hierbei, dass die Messungen an der gleichen Beobachtung stattfinden. Beide messen also zweimal an den gleichen Hühnern. Hier muss dann wohl auf ein normalverteiltes Outcome ($Y$) ein gepaarter t-Test gerechnet werden. Leider kennen sich beide nicht sehr gut in \Rlogo aus.

\begin{knitrout}
\definecolor{shadecolor}{rgb}{0.969, 0.969, 0.969}\color{fgcolor}\begin{kframe}
\begin{verbatim}
## 
## 	Paired t-test
## 
## data:  Protein/Fettrate by Bestandsdichte
## t = -4.256, df = 8, p-value = 0.002776
## alternative hypothesis: true  is not equal to [condensed]
## 95 percent confidence interval:
##  -16.754508  -4.978825
## sample estimates:
## mean difference 
##       -10.86667
\end{verbatim}
\end{kframe}
\end{knitrout}

Jetzt brauchen Mark und Yuki Ihre Hilfe bei der Berechnung eines gepaarten t-Tests in \Rlogo um ihre Arbeit dann in diesem Semester noch abschließen zu können.

\begin{enumerate}
  \item Formulieren Sie die wissenschaftliche Fragestellung! \textbf{(1 Punkt)}
  \item Formulieren Sie das statistische Hypothesenpaar! \textbf{(1 Punkt)}
\item Liegt ein signifikanter Unterschied zwischen den Gruppen vor?
  Begründen Sie Ihre Antwort! \textbf{(2 Punkte)}
\item Skizzieren Sie das sich ergebende 95\% Konfidenzintervall! \textbf{(2 Punkte)}
\item Interpretieren Sie den Effekt des gepaarten t-Tests! \textbf{(2 Punkte)}
\item Skizzieren Sie den sich ergebenden Boxplot der Differenzen! Welche Annahmen an die Daten haben Sie getroffen? Begründen Sie Ihre Antwort! \textbf{(2 Punkte)} 
\end{enumerate}
 
\clearpage
% -----------------------------------------------------------------------
\part{Die einfaktorielle \& zweifaktorielle ANOVA}
% -----------------------------------------------------------------------

\section{Aufgabe \hfill (11 Punkte)}

\textit{Geben Sie grundsätzlich Formeln und Rechenweg zur Lösung der Teilaufgaben mit an!} \\[1Ex]
 

 
%% --------------------------------------------------------------------
\begin{minipage}[t]{0.5\textwidth}
\includegraphics[width = 1.3cm]{/Users/kruppajo/work/GitHub/exam/avatare/Jessica.png}\hspace{-4mm}\includegraphics[width = 1.3cm]{/Users/kruppajo/work/GitHub/exam/avatare/Nilufar.png}
\end{minipage}
\begin{minipage}[t]{0.5\textwidth}
\hfill
\href{https://youtu.be/kHmfEmU6lrk}{\includegraphics[width = 2cm]{img/youtube}}
\end{minipage}
\vspace{1ex}
%% --------------------------------------------------------------------



'Uff... die einfaktorielle ANOVA. Und wir können jetzt anhand der Visualisuierung sehen, ob da schon was signifikant ist?', Jessica hebt die Augenbraue. 'Ja, können wir. Dafür müssen wir aber erstmal in \texttt{\{ggplot\}} uns die Daten anschauen. Oder wir zeichnen es flott mit der Hand. Geht auch.', meint Nilufar dazu. Jessica hatte sich in ein Kreuzungsexperiment verschiedene Hühnern angeschaut. Dabei ging es herauszufinden, ob es einen Zusammenhang zwischen der Behandlung Bestandsdichte ($eng$, $weit$ und $kontakt$) und dem Messwert Protein/Fettrate [\%/kg] gibt.

\begin{knitrout}
\definecolor{shadecolor}{rgb}{0.969, 0.969, 0.969}\color{fgcolor}\begin{table}[!h]
\centering
\begin{tabular}{cc}
\toprule
Bestandsdichte & Protein/Fettrate\\
\midrule
weit & 40\\
eng & 35\\
eng & 35\\
eng & 36\\
weit & 39\\
\addlinespace
eng & 33\\
kontakt & 25\\
eng & 34\\
weit & 40\\
eng & 34\\
\addlinespace
eng & 33\\
kontakt & 25\\
kontakt & 26\\
weit & 40\\
weit & 40\\
\addlinespace
kontakt & 25\\
kontakt & 25\\
\bottomrule
\end{tabular}
\end{table}

\end{knitrout}

Leider kennen sich Jessica und Nilufar mit Darstellung einer einfaktoriellen ANOVA überhaupt nicht aus. Deshalb brauchen beide bei der Erstellung Ihre Hilfe! 

\begin{enumerate}
\item Erstellen  Sie  eine  Visualisierung  der  Datentabelle! Beschriften  Sie  die  Abbildung! \textbf{(2 Punkte)}
\item Benennen Sie die Visualisierung mit dem korrekten, statistischen Fachbegriff! \textbf{(1 Punkt)}
\item Zeichnen Sie folgende statistischen Maßzahlen passend ein! 
  \begin{itemize}
  \item Globale Mittelwert: $\beta_0$ \textbf{(1 Punkt)}
  \item Mittelwerte der einzelnen Behandlungsstufen: $\bar{y}_{0.5}$, $\bar{y}_{1.5}$ und $\bar{y}_{2.5}$ \textbf{(1 Punkt)}
  \item Mittelwertsdifferenz der einzelnen Behandlungsstufen: $\beta_{0.5}$, $\beta_{1.5}$ und $\beta_{2.5}$ \textbf{(1 Punkt)}
  \item Residuen oder Fehler: $\epsilon$ \textbf{(1 Punkt)}
  \end{itemize}
\item Liegt ein \textit{vermutlicher} signifikanter Unterschied vor? Begründen Sie Ihre Antwort! \textbf{(2 Punkte)}
\item Schätzen Sie die Effekte der Behandlungsstufen! \textbf{(2 Punkte)}
\end{enumerate}
 
\clearpage
% -----------------------------------------------------------------------

\section{Aufgabe \hfill (9 Punkte)}

\textit{Geben Sie grundsätzlich Formeln und Rechenweg zur Lösung der Teilaufgaben mit an!} \\[1Ex]
 

 
%% --------------------------------------------------------------------
\begin{minipage}[t]{0.5\textwidth}
\includegraphics[width = 1.3cm]{/Users/kruppajo/work/GitHub/exam/avatare/Jonas.png}\hspace{-4mm}\includegraphics[width = 1.3cm]{/Users/kruppajo/work/GitHub/exam/avatare/Tina.png}
\end{minipage}
\begin{minipage}[t]{0.5\textwidth}
\hfill
\href{https://youtu.be/IhecxMcCENY}{\includegraphics[width = 2cm]{img/youtube}}
\end{minipage}
\vspace{1ex}
%% --------------------------------------------------------------------



'Als erstes bauen wir uns aus unsere Daten die ANOVA Tabelle dann sehen wir schon, ob unser Gruppenvergleich in der ANOVA signifikant ist.', Jonas schaut Tina fragend an und hofft auf eine positive Regung im Gesicht. Wird aber enttäuscht. Tina tut sich auch sehr schwer mit der einfaktoriellen ANOVA. Beide waren im Oldenburger Land um ein Stallexperiment mit Hühnern durchzuführen. Dabei ging es herauszufinden, ob es einen Zusammenhang zwischen der Behandlung Bestandsdichte ($standard$, $eng$, $weit$ und $kontakt$) und dem Messwert Protein/Fettrate [\%/kg] gibt.



\vspace{1ex}

Leider kennen sich Jonas und Tina mit Berechnung einer einfaktoriellen ANOVA überhaupt nicht aus. Deshalb brauchen beide bei der Erstellung Ihre Hilfe! 

\begin{enumerate}
  \item Formulieren Sie die wissenschaftliche Fragestellung! \textbf{(1 Punkt)}
  \item Formulieren Sie das statistische Hypothesenpaar! \textbf{(1 Punkt)}
\item Füllen Sie die unterstehende einfaktorielle ANOVA Ergebnistabelle aus! \textbf{(3 Punkte)}
\end{enumerate}

\vspace{1Ex}

\begin{center}
  \Large
  \begin{tabular}{lccccp{3cm}}
\toprule
     & \textbf{Df} & \textbf{Sum Sq} & \textbf{Mean Sq} & \textbf{F value} & \textbf{Pr(>F)} \strut\\
    \midrule
   \textbf{Bestandsdichte}  & 3 & 15.35 &  &  &  \strut\\
   \textbf{error}  & 21 & 312.89 &  &  &  \strut\\
   \textbf{Total}  & 24 &  &  &  &  \strut\\
\bottomrule
  \end{tabular}
\end{center}

\vspace{1Ex}

\begin{enumerate}
  \setcounter{enumi}{3}
\item Schätzen Sie den p-Wert der Tabelle mit $F_{\alpha = 5\%} = 3.07$ ab. Begründen Sie Ihre Antwort! \textbf{(2 Punkte)}
\item Berechen Sie den Effektschätzer $\eta^2$. Was sagt Ihnen der Wert von $\eta^2$ aus? \textbf{(2 Punkte)}
\end{enumerate}



 
\clearpage
% -----------------------------------------------------------------------

\section{Aufgabe \hfill (12 Punkte)}

\textit{Geben Sie grundsätzlich Formeln und Rechenweg zur Lösung der Teilaufgaben mit an!} \\[1Ex]
 

 
%% --------------------------------------------------------------------
\begin{minipage}[t]{0.5\textwidth}
\includegraphics[width = 1.3cm]{/Users/kruppajo/work/GitHub/exam/avatare/Jessica.png}\hspace{-4mm}\includegraphics[width = 1.3cm]{/Users/kruppajo/work/GitHub/exam/avatare/Tina.png}
\end{minipage}
\begin{minipage}[t]{0.5\textwidth}
\hfill
\href{https://youtu.be/49hvImMwVyE}{\includegraphics[width = 2cm]{img/youtube}}
\end{minipage}
\vspace{1ex}
%% --------------------------------------------------------------------



Tina und Jessica schauen sich etwas entnervt an. Gemeinsam schreiben die beiden ihre Abschlussarbeit und sollen nun als erstes einmal die Daten mit eine einfaktoriellen ANOVA auswerten damit abgeschätzt werden kann, ob überhaupt signifikante Ergebnisse in den multipen Gruppenvergleichen zu erwarten sind. Nun möchte erstmal ihre Betreuung der Arbeit eine ANOVA Tabelle sehen. Was immer da auch drin zu erkennen sein mag. Tina schaut Jessica sehen erstmla gar nichts. Die beiden waren im Oldenburger Land um einen Leistungssteigerungsversuch mit Hühnern durchzuführen. Dabei haben Tina und Jessica den Messwert Gewichtszuwachs in der 1LW unter der Behandung Ernährungszusatz ($ctrl$, $fedX$ und $getIt$) ermittelt.



\vspace{1ex}

Leider kennen sich Tina und Jessica mit Berechnung einer einfaktoriellen ANOVA überhaupt nicht aus. Deshalb brauchen beide bei der Erstellung Ihre Hilfe! 

\begin{enumerate}
  \item Formulieren Sie die wissenschaftliche Fragestellung! \textbf{(1 Punkt)}
  \item Formulieren Sie das statistische Hypothesenpaar! \textbf{(1 Punkt)}
\item Füllen Sie die unterstehende einfaktorielle ANOVA Ergebnistabelle aus! \textbf{(3 Punkte)}
\end{enumerate}

\vspace{1Ex}

\begin{center}
  \Large
  \begin{tabular}{lccccp{3cm}}
\toprule
     & \textbf{Df} & \textbf{Sum Sq} & \textbf{Mean Sq} & \textbf{F value} & \textbf{Pr(>F)} \strut\\
    \midrule
   \textbf{Ernährungszusatz}  & 2 & 5593.58 &  &  &  \strut\\
   \textbf{Error}  & 21 & 694.38 &  &  &  \strut\\
\bottomrule
  \end{tabular}
\end{center}

\vspace{1Ex}

\begin{enumerate}
  \setcounter{enumi}{3}
\item Schätzen Sie den p-Wert der Tabelle mit $F_{\alpha = 5\%} = 3.47$ ab. Begründen Sie Ihre Antwort! \textbf{(2 Punkte)}
\item Was bedeutet ein signifikantes Ergebnis in einer einfaktoriellen ANOVA? \textbf{(1 Punkt)}
\item Berechnen Sie \textit{einen} Student t-Test für den \textit{vermutlich} signifikantesten Gruppenvergleich anhand der untenstehenden Tabelle mit $T_{\alpha = 5\%} = 2.03$. Begründen Sie Ihre Auswahl! \textbf{(3 Punkte)}
\end{enumerate}


\begin{knitrout}
\definecolor{shadecolor}{rgb}{0.969, 0.969, 0.969}\color{fgcolor}\begin{table}[!h]
\centering\begingroup\fontsize{11}{13}\selectfont

\begin{tabular}{cccc}
\toprule
\textbf{Ernährungszusatz} & \textbf{Fallzahl (n)} & \textbf{Mittelwert} & \textbf{Standardabweichung}\\
\midrule
ctrl & 8 & 5.12 & 2.10\\
fedX & 8 & 35.25 & 7.52\\
getIt & 8 & 1.00 & 6.19\\
\bottomrule
\end{tabular}
\endgroup{}
\end{table}

\end{knitrout}


\begin{enumerate}
  \setcounter{enumi}{6}
\item Gegebenen der ANOVA Tabelle war das Ergebnis des Student t-Tests zu erwarten? Begründen Sie Ihre Antwort! \textbf{(2 Punkte)}
\end{enumerate}

 
\clearpage
% -----------------------------------------------------------------------

\section{Aufgabe \hfill (9 Punkte)}

\textit{Geben Sie grundsätzlich Formeln und Rechenweg zur Lösung der Teilaufgaben mit an!} \\[1Ex]
 

 
%% --------------------------------------------------------------------
\begin{minipage}[t]{0.5\textwidth}
\includegraphics[width = 1.3cm]{/Users/kruppajo/work/GitHub/exam/avatare/Nilufar.png}
\end{minipage}
\begin{minipage}[t]{0.5\textwidth}
\hfill
\href{https://youtu.be/aXvxGC4YLqk}{\includegraphics[width = 2cm]{img/youtube}}\\[1Ex]
\end{minipage}
\vspace{1ex}
%% --------------------------------------------------------------------



Nilufar schaut entnervt auf und klappt den Laptop zu. Nun möchte ihr Betreuer ihrem Projektbericht erstmal eine ANOVA sehen und \textit{dann} die Ergebnisse präsentiert bekommen bevor es überhaupt mit der Abschlussarbeit weitergeht. Dabei war sie extra im Oldenburger Land um ein Kreuzungsexperiment mit Hühnern durchzuführen. Und dort was es wirklich nicht schön geschweige denn spannend wie bei ihren Kommilitonen, die in Almería waren. Hätte sie es vorher gewusst, dann hätte sie die Abschlussarbeit bei wem anders geschrieben. Aber gut, jetzt als die ANOVA in \Rlogo.

\begin{knitrout}
\definecolor{shadecolor}{rgb}{0.969, 0.969, 0.969}\color{fgcolor}\begin{kframe}
\begin{verbatim}
## Analysis of Variance Table
## 
## Response: Gewichtszuwachs
##            Df Sum Sq Mean Sq F value    Pr(>F)
## Elterlinie  3 967.08  322.36  12.936 2.311e-05
## Residuals  26 647.89   24.92
\end{verbatim}
\end{kframe}
\end{knitrout}

\vspace{1ex}

Leider kennen sich Nilufar mit Berechnung einer einfaktoriellen ANOVA überhaupt nicht aus. Deshalb braucht sie bei der Erstellung Ihre Hilfe! 

\begin{enumerate}
  \item Formulieren Sie die wissenschaftliche Fragestellung! \textbf{(1 Punkt)}
  \item Formulieren Sie das statistische Hypothesenpaar! \textbf{(1 Punkt)}
\item Interpretieren Sie das Ergebnis der einfaktoriellen ANOVA! \textbf{(2 Punkte)} 
\item Berechnen Sie den Effektschätzer $\eta^2$. Was sagt Ihnen der Wert von $\eta^2$ aus? \textbf{(2 Punkte)}
\item Skizzieren Sie eine Abbildung, der dem obigen Ergebnis der
  einfaktoriellen ANOVA näherungsweise entspricht! \textbf{(3 Punkte)}
\end{enumerate}

 
\clearpage
% -----------------------------------------------------------------------

\section{Aufgabe \hfill (12 Punkte)}

\textit{Geben Sie grundsätzlich Formeln und Rechenweg zur Lösung der Teilaufgaben mit an!} \\[1Ex]
 

 
%% --------------------------------------------------------------------
\begin{minipage}[t]{0.5\textwidth}
\includegraphics[width = 1.3cm]{/Users/kruppajo/work/GitHub/exam/avatare/Steffen.png}
\end{minipage}
\begin{minipage}[t]{0.5\textwidth}
\hfill
\href{https://youtu.be/8Pb2sKUIMyk}{\includegraphics[width = 2cm]{img/youtube}}\\[1Ex]
\end{minipage}
\vspace{1ex}
%% --------------------------------------------------------------------



Wie absolut ärgerlich. Jetzt stellt sich tatsächlich heraus, dass seinem Betreuer keine Anhnung von der zweifaktoriellen ANOVA hat. Woher soll Steffen jetzt das Wissen nehmen? Immerhin muss er ja noch mit seinem Projektbericht dieses Jahr fertig werden. In ein Kreuzungsexperiment hatte er Hühnern mit der Behandlung Lüftungssystem ($keins$, $storm$, $tornado$ und $thunder$) sowie der Behandlung Flüssignahrung ($ctrl$ und $flOw$) im Oldenburger Land untersucht. Es wurde als Messwert Schlachtgewicht [kg] bestimmt. Jetzt muss er erstmal die zweifaktorielle ANOVA verstehen.



\vspace{1ex}

Leider kennen sich Steffen mit Berechnung einer zweifaktoriellen ANOVA überhaupt nicht aus. Deshalb braucht er bei der Erstellung Ihre Hilfe! 

\begin{enumerate}
  \item Formulieren Sie die wissenschaftliche Fragestellung! \textbf{(1 Punkt)}
  \item Formulieren Sie das statistische Hypothesenpaar! \textbf{(1 Punkt)}
\item Füllen Sie die unterstehende einfaktorielle ANOVA Ergebnistabelle aus! \textbf{(3 Punkte)}
\end{enumerate}

\vspace{1Ex}

\begin{center}
  \Large
  \begin{tabular}{lccccc}
  \toprule
     & \textbf{Df} & \textbf{Sum Sq} & \textbf{Mean Sq} & \textbf{F value} & \textbf{Pr(>F)} \strut\\
    \midrule
   \textbf{Lüftungssystem}  & 3 & 577.75 &  &  &  \strut\\
    \textbf{Flüssignahrung}  & 1 & 4.77 &  &  &  \strut\\
    \textbf{Lüftungssystem:Flüssignahrung}  & 3 & 117.22 &  &  &  \strut\\
   \textbf{Error}  & 18 & 390.71 &  &  &  \strut\\
\bottomrule
  \end{tabular}
\end{center}

\vspace{1Ex}

\begin{enumerate}
  \setcounter{enumi}{3}
\item Schätzen Sie den p-Wert der Tabelle ab. Begründen Sie Ihre
  Antwort! \textbf{(3 Punkte)}
\end{enumerate}
  
\begin{center}
    \Large
\begin{tabular}{lc}
  \toprule
     & $\boldsymbol{F_{\alpha = 5\%}}$ \\
\midrule
  \textbf{Lüftungssystem} & $4.26$ \\
  \textbf{Flüssignahrung} & $3.40$ \\
  \textbf{Lüftungssystem:Flüssignahrung} & $5.23$ \\
  \bottomrule
  \end{tabular}
\end{center}

\begin{enumerate}
  \setcounter{enumi}{4}
\item Was bedeutet ein signifikantes Ergebnis in einer zweifaktoriellen ANOVA? \textbf{(2 Punkte)}
\item Was sagt der Term \textit{Lüftungssystem:Flüssignahrung} aus? Interpretieren Sie das Ergebnis! \textbf{(2 Punkte)}
\end{enumerate}
 
\clearpage
% -----------------------------------------------------------------------

\section{Aufgabe \hfill (10 Punkte)}

\textit{Geben Sie grundsätzlich Formeln und Rechenweg zur Lösung der Teilaufgaben mit an!} \\[1Ex]
 

 
%% --------------------------------------------------------------------
\begin{minipage}[t]{0.5\textwidth}
\includegraphics[width = 1.3cm]{/Users/kruppajo/work/GitHub/exam/avatare/Steffen.png}
\end{minipage}
\begin{minipage}[t]{0.5\textwidth}
\hfill
\href{https://youtu.be/rWTyHXXlYjY}{\includegraphics[width = 2cm]{img/youtube}}\\[1Ex]
\end{minipage}
\vspace{1ex}
%% --------------------------------------------------------------------



'Mit der zweifaktoriellen ANOVA lässt sich die Interaktion zwischen den beiden Behandlungen nachweisen!', sein Betreuer scheint die zweifaktoriellen ANOVA zu verstehen. Warum jetzt er jetzt nochmal alles wiederkäuen muss, wird Steffen echt nicht so klar. Wenn es doch so klar ist? Steffen war im Oldenburger Land und hatte dort einen Leistungssteigerungsversuch mit Hühnern durchgeführt. Die Komune wo er untergekommen war, war cool gewesen. Nur jetzt muss eben das Experiment fertig ausgewertet werden. Steffen hatte zwei Behandlungen auf Hühnern angewendet. Einmal Genotypen ($AA$, $AB$ und $BB$) sowie als zweite Behandlung Flüssignahrung ($ctrl$ und $flOw$). Gemessen wurde der Messwert ($Y$) Fettgehalt [\%/kg]. Jetzt muss das hier zu einem Ende kommen!

\begin{knitrout}
\definecolor{shadecolor}{rgb}{0.969, 0.969, 0.969}\color{fgcolor}\begin{kframe}
\begin{verbatim}
## Analysis of Variance Table
## 
## Response: Fettgehalt
##                          Df Sum Sq Mean Sq F value    Pr(>F)
## Genotypen                 2 693.20  346.60 16.9527 7.253e-05
## Flüssignahrung            1   1.86    1.86  0.0911   0.76628
## Genotypen:Flüssignahrung  2 150.17   75.08  3.6725   0.04596
## Residuals                18 368.01   20.45
\end{verbatim}
\end{kframe}
\end{knitrout}

\vspace{1ex}

Leider kennen sich Steffen mit Berechnung einer zweifaktoriellen ANOVA überhaupt nicht aus. Deshalb braucht er bei der Erstellung Ihre Hilfe! 

\begin{enumerate}
  \item Formulieren Sie die wissenschaftliche Fragestellung! \textbf{(1 Punkt)}
  \item Formulieren Sie das statistische Hypothesenpaar! \textbf{(1 Punkt)}
\item Interpretieren Sie das Ergebnis der einfaktoriellen ANOVA! \textbf{(3 Punkte)} 
\item Zeichnen Sie eine Abbildung, der dem obigen Ergebnis der
  zweifaktoriellen ANOVA näherungsweise entspricht! \textbf{(5 Punkte)}
\end{enumerate}
 
\clearpage
% -----------------------------------------------------------------------

\section{Aufgabe \hfill (12 Punkte)}

\textit{Geben Sie grundsätzlich Formeln und Rechenweg zur Lösung der Teilaufgaben mit an!} \\[1Ex]
 

 
%% --------------------------------------------------------------------
\begin{minipage}[t]{0.5\textwidth}
\includegraphics[width = 1.3cm]{/Users/kruppajo/work/GitHub/exam/avatare/Jessica.png}
\end{minipage}
\begin{minipage}[t]{0.5\textwidth}
\hfill
\href{https://youtu.be/FjjJXkFJfIY}{\includegraphics[width = 2cm]{img/youtube}}\\[1Ex]
\end{minipage}
\vspace{1ex}
%% --------------------------------------------------------------------



In ein Stallexperiment wurden Hühnern mit der Behandlung Elterlinie ($ctrl$, $Standard$, $Yray$ und $Xray$) sowie der Behandlung Bestandsdichte ($standard$ und $kontakt$) untersucht. Jessica schaut konzentriert auf die Formeln der ANOVA und des t-Tests. In ihrem Experiment wurde als Messwert Fettgehalt [\%/kg] bestimmt. Jessica und der Mangel, eine unendliche Geschichte mit kniffeligen Wendungen. Dann wäre es nicht noch komplizierter. Was war da jetzt nochmal der Zusammenhang zwischen den beiden statistischen Verfahren? Beide Verfahren haben ja irgendwie etwas miteinander zu tun und ihre Betreuerin möchte das jetzt auch noch verstehen. Muss das nicht eigentlich klar sein? Immerhin ist Jessica nicht die erste Betreuung. Immerhin hat sie die beiden Formeln vorliegen. Schon dutzende Male gesehen: Herr der Ringe. Aber immer noch großartig zusammen mit Schokobons.

\begin{graybox}{Gegebene Formeln für $\boldsymbol{F_D}$ und $\boldsymbol{T_D}$}
\begin{center}
  \begin{tabular}{cc}
    $F_{D} = \cfrac{MS_{treatment}}{MS_{error}}$ & $T_{D} = \cfrac{\bar{y}_1 - \bar{y}_2}{s_p \cdot \sqrt{2/n_g}}$\\
  \end{tabular}
\end{center}
\end{graybox}

\vspace{1ex}

Leider kennen sich Jessica mit dem Zusammenhang zwischen der ANOVA und dem t-Test nicht aus. Deshalb braucht sie bei der Erstellung Ihre Hilfe! 

\begin{enumerate}
\item Was testet der t-Test und was testet die ANOVA? Begründen Sie Ihre Antwort! \textbf{(2 Punkte)}
\item Erklären Sie den konzeptionellen Zusammenhang zwischen der Berechnung der $F_{D}$ Statistik und $T_{D}$ Statistik! \textbf{(3 Punkte)}
\item Visualisieren Sie eine nicht signifikante $F_{D}$ Statistik sowie eine signifikante $F_{D}$ Statistik anhand von $MS_{treatment}$ und $MS_{error}$! Beschriften Sie die Abbildung! \textbf{(3 Punkte)}
\item Nennen Sie das Minimum der F-Statistik! Begründen Sie Ihre Antwort! \textbf{(2 Punkte)}
\item Wenn die F-Statistik minimal ist, welche Aussage erhalten Sie über die Nullhypothese? Begründen Sie Ihre Antwort! \textbf{(2 Punkte)}
\end{enumerate}

 
\clearpage
% -----------------------------------------------------------------------

\section{Aufgabe \hfill (8 Punkte)}

\textit{Geben Sie grundsätzlich Formeln und Rechenweg zur Lösung der Teilaufgaben mit an!} \\[1Ex]
 

 
%% --------------------------------------------------------------------
\begin{minipage}[t]{0.5\textwidth}
\includegraphics[width = 1.3cm]{/Users/kruppajo/work/GitHub/exam/avatare/Jessica.png}
\end{minipage}
\begin{minipage}[t]{0.5\textwidth}
\hfill
\href{https://youtu.be/2qG1Dws0MJo}{\includegraphics[width = 2cm]{img/youtube}}\\[1Ex]
\end{minipage}
\vspace{1ex}
%% --------------------------------------------------------------------



Es ist schon kurz nach fünf und Jessica wird langsam nervös. Jessica wollte heute Abend noch seine E-Sport Qualifikation schauen. Stattdessen versucht ihr Betreuer die Ausgabe der zweifaktoriellen ANOVA zu visualieren und zu überprüfen, ob es mit der Visualisierung der Daten als Boxplots zusammenpasst. Jessica hatte im Oldenburger Land ein Stallexperiment mit Hühnern durchgeführt. Es gab dabei zwei Behandlungen. Einmal Ernährungszusatz ($ctrl$, $fedX$ und $getIt$) sowie als zweite Behandlung Flüssignahrung ($ctrl$ und $flOw$). Gemessen wurde der Messwert ($Y$) Protein/Fettrate [\%/kg]. So kompliziert kann das jetzt doch nicht sein!


Sie rechnen eine zweifaktorielle ANOVA und erhalten einen signifikanten
Interaktionseffekt zwischen den beiden Faktoren $f_1$ und $f_2$. Der Faktor
$f_1$ hat drei Level. Der Faktor $f_2$ hat dagegen nur zwei Level.

\vspace{1ex}


Leider kennen sich Jessica mit Berechnung einer zweifaktoriellen ANOVA überhaupt nicht aus. Deshalb braucht sie bei der Erstellung Ihre Hilfe! 

\begin{enumerate}
\item Visualisieren Sie eine keine, eine schwache und eine starke Interaktion zwischen den Faktoren $f_1$ und $f_2$! \textbf{(4 Punkte)}
\item Erklären Sie den Unterschied zwischen den beiden Stärken der Interaktion! \textbf{(2 Punkte)}
\item Wenn eine signifikante Interaktion in den Daten vorliegt, wie ist dann das weitere Vorgehen bei einem Post-hoc-Test? \textbf{(2 Punkte)}
\end{enumerate}

 
\clearpage
% -----------------------------------------------------------------------

\section{Aufgabe \hfill (9 Punkte)}

%% --------------------------------------------------------------------
\hfill\href{https://youtu.be/M9Uhm67ndxM}{\includegraphics[width =
  2cm]{img/youtube}}\\[1Ex]
%% --------------------------------------------------------------------

\textit{Geben Sie grundsätzlich Formeln und Rechenweg zur Lösung der Teilaufgaben mit an!} \\[1Ex]
 

 
%% --------------------------------------------------------------------
\begin{minipage}[t]{0.5\textwidth}
\includegraphics[width = 1.3cm]{/Users/kruppajo/work/GitHub/exam/avatare/Jessica.png}
\end{minipage}
\begin{minipage}[t]{0.5\textwidth}
\hfill
\href{https://youtu.be/M9Uhm67ndxM}{\includegraphics[width = 2cm]{img/youtube}}\\[1Ex]
\end{minipage}
\vspace{1ex}
%% --------------------------------------------------------------------



In einen Leistungssteigerungsversuch wurden Hühnern mit der Behandlung Bestandsdichte ($standard$, $eng$, $weit$ und $kontakt$) sowie der Behandlung Genotypen ($AA$ und $BB$) untersucht. Es wurde als Messwert Gewichtszuwachs in der 1LW bestimmt. Jetzt starrt Jessica mit auf die \Rlogo Ausgabe einer zweifaktoriellen ANOVA. Leider starrt ihre Betreuerin in der gleichen Art Jessica zurück an. Das wird ein langer Nachmmittag, denkt sie sich und kreuselt ihren Mund. 'Und was machen wir jetzt?' entfährt es ihr überrascht entnervt. Immerhin war geht es ja um ihre Abschlussarbeit. Jessica hätte doch nichts mit Hühnern machen sollen. Hühnern -- was soll das auch bedeutendes sein?




Sie rechnen eine einfaktorielle ANOVA mit einem Faktor $f_1$ mit
drei Leveln. Nachdem Sie die einfaktorielle ANOVA gerechnet
haben, erhalten Sie einen p-Wert von $0.078$ und eine F Statistik mit
$F_{D} = 1.2$. Als Sie sich die Boxplots der Behandlungen anschauen,
stellen Sie fest, dass es eigentlich einen Mittelwertsunterschied zwischen
dem ersten und zweiten Level geben müsste. Die
$IQR$-Bereiche überlappen sich nicht und die Mediane liegen auch weit vom
globalen Mittel entfernt.

\vspace{1ex}

Leider kennen sich Jessica mit Berechnung einer zweifaktoriellen ANOVA überhaupt nicht aus. Deshalb braucht sie bei der Erstellung Ihre Hilfe! 

\begin{enumerate}
\item Erklären Sie die Annahme der Normalverteilung und die Annahme der Varianzhomogenität für eine ANOVA an einer passenden Abbildung! \textbf{(3 Punkte)}
\item Visualisieren Sie die Berechnung von $F_{D}$ am obigen Beispiel! \textbf{(3 Punkte)}
\item Erklären Sie das Ergebnis der obigen einfaktoriellen ANOVA unter der Berücksichtigung der Annahmen an eine ANOVA! \textbf{(3 Punkte)}
\end{enumerate}

 
\clearpage
% -----------------------------------------------------------------------
\part{Multiple Gruppenvergleiche}
% ----------------------------------------------------------------------- 

\section{Aufgabe \hfill (12 Punkte)}

\textit{Geben Sie grundsätzlich Formeln und Rechenweg zur Lösung der Teilaufgaben mit an!} \\[1Ex]
 

 
%% --------------------------------------------------------------------
\begin{minipage}[t]{0.5\textwidth}
\includegraphics[width = 1.3cm]{/Users/kruppajo/work/GitHub/exam/avatare/Alex.png}\hspace{-4mm}\includegraphics[width = 1.3cm]{/Users/kruppajo/work/GitHub/exam/avatare/Mark.png}
\end{minipage}
\begin{minipage}[t]{0.5\textwidth}
\hfill
\href{https://youtu.be/kHmfEmU6lrk}{\includegraphics[width = 2cm]{img/youtube}}
\end{minipage}
\vspace{1ex}
%% --------------------------------------------------------------------



In einen Leistungssteigerungsversuch mit Hühnern wurde die Behandlung Bestandsdichte ($effizient$, $standard$, $eng$, $weit$ und $kontakt$) gegen die Ergebnisse einer früheren Studie von Meyer et al. (2021) verglichen. Im Rahmen des Experiments haben Alex und Mark verschiedene Student t-Tests für den Mittelwertsvergleich für den Messwert Protein/Fettrate [\%/kg] gerechnet. Es ergab sich dann die folgende Tabelle der rohen p-Werte für die Vergleiche zu Meyer et al. (2021). Jetzt sollen die beiden einmal schauen, was in den Daten so drin ist.

\begin{knitrout}
\definecolor{shadecolor}{rgb}{0.969, 0.969, 0.969}\color{fgcolor}\begin{table}[!h]
\centering\begingroup\fontsize{10}{12}\selectfont

\begin{tabular}{ccc}
\toprule
\textbf{Rohen p-Werte} & \textbf{Adjustierte p-Werte} & \textbf{Nullhypothese ablehnen?}\\
\midrule
0.030 &  & \\
0.340 &  & \\
0.760 &  & \\
0.012 &  & \\
0.020 &  & \\
\bottomrule
\end{tabular}
\endgroup{}
\end{table}

\end{knitrout}

Leider kennen sich Alex und Mark mit der Adjustierung von $p$-Werten und dem Signifikanzniveau $\alpha$ überhaupt nicht aus. Deshalb brauchen die beiden bei der Erstellung Ihre Hilfe!

\begin{enumerate}
  \item Formulieren Sie die wissenschaftliche Fragestellung! \textbf{(1 Punkt)}
  \item Formulieren Sie die statistischen Hypothesen! \textbf{(1 Punkt)}
\item Füllen Sie die Spalte \textit{Adjustierte p-Werte} nach der Bonferoni-Methode aus! \textbf{(2 Punkte)}
\item Entscheiden Sie, ob nach der Adjustierung die Nullhypothese abgelehnt werden kann! Begründen Sie Ihre Antwort! \textbf{(2 Punkte)}
\item Wie ist Ihr Vorgehen, wenn Sie anstatt der $p$-Werte das Signifikanzniveau $\alpha$ adjustieren? \textbf{(2 Punkte)}
\item Erklären Sie warum die $p$-Werte oder das Signifikanzniveau $\alpha$ bei multiplen Vergleichen adjustiert werden müssen! \textbf{(2 Punkte)}
\item Würden Sie die Adjustierung der $p$-Werte oder die Adjustierung des Signifikanzniveaus $\alpha$ vorziehen? Begründen Sie Ihre Antwort! \textbf{(2 Punkte)}
\end{enumerate}


 
\clearpage
% ----------------------------------------------------------------------- 

\section{Aufgabe \hfill (10 Punkte)}

\textit{Geben Sie grundsätzlich Formeln und Rechenweg zur Lösung der Teilaufgaben mit an!} \\[1Ex]
 

 
%% --------------------------------------------------------------------
\begin{minipage}[t]{0.5\textwidth}
\includegraphics[width = 1.3cm]{/Users/kruppajo/work/GitHub/exam/avatare/Nilufar.png}
\end{minipage}
\begin{minipage}[t]{0.5\textwidth}
\hfill
\href{https://youtu.be/xq29O8qDrg0}{\includegraphics[width = 2cm]{img/youtube}}
\end{minipage}
\vspace{1ex}
%% --------------------------------------------------------------------



Nilufar betrachtet in sich gekehrt die Poster vor dem Büro von ihrer Betreuerin. Viele der explorativen Abbildungen sagen ihr etwas. Die Barplots und die Boxplots könnte sie dann schon nachbauen. Das macht sie dann zuversichtlich die Abschlussarbeit auch hinzukriegen. Etwas komischer sind die seltsamen Buchstaben über den Barplots. Nilufar betrachtet ein Poster das sich mit Hühnern beschäftigt. Bestandsdichte ($effizient$, $standard$, $eng$, $weit$ und $kontakt$) und Gewichtszuwachs in der 1LW wurden dort bestimmt. So richtig schlau, wird sie daraus nicht.

\begin{knitrout}
\definecolor{shadecolor}{rgb}{0.969, 0.969, 0.969}\color{fgcolor}\begin{table}[!h]
\centering\begingroup\fontsize{10}{12}\selectfont

\begin{tabular}{cc}
\toprule
\textbf{Behandlung} & \textbf{Compact letter display}\\
\midrule
effizient & a\\
standard & a\\
eng & b\\
weit & a\\
kontakt & c\\
\bottomrule
\end{tabular}
\endgroup{}
\end{table}

\end{knitrout}

Leider kennen sich Nilufar mit dem \textit{Compact letter display (CLD)} überhaupt nicht aus. Deshalb braucht sie bei der Erstellung Ihre Hilfe!

\begin{enumerate}
  \item Formulieren Sie die wissenschaftliche Fragestellung! \textbf{(1 Punkt)}
  \item Formulieren Sie die statistischen Hypothesen! \textbf{(1 Punkt)}
\item Zeichnen Sie die sich anhand des \textit{Compact letter display (CLD)} ergebenden Barplots! \textbf{(2 Punkte)}
\item Ergänzen Sie das \textit{Compact letter display (CLD)} zu den Barplots! \textbf{(1 Punkt)}
\item Erklären Sie \textit{einen} Vorteil und \textit{einen} Nachteil des \textit{Compact letter display (CLD)}! \textbf{(2 Punkte)}
\item Erstellen Sie eine Matrix mit den paarweisen $p$-Werten eines Student t-Tests, die sich näherungsweise aus dem \textit{Compact letter display (CLD)} ergeben würde! Begründen Sie Ihre Antwort! \textbf{(3 Punkte)}
\end{enumerate}

 
\clearpage
% ----------------------------------------------------------------------- 

\section{Aufgabe \hfill (12 Punkte)}

\textit{Geben Sie grundsätzlich Formeln und Rechenweg zur Lösung der Teilaufgaben mit an!} \\[1Ex]
 

 
%% --------------------------------------------------------------------
\begin{minipage}[t]{0.5\textwidth}
\includegraphics[width = 1.3cm]{/Users/kruppajo/work/GitHub/exam/avatare/Alex.png}
\end{minipage}
\begin{minipage}[t]{0.5\textwidth}
\hfill
\href{https://youtu.be/RagTFFKFbFg}{\includegraphics[width = 2cm]{img/caution}}
\end{minipage}
\vspace{1ex}
%% --------------------------------------------------------------------



Alex sitzt schon etwas länger bei sein Betreuer. So langsam macht Alex sich Gedanken, ob er nicht doch mal anmerken sollte, dass er von CLD noch nie was gehört hat. Aber noch kann gelauscht werden, ein Ende ist erstmal nicht in Sicht! Alex hatte in seine Abschlussarbeit ein Stallexperiment durchgeführt. Deshalb sitzt er hier. Also eigentlich nein, deshalb nicht. Alex will fertig werden. Hat er sich doch mit Flüssignahrung ($ctrl$, $superIn$, $compostIn$ und $flOw$) und Gewichtszuwachs in der 1LW schon eine Menge angeschaut. Alex beugt sich leicht nach vorne. Nein, doch keine Pause. Weiter warten auf eine Lücke im Fluss... 'Wir müssen als erstes die Gruppen nach absteigender Effektstärke sortieren!', hört Alex noch aus der Ferne bevor er einnickt.

\begin{knitrout}
\definecolor{shadecolor}{rgb}{0.969, 0.969, 0.969}\color{fgcolor}\begin{table}[!h]
\centering\begingroup\fontsize{10}{12}\selectfont

\begin{tabular}{cccc}
\toprule
\textbf{Flüssignahrung} & \textbf{Fallzahl (n)} & \textbf{Mittelwert} & \textbf{Standardabweichung}\\
\midrule
ctrl & 9 & 4.30 & 3.02\\
superIn & 7 & 4.36 & 2.30\\
compostIn & 8 & 5.22 & 2.57\\
flOw & 7 & 5.99 & 1.26\\
\bottomrule
\end{tabular}
\endgroup{}
\end{table}

\end{knitrout}

Leider kennen sich Alex mit dem \textit{Compact letter display (CLD)} überhaupt nicht aus. Deshalb braucht er bei der Erstellung Ihre Hilfe!

\begin{enumerate}
  \item Formulieren Sie die wissenschaftliche Fragestellung! \textbf{(1 Punkt)}
  \item Formulieren Sie die statistischen Hypothesen! \textbf{(1 Punkt)}
\item Zeichnen Sie die sich ergebenden Barplots! \textbf{(1 Punkt)}
\item Berechnen Sie die Matrix der $p$-Werte anhand von Student t-Tests! \textbf{(4 Punkte)}
\item Ergänzen Sie das \textit{Compact letter display (CLD)} zu den gezeichneten Barplots! Begründen Sie Ihre Antwort! \textbf{(4 Punkte)}
\item Interpretieren Sie das \textit{Compact letter display (CLD)} für Alex und Mark! \textbf{(1 Punkt)} 
\end{enumerate}

 
\clearpage
% -----------------------------------------------------------------------

\section{Aufgabe \hfill (10 Punkte)}

\textit{Geben Sie grundsätzlich Formeln und Rechenweg zur Lösung der Teilaufgaben mit an!} \\[1Ex]
 

 
%% --------------------------------------------------------------------
\begin{minipage}[t]{0.5\textwidth}
\includegraphics[width = 1.3cm]{/Users/kruppajo/work/GitHub/exam/avatare/Mark.png}
\end{minipage}
\begin{minipage}[t]{0.5\textwidth}
\hfill
\href{https://youtu.be/RagTFFKFbFg}{\includegraphics[width = 2cm]{img/youtube}}
\end{minipage}
\vspace{1ex}
%% --------------------------------------------------------------------



'Okay, dann nochmal für mich. Ich habe jetzt alles in SPSS gemacht, aber das Wichtigste, was gemacht werden soll, nämlich das CLD, das kann ich nicht in SPSS machen?', Mark muss sich echt beherrschen. Immerhin betreut seine Betreuerin ja erst nicht seit gestern Abschlussarbeiten und wusste ja was gemacht werden soll! Mark hatte sich zwei Variablen mit Ernährungszusatz ($ctrl$, $fedX$, $proteinX$ und $getIt$) und Protein/Fettrate [\%/kg] in einen Leistungssteigerungsversuch mit Hühnern angeschaut. Jetzt möchte er eigentlich fertig werden und nicht nochmal alles neu in \Rlogo und \texttt\{emmeans\} machen. Deshalb soll jetzt das CLD per Hand aus der Matrix der $p$-Wert abgeleitet werden. 'Ich glaube ich wechsel nochmal das Thema...', denkt Mark, verwirft dann aber den Gedanken.

\begin{knitrout}
\definecolor{shadecolor}{rgb}{0.969, 0.969, 0.969}\color{fgcolor}\begin{table}[!h]
\centering\begingroup\fontsize{10}{12}\selectfont

\begin{tabular}{>{}lcccc}
\toprule
\textbf{ } & \textbf{ctrl} & \textbf{fedX} & \textbf{proteinX} & \textbf{getIt}\\
\midrule
\textbf{ctrl} & 1.0000000 & 0.3695655 & 0.1202022 & 0.9728293\\
\textbf{fedX} & 0.3695655 & 1.0000000 & 0.0254695 & 0.4074634\\
\textbf{proteinX} & 0.1202022 & 0.0254695 & 1.0000000 & 0.1281251\\
\textbf{getIt} & 0.9728293 & 0.4074634 & 0.1281251 & 1.0000000\\
\bottomrule
\end{tabular}
\endgroup{}
\end{table}

\end{knitrout}

Leider kennen sich Mark mit dem \textit{Compact letter display (CLD)} überhaupt nicht aus. Deshalb braucht er bei der Erstellung Ihre Hilfe!

\begin{enumerate}
  \item Formulieren Sie die wissenschaftliche Fragestellung! \textbf{(1 Punkt)}
  \item Formulieren Sie die statistischen Hypothesen! \textbf{(1 Punkt)}
\item Zeichnen Sie die sich anhand der Matrix der $p$-Werte ergebenden Barplots! \textbf{(2 Punkte)}
\item Ergänzen Sie das \textit{Compact letter display (CLD)}! Begründen Sie Ihre Antwort! \textbf{(4 Punkte)}
\item Interpretieren Sie das \textit{Compact letter display (CLD)} für Mark und Mark! \textbf{(2 Punkte)} 
\end{enumerate}

 
\clearpage
% -----------------------------------------------------------------------
\part{Der Chi-Quadrat-Test \& Der diagnostische Test}
% -----------------------------------------------------------------------

\section{Aufgabe \hfill (12 Punkte)}

\textit{Geben Sie grundsätzlich Formeln und Rechenweg zur Lösung der Teilaufgaben mit an!} \\[1Ex]
 

 
%% --------------------------------------------------------------------
\begin{minipage}[t]{0.5\textwidth}
\includegraphics[width = 1.3cm]{/Users/kruppajo/work/GitHub/exam/avatare/Steffen.png}
\end{minipage}
\begin{minipage}[t]{0.5\textwidth}
\hfill
\href{https://youtu.be/-Kva5wc5Elw}{\includegraphics[width = 2cm]{img/youtube}}\\[1Ex]
\end{minipage}
\vspace{1ex}
%% --------------------------------------------------------------------



Steffen hat sich ein Herz gefasst und war für seinem Projektbericht in die Niederlande gegangen. Das war eine super Zeit in der er viel gelernt hat. Klar gab es auch die ein oder andere Besonderheit, aber das gehört hier eher nicht hin. Steffen ist schon eine ganze Zeit im Büro, da seine Betreuerin möchte, dass er jetzt auf seinen Daten mit $n = 100$ Beobachtungen von Hühnern einen $\mathcal{X}^2$-Test rechnet. Das ginge, da er als Behandlung \textit{Ökologisch [ja/nein]} bestimmt und zum anderen die Variable \textit{Protein/Fettrate im Zielbereich [ja/nein]} ermittelt hat. Wie genau, das ist jetzt eine andere Frage.


\vspace{5Ex}

\begin{center}
  \huge
  \begin{tabular}{c|c|c|c}
     & \phantom{\textbf{Erkrankt (ja)}} & \phantom{\textbf{Erkrankt (ja)}} & \phantom{\textbf{Erkrankt (ja)}} \strut\\
    \hline
    \phantom{\textbf{Pestizid (ja)}} & 24  & 19  &     \strut\\
    \hline
    \phantom{\textbf{Pestizid (ja)}} & 13  & 44  &      \strut\\
    \hline
     \phantom{100} & \phantom{100}  & \phantom{100}  &  \phantom{100}  \strut\\
  \end{tabular}
\end{center}

\vspace{5Ex}

Leider kennt sich Steffen mit der Berechnung eines $\mathcal{X}^2$-Test für kategoriale Daten überhaupt nicht aus. Deshalb braucht er bei der Erstellung Ihre Hilfe!

\begin{enumerate}
\item Formulieren Sie die wissenschaftliche Fragestellung! \textbf{(1 Punkt)}
\item Ergänzen Sie die Tabelle um die fehlenden Informationen! \textbf{(1 Punkt)} 
\item Visualisieren Sie den Zusammenhang zwischen den beiden kategorialen Variablen! \textbf{(2 Punkte)}
\item Berechnen Sie die Teststatistik eines Chi-Quadrat-Test! \textbf{(2 Punkte)}
\item Treffen Sie eine Entscheidung im Bezug zu der Nullhypothese gegeben
  einem $\mathcal{X}^2_{\alpha = 5\%} = 3.841$! Begründen Sie Ihre Antwort!
  \textbf{(2 Punkte)}
\item Skizzieren Sie die $\mathcal{X}^2$-Verteilung, wenn die $H_0$ wahr ist! Ergänzen Sie  $\mathcal{X}^2_{\alpha = 5\%}$ und $\mathcal{X}^2_{D}$ in der Abbildung! \textbf{(2 Punkte)}
\item Berechnen Sie den Effektschätzer $Cramers\; V$! Interpretieren Sie den
  Effektschätzer! \textbf{(2 Punkte)}
\end{enumerate} 
\clearpage
% -----------------------------------------------------------------------

\section{Aufgabe \hfill (10 Punkte)}

\textit{Geben Sie grundsätzlich Formeln und Rechenweg zur Lösung der Teilaufgaben mit an!} \\[1Ex]
 

 
%% --------------------------------------------------------------------
\begin{minipage}[t]{0.5\textwidth}
\includegraphics[width = 1.3cm]{/Users/kruppajo/work/GitHub/exam/avatare/Tina.png}
\end{minipage}
\begin{minipage}[t]{0.5\textwidth}
\hfill
\href{https://youtu.be/jakM7fHyZfU}{\includegraphics[width = 2cm]{img/youtube}}\\[1Ex]
\end{minipage}
\vspace{1ex}
%% --------------------------------------------------------------------



Am Ende war es für Tina in ihrer Abschlussarbeit dann doch kein normalverteiltes Outcome. Das was jetzt etwas doff, da er sich auf eine ANOVA gefreut hatte. Prinzipiell ginge das auch irgendwie, aber nun möchte ihre Betreuerin gerne einen $\mathcal{X}^2$-Test auf einer $2x2$-Kreuztabelle berechnet bekommen. Tina hatte sich in ein Kreuzungsexperiment $n = 162$ Beobachtungen von Hühnern angeschaut. Dabei hat sie als Behandlung \textit{Ökologisch [ja/nein]} bestimmt und zum anderen die Variable \textit{Fettgehalt erreicht [ja/nein]} ermittelt. Jetzt muss Tina mal schauen, wie sie das jetzt rechnet.

\vspace{5Ex}

\begin{center}
  \huge
  \begin{tabular}{c|c|c|c}
     & \phantom{\textbf{Erkrankt (ja)}} & \phantom{\textbf{Erkrankt (ja)}} & \phantom{\textbf{Erkrankt (ja)}} \strut\\
    \hline
   \phantom{\textbf{Pestizid (ja)}} & \phantom{100}  & \phantom{100}  &   67  \strut\\
    \hline
    \phantom{\textbf{Pestizid (ja)}} & \phantom{100}  & \phantom{100}  &   95   \strut\\
    \hline
     &  104 &  58 &  162  \strut\\
  \end{tabular}
\end{center}

\vspace{5Ex}

Leider kennt sich Tina mit der Berechnung eines $\mathcal{X}^2$-Test für kategoriale Daten überhaupt nicht aus. Deshalb braucht sie bei der Erstellung Ihre Hilfe!

\begin{enumerate}
  \item Formulieren Sie die wissenschaftliche Fragestellung! \textbf{(1 Punkt)}
\item Ergänzen Sie die Tabelle um die fehlenden Informationen! \textbf{(1 Punkt)} 
\item Ergänzen Sie die Felder innerhalb der $2x2$ Kreuztabelle, so dass \textit{kein} signifikanter Effekt zu erwarten wäre! \textbf{(2 Punkte)}
\item Begründen Sie Ihr Vorgehen an der Formel des Chi-Quadrat-Tests. Erklären Sie Ihr Vorgehen an einem Beispiel! \textbf{(2 Punkte)}
\item Visualisieren Sie den Zusammenhang zwischen den beiden kategorialen Variablen! \textbf{(2 Punkte)}
\item Was ist die Mindestanzahl an Beobachtungen je Zelle? Wenn in einer der Zellen weniger Beobachtungen auftreten, welchen Test können Sie anstatt des Standard Chi-Quadrat-Tests anwenden? \textbf{(2 Punkte)}
\end{enumerate} 
\clearpage
% -----------------------------------------------------------------------

\section{Aufgabe \hfill (10 Punkte)}

\textit{Geben Sie grundsätzlich Formeln und Rechenweg zur Lösung der Teilaufgaben mit an!} \\[1Ex]
 

 
%% --------------------------------------------------------------------
\begin{minipage}[t]{0.5\textwidth}
\includegraphics[width = 1.3cm]{/Users/kruppajo/work/GitHub/exam/avatare/Alex.png}
\end{minipage}
\begin{minipage}[t]{0.5\textwidth}
\hfill
\href{https://youtu.be/ghArbetOr_E}{\includegraphics[width = 2cm]{img/youtube}}\\[1Ex]
\end{minipage}
\vspace{1ex}
%% --------------------------------------------------------------------



Am Ende war es für Alex in seiner Hausarbeit dann doch kein normalverteiltes Outcome. Das was jetzt etwas doff, da er sich auf eine ANOVA gefreut hatte. Prinzipiell ginge das auch irgendwie, aber nun möchte sein Betreuer gerne einen $\mathcal{X}^2$-Test auf einer $2x2$-Kreuztabelle berechnet bekommen. Alex hatte sich in ein Stallexperiment $n = 162$ Beobachtungen von Hühnern angeschaut. Dabei hat er als Behandlung \textit{Außenklimakontakt [ja/nein]} bestimmt und zum anderen die Variable \textit{Fettgehalt erreicht [ja/nein]} ermittelt. Jetzt muss Alex mal schauen, wie er das jetzt rechnet. Nach seinem Experiment erhielt er folgende $2x2$ Kreuztabelle aus seinen erhobenen Daten.

\begin{knitrout}
\definecolor{shadecolor}{rgb}{0.969, 0.969, 0.969}\color{fgcolor}\begin{kframe}
\begin{verbatim}
##                    Außenklimakontakt
## Fettgehalt erreicht ja nein
##                ja    6   12
##                nein 10    7
\end{verbatim}
\end{kframe}
\end{knitrout}

Dann rechnete Alex den Fisher-Exakt-Test auf der $2x2$-Kreuztabelle in \Rlogo und erhielt folgende \Rlogo Ausgabe der Funktion \texttt{fisher.test()}.

\begin{knitrout}
\definecolor{shadecolor}{rgb}{0.969, 0.969, 0.969}\color{fgcolor}\begin{kframe}
\begin{verbatim}
## 
## 	Fisher's Exact Test for Count Data
## 
## data:  Fettgehalt erreicht
## p-value = 0.1811
## alternative hypothesis: true odds ratio is not equal to 1
## 95 percent confidence interval:
##  0.07075797 1.67575062
## sample estimates:
## odds ratio 
##  0.3611682
\end{verbatim}
\end{kframe}
\end{knitrout}

Leider kennt sich Alex mit der Berechnung eines $\mathcal{X}^2$-Test für kategoriale Daten überhaupt nicht aus. Deshalb braucht er bei der Erstellung Ihre Hilfe!

\begin{enumerate}
\item Formulieren Sie die wissenschaftliche Fragestellung! \textbf{(1 Punkt)}
\item Visualisieren Sie den Zusammenhang zwischen den beiden kategorialen Variablen! \textbf{(2 Punkte)}
\item Liegt ein signifikanter Unterschied zwischen den Gruppen vor? Begründen Sie Ihre Antwort! \textbf{(2 Punkte)}
\item Skizzieren Sie das sich ergebende 95\% Konfidenzintervall! \textbf{(2 Punkte)}
\item Beschriften Sie die Abbildung des 95\% Konfidenzintervalls! \textbf{(1 Punkt)} 
\item Interpretieren Sie das \textit{Odds ratio} im Kontext der wissenschaftlichen Fragestellung! \textbf{(2 Punkte)} 
\end{enumerate}
 
\clearpage
% -----------------------------------------------------------------------

\section{Aufgabe \hfill (11 Punkte)}

%% --------------------------------------------------------------------
\hfill\href{https://youtu.be/VQlNl8hvRII}{\includegraphics[width =
  2cm]{img/youtube}}\\[1Ex]
%% --------------------------------------------------------------------


Die Pr{\"a}valenz von Klauenseuche bei Wollschweinen wird mit
2\% angenommen. In 85\% der F{\"a}lle ist ein Test positiv, wenn das Wollschwein erkrankt
ist. In 8\% der F{\"a}lle ist ein Test positiv,
wenn das Wollschwein \textit{nicht} erkrankt ist und somit gesund ist. Sie
werten 4000 Wollschweine mit einem
diagnostischen Test auf Klauenseuche aus.



\begin{enumerate}
\item F{\"u}llen und beschriften Sie den untenstehenden Doppelbaum! Beschriften
  Sie auch die {\"A}ste des Doppelbaumes, mit denen Ihnen bekannten
  Informationen!  \textbf{(8 Punkte)}
\item Berechnen Sie die Wahrscheinlichkeit $Pr(K^+|T^+)$! \textbf{(2 Punkte)}
\item Was sagt Ihnen die Wahrscheinlichkeit $Pr(K^+|T^+)$ aus? \textbf{(1 Punkt)}
\end{enumerate}

\vspace{1cm}

\begin{center}
  \includegraphics[width=17cm]{/Users/kruppajo/work/GitHub/exam/question/img/diag-doppelbaum}
\end{center}



 
\clearpage
% -----------------------------------------------------------------------

\section{Aufgabe \hfill (12 Punkte)}


%% --------------------------------------------------------------------
\hfill\href{https://youtu.be/_7s44pbOc00}{\includegraphics[width =
  2cm]{img/youtube}}\\[1Ex]
%% --------------------------------------------------------------------





Folgender diagnostischer Doppelbaum nach der Testung auf Klauenseuche bei
Fleckvieh ist gegeben.

\begin{enumerate}
\item F{\"u}llen und beschriften Sie den untenstehenden Doppelbaum! \textbf{(4
    Punkte)}
\item Berechnen Sie die Wahrscheinlichkeit $Pr(K^+|T^+)$! \textbf{(2 Punkte)}
\item Berechnen Sie die Pr{\"a}valenz f{\"u}r Klauenseuche! \textbf{(2 Punkte)}
\item Berechnen Sie die Sensifit{\"a}t und Spezifit{\"a}t des diagnostischen Tests
  f{\"u}r Klauenseuche! Erstellen Sie daf{\"u}r zun{\"a}chst eine 2x2 Kreuztabelle aus
  dem ausgef{\"u}llten Doppelbaum!
  \textbf{(4 Punkte)}
\end{enumerate}

\vspace{1cm}
 
\begin{tikzpicture}
  \node (image) at (0,0) {
    \includegraphics[width=\textwidth]{/Users/kruppajo/work/GitHub/exam/question/img/diag-doppelbaum}
  };
  \node[] at (-4.8,0) {\huge 180};
  \node[] at (-1.7,0) {\huge 60};
  \node[] at (1.7,0) {\huge 850};
  \node[] at (4.75,0) {\huge 1800};
\end{tikzpicture}




 
\clearpage
% -----------------------------------------------------------------------
\part{Lineare Regression \& Korrelation}
% -----------------------------------------------------------------------

\section{Aufgabe \hfill (10 Punkte)}

\textit{Geben Sie grundsätzlich Formeln und Rechenweg zur Lösung der Teilaufgaben mit an!} \\[1Ex]
 

 
%% --------------------------------------------------------------------
\begin{minipage}[t]{0.5\textwidth}
\includegraphics[width = 1.3cm]{/Users/kruppajo/work/GitHub/exam/avatare/Jessica.png}\hspace{-4mm}\includegraphics[width = 1.3cm]{/Users/kruppajo/work/GitHub/exam/avatare/Steffen.png}
\end{minipage}
\begin{minipage}[t]{0.5\textwidth}
\hfill
\href{https://youtu.be/kHmfEmU6lrk}{\includegraphics[width = 2cm]{img/youtube}}
\end{minipage}
\vspace{1ex}
%% --------------------------------------------------------------------



'Wichtig ist es, dass wir jetzt eine Gerade durch die Punkte zeichnen!', ruft Steffen. 'Ich sehe nur eine Zahlen und keine Punkte. Wie soll ich da denn jetzt eine Gerade durchzeichnen?', fragt Jessica. Steffen atmet schwer ein. Die beiden hatten ein Kreuzungsexperiment im Emsland mit Hühnern durchgeführt. Dabei wurden die beiden folgenden Variablen gemessen: mittlerer Anzahl an weißen Blutkörperchen [LEU/ml] und Fettgehalt [\%/kg]. Jetzt will die Betreuung von den beiden einmal die Visualisierung der Daten und auch gleich noch die lineare Regression gerechnet bekommen.

\begin{table}[!h]
\centering
\begin{tabular}{cc}
\toprule
Mittlerer Anzahl an weißen Blutkörperchen [LEU/ml] & Fettgehalt [\%/kg]\\
\midrule
23.5 & 25.8\\
16.7 & 19.2\\
21.3 & 20.2\\
24.2 & 24.5\\
20.4 & 19.2\\
\addlinespace
20.7 & 19.7\\
18.4 & 19.6\\
18.6 & 20.7\\
16.5 & 19.2\\
22.5 & 23.7\\
\addlinespace
25.7 & 23.9\\
\bottomrule
\end{tabular}
\end{table}



Leider kennen sich Steffen und Jessica mit der linearen Regression für kontinuierliche Daten überhaupt nicht aus. Deshalb brauchen beide bei der Erstellung Ihre Hilfe!

\begin{enumerate}
\item Formulieren Sie die wissenschaftliche Fragestellung! \textbf{(1 Punkt)}
\item Erstellen  Sie  eine  Visualisierung  für  die  Datentabelle.  Beschriften  Sie  die  Achsen! \textbf{(2 Punkte)}
\item Schätzen Sie die Regressionsgleichung aus der obigen Abbildung ab! \textbf{(2 Punkte)}
\item Beschriften Sie die Grade mit den statistischen Maßzahlen der linearen Regressionsgleichung! \textbf{(2 Punkte)}
\item Liegt ein Zusammenhang zwischen $x$ und $y$ vor? Begründen Sie Ihre Antwort! \textbf{(2 Punkte)}
\item Wenn kein Zusammenhang zu beobachten wäre, wie würde die Grade aussehen? \textit{Antworten Sie mit einer Skizze der Geraden!} \textbf{(1 Punkt)}
\end{enumerate} 
\clearpage
% -----------------------------------------------------------------------

\section{Aufgabe \hfill (12 Punkte)}

\textit{Geben Sie grundsätzlich Formeln und Rechenweg zur Lösung der Teilaufgaben mit an!} \\[1Ex]
 

 
%% --------------------------------------------------------------------
\begin{minipage}[t]{0.5\textwidth}
\includegraphics[width = 1.3cm]{/Users/kruppajo/work/GitHub/exam/avatare/Alex.png}\hspace{-4mm}\includegraphics[width = 1.3cm]{/Users/kruppajo/work/GitHub/exam/avatare/Tina.png}
\end{minipage}
\begin{minipage}[t]{0.5\textwidth}
\hfill
\href{https://youtu.be/lJp8rFmMnrs}{\includegraphics[width = 2cm]{img/youtube}}
\end{minipage}
\vspace{1ex}
%% --------------------------------------------------------------------



'Ich glaube du bringst da was durcheinander. Wir nutzen zwar auch für die ANOVA die Funktion \texttt{lm()} aber hier wollen wir, glaube ich, eine Gerade durch die Punkte zeichnen.', merkt Alex an. 'Ich sehe keine Punkte... ich sehe nur zwei Zeilen einer Tabelle und ich glaube du hast gerade was gelöscht.', antwortet Tina sichtlich übernächtigt. 'Wir müssen die Koeffizienten der linearen Regression ja auch erst interpretieren!', spricht Alex sehr deutlich und langsam. Die beiden hatten ein Stallexperiment im Oldenburger Land mit Hühnern durchgeführt. Dabei wurden die beiden folgenden Variablen gemessen: mittlere Eisenkonzentration [Fe/ml] und Fettgehalt [\%/kg]. Jetzt wollen sie erstmal schauen, ob es einen Zusammenhang gibt und das soll mit der \Rlogo Ausgabe möglich sein.

\begin{table}[!h]
\centering\begingroup\fontsize{12}{14}\selectfont

\begin{tabular}{ccccc}
\toprule
term & estimate & std.error & t statistic & p-value\\
\midrule
(Intercept) & 0.02 & 2.09 &  & \\
Mittlere Eisenkonzentration & 0.72 & 0.20 &  & \\
\bottomrule
\end{tabular}
\endgroup{}
\end{table}



Leider kennen sich Alex und Tina mit der linearen Regression für kontinuierliche Daten in \Rlogo überhaupt nicht aus. Deshalb brauchen beide bei der Erstellung Ihre Hilfe!

\begin{enumerate}
\item Formulieren Sie die wissenschaftliche Fragestellung! \textbf{(1 Punkt)}
\item Erstellen  Sie  eine  Visualisierung  der \texttt{lm()}-Ausgabe.  Beschriften  Sie  die  Achsen! \textbf{(2 Punkte)}
\item Beschriften Sie die Visualisierung mit den statistischen Maßzahlen der der \texttt{lm()}-Ausgabe! \textbf{(2 Punkte)}
\item Formulieren Sie die Regressionsgleichung! \textbf{(1 Punkt)}
\item Ergänzen Sie die t Statistik in der \texttt{lm()}-Ausgabe! \textbf{(2 Punkte)}
\item Ergänzen Sie den $p$-Wert in der \texttt{lm()}-Ausgabe mit $T_{\alpha = 5\%} = 1.96$!  \textbf{(2 Punkte)}
\item Interpretieren Sie den $p$-Wert im Kontext der wissenschaftlichen Fragestellung! \textbf{(1 Punkt)}  
\item Wie groß ist der Effekt im Kontext der wissenschaftlichen Fragestellung? \textbf{(1 Punkt)}
\end{enumerate} 
\clearpage
% -----------------------------------------------------------------------

\section{Aufgabe \hfill (11 Punkte)}

\textit{Geben Sie grundsätzlich Formeln und Rechenweg zur Lösung der Teilaufgaben mit an!} \\[1Ex]
 

 
%% --------------------------------------------------------------------
\begin{minipage}[t]{0.5\textwidth}
\includegraphics[width = 1.3cm]{/Users/kruppajo/work/GitHub/exam/avatare/Jessica.png}\hspace{-4mm}\includegraphics[width = 1.3cm]{/Users/kruppajo/work/GitHub/exam/avatare/Steffen.png}
\end{minipage}
\begin{minipage}[t]{0.5\textwidth}
\hfill
\href{https://youtu.be/tNNzcndrpSk}{\includegraphics[width = 2cm]{img/youtube}}
\end{minipage}
\vspace{1ex}
%% --------------------------------------------------------------------



'Wichtig ist es, dass wir jetzt eine Gerade durch die Punkte zeichnen!', ruft Steffen. 'Ich sehe nur Kauderwelsch und keine Punkte. Wie soll ich da denn jetzt eine Gerade durchzeichnen? Und warum überhaupt? War das unsere Fragestellung?', fragt Jessica. Steffen atmet schwer ein und starrt auf die \Rlogo Ausgabe der Funktion \texttt{lm()}. Die beiden hatten ein Kreuzungsexperiment im Oldenburger Land mit Hühnern durchgeführt. Dabei wurden die beiden folgenden Variablen gemessen: mittlere Eisenkonzentration [Fe/ml] und Schlachtgewicht [kg]. Jetzt will die Betreuung von den beiden die Interpretierung der Daten in Form einer linearen Regression gerechnet bekommen. Das haben beide in \Rlogo gemacht, aber wie soll das jetzt gehen? Das mit der Interpretation?

\begin{knitrout}
\definecolor{shadecolor}{rgb}{0.969, 0.969, 0.969}\color{fgcolor}\begin{kframe}
\begin{verbatim}
## 
## Call:
## Schlachtgewicht ~ Mittlere_Eisenkonzentration
## 
## Residuals:
##     Min      1Q  Median      3Q     Max 
## -1.8308 -0.8170 -0.1963  0.8298  3.0127 
## 
## Coefficients:
##                             Estimate Std. Error t value Pr(>|t|)
## (Intercept)                   0.4221     1.3413   0.315    0.755
## Mittlere_Eisenkonzentration   0.6783     0.1353   5.015  1.2e-05
## 
## Residual standard error: 1.113 on 39 degrees of freedom
## Multiple R-squared:  0.392,	Adjusted R-squared:  0.3764 
## F-statistic: 25.15 on 1 and 39 DF,  p-value: 1.195e-05
\end{verbatim}
\end{kframe}
\end{knitrout}

Leider kennen sich Steffen und Jessica mit der linearen Regression für kontinuierliche Daten in \Rlogo überhaupt nicht aus. Deshalb brauchen beide bei der Erstellung Ihre Hilfe!


\begin{enumerate}
\item Formulieren Sie die wissenschaftliche Fragestellung! \textbf{(1 Punkt)}
\item Wie groß ist der Effekt im Kontext der wissenschaftlichen Fragestellung? \textbf{(2 Punkte)} 
\item Interpretieren Sie die $p$-Werte im Kontext der wissenschaftlichen Fragestellung! \textbf{(2 Punkte)}
\item Visualisieren Sie die Verteilung der Residuen! \textbf{(2 Punkte)} 
\item Ist die Annahme der Normalverteilung erfüllt? Begründen Sie die Antwort! \textbf{(2 Punkte)}
\item Erklären Sie \textit{kurz} den Begriff \texttt{R-squared}! Was sagt Ihnen der Wert $0.39$ aus? \textbf{(2 Punkte)}
\end{enumerate}
 
\clearpage
% -----------------------------------------------------------------------

\section{Aufgabe \hfill (9 Punkte)}

\textit{Geben Sie grundsätzlich Formeln und Rechenweg zur Lösung der Teilaufgaben mit an!} \\[1Ex]
 

 
%% --------------------------------------------------------------------
\begin{minipage}[t]{0.5\textwidth}
\includegraphics[width = 1.3cm]{/Users/kruppajo/work/GitHub/exam/avatare/Paula.png}
\end{minipage}
\begin{minipage}[t]{0.5\textwidth}
\hfill
\href{https://youtu.be/C9skfFRTHhI}{\includegraphics[width = 2cm]{img/youtube}}
\end{minipage}
\vspace{1ex}
%% --------------------------------------------------------------------



'Wichtig ist es, dass wir jetzt eine Gerade durch die Punkte zeichnen...', denkt Paula. 'Ich sehe nur Kauderwelsch und keine Punkte. Ich glaube das war jetzt doch eine Korrelation, die ich rechnen sollte. Und warum überhaupt? War das unsere Fragestellung?', denkt sich Paula. Paula atmet schwer ein und starrt auf die \Rlogo Ausgabe der Funktion \texttt{cor.test()}. Paula hatte ein Stallexperiment im Emsland mit Hühnern durchgeführt. Dabei wurden die beiden folgenden Variablen gemessen: durchschnittlicher Bewegungsscore [Movement/h] und Gewichtszuwachs in der 1LW. Jetzt will die Betreuung von ihr die Interpretierung der Daten in Form einer Korrelation berechnet bekommen. Das hat Paula in \Rlogo gemacht, aber wie soll das jetzt gehen? Das mit der Interpretation?


\begin{knitrout}
\definecolor{shadecolor}{rgb}{0.969, 0.969, 0.969}\color{fgcolor}\begin{kframe}
\begin{verbatim}
## 
## 	Pearson's correlation
## 
## data:  Durchschnittlicher Bewegungsscore [Movement/h] and Gewichtszuwachs in der 1LW
## t = -2.279, df = 8, p-value = 0.05215
## alternative hypothesis: true correlation is not equal to 0
## 95 percent confidence interval:
##  -0.901082605  0.003646649
## sample estimates:
##        cor 
## -0.6274202
\end{verbatim}
\end{kframe}
\end{knitrout}

Leider kennt sich Paula mit der Korrelationsanalyse in \Rlogo überhaupt nicht aus. Deshalb braucht sie bei der Erstellung Ihre Hilfe!

\begin{enumerate}
  \item Formulieren Sie die wissenschaftliche Fragestellung! \textbf{(1 Punkt)}
  \item Formulieren Sie das statistische Hypothesenpaar! \textbf{(1 Punkt)}
\item Erstellen Sie eine Visualisierung für den Korrelationskoeffizienten! Beschriften Sie die Abbildung! \textbf{(2 Punkte)}
\item Nennen Sie die zwei Eigenschaften des Korrelationskoeffizienten! \textbf{(2 Punkte)}
\item Interpretieren Sie den Korrelationskoefizienten hinsichtlich des
  Effekts und der Signifikanz! Begründen Sie Ihre Antwort! \textbf{(2 Punkte)}
\item Visualisieren Sie das 95\% Konfidenzintervall! Beschriften Sie die Abbildung! \textbf{(2 Punkte)} 
\end{enumerate} 
\clearpage
% -----------------------------------------------------------------------

\section{Aufgabe \hfill (9 Punkte)}

%% --------------------------------------------------------------------
\hfill\href{https://youtu.be/fB6nF4dxodA}{\includegraphics[width =
  2cm]{img/youtube}}\\[1Ex]
%% --------------------------------------------------------------------


In den folgenden Abbildungen sehen Sie drei leere Scatterplots. F{\"u}llen Sie diese
Scatterplots nach folgenden Anweisungen.

\begin{enumerate}
\item Zeichnen Sie f{\"u}r die angegebene $\rho$-Werte eine Gerade in die
  entsprechende Abbildung! \textbf{(3 Punkte)}
\item Zeichnen Sie f{\"u}r die angegebenen $R^2$-Werte die entsprechende
  Punktewolke um die Gerade. \textbf{(3 Punkte)}
\item Sie rechnen ein statistisches Modell. Was sagen Ihnen die $R^2$-Werte
  {\"u}ber das jeweilige Modell? \textbf{(3 Punkte)}
\end{enumerate}




{\centering \includegraphics[width=\maxwidth]{img/correlation-01-1} 

}



 
\clearpage
% -----------------------------------------------------------------------

\section{Aufgabe \hfill (9 Punkte)}

%% --------------------------------------------------------------------
\hfill\href{https://youtu.be/2QJa19ZwLls}{\includegraphics[width =
  2cm]{img/youtube}}\\[1Ex]
%% --------------------------------------------------------------------

In den folgenden Abbildungen sehen Sie vier Scatterplots. Erg{\"a}nzen Sie die {\"U}berschriften
der jeweiligen Scatterplots.


\begin{enumerate}
\item Sch{\"a}tzen Sie die $\rho$-Werte in der entsprechenden
  Abbildung! \textbf{(4 Punkte)}
\item Sch{\"a}tzen Sie die $R^2$-Werte in der entsprechenden
  Punktewolke um die Gerade! \textbf{(4 Punkte)}
\item Sie rechnen ein statistisches Modell. Was sagen Ihnen die $R^2$-Werte
  {\"u}ber das jeweilige Modell? \textbf{(1 Punkt)}
\end{enumerate}




{\centering \includegraphics[width=\maxwidth]{img/correlation-02-1} 

}



 
\clearpage
% -----------------------------------------------------------------------

\section{Aufgabe \hfill (10 Punkte)}

%% --------------------------------------------------------------------
\hfill\href{https://youtu.be/dyQlYV9nOqY}{\includegraphics[width =
  2cm]{img/youtube}}\\[1Ex]
%% --------------------------------------------------------------------

Sie rechnen eine lineare Regression um nach einem Feldexperiment den
Zusammenhang zwischen Trockengewicht kg/m$^2$ (\textit{drymatter}) und
Wassergabe l/m$^2$ (\textit{water}) bei Spargel zu bestimmen. Sie erhalten
folgende Datentabelle.

\begin{knitrout}
\definecolor{shadecolor}{rgb}{0.969, 0.969, 0.969}\color{fgcolor}\begin{table}[!h]
\centering\begingroup\fontsize{12}{14}\selectfont

\begin{tabular}{ccccc}
\toprule
.id & drymatter & water & .fitted & .resid\\
\midrule
1 & 21.6 & 7.7 & 21.2 & \\
2 & 19.5 & 6.6 & 19.5 & \\
3 & 34.5 & 17.5 & 36.7 & \\
4 & 17.4 & 5.6 & 18.0 & \\
5 & 35.8 & 16.5 & 35.0 & \\
\addlinespace
6 & 22.3 & 9.0 & 23.3 & \\
7 & 26.8 & 12.0 & 28.0 & \\
8 & 22.2 & 7.2 & 20.4 & \\
9 & 19.6 & 7.6 & 21.1 & \\
10 & 22.5 & 8.2 & 22.1 & \\
\addlinespace
11 & 27.3 & 11.2 & 26.7 & \\
12 & 36.3 & 16.2 & 34.6 & \\
13 & 30.3 & 13.0 & 29.6 & \\
\bottomrule
\end{tabular}
\endgroup{}
\end{table}

\end{knitrout}

\begin{enumerate}
\item Erg{\"a}nzen Sie die Werte in der Spalte \texttt{.resid} in der obigen
  Tabelle. Geben Sie den Rechenweg und Formel mit an! \textbf{(4 Punkte)}
\item Zeichnen Sie den sich aus der obigen Tabelle ergebenden
  Residualplot. Beschriften Sie die Abbildung! \textbf{(4 Punkte)}
\item Gibt es auff{\"a}llige Werte anhand des Residualplots? Begr{\"u}nden Sie Ihre
  Antwort! \textbf{(2 Punkte)}
\end{enumerate}
 
\clearpage
% -----------------------------------------------------------------------

\section{Aufgabe \hfill (12 Punkte)}

%% --------------------------------------------------------------------
\hfill\href{https://youtu.be/lHzRgm7hPw0}{\includegraphics[width =
  2cm]{img/youtube}}\\[1Ex]
%% --------------------------------------------------------------------



\begin{enumerate}
\item Zeichen Sie in die drei untenstehenden, leeren Abbilungen die Zeile des
  Regressionskreuzes der Poissonverteilung. W{\"a}hlen Sie die Beschriftung der
  y-Achse sowie der x-Achse entsprechend aus! \textbf{(6 Punkte)}
\item Erg{\"a}nzen Sie die jeweiligen statistischen Methoden zu der Abbildung! \textbf{(2 Punkte)}
\item Welchen Effektsch{\"a}tzer erhalten Sie aus der entsprechend linearen
  Regression bzw. den Gruppenvergleich? Geben Sie ein Beispiel! \textbf{(2 Punkte)}
\item Wenn Sie keinen Effekt erwarten, welchen \textit{Zahlenraum} nimmt dann
  der Effektsch{\"a}tzer ein? Geben Sie ein Beispiel! \textbf{(2 Punkte)}
\end{enumerate}



{\centering \includegraphics[width=\maxwidth]{img/regression-01-1} 

}



 
\clearpage
% -----------------------------------------------------------------------

\section{Aufgabe \hfill (9 Punkte)}

%% --------------------------------------------------------------------
\hfill\href{https://youtu.be/AwQEcQWLFCw}{\includegraphics[width =
  2cm]{img/youtube}}\\[1Ex]
%% --------------------------------------------------------------------



Ein Feldexperiment wurde mit $n = 200$ Pflanzen durchgef{\"u}hrt. Folgende
Einflussvariablen ($x$) wurden erhoben: rainfall, region und weight. Als m{\"o}gliche Outcomevariablen stehen Ihnen nun
folgende gemessene Endpunkte zu Verf{\"u}gung: drymatter, yield, count, quality\_score und dead.

\begin{enumerate}
\item W{\"a}hlen Sie ein Outcome was zu der Verteilungsfamilie
  \textit{Poisson} geh{\"o}rt! \textbf{(1 Punkt)}
\item Schreiben Sie das Modell in der Form $y \sim x$ wie es in \Rlogo in
  der Funktion \texttt{glm()}
  {\"u}blich ist \textit{ohne Interaktionsterm}! \textbf{(3 Punkte)}
\item Schreiben Sie das Modell in der Form $y \sim x$ wie es in \Rlogo
  {\"u}blich ist und erg{\"a}nzen Sie \textit{einen} Interaktionsterm nach Wahl! \textbf{(1 Punkt)} 
\item Zeichen Sie eine \textit{starke}
  Interaktion in die Abbildung unten f{\"u}r den Endpunkt
  \textit{yield}. Erg{\"a}nzen Sie eine aussagekr{\"a}ftige Legende. Wie erkennen
  Sie eine Interaktion? Begr{\"u}nden Sie Ihre Antwort! \textbf{(4 Punkte)}
\end{enumerate}



{\centering \includegraphics[width=\maxwidth]{img/modeling-R-01-1} 

}


 
\clearpage
% -----------------------------------------------------------------------
\part{Experimentelles Design}
% -----------------------------------------------------------------------

\section{Aufgabe \hfill (16 Punkte)}


 
%% --------------------------------------------------------------------
\begin{minipage}[t]{0.5\textwidth}
\includegraphics[width = 1.3cm]{/Users/kruppajo/work/GitHub/exam/avatare/Jonas.png}\hspace{-4mm}\includegraphics[width = 1.3cm]{/Users/kruppajo/work/GitHub/exam/avatare/Steffen.png}\hspace{-4mm}\includegraphics[width = 1.3cm]{/Users/kruppajo/work/GitHub/exam/avatare/Yuki.png}
\end{minipage}
\begin{minipage}[t]{0.5\textwidth}
\hfill
\href{https://youtu.be/wJqsNV1hOW8}{\includegraphics[width = 2cm]{img/caution}}
\end{minipage}
\vspace{1ex}
%% --------------------------------------------------------------------



Steffen und Yuki sind bei Jonas um sich Hilfe für eine Versuchsplanung in \Rlogo zu holen. Dabei geht es um den Zusammenhang zwischen der Behandlung Lüftungssystem ($keins$, $storm$, $tornado$ und $thunder$) sowie Bestandsdichte ($standard$ und $kontakt$) und dem Messwert Fettgehalt [\%/kg] in Hühnern. Der Versuch soll in einem Stallexperiment in der Uckermark durchgeführt werden. Nach dem Dozenten ist der Messwert Fettgehalt [\%/kg] normalverteilt. Die beiden entschieden sich für ein faktorielles Versuchsdesign. Im ersten Schritt überlegt Jonas ein einfaches experimentelles Design zu probieren. Daher entscheiden sich alle drei für ein \textit{Randomized complete block design (RCBD)} mit nur einem der beiden Faktoren. Das sollte für den anfang erstmal reichen. 'Und jetzt, was machen wir jetzt?', Yuki schaut die anderen beiden mit großen Augen an. Die zucken mit der Schulter.\\

Leider kennen sich Jonas, Steffen und Yuki mit dem \textit{Randomized complete block design (RCBD)} überhaupt nicht aus. Deshalb brauchen die Drei bei der Erstellung Ihre Hilfe!

\begin{enumerate}
  \setcounter{enumi}{0}
  \item Formulieren Sie die wissenschaftliche Fragestellung! \textbf{(1 Punkt)}
  \item Formulieren Sie das statistische Hypothesenpaar! \textbf{(1 Punkt)}
  \item Skizzieren Sie das faktorielle Versuchsdesign! \textbf{(3 Punkte)}
  \item Skizzieren Sie eine Datentabelle für das faktorielle Versuchsdesign in \Rlogo! \textbf{(2 Punkte)}
  \item Erstellen Sie das statistische Modell in der in \Rlogo üblichen Schreibweise für eine ANOVA! Skizzieren Sie die notwendige Funktionen in \Rlogo! \textbf{(3 Punkte)}
  \item Skizzieren Sie die weitere Datenanalyse hinsichtlich eines multiplen Gruppenvergleiches! \textbf{(2 Punkte)}
  \item Skizzieren Sie eine mögliche Abbildung im Kontext der wissenschaftlichen Fragestellung! Beschriften Sie die Abbildung! \textbf{(2 Punkte)}
  \item Ergänzen Sie zu der Abbildung ein mögliches Ergebnis des multiplen Gruppenvergleichs! Begründen Sie Ihre Antwort! \textbf{(2 Punkte)}
\end{enumerate}


 
\clearpage
% -----------------------------------------------------------------------

\section{Aufgabe \hfill (20 Punkte)}


 
%% --------------------------------------------------------------------
\begin{minipage}[t]{0.5\textwidth}
\includegraphics[width = 1.3cm]{/Users/kruppajo/work/GitHub/exam/avatare/Nilufar.png}\hspace{-4mm}\includegraphics[width = 1.3cm]{/Users/kruppajo/work/GitHub/exam/avatare/Paula.png}\hspace{-4mm}\includegraphics[width = 1.3cm]{/Users/kruppajo/work/GitHub/exam/avatare/Yuki.png}
\end{minipage}
\begin{minipage}[t]{0.5\textwidth}
\hfill
\href{https://youtu.be/wJqsNV1hOW8}{\includegraphics[width = 2cm]{img/caution}}
\end{minipage}
\vspace{1ex}
%% --------------------------------------------------------------------



Nilufar und Paula sind bei Yuki um sich Hilfe für eine Versuchsplanung in \Rlogo zu holen. Dabei geht es um den Zusammenhang zwischen der Behandlung Flüssignahrung ($ctrl$, $superIn$ und $flOw$) sowie Bestandsdichte ($standard$ und $kontakt$) sowie drei Blöcken und dem Messwert Fettgehalt [\%/kg] in Hühnern. Der Versuch soll in einem Stallversuch im Emsland durchgeführt werden. Nach der Dozentin ist der Messwert Fettgehalt [\%/kg] normalverteilt. Die beiden entschieden sich für ein faktorielles Versuchsdesign. Im ersten Schritt überlegt Yuki ein komplexeres experimentelles Design zu probieren. Daher entscheiden sich alle drei für ein \textit{Randomized complete block design} mit Berücksichtigung einer Interaktion. Das sollte für den anfang erstmal reichen. 'Und jetzt, was machen wir jetzt?', Paula schaut die anderen beiden mit großen Augen an. Die zucken mit der Schulter.\\

Leider kennen sich Yuki, Nilufar und Paula mit dem \textit{Randomized complete block design} überhaupt nicht aus. Deshalb brauchen die Drei bei der Erstellung Ihre Hilfe!

\begin{enumerate}
  \setcounter{enumi}{0}
  \item Formulieren Sie die wissenschaftliche Fragestellung! \textbf{(1 Punkt)}
  \item Formulieren Sie die statistische Hypothesenpaare! \textbf{(2 Punkte)}
  \item Skizzieren Sie das faktorielle Versuchsdesign! \textbf{(3 Punkte)}
  \item Skizzieren Sie eine Datentabelle für das faktorielle Versuchsdesign in \Rlogo! \textbf{(2 Punkte)}
  \item Erstellen Sie das statistische Modell in der in \Rlogo üblichen Schreibweise für eine ANOVA! Skizzieren Sie die notwendige Funktionen in \Rlogo! \textbf{(4 Punkte)}
  \item Skizzieren Sie die weitere Datenanalyse hinsichtlich eines multiplen Gruppenvergleiches! \textbf{(2 Punkte)}
  \item Skizzieren Sie eine mögliche Abbildung im Kontext der wissenschaftlichen Fragestellung! Beschriften Sie die Abbildung! \textbf{(3 Punkte)}
  \item Ergänzen Sie zu der Abbildung ein mögliches Ergebnis des multiplen Gruppenvergleichs! Welche Annahme hinsichtlich der Modellierung haben Sie getroffen? Begründen Sie Ihre Antwort! \textbf{(3 Punkte)}
\end{enumerate} 
\clearpage
% -----------------------------------------------------------------------
\part{Programmieren in R}
% -----------------------------------------------------------------------

\section{Aufgabe \hfill (9 Punkte)}



 
%% --------------------------------------------------------------------
\begin{minipage}[t]{0.5\textwidth}
\includegraphics[width = 1.3cm]{/Users/kruppajo/work/GitHub/exam/avatare/Paula.png}
\end{minipage}
\begin{minipage}[t]{0.5\textwidth}
\hfill
\href{https://youtu.be/C9skfFRTHhI}{\includegraphics[width = 2cm]{img/caution}}
\end{minipage}
\vspace{1ex}
%% --------------------------------------------------------------------



'Unter den Blinden ist der Einäuge König!', ruft Ihnen Paula entgegen. Leider kennt sich Paula überhaupt nicht mit den Grundlagen in \Rlogo aus aber ihre Betreuerin möchte gerne, dass die Auswertung in \Rlogo gemacht wird. Da müssen Sie dann wohl mal ran und helfen.\\[1Ex]

Paula: \textit{Ich verstehe den Unterschied zwischen \texttt{library()} und \texttt{Packages} nicht. Warum gibt es die?} \textbf{(1 Punkt)}\\[1ex]
Sie antworten:\\[2Ex]

Paula: \textit{Ich habe doch die Spalte mutiert und geändert. Warum sehe ich das in R aber mein Datensatz ändert sich nicht?} \textbf{(1 Punkt)}\\[1ex]
Sie antworten:\\[2Ex]

Paula: \textit{Wir brauchen recht häufig die Tilde ($\sim$) in R. Wo wird die nochmal angewandt und genutzt?} \textbf{(1 Punkt)}\\[1ex]
Sie antworten:\\[2Ex]

Paula: \textit{Was ist eigentlich ein Faktor in \Rlogo?} \textbf{(1 Punkt)}\\[1ex]
Sie antworten:\\[2Ex]

Paula: \textit{In R gibt es Objekte, Wörter und Funktionen. Wie unterscheiden sich diese voneinander?} \textbf{(1 Punkt)}\\[1ex]
Sie antworten:\\[2Ex]

Paula: \textit{Ich habe den Namen der Funktion, die intern Daten speichert, vergessen. Was waren da nochmal die Vorteile?} \textbf{(1 Punkt)}\\[1ex]
Sie antworten:\\[2Ex]

Paula: \textit{Der Pipe-Operator wird sehr häufig genutzt. Wie sieht der aus und wie funktioniert der an einem Beispiel?} \textbf{(1 Punkt)}\\[1ex]
Sie antworten:\\[2Ex]

Paula: \textit{Ich verstehe den Zuweisungs-Operator nicht. Wie sieht der aus und was macht der? Gebe mal ein Beispiel!} \textbf{(1 Punkt)}\\[1ex]
Sie antworten:\\[2Ex]

Paula: \textit{Wie heißen nochmal die beiden \Rlogo Pakete, die wir fast immer laden, wenn wir \Rlogo nutzen wollen?} \textbf{(1 Punkt)}\\[1ex]
Sie antworten:\\[2Ex] 
\clearpage
% -----------------------------------------------------------------------

\section{Aufgabe \hfill (9 Punkte)}



 
%% --------------------------------------------------------------------
\begin{minipage}[t]{0.5\textwidth}
\includegraphics[width = 1.3cm]{/Users/kruppajo/work/GitHub/exam/avatare/Jessica.png}
\end{minipage}
\begin{minipage}[t]{0.5\textwidth}
\hfill
\href{https://youtu.be/C9skfFRTHhI}{\includegraphics[width = 2cm]{img/caution}}
\end{minipage}
\vspace{1ex}
%% --------------------------------------------------------------------



Jessica muss ihrem Projektbericht mit \Rlogo arbeiten. Leider ist die Analyse etwas komplexer, so dass es eben in Excel dann nicht mehr geht. Deshalb also gleich alles in \Rlogo. Das ist auch der Grund warum sie jetzt mit Ihnen in der Küche sitzt und einige vertiefende Fragen zu \Rlogo an Sie hat! Na dann wollen Sie mal helfen. Immerhin will ihr Betreuer, dass \Rlogo genutzt wird und die Abgabe ist dann auch schon in gut einem Monat.\\[1Ex]

Jessica fragt: \textit{Datumsangaben sind schwierig, da es nur ein gültiges Format gibt, was zwischen Programmen funktioniert. Wie lautet das Format? \textbf{(1 Punkt)}}\\[1ex]
Sie antworten:\\[2Ex]

Jessica fragt: \textit{Ich will das R Paket \texttt{\{ggplot\}} nutzen, da war so eine Analogie an die ich mich nicht erinnern kann. Was war noch gleich das Prinzip von \texttt{\{ggplot\}}? Wie funktioniert \texttt{\{ggplot\}} konzeptionell? \textbf{(2 Punkte)}}\\[1ex]
Sie antworten:\\[2Ex]

Jessica fragt: \textit{Ich möchte ein CLD erstellen. Welche Funktionen muss ich in welcher Reihenfolge nutzen? \textbf{(2 Punkte)}}\\[1ex]
Sie antworten:\\[2Ex]

Jessica fragt: \textit{Ich baue mir ja meinen Datensatz in Excel. Was muss ich da im Bezug auf die Namen der Spalten beachten? \textbf{(1 Punkt)}}\\[1ex]
Sie antworten:\\[2Ex]

Jessica fragt: \textit{Wie spezifizieren wir nochmal eine Interaktion in einem Modell mit zwei Faktoren $f_1$ und $f_2$? \textbf{(1 Punkt)}}\\[1ex]
Sie antworten:\\[2Ex]

Jessica fragt: \textit{Ich will eine ANOVA in R rechnen. Dazu brauche ich zwei Funktionen. Welche waren das noch gleich und wie war die Reihenfolge? \textbf{(1 Punkt)}}\\[1ex]
Sie antworten:\\[2Ex]

Jessica fragt: \textit{Oh, wie baue ich mir nochmal die Daten in R? Wie heißt das Dateiformat? Gerne einmal mit Beispiel! \textbf{(1 Punkt)}}\\[1ex]
Sie antworten:\\[2Ex]



 
\clearpage
% -----------------------------------------------------------------------
\part{Forschendes Lernen}

Das forschende Lernen basiert zum einen auf den folgenden wissenschaftlichen Veröffentlichungen. Für die Prüfung wird die vertiefende Kenntnis der folgenden Veröffentlichungen vorausgesetzt.\\

\textit{In der Prüfung erhalten Sie einen Auszug der wissenschaftlichen Veröffentlichung. Für die Einarbeitung in die Veröffentlichung ist in der Prüfung \underline{ausdrücklich} keine Zeit vorgesehen.}

\begin{itemize}[noitemsep]
\item Sánchez, M., Velásquez, Y., González, M., \& Cuevas, J. (2022). Hoverfly pollination enhances yield and fruit quality in mango under protected cultivation. Scientia Horticulturae, 304, 111320. [\href{https://www.sciencedirect.com/science/article/pii/S0304423822004411}{Link}]
% \item Salinas, I., Hueso, J. J., Força Baroni, D., \& Cuevas, J. (2023). Plant growth, yield, and fruit size improvements in ‘Alicia’papaya multiplied by grafting. Plants, 12(5), 1189. [\href{https://www.mdpi.com/2223-7747/12/5/1189}{Link}]
\item Petersen, F., Demann, J., Restemeyer, D., Olfs, H. W., Westendarp, H., Appenroth, K. J., \& Ulbrich, A. (2022). Influence of light intensity and spectrum on duckweed growth and proteins in a small-scale, re-circulating indoor vertical farm. Plants, 11(8), 1010. [\href{https://www.mdpi.com/2223-7747/11/8/1010}{Link}]
\item Selle, P. H., Cadogan, D. J., Li, X., \& Bryden, W. L. (2010). Implications of sorghum in broiler chicken nutrition. Animal Feed Science and Technology, 156(3-4), 57-74. [\href{https://www.sciencedirect.com/science/article/pii/S0377840110000209}{Link}]
\item Wu, G., Knabe, D. A., \& Kim, S. W. (2004). Arginine nutrition in neonatal pigs. The Journal of Nutrition, 134(10), 2783S-2790S. [\href{https://www.sciencedirect.com/science/article/pii/S0022316623031279}{Link}]
\end{itemize}

Das forschende Lernen basiert zum anderen auf den folgenden wissenschaftlichen Datensätzen und deren vertiefende Analyse werden als bekannt vorausgesetzt. Die Teilaufgaben der Aufgaben stellen nur eine zufällige Auswahl an möglichen Fragen dar. Die Datensätze werden über ILIAS bereitgestellt.\\

\textit{In der Prüfung erhalten Sie \underline{keinen Auszug} aus den wissenschaftlichen Daten. Die Datensätze werden als \underline{bekannt} in der Prüfung vorgesetzt. Sie haben sich vorab Notizen und Anmerkungen gemacht.}

\begin{itemize}[noitemsep]
\item bar
\end{itemize}

\clearpage
% -----------------------------------------------------------------------

\section{Aufgabe \hfill (20 Punkte)}

\textit{Geben Sie grundsätzlich Formeln und Rechenweg zur Lösung der Teilaufgaben mit an!} \\[1Ex]
 

 
%% --------------------------------------------------------------------
\begin{minipage}[t]{0.5\textwidth}
\includegraphics[width = 1.3cm]{/Users/kruppajo/work/GitHub/exam/avatare/Jonas.png}
\end{minipage}
\begin{minipage}[t]{0.5\textwidth}
\hfill
\href{https://youtu.be/C9skfFRTHhI}{\includegraphics[width = 2cm]{img/caution}}
\end{minipage}
\vspace{1ex}
%% --------------------------------------------------------------------



Jonas hält die wissenschaftliche Veröffentlichung \textit{Sánchez, M., et al. (2022). Hoverfly pollination enhances yield and fruit quality in mango under protected cultivation} unter einem Schnaufen in die Luft. 'Worum geht es denn eigentlich in dieser Arbeit?', fragt er stirnrunzelnd. Jonas soll die Veröffentlichung nutzen um das eigene Experiment zu planen. Als eine Vorlage sozusagen. Daher möchte seine Betreuerin, dass er einmal die Veröffentlichung sinnvoll zusammenfasst. Das sollte dann doch etwas aufwendiger werden.\\

Leider kennt sich Jonas mit dem Lesen einer wissenschaftlichen Veröffentlichung mit Fokus auf die Statistik überhaupt nicht aus. Deshalb braucht er bei der Erstellung Ihre Hilfe! Glücklicherweise kennen Sie die wissenschaftliche Veröffentlichung schon im Detail und können sofort helfen.

\begin{enumerate}
  \setcounter{enumi}{0}
  \item Erläutern Sie die wissenschaftliche Fragestellung der wissenschaftlichen Veröffentlichung anhand des OCAR Prinzips nach Schimel (2012)\footnote{Schimel, J. (2012). Writing science: how to write papers that get cited and proposals that get funded. OUP USA.} \textbf{(4 Punkte)}
  \item Nennen Sie die untersuchten Endpunkte in der wissenschaftlichen Veröffentlichung! Wie lautet der primäre Endpunkt? \textbf{(2 Punkte)} 
\item Erstellen Sie das statistische Modell in der in \Rlogo üblichen Schreibweise! \textbf{(2 Punkte)}
  \item Nennen Sie eine Auswahl an bedeutenden statistischen Maßzahlen in der wissenschaftlichen Veröffentlichung! \textbf{(1 Punkt)}
  \item Interpretieren Sie die Hauptaussage der wissenschaftlichen Veröffentlichung hinsichtlich der Signifkanz für den primären Endpunkt! \textbf{(2 Punkte)}
  \item Interpretieren Sie die Hauptaussage der wissenschaftlichen Veröffentlichung hinsichtlich der Effektstärke für den primären Endpunkt! \textbf{(2 Punkte)}
  \item Diskutieren Sie die ökonomische Relevanz der Hauptaussage der wissenschaftlichen Veröffentlichung im Bezug auf Signifikanz und Effektstärke für den primären Endpunkt! \textbf{(1 Punkt)}
  \item Skizzieren Sie für den primären Endpunkt den sich ergebenden Datensatz in \Rlogo für eine ausgewählte Abbildung! \textbf{(2 Punkte)}
\item Skizzieren Sie einen möglichen Versuchsplan für den primären Endpunkt! \textbf{(2 Punkte)}
  \item Schätzen Sie die benötigte Fallzahl für ein zukünftiges Experiment anhand der Ergebnisse in der wisenschaftlichen Veröffentlichung für den primären Endpunkt! \textbf{(2 Punkte)}
\end{enumerate} 
\clearpage
% -----------------------------------------------------------------------

\section{Aufgabe \hfill (20 Punkte)}

\textit{Geben Sie grundsätzlich Formeln und Rechenweg zur Lösung der Teilaufgaben mit an!} \\[1Ex]
 

 
%% --------------------------------------------------------------------
\begin{minipage}[t]{0.5\textwidth}
\includegraphics[width = 1.3cm]{/Users/kruppajo/work/GitHub/exam/avatare/Steffen.png}
\end{minipage}
\begin{minipage}[t]{0.5\textwidth}
\hfill
\href{https://youtu.be/C9skfFRTHhI}{\includegraphics[width = 2cm]{img/caution}}
\end{minipage}
\vspace{1ex}
%% --------------------------------------------------------------------



Vor dem Start der eigenen Arbeit möchte sein Betreuer, dass Steffen einmal die wissenschaftlichen Daten \textit{data3} sinnvoll zusammenfasst. Dann würde die eigene Arbeit auch leichter von der Hand gehen und Steffen hätte dann schon eine Vorlage um die eigenen erhobenen Daten in eine Tabelle eintragen zu können. 'Das ist jetzt aber umfangreicher als gedacht!', schnauft er und runzelt die Stirn als er in seinen Laptop starrt.\\

Leider kennt sich Steffen mit der Analyse eines wissenschaftlichen Datensatzes überhaupt nicht aus. Deshalb braucht er bei der Auswertung Ihre Hilfe! Glücklicherweise kennen Sie den wissenschaftlichen Datensatz aus Ihren eigenen Analysen schon im Detail und können sofort helfen.

\begin{enumerate}
  \setcounter{enumi}{0}
  \item Formulieren Sie die wissenschaftliche Fragestellung des Datensatzes in Form einer PowerPoint Folie! \textbf{(2 Punkte)}
  \item Nennen Sie zwei Besonderheiten des Datensatzes! Begründen Sie Ihre Antwort! \textbf{(2 Punkte)}
  \item Nennen Sie die untersuchten Endpunkte in dem Datensatz! Wie lautet der primäre Endpunkt für die Auswertung? \textbf{(2 Punkte)}
  \item Skizzieren Sie die großen Analysebereiche der Statistik! Beschriften Sie die Abbildungen! \textbf{(2 Punkte)}
  \item In welchen der großen Analysebereiche der Statistik fällt die Auswertung des primären Endpunktes? Begründen Sie Ihre Antwort! \textbf{(2 Punkte)}
  \item Skizzieren Sie eine ikonische Abbildung für den primären Endpunkt im Kontext der wissenschaftlichen Fragestellung! \textbf{(2 Punkte)}
  \item Erstellen Sie das statistische Modell in der in \Rlogo üblichen Schreibweise! \textbf{(2 Punkte)}
  \item Skizzieren Sie die Datenanalyse hinsichtlich der Signifkanz für den primären Endpunkt! \textbf{(2 Punkte)}
  \item Skizzieren Sie die Berechnung der Effektstärke für den primären Endpunkt! \textbf{(2 Punkte)}
  \item Skizzieren Sie einen möglichen Versuchsplan für den primären Endpunkt! \textbf{(2 Punkte)}
\end{enumerate} 
\clearpage
% -----------------------------------------------------------------------
\part{Mathematik}
% -----------------------------------------------------------------------  

\section{Aufgabe \hfill (10 Punkte)}

\textit{Geben Sie grunds{\"a}tzlich Formeln und Rechenweg zur L{\"o}sung der
  Teilaufgaben mit an!} \\[1Ex]

%% --------------------------------------------------------------------
\hfill\href{https://youtu.be/Fu8kN0Uj13Y}{\includegraphics[width =
  2cm]{img/youtube}} %%youtube
\hspace{2Ex}
%% --------------------------------------------------------------------

\paragraph{Herodot – der Schimmel aus Ivenack}

W{\"a}hrend der Besetzung Mecklenburgs durch die Franzosen kamen Napoleon die
Geschichten des ber{\"u}hmten Apfelschimmels Herodot aus Ivenack zu
Geh{\"o}r. Herodot lief zwar niemals Rennen, war aber eines der ber{\"u}hmtesten
Pferde dieser Zeit. Napoleon selbst gab den Auftrag, diesen
Schimmel durch die Armee nach Frankreich zu bringen. Der Legende nach
sollen Arbeiter den Schimmel im hohlen Stamm einer 1000-j{\"a}hrigen Eiche aus Ivenack vor
den Franzosen versteckt haben. Doch Herodot verriet sein Versteck durch
lautes Wiehern, woraufhin die franz{\"o}sische Armee den Schimmel
beschlagnahmte und nach Frankreich f{\"u}hrte\footnote{Die Quelle der Inspiration
  für die Aufgabe war eine Fahrt an die Ostsee und folgender Artikel:
  \href{https://www.wald-mv.de/landingpage/ivenacker-eichen/}{Entdecke das erste Nationale Naturmonument Deutschlands - Ivenacker Eichen und Hutewald}}. \\



\textit{Forschungsfrage: "Konnten die Ivenacker den Apfelschimmel Herodot
  vor dem Zugriff von Napoleon in der 1000-j{\"a}hrigen Eiche verstecken?"} \\

Gehen Sie von einem radialen Wachstum der 1000-j{\"a}hrigen Eiche von
$0.9mm$ pro Jahr aus. Es ist bekannt, dass die Eiche im
Jahr 2022 einen Umfang von $12.5m$ in Brusth{\"o}he hatte.

\begin{enumerate}
\item Wie gro{\ss} war der Durchmesser in $m$ der Eiche im Jahr $1810$ als
  Herodot in der Eiche versteckt werden sollte?
  \textbf{(2 Punkte)}
\item Skizzieren Sie in einer Abbildung einen linearen Zusammenhang und einen
exponentiellen Zusammenhang f{\"u}r das Wachstum der 1000-j{\"a}hrigen Eiche. Erkl{\"a}ren Sie die
Auswirkungen der Entscheidung f{\"u}r linear oder exponentiell auf Ihre
Berechnungen! \textbf{(2 Punkte)}
\end{enumerate}
 
Herodot hatte eine Schulterh{\"o}he von $195$cm, eine Breite von
$85$cm sowie eine L{\"a}nge von  $220$cm.

\begin{enumerate}
  \setcounter{enumi}{2}
\item Berechnen Sie das effektive Volumen von Herodot in $m^3$, welches
  Herodot in der 1000-j{\"a}hrigen Eiche einnehmen w{\"u}rde! \textbf{(2 Punkte)}
\end{enumerate}

Es wurde berichtet, dass sich Herodot in der 1000-j{\"a}hrigen Eiche
$m{"u}hsam$ um die eigene Achse drehen konnte.

\begin{enumerate}
  \setcounter{enumi}{3}
\item Berechnen Sie die Dicke der Eichenwand in $cm$! Verdeutlichen Sie Ihre
  Berechnungen an einer aussagekr{\"a}ftigen Skizze f{\"u}r Pferd und Eiche! \textbf{(2 Punkte)} 
\item Unter einer Dicke der Eichenwand von $15cm$ bricht
  die Eiche zusammen. Beantworten Sie die Forschungsfrage! Begr{\"u}nden Sie
  Ihre Antwort! \textbf{(2 Punkte)} 
\end{enumerate}
 
\clearpage
% ----------------------------------------------------------------------- 

\section{Aufgabe \hfill (10 Punkte)}

\textit{Geben Sie grunds{\"a}tzlich Formeln und Rechenweg zur L{\"o}sung der
  Teilaufgaben mit an!} \\[1Ex]

%% --------------------------------------------------------------------
\hfill\href{https://youtu.be/57B-yYoFSk0}{\includegraphics[width =
  2cm]{img/youtube}} %%youtube
\hspace{2Ex}
%% --------------------------------------------------------------------

\paragraph{Von T{\"o}pfen auf Tischen}



In einem Experiment wollen Sie die Wuchsh{\"o}he von 180
Stockrosen bestimmen. Bevor Sie {\"u}berhaupt mit dem Experiment beginnen
k{\"o}nnen, gibt es aber ein paar Absch{\"a}tzungen {\"u}ber die Kosten und den Aufwand
zu treffen. Zum einen m{\"u}ssen Sie die Stockrosen einpflanzen und m{\"u}ssen
daf{\"u}r Substrat bestellen. Zum anderen m{\"u}ssen Sie die Stockrosen auch
bewegen und in ein Gew{\"a}chshaus platzieren. Die T{\"o}pfe f{\"u}r die Keimung haben
einen Durchmesser von 8.5cm und eine H{\"o}he von 10cm. Der
Kubikmeterpreis f{\"u}r Torf liegt bei 290 EUR.

\begin{enumerate}
\item Skizzieren Sie den Versuchsplan auf \textit{zwei} Tischen im
  Gew{\"a}chshaus! \textbf{(2 Punkte)}
\item Berechnen Sie die ben{\"o}tigte Anzahl an Pflanzt{\"o}pfen, wenn Sie
  Randpflanzen mit ber{\"u}cksichtigen wollen! \textbf{(1 Punkt)}
\item Welche $Pflanztopf$fl{\"a}che in $m^2$ gegeben der Anzahl an
  Pflanzt{\"o}pfen inklusive Randpflanzen ben{\"o}tigen Sie im Gew{\"a}chshaus am
  Anfang der Keimungsphase?  \textbf{(3 Punkte)}
\item Berechnen Sie die ben{\"o}tigte Menge an Torf in Liter $l$, die Sie f{\"u}r
  das Bef{\"u}llen der Pflanzt{\"o}pfe ben{\"o}tigen! Gehen Sie von \textit{einem
    Zylinder} f{\"u}r die Pflanzt{\"o}pfe aus!  \textbf{(3 Punkte)}
\item Berechnen Sie die Kosten in EUR f{\"u}r Ihre Torfbestellung! \textbf{(1
    Punkt)}
\end{enumerate}



 
\clearpage
% ----------------------------------------------------------------------- 

\section{Aufgabe \hfill (10 Punkte)}

\textit{Geben Sie grunds{\"a}tzlich Formeln und Rechenweg zur L{\"o}sung der
  Teilaufgaben mit an!} \\[1Ex]

%% --------------------------------------------------------------------
\hfill\href{https://youtu.be/aBxLkdF-c4M}{\includegraphics[width =
  2cm]{img/youtube}} %%youtube
\hspace{2Ex}
%% --------------------------------------------------------------------

\paragraph{Solar- \& Biogasanlagen}



Um die Energiekosten Ihres Betriebes zu senken, wollen Sie eine Solaranlage
auf den Schweinestall montieren lassen. Sie messen Ihren Stall und finden
folgende Ma{\ss}e wieder. Die vordere Seite des Schweinestall hat eine H{\"o}he
$h_v$ von $7m$. Die hintere Seite des Schweinestall hat eine
H{\"o}he $h_b$ von $8m$. Der Schweinestall hat eine Tiefe $t$ von
$13m$ und eine Breite $b$ von $30m$.

\begin{enumerate}
\item Skizzieren Sie den Schweinestall auf dem die Solaranlage montiert
  werden soll! Erg{\"a}nzen Sie die Angaben f{\"u}r die H{\"o}hen $h_v$, $h_b$, die
  Tiefe $t$ und die Breite $b$ des Stalls!  \textbf{(2 Punkte)}
\item Berechnen Sie die Fl{\"a}che der schr{\"a}gen, neuen Solaranlage auf dem
  Schweinestall! \textbf{(3 Punkte)}
\end{enumerate}

Ebenfalls planen Sie eine neue Biogasanlage f{\"u}r Ihren Betrieb. Der neue
Methantank hat einen Radius $r$ von $1m$. Leider gibt es ein
paar bauliche Beschr{\"a}nkungen auf dem Grundst{\"u}ck. Ihr Fundament des
zylindrischen Methantanks kann nur ein Gewicht von maximal
$14t$ aushalten bevor der Tank wegbricht. Sie rechnen eine
Sicherheitstoleranz von $10\%$ ein beinhaltend das Gewicht des
Methantanks. In fl{\"u}ssiger Form hat
Methan bei $-80^\circ\text{C}$ eine Dichte von
$235kg/m^3$. Bei $-100^\circ\text{C}$ hat Methan eine Dichte
von $290kg/m^3$. Sie betrieben Ihre Anlage bei
$-85^\circ\text{C}$.

\begin{enumerate}
  \setcounter{enumi}{2}
\item Extrapolieren Sie die effektive Dichte des Methans in Ihrem
  Methantank! Welche Annahme haben Sie getroffen? \textbf{(1 Punkt)}
\item Berechnen Sie wie viel Kubikmeter $m^3$ Sie in den Methantank f{\"u}llen
  k{\"o}nnen, bevor das Fundament nachgibt! \textbf{(2 Punkte)}
\item Berechnen Sie die maximale H{\"o}he $h_{max}$ in $m$ f{\"u}r den gef{\"u}llten
  Methantank mit dem Radius $r$, bevor das Fundament wegbricht! \textbf{(2
    Punkte)}
\end{enumerate}

 
\clearpage
% -----------------------------------------------------------------------

\section{Aufgabe \hfill (10 Punkte)}

\textit{Geben Sie grunds{\"a}tzlich Formeln und Rechenweg zur L{\"o}sung der
  Teilaufgaben mit an!} \\[1Ex]

%% --------------------------------------------------------------------
\hfill\href{https://youtu.be/https://youtu.be/k2G52hMIfqk}{\includegraphics[width = 2cm]{img/youtube}} %%youtube
\hspace{2Ex}
%% --------------------------------------------------------------------

%% --------------------------------------------------------------------
{\tiny\textbf{Stichworte:} Riesenfaultier $\bullet$ Evolution der Avocado $\bullet$ Bluetooth $ \bullet$ Blauzahn $\bullet$ Colonia Dignidad $\bullet$ ODESSA $\bullet$ Rattenlinie $\bullet$ Adolf Eichmann}
%% --------------------------------------------------------------------


\paragraph{Aligatorenbirnen und Blaubeeren}



"'Sind Sie ein Riesenfautier oder warum kaufen Sie so viele
Aligatorenbirnen?"', spricht es hinter Ihnen. Irritiert drehen Sie sich um
und blicken in das puderrote Gesicht von einer Rentnerin mit Monokel. "'Wieso?"', entfährt es
Ihnen und Sie bereuen sogleich die Frage. Sofort werden Sie zu einem
Whiteboard voller roter Schnüre geschliffen und müssen folgenden
mathematischen untermauerten Argumenten im Netto über sich ergehen
lassen. Da kommen Sie nicht mehr raus, also können Sie auch gleich
mitmachen. Das Problem liegt in Chile\footnote{Die Quelle der Inspiration
  für die Aufgabe waren folgende Reportagen:
  \href{https://www.amnesty.ch/de/ueber-amnesty/publikationen/magazin-amnesty/2021-3/bis-zum-letzten-tropfen}{"`Bis
  zum letzten Tropfen"' in AMNESTY – Magazin der Menschenrechte vom August
  2021} und
\href{https://www.welthungerhilfe.de/welternaehrung/rubriken/klima-ressourcen/wassernot-in-chile-eine-folge-der-privatisierung}{"`Wasserknappheit
  in Chile: Eine Folge der Privatisierung?"' in Die Welternährung dem
  Fachjournal der Welthungerhilfe vom April 2022.}}. Tja, die Deutschen und Südamerika.\\

Zuerst werden Ihre Fähigkeiten getestet, der Mathematik folgen zu können. Oder berechnen Sie gerade den Einkauf von einer Rentnerin mit Monokel?\\

\begin{enumerate}
\item Wenn 3 Blaubeerschalen 5.97 Euro kosten,  wie viel kosten 6 Schalen? \textbf{(2 Punkte)}
\item Wenn Sie die 6 Blaubeerschalen gekauft haben, wie viele Aligatorbirnen zu je 0.99 EUR können Sie sich dann noch für 50 EUR leisten? \textbf{(1 Punkt)}
\end{enumerate}

Das Whiteboard beinhaltet folgende Liste mit Informationen zum Wasserverbrauch bei der Produktion von Gemüse aus Chile. Seltsam, was man so alles in einem Netto über Gemüse erfährt.
  
\begin{itemize}[noitemsep]
\item Ein Kilo Strauchtomaten benötigt 160l Wasser. Eine Strauchtomate wiegt 90 - 130g.
\item Ein Kilo Salat benötigt 100l Wasser. Ein Salatkopf wiegt 280 - 490g.
\item Ein Kilo Avocado benötigt 980l Wasser. Eine Avocado wiegt 120 - 410g.
\item Ein Kilo Blaubeeren benötigt 880l Wasser. Eine Blaubeere wiegt 3.1 - 3.7g.
\end{itemize}

\begin{enumerate}
  \setcounter{enumi}{2}
\item Berechnen Sie den Wasserverbrauch für die Produktion für jeweils eine Strauchtomate, einem Salat, einer Avocado und einer Blaubeeren. Stellen Sie das Ergebnis als Tabelle dar! \textbf{(3 Punkte)}
\end{enumerate}

Chile exportiert im großem Ausmaß Blaubeeren und Avocados. In dem Exportjahr 2021 blieben die Erträge von Blaubeeren mit \ensuremath{8\times 10^{4}}t in dem prognostizierten Rahmen. Die Menge steigerte sich um 6.8\%. Die Exporte für Avocados stiegen in dem gleichen Zeitraum um 21.2\% auf \ensuremath{2.3\times 10^{5}}t.

\begin{enumerate}
  \setcounter{enumi}{3}
\item Wie viele Liter Wasser hat Chile in dem Exportjahr 2020 exportiert? \textbf{(2 Punkte)}
\end{enumerate}

Chile ist eines der wenigen Länder der Welt, die ihr Wasser komplett privatisiert haben. Derzeit sind nur ein Prozent des Wassers des Landes für den häuslichen Verbrauch vorgesehen. In den Dörfern der Anbauregionen versorgen Tankwagen die Bevölkerung jede Woche mit Wasser, es gibt etwa 52 Liter Wasser pro Kopf für den täglichen Bedarf. In \textit{Deutschland} liegt der Verbrauch bei 10 - 15 Liter pro Minute Duschen und 3 - 12 Liter pro Minute H{"a}ndewaschen.

\begin{enumerate}
  \setcounter{enumi}{4}
\item Mit der rationierten Wassermenge aus Chiles Anbaugebieten können Sie in \textit{Deutschland} wie oft Ihren Bedarf stillen? \textbf{(1 Punkt)}
\end{enumerate}

Das alles hätten Sie nicht von einer Rentnerin mit Monokel erwartet. Ganz schön viele Informationen wurden da zusammengetragen.

\begin{enumerate}
  \setcounter{enumi}{5}  
  \item Nennen Sie eine \textit{Daten}quelle im Internet, wo Sie mehr Informationen zu landwirtschaftlichen Daten oder klimatischen, wirtschaftlichen und gesellschaftlichen Daten erhalten! \textbf{(1 Punkt)}
\end{enumerate} 
\clearpage
% ----------------------------------------------------------------------- 

\section{Aufgabe \hfill (12 Punkte)}

\textit{Geben Sie grunds{\"a}tzlich Formeln und Rechenweg zur L{\"o}sung der
  Teilaufgaben mit an!} \\[1Ex]

%% --------------------------------------------------------------------
\hfill\href{https://youtu.be/WZSxntiNF8s}{\includegraphics[width = 2cm]{img/youtube}} %%youtube
\hspace{2Ex}
%% --------------------------------------------------------------------

%% --------------------------------------------------------------------
{\tiny\textbf{Stichworte:} Kardaschow-Skala $\bullet$ Dyson-Sphäre $\bullet$ Hohlerde $\bullet$ Entropie $\bullet$ Proton $r_P = 1.7 \times 10e-15$ $\bullet$ Wasserstoff $r_H = 5.3\times 10e-11$}
%% --------------------------------------------------------------------

\paragraph{Die Dampfnudelerde}



"'Was für einen Unsinn!"', rufen Sie. Jetzt haben Sie kostbaren Schlaf prokrastiniert um einem Ernährungswissenschaftler auf YouTube über die Erde als Dampfnudel zu lauschen. Irgendwie passt es dann doch mit der Analogie. Übermüdet müssen Sie darüber nachdenken, warum vor 67 Millionen Jahren die Dinosaurier - so groß sie auch waren - nicht von der Schwerkraft zu Boden gerissen wurden. Hat der Dampfplauderer etwa recht und war die Schwerkraft vor Millionen von Jahren eine andere?  Sind deshalb alle Lebewesen auf der Erde \textit{heutzutage} so viel kleiner, weil die Schwerkraft größer ist als damals? War die Erde kleiner und hatte weniger Masse? Oder ist es nur ein Rechenfehler wie bei der Theorie der Hohlerde von Edmond Halley aus dem 17.–18. Jahrhundert? Müde reiben Sie sich die Augen. So wird es nichts mehr mit dem Schlafen, dann können Sie auch mal etwas rechnen\footnote{Die Quelle der Inspiration
  für die Aufgabe war folgender Artikel:
  \href{https://hpd.de/artikel/erde-dampfnudel-22236}{"Skeptische Anmerkungen --- Die Erde als Dampfnudel" in Der Humanistische Pressedienst}}.  \\

Betrachten wir die Schwerkraft oder Gewichtskraft, die auf Lebewesen damals und heute gewirkt haben soll. Nehmen Sie für die Fallbeschleunigung $g$ der Erde \textit{heutzutage} einen Wert von 9.78m/s$^2$ an. Im Weiteren hat die Erde einen ungefähren Durchmesser von \ensuremath{1.2742\times 10^{4}}km und eine mittlere Dichte $\rho$ von 5.21g/cm$^3$. Das Gewicht von einem heute lebenden afrikanischen Elefanten liegt bei 5t bis 7t und das Gewicht von einem Tyrannosaurus rex (T. rex) bei 4.5t bis 8t.

\begin{enumerate}
\item Welchen Durchmesser müsste die Erde vor 67 Millionen Jahren gehabt haben, wenn Dinosaurier und Elefanten die gleiche Gewichtskraft $\overrightarrow{F_G}$ damals und heute erfahren hätten? \textit{Beantworten Sie die Frage anhand der folgenden Teilaufgaben!}
\begin{enumerate}
\item Berechnen Sie die Fallbeschleunigung von vor 67 Millionen Jahren unter der obigen Annahme gleich wirkender Gewichtskraft $\overrightarrow{F_G}$ auf Elefant und Dinosaurier! \textbf{(1 Punkt)}
\item Berechnen Sie Masse der heutigen Erde! \textbf{(2 Punkte)}
\item Schließen Sie über die Masse auf den Durchmesser der Erde vor 67 Millionen Jahren! \textbf{(2 Punkte)}
\end{enumerate}
\item Beantworten Sie die Eingangsfrage mit 1-2 Antwortsätzen! \textbf{(1 Punkt)}
\end{enumerate}

Die Distanz zwischen Sonne und Erde entspricht 1.01 astronomische Einheiten ($AE$). Die Einheit 1 AE wird mit \ensuremath{1.55\times 10^{8}}km angegeben. Der \textit{massebehaftete} Sonnenwind besteht aus 84\% Wasserstoffkernen mit einer molaren Masse von 1.02g/mol, 12\% Heliumkernen mit 4.11g/mol sowie 4\% weiteren Atomkernen mit  145.31g/mol. Die Teilchendichte bei Eintritt in die Erdatmosphäre liegt zwischen 0.4 bis 100 Teilchen cm$^{-3}$ pro Sekunde mit einer mittleren Teilchendichte von 7cm$^{-3}$ pro Sekunde. \\

\textit{Lösen Sie den folgenden Aufgabenteil mit einer aussagekräftigen Skizze!}

\begin{enumerate}
  \setcounter{enumi}{3}
\item Berechnen Sie die Anzahl an massebehafteten Teilchen des Sonnenwindes, die die gesamte Erde pro Sekunde treffen! \textbf{(2 Punkte)}
\item Berechnen Sie die Anzahl an massebehafteten Teilchen des Sonnenwindes, die die Sonne pro Sekunde in alle Richtungen aussendet! \textbf{(2 Punkte)}
\item Berechnen Sie die Masse, die die Erde pro Jahr durch die \textit{massebehafteten} Teilchen des Sonnenwind zunimmt! \textbf{(2 Punkte)}
\end{enumerate}

%\blfootnote{\tiny\textbf{Stichworte:} Kardaschow-Skala $\bullet$ Dyson-Sphäre $\bullet$ Hohlerde $\bullet$ Entropie}

% https://de.wikipedia.org/wiki/Kardaschow-Skala 
\clearpage
% ----------------------------------------------------------------------- 

\section{Aufgabe \hfill (10 Punkte)}

\textit{Geben Sie grunds{\"a}tzlich Formeln und Rechenweg zur L{\"o}sung der
  Teilaufgaben mit an!} \\[1Ex]

%% --------------------------------------------------------------------
\hfill\href{https://youtu.be/n451XnhtSh4}{\includegraphics[width = 2cm]{img/youtube}} %%youtube
\hspace{2Ex}
%% --------------------------------------------------------------------

\paragraph{Entschuldigung, ist das Ihre Feder in meinem Auge?}



So h{\"o}rt man h{\"a}ufiger h{\"o}fliche Puten in Mastst{\"a}llen sagen. Das
ist nat{\"u}rlich etwas ung{\"u}nstig, den dann kommt es zu Picken und
Kannibalismus. Denn wenn der Nachbar nervt, dann muss zu Ma{\ss}nahmen
gegriffen werden. Kennt jeder aus einer mittelm{\"a}{\ss}igen Wohngemeinschaft. Das
wollen wir aber als vorsorgliche Puten-Halter:innen
nicht\footnote{Die Quelle der Inspiration f{\"u}r die Aufgabe war der folgende
  wissenschaftliche Artikel:
  \href{https://www.efsa.europa.eu/en/efsajournal/pub/7788}{EFSA Panel on
    Animal Health and Welfare, et al. (2023) Welfare of broilers on
    farm. EFSA Journal 21.2}.}. Betrachten wir also einmal das Platzangebot
(eng. \textit{space allowance}, abk. \textit{SA}) der Puten
f{\"u}r vier T{\"a}tigkeiten und versuchen die notwendige Fl{\"a}che zu optimieren. Wie
immer gibt es daf{\"u}r eine mathematische Formel:


\begin{center}
  \begin{tabular}{cc}
    $SA = \sum^n_{i = 1} (A_i \times PB_i)$ & $A_i = \pi \times (r_i + R_i)^2$\\
  \end{tabular}
\end{center}

\vspace{-2Ex}

mit

\begin{itemize}[noitemsep]
\item $SA$ dem ben{\"o}tigten Platzangebot aller aufsummierten Verhalten $i$.
\item $A_i$ dem ben{\"o}tigten Platz f{\"u}r ein Verhalten $i$. 
\item $PB_i$ dem Anteil des Auftretens eines Verhaltens $i$.
\item $r_i$ dem Radius Pute plus dem ben{\"o}tigten Radius f{\"u}r das Verhalten $i$.
\item $R_i$ dem notwendigen Abstand zu den Nachbarn f{\"u}r das Verhalten $i$.    
\item $i$ dem Verhalten: (1) drinking/eating, (2) sitting, (3)
  foraging incl. scratching und (4) standing.
\end{itemize}

In der folgenden Tabelle 1 sind die Werte f{\"u}r $r_i$, $R_i$ und $PB_i$ f{\"u}r
ein spezifisches Verhalten $i$ aus drei wissenschaftlichen
Ver{\"o}ffentlichungen dargestellt.

\vspace{-1Ex}

{\small
\begin{knitrout}
\definecolor{shadecolor}{rgb}{0.969, 0.969, 0.969}\color{fgcolor}\begin{table}[!h]
\centering
\begin{tabular}{llll}
\toprule
  & Aldridge et al. (2021) & Baxter et al. (2022) & Jabcobs et al. (2019)\\
\midrule
drinking/eating & 26cm; 21cm; 3.5\% & 30cm; 24cm; 3.5\% & 43cm; 19cm; 12.1\%\\
sitting & 31cm; 28cm; 25.4\% & 33cm; 26cm; 25.4\% & 38cm; 37cm; 50.1\%\\
foraging incl. scratching & 34cm; 20cm; 0.4\% & 34cm; 28cm; 0.2\% & 31cm; 24cm; 0.8\%\\
standing & 35cm; 34cm; 18.1\% & 31cm; 23cm; 16.4\% & 40cm; 23cm; 12.3\%\\
\bottomrule
\end{tabular}
\end{table}

\end{knitrout}
}

\begin{enumerate}
\item Erstellen Sie eine zusammenfassende Tabelle mit den mittleren Werten
  f{\"u}r $r$, $R$ und $PB$ aus der obigen Tabelle 1 f{\"u}r die jeweiligen
  Verhalten! \textbf{(3 Punkte)}
\item Erg{\"a}nzen Sie eine Spalte mit dem ben{\"o}tigten Platz $A$ f{\"u}r das jeweilige
  Verhalten, welches sich aus den mittleren Werten ergibt! \textbf{(1 Punkt)}
\item Berechnen Sie das ben{\"o}tigte Platzangebot $SA$ f{\"u}r alle betrachteten
  Verhalten! \textbf{(1 Punkt)}
\item Skizzieren Sie die Werte $r_i$, $R_i$ und $A_i$ f{\"u}r zwei
  nebeneinander agierende Puten f{\"u}r ein Verhalten $i$. Nutzen
  Sie hierf{\"u}r vereinfachte geometrische Formen! \textbf{(2 Punkte)}
\item Sie entnehmen der Literatur folgende Aussage zur Verteilung der
  Puten in der Fl{\"a}che $A$: \textit{"`Assuming, that the
    animals will optimally and equally distribute in an area $A$, we observe a
    small part, which is not covered. This area is called
    $\omega$ and is calculated with $\omega = \tfrac{A}{0.9069}$."'
  } Veranschaulichen Sie die Fl{\"a}che $\omega$ in einer aussagekr{\"a}ftigen
  Abbildung!  \textbf{(1 Punkt)}
\item Ein Tier braucht Platz f{\"u}r sich selbst. Berechnen Sie nun die
  K{\"o}rperfl{\"a}che $a$, die ein Tier einnimmt. Welche Annahmen haben Sie f{\"u}r
  die Berechnung der K{\"o}rperfl{\"a}che getroffen? \textbf{(2 Punkte)}
\end{enumerate}



 
\clearpage
% ----------------------------------------------------------------------- 

\section{Aufgabe \hfill (8 Punkte)}

\textit{Geben Sie grunds{\"a}tzlich Formeln und Rechenweg zur L{\"o}sung der
  Teilaufgaben mit an!} \\[1Ex]

%% --------------------------------------------------------------------
\hfill\href{https://youtu.be/1B53cVFIU7Q}{\includegraphics[width =
  2cm]{img/youtube}} %%youtube
\hspace{2Ex}
%% --------------------------------------------------------------------

\paragraph{Nelken von den Molukken}



In der Ausstellung "`Europa und das Meer"' im Deutschen Historischen Museum in
Berlin gab es folgendes Zitat {\"u}ber die Probleme der fr{\"u}hen Hochseeschifffahrt.

\begin{quote}
  >>Ohne ausreichende Zufuhr von Vitamin C stellen sich nach 40 Tagen die
  ersten Symptome ein; die ersten Toten sind nach 70 Tagen zu beklagen;
  nach 100 Tagen rafft die Skorbut eine ganze Schiffsbesatzung dahin<<
\end{quote}

Ferdinand Magellan stach im Jahre 1519 in See um eine Passage durch den
s{\"u}damerikanischen Kontinent zu finden. Zu seiner Flotte geh{\"o}rten
f{\"u}nf Schiffe - das Flaggschiff Trinidad, die San Antonio, die Victoria, die
Concepci{\'o}n und die Santiago - mit einer Besatzung von insgesamt
237 Mann. 

\begin{enumerate}
\item Stellen Sie den Verlauf der Anzahl an Matrosen auf einem Schiff der
  Flotte in der Form einer {\"U}berlebenszeitkurve dar! Beschriften Sie die
  Achsen entsprechend! \textbf{(2 Punkte)} 
\item Was ist die Besonderheit der Überlebenszeitkurve? Begründen Sie Ihre Antwort! \textbf{(2 Punkte)} 
\item Sch{\"a}tzen Sie die {\"U}berlebenswahrscheinlichkeit nach 95 Tagen
  aus Ihrer Abbildung ab! \textbf{(1 Punkt)} 
\end{enumerate}

Der Chronist an Bord der Trinidad, Antonio Pigafetta, schrieb in seinem
Bericht "`[...] Um nicht Hungers zu sterben, a{\ss}en wir das Leder, mit dem
die gro{\ss}e Rahe zum Schutz der Taue umwunden war."' Insbesondere die
Mannschaft der Concepci{\'o}n erlitt gro{\ss}e Verluste durch die Skrobut bei der
{\"U}berquerung des Pazifiks, da durch Erkundungsfahrten weniger Zeit blieb, um
wilden Sellerie aufzunehmen. Wilder Sellerie enth{\"a}lt
$5000\mu g/50mg$ Vitamin C. Der Bedarf liegt bei
$120mg$ pro Tag f{\"u}r M{\"a}nner.

\begin{enumerate}
  \setcounter{enumi}{2}
\item Berechnen Sie die notwendige Menge in $t$ an
  aufzunehmenden wilden Sellerie auf die Concepci{\'o}n f{\"u}r die ununterbrochene
  Fahrt von drei Monate und 24 Tage {\"u}ber den Pazifik!
  \textbf{(3 Punkte)}
\item Skizzieren Sie die {\"U}berlebenszeitkurve f{\"u}r die Concepci{\'o}n im
  Vergleich zu der {\"U}berlebenszeitkurve der Trinidad! Beschriften Sie die
  Achsen! \textbf{(2 Punkte)}
\end{enumerate}

 
\clearpage
% ----------------------------------------------------------------------- 

\section{Aufgabe \hfill (10 Punkte)}

\textit{Geben Sie grunds{\"a}tzlich Formeln und Rechenweg zur L{\"o}sung der
  Teilaufgaben mit an!} \\[1Ex]

%% --------------------------------------------------------------------
\hfill\href{https://youtu.be/q-qYK4Chslg}{\includegraphics[width =
  2cm]{img/youtube}} %%youtube
\hspace{2Ex}
%% --------------------------------------------------------------------

\paragraph{Event Horizon -- Am Rande des Universums}



Die Sonne hat eine aktuelle, angenommene Masse von $\ensuremath{2\times 10^{29}}$kg. Wenn
die Sonne nun am Ende ihrer Lebenszeit zu einem schwarzen Loch mit dem
Radius von $3000$m kollabiert, wird die Sonne $40$\%
der aktuellen Masse verloren haben. Ein Lichtteilchen mit der Masse $m_f$
und der Fluchtgeschwindigkeit $v_f$ will dem schwarzen Loch entkommen.  Sie
haben folgende Formeln f{\"u}r die kinetische Energie des Lichtteilchens
$E_{kin}$ und der Graviationsenergie des schwarzen Lochs $E_{grav}$
gegeben\footnote{Die Quelle der Inspiration
  für die Aufgabe war ein Montagnachtfilm:
  \href{https://de.wikipedia.org/wiki/Event_Horizon_–_Am_Rande_des_Universums}{Event Horizon – Am Rande des Universums}}.

\begin{center}
  \begin{tabular}{cc}
    $E_{kin} = \cfrac{1}{2}m_fv_f^2$ & $E_{grav} = \cfrac{Gm_sm_f}{r_s}$\\
  \end{tabular}
\end{center}

mit

\begin{itemize}[noitemsep]
\item $m_f$, gleich der Masse [kg] des fliehenden Objektes
\item $m_s$, gleich der Masse [kg] des station{\"a}ren Objekts
\item $r_s$, gleich dem Radius [m] des station{\"a}ren Objekts  
\item $G$, gleich der Gravitationskonstante mit $6.165 \cdot 10^{-11}
  m^3(kg \cdot s^2)^{-1}$ 
\end{itemize}

Im Folgenden wollen wir uns mit der Frage besch{\"a}ftigen, ob das
Lichtteilchen der Gravitation des schwarzen Lochs entkommen kann.

\begin{enumerate}
\item Geben Sie die Formel f{\"u}r die Fluchtgeschwindigkeit $v_f$ an! 
  \textbf{(1 Punkt)}
\item {\"U}berpr{\"u}fen Sie Ihre umgestellte Formel nach $v_f$ anhand der Einheiten!
  \textbf{(1 Punkt)} 
\item Berechnen Sie die notwendige Fluchtgeschwindigkeit $v_f$ des
  Lichtteilchens mit den angegebenen Informationen! \textbf{(2 Punkte)}
\item Gehen Sie von einer Lichtgeschwindigkeit von $\ensuremath{2.8\times 10^{8}}m/s$
  aus. Kann das Lichtteilchen der Gravitation des schwarzen Lochs
  entkommen? Begr{\"u}nden Sie Ihre Antwort! \textbf{(2 Punkte)}
\item Stellen Sie den Zusammenhang zwischen dem sich verringernden Radius
  $r$ des schwarzen Lochs bei gleichbleibender Masse $m_s$
  und der notwendigen Fluchtgeschwindigkeit $v_f$ in einer Abbildung dar!
  \textbf{(2 Punkte)}
 \item Ein Amboss und ein Lolli st{\"u}rzen aus gro{\ss}er und gleicher H{\"o}he
  in ein schwarzes Loch. Welches der beiden Objekte {\"u}berschreitet zuerst
  den Ereignishorizont des schwarzes Loches? Begr{\"u}nden
  Sie Ihre Antwort mathematisch! \textbf{(2 Punkte)}  
\end{enumerate}

 
\clearpage
% -----------------------------------------------------------------------

\section{Aufgabe \hfill (10 Punkte)}

\textit{Geben Sie grunds{\"a}tzlich Formeln und Rechenweg zur L{\"o}sung der
  Teilaufgaben mit an!} \\[1Ex]

%% --------------------------------------------------------------------
\hfill\href{https://youtu.be/iCQogS6KhPM}{\includegraphics[width =
  2cm]{img/youtube}} %%youtube
\hspace{2Ex}
%% --------------------------------------------------------------------

\paragraph{Das Fermi Paradoxon}



Der Kernphysiker Enrico Fermi diskutierte 1950 auf dem Weg zum Mittagessen
im Los Alamos National Laboratory mit seinen Kollegen angebliche
UFO-Sichtungen und fragte schlie{\ss}lich: "`Where is everybody?"'. Warum seien
weder Raumschiffe anderer Weltraumbewohner noch andere Spuren
extraterrestrischer Technik zu beobachten? Wie lange w{\"u}rde eine au{\ss}erirdische
Zivilisation ben{\"o}tigen um die gesamte Milchstra{\ss}e zu
besuchen, wenn das maximale Reisetempo die Geschwindigkeit der Voyager 1 Sonde w{\"a}re?\footnote{Die Quelle der Inspiration
  für die Aufgabe war folgender Wikipediaeintrag:
  \href{https://de.wikipedia.org/wiki/Fermi-Paradoxon}{Fermi-Paradoxon}}\\[-1ex]

Wir treffen folgende Annahmen. Eine au{\ss}erirdische Zivilisation schickt $vier$
Voyager 1 {\"a}hnliche Sonden mit der Geschwindigkeit von $\ensuremath{6.2587\times 10^{4}}km/h$
los um sich auf den erreichten Planeten selbst zu replizieren. Nach
$500$ Jahren ist die Replikation abgeschlossen und wiederum
$vier$ Sonden werden ausgesendet. Gehen Sie von
$6.23$ Lichtjahren als mittlerer Abstand der Sterne in der
Milchstra{\ss}e aus. Es gibt $\ensuremath{1.5\times 10^{11}}$ Sterne in der Milchstra{\ss}e. Nehmen
Sie eine Lichtgeschwindigkeit von $\ensuremath{2.9\times 10^{8}}m/s$ an.

\begin{enumerate}
\item Skizzieren Sie in einer Abbildung die ersten vier Schritte der
  Vervielf{\"a}ltigung der Sonden in der Galaxie! Beschriften Sie die Abbildung
  mit der Dauer und der Anzahl an Sonden f{\"u}r jeden Schritt der Vervielf{\"a}ltigung! \textbf{(4 Punkte)}
\item Berechnen Sie die theoretische Anzahl an Vervielf{\"a}ltigungsschritten die
  ben{\"o}tigt werden um mit \textit{einem einzigen Vervielf{\"a}ltigungsschritt} die
  gesamten Sterne der Milchstra{\ss}e mit Sonden zu besuchen! \textbf{(2 Punkte)}
\item Berechnen Sie die Dauer, die eine au{\ss}erirdische Zivilisation
  ann{\"a}hrungsweise ben{\"o}tigt um die gesamten Sterne der Milchstra{\ss}e mit
  Sonden zu besuchen! \textbf{(2 Punkte)}
\item Bei einem vermutetet Alter der Erde von $\ensuremath{4.6\times 10^{9}}$ Jahren,
  wie oft war dann eine Sonde einer au{\ss}erirdischen Zivilisation schon zu
  Besuch? Korrigieren Sie Ihre Antwort mit dem Wissen, dass sich die
  Kontinentalplatten einmal alle $\ensuremath{8\times 10^{7}}$ Jahre vollst{\"a}ndig im
  Erdinneren umgewandelt haben! \textbf{(2 Punkte)}
\end{enumerate}


 
\clearpage
% -----------------------------------------------------------------------

\section{Aufgabe \hfill (10 Punkte)}

\textit{Geben Sie grunds{\"a}tzlich Formeln und Rechenweg zur L{\"o}sung der
  Teilaufgaben mit an!} \\[1Ex]

%% --------------------------------------------------------------------
\hfill\href{https://youtu.be/tDgr6fpkkYA}{\includegraphics[width =
  2cm]{img/youtube}} %%youtube
\hspace{2Ex}
%% --------------------------------------------------------------------

\paragraph{Pyramiden bauen}



Es stehen die mecklemburgischen Pyramidentage an und Sie sind auf
abenteuerlichen Wegen f{\"u}r den Bau der Pyramiden zust{\"a}ndig. Zu allem
{\"U}berfluss handelt es sich auch noch eine \textit{Reenactment}
Veranstaltung. Thema der diesj{\"a}hrigen Pyramidentage sind die Pyramiden von
Meroe, die den K{\"o}nigen und K{\"o}niginnen des historischen Reiches von Kusch in
Nubien, dem heutigen Sudan, als Grabst{\"a}tten dienten. Die Pyramiden in Meroe
fallen durch ihren steilen Winkel von 73 Grad im Vergleich zu
den {\"a}gyptischen Pyramiden mit 54 Grad auf. Die durchschnittliche
Seitenl{\"a}nge der Grundfl{\"a}che einer Pyramide betr{\"a}gt 38 K{\"o}nigsellen. Eine K{\"o}nigselle
misst 52.6cm.\\

\textit{L{\"o}sen Sie diese Aufgabe mit Hilfe einer Skizze der Pyramide. Bezeichnen
  Sie Seiten und die Winkel der Pyramide entsprechend!}

\begin{enumerate}
\item Bei der K{\"o}nigspyramide von Meroe soll eine Seitenl{\"a}nge der Grundfl{\"a}che
  38 K{\"o}nigsellen lang sein. Welche H{\"o}he der K{\"o}nigspyramide in $m$
  ergibt sich? \textbf{(1 Punkt)}
\item Die Au{\ss}enfl{\"a}chen der Pyramide soll begr{\"u}nt werden. F{\"u}r die
  Bepflanzung muss eine 7cm dicke Torfschicht auf die
  Pyramide aufgebracht werden. Berechnen Sie die ungef{\"a}hre Menge an
  ben{\"o}tigten Torf in $m^3$! \textbf{(2 Punkte)}
\end{enumerate}

Wie in jedem guten \textit{Reenactment} gibt es viel Oberschicht, aber nur
4 Sklaven, die Ihnen bei dem Bef{\"u}llen der Pyramide mit Schutt
zu Seite stehen. Leider haben Ihre Sklaven zu allem {\"U}berfluss auch noch
chronische Schulterschmerzen entwickelt, als sie von der
anstehenden Aufgabe erfahren haben. Gehen Sie daher von einer Effizienz der
Sklaven von 70\% aus. In eine Schubkarre passen
100 Liter.

\begin{enumerate}
  \setcounter{enumi}{2}
\item Wie oft m{\"u}ssen Ihre maladen Sklaven die Rampe mit der Schubkarre
  zur Spitze der Pyramide hochfahren um die Pyramide mit Schutt zu f{\"u}llen? \textbf{(1 Punkt)}
\item Berechnen Sie die L{\"a}nge der Rampe zur Spitze der Pyramide mit einem
  Anstellwinkel von $9^\circ$! \textbf{(2 Punkte)}
\item Wie weit reicht Ihre Rampe vom Fu{\ss} der Pyramide in die mecklemburgische
  Landschaft?  \textbf{(2 Punkte)}
\end{enumerate}

Bei der Besichtigung der Pyramide teilt Ihnen der leicht {\"u}bergewichtige
Pharao (Nebenberuf \textit{Mittelst{"a}ndler}) mit, das die Pyramide zu
flach sei und somit nicht in die mecklemburgische Landschaft passen
w{\"u}rde. Sie m{\"u}ssen nochmal ran.

\begin{enumerate}
  \setcounter{enumi}{5}
\item Die Grundfl{\"a}che der Pyramide {\"a}ndert sich nicht. Berechnen Sie die
  {\"A}nderung der H{\"o}he in K{\"o}nigsellen, wenn sich der Anstellwinkel der
  Pyramide um $5^\circ$ {\"a}ndert!  \textbf{(2 Punkte)}
\end{enumerate}



% https://de.wikipedia.org/wiki/Rechtwinkliges_Dreieck
% https://www.matheretter.de/wiki/pyramide
% https://de.wikipedia.org/wiki/Knickpyramide 
\clearpage
% -----------------------------------------------------------------------

\section{Aufgabe \hfill (12 Punkte)}

\textit{Geben Sie grunds{\"a}tzlich Formeln und Rechenweg zur L{\"o}sung der
  Teilaufgaben mit an!} \\[1Ex]

%% --------------------------------------------------------------------
\hfill\href{https://youtu.be/3LAq3R0rS14}{\includegraphics[width =
  2cm]{img/youtube}} %%youtube
\hspace{2Ex}
%% --------------------------------------------------------------------

\paragraph{Geocaching -- Von Satelliten und Plastikdosen}



Es ist Wochenende und das Wetter ist \textit{sweet}. Sie schwingen sich auf
Ihr Cachermobil um mit 15km/h, geleitet von modernster
Satellietentechnologie und einem Supercompter aus dem Jahr 2000 in Ihren
H{\"a}nden, Plastikdosen in der Natur und an sehenswerten Orten zu finden. Sie
wollen diesmal endlich die aufw{"a}rts Terrainchallenge
durchf{\"u}hren. Die Reihenfolge der Caches nach Terrainwertung gibt daher
die von Ihnen abzufahrenden Orte vor. Die Terrain- und
Schwierigkeitswertungen laufen von 1 (leichteste Wertung) bis 5
(schwierigste Wertung) in 0.5 Schritten. Folgende Informationen zu den
Orten und den entsprechenden Caches stehen Ihnen f{\"u}r Ihre Planung der Route
zu Verf{\"u}gung\footnote{Die Quelle der Inspiration
  für die Aufgabe war folgende Tätigkeit:
  \href{https://www.geocaching.com/play}{Geocaching -- Mach mit bei der weltweit größten Schatzsuche.}}.

\begin{center}
  \begin{tabular}{ ccc }
    \toprule
    Ort & Cache & Wertung (S|T|G) \\
    \midrule
    A & GCYHV18 & 2.0 | 1.5 | Mikro \\
    B & GC9JNE0 & 1.0 | 4.5 | Normal \\ 
    C & GCBZXFR & 3.0 | 2.5 | Klein \\ 
    D & GCQP8O4 & 3.5 | 5.0 | Klein \\ 
    E & GC6ZGC8 & 2.5 | 4.0 | Mikro \\     
 \bottomrule
\end{tabular}
\end{center}

Im Weiteren sind Ihnen folgende Informationen zu den Entfernungen der Orte
zug{\"a}nglich. Der Entfernungsvektor $\overrightarrow{AC}$ ist
$6$km. Im Weiteren ist Ihnen der Entfernungsvektor
$\overrightarrow{CB}$ mit $7.5$km bekannt. Der
Entfernungsvektor $\overrightarrow{BE}$ ist das $1.3$-fache
des Entfernungsvektor $\overrightarrow{CB}$. Wenn Sie von dem Ort A den Ort
C anpeilen, so liegt der Ort B ungef{\"a}hr $25^\circ$
n{"o}rdlich. Wenn Sie von dem Ort C den Ort B anpeilen, so liegt
der Ort D ungef{\"a}hr $35^\circ$ {\"o}stlich. Vom Ort B
betrachtet, bilden die Orte C und D einen rechten Winkel am Ort B. Der Ort
B liegt auf gerader Linie zwischen den Orten C und E. Somit liegt der Ort E
n{"o}rdlich von B. Die Strecke zwischen A und E ist nicht
passierbar. Sie starten an dem Ort B Ihre Cachertour. \\

\begin{enumerate}
\item L{\"o}sen Sie diese Aufgabe mit Hilfe einer aussagekr{\"a}ftigen Skizze der
  Orte und Caches. Bezeichnen Sie die Strecken und die Winkel Ihrer Skizze
  entsprechend! \textbf{(2 Punkte)}
\item Welche Strecke in $km$ legen Sie bei der Bew{\"a}ltigung der
  aufw{"a}rts Terrainchallenge zur{\"u}ck? \textbf{(5
    Punkte)}
\item Gehen Sie von einer zus{\"a}tzlichen Suchzeit in Stunden f{\"u}r die
  Caches an den jeweiligen Orten zur reinen Reisezeit mit Ihrem Cachermobil
  aus. Die Suchzeit in Stunden f{\"u}r \textit{jeden einzelnen} Cache wird durch die
  Funktion
  \begin{equation*}
    Suchzeit = 0.05 + 0.13 \cdot Schwierigkeit
  \end{equation*}  
  beschreiben.  Wie lange in Stunden ben{\"o}tigen Sie um die
  aufw{"a}rts Terrainchallenge zu erf{\"u}llen? \textbf{(3 Punkte)}
\item An der h{\"o}chsten Schwierigkeit m{\"u}ssen Sie angeln. Ihre Angel ist
  ausgefahren 6m lang. Erreichen Sie einen Cache in der H{\"o}he
  von 7.6m?  Berechnen Sie dazu Ihre maximale m{\"o}gliche
  Angelh{\"o}he! Welche Annahmen mussten Sie treffen um die Aufgabe zu l{\"o}sen? \textbf{(2 Punkte)} 
\end{enumerate}

 
\clearpage
% -----------------------------------------------------------------------

\section{Aufgabe \hfill (10 Punkte)}

\textit{Geben Sie grundsätzlich Formeln und Rechenweg zur L{\"o}sung der
  Teilaufgaben mit an!} \\[1Ex]

%% --------------------------------------------------------------------
\hfill\href{https://youtu.be/4-dSaPMhK9s}{\includegraphics[width =
  2cm]{img/youtube}} %%youtube
\hspace{2Ex}
%% --------------------------------------------------------------------

%% --------------------------------------------------------------------
{\tiny\textbf{Stichworte:} Brot aus Luft $\bullet$ Walöl $\bullet$ Haber-Bosch-Verfahren $\bullet$ 1. Weltkrieg $\bullet$ 40\% N im menschlichen Körper $\bullet$ Positivist}
%% --------------------------------------------------------------------

\paragraph{Die atmende Wand und Brot aus Luft}



Als Kellerkind vom Dorf wollen das Ausmaß der Radonbelastung in ihrem Kellerzimmer bestimmen und
lüften daher nicht. Spart dann auch Energie und lüften wird sowieso überschätzt. Während einer Messperiode von 7:00 Uhr bis
21:00 bestimmen Sie dreimal automatisch die Radonbelastung in
Ihrem Kellerraum in $Bq/m^3$. Es ergibt sich folgende Abbildung\footnote{Die Quelle der Inspiration
  für die Aufgabe war folgender Artikel:
  \href{https://de.wikipedia.org/wiki/Atmende_Wand}{Atmende Wand}}.

\begin{knitrout}
\definecolor{shadecolor}{rgb}{0.969, 0.969, 0.969}\color{fgcolor}

{\centering \includegraphics[width=\maxwidth]{img/math-10-1} 

}


\end{knitrout}

\vspace{-0.75cm}

\begin{enumerate}
\item Wie lange dauert es in Stunden bis Sie eine kritische Belastung von
  320$Bq/m^3$ in Ihrem ungelüfteten Kellerraum erreicht haben?
  \textbf{(2 Punkte)}
\end{enumerate}

Radon zerfällt mit einer Halbwertszeit von 3.5d zu
Polonium. Polonium wiederum zerfällt mit einer Halbwertszeit von
180d zu Blei. Nur Radon und Polonium tragen zur
radioaktiven Strahlenbelastung bei.

\begin{enumerate}
  \setcounter{enumi}{1}
\item Wie lange dauert es in Stunden bis Ihre kritische Radonbelastung von
  320$Bq/m^3$ auf unter 80$Bq/m^3$ gefallen ist?
  \textbf{(4 Punkte)}
\end{enumerate}

Folgende Tabelle enthält die Informationen zur Zusammensetzung der
normalen Umgebungsluft.

\begin{center}
  \begin{tabular}{ c|c|c|c }
     & Vol-\% & M [g/mol] & ppm \\
    \hline
    Stickstoff & 79.7 & 28.1 &
                                                    \phantom{1000000000000}\strut\\
        \hline
    Sauerstoff & 19.5 & 16.2 &
                                                    \phantom{10000000}\strut\\
        \hline
    Kohlenstoffdioxid & 0.045 & 12.1 & \phantom{10000000}\strut\\     
     \hline
\end{tabular}
\end{center}

\begin{enumerate}
   \setcounter{enumi}{2}
\item Rechnen Sie die Volumenprozente (Vol-\%) der Umgebungsluft in die
  entsprechenden ppm-Werte um und ergänzen Sie die berechneten ppm-Werte in
  die Tabelle!  \textbf{(1 Punkt)}
\end{enumerate}

Während Sie Ihr etwas pappiges Toastbrot mampfen kommt Ihnen die Dokumentation über Brot aus Luft in den Sinn. Sie denken darüber ein wenig nach. Für die Umwandlung von Stickstoff $N_2$ mit Wasserstoff $H_2$ zu Ammoniak
$NH_3$ gilt folgende Reaktionsgleichung\footnote{Die Quelle der Inspiration
  für die Aufgabe war folgender Artikel:
  \href{https://www.gdch.de/netzwerk-strukturen/fachstrukturen/ag-chemie-und-gesellschaft/projekte-und-veranstaltungen/cartoons/haber-bosch-verfahren-brot-aus-luft.html}{Haber-Bosch-Verfahren – Brot aus Luft}}:

\begin{equation*}
  N_2 + 3H_2 \rightarrow 2NH_3
\end{equation*}  

Ein Mol eines beliebigen Gases hat bei normalen Umweltbedingungen ein
Volumen von 22.4 Liter. % Ein Mol enthält $6 \cdot 10^{23}$ Teilchen.

\begin{enumerate}
  \setcounter{enumi}{3}
\item Welche Masse an Ammoniak in Kilogramm $kg$ können Sie aus einem
  Kubikmeter $m^3$ Luft unter normalen Umweltbedingungen gewinnen?
  \textbf{(2 Punkte)}
\item Wieviel Ammoniak in $mol$ erhalten Sie aus einem Kubikmeter Luft?
  \textbf{(1 Punkt)}
\end{enumerate}

 
\clearpage
% -----------------------------------------------------------------------

\section{Aufgabe \hfill (10 Punkte)}

\textit{Geben Sie grunds{\"a}tzlich Formeln und Rechenweg zur L{\"o}sung der
  Teilaufgaben mit an!} \\[1Ex]

%% --------------------------------------------------------------------
\hfill\href{https://youtu.be/Bbu6n8MXxQk}{\includegraphics[width =
  2cm]{img/youtube}} %%youtube
\hspace{2Ex}
%% --------------------------------------------------------------------

\paragraph{Armee der Finsternis}



Ihr Studentenjob war nach Ladenschluss bei Kaufland die Regale
einzur{\"a}umen. Dabei ist Ihnen in der Auslage der Sonderangebote das
Necronomicon\footnote{Ein wirklich gef{\"a}hrliches Buch ist:
  \textit{Du bist genug: Vom Mut, gl{\"u}cklich zu sein} von Fumitake Koga und
  Ichiro Kishimi} in die H{\"a}nde gefallen. Nun sind Sie ein Magier der
Zeichen geworden! Also eigentlich k{\"o}nnen Sie nur Mathe und das d{\"a}mliche
Necronomicon hat Sie in die Vergangenheit geschleudert... aber gut, was tut
man nicht alles im Jahr 1041 n. Chr. f{\"u}r den neuen Lehnsherren
Henry dem Roten. Sie bauen nat{\"u}rlich einen Schrottkugelturm um sich den
Horden der Finsternis mit genug Schrott erwehren zu k{\"o}nnen! Ihnen stehen
zwei m{\"a}chtige magische Formeln zur Unterst{\"u}tzung zu Verf{\"u}gung.

\begin{center}
  \begin{tabular}{cc}
    $E_{kin} = \cfrac{1}{2}\cdot m \cdot v^2$ & $E_{pot} = m \cdot g \cdot h$\\
  \end{tabular}
\end{center}

mit

\begin{itemize}
\item $m$, gleich der Masse [kg] des Objekts
\item $h$, gleich der H{\"o}he [m] des ruhenden Objekts
\item $v$, gleich der Geschwindigkeit [m/s] des Objekts
\item $g$, gleich der Erdbeschleunigung mit $9.81 \tfrac{m}{s^2}$ 
\end{itemize}

Als erstes m{\"u}ssen Sie die H{\"o}he des zu bauenden Schrottkugelturmes
bestimmen. Hierf{\"u}r ist wichtig zu wissen, dass sich die
Blei\textit{tropfen} mit einem Gewicht von $10mg$ zu
gleichf{\"o}rmigen Blei\textit{kugeln} bei einer Geschwindigkeit von
$11m/s$ bilden.

\begin{enumerate}
\item Wie hoch m{\"u}ssen Sie den Schrottkugelturm bauen lassen, damit sich
  runde Bleikugeln durch die Fallgeschwindigkeit von $11m/s$ bilden?
  \textbf{(3 Punkte)}
\end{enumerate}

Ihre erstellten Schrottkugeln sind leider zu gro{\ss} und somit sind zu wenige
Schrottkugeln in einer Ladung. Damit k{\"o}nnen Sie die Armee der Finsternis
nicht aufhalten. Die Sachlage m{\"u}ssen Sie einmal mathematisch untersuchen.

\begin{enumerate}
  \setcounter{enumi}{1}
\item Nennen Sie die beiden geometrischen Formen aus denen sich
  n{\"a}herungsweise ein Tropfen zusammensetzt! Erstellen Sie eine beschriftete 
  Skizze des Tropfens!
  \textbf{(2 Punkte)}
\item Sie messen eine L{\"a}nge des Tropfens von 2.8mm. Die L{\"o}cher
  im Sieb erlauben ein Tropfendurchmesser von 1.7mm. Welchen
  Durchmesser in mm haben Ihre produzierten Blei\textit{kugeln}?  \textbf{(3 Punkte)}
\end{enumerate}

Sie haben jetzt die \ensuremath{2.3\times 10^{5}} Bleikugeln zusammen. Blei hat eine Dichte
von $11.23g/cm^3$.

\begin{enumerate}
  \setcounter{enumi}{3}
\item Wie schwer in Kilogramm $kg$ sind die \ensuremath{2.3\times 10^{5}} produzierten
  Bleikugeln, die Sie jetzt auf die Burgmauer transportieren m{\"u}ssen?
  \textbf{(1 Punkt)}
\end{enumerate}

Am Ende m{\"u}ssen Sie noch die Produktion von dem Bleischrott im Turm optimieren.

\begin{enumerate}
  \setcounter{enumi}{4}
\item Wie gro{\ss} in $cm^2$ ist Ihr quadratisches Sieb am oberen Ende des Turms,
  wenn Sie pro Fall ca. 1200 Bleikugeln produzieren wollen und
  die Bleikugel im Fall 1.2cm Abstand haben m{\"u}ssen?  \textbf{(1
    Punkt)}
\end{enumerate}
 
\clearpage
% -----------------------------------------------------------------------

\section{Aufgabe \hfill (10 Punkte)}

\textit{Geben Sie grunds{\"a}tzlich Formeln und Rechenweg zur L{\"o}sung der
  Teilaufgaben mit an!} \\[1Ex]

%% --------------------------------------------------------------------
\hfill\href{https://youtu.be/Mr6eslls4J0}{\includegraphics[width =
  2cm]{img/youtube}} %%youtube
\hspace{2Ex}
%% --------------------------------------------------------------------

\paragraph{Armee der Kaninchen}



Leider hat es mit Ihrer Faultierpension in Down Under nicht geklappt. War vielleicht
auch nicht \textit{so} die beste Idee... aber daf{\"u}r haben Sie eine Neue!
Oder wie es Mike Tyson zugeschrieben wird: "`Ich wurde nie
niedergeschlagen, ich war immer am Aufstehen!"'. Daher machen Sie jetzt
einen Gro{\ss}handel mit Kaninchenfleisch und damit dem teuersten Fleisch in
Australien auf. Moment, hopsen hier nicht, seit Thomas Austin im Jahr
1875 ungef{\"a}hr 32 Kaninchen entlassen hat,
Millionen von Kaninchen rum? Wieso ist das Kaninchenfleisch dann so
exklusiv? Dem wollen wir mal mathematisch nachgehen!\footnote{Die Quelle
  der Inspiration f{\"u}r die Aufgabe war der folgendes YouTube Video:
  \href{https://youtu.be/38fuOr3tdgc?si=Li7NL_FoByML8JtT}{ Incredible
    Stories -- Why don't they eat wild rabbits in Australia? They have
    millions of them! The reason is surprising...}}
\\

Forscherinnen fand folgende S{\"a}ttigungsfunktion f{\"u}r das j{\"a}hrliche Wachstum
der gesamten Kaninchenpopulation im westlichen Australien.
\begin{equation*}
  f(t) = \ensuremath{1.1\times 10^{10}} - \ensuremath{9\times 10^{8}} \cdot 2.2^{-0.3 \cdot t + 2.7}
\end{equation*}

\begin{enumerate}
\item Skizzieren Sie die S{\"a}ttigungsfunktion \textit{ann{\"a}herungsweise} in einer Abbildung! \textbf{(1
    Punkt)}
\item Wie viele Kaninchen k{\"o}nnen nach der S{\"a}ttigungsfunktion maximal im
  westlichen Australien leben? Erg{\"a}nzen Sie den Wert in Ihrer Abbildung! \textbf{(2 Punkte)}
\item Wie viele Millionen Kaninchen leben nach der S{\"a}ttigungsfunktion
  nach 9 Jahren auf dem australischen Kontinent? \textbf{(1
    Punkt)}
\end{enumerate}

Um den Kaninchen Einhalt zu gebieten wurde das Myxoma Virus und das Rabbit
Haemorrhagic Disease Virus (RHDV) in 12 Kaninchen
ausgebracht. Da die Kaninchen keine Ma{\ss}nahmen gegen die Ausbreitung
vornehmen k{\"o}nnen, verl{\"a}uft die Ausbreitung mit einem w{\"o}chentlichen
Wachstumsfakor von 1.6 nach folgender Formel.

\begin{equation*}
  N(t) = N(0) \cdot a^t
\end{equation*}

\begin{enumerate}
  \setcounter{enumi}{2}
\item Wie viele Wochen ben{\"o}tigen die Viren um theoretisch die gesamte
  Kaninchenpopulation nach 7 Jahren Wachstum zu
  durchseuchen? \textbf{(1 Punkt)}
\end{enumerate}

Das Myxoma Virus und das RHDV t{\"o}ten 99.9\% der
Kaninchenpopulation innerhalb weniger Wochen.

\begin{enumerate}
  \setcounter{enumi}{3}  
\item Wie lange in Jahren dauert es bis eine Kaninchenpopulation nach einer
  Viruspandemie wieder auf 70\% der ges{\"a}ttigten
  Kaninchenpopulation angewachsen ist?  \textbf{(2 Punkte)}
\end{enumerate}

Thomas Austin entlie{\ss} die Kaninchen im {\"a}u{\ss}ersten Osten von
Australien. Australien hat eine West-Ost-Ausdehnung von 4100km
und eine Nord-S{\"u}d-Ausdehnung von knapp 3600km. Die Kaninchen
breiten sich radial mit einer Geschwindigkeit von 9.8km pro
Jahr aus.

\begin{enumerate}
  \setcounter{enumi}{4}
\item Wie lange dauert es in Jahren bis die Kaninchen jeden Ort
  in Australien erreicht haben? \textit{L{\"o}sen Sie die Aufgabe unter der
    Verwendung einer schematischen Skizze!} \textbf{(2 Punkte)}
\end{enumerate}

Eine j{\"a}hrliche Impfung gegen das Myxoma Virus und das Rabbit Haemorrhagic
Disease Virus (RHDV) kosten 9\$ pro Tier und der
durchf{\"u}hrende Arzt verlangt ca. 40\$ pro Tier.

\begin{enumerate}
  \setcounter{enumi}{5}
\item In Ihrem Stall leben 800 Mastkaninchen. Mit welchen
  j{\"a}hrlichen Zusatzkosten f{\"u}r die Impfungen der Kaninchen m{\"u}ssen Sie daher
  kalkulieren? \textbf{(1 Punkt)}
\end{enumerate}
 
\clearpage
% -----------------------------------------------------------------------

\section{Aufgabe \hfill (12 Punkte)}

\textit{Geben Sie grunds{\"a}tzlich Formeln und Rechenweg zur L{\"o}sung der
  Teilaufgaben mit an!} \\[1Ex]

%% --------------------------------------------------------------------
\hfill\href{https://youtu.be/fiWGgCX-cE4}{\includegraphics[width =
  2cm]{img/youtube}} %%youtube
\hspace{2Ex}
%% --------------------------------------------------------------------



\paragraph{Ostfriesland. Unendliche Weiten.}



Wir schreiben das Jahr 2024. Dies sind die Abenteuer
der Kuh Frida und Ihnen. Gr{\"u}nes Gras unter Ihren F{\"u}{\ss}en und
ein strammer Wind im Gesicht, egal wohin Sie schauen. Ein schmatzendes
Ger{\"a}usch ert{\"o}nt unter Ihnen. Sie sinnieren, sollten Sie Ihre weiten
Graslandschaften jetzt schon d{\"u}ngen?  Daf{\"u}r ben{\"o}tigen Sie die
\textit{Gr{\"u}nlandtemperatur}! Die Gr{\"u}nlandtemperatur (GLT) ist die Summe aller
positiven Tagesmitteltemperaturen seit Jahresbeginn. Ab einer GLT von
200$^\circ$ kann mit der Stickstoffd{\"u}ngung begonnen werden. Sie sehen nicht
ein, Geld f{\"u}r einen Agrarmetrologen zu bezahlen. Also rechnen Sie mit
folgenden Informationen zu Monatsmultiplikatoren des GLT-Wertes: Januar mit
$0.6\times$, Februar mit $0.7\times$ und M{\"a}rz mit
$1.1\times$. Sie haben noch im letzten Jahr folgende Temperaturen
gemessen.

\begin{center}
\begin{tabular}{cc}
  \toprule
  Datum & C$^\circ$ \\
  \midrule
  01. Jan 2023 & 0.1\\
  01. Feb 2023 & 1.5\\
  01. Mrz 2023 & 3.1\\
  01. Apr 2023 & 4.3\\
  \bottomrule
\end{tabular}
\end{center}

\begin{enumerate}
\item Erstellen Sie eine Skizze aus den Informationen aus der
  Temperaturtabelle!  \textbf{(1 Punkt)}
\item Stellen Sie die linearen Funktionen $f_1(t)$, $f_2(t)$ und
  $f_3(t)$ aus der obigen Temperaturtabelle auf!  \textbf{(1 Punkt)}
\item Bestimmen Sie die Stammfunktionen $F_1(t)$, $F_2(t)$ und $F_3(t)$ f{\"u}r
  Ihre linearen Funktionen aus der obigen Temperaturtabelle!  \textbf{(1
    Punkt)}
\item Osterglocken beginnen ab einer GLT von 190$^\circ C$ zu bl{\"u}hen. An
  welchem Tag im 1. Quartal des Jahres 2023 war dies der
  Fall? \textit{Ignorieren Sie ein eventuelles Schaltjahr in Ihrer Berechnung.} \textbf{(4 Punkte)}
\end{enumerate}

Auf dem Weg zu Ihrer Jonagoldplantage wurden Sie mit Ihrem Trecker von
einer Gruppe elektrifizierter Renter abgedr{\"a}ngt. Der Trecker muss wieder
aus dem Graben! Frida und die elektrifizierten Rentner ziehen an zwei,
separaten Seilen. Dabei zieht Frida mit $230N$. Die
elektrifizierter Renter  bringen eine Kraft von $190N$ auf.\\

\textit{L{\"o}sen Sie diese Aufgabe mit Hilfe einer aussagekr{\"a}ftigen Skizze der
  Kraftvektoren. Bezeichnen Sie die Kraftvektoren und die Winkel Ihrer
  Skizze entsprechend!}

\begin{enumerate}
  \setcounter{enumi}{4}  
\item Im ersten Versuch legen Sie das Seil f{\"u}r Frida lotrecht {\"u}ber einen
  Ast oberhalb des Treckers. Die Rentner ziehen in einer geraden Linie {\"u}ber
  die B{\"o}schung hinweg am anderen Seil. Welche Kraft wird aufgebracht?
  \textbf{(2 Punkte)}
\item Im zweiten Versuch ziehen Frida und die Rentner mit einem
  $30^\circ$ Winkel mit ihrem Seil an dem Trecker. Welche Kraft
  wird aufgebracht? \textbf{(2 Punkte)}
\item Mit welcher Beschleunigung ziehen Sie den $1.2t$ schweren
  Trecker \textit{jeweils} aus dem Graben, wenn $F = m \cdot a$ gilt?
  \textbf{(1 Punkt)}
\end{enumerate}

 
\clearpage
% -----------------------------------------------------------------------

\section{Aufgabe \hfill (10 Punkte)}

\textit{Geben Sie grunds{\"a}tzlich Formeln und Rechenweg zur L{\"o}sung der
  Teilaufgaben mit an!} \\[1Ex]

%% --------------------------------------------------------------------
\hfill\href{https://youtu.be/RuzMjwvwT-4}{\includegraphics[width =
  2cm]{img/youtube}} %%youtube
\hspace{2Ex}
%% --------------------------------------------------------------------

\paragraph{In der Kartonagenfabrik}



Wenn man sich zu sp{\"a}t anmeldet, dann ist die Exkursion nicht so toll. Also
geht es mit Rektor Skinner und Mrs. Krabappel in die Kartonagenfabrik. Wie
schon im vorherigen Semester... In der Kartonagenfabrik angekommen erfahren
Sie, dass die Kartons zum Versand von N{\"a}geln nicht hier zusammengebaut
werden sondern das sich die Endfertigung in Flint, Michigan befindet. Unter
anderem wird dort der ber{\"u}hmte \textit{Doppelt gewellte,
  5-mal-gefaltete, 0.7mm, 30-cm-Karton} durch
Falzung hergestellt. Beim letzten Mal war Rektor Skinner die Stimmung zu
schlecht und deshalb geht es erst nach Hause, wenn ein paar Aufgaben gel{\"o}st
sind. Martin gef{\"a}llt das. An dem Vorrat
an Zigaretten von Mrs. Krabappel meinen Sie wenig Zuversicht zu erkennen.\\

Jetzt hei{\ss}t es Kartons optimieren. Der nun zu optimierende, flache Karton
hat eine L{\"a}nge von 30cm und eine Breite von 18cm. Die
Kartonagenmaschine in Flint soll dann einen quadratischen Eckenausschnitt
der L{\"a}nge $x$ falzen.

\begin{enumerate}
\item Erstellen Sie eine Skizze des Karton\textit{blatt}rohlings!
  Beschriften Sie die Skizze mit den entsprechenden L{\"a}ngenangaben
  \textbf{(1 Punkt)}
\item Berechnen Sie die Falztiefe $x$ f{\"u}r ein maximales Volumen des flachen
  Kartons! \textbf{(3 Punkte)}
\item Welches Volumen in Liter ergibt sich mit der von Ihnen berechneten
  Falztiefe $x$?  \textbf{(1 Punkt)}
\item Sie wollen noch einen b{\"u}ndig mit dem Boden abschlie{\ss}enden Deckel f{\"u}r
  den Karton stanzen lassen. Wie gro{\ss} ist die Fl{\"a}che des
  Kartondeckel\textit{blatt}rohlings in $cm^2$? \textbf{(2 Punkte)}
\end{enumerate}

Rektor Skinner m{\"o}chte sich gerne wieder in seinem Vorgarten aufhalten und
nicht die ganze Zeit von Bart mit Erdn{\"u}ssen beworfen werden. Deshalb m{\"o}chte
er einen ger{\"a}umigen Teil seines Vorgartens einz{\"a}unen. Ein Teil der
Umz{\"a}unung bildet seine Vorderhauswand. Wegen Lieferschwierigkeiten stehen
Rektor Skinner nur 100m Zaun zu Verf{\"u}gung. Sie wollen nun die
maximale Fl{\"a}che des abgeschirmten Vorgartens in Abh{\"a}ngigkeit der
Seitenl{\"a}ngen bei der Verwendung von 100m Zaun bestimmen!

\begin{enumerate}
  \setcounter{enumi}{4}  
\item  Welche Seitenl{\"a}ngen f{\"u}r den Zaun ergeben sich f{\"u}r die
  maximale Fl{\"a}che des abgeschirmten Vorgartens? \textbf{(2 Punkte)}
\item Berechnen Sie die Fl{\"a}che des abgeschirmten Vorgartens! \textbf{(1
    Punkt)}
\end{enumerate}

 
\clearpage
% -----------------------------------------------------------------------

\section{Aufgabe \hfill (10 Punkte)}

\textit{Geben Sie grunds{\"a}tzlich Formeln und Rechenweg zur L{\"o}sung der
  Teilaufgaben mit an!} \\[1Ex]

%% --------------------------------------------------------------------
\hfill\href{https://youtu.be/OhyuH6hzEhY}{\includegraphics[width = 2cm]{img/youtube}} %%youtube
\hspace{2Ex}
%% --------------------------------------------------------------------




\paragraph{Ein Pfund Insekten, bitte!} Das wird wohl h{\"a}ufiger geh{\"o}rt
werden, wenn wir die Menschheit mit Proteinen ausreichend ern{\"a}hren
wollen\footnote{Die Quelle der Inspiration f{\"u}r die Aufgabe war der folgende
  Artikel aus dem Spiegel:
  \href{https://www.spiegel.de/ausland/ueberbevoelkerung-acht-milliarden-sind-wir-bald-zu-viele-menschen-auf-der-erde-a-3f20c7bc-3d60-4440-9f52-eb338db207f5}{Acht
    Milliarden - sind wir bald zu viele Menschen auf der Erde?}}. Schauen
wir uns dazu einmal den Vergleich Deutschland zu Nigeria an. Nigeria hat
eine der am schnellsten wachsenden Bev{\"o}lkerungen der Welt und wird
vermutlich im Jahr 2100 zu den Top 5 der bev{\"o}lkerungsreichsten L{\"a}nder
z{\"a}hlen. Im Jahr 2021 leben ca. \ensuremath{8.4\times 10^{7}} Menschen in
Deutschland und ca. \ensuremath{1.79\times 10^{8}} Menschen in Nigeria. Mit den
Informationen wollen wir anfangen und dann eine Prognose f{\"u}r den
Fleischkonsum im Jahr 2050 zu treffen. \\ 


Im folgenden ist Abbildung des Fleischkonsums im Jahr 2021 in
Deutschland und Nigeria in [kg] einmal dargestellt. 

\begin{knitrout}
\definecolor{shadecolor}{rgb}{0.969, 0.969, 0.969}\color{fgcolor}

{\centering \includegraphics[width=\maxwidth]{img/math-14-a-1} 

}


\end{knitrout}
  
\begin{enumerate}
\item Stellen Sie den Fleischkonsum in Deutschland und Nigeria im Jahr
  2021 \textit{pro Kopf} in einer aussagekr{\"a}ftigen Tabelle dar! 
  \textbf{(2 Punkte)}
\item Erg{\"a}nzen Sie in der Tabelle eine Spalte in der Sie f{\"u}r den
  Fleischkonsum in Nigeria auf Deutschland normieren, daher ins Verh{\"a}ltnis
  Nigeria/Deutschland, setzen! \textbf{(1 Punkt)}
\end{enumerate}

In der n{\"a}chsten Abbildung finden Sie die CO$_2$-Emission in [kg] nach
Lebensmittel, die durch die Produktion entsteht, abgebildet.

\begin{knitrout}
\definecolor{shadecolor}{rgb}{0.969, 0.969, 0.969}\color{fgcolor}

{\centering \includegraphics[width=\maxwidth]{img/math-14-b-1} 

}


\end{knitrout}

\begin{enumerate}
  \setcounter{enumi}{2}  
\item Stellen Sie in einer Tabelle die Treibhausgasemissionen an $CO_2$ pro
  Kopf, die durch den Fleischkonsum in Deutschland und Nigeria im Jahr
  2021 entstehen, dar! Erg{\"a}nzen Sie auch hier das Verh{\"a}ltnis
  Nigeria zu Deutschland! \textbf{(2 Punkte)}
\end{enumerate}

\newpage

In der folgenden Abbildung sehen Sie die
Bev{\"o}lkerungsentwicklung [Millionen] in Nigeria von 1950 bis ins Jahr 2030 fortgef{\"u}hrt.

\begin{knitrout}
\definecolor{shadecolor}{rgb}{0.969, 0.969, 0.969}\color{fgcolor}

{\centering \includegraphics[width=\maxwidth]{img/math-14-c-1} 

}


\end{knitrout}

\begin{enumerate}
  \setcounter{enumi}{3}  
\item Sch{\"a}tzen Sie graphisch die zu erwartende Bev{\"o}lkerung [Millionen] in Nigeria im
  Jahr 2050, die sich anhand der Informationen aus der Abbildung ergibt!
\begin{enumerate}
\item Ohne Ber{\"u}cksichtigung der Covid-19-Pandemie! \textbf{(1
    Punkt)}
\item Unter Ber{\"u}cksichtigung der Covid-19-Pandemie! \textbf{(1
    Punkt)}
\end{enumerate}
\item Berechnen Sie den gesch{\"a}tzten Fleischkonsum von Nigeria im Jahr
  2050 unter der Annahme 70\%-iger Angleichung der Lebensbedingungen zu
  Deutschland im Jahr 2021! \textbf{(1 Punkt)}
\item Berechnen Sie die prozentuale Steigerung der Treibhausgasemissionen
  an $CO_2$ in Nigeria im Jahr 2050 im Vergleich zum Jahr 2021, der
  sich durch den angeglichenen Fleischkonsum ergibt! \textbf{(1 Punkt)}
\item Berechnen Sie die prozentuale Steigerung der Treibhausgasemissionen
  an $CO_2$ in Nigeria, wenn die gesamte Proteinaufnahme 
  durch Insekten ersetzt w{\"u}rde! \textbf{(1
    Punkt)}
\end{enumerate}







 
\clearpage
% -----------------------------------------------------------------------

\section{Aufgabe \hfill (12 Punkte)}

\textit{Geben Sie grunds{\"a}tzlich Formeln und Rechenweg zur L{\"o}sung der
  Teilaufgaben mit an!} \\[1Ex]

%% --------------------------------------------------------------------
\hfill\href{https://youtu.be/flRBo1FWQy0}{\includegraphics[width = 2cm]{img/youtube}} %%youtube
\hspace{2Ex}
%% --------------------------------------------------------------------

\paragraph{T{\"o}dliche Seuche AIDS -- Die r{\"a}tselhafte Krankheit}




Irritiert legen Sie die historische Ausgabe des Spiegels aus den 80zigern
beiseite. Sie sind bei Ihrem Augenarzt und wollen einen AIDS-Test machen
lassen. Woanders leider keinen Termin gekriegt... Immerhin denken Sie und
Ihre Partnerin {\"u}ber Nachwuchs nach und da geht es eben nur durch
ungesch{\"u}tzten Sex. Was wissen Sie nun aber {\"u}ber AIDS und dem diagnostischen
AIDS-Test, den Sie nun machen werden?\\

Die Pr{\"a}valenz von AIDS bei einem Menschen in Europa wird mit
0.5\% angenommen. In 95\% der F{\"a}lle ist ein
HIV-Test positiv, wenn der Patient erkrankt ist. In 0.5\%
der F{\"a}lle ist ein HIV-Test positiv, wenn der Patient \textit{nicht}
erkrankt ist und somit gesund ist. Sie stutzen. Wie wahrscheinlich ist es
denn eigentlich an AIDS erkrankt zu sein $(K^+)$, wenn Sie einen positiven
AIDS-Test vorliegen haben $(T^+)$? Gehen Sie f{\"u}r die folgenden Berechnungen
von $n = \ensuremath{10^{4}}$ Patienten mit einem diagnostischen Test f{\"u}r AIDS
aus. Sie nehmen sich also einen Kuli und fangen an auf der historischen
Ausgabe des Spiegels zu rechnen\footnote{Die Quelle der Inspiration f{\"u}r die
  Aufgabe war der folgende wissenschaftlicher Artikel: \textit{Binder et
    al. (2022) Von Baumdiagrammen {\"u}ber Doppelb{\"a}ume zu H{\"a}ufigkeitsnetzen --
    kognitive {\"U}berlastung oder didaktische Unterst{\"u}tzung? Journal f{\"u}r
    Mathematik-Didaktik, 1-33}}.

\begin{enumerate}
\item Welche Wahrscheinlichkeit $Pr$ wollen Sie berechnen? \textbf{(1 Punkt)}
\item Zeichnen Sie einen H{\"a}ufigkeitsdoppelbaum zur Bestimmung der gesuchten
  Wahrscheinlichkeit $Pr$! \textbf{(2 Punkte)} 
\item Beschriften Sie den H{\"a}ufigkeitsdoppelbaum, mit denen Ihnen bekannten
  Informationen zu der AIDS Erkrankung und dem AIDS-Test! \textbf{(1 Punkt)}
\item F{\"u}llen Sie den H{\"a}ufigkeitsdoppelbaum mit den sich ergebenden,
  absoluten Patientenzahlen $n$ aus! \textbf{(2 Punkte)}
\item Berechnen Sie die gesuchte Wahrscheinlichkeit $Pr$! \textbf{(1 Punkt)}
\end{enumerate}

Bei dem folgenden Arztgespr{\"a}ch erfahren Sie, dass beim diagnostischen
Testen \textit{True Positives (TP)}, \textit{True Negatives (TN)},
\textit{False Positives (FP)} und \textit{False Negatives (FN)}
auftreten. Das verstehen Sie so noch nicht und deshalb stellen Sie f{\"u}r sich
den Zusammenhang in einer 2x2 Kreuztabelle dar.

\begin{enumerate}
  \setcounter{enumi}{5}
\item Tragen Sie \textit{TP}, \textit{TN}, \textit{FP} und \textit{FN} in
  eine 2x2 Kreuztablle ein. Beschriften Sie die Tabelle entsprechend!
  \textbf{(1 Punkt)}
\item Berechnen Sie die Sensitivit{\"a}t und Spezifit{\"a}t des diagnostischen Tests
  f{\"u}r AIDS! F{\"u}llen Sie daf{\"u}r die 2x2 Kreuztabelle mit den Informationen aus
  dem H{\"a}ufigkeitsdoppelbaum aus! \textbf{(2 Punkte)}
\item Was beschreibt die Sensitivit{\"a}t und die Spezifit{\"a}t im Bezug auf die
  Gesunden und Kranken? Stellen Sie beide diagnostische Ma{\ss}zahlen als
  Wahrscheinlichkeiten $Pr$ dar! \textbf{(2 Punkte)} 
\end{enumerate}



 
\clearpage
% -----------------------------------------------------------------------

\section{Aufgabe \hfill (12 Punkte)}

\textit{Geben Sie grunds{\"a}tzlich Formeln und Rechenweg zur L{\"o}sung der
  Teilaufgaben mit an!} \\[1Ex]

%% --------------------------------------------------------------------
\hfill\href{https://youtu.be/SZqp_vy3rgI}{\includegraphics[width = 2cm]{img/youtube}} %%youtube
\hspace{2Ex}
%% --------------------------------------------------------------------

\paragraph{Network-Marketing oder Schneeballschlacht!}




Eine M{\"o}glichkeit, leicht Geld zu verdienen, ist es anderen Menschen f{\"u}r Geld
zu versprechen, wie man leicht reich werden kann. Am besten nat{\"u}rlich ohne
viel Aufwand und ortsunabh{\"a}ngig. Dann wollen wir mal loslegen. Schnell
ein YouTube-Werbevideo gedreht und auf geht es mit unserem
Network-Marketing. Aber Moment, wie funktioniert Network-Marketing
eigentlich und was hat das alles mit einer Schneeballschlacht zu tun? Wir
wollen hier einmal in die Untiefen des "`passiven Einkommens"'
abtauchen\footnote{Die Quellen der Inspiration f{\"u}r die Aufgabe waren
  folgendes YouTube Video:
  \href{https://youtu.be/UOKkZF_qK9M?si=uf4foJVFKfeQMwSw}{Simplicissimus --
    Die meistgesuchte Betr{\"u}gerin der Welt} und der Artikel:
  \href{https://www.deutschlandfunkkultur.de/netzwerk-marketing-die-illusion-schnell-reich-zu-werden-100.html}{
    Deutschlandfunk Kultur -- Die Illusion, schnell reich zu werden}}.\\

Das Jahr 2022 war das erfolgreichste Jahr in der Geschichte von
Up/Down Systems and Networking (UDSysNet). Das Unternehmen steigerte den Umsatz um rund
27 Prozent von 275 Millionen Euro im Jahr
2021. Doch wie viel kommt bei den Partnern an? Laut
UDSysNet habe das Unternehmen \ensuremath{3.6\times 10^{5}} aktive Partner.

\begin{enumerate}
\item Berechnen Sie zuerst den Umsatz der Firma UDSysNet im
  Jahr 2022! \textbf{(1 Punkt)}
\item Wie viel von dem Umsatz im Jahr 2022 wird im Durchschnitt von jedem
  aktiven Partner erwirtschaftet? \textbf{(1 Punkt)}
\item Welche \textit{monatlicher} Umsatz ergibt sich dadurch im
  Durchschnitt f{\"u}r jeden aktiven Partner bei einer direkten Provision von
  20\%? \textbf{(1 Punkt)}
\end{enumerate}

Ihr zu vermarkendes Produkt, hinter dem Sie voll stehen, kostet
75EUR pro Einheit im Direktverkauf. Die direkte Provision
f{\"u}r die erste Stufe betr{\"a}gt 30\%. F{\"u}r die zweite, dritte und
vierte Stufe betragen die indirekten Provisionen jeweils 3.25\%,
2.25\% und 1.75\%. Jeder Ihrer angeworbenen "`Partner"'
wirbt wiederum drei Partner f{\"u}r sich selbst an. Pro Monat
werden im Schnitt zwei Einheiten vom Produkt verkauft. Sie wollen nun
2100EUR im Monat \textit{passiv} -- also durch indirekte
Provisionen -- erwirtschaften.

\begin{enumerate}
  \setcounter{enumi}{3}
\item Erg{\"a}nzen Sie die folgende Tabelle mit den obigen Informationen! \textbf{(2 Punkte)}
\end{enumerate}

\begin{center}
\begin{tabular}{c|c|c|c}
  \toprule
  \textbf{Stufe} & \textbf{Anzahl Partner}  & \textbf{Umsatz/Stufe} & \textbf{Provision}\\
  \midrule
  1 & Sie selber  &  & \\ \midrule
  2 &   &  &  \\ \midrule
  3 &   &  &  \\ \midrule
  4 &   &  &  \\
  \bottomrule
\end{tabular}
\end{center}

\begin{enumerate}
  \setcounter{enumi}{4}
\item Wie viele Partner m{\"u}ssen Sie auf der 2 Stufe anwerben um Ihr passives
  Einkommen durch indirekte Provision zu erreichen? Wie viele Menschen
  arbeiten am Ende indirekt f{\"u}r Sie? Stellen Sie den
  Zusammenhang graphisch dar!  \textbf{(3 Punkte)}
\end{enumerate}

Sie mussten zum Einstieg bei UDSysNet Einheiten des Produkts
f{\"u}r 2250EUR kaufen. Diese Einheiten k{\"o}nnen Sie nur direkt
verkaufen. Leider mussten Sie den Kauf {\"u}ber einen Kredit {\"u}ber
5\% p.a. {\"u}ber 72 Monate finanzieren.

\begin{enumerate}
  \setcounter{enumi}{5}
\item Berechnen Sie die Gesamtsumme, die Sie als Kredit abbezahlen m{\"u}ssen! \textbf{(2 Punkte)}
\item Wie viele Einheiten m{\"u}ssen Sie pro Monat verkaufen um die anfallenden Zinsen
  durch die direkte Provision zu erwirtschaften? \textbf{(1 Punkt)}
\item Wie lange in
  Monaten ben{\"o}tigen Sie um den Kredit durch die direkte
  Provision abzubezahlen? \textbf{(1 Punkt)}
\end{enumerate} 
\clearpage
% -----------------------------------------------------------------------

\section{Aufgabe \hfill (10 Punkte)}

\textit{Geben Sie grunds{\"a}tzlich Formeln und Rechenweg zur L{\"o}sung der
  Teilaufgaben mit an!} \\[1Ex]

%% --------------------------------------------------------------------
\hfill\href{https://youtu.be/8Pb2sKUIMyk}{\includegraphics[width =
  2cm]{img/youtube}} %%youtube
\hspace{2Ex}
%% --------------------------------------------------------------------

\paragraph{H{\"o}hlen \& Drachen}



Nachdem Sie sich begeistert in der Serie \textit{Stranger Thinks} verloren
haben, wollen Sie bei einer Ihrer Freundinnen einmal \textit{H{\"o}hlen \& Drachen}
ausprobieren. Um Geld zu sparen, das Zeug kostet echt, wurde etwas an den
Regeln gebastelt. Schnell stellen Sie fest, dass hier ganz sch{\"o}n viele
unterschiedliche W{\"u}rfel durch die Gegend fliegen. Daher m{\"u}ssen Sie sich
jetzt einiges an Fragen stellen. \\%[-1ex]

In dem Spiel haben Sie nun auf einmal 4 sechseitige W{"u}rfel (4d6) zum w{\"u}rfeln in der Hand. Wenn Sie eine 6 w{\"u}rfeln,
haben Sie einen Erfolg.

\begin{enumerate}
\item Berechnen Sie die Wahrscheinlichkeit \textit{genau}
  3 Erfolge zu erzielen!  \textbf{(2 Punkte)}
\item Berechnen Sie die Wahrscheinlichkeit keinen Erfolg zu erzielen!
  \textbf{(1 Punkt)}
\end{enumerate}

Sie betrachten nun aufmerksam die ausufernden Ausr{\"u}stungstabellen. Ihnen
wird aber geholfen und Sie m{\"u}ssen sich jetzt nur zwischen der Axt oder dem
Schwert entscheiden.

\begin{enumerate}
  \setcounter{enumi}{2}
\item W{\"u}rden Sie die Axt mit zwei zw{"o}lfseitigen W{"u}rfeln (2d12) als Schaden oder
  das Schwert mit einem vierseitigen W{"u}rfel plus 5 (1d4+5) als Schaden bevorzugen?
  Begr{\"u}nden Sie Ihre Antwort mathematisch! \textbf{(1 Punkt)}
\end{enumerate}

Jetzt wird es immer wilder, da Sie sich jetzt {\"u}berlegen m{\"u}ssen, wie
wahrscheinlich es ist, dass Ihr Rettungswurf gegen den zaubernden Hexer
funktioniert. Sie haben folgende Wahrscheinlichkeiten gegeben. Die
Wahrscheinlichkeit f{\"u}r das Ereignis $A$, der Rettungswurf ist erfolgreich,
ist $Pr(A) = 0.7$, die Wahrscheinlichkeit f{\"u}r das Ereignis $B$,
der Zauberwurf des Hexers ist erfolgreich, ist $Pr(B) = 0.75$. Sie
haben mitgez{\"a}hlt und festgestellt, dass in $40$ von 100 F{\"a}llen
Ihr Rettungswurf bei einem erfolgeichen Zauber funktioniert hat.  

\begin{enumerate}
  \setcounter{enumi}{3}
\item Erstellen Sie eine 2x2 Kreuztabelle mit den Ereignissen $A$ und $B$
  sowie den Gegenereignissen $\bar{A}$ und $\bar{B}$ mit einen
  $\Omega = 100$. Beachten Sie hierbei die entsprechenden
  Wahrscheinlichkeiten f{\"u}r die Ereignisse $A$ und $B$! \textbf{(2 Punkte)}
\item Bestimmen Sie $Pr(A \cap B)$! \textbf{(1 Punkt)}
\item Erstellen Sie ein Baumdiagramm mit den passenden Informationen aus der 2x2
  Kreuztabelle! \textbf{(2 Punkte)}
\item Bestimmen Sie Wahrscheinlichkeit $Pr(A|B)$, dass Ihr Rettungswurf gelingt, wenn
  der Hexer erfolgreich gezaubert hat! \textbf{(1 Punkt)}
\end{enumerate}

  
\clearpage
% -----------------------------------------------------------------------

\section{Aufgabe \hfill (12 Punkte)}

\textit{Geben Sie grunds{\"a}tzlich Formeln und Rechenweg zur L{\"o}sung der
  Teilaufgaben mit an!} \\[1Ex]

%% --------------------------------------------------------------------
\hfill\href{https://youtu.be/9DQKaXdxT_g}{\includegraphics[width =
  2cm]{img/youtube}} %%youtube
\hspace{2Ex}
%% --------------------------------------------------------------------

\paragraph{Retrocheck im TV}



"`Und hier ist sie wieder, die Show der fantastischen Preise. Seien Sie mit
dabei, wenn es wieder hei{\ss}t: Der Preis ist hei{\ss}!"', ert{\"o}nt es und Sie
fragen sich, ob Sie nicht doch lieber bezahlter Gast bei Barbara Salesch
h{\"a}tten sein sollten. Aber Sie brauchen das Geld und jetzt hei{\ss}t es
Spielschows farmen! Erstmal eine Kaffemaschine von Mitropa
gewinnen. Ein Kandidat gewinnt die Kaffeemaschine von Mitropa, wenn nicht alle
Kandidaten {\"u}berbieten (eng. \textit{outbid}). Mit Ihnen bilden
Frida und Thorsten das Team der drei Kandidaten.

\begin{center}
\begin{tabular}{ccc}
  \toprule
  Name & $P(win)$ & $P(outbid)$\\
  \midrule
  Frida & 0.2 & 0.043\\
  Thorsten & 0.1 & 0.08 \\
  \bottomrule
\end{tabular}
\end{center}

\begin{enumerate}
\item Mit welcher Wahrscheinlichkeit gewinnen Sie die Kaffeemaschine von
  Mitropa, wenn keiner der Kandidaten {\"u}berbietet? \textbf{(1 Punkt)}
\item Wenn Ihre {\"U}berbietungswahrscheinlichkeit $P(outbid)$ bei
  0.12 liegt, mit welcher Wahrscheinlichkeit gewinnt
  \textit{keiner} die Kaffeemaschine von Mitropa? \textbf{(1 Punkt)}
\end{enumerate}

Gl{\"u}cksrad f{\"u}r Arme auf der Kirmes! Leider hat es f{\"u}r Maren Gilzer nicht
gereicht. Deshalb sind Sie jetzt auf der Kirmes und spielen mit
einem ein{"a}ugen Piraten um das gro{\ss}e Geld. Das Gl{\"u}cksrad hat 20
Felder. Sie drehen das Gl{\"u}cksrad zweimal. Auf 10 Feldern
gewinnen Sie 5000EUR sonst 1500EUR. Ganz sch{\"o}n viel Geld
und ganz sch{\"o}n zwielichtig hier...

\begin{enumerate}
  \setcounter{enumi}{2}  
\item Skizzieren Sie das Gl{\"u}cksrad und erg{\"a}nzen Sie die
  Wahrscheinlichkeiten! \textbf{(1 Punkt)}
\item Zeichnen Sie das zugeh{\"o}rige Baumdiagramm f{\"u}r das zweimalige Drehen!
  Erg{\"a}nzen Sie die Wahrscheinlichkeiten und die entsprechenden Ereignisse
  \textbf{(2 Punkte)}
\item Mir welcher Wahrscheinlichkeit gewinnen Sie 6500EUR? \textbf{(1
    Punkt)}
\end{enumerate}

Nach Ihrem Fiebertraum reisen Sie im Zug nach K{\"o}ln um bei "`Geh aufs
Ganze!"' mitzuspielen. Sie schaffen es tats{\"a}chlich ins Finale und k{\"o}nnen
als Hauptgewinn ein Auto hinter einer der drei T{\"u}ren gewinnen. 

\begin{enumerate}
  \setcounter{enumi}{5}  
\item Bevor die Show beginnt, wird das Auto hinter eine zuf{\"a}llig bestimmte
  T{\"u}r gestellt. Mit welcher Wahrscheinlichkeit wird jeweils eine der drei
  T{\"u}ren ausgew{\"a}hlt? Zeichnen Sie ein Baumdiagramm! \textbf{(1 Punkt)}
\item Mit welcher Wahrscheinlichkeit w{\"a}hlen Sie sofort die T{\"u}r mit
  dem Auto? Erweitere Sie das Baumdiagramm entsprechend! \textbf{(1
    Punkt)}
\item Der Moderator {\"o}ffnet nun eine der nicht gew{\"a}hlten T{\"u}ren, aber
  nat{\"u}rlich nicht die mit dem Auto. Mit welcher Wahrscheinlichkeit steht
  das Auto hinter der anderen T{\"u}r? Erweitern Sie das Baumdiagramm
  entsprechend! \textbf{(2 Punkte)}
\item L{\"o}sen Sie nun das "`Ziegenproblem"'! Berechne Sie dazu die
  Wahrscheinlichkeiten der einzelnen Pfade. Lohnt sich ein Wechsel der
  anfangs gew{\"a}hlte T{\"u}r? Begr{\"u}nden Sie Ihre Antwort mathematisch!
  \textbf{(2 Punkte)}
\end{enumerate}
 

%https://unterrichten.zum.de/wiki/Laplace-Wahrscheinlichkeit_wiederholen_und_vertiefen/Ziegen

%Buchstabenh{\"a}ufigkeit im Deutschen

%%https://www.mathelike.de/abiturloesungen-mathematik-bayern/mathematik-abitur-bayern-2020/pruefungsteil-a/stochastik-2/teilaufgabe-a.html

%%https://de.wikipedia.org/wiki/Buchstabenh%C3%A4ufigkeit


 
\clearpage
% -----------------------------------------------------------------------
\part{Angewandte Nutztier- und Pflanzenwissenschaften (M.Sc.)}
% -----------------------------------------------------------------------  

\section{Aufgabe \hfill (6 Punkte)}
Vergleichen Sie die Standardabweichung mit dem Standardfehler und grenzen
Sie die beiden Kennzahlen voneinander ab.



\section{Aufgabe \hfill (8 Punkte)}
Ihnen liegt folgendes Varianzanalysemodell mit der {\"u}blichen Beschreibung
zur Auswertung des Merkmals fett- und eiwei{\ss}korrigierte Milchleistung pro
Kuh und Jahr in kg vor:

\begin{equation*}
  Y_{ijkl} = \mu + Var_i + EKA_j + VarEKA_{ij} + V_k + b(L_{ij} - L) + e_{ijkl}
\end{equation*}

mit

\begin{itemize}
\item $Y_{ijkl}$: l-te Beobachtung
\item $\mu$: Populationsmittel 
\item $Var_i$: fixer Effekt der i-ten Variante (i: Kontrolle,
  Versuchsgruppe 1, Versuchsgruppe 2)
\item $EKA_j$: fixer Effekt der j-ten Erstkalbealtergruppe (j: EKA $\leq$ 25
  Monate, EKA > 25 Monate)
\item $VarEKA_{ij}$: fixer Effekt der Interaktion Variante x
  Erstkalbealtergruppe 
\item $V_k$: zuf{\"a}lliger Effekt des Vaters 
\item $b(L_{ij} - L)$: lineare Kovariable Laktationsnummer 
\item $e_{ijkl}$: zuf{\"a}lliger
  Restfehler
\end{itemize}
Erl{\"a}utern Sie anhand dieses Beispiels die Begriffe fixer Effekt,
Interaktion, zuf{\"a}lliger Effekt und Kovariable und grenzen Sie diese
Begriffe voneinander ab.



\section{Aufgabe \hfill (6 Punkte)}
Wie bestimmen Sie die richtige Stichprobengr{\"o}{\ss}e? Welche Kennzahlen /
statistische Ma{\ss}zahlen ben{\"o}tigen Sie dabei und nennen Sie die
Voraussetzungen.

 
\end{document}
% -----------------------------------------------------------------------


  
