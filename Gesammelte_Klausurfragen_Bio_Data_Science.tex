\documentclass[a4paper, 9pt]{scrartcl}\usepackage[]{graphicx}\usepackage[]{xcolor}
% maxwidth is the original width if it is less than linewidth
% otherwise use linewidth (to make sure the graphics do not exceed the margin)
\makeatletter
\def\maxwidth{ %
  \ifdim\Gin@nat@width>\linewidth
    \linewidth
  \else
    \Gin@nat@width
  \fi
}
\makeatother

\definecolor{fgcolor}{rgb}{0.345, 0.345, 0.345}
\newcommand{\hlnum}[1]{\textcolor[rgb]{0.686,0.059,0.569}{#1}}%
\newcommand{\hlstr}[1]{\textcolor[rgb]{0.192,0.494,0.8}{#1}}%
\newcommand{\hlcom}[1]{\textcolor[rgb]{0.678,0.584,0.686}{\textit{#1}}}%
\newcommand{\hlopt}[1]{\textcolor[rgb]{0,0,0}{#1}}%
\newcommand{\hlstd}[1]{\textcolor[rgb]{0.345,0.345,0.345}{#1}}%
\newcommand{\hlkwa}[1]{\textcolor[rgb]{0.161,0.373,0.58}{\textbf{#1}}}%
\newcommand{\hlkwb}[1]{\textcolor[rgb]{0.69,0.353,0.396}{#1}}%
\newcommand{\hlkwc}[1]{\textcolor[rgb]{0.333,0.667,0.333}{#1}}%
\newcommand{\hlkwd}[1]{\textcolor[rgb]{0.737,0.353,0.396}{\textbf{#1}}}%
\let\hlipl\hlkwb

\usepackage{framed}
\makeatletter
\newenvironment{kframe}{%
 \def\at@end@of@kframe{}%
 \ifinner\ifhmode%
  \def\at@end@of@kframe{\end{minipage}}%
  \begin{minipage}{\columnwidth}%
 \fi\fi%
 \def\FrameCommand##1{\hskip\@totalleftmargin \hskip-\fboxsep
 \colorbox{shadecolor}{##1}\hskip-\fboxsep
     % There is no \\@totalrightmargin, so:
     \hskip-\linewidth \hskip-\@totalleftmargin \hskip\columnwidth}%
 \MakeFramed {\advance\hsize-\width
   \@totalleftmargin\z@ \linewidth\hsize
   \@setminipage}}%
 {\par\unskip\endMakeFramed%
 \at@end@of@kframe}
\makeatother

\definecolor{shadecolor}{rgb}{.97, .97, .97}
\definecolor{messagecolor}{rgb}{0, 0, 0}
\definecolor{warningcolor}{rgb}{1, 0, 1}
\definecolor{errorcolor}{rgb}{1, 0, 0}
\newenvironment{knitrout}{}{} % an empty environment to be redefined in TeX

\usepackage{alltt}
\usepackage[ngerman]{babel}

% -----------------------------------------------------------------------

% -----------------------------------------------------------------------
%% ------------------------------------------------------------
%% by J.Kruppa on Friday, February 11, 2022 (11:31)
%% \def\mainDir{\Sexpr{exam_path}}
\def\source{/Users/jokruppa/source/tex}
\usepackage[margin=2cm, includefoot]{geometry}
\setlength{\parindent}{0cm}
\usepackage{booktabs}
\usepackage{amsmath}
\usepackage{scalerel,amssymb}
\usepackage{setspace}
\def\csquare{{\Large $\boxtimes$}}
\def\msquare{{\Large $\square$}}
\usepackage[normalem]{ulem}
\usepackage{array}
\usepackage{xcolor}
\usepackage{float}
\usepackage{currfile}
\usepackage{tikz}
\usepackage[nomessages]{fp}

%% beamer defs
\def\lecture{Klausurfragen der Bio Data Science}

%% exam defs
\def\examtitle{\lecture}
\def\exammodule{
\vspace{-1.75cm}  
\begin{graybox}{}
\vspace{2Ex}
\textbf{\large Name:} \rule[0ex]{16.75em}{.4pt}
\hfill \textnormal{\textit{Nicht bestanden:}} \msquare \\[2.5Ex]
\textbf{\large Vorname:} \rule[0ex]{15em}{.4pt} \\[2.5Ex]
\textbf{\large Matrikelnummer:} \rule[0ex]{10.8em}{.4pt}
\hfill Endnote: \rule[0ex]{7em}{.4pt} 
\end{graybox}
\vspace{3Ex}
\phantom{text}
}
\def\examsemester{Sommersemester \& Wintersemester}
\def\examdate{\today}
%% ------------------------------------------------------------
\definecolor{darkblue}{rgb}{0,0,.5}
\definecolor{darkpurple}{rgb}{0.4117, 0.2, 0.4117}
\definecolor{uni}{rgb}{0,0.3137,0.6078}
\definecolor{gray}{gray}{0.7}

\usepackage{tcolorbox}
\definecolor{logo1}{RGB}{0, 158, 227}
\definecolor{gray5}{RGB}{247, 247, 247}
\definecolor{gray2}{RGB}{102, 102, 102}

\newtcolorbox{graybox}[1]{
  colback=gray5,%%red!5!white,
  colframe=gray2,%%red!75!black,
  fonttitle=\bfseries\Large,
  %%valign=center,
  fontupper=\large,
  before skip=10pt plus 2pt,
  after skip=20pt plus 4pt,
  title=#1}

\newtcolorbox{takehomebox}[1]{
  colback=gray5,%%red!5!white,
  colframe=logo1,%%red!75!black,
  fonttitle=\bfseries\Large,
  %%valign=center,
  fontupper=\large,
  before skip=10pt plus 2pt,
  after skip=10pt plus 2pt,
  title=#1}

\def\Rlogo{\includegraphics[width = 0.5cm]{\string~/Documents/GitHub/exam/img/Rlogo}\;}

\usepackage[scaled=.90]{helvet} 
\usepackage{fancyhdr}
\usepackage{lastpage}
\usepackage{hyperref}
\hypersetup{
    colorlinks=true,       % false: boxed links; true: colored links
    linkcolor=black,          % color of internal links 
    urlcolor=magenta           % color of external links
}
\renewcommand{\familydefault}{\sfdefault}

\title{
\large \exammodule \\[5Ex]
\Huge \examtitle \\[2Ex] 
\Large Hochschule Osnabr{\"u}ck
}
\author{Pr{\"u}fer: Prof. Dr. Jochen Kruppa \\
Fakult{\"a}t f{\"u}r Agrarwissenschaften und Landschaftsarchitektur \\ 
j.kruppa@hs-osnabrueck.de}
\date{Version vom \examdate}

%% ------------------------------------------------------------
%% by J.Kruppa on Tuesday, September 23, 2014 (12:50)
%% Header
\renewcommand{\headrulewidth}{0pt}
\renewcommand{\footrulewidth}{0pt}
\pagestyle{fancy}

\fancyhf{}
\fancyhead[L]{}
\fancyhead[R]{}
\fancyfoot[R]{\thepage}
\fancyfoot[L]{\footnotesize \examtitle}

\fancypagestyle{empty}{
 \fancyhf{}
 \fancyhead[L]{}
 \fancyhead[R]{}
 \fancyfoot[R]{\thepage}
 \fancyfoot[L]{\footnotesize \examtitle}
}

\usepackage{arevtext,arevmath}

\newcommand\Tstrut{\rule{0pt}{2.6ex}}         % = `top' strut
\newcommand\Bstrut{\rule[-0.9ex]{0pt}{0pt}}   % = `bottom' strut
\def\strut{\Tstrut\Bstrut}

% -----------------------------------------------------------------------
\IfFileExists{upquote.sty}{\usepackage{upquote}}{}
\begin{document}
\date{Wintersemester 2024/25 
\vfill
\begin{center}
\includegraphics[width = 1.9cm]{avatare/Alex}\hspace{-8mm}
\includegraphics[width = 1.9cm]{avatare/Jessica}\hspace{-8mm}
\includegraphics[width = 1.9cm]{avatare/Jonas}\hspace{-8mm}
\includegraphics[width = 1.9cm]{avatare/Mark}\hspace{-8mm}
\includegraphics[width = 1.9cm]{avatare/Nilufar}\hspace{-8mm}
\includegraphics[width = 1.9cm]{avatare/Paula}\hspace{-8mm}
\includegraphics[width = 1.9cm]{avatare/Steffen}\hspace{-8mm}
\includegraphics[width = 1.9cm]{avatare/Tina}\hspace{-8mm}
\includegraphics[width = 1.9cm]{avatare/Yuki}\\
\small
\vspace{1.5Ex}
\textit{"`The test of a student is not how much she or he knows,\\ but how much the student wants to know."'\\ --- Alice W. Rollins}
\end{center}}
% -----------------------------------------------------------------------
\maketitle
\fancypagestyle{empty}{
  \fancyfoot[L]{\tiny $\square\!\square\!\square\!\blacksquare\!\square\!\square\!\square\!\square\!\square\!\square\!\square\!\square\!\blacksquare\!\square\!\square\!\square\!\square\!\square\!\blacksquare\!\square$}
}
\thispagestyle{empty}
\clearpage
% -----------------------------------------------------------------------
\begin{minipage}[c]{0.125\textwidth}
\includegraphics[width = 1.9cm]{avatare/Alex}
\end{minipage}
\begin{minipage}[c]{0.875\textwidth}
\textit{Alex studiert im 3. Semester und wiederholt das Modul, da er im ersten Jahr andere Prioritäten für sich gesetzt hat. Das musste sein, da er sich ziemlich im Abitur verausgabt hat. Darüber hinaus war die WG auch eher auf Party angelegt. Alex hofft jetzt über Pünktlichkeit wieder in die Bahn zu kommen. Dafür steht er jetzt immer um 5 Uhr auf! Freunde von ihm beschreiben ihn eher als extrovertiert. Er kennt Paula noch aus der Schulzeit und er überlegt, ob nicht beide Mal nach Mallorca sollten.} 
\end{minipage}\\[2.75Ex]
% -----------------------------------------------------------------------
\begin{minipage}[c]{0.875\textwidth}
\textit{Nach zwei Semestern Studium an der Universität Osnabrück war es dann Jessica doch viel zu theoretisch. Etwas angewandtes sollte es sein, wo sie auch eine Fähigkeit lernt, die frau nutzen kann. Deshalb hat sich Jessica an der Hochschule eingeschrieben. Hoffentlich lernt sie etwas nützliches, wo andere für Geld geben würden. Wer nützlich ist, ist wertvoll. Ihr Traum ist ja eine Hundeschule aufzumachen. Die großen Parties hat sie immer gemieden. Sie ist lieber mit ihrer Hündin im Teuteburgerwald.}
\end{minipage}
\begin{minipage}[c]{0.125\textwidth}
\includegraphics[width = 1.9cm]{avatare/Jessica}
\end{minipage}\\[2.75Ex]
% -----------------------------------------------------------------------
\begin{minipage}[c]{0.125\textwidth}
\includegraphics[width = 1.9cm]{avatare/Jonas}
\end{minipage}
\begin{minipage}[c]{0.875\textwidth}
\textit{Das ist jetzt der letzte Versuch mit einem Studium. Wenn es nicht klappt dann überlegt Jonas das \href{https://www.ihk.de/osnabrueck/aus-und-weiterbildung/ausbildung/ausbildungsbetriebe/projekt-neustart-1087206}{Programm der IHK zu Ausbildungsvermittlung} zu nutzen. Aber eine Runde gibt er sich noch. Struktur ist eigentlich das Wichtigste und diesmal hat er sich alle Altklausuren der Fachschaft besorgt. Dann ist er auch noch gleich der Fachschaft beigetreten um mehr Informationen abzugreifen. Und er versucht nicht seine Zeit mit Alex zu verdaddeln oder in der Fachschaft bei einem Bier oder so...}
\end{minipage}\\[2.75Ex]
% -----------------------------------------------------------------------
\begin{minipage}[c]{0.875\textwidth}
\textit{Nächstes Semester geht es nach Kanada davon hat er schon auf der Berufsschule geträumt. Deshalb konzentriert er sich sehr auf die Prüfungen. Ein Schiff ist im Hafen sicher, aber dafür ist es nicht gebaut worden. Das \href{https://www.hs-osnabrueck.de/wir/fakultaeten/aul/international/}{International Faculty Office} der Fakultät Agrarwissenschaften und Landschaftsarchitektur hat super geholfen, aber es waren einiges an Unterlagen. Jetzt hofft er, dass Tina dann doch noch mitkommt. Aber sonst macht er das eben alleine. Obwohl das eher nicht so seine Art ist. Vielleicht sollte er sich mal einen Tipp bei Tina holen, sie wirkt sehr entschlossen.} 
\end{minipage}
\begin{minipage}[c]{0.125\textwidth}
\includegraphics[width = 1.9cm]{avatare/Mark}
\end{minipage}\\[2.75Ex]
% -----------------------------------------------------------------------
\begin{minipage}[c]{0.125\textwidth}
\includegraphics[width = 1.9cm]{avatare/Nilufar}
\end{minipage}
\begin{minipage}[c]{0.875\textwidth}
\textit{Nach der Ausbildung wollte Nilufar eigentlich gleich anfangen zu arbeiten, aber nach einem Praktikum und der Probezeit stellte sie fest, dass es ohne einen Hochschulabschluss schwer wird Führungsverantwortung zu übernehmen. Mit Menschen kann sie schon immer und dann auch eigene Projekte mit anderen verwirklichen, dass ist doch was. Mit dem notwendigen Abschluss sollte der Start um so einfacher sein. Dann ist sie keine Befehlsempfängerin mehr sondern gibt die Marschrichtung vor. Schon jetzt koordiniert Nilufar das Studium von anderen.}
\end{minipage}\\[2.75Ex]
% -----------------------------------------------------------------------
\begin{minipage}[c]{0.875\textwidth}
\textit{Paula möchte die Welt zu einem besseren Ort machen. Wenn da nicht die anderen Mitmenschen wären. Paula muss das Modul nochmal wiederholen, da es dann am Ende des Semesters zu viel für sie wurde. Eine Lerngruppe hätte geholfen, aber das ist dann gar nicht so einfach eine zu finden. Zwar kennt sie schon Nilufar, aber Nilufar ist ihr manchmal zu forsch. Ihr schwant aber, dass alleine das Studium sehr schwer werden wird. Das Abitur war schon so ein Lernhorror, das möchte sie nicht nochmal. Alex sieht sie da als Vorbild.}
\end{minipage}
\begin{minipage}[c]{0.125\textwidth}
\includegraphics[width = 1.9cm]{avatare/Paula}
\end{minipage}\\[2.75Ex]
% -----------------------------------------------------------------------
\begin{minipage}[c]{0.125\textwidth}
\includegraphics[width = 1.9cm]{avatare/Steffen}
\end{minipage}
\begin{minipage}[c]{0.875\textwidth}
\textit{Sommer, Sonne, Natur. Das ist es was Steffen mag. Raus in die Komune und die Natur genießen. Leider hat Steffen noch andere Bedürfnisse, die ein Einkommen benötigen. Da Studierte mehr verdienen, würde dann in Teilzeit auch mehr rausspringen. Wenn er dann privat was anbauen kann, dann spart er gleich doppelt. Leider sind viele seiner Kommilitonen total verkrampfte Karrieristen. Es geht nur ums Äußere. Dabei verliert sich Steffen gerne im Prozess. Das hat auch seinen Schulabschluss etwas verzögert. Steffen lässt sich eben Zeit.}
\end{minipage}\\[2.75Ex]
% -----------------------------------------------------------------------
\begin{minipage}[c]{0.875\textwidth}
\textit{Wille  war es, die es Tina hierher gebracht hat und Wille wird es sein, die Tina dann auch zum Abschluß treibt. Nach einem Rückschlag muss Tina jetzt einige Module wiederholen, damit sie dann auch fertig wird. Ab und zu ist sie schwach gewesen und das hat dann Zeit gekostet. Das Tina es dann manchmal übertreibt, weiß sie nur zu gut, aber irgendwie muss die Kontrolle ja erhalten bleiben? Insbesondere, wenn sie mal wieder die Nacht durchgefeiert hat, verachtet Tina sich. Dann baut Nilufar sie dann bei einem Tee wieder auf.}
\end{minipage}
\begin{minipage}[c]{0.125\textwidth}
\includegraphics[width = 1.9cm]{avatare/Tina}
\end{minipage}\\[2.75Ex]
% -----------------------------------------------------------------------
\begin{minipage}[c]{0.125\textwidth}
\includegraphics[width = 1.9cm]{avatare/Yuki}
\end{minipage}
\begin{minipage}[c]{0.875\textwidth}
\textit{Für Yuki war es nicht einfach. Teilweise war die Krankheit sehr hinderlich, dann war es Yuki selber. Dann muss man auch wieder auf die Beine kommen und es dauert eben seine Zeit. Aber immerhin hat Yuki es jetzt den Abschluss gekriegt und hat einen Studienplatz. Jetzt heißt es in den Rhythmus kommen und schauen, was noch so passiert. Immerhin hat Yuki schon eine kleine Gruppe gefunden, in der Yuki dann Hilfe findet. Ist aber auch sehr unübersichtlich so ein Studium. Steffen ist immer super entspannt.}
\end{minipage}
\clearpage
% -----------------------------------------------------------------------


\begin{graybox}{Erlaubte Hilfsmittel}
  \vspace{1Ex}
  \begin{itemize}
  \item Normaler Taschenrechner ohne Möglichkeit der Kommunikation mit anderen
    Geräten! Ausdrücklich kein Handy!
  \item Eine DIN A4-Seite als beidseitig, selbstgeschriebene,
    handschriftliche Formelsammlung. Keine digitalen Ausdrucke! 
  \item \textbf{\textcolor{red}{Die Verwendung eines roten Farbstiftes ist nicht gestattet! Korrekturfarbe!}}
  \item \textit{You can answer the questions in English without any consequences.}  
  \end{itemize}
\end{graybox}
\vfill

\begin{graybox}{Endnote}
  \vspace{1Ex}
  \begin{itemize}
  \item[] \rule[0ex]{3em}{.4pt}\, von 20\, Punkten sind aus den Multiple
    Choice Aufgaben erreicht.
  \item[] \rule[0ex]{3em}{.4pt}\, von 86 Punkten sind aus den Rechen- und
    Textaufgaben erreicht. 
  \item[] \rule[0ex]{3em}{.4pt}\, von 106 Punkten in Summe.
  \item[] Es wird folgender Notenschlüssel angewendet.   
  \end{itemize}
  \vspace{1ex}
\begin{center}
  \begin{tabular}[c]{cc}
    \toprule
    \textbf{Punkte}	&	\textbf{Note}	\\
    \midrule
    101.5 - 106	&	1,0	\\
    96.0 - 101	&	1,3	\\
    90.5 - 95.5	&	1,7	\\
    85.5 - 90.0	&	2,0	\\
    80.0 - 85.0	&	2,3	\\
    74.5 - 79.5	&	2,7	\\
    69.5 - 74.0	&	3,0	\\
    64.0 - 69.0	&	3,3	\\
    59.0 - 63.5	&	3,7	\\
    53.0 - 58.5	&	4,0	\\
    \bottomrule
  \end{tabular}
\end{center}
  \vspace{1ex}
\begin{itemize}
\item[] Es ergibt sich eine Endnote von \rule[0ex]{4em}{.4pt}.
\end{itemize}
  \vspace{1Ex}
\end{graybox}

% -----------------------------------------------------------------------
\newpage
% -----------------------------------------------------------------------

\begin{graybox}{Multiple Choice Aufgaben}
  \begin{itemize}
  \item Pro Multipe Choice Frage ist \emph{genau} eine Antwort richtig.
  \item \textbf{Übertragen Sie Ihre Kreuze in die Tabelle auf
      dieser Seite.}
  \end{itemize}

\begin{center}
  \large
  \begin{tabular}{|l|c|c|c|c|c?c|}
    \hline
    & \textbf{A} & \textbf{B} & \textbf{C} & \textbf{D} & \textbf{E} & $\boldsymbol{\checkmark}$\strut\\
    \hline
    \textbf{Aufgabe 1} &   &   &   &   &   & \strut\\
    \hline
    \textbf{Aufgabe 2} &   &   &   &   &   & \strut\\
    \hline
    \textbf{Aufgabe 3} &   &   &   &   &   & \strut\\
    \hline
    \textbf{Aufgabe 4} &   &   &   &   &   & \strut\\
    \hline
    \textbf{Aufgabe 5} &   &   &   &   &   & \strut\\
    \hline
    \textbf{Aufgabe 6} &   &   &   &   &   & \strut\\
    \hline
    \textbf{Aufgabe 7} &   &   &   &   &   & \strut\\
    \hline
    \textbf{Aufgabe 8} &   &   &   &   &   & \strut\\
    \hline
    \textbf{Aufgabe 9} &   &   &   &   &   & \strut\\
    \hline
    \textbf{Aufgabe 10} &   &   &   &   &   & \strut\\
    \hline
  \end{tabular}
\end{center}

\begin{itemize}
\item Es sind \rule[0ex]{2em}{.4pt}\, von 20 Punkten erreicht worden.
\end{itemize}
\end{graybox}

\vfill

\begin{graybox}{Rechen- und Textaufgaben}
  \begin{center}
    \large
    \begin{tabular}{|l|c|c|c|c|c|c|c|}
      \hline
      \textbf{Aufgabe} & \textbf{11} & \textbf{12} & \textbf{13} & \textbf{14} & \textbf{15} & \textbf{16} & \textbf{17} \strut\\
      \hline
      \textbf{Punkte} & 
      \hspace{1Ex}\Large\textcolor{gray!70}{10}\hspace{1Ex}  & 
      \hspace{1Ex}\Large\textcolor{gray!70}{10}\hspace{1Ex}  & 
      \hspace{1Ex}\Large\textcolor{gray!70}{16}\hspace{1Ex}  & 
      \hspace{1Ex}\Large\textcolor{gray!70}{20}\hspace{1Ex}  & 
      \hspace{1Ex}\Large\textcolor{gray!70}{9}\hspace{1Ex}  & 
      \hspace{1Ex}\Large\textcolor{gray!70}{11}\hspace{1Ex}  & 
      \hspace{1Ex}\Large\textcolor{gray!70}{10}\hspace{1Ex} \strut\\
      \hline
  \end{tabular}
\end{center}
\begin{itemize}
\item Es sind \rule[0ex]{2em}{.4pt}\, von 86 Punkten erreicht worden.
\end{itemize}
\end{graybox}

% -----------------------------------------------------------------------
\clearpage
% -----------------------------------------------------------------------
\begin{graybox}{Multiple Choice Aufgaben}
Die Multiple Choice Aufgaben \textcolor{red}{unterliegen dem Zufall}. Die Reihenfolge der Antworten ist zufällig. Die Fragen und Antworten sind semantisch zufällig und haben somit \textcolor{red}{verschiedene Textvarianten}. Insbesondere die reinen Textaufgaben haben verschiedene Textvarianten. Die Semeantik mag sich unterscheiden, die Inhalte sind aber gleich.
\end{graybox}
\section*{Programmieren in R} 

\section{Aufgabe \hfill (2 Punkte)}



Sie wollen Ihren Datensatz in \Rlogo einlesen und stehen nun vor einem Problem. Sie stellen fest, dass die Hilfeseiten alle in englischer Sprache verfasst sind. Warum mag die Nutzung von Deutsch problematisch sein?



\begin{enumerate}
\item [\textbf{A} \msquare] Im Allgemeinen haben Programmiersprachen Probleme mit Umlauten und Sonderzeichen, die in der deutschen Sprache vorkommen. Eine Nutzung der englischen Sprache umgeht dieses Problem auf einfache Art.
\item [\textbf{B} \msquare] Die Spracherkennung von \Rlogo ist nicht in der Lage Deutsch zu verstehen.
\item [\textbf{C} \msquare] \Rlogo Pakete sind nur in englischer Sprache verfasst. Es macht keinen Sinn \Rlogo daher in Deutsch zu bedienen.
\item [\textbf{D} \msquare] Programmiersprachen können nur englische Begriffe verarbeiten. Zusätzliche Pakete können zwar geladen werden, aber meist funktionieren diese Pakete nicht richtig. Deutsch ist International nicht bedeutend genug.
\item [\textbf{E} \msquare] Programmiersprachen haben Probleme mit Umlauten und Sonderzeichen der deutschen Sprache. Daher ist die Nutzung in Deutsch in den AGBs von \Rlogo untersagt.
\end{enumerate} 

\section{Aufgabe \hfill (2 Punkte)}



In \Rlogo müssen die Daten eines Experiments in einem bestimmten Format vorliegen. Wir sprechen auch von dem log-Format der Daten. Welche Aussage zu dem Long-Format ist richtig?



\begin{enumerate}
\item [\textbf{A} \msquare] Das Long-Format hat die gleiche Anzahl an Spalten wie auch an Zeilen.
\item [\textbf{B} \msquare] Das Long-Format beschreibt in den Spalten die Beobachtungen sowie in den Zeilen die \textit{unabhängigen} Beobachtungen.
\item [\textbf{C} \msquare] In den Spalten finden sich die einzelnen Beobachtungen und in den Zeilen die gemessenen Variablen ($Y$) sowie die experimentellen Faktoren $X$.
\item [\textbf{D} \msquare] Wichtig ist, dass sich in den Zeilen die Beobachtungen finden und in den Spalten die Variablen, wie die Messwerte $Y$ sowie die experimentellen Faktoren $X$.
\item [\textbf{E} \msquare] In den Spalten sind die Beobachtungen in den Zeilen die Variablen, wie die Messwerte und experiementellen Faktoren.
\end{enumerate} 

\section{Aufgabe \hfill (2 Punkte)}



In Ihrer Abschlussarbeit wollen Sie zu Beginn eine explorativen Datenanalyse (EDA) in \Rlogo rechnen. Dafür gibt es eine generelle Abfolge von Prozessschritten. Welche ist hierbei die richtige Reihenfolge?



\begin{enumerate}
\item [\textbf{A} \msquare] Die Funktionsreihenfolge ist wie folgt: \texttt{read\_excel()} ->  \texttt{mutate()} -> \text{ggplot()}. Dabei ist bei der Transformation der Daten darauf zu achten, dass keine Faktoren erstellt werden.
\item [\textbf{B} \msquare] Die Funktionsreihenfolge ist wie folgt: \texttt{read\_excel()} ->  \texttt{mutate()} -> \text{ggplot()}. Dabei ist bei der Transformation der Daten darauf zu achten, dass die Faktoren richtig erstellt werden.
\item [\textbf{C} \msquare] Wir lesen als erstes die Daten über \texttt{read\_excel()} ein, transformieren die Spalten über \texttt{mutate()} in die richtige Form und können dann  über \text{ggplot()} uns die Abbildungen erstellen lassen. Wichtig ist, dass wir keine Faktoren sondern nur numerische Variablen vorliegen haben.
\item [\textbf{D} \msquare] Für eine explorativen Datenanalyse (EDA) in \Rlogo müssen wir als erstes die Daten über \texttt{read\_excel()} einlesen. Danach müssen wir schauen, dass wir die Zeilen richtig über \texttt{mutate()} transformiert haben. Insbesondere müssen Variablen mit kontinuierlichen Werten in einen Faktor umgewandelt werden. Am Ende nutzen wir die Funktion \text{ggplot()} für die eigentlich EDA.
\item [\textbf{E} \msquare] Wir lesen die Daten ein und mutieren die Daten. Dabei ist wichtig, dass wir nicht das Paket \texttt{tidyverse} nutzen, da dieses Paket veraltet ist. über die Funktion \texttt{library(tidyverse)} entfernen wir das Paket von der Analyse.
\end{enumerate} 
\section*{Deskriptive Statistik \& Explorative Datenanalyse}

\section{Aufgabe \hfill (2 Punkte)}




Gegeben ist $y$ mit 8, 12, 15, 5 und 18. Berechnen Sie den Mittelwert und Standardabweichung.



\begin{enumerate}
\item [\textbf{A} \msquare] Es ergibt sich 12.6 +/- 2.61
\item [\textbf{B} \msquare] Sie erhalten 11.6 +/- 2.61
\item [\textbf{C} \msquare] Sie erhalten 11.6 +/- 5.22
\item [\textbf{D} \msquare] Es berechnet sich 11.6 +/- 27.3
\item [\textbf{E} \msquare] Sie erhalten 11.6 +/- 2.28
\end{enumerate} 

\section{Aufgabe \hfill (2 Punkte)}




Wie lautet der Median, das $1^{st}$ Quartile sowie das $3^{rd}$ Quartile von $y$ mit 30, 13, 15, 29, 32, 9, 2, 37, 26, 11 und 63.




\begin{enumerate}
\item [\textbf{A} \msquare] Es berechnet sich 24 [12; 33]
\item [\textbf{B} \msquare] Sie erhalten 26 [9; 30]
\item [\textbf{C} \msquare] Es ergibt sich 26 +/- 11
\item [\textbf{D} \msquare] Sie erhalten 26 [11; 32]
\item [\textbf{E} \msquare] Es ergibt sich 24 +/- 11
\end{enumerate} 

\section{Aufgabe \hfill (2 Punkte)}



Um die Standardabweichung zu berechnen müssen wir folgende Rechenoperationen durchführen.



\begin{enumerate}
\item [\textbf{A} \msquare] Den Mittelwert berechnen und die Abstände quadrieren. Die Summe mit der Fallzahl multiplizieren.
\item [\textbf{B} \msquare] Als erstes berechnen wir den Mittelwert. Dann bilden wir die Summe der quadratischen Abstände zu dem Mittelwert. Abschließend subtrahieren wir die Fallzahl.
\item [\textbf{C} \msquare] Als erstes berechnen wir den Mittelwert. Dann bilden wir die Summe der quadratischen Abstände zu dem Mittelwert. Abschließend teilen wir durch die Fallzahl.
\item [\textbf{D} \msquare] Wir berechnen erst den Mittelwert und dann die absoluten Abstände zu dem Mittelwert. Diese quadratischen Abstände summieren wir auf und teilen am Ende durch die Fallzahl.
\item [\textbf{E} \msquare] Wir berechnen erst den Mittelwert und dann die quadratischen Abstände zu dem Mittelwert. Diese quadratischen Abstände summieren wir auf und teilen am Ende durch die Fallzahl. Als letzten Schritt ziehen wir die quadratische Wurzel.
\end{enumerate} 

\section{Aufgabe \hfill (2 Punkte)}



Die Normalverteilung wird durch den Mittelwert und die Standardabweichung definiert. Welche Aussage im Bezug zur Standardabweichung in einer Normalverteilung ist richtig?



{\centering \includegraphics[width=\maxwidth]{img/mc-distribution-02-a-1} 

}







\begin{enumerate}
\item [\textbf{A} \msquare] Die Fläche rechts von $2\sigma$ ist der p-Wert mit $Pr(D|H_0)$ in der obigen Abbildung.
\item [\textbf{B} \msquare] Die Fläche zwischen $-1\sigma$ und $1\sigma$ ist 0.95 und 95\% der Beobachtungen liegen somit zwischen $\bar{y}\pm\sigma$ in der obigen Verteilung.
\item [\textbf{C} \msquare] Die Fläche zwischen $-2\sigma$ und $2\sigma$ ist 0.68 und 68\% der Beobachtungen liegen somit zwischen $\bar{y}\pm\sigma$ in der obigen Verteilung.
\item [\textbf{D} \msquare] Die Fläche unter der Kurve ist 1, wenn die Nullhypothese falsch ist. Wenn die Nullhypothese gilt, dann ist die Fläche $1-\alpha$. Somit ergibt sich auch eine Standardabweichung $\sigma$ mit $\alpha$ gleich 0.05 in beiden Fällen.
\item [\textbf{E} \msquare] Die Fläche zwischen $-2\sigma$ und $2\sigma$ ist 0.95 und 95\% der Beobachtungen liegen somit zwischen $\bar{y}\pm\sigma$ in der obigen Verteilung.
\end{enumerate} 

\section{Aufgabe \hfill (2 Punkte)}



Nachdem Sie eine ANOVA und die paarweisen t-Tests über das \Rlogo Paket \{emmeans\} durchgeführt haben, müssen Sie Ihre Daten nochmal zur Überprüfung visualisieren. Sie entscheiden sich für den Boxplot. Welche statistischen Maßzahlen stellt der Boxplot dar?

 



\begin{enumerate}
\item [\textbf{A} \msquare] Durch die Abbildung des Boxplot erhalten wir die Informationen über den Median und die Quartile.
\item [\textbf{B} \msquare] Den Median und die Standardabweichung.
\item [\textbf{C} \msquare] Der Boxplot stellt die Mittelwerte und die Varianz dar.
\item [\textbf{D} \msquare] Der Boxplot stellt den Median und die Streuung dar.
\item [\textbf{E} \msquare] Der Boxplot stellt die Mittelwerte und die Standardabweichung dar.
\end{enumerate}

\section{Aufgabe \hfill (2 Punkte)}



Der Mittelwert $\bar{y}$ und der Median $\tilde{y}$ unterscheiden sich nicht in Ihren Feldexperiment zu Leistungssteigerung von Brokoli.  Welche Aussage ist richtig?



\begin{enumerate}
\item [\textbf{A} \msquare] Der Mittelwert und der Median sollten gleich sein, wenn keine Outlier in den Daten vorliegen. 
\item [\textbf{B} \msquare] Wenn sich der Mittelwert und der Median nicht unterscheiden, liegen vermutlich keine Outlier in den Daten vor.
\item [\textbf{C} \msquare] Wenn sich der Mittelwert und der Median nicht unterscheiden, liegen vermutlich Outlier in den Daten vor.
\item [\textbf{D} \msquare] Da sich der Mittelwert und der Median unterscheiden, ist der Datensatz nicht zu verwenden. Mittelwert und Median müssen gleich sein.
\item [\textbf{E} \msquare] Wenn sich der Mittelwert und der Median unterscheiden, liegen vermutlich keine Outlier in den Daten vor.
\end{enumerate} 

\section{Aufgabe \hfill (2 Punkte)}



Mit einem Dotplot  können Sie sehr gut die Verteilung von Daten visualisieren. Die empfohlene Mindestanzahl an Beobachtungen ist dabei?



\begin{enumerate}
\item [\textbf{A} \msquare] Wir sollten zwei bis fünf Beobachtungen mindestens pro Gruppe vorliegen haben.
\item [\textbf{B} \msquare] Wir sollten eine Beobachtung mindestens pro Gruppe vorliegen haben.
\item [\textbf{C} \msquare] Wir brauchen fünf oder mehr Beobachtungen.
\item [\textbf{D} \msquare] Die untere Grenze liegt bei einer Beobachtung.
\item [\textbf{E} \msquare]  erhalten, sollten wir mindestens zwanzig Beobachtungen haben.
\end{enumerate}

\section{Aufgabe \hfill (2 Punkte)}




In der Statistik müssen wir häufig überprüfen, ob unser Outcome einer bestimmten Verteilung folgt. Meistens überprüfen wir, ob eine
Normalverteilung vorliegt. Folgende drei Abbildungen eigenen sich im Besonderen für die Überprüfung einer Verteilungsannahme an eine Variable.





\begin{enumerate}
\item [\textbf{A} \msquare] Boxplot, Violinplot, Mosaicplot
\item [\textbf{B} \msquare] Scatterplot, Densityplot, Barplot
\item [\textbf{C} \msquare] Boxplot, Densityplot, Violinplot
\item [\textbf{D} \msquare] Histogramm, Densityplot, Dotplot
\item [\textbf{E} \msquare] Scatterplot, Mosaicplot, Boxplot
\end{enumerate}

\section{Aufgabe \hfill (2 Punkte)}



Ihre Betreuung der Abschlussarbeit fragt überraschend in der letzten Besprechung, ob Ihre Messwerte einer Varianzhomogenität genügen. Sonst könnten Sie ja gar nicht einen t-Test rechnen. Da Ihnen die Zeit wegrennt, entscheiden Sie sich für eine schnelle Visualisierung im Anhang. Welche Visualisierung nutzen Sie und welche Regel kommt zur Abschätzung einer Varianzhomogenität zur Anwendung?



\begin{enumerate}
\item [\textbf{A} \msquare] Einen Dotplot. Die Punkte müssen sich wie an einer Perlenschnurr audreihen. Eine Abweichung führt zur Ablehnung der Annahme einer Varianzhomogenität.
\item [\textbf{B} \msquare] Nach dem Einlesen der Daten nutzen wir einen Boxplot um zu schauen, ob alle Boxen über alle Behandlungen in etwa gleich groß sind. Damit ist dann auch das IQR in allen Behandlungen in etwa gleich.
\item [\textbf{C} \msquare] Nach dem Einlesen der Daten nutzen wir einen Barplot um zu schauen, ob alle Mittelwerte über alle Behandlungen in etwa gleich groß sind. Damit ist dann auch die Varianz in allen Behandlungen in etwa gleich.
\item [\textbf{D} \msquare] In einer explorativen Datanalyse nutzen wir den Violinplot. Dabei sollte der Bauch am Rand liegen. Dann können wir von einer Varianzhomogenität ausgehen.
\item [\textbf{E} \msquare] In einer explorativen Datanalyse nutzen wir den Boxplot. Dabei sollte der Median als dicke Linie in der Mitte der Box liegen. Dann können wir von einer Varianzhomogenität ausgehen.
\end{enumerate}
\section*{Statistische Testtheorie}  

\section{Aufgabe \hfill (2 Punkte)}




Geben ist $Pr(D|H_0)$ als mathematischer Ausdruck, welche Aussage ist richtig?



\begin{enumerate}
\item [\textbf{A} \msquare] $Pr(D|H_0)$ ist die Wahrscheinlichkeit nicht die Daten $D$ zu beobachten sondern die Nullhypothese, wenn diese wahr ist.
\item [\textbf{B} \msquare] $Pr(D|H_0)$ stellt die Wahrscheinlichkeit die Daten $D$ und somit die Teststatistik $T_D$ zu beobachten dar, wenn die Nullhypothese wahr ist.
\item [\textbf{C} \msquare] $Pr(D|H_0)$ stellt die Wahrscheinlichkeit die Teststatistik $T$ zu beobachten dar, wenn die Nullhypothese falsch ist.
\item [\textbf{D} \msquare] Die Wahrscheinlichkeit für die Nullhypothese, wenn die Daten wahr sind.
\item [\textbf{E} \msquare] Die Wahrscheinlichkeit der Daten unter der Nullhypothese in der Grundgesamtheit.
\end{enumerate} 

\section{Aufgabe \hfill (2 Punkte)}



Das statistische Testen basiert auf dem Falsifikationsprinzip. Es besagt,



\begin{enumerate}
\item [\textbf{A} \msquare] ... dass in der Wissenschaft immer etwas falsch sein muss. Sonst gebe es keinen Fortschritt.
\item [\textbf{B} \msquare] ... dass ein schlechtes Modell durch das Falsifikationsprinzip durch ein noch schlechteres Modell ersetzt wird. Die Wissenschaft lehnt ab und verifiziert nicht.
\item [\textbf{C} \msquare] ... dass ein schlechtes Modell durch ein schlechteres Modell ersetzt wird. Die Wissenschaft lehnt ab und verifiziert nicht.
\item [\textbf{D} \msquare] ... dass Fehlerterme in statistischen Modellen nicht verifiziert werden können.
\item [\textbf{E} \msquare] ... dass ein schlechtes Modell durch ein weniger schlechtes Modell ersetzt wird. Die Wissenschaft lehnt ab und verifiziert nicht.
\end{enumerate} 

\section{Aufgabe \hfill (2 Punkte)}



Der Fehler 1. Art oder auch Signifikanzniveau $\alpha$ genannt, liegt bei
5\%. Welcher der folgenden Gründe für diese Festlegeung auf 5\% als Signifikanzschwelle ist richtig?



\begin{enumerate}
\item [\textbf{A} \msquare] In der Wissenschaft gibt es neben der Naturkonstante, die sich aus der Beobachtung der Welt ergibt, noch die Kulturkonstante, die von einer Gruppe Menschen selbstgewählt wird. Dabei ist $\alpha = 5\%$ eine Kulturkonstante und wurde somit eher zufällig gewählt.
\item [\textbf{B} \msquare] Als Kulturkonstante hat $\alpha = 5\%$ den Rang einer Naturkonstante und wurde nach langer Diskussion in der UN im Jahre 1983 festgesetzt. Damals auch schon mit der Zustimmung der UdSSR.
\item [\textbf{C} \msquare] Da Wissenschaftler eine Schwelle für die statistische Testentscheidung benötigen wurde $\alpha$ in einer großen Konferenz 1945 gewählt. Damit ist $\alpha = 5\%$ eine Kulturkonstante mit einem Rank einer Naturkonstante.
\item [\textbf{D} \msquare] Auf einer Statistikkonferenz in Genf im Jahre 1942 wurde dieser Cut-Off nach langen Diskussionen festgelegt. Bis heute ist der Cut Off aber umstritten, da wegen dem 2. Weltkrieg viele Wissenschaftler nicht teilnehmen konnten.
\item [\textbf{E} \msquare] Der Wert ergab sich aus einer Auswertung von 1042 wissenschaftlichen Veröffentlichungen zwischen 1914 und 1948. Der Wert $5\%$ wurde in $28\%$ der Veröffentlichungen genutzt. Daher legte man sich auf diese Zahl fest.
\end{enumerate}

\section{Aufgabe \hfill (2 Punkte)}

Betrachten wir die Teststatistik aus einem abstrakteren Blickwinkel. Beim
statistischen Testen wird das \textit{"`signal"'} mit dem
\textit{"`noise"'} aus den Daten $D$ zu einer Teststatistik $T_D$ verrechnet. Welche der Formel
berechnet korrekt die Teststatistik $T_D$?



\begin{enumerate}
\item [\textbf{A} \msquare] Es gilt $T_D = \cfrac{signal}{noise^2}$
\item [\textbf{B} \msquare] Es gilt $T_D = signal \cdot noise$
\item [\textbf{C} \msquare] Es gilt $T_D = \cfrac{noise}{signal}$
\item [\textbf{D} \msquare] Es gilt $T_D = \cfrac{signal}{noise}$
\item [\textbf{E} \msquare] Es gilt $T_D = (signal \cdot noise)^2$
\end{enumerate}

%% ------------------------------------------------------------

\section{Aufgabe \hfill (2 Punkte)}



Eine Analogie kann helfen einen Sachverhalt besser zu verstehen. Wie kann folgende Aussage richtig in die Analogie der statistischen Testtheorie gesetzt werden?

\begin{center}
\textit{$H_0$ beibehalten obwohl die $H_0$ falsch ist}
\end{center}



\begin{enumerate}
\item [\textbf{A} \msquare] In die Analogie eines Feuerwehrautos: \textit{Car without noise}.
\item [\textbf{B} \msquare] \textit{Alarm with fire}, dem $\alpha$-Fehler in der Analogie von Feuer.
\item [\textbf{C} \msquare] \textit{Fire without alarm}, dem $\beta$-Fehler als Analogie eines Rauchmelders.
\item [\textbf{D} \msquare] In die Analogie eines Rauchmelders: \textit{Alarm without fire}, dem $\alpha$-Fehler.
\item [\textbf{E} \msquare] Dem $\beta$-Fehler mit der Analogie eines brennenden Hauses: \textit{Fire without alarm}.
\end{enumerate}

\section{Aufgabe \hfill (2 Punkte)}



Sie sollen in Ihrer Abschlussarbeit die Relevanz und die Signifikanz in einer statistischen Maßzahl vereinen. Welche Aussage ist richtig?



\begin{enumerate}
\item [\textbf{A} \msquare] Das $\Delta$. Durch die Effektstärke haben wir einen Wert für die Relevanz, die vom Anwender bewertet werden muss. Da $\Delta$ antiproportional zum p-Wert ist, bedeutet auch ein hohes $\Delta$ ein sehr kleinen p-Wert.
\item [\textbf{B} \msquare] Einem Konfidenzintervall. Das Konfidenzinterval bringt durch eine Visualisierung und zwei Intervallgrenzen die Möglichkeit mit, eine Relevanzschwelle neben der definierten Signifikanzschwelle zu definieren.
\item [\textbf{C} \msquare] Die Teststatistik. Durch den Vergleich von $T_c$ zu $T_k$ ist es m{"o}glich die $H_0$ abzulehnen. Die Relevanz ergibt sich aus der Fläche rechts vom dem $T_c$-Wert.
\item [\textbf{D} \msquare] Der p-Wert. Durch den Vergleich mit $\alpha$ lässt sich über die Signifikanz entscheiden und der $\beta$-Fehler erlaubt über die Power eine Einschätzung der Relevanz.
\item [\textbf{E} \msquare] Das OR. Als Chancenverhältnis gibt es das Verhältnis von Relevanz und Signifikanz wieder.
\end{enumerate}

\section{Aufgabe \hfill (2 Punkte)}



Ein statistischer Test produziert für einen Gruppenvergleich einen $p$-Wert. Welche Aussage zusammen mit dem Signifikanzniveau $\alpha$ gleich 5\% stimmt?



\begin{enumerate}
\item [\textbf{A} \msquare] Wir machen ein Aussage über die Flächen und der Kurve der Teststatistik, wenn die $H_0$ gilt. Dabei werden Wahrscheinlichkeiten vergleichen, die durch die Flächen unter der Kurve repräsentiert werden.
\item [\textbf{B} \msquare] Wir schauen, ob der $p$-Wert größer ist als das Signifikanzniveau $\alpha$ und vergleichen somit Wahrscheinlichkeiten. Die Wahrscheinlichkeiten werden als Flächen unter der Kurve der Teststaistik dargestellt, wenn die $H_A$ gilt.
\item [\textbf{C} \msquare] Wir machen eine Aussage über die indivduelle Wahrscheinlichkeit des Eintretens der Nullhypothese $H_0$. Der $p$-Wert wird mit dem Signifikanzniveau verglichen und bewertet.
\item [\textbf{D} \msquare] Wir vergleichen mit dem $p$-Wert und dem Signifikanzniveau $\alpha$ Wahrscheinlichkeiten und damit die absoluten Werte auf einem Zahlenstrahl, wenn die $H_0$ gilt.
\item [\textbf{E} \msquare] Wir vergleichen mit dem $p$-Wert und dem Signifikanzniveau $\alpha$ absolute Werte auf einem Zahlenstrahl und damit den Unterschied der Teststatistiken, wenn die $H_0$ gilt.
\end{enumerate}

\section{Aufgabe \hfill (2 Punkte)}



Um die Ergebnisse eines statistischen Tests und die damit verbundene Theorie besser zu verstehen, kann eine Analogie zur Wettervorhersage genutzt werden. Welche Analogie zu der Testtheorie trifft am meisten zu?



\begin{enumerate}
\item [\textbf{A} \msquare] In der Analogie der Maximaltemperatur: Was ist der maximale Unterschied zwischen zwei Gruppen. Wir erhalten hier eine Aussage über die Spannweite und den maximalen Effekt.
\item [\textbf{B} \msquare] In der Analogie des Niederschlags oder Regenmenge: ein statistischer Test gibt die Stärke eines Effektes wieder. Zum Beispiel, wie hoch ist der Mittelwertsunterschied.
\item [\textbf{C} \msquare] In der Analogie der Durchschnittstemperatur: Wie oft tritt ein Effekt durchschnittlich ein? Wir erhalten eine Wahrscheinlichkeit für die Effekte. Zum Beispiel, wie hoch ist die Wahrscheinlichkeit für einen Mittelwert als Durchschnitt.
\item [\textbf{D} \msquare] In der Analogie der Sonnenscheindauer: Wie lange kann mit einem entsprechenden Effekt gerechnet werden? Die Wahrscheinlichkeit für den Effekt gibt der statistische Test wieder.
\item [\textbf{E} \msquare] In der Analogie der Regenwahrscheinlichkeit: ein statistischer Test gibt die Wahrscheinlichkeit für das Auftreten eines Ereignisses wieder. Die Stärke des Effektes wird nicht wiedergeben.
\end{enumerate}

\section{Aufgabe \hfill (2 Punkte)}



Sie wollen eine Aussage über die untersuchte Population treffen. Dazu nutzen Sie einen statistischen Test. Können Sie eine valide Aussage aus einem statistischen Test erhalten?



\begin{enumerate}
\item [\textbf{A} \msquare] Ja, wir erhalten eine Aussage. Müssen aber das Individuum im Kontext der Population adjustieren.
\item [\textbf{B} \msquare] Nein, wir erhalten nur eine Aussage zu zwei Individuen. Ein statistischer Test liefert Informationen zu einem Individuum im Vergleich zu einem anderen Individuum.
\item [\textbf{C} \msquare] Nein, es ist nicht möglich die untersuchte Population mit einem t-Test auszuwerten. Wir erhalten dann leider keine Aussage zur Population.
\item [\textbf{D} \msquare] Nein, wir können die untersuchte Population nicht mit einer ANOVA auswerten. Wir erhalten keine Aussage zur Population. Wir können aber den Test adjustieren und so die Auswertung ermöglichen.
\item [\textbf{E} \msquare] Ja, die untersuchte Population können wir mit einem statistischen Test auswerten. Wir erhalten dann eine Aussage zur Population.
\end{enumerate}

\section{Aufgabe \hfill (2 Punkte)}



Sie haben die \textit{Power} berechnet. Was sagt Ihnen dieser statistische Begriff aus?



\begin{enumerate}
\item [\textbf{A} \msquare] Die Power ist nicht in der aktuellen Testthorie mehr vertreten. Wir rechnen nur noch mit dem Fehler 1. Art.
\item [\textbf{B} \msquare] Es gilt $\alpha + \beta = 1$ und somit liegt $\beta$ meist bei 95\%.
\item [\textbf{C} \msquare] Alle statistischen Tests sind so konstruiert, dass die $H_A$ mit 20\% \textit{bewiesen wird}. Die Power ist $1-\beta$ mit $\beta$ gleich 80\% gesetzt.
\item [\textbf{D} \msquare] Die Power $1-\beta$ wird auf 80\% gesetzt. Damit liegt die Wahrscheinlichkeit für die $H_0$ bei 20\%.
\item [\textbf{E} \msquare] Die Power wird nicht berechnet sondern ist eine Eigenschaft des Tests. Die Power wird auf $80\%$ gesetzt und beschreibt mit welcher Wahrscheinlichkeit $H_A$ \textit{bewiesen wird}
\end{enumerate}

\section{Aufgabe \hfill (2 Punkte)}



Sie rechnen einen statistischen Test und erhalten neben dem p-Wert noch einen Effekt wiedergegeben. Welche Aussage zum Effekt ist richtig?



\begin{enumerate}
\item [\textbf{A} \msquare] Der Forschende muss am Ende wissen, ob das Eregbnis eines Experiments relevant für seine Forschung ist. Dafür kann der Effekt eines statistischen Tests genutzt werden. Damit beschreibt der Effekt den biologischen interpretierbaren Teil einer Ausgabe eines Tests. Zum Beispiel der Unterschied zwischen zwei Anteilen.
\item [\textbf{B} \msquare] Der Effekt eines statistischen Tests beschreibt die biologisch interpretierbare Ausgabe eines Tests. Moderen Algorithmen liefern keine Effekte mehr sondern nur noch bedingte Wahrscheinlichkeiten. Der Effekt spielt in der modernen Statistik keine Rollen mehr.
\item [\textbf{C} \msquare] Der Effekt eines statistischen Tests beschreibt die biologisch interpretierbare Ausgabe eines Tests. Damit ist der Effekt direkt mit dem Begriff der Signifikanz verbunden. Die Entscheidung über die Signifikanz trifft der Forschende unabhängig von der Relevanz eines statistsichen Tests.
\item [\textbf{D} \msquare] Der Effekt eines statistischen Tests beschreibt die mathematisch interpretierbare Ausgabe eines Tests. Damit ist der Effekt direkt mit dem Begriff der Signifikanz verbunden. Die Entscheidung über die Signifikanz trifft der Forschende unabhängig von der Relevanz eines statistsichen Tests.
\item [\textbf{E} \msquare] Der Effekt eines statistischen Tests beschreibt den Output oder die Wiedergabe eines Tests in einem Computer.
\end{enumerate}

\section{Aufgabe \hfill (2 Punkte)}



Welche Aussage über die Entscheidung anhand der berechneten Teststatistik gegen die
Nullhypothese ist richtig?



\begin{enumerate}
\item [\textbf{A} \msquare] Ist $Pr(D|H_0)$ kleiner als das Signifikanzniveau $\alpha$ gleich $5\%$ dann wird die Nullhypothese $H_0$ abgelehnt.
\item [\textbf{B} \msquare] Ist in dem 95\%-Konfidenzintervall nicht die Null enthalten dann wird die Nullhypothese $H_0$ abgelehnt.
\item [\textbf{C} \msquare] Ist $T_{D}$ h{"o}her als der kritische Wert $T_{\alpha = 5\%}$ dann wird die Nullhypothese $H_0$ abgelehnt.
\item [\textbf{D} \msquare] Anhand der berechneten Teststatistik lässt sich wie folgt eine Entscheidung treffen. Liegt der Wert über oder gleich dem Signifikanzniveau $\alpha$ dann kann die Nullhypothese abgelehnt werden.
\item [\textbf{E} \msquare] Anhand der berechneten Teststatistik lässt sich wie folgt eine Entscheidung treffen. Liegt der Wert in dem Signifikanzniveauintervall $\alpha$ dann kann die Nullhypothese abgelehnt werden.
\end{enumerate}

\section{Aufgabe \hfill (2 Punkte)}



Ein statistischer Test benötigt für die richtige Durchführung Hypothesen $H$, sonst ist der Test nicht zu interpretieren. Welche Aussage ist richtig?



\begin{enumerate}
\item [\textbf{A} \msquare] Mit der Nullhypothese $H_0$ und der Alternativehypothese $H_A$ oder $H_1$ gibt es zwei Hypothesen.
\item [\textbf{B} \msquare] Ein statistisches Hypothesenpaare gibt es. Zum einen die Nullhypothese und zum anderen die Alternativehypothese. Es ist aber nur notwendig die Alternative anzugeben, da die Nullhypothese nicht beim Testen benötigt wird.
\item [\textbf{C} \msquare] Es gibt ein statistisches Hypothesenpaar mit der Hypothese für und gegen die wissenschaftliche Fragestellung. Die Hypothesen werden $H_{pro}$ und $H_{contra}$ bezeichnet.
\item [\textbf{D} \msquare] Die Hypothesen $H_0$ und $H_A$ sind rein prosarischer Natur und bilden keinen mathematischen Hintergrund ab. In der Statistik wird die wissenschaftliche Fragestellung getestet. Daher stehen auch die verständlichen Hypothesen im Mittelpunkt der biologischen Interpretation.
\item [\textbf{E} \msquare] Es gibt ein Hypothesenset bestehend aus $k$ Hypothesen. Meistens wird die Nullhypothese $H_0$ und die Alternativhypothese $H_A$ verwendet. Wegen des Falsifikationsprinzips ist es wichtig, die bekannte falsche und unbekannte richtige Hypothese mit in das Set zu nehmen.
\end{enumerate}

\section{Aufgabe \hfill (2 Punkte)}

Die Randomisierung von Beobachtungen zu den Versuchseinheiten
ist bedeutend in der Versuchsplanung. Welche der folgenden Aussagen ist richtig?



\begin{enumerate}
\item [\textbf{A} \msquare] Randomisierung ist die direkte Folge von Strukturgleichheit. Die Strukturgleichheit erlaubt es erst von der Stichprobe auf die Grundgesamtheit zurückzuschliessen.
\item [\textbf{B} \msquare] Randomisierung erlaubt erst die Varianzen zu schätzen. Ohne eine Randomisierung ist die Berechnung von Mittelwerten und Varianzen nicht möglich. Dadurch lässt sich erst ein Experiment auswerten.
\item [\textbf{C} \msquare] Randomisierung sorgt für Strukturgleichheit und erlaubt erst von der Stichprobe auf die Grundgesamtheit zurückzuschliessen.
\item [\textbf{D} \msquare] Strukturgleichheit ist durch Randomisierung gegeben. Leider hilft die Randomisierung noch nicht um von der Stichprobe auf die Grundgesamtheit zu schließen. Deshalb wurde das Falsifikationsprinzip entwickelt.
\item [\textbf{E} \msquare] Randomisierung war bis 1952 bedeutend, wurde dann aber in Folge besserer Rechnerleistung nicht mehr verwendet. Aktuelle Statistik nutzt keine Randomisierung mehr.
\end{enumerate}

\section{Aufgabe \hfill (2 Punkte)}



In der Bio Data Science wird häufig mit sehr großen Datensätzen gerechnet. Historisch ergibt sich nun ein Problem bei der Auswertung der Daten und deren Bewertung hinsichtlich der Signifikanz. Welche Aussage ist richtig?





\begin{enumerate}
\item [\textbf{A} \msquare] Relevanz und Signifikanz haben nichts miteinander zu tun. Daher gibt es auch keinen Zusammenhang zwischen hoher Fahlzahl (n > 10000) und einem signifikanten Test. Ein Effekt ist immer relevant und somit signifikant.
\item [\textbf{B} \msquare] Big Data ist ein Problem der parametrischen Statistik. Parameter lassen sich nur auf kleinen Datensätzen berechnen, da es sich sonst nicht mehr um eine Stichprobe im engen Sinne der Statistik handelt.
\item [\textbf{C} \msquare] Mehr Fallzahl in Datensätzen bedeutet mehr signifikante Ergebnisse, da in mehr Daten auch mehr Informationen beinhaltet sind. Deshalb lohnen sich riesige Datensätze, die durch die vielen signifikanten Ergebnisse auch eine Menge an relevanten Erkenntnissen liefern.
\item [\textbf{D} \msquare] Aktuell werden immer größere Datensätze erhoben. Dadurch wird auch die Varianz immer höher was automatisch zu mehr signifikanten Ergebnissen führt.
\item [\textbf{E} \msquare] Eine große Fallzahl führt zu mehr signifikanten Ergebnissen auch bei kleinen Effekten. Daher werden fast alle Vergleich esignifikant, wenn die Fallzahl nur groß genug wird.
\end{enumerate}

\section{Aufgabe \hfill (2 Punkte)}



In einem Zuchtexperiment messen wir die Ferkel verschiedener Sauen. Die Ferkel einer Muttersau sind daher im statistischen Sinne...



\begin{enumerate}
\item [\textbf{A} \msquare] Untereinander abhängig, wenn die Mütter ebenfalls miteinander verwandt sind. Erst die Abhängigkeit 2. Grades wird in der Statistik modelliert.
\item [\textbf{B} \msquare] Die Ferkel stammen von der gleichen Sau und sind somit untereinander unabhängig.
\item [\textbf{C} \msquare] Abhängig von der Stallanlage und des Experiments können die Ferkel abhängig oder unabhängig sein. Allgmein gilt, dass Ferkel von unterschiedlichen Sauen näher miteinander verwandt sind als Ferkel von gleichen Sauen. Das Fisher-Axiom.
\item [\textbf{D} \msquare] Die Ferkel stammen vom gleichen Muttertier und haben vermutlich eine ähnlichere Varianzstruktur als die Ferkel von anderen Sauen. Die Ferkel sind untereinander über die Mutter abhängig.
\item [\textbf{E} \msquare] Untereinander stark korreliert. Die Ferkel sind von einer Mutter und sommit miteinander korreliert. Dies wird in der Statistik jedoch meist nicht modelliert.
\end{enumerate}
\section*{Statistische Tests für Gruppenvergleiche} 

\section{Aufgabe \hfill (2 Punkte)}



Nach einem Feldexperiment wollen Sie zwei Gruppen mit einem Welch t-Test vergleichen. Welche Aussage ist auch für den Student t-Test richtig?



\begin{enumerate}
\item [\textbf{A} \msquare] Der t-Test vergleicht zwei oder mehr Gruppen indem die Mittelwerte miteinander verglichen werden.
\item [\textbf{B} \msquare] Der t-Test vergleicht die Mittelwerte von zwei Gruppen.
\item [\textbf{C} \msquare] Der t-Test ist ein Vortest der ANOVA und basiert daher auf dem Vergleich von Streuungsparametern
\item [\textbf{D} \msquare] Der t-Test berechnet die Differenz von zwei Mittelwerten als Effekt und gibt eine Entscheidung, ob sich die beiden Mittelwerte \textit{jeweils} von Null unterscheiden.
\item [\textbf{E} \msquare] Der t-Test vergleicht die Mittelwerte von zwei Gruppen unter der strikten Annahme von Varianzhomogenität. Sollte keine Varianzhomogenität vorliegen, so gibt es keine Möglichkeit den t-Test in einer Variante anzuwenden.
\end{enumerate}

\section{Aufgabe \hfill (2 Punkte)}



Die folgende Abbildung enthält die Daten aus einer Studie zur Bewertung der Wirkung des Mikronährstoff Nitrat auf den Ertrag in t/ha von Mango im Vergleich zu einer Kontrolle. Der Versuch wurde in 9 Parzellen pro Gruppe durchgeführt. Welche Aussage im Bezug auf eine statistische Auswertung ist richtig?



{\centering \includegraphics[width=\maxwidth]{img/mc-testing-ttest-02-1} 

}







\begin{enumerate}
\item [\textbf{A} \msquare] Nach Betrachtung des Barplots liegt kein signifikanter Unterschied vor. Der Effekt kann nicht bei einem t-Test aus Barplots bestimmt werden.
\item [\textbf{B} \msquare] Der Effekt und die Signifikanz lassen sich nicht aus Barplots abschätzen. Höchtens der Effekt als relativer Unterschied zwischen der Höhe der Barplots. Standard ist der mediane Unterschied aus Boxplots.
\item [\textbf{C} \msquare] Es liegt ein signifikanter Unterschied vor. Der Effekt liegt bei -6.
\item [\textbf{D} \msquare] Die Barplots deuten auf keinen signifikanten Unterschied. Der Effekt liegt vermutlich bei -6 unter einer groben Abschätzung. Wir müssen aber eine ANOVA rechnen um den Effekt wirklich bestimmen zu können.
\item [\textbf{E} \msquare] Der Test deutet auf keinen signifikanten Unterschied hin. Der Effekt liegt vermutlich bei -6.
\end{enumerate}

\section{Aufgabe \hfill (2 Punkte)}




Sie rechnen einen gepaarten t-Test, da Ihre Beobachtungen verbunden sind. Welche der folgenden Aussagen ist richtig?



\begin{enumerate}
\item [\textbf{A} \msquare] Abhängige Beobachtungen müssen gesondert in einem gepaarten t-Test modelliert werden. Wenn wiederholt an dem gleichen Tier oder Pflanze gemessen wird, dann bilden wir den Quotienten zwischen den beiden Zeitpunkten. Auf den Quotienten rechnen wir den gepaarten t-Test.
\item [\textbf{B} \msquare] Der gepaarte t-Test wird genutzt, wenn die Differenzen der Beobachtungen verbunden sind und wir dadurch die Unabhäängigkeit nicht mehr vorliegen haben.
\item [\textbf{C} \msquare] Abhängige Beobachtungen müssen gesondert in einem gepaarten t-Test modelliert werden. Wenn wiederholt an dem gleichen Tier oder Pflanze gemessen wird, dann bilden wir die Differenz zwischen den beiden Zeitpunkten. Auf den Differenzen rechnen wir den gepaarten t-Test.
\item [\textbf{D} \msquare] Beim gepaarten t-Test kombinieren wir die Vorteile des Student t-Test für Varianzhomogenität mit den Vorteilen des Welch t-Test für Varianzheterogenität. Wir bilden dafür die Differenz der Einzelbeobachtungen.
\item [\textbf{E} \msquare] Der gepaarte t-Test nutzt die Varianz der Beobachtungen jeweils paarweise und bildet dafür eine verbundene Stichprobe. Dieser Datensatz $d$ dient dann zur Differenzbildung.
\end{enumerate}

\section{Aufgabe \hfill (2 Punkte)}



Nach einem Experiment mit fünf Weizensorten ergibt eine ANOVA ($p = 0.048$) einen signifikanten Unterschied für den Ertrag. Sie führen anschließend die paarweisen t-Tests für alle Vergleiche der verschiedenen Weizensorten durch. Nach der Adjustierung für multiples Testen ist kein p-Wert unter der $\alpha$-Schwelle. Sie schauen sich auch die rohen, unadjustierten p-Werte an und finden hier als niedrigsten p-Wert $p_{3-2} = 0.051$. Welche Aussage ist richtig?




\begin{enumerate}
\item [\textbf{A} \msquare] Der Fehler liegt in den t-Tests. Wenn eine ANOVA signifikant ist, dann muss zwangsweise auch ein t-Test signifikant sein.
\item [\textbf{B} \msquare] Die adjustierten p-Werte deuten in die richtige Richtung. Zusammen mit den nicht signifikanten rohen p-Werten ist von einem Fehler in der ANOVA auszugehen.
\item [\textbf{C} \msquare] Es gibt einen Fehler in der Varianzstruktur. Daher kann die ANOVA nicht richtig sein und paarweise t-Tests liefern das richtige Ergebnis.
\item [\textbf{D} \msquare] Das Beispiel kann so nicht auftreten, da die ANOVA und die t-Tests algorithmisch miteinander verschränkt sind.
\item [\textbf{E} \msquare] Das ist kein Wunder. Die ANOVA testet auf der gesamten Fallzahl und die paarweisen t-Tests verlieren immer eine oder mehr Gruppen als Fallzahl. Mit steigender Fallzahl sind mehr signifikante Unterschiede zu erwarten. Die p-Werte unterscheiden sich numerisch auch kaum.
\end{enumerate}
\section*{ANOVA}

\section{Aufgabe \hfill (2 Punkte)}





Sie führen einen Versuch mit einer Behandlung und drei Faktorleveln durch. Danach rechnen Sie eine einfaktorielle ANOVA und es ergibt sich ein $\eta^2 = 0.31$. Welche Aussage ist richtig?



\begin{enumerate}
\item [\textbf{A} \msquare] Das $\eta^2$ ist die Korrelation der ANOVA. Mit der Ausnahme, dass 0 der beste Wert ist.
\item [\textbf{B} \msquare] Das $\eta^2$ ist damit mit dem $R^2$ aus der linearen Regression zu vergleichen und beschreibt den Anteil der Varianz, der von den Behandlungsbedingungen erkl{"a}rt wird.
\item [\textbf{C} \msquare] Das $\eta^2$ beschreibt den Anteil der Varianz, der von den Behandlungsbedingungen nicht erkl{"a}rt wird. Somit der Rest an nicht erkl{"a}rbarer Varianz.
\item [\textbf{D} \msquare] Die Berechnung von $\eta^2$ ist ein Wert f{"u}r die Interaktion.
\item [\textbf{E} \msquare] Das $\eta^2$ ist ein Wert f{"u}r die G{"u}te der ANOVA. Je kleiner desto besser. Ein $\eta^2$ von 0 bedeutet ein perfektes Modell mit keiner Abweichung. Die Varianz ist null.
\end{enumerate}

\section{Aufgabe \hfill (2 Punkte)}



Sie führen ein Feldexperiment durch um das Gewicht von Lauch zu
steigern. Die Pflanzen wachsen unter einer Kontrolle und zwei verschiedenen
Behandlungsbedingungen. Nach der Berechnung einer einfaktoriellen ANOVA
ergibt sich ein $\eta^2 = 0.19$. Welche Aussage ist richtig?



\begin{enumerate}
\item [\textbf{A} \msquare] Es werden 81\% der Varianz durch die Behandlung erklärt. Das $\eta^2$ beschreibt den Anteil der Varianz, der von den unterschiedlichen Behandlungsbedingungen nicht erklärt wird.
\item [\textbf{B} \msquare] Es werden 19\% der Varianz durch den Versuch erklärt. Das $\eta^2$ beschreibt den Anteil der Varianz, der durch Fehler in der Versuchsdurchführung entsteht.
\item [\textbf{C} \msquare] Das $\eta^2$ beschreibt den Anteil der Varianz, der von den Umweltbedingungen erklärt wird. Daher werden 19\% der Varianz durch die Umweltbedingungen erklärt. Der Anteil der Varianz durch die Behandlungsgruppen ist dann 81\%.
\item [\textbf{D} \msquare] Das $\eta^2$ beschreibt den Anteil der Varianz, der von den Behandlungsbedingungen erklärt wird. Daher werden 19\% der Varianz durch die Behandlungsgruppen erklärt.
\item [\textbf{E} \msquare] Das $\eta^2$ beschreibt den Anteil der Varianz, der durch den Forschenden entsteht. Es gilt die Regel, dass ca. 70\% der Varianz eines Versuches durch die Versuchsdurchführung entstehen sollen.
\end{enumerate}

\section{Aufgabe \hfill (2 Punkte)}



Sie rechnen eine einfaktorielle ANOVA und erhalten eine Teststatistik. Nun müssen Sie diese Teststatistik interpretieren. Welche Aussage ist richtig?



\begin{enumerate}
\item [\textbf{A} \msquare] Die ANOVA berechnet die F-Statistik indem die MS des Fehlers durch die MS der Behandlung geteilt werden. Wenn die F-Statistik sich der 0 ann{"a}hert kann die Nullhypothese abgelehnt werden.
\item [\textbf{B} \msquare] Die ANOVA berechnet die T-Statistik indem den Mittelwertsunterschied der Gruppen simultan durch die Standardabweichung der Gruppen teilt. Wenn die T-Statistik h{"o}her als 1.96 ist, kann die Nullhypothese abgelehnt werden.
\item [\textbf{C} \msquare] Die ANOVA berechnet die T-Statistik aus der Multiplikation der MS Behandlung mit der MS der Fehler. Wenn die F-Statistik genau 0 ist, kann die Nullhypothese nicht abgelehnt werden.
\item [\textbf{D} \msquare] Die ANOVA berechnet die F-Statistik indem die MS des Fehlers durch die MS der Behandlung geteilt werden. Wenn die F-Statistik sich der 1 ann{"a}hert kann die Nullhypothese nicht abgelehnt werden.
\item [\textbf{E} \msquare] Die F-Statistik wird berechnet indem die MS der Behandlung durch die MS des Fehlers geteilt werden. Wenn die F-Statistik sich kaum von der Null unterscheidet kann die Nullhypothese nicht abgelehnt werden.
\end{enumerate}

\section{Aufgabe \hfill (2 Punkte)}




Viele statistische Verfahren nutzen eine Teststatistik um eine Aussage über den Zusammenhang zwischen der Grundgesamthat und der Stichprobe abzubilden. Ein statistisches Testwerkzeug ist hierbei die ANOVA. Die ANOVA rechnet dabei...



\begin{enumerate}
\item [\textbf{A} \msquare] ... den Unterschied zwischen der Varianz aus verschiedenen Behandlungsguppen und der Varianz über alle Behandlungsgruppen. Wenn die ANOVA signifikant ist, muss über einen Posthoc-Test nachgedacht werden um den signifikanten Unterschied in den Gruppen exakt zu bestimmen.
\item [\textbf{B} \msquare] ... den Unterschied zwischen der Varianz in den verschiedenen Behandlungsguppen und der Varianz in einer der Behandlungsgruppen. Wenn die ANOVA signifikant ist, muss über einen Posthoc-Test nachgedacht werden um den signifikanten Unterschied in einer der Gruppen exakt zu bestimmen.
\item [\textbf{C} \msquare] ... den Unterschied zwischen der Varianz ausgelöst durch alle Behandlungsgruppen und der Varianz aus globalen Behandlungsguppen der Kontrollen. Wenn die ANOVA nicht signifikant ist, muss ein Posthoc-Test ausgeschlossen werden.
\item [\textbf{D} \msquare] ... den Unterschied zwischen der Mittelwerte und der Varianz aus verschiedenen Behandlungsguppen. Wenn die ANOVA signifikant ist, ist bekannt welcher Vergleich konkret unterschiedlich ist.
\item [\textbf{E} \msquare] ... den Unterschied zwischen der Varianz durch verschiedene Behandlungsguppen unter der Varianz über alle Behandlungsgruppen. Wenn die ANOVA signifikant ist, kann kein Effekt $\eta^2$ bestimmt werden.
\end{enumerate}

\section{Aufgabe \hfill (2 Punkte)}



Ein Versuch wurde an 70 Tieren durchgeführt, wobei jedes Tier eine von drei Vitamin-C-Dosen (0.5, 1 und 1.5 mg/Tag) über eine von zwei Verabreichungsmethoden erhielt. Die folgende Abbildung enthält die Daten aus diesem Versuch zur Bewertung der Wirkung von Vitamin E auf das Zahnwachstum bei Hasen.  Welche Aussage ist richtig, wenn Sie eine zweifaktorielle ANOVA rechnen?



{\centering \includegraphics[width=\maxwidth]{img/mc-anova-02-a-1} 

}







\begin{enumerate}
\item [\textbf{A} \msquare] Eine mittlere bis starke Interaktion liegt vor $(p \leq 0.05)$
\item [\textbf{B} \msquare] Keine Interaktion liegt vor $(p \leq 0.05)$.
\item [\textbf{C} \msquare] Das Bestimmtheitsmaß $R^2$ ist klein.
\item [\textbf{D} \msquare] Die Koeffizienten sind positiv $(\beta_0 > 0; \beta_1 > 0)$.
\item [\textbf{E} \msquare] Keine Korrelation liegt vor $(p \geq 0.05)$.
\end{enumerate}
\section*{Multiple Gruppenvergleiche} 

\section{Aufgabe \hfill (2 Punkte)}



Sie haben folgende unadjustierten p-Werte gegeben: 0.03, 0.42, 0.34, 0.001 und 0.01. Sie adjustieren die p-Werte nach
Bonferroni. Welche Aussage ist richtig?



\begin{enumerate}
\item [\textbf{A} \msquare] Nach der Bonferroni-Adjustierung ergeben sich die adjustierten p-Werte von 0.15, 1, 1, 0.005 und 0.05. Die adjustierten p-Werte werden zu einem $\alpha$-Niveau von 5\% verglichen.
\item [\textbf{B} \msquare] Nach der Bonferroni-Adjustierung ergeben sich die adjustierten p-Werte von 0.15, 2.1, 1.7, 0.005 und 0.05. Die adjustierten p-Werte werden zu einem $\alpha$-Niveau von 5\% verglichen.
\item [\textbf{C} \msquare] Nach der Bonferroni-Adjustierung ergeben sich die adjustierten p-Werte von 0.15, 1, 1, 0.005 und 0.05. Die adjustierten p-Werte werden zu einem $\alpha$-Niveau von 1\% verglichen.
\item [\textbf{D} \msquare] Nach der Bonferroni-Adjustierung ergeben sich die adjustierten p-Werte von 0.006, 0.084, 0.068, 2e-04 und 0.002. Die adjustierten p-Werte werden zu einem $\alpha$-Niveau von 5\% verglichen.
\item [\textbf{E} \msquare] Nach der Bonferroni-Adjustierung ergeben sich die adjustierten p-Werte von 0.006, 0.084, 0.068, 2e-04 und 0.002. Die adjustierten p-Werte werden zu einem $\alpha$-Niveau von 1\% verglichen.
\end{enumerate}

\section{Aufgabe \hfill (2 Punkte)}



Auf wissenschaftlichen Postern finden Sie unter Abbildungen häufig die Abbkürzung \textit{CLD}. Für welchen statistischen Fachbegriff steht die Abbkürzung und wie interpretieren Sie ein \textit{CLD}?



\begin{enumerate}
\item [\textbf{A} \msquare] Compact letter display. Gleiche Buchstaben bedeuten, dass sich die Behandlungen unterscheiden. Daher ist das CLD sehr unintuitiv. Es wäre besser, wenn gleiche Buchstaben Gleichheit anzeigen würden. Dies ist aber leider in der statistischen Testtheorie nicht möglich.
\item [\textbf{B} \msquare] Compound letter display. Gleichheit in dem Outcomes wird durch den gleichen Buchstaben oder Symbol dargestellt. Teilweise ist die Interpretation des Verbunds (eng. compound) herausfordernd, da wir ja nach dem Unterschied suchen.
\item [\textbf{C} \msquare] Compact letter display. Gleichheit in den Behandlungen wird durch den gleichen Buchstaben oder Symbol dargestellt. Teilweise ist die Interpretation des CLD herausfordernd, da wir ja nach dem Unterschied suchen.
\item [\textbf{D} \msquare] Compact line display. Gleichheit in den Behandlungen wird durch den gleichen Buchstaben oder Symbol dargestellt. Früher wurden keine Buchstaben sondern eine durchgezogene Linie verwendet. Bei mehr als drei Gruppen funktioniert die Linie aber graphisch nicht mehr.
\item [\textbf{E} \msquare] Compact letter detection. Gleichheit in den Behandlungen wird durch den gleichen Buchstaben oder Symbol dargestellt.
\end{enumerate}

\section{Aufgabe \hfill (2 Punkte)}




Sie haben eine zweifaktorielle ANOVA gerechnet und wollen nach einem signifikanten Ergebnis in dem Gruppenfaktor einen Posthoc-Test rechnen. Welches R Paket nutzen Sie dafür und welche Eigenschaften des Paktes sind korrekt?



\begin{enumerate}
\item [\textbf{A} \msquare] Das R Paket \{hmisc\} erlaubt die Durchführung eines multiplen Gruppenvergleichs aus verschiedenen Modellen heraus. Aus einem hmisc Objekt lässt sich recht einfach das CLD erstellen und so über Barplots eine schnelle Interpration der statistischen Auswertung durchführen.
\item [\textbf{B} \msquare] Das R Paket \{ggplot\}. Wir erhalten hier sofort eine Visualisierung der Daten. Anhand der Visualisierung lässt sich eine explorative Datenanalyse durchführen, die gleichwertig zu einem Posthoc-Test ist.
\item [\textbf{C} \msquare] Da Sie für Ihre Bachelorarbeit einen Barplot mit CLD brauchen nutzen Sie das R Paket \{emmeans\} welches Ihnen schnell die notwenidigen Informationen liefert um einen Barplot zu erstelen. Die Berechnung eines CLD ist hierbei auch einfach.
\item [\textbf{D} \msquare] Das R Paket \{emmeans\} erlaubt die Durchführung eines multiplen Gruppenvergleichs. Aus einem emmeans Objekt lässt sich leider kein CLD erstellen. Dennoch ist das Paket einfach zu bedienen und wird deshalb genutzt. Die Interpretation der statistischen Auswertung wird über einen Barplot abgebildet.
\item [\textbf{E} \msquare] Das R Paket \{lm\}. Das Paket \{lm\} erstellt selbstständig Konfidenzintervalle und entsprechende p-Werte. Da wir in dem Paket nicht adjustieren müssen, ist es bei Anwendern sehr beliebt.
\end{enumerate}

\section{Aufgabe \hfill (2 Punkte)}



Bei einem multiplen Vergleich oder Posthoc Test kann es zu einer Besonderheit beim statistischen Testen kommen. Wie nennt man diese Besonderheit beim statistischen Testen und wie kann man mit ihr umgehen?



\begin{enumerate}
\item [\textbf{A} \msquare] Das globale Signifikanzniveau explodiert und erreicht Werte größer als Eins. Es kommt zu einer $\alpha$-Inflation. Dagegen kann mit der Adjustierung der $\alpha$-Werte nach Bonferroni vorgegangen werden.
\item [\textbf{B} \msquare] Beim multiplen Testen kann es zu einer $\alpha$-Inflation kommen. Das globale Signifikanzniveau liegt nicht mehr bei $5\%$ sondern weit darunter. Daher müssen die p-Werte entsprechend adjustiert werden. Hierfür gibt es verschiedene Verfahren, wobei das Verfahren zur Adjustierung der p-Werte nach Welch das bekanneste Verfahren ist.
\item [\textbf{C} \msquare] Die Adjustierung der p-Werte nach Bonferroni erlaubt es gegen die $\beta$-Inflation vorzugehen, die häufig beim multiplen Testen auftritt. Das globale Powerniveau liegt nicht mehr bei $80\%$ sondern sehr viel niedriger.
\item [\textbf{D} \msquare] Beim multiplen Testen kann es zu Varianzheterogenität kommen. Das globale Signifikanzniveau liegt nicht mehr bei $5\%$. Daher müssen die p-Werte entsprechend adjustiert werden. Das Verfahren nach Welch, bekannt aus dem t-Test, ist hier häufig anzuwenden.
\item [\textbf{E} \msquare] Beim multiplen Testen kann es zu einer $\alpha$-Inflation kommen. Das globale Signifikanzniveau liegt nicht mehr bei $5\%$ sondern sehr viel höher. Daher müssen die p-Werte entsprechend adjustiert werden. Hierfür gibt es verschiedene Verfahren, wobei das Verfahren zur Adjustierung der p-Werte nach Bonferroni das bekanneste Verfahren ist.
\end{enumerate}

\section{Aufgabe \hfill (2 Punkte)}




Sie rechnen mehrere t-Tests für einen multiplen Vergleich nachdem eine einfaktorielle ANOVA sich als signifikant herausgestellt hat. Welche Aussage im Bezug auf den Effekt ist richtig? 



\begin{enumerate}
\item [\textbf{A} \msquare] Wenn ein multipler Test gerechnet wird, dann muss der Effekt $\Delta$ adjustiert werden im Gegensatz zu den p-Werten.
\item [\textbf{B} \msquare] Beim multiplen Testen werden die Effekte der paarweisen Vergleiche ignoriert. Der Nachteil des multiplen Testens ist ja auch, dass wir am Ende keine Effekte mehr vorliegen haben. Eine ANOVA liefert hier bessere Informationen.
\item [\textbf{C} \msquare] Wenn ein multipler Test gerechnet wird, dann muss der Effekt $\Delta$ nach Bonferroni adjustiert werden. Dafür wird der Effekt mit der Anzahl an Vergleichen $k$ multipliziert. Dies geschiet analog zu den p-Werten.
\item [\textbf{D} \msquare] Wenn ein multipler Test gerechnet wird, dann muss der Effekt $\Delta$ nicht adjustiert werden. Bei einem Effekt im multiplen Testen handelt es sich um eine Wahrscheinlichkeit für das Auftreten der Nullhypothese.
\item [\textbf{E} \msquare] Wenn ein multipler Test gerechnet wird, dann muss der Effekt $\Delta$ nicht adjustiert werden im Gegensatz zu den p-Werten.
\end{enumerate}
\section*{Lineare Regression \& Korrelation}

\section{Aufgabe \hfill (2 Punkte)}



Nach der Modellierung einer Regression stellt sich die Frage, ob die Residuen approximativ einer Normalverteilung folgen. Sie können einen QQ-Plot für die visuelle Überprüfung der Annahme an die Residuen nutzen. Welche Aussage ist richtig?



{\centering \includegraphics[width=\maxwidth]{img/mc-regression-05-a-1} 

}







\begin{enumerate}
\item [\textbf{A} \msquare] Wir betrachten die Gerade und dabei insbesondere die beiden Enden der Gerade. Hier sollten die Punkte auf der Geraden liegen, dann ist die Annahme an die Normalverteilung der Residuen erfüllt. Diese Annahme ist nicht erfüllt.
\item [\textbf{B} \msquare] Die Annahme der normalverteilten Residuen ist erfüllt. Die Punkte liegen zum überwiegenden Teil nicht auf der Geraden.
\item [\textbf{C} \msquare] Wir betrachten die Gerade und dabei insbesondere die beiden Enden der Gerade. Hier sollten die Punkte auf der Geraden liegen, dann ist die Annahme an die Normalverteilung der Residuen erfüllt.
\item [\textbf{D} \msquare] Wir betrachten die Punkte. Wenn die Punkte einigermaßen gleichmäßig verteilt liegen, dann gehen wir von normalen Residuen aus.
\item [\textbf{E} \msquare] Die Annahme der normalverteilten Residuen ist nicht erfüllt. Die Punkte liegen zum überwiegenden Teil nicht auf der Geraden.
\end{enumerate}

\section{Aufgabe \hfill (2 Punkte)}



In Ihrer Abschlussarbeit wollen Sie ein kausales Modell rechnen. Jetzt stellt sich die Frage, was diese Entscheidung für Ihre Auswertung bedeutet. Welche Aussage ist richtig?



\begin{enumerate}
\item [\textbf{A} \msquare] Wir modellieren den Zusammenhang zwischen $X$ und $Y$ wenn ein kausales Modell rerechnet wird. Dabei kann der gesamte Datensatz genutzt werden. Eine Aufteilung wie in einem prädiktiven Modell ist nicht notwendig.
\item [\textbf{B} \msquare] Ein kausales Modell benötigt mindestens eine Fallzahl von über 100 Beobachtungen und darf keine fehlenden Werte beinhalten. Die Varianzkomponenten müssen homogen sein.
\item [\textbf{C} \msquare] Ein kausales Modell schliesst grundsätzlich lineare Modell aus. Es muss ein Graph gefunden werden, der alle Punkte beinhaltet. Erst dann kann das $R^2$ berechnet werden.
\item [\textbf{D} \msquare] Wenn ein kausales Modell gerechnet werden soll, dann muss zum einen ein Traingsdatensatz sowie ein Testdatensatz definiert werden. Dabei ist der Trainingsdatensatz meist 2/3 und der Testdatensatz 1/3 der Fallzahl groß. Der Testdatensatz dient zur Validierung.
\item [\textbf{E} \msquare] Wir modellieren den Zusammenhang zwischen $X$ und $Y$ wenn ein kausales Modell rerechnet wird. Dabei kann nicht der gesamte Datensatz genutzt werden. Es wird ein Trainingsdatensatz zum Trainieren des Modells benötigt.
\end{enumerate}

\section{Aufgabe \hfill (2 Punkte)}



Nach der Modellierung einer Regression stellt sich die Frage, ob die Residuen (\texttt{.resid}) gleichmäßig um die gefitte Gerade liegen. Sie können folgende Abbildung für die visuelle Überprüfung der Residuen nutzen. Welche Aussage ist richtig?



{\centering \includegraphics[width=\maxwidth]{img/mc-regression-06-a-1} 

}







\begin{enumerate}
\item [\textbf{A} \msquare] Wenn wir die Nulllinie betrachten so müssen die Punkte gleichmäßig unter der Nulllinie liegen. Unser Modell erfüllt somit nicht die Annahme von normalverteilten Residuen mit einem Mittelwert von $>0$ und einer Streuung von $s$.
\item [\textbf{B} \msquare] Die Annahme der normalverteilten Residuen ist nicht erfüllt. Ein klares Muster ist zu erkennen und/oder einige Outlier sind zu beobachten.
\item [\textbf{C} \msquare] Die Punkte müssen gleichmäßig in dem positiven Bereich liegen. Dies ist hier klar nicht der Fall. Einzelne Ausreißer können beobachtet werden. Die Analyse ist gescheitert.
\item [\textbf{D} \msquare] Die Annahme der normalverteilten Residuen ist erfüllt. Die Punkte liegen zum überwiegenden Teil auf der Diagonalen. Damit ist das Modell erfolgreich geschätzt worden.
\item [\textbf{E} \msquare] Die Annahme der normalverteilten Residuen ist erfüllt. Kein Muster ist zu erkennen und keine Outlier zu beobachten.
\end{enumerate}

\section{Aufgabe \hfill (2 Punkte)}




Welche Aussage über den Korrelationskoeffizienten $\rho$ ist richtig?




\begin{enumerate}
\item [\textbf{A} \msquare] Korrelationskoeffizienten $\rho$ liegt zwischen 0 und 1. Darüber hinaus ist der Korrelationskoeffizienten $\rho$ einheitslos und kann als Standardisierung verstanden werden.
\item [\textbf{B} \msquare] Der Korrelationskoeffizienten $\rho$ ist eine standardisierte, statistische Maßzahl, die zwischen -1 und 1 liegt. Dabei ist Korrelationskoeffizienten $\rho$ einheitslos.
\item [\textbf{C} \msquare] Der Korrelationskoeffizienten $\rho$ ist eine standardisierte, statistische Maßzahl, die zwischen 0 und 1 liegt. Dabei ist Korrelationskoeffizienten $\rho$ einheitslos. Eine Signifikanz kann nicht nachgewiesen werden.
\item [\textbf{D} \msquare] Der Korrelationskoeffizienten $\rho$ liegt zwischen -1 und 1. Darüber hinaus ist der Korrelationskoeffizienten $\rho$ als standardisierte Steigung zu verstehen, wenn eine Standardisierung durchgeführt wurde. Diese Adjustierung nach Fischer muss am Anschluß der Berechnung der Korrelation durchgeführt werden.
\item [\textbf{E} \msquare] Der Korrelationskoeffizienten $\rho$ ist eine veraltete Darstellungsform von Effekten in der linearen Regression und wird wie das $\eta^2$ aus der ANOVA interpretiert. Der Korrelationskoeffizienten $\rho$ beschreibt den Anteil an erklärter Varianz durch die Regression.
\end{enumerate}

\section{Aufgabe \hfill (2 Punkte)}



In einer lineren Regression kann es vorkommen, dass der Effekt repräsentiert durch den $\beta$ Koeffizienten nicht so richtig von der Größenordnung zu dem p-Wert passen will. So liefert eine Untersuchung des Einflusses von der $Fe_3O_4$-Konzentration in [$\mu g$] im Wasser auf das Trockengewicht in [$kg$] an Erbsen folgende Effekte und p-Werte: $0.00032$ als p-Wert und einen $\beta_{Fe_3O_4}$ Koeffizienten von $7.4\times 10^{-6}$. Welche Aussage ist richtig?




\begin{enumerate}
\item [\textbf{A} \msquare] Die Einheit der $Fe_3O_4$-Konzentration ist zu klein gewählt. Die Erhöhung der $Fe_3O_4$-Konzentration um 1 Einheit führt nur zu einem sehr winzigen Anstieg von $\beta_{Fe_3O_4}$ im Gewicht der Wasserlinsen. Die Einheit [$\mu g$] muss besser gewählt werden.
\item [\textbf{B} \msquare] Wenn der Effekt $\beta_{Fe_3O_4}$ winzig ist, dann kann es an einer falsch gewählten Einheit liegen. Der Anstieg von einer Einheit in $X$ führt ja zu einer Änderung von $\beta_{Fe_3O_4}$ in $x$. Wir müssen daher die Einheit von $y$ entsprechend anpassen.
\item [\textbf{C} \msquare] Das Gewicht und die $Fe_3O_4$-Konzentration korrelieren sehr stark, deshalb wird der $\beta_{Fe_3O_4}$ Koeffizient sehr klein. Mit einer ANOVA kann für die Korrelation korrigiert werden und der Effektschätzer passt dann zum p-Wert.
\item [\textbf{D} \msquare] Die Fallzahl ist zu hoch angesetzt. Je höher die Fallzahl ist, desto kleiner ist die Teststatistik und damit ist dann auch der $p$-Wert sehr klein. Es sollte über eine Reduzierung der Fallzahl nachgedacht werden. Dann sollte der Effekt zum p-Wert passen.
\item [\textbf{E} \msquare] Die Fallzahl ist zu klein angesetzt. Je kleiner die Fallzahl ist, desto höher ist die Teststatsitik und damit auch der $p$-Wert kleiner. Wir brauchen also mehr Fallzahl um den geringen Effekt noch signifikant zu krigen.
\end{enumerate}

\section{Aufgabe \hfill (2 Punkte)}



Sie wollen nach der explorativen Datenanalyse (EDA) Ihre Daten in der Abschlussarbeit auswerten. Nach einiger Rechereche finden Sie heraus, dass Sie zuerst die Daten mit der Funktion \texttt{lm()} in \Rlogo modellieren müssen. Welche Anwendung folgt drauf?





\begin{enumerate}
\item [\textbf{A} \msquare] Ist die Einflussvariable $X$ ein Faktor so werden die Gruppenmittelwerte geschätzt und eine anschließende ANOVA sowie multipler Gruppenvergleich mit \{emmeans\} ist möglich. Dennoch muss zuerst ein lineares Modell mit der Funktion \texttt{lm()} in \Rlogo gerechnet werden.
\item [\textbf{B} \msquare] Die Funktion \texttt{lm()} berechnet die Varianzstruktur für eine ANOVA. Dannach kann dann über eine explorative Datenalayse nochmal eine Signifikanz berechnet werden. Sollte vor der Verwendung der Funktion \texttt{lm()} schon eine EDA gerechnet worden sein, so ist die Analyse wertlos.
\item [\textbf{C} \msquare] Ist die Einflussvariable $X$ numerisch so werden die Gruppenmittelwerte geschätzt und eine anschließende ANOVA sowie multipler Gruppenvergleich mit \{emmeans\} ist möglich.
\item [\textbf{D} \msquare] Die Funktion \texttt{lm()} in \Rlogo wird klassischerweise für die nicht-lineare Regression genutzt. Ist die Einflussvariable $X$ numerisch so werden die Gruppenmittelwerte geschätzt.
\item [\textbf{E} \msquare] Neben der klassichen Verwendung der Funktion \texttt{lm()} in der linearen Regression kann auch ein Gruppenvergleich gerechnet werden. Dafür müssen aber alle Faktoren aus den Daten entfernt und numerishc umgewandelt werden. Dann kann das R Paket \{emmeans\} genutzt werden um die Korrelation zu berechnen. Eine Adjustierung ist dann nicht mehr notwendig.
\end{enumerate}

\section{Aufgabe \hfill (2 Punkte)}



Welche Aussage über das \textit{generalisierte lineare Modell (GLM)} ist richtig?




\begin{enumerate}
\item [\textbf{A} \msquare] In \Rlogo ist mit dem \textit{generalisierten linearen Modell (GLM)} eine Modellierung implementiert, die die Poissonverteilung für Zähldaten oder die Binomialverteilung für 0/1-Daten modellieren kann. Weitere Modellierungen sind in \Rlogo auch mit zusätzlich geladenen Paketen nicht möglich.
\item [\textbf{B} \msquare] Das \textit{generalisierte lineare Modell (GLM)} erlaubt auch weitere Verteilungsfamilien für das $Y$ bzw. das Outcome in einer linearen Regression zu wählen.
\item [\textbf{C} \msquare] Dank dem \textit{generalisierten linearen Modell (GLM)} können auch andere Verteilungsfamilien -- außer die Normalverteilung -- mit einer linearen Regression modelliert werden. Dafür werden alle Verteilungen in eine Normalverteilung überführt und anschließend standardisiert.
\item [\textbf{D} \msquare] Das GLM erlaubt auch nicht normalverteilte Residuen in der Schätzung der Regressionsgrade.
\item [\textbf{E} \msquare] Das \textit{generalisierte lineare Modell (GLM)} erlaubt auch weitere Verteilungsgruppen für das $X$ bzw. die Einflussvariablen in einer linearen Regression zu wählen.
\end{enumerate}

\section{Aufgabe \hfill (2 Punkte)}



In einer Studie wollen Sie den Effektschätzer Risk ratio berechnen. Sie finden in Ihrem Experiment zur Behandlung von Klaueninfektionen bei Schafe in 3 Tieren Erkrankung der Klauen vor. 8 Tiere sind gesund. Welche Aussage ist richtig?



\begin{enumerate}
\item [\textbf{A} \msquare] Das Verhältnis der Chancen Risk ratio ergibt ein Chancenverhältnis von 0.27. Wir sind an der Chance krank zu sein interessiert.
\item [\textbf{B} \msquare] Das Verhältnis der Anteile Risk ratio ergibt ein Anteilsverhältnis von 0.27. Wir sind am Anteil der Kranken interessiert.
\item [\textbf{C} \msquare] Es ergibt sich ein Risk ratio von 0.38, da es sich um ein Anteil handelt.
\item [\textbf{D} \msquare] Es ergibt sich ein Risk ratio von 0.27, da es sich um eine Chancenverhältnis handelt.
\item [\textbf{E} \msquare] Das Verhältnis von Chancen Risk ratio ergibt ein Chancenverhältnis von 0.38.
\end{enumerate}
    
% -----------------------------------------------------------------------
\clearpage
% -----------------------------------------------------------------------
\part{Programmieren in R}
% -----------------------------------------------------------------------

\section{Aufgabe \hfill (9 Punkte)}



 
%% --------------------------------------------------------------------
\begin{minipage}[t]{0.5\textwidth}
\includegraphics[width = 1.3cm]{/Users/kruppajo/work/GitHub/exam/avatare/Nilufar.png}
\end{minipage}
\begin{minipage}[t]{0.5\textwidth}
\hfill
\href{https://www.youtube.com/playlist?list=PLe51bCp9JvEFUnFqaJG5aRmON9i1ZbOYC}{\includegraphics[width = 2cm]{img/youtube}}
\end{minipage}
\vspace{-3ex}
%% --------------------------------------------------------------------



\paragraph{Grundlegende Kenntnisse der Programierung in \Rlogo}

Nilufar muss ihrer Abschlussarbeit mit \Rlogo arbeiten. Deshalb sitzt sie jetzt mit Ihnen zusammen und hat einige Fragen zu den Grundlagen in \Rlogo an Sie! Na dann wollen Sie mal helfen. Immerhin will ihr Betreuer, dass \Rlogo genutzt wird.\\[1Ex]

Nilufar: \textit{Ich habe gehört, dass es Vorteile gibt \Rlogo zu nutzen. Nenne mir mal einen Vorteil!} \textbf{(1 Punkt)}\\[1ex]
Sie antworten:\\[2Ex]

Nilufar: \textit{Wie sieht der Zuweisungs-Operator aus und was ist seine Funktion? Gerne mit Beispiel!} \textbf{(1 Punkt)}\\[1ex]
Sie antworten:\\[2Ex]

Nilufar: \textit{Jetzt sehe ich wieder diese Tilde ($\sim$) in R. Wo brauchen wir diese denn nochmal?} \textbf{(1 Punkt)}\\[1ex]
Sie antworten:\\[2Ex]

Nilufar: \textit{Ich verstehe den Unterschied zwischen \texttt{library()} und \texttt{Packages} nicht. Warum gibt es die?} \textbf{(1 Punkt)}\\[1ex]
Sie antworten:\\[2Ex]

Nilufar: \textit{Ich habe den Namen der Funktion, die intern Daten speichert, vergessen. Was waren da nochmal die Vorteile?} \textbf{(1 Punkt)}\\[1ex]
Sie antworten:\\[2Ex]

Nilufar: \textit{In R hat sich doch mein Datensatz geändert. Warum ist er jetzt wieder so wie vorher, wenn ich den Datensatz wieder aufrufe? Was mache ich falsch?} \textbf{(1 Punkt)}\\[1ex]
Sie antworten:\\[2Ex]

Nilufar: \textit{In R gibt es Objekte, Wörter und Funktionen. Wie unterscheiden sich diese voneinander?} \textbf{(1 Punkt)}\\[1ex]
Sie antworten:\\[2Ex]

Nilufar: \textit{Wenn ich was in \Rlogo machen möchte, dann lade ich nochmal welche zwei \Rlogo Pakete sehr häufig?} \textbf{(1 Punkt)}\\[1ex]
Sie antworten:\\[2Ex]

Nilufar: \textit{Wie sieht der Pipe-Operator aus und was ist seine Funktion? Gerne mit Beispiel!} \textbf{(1 Punkt)}\\[1ex]
Sie antworten:\\[2Ex] 
\clearpage
% -----------------------------------------------------------------------

\section{Aufgabe \hfill (9 Punkte)}



 
%% --------------------------------------------------------------------
\begin{minipage}[t]{0.5\textwidth}
\includegraphics[width = 1.3cm]{/Users/kruppajo/work/GitHub/exam/avatare/Yuki.png}
\end{minipage}
\begin{minipage}[t]{0.5\textwidth}
\hfill
\href{https://www.youtube.com/playlist?list=PLe51bCp9JvEFUnFqaJG5aRmON9i1ZbOYC}{\includegraphics[width = 2cm]{img/youtube}}
\end{minipage}
\vspace{1ex}
%% --------------------------------------------------------------------



\paragraph{Fortgeschrittene Kenntnisse der Programierung in \Rlogo}

'Hm...am Ende ist dann \Rlogo eigentlich gar nicht so schwer, wenn ich Hilfe habe.', meint  Yuki stolz und lacht Sie an. Nur leider kennt er sich überhaupt nicht mit \Rlogo aus! Das heißt, Sie müssen hier einmal Rede und Antwort stehen und helfen. Sonst wird es für Yuki dann in seiner Hausarbeit nichts mit der Auswertung und Abgabe. Das kann auch keine Lösung für Yuki und Sie sein. Immerhin haben Sie schon so viel gelernt.\\[1Ex]

Yuki fragt: \textit{Wenn ich zwei Faktoren $f_1$ und $f_2$ habe, wie spezifiziere ich nochmal eine Interaktion in einem Modell? \textbf{(1 Punkt)}}\\[1ex]
Sie antworten:\\[2Ex]

Yuki fragt: \textit{Viele Abbildungen gehen ja auch in Excel, aber \texttt{\{ggplot\}} ist schon besser. Wie verbindet \texttt{\{ggplot\}} eigentlich nochmal die einzelnen Ebenen? \textbf{(1 Punkt)}}\\[1ex]
Sie antworten:\\[2Ex]

Yuki fragt: \textit{Wenn ich Daten in R mit Gruppen eingelesen habe, welche Funktion nutze ich dann meistens als erstes und warum muss ich das machen? Was muss ich da machen? \textbf{(2 Punkte)}}\\[1ex]
Sie antworten:\\[2Ex]

Yuki fragt: \textit{Oh, wie baue ich mir nochmal die Daten in R? Wie heißt das Dateiformat? Gerne einmal mit Beispiel! \textbf{(1 Punkt)}}\\[1ex]
Sie antworten:\\[2Ex]

Yuki fragt: \textit{Man kann doch die Funktion \texttt{emmeans()} von Varianzhomogenität auf Varianzheterogenität umstellen. Wie ging das noch gleich? \textbf{(1 Punkt)}}\\[1ex]
Sie antworten:\\[2Ex]

Yuki fragt: \textit{Was muss ich bei der Eingabe eines Datums in Excel beachten, wenn ich später die Exceldatei in R einlesen will? Wie lautet das Format? \textbf{(1 Punkt)}}\\[1ex]
Sie antworten:\\[2Ex]

Yuki fragt: \textit{Okay, und für was brauche ich nochmal die R Pakete  \texttt{\{emmeans\}}, \texttt{\{ggplot\}} und \texttt{\{readxl\}}? \textbf{(2 Punkte)}}\\[1ex]
Sie antworten:\\[2Ex]



 
\clearpage
% -----------------------------------------------------------------------
\part{Deskriptive Statistik \& Explorative Datenanalyse}
% -----------------------------------------------------------------------

\section{Aufgabe \hfill (8 Punkte)}

\textit{Geben Sie grundsätzlich Formeln und Rechenweg zur Lösung der Teilaufgaben mit an!} \\[1Ex]
 

 
%% --------------------------------------------------------------------
\begin{minipage}[t]{0.5\textwidth}
\includegraphics[width = 1.3cm]{/Users/kruppajo/work/GitHub/exam/avatare/Jonas.png}
\end{minipage}
\begin{minipage}[t]{0.5\textwidth}
\hfill
\href{https://youtu.be/t0WYa_LVc5o}{\includegraphics[width = 2cm]{img/youtube}}
\end{minipage}
\vspace{-3ex}
%% --------------------------------------------------------------------



\paragraph{Zerforschen des Barplots}

Eine echte Herausforderung für ihn war schon immer die Erschöpfung gewesen. Ein leidiges Lied. Deshalb gilt anschauen, was andere vor einem gemacht haben. Für Jonas ist es eine Möglichkeit schneller ans Ziel zu gelangen. Jonas soll in seiner Hausarbeit Maiss untersuchen. Die Behandlung in seiner Hausarbeit werden verschiedene Lichtstufen ($none$, $200lm$ und $600lm$) sein. Erheben wird Jonas als Endpunkt ($Y$) \textit{Proteingehalt} benannt als \textit{protein} in seiner Exceldatei. Von seiner Betreuer erhält er nun folgende Abbildung von Barplots, die er erstmal zur Übung nachbauen soll, bevor er mit dem eigentlichen Versuch beginnt.



{\centering \includegraphics[width=\maxwidth]{img/barplot-02-1} 

}




Leider kennt sich Jonas mit der Erstellung von Barplots in \Rlogo nicht aus. Deshalb braucht er bei der Visualisierung Ihre Hilfe!

\begin{enumerate}
\item Formulieren Sie die wissenschaftliche Fragestellung! \textbf{(1 Punkt)}
\item Erstellen Sie eine Tabelle mit den statistischen Maßzahlen der drei Barplots! \textit{Beachten Sie die korrekte Darstellungsform der statistischen Maßzahlen!} \textbf{(3 Punkte)}
\item Erstellen Sie einen beispielhaften Datensatz im \Rlogo üblichen Format, aus dem die drei Barplots \textit{möglicherweise} erstellt wurden! \textbf{(2 Punkte)}
\item Kann Jonas einen Unterschied zwischen den Behandlungen erwarten? Begründen Sie Ihre Antwort! \textbf{(2 Punkte)}
\end{enumerate} 
\clearpage
% -----------------------------------------------------------------------

\section{Aufgabe \hfill (8 Punkte)}

\textit{Geben Sie grundsätzlich Formeln und Rechenweg zur Lösung der Teilaufgaben mit an!} \\[1Ex]
 

 
%% --------------------------------------------------------------------
\begin{minipage}[t]{0.5\textwidth}
\includegraphics[width = 1.3cm]{/Users/kruppajo/work/GitHub/exam/avatare/Alex.png}
\end{minipage}
\begin{minipage}[t]{0.5\textwidth}
\hfill
\href{https://youtu.be/vXnLttRL_VI}{\includegraphics[width = 2cm]{img/youtube}}
\end{minipage}
\vspace{-3ex}
%% --------------------------------------------------------------------



\paragraph{Visualisierung des Barplots}


Alex steht vor einem ersten Problem, denn wenn es nach seiner Betreuerin geht, soll er in einem einem Freilandversuch Erbsen auswertet. Soweit eigentlich alles passend. Besser wäre was anderes gewesen. Alex liebt Starcraft. Darin kann er sich wirklich verlieren und immer wieder neu begeistern. Die Behandlung waren verschiedene Bewässerungstypen ($low$, $mid$ und $high$). In seiner Exceldatei hat er den Outcome ($Y$) \textit{Frischegewicht} als \textit{freshmatter} aufgenommen. Nun soll Alex die Daten eimal als Barplots in einer Präsentation visualisieren, damit seiner Betreuerin wieder klar wird, was er eigentlich nochmal gemacht hat und was für ein Ergbnis in einem statistischen Test zu erwarten wäre. Wäre da nicht noch etwas. Alex und die Gefälligkeit, eine unendliche Geschichte mit kniffeligen Wendungen. Aber egal. Alex will später nochmal raus um zu Laufen. Druck ablassen, dass muss er auch.

\begin{table}[!h]
\centering
\begin{tabular}{cc}
\toprule
treatment & freshmatter\\
\midrule
high & 43.8\\
mid & 28.0\\
mid & 39.1\\
mid & 31.4\\
low & 4.9\\
\addlinespace
high & 34.6\\
low & 26.2\\
low & 33.0\\
high & 39.6\\
high & 41.9\\
\addlinespace
mid & 48.3\\
low & 19.9\\
\bottomrule
\end{tabular}
\end{table}



Leider kennt sich Alex mit der Erstellung von Barplots nicht aus. Deshalb braucht er bei der Visualisierung Ihre Hilfe!

\begin{enumerate}
\item Formulieren Sie die wissenschaftliche Fragestellung! \textbf{(1 Punkt)}
\item Zeichnen Sie in \textit{einer} Abbildung die Barplots für die Behandlung von Erbsen! Beschriften Sie die Achsen entsprechend!\textbf{(4 Punkte)}
\item Beschriften Sie \textit{einen} Barplot mit den gängigen statistischen Maßzahlen! \textbf{(2 Punkte)}
\item Wenn Alex \textit{keinen Effekt} zwischen den Behandlungen von Erbsen erwarten würde, wie sehen dann die Barplots aus? \textit{Antworten Sie mit einer Skizze der Barplots!}
  \textbf{(1 Punkt)}
\end{enumerate} 
\clearpage
% -----------------------------------------------------------------------

\section{Aufgabe \hfill (9 Punkte)}

\textit{Geben Sie grundsätzlich Formeln und Rechenweg zur Lösung der Teilaufgaben mit an!} \\[1Ex]
 

 
%% --------------------------------------------------------------------
\begin{minipage}[t]{0.5\textwidth}
\includegraphics[width = 1.3cm]{/Users/kruppajo/work/GitHub/exam/avatare/Tina.png}
\end{minipage}
\begin{minipage}[t]{0.5\textwidth}
\hfill
\href{https://youtu.be/Xf0yE-o7bEU}{\includegraphics[width = 2cm]{img/youtube}}
\end{minipage}
\vspace{-3ex}
%% --------------------------------------------------------------------



\paragraph{Zerforschen des Boxplots}

Tina steht vor einem ersten Problem, denn wenn es nach ihrem Betreuer geht, soll sie in einem einem Versuch in einer Klimakammer Maiss auswertet. Soweit eigentlich alles passend. Besser wäre was anderes gewesen. Tina liebt Astronomie. Darin kann sie sich wirklich verlieren und immer wieder neu begeistern. Das heißt erstmal überlegen für Tina. Aus den Boxen wummert Tocotronic und ihr Mund ist verklebt von Katjes. 'Herrlich', denkt Tina. Die Behandlung werden verschiedene Bewässerungstypen ($low$, $mid$ und $high$) sein. In ihrer Exceldatei wird sie den Endpunkt ($Y$) \textit{Proteingehalt} als \textit{protein} aufnehmen. Vorab soll Tina aber eimal die folgenden Boxplots ihrem Betreuer nachbauen, damit sie den \Rlogo Code schonmal für später vorliegen hat. Damit geht das Problem schon los. Eine echte Herausforderung für sie war schon immer die Wut gewesen. Ein leidiges Lied.



{\centering \includegraphics[width=\maxwidth]{img/boxplot-02-zer-1} 

}




Leider kennt sich Tina mit der Erstellung von Boxplots in \Rlogo nicht aus. Deshalb braucht sie bei der Visualisierung Ihre Hilfe!

\begin{enumerate}
\item Erstellen Sie eine Tabelle mit den statistischen Maßzahlen aus der obigen Abbildung der drei Boxplots! \textit{Beachten Sie die korrekte Darstellungsform der statistischen Maßzahlen!} \textbf{(3 Punkte)}
\item Beschriften Sie \textit{einen} der Boxplots mit den gängigen statistischen Maßzahlen! \textbf{(2 Punkte)}
\item Erstellen Sie einen beispielhaften Datensatz, aus dem die drei Boxplots \textit{möglicherweise} erstellt wurden, im \Rlogo üblichen Format! \textbf{(2 Punkte)}
\item Kann Tina einen Unterschied zwischen den Behandlungen erwarten? Begründen Sie Ihre Antwort! \textbf{(2 Punkte)}
\end{enumerate} 
\clearpage
% -----------------------------------------------------------------------

\section{Aufgabe \hfill (9 Punkte)}

\textit{Geben Sie grundsätzlich Formeln und Rechenweg zur Lösung der Teilaufgaben mit an!} \\[1Ex]
 

 
%% --------------------------------------------------------------------
\begin{minipage}[t]{0.5\textwidth}
\includegraphics[width = 1.3cm]{/Users/kruppajo/work/GitHub/exam/avatare/Paula.png}
\end{minipage}
\begin{minipage}[t]{0.5\textwidth}
\hfill
\href{https://youtu.be/0xc0jIPeiyw}{\includegraphics[width = 2cm]{img/youtube}}
\end{minipage}
\vspace{-3ex}
%% --------------------------------------------------------------------



\paragraph{Visualisierung des Boxplots}

Boxplots sind bedeutend in der Darstellung von wissenschaftlichen Ergebnissen. Leider hat sich Paula nicht gemerkt, welche statistischen Maßzahlen für einen Boxplot erhoben werden müssen. Besser wäre was anderes gewesen. Am Ende dann doch besser Harry Potter. Wunderbar. Eine echte Ablenkung für Paula. Das ist in soweit doof, da nach ihrem Betreuer nun Boxplots aus ihren Daten gebaut werden sollen, bevor es mit dem statistischen Testen weitergeht. Anhand von Boxplots lässt sich eine Aussage über die Normalverteilung von $Y$ treffen. Die Behandlung für Brokoli waren verschiedene Genotypen ($AA$ und $BB$). Erfasst wurde von Paula als Endpunkt ($Y$) \textit{Proteingehalt}. Paula hat dann \textit{protein} in ihrer Exceldatei eintragen. Aber nur in passender Atmospäre! Hm, lecker Smarties und dazu dann im Hintergrund Jagd auf roter Oktober laufen lassen.

\begin{table}[!h]
\centering
\begin{tabular}{cc}
\toprule
treatment & drymatter\\
\midrule
AA & 49.3\\
BB & 21.7\\
BB & 29.8\\
AA & 47.8\\
BB & 21.5\\
\addlinespace
BB & 34.6\\
AA & 40.6\\
AA & 46.1\\
BB & 19.6\\
AA & 45.8\\
\addlinespace
AA & 44.5\\
AA & 42.1\\
BB & 26.0\\
BB & 35.7\\
AA & 49.1\\
\bottomrule
\end{tabular}
\end{table}



Leider kennt sich Paula mit der Erstellung von Boxplots nicht aus. Deshalb braucht sie bei der Visualisierung Ihre Hilfe!

\begin{enumerate}
\item Zeichnen Sie in \textit{einer} Abbildung die beiden Boxplots für die zwei Behandlungen von Brokoli! Beschriften Sie die Achsen entsprechend! \textbf{(5 Punkte)} 
\item Wie ist Ihr Vorgehen, wenn Sie eine \textit{gerade} Anzahl an
  Beobachtungen pro Gruppe haben? \textbf{(1 Punkt)}
\item Beschriften Sie \textit{einen} der beiden Boxplots mit den gängigen
  statistischen Maßzahlen! \textbf{(2 Punkte)}
\item Wenn Sie \textit{keinen Effekt} zwischen den Behandlungen von
  Brokoli erwarten würden, wie sehen dann die beiden Boxplots aus?
  \textit{Antworten Sie mit einer Skizze der Boxplots!}
  \textbf{(1 Punkt)}
\end{enumerate} 
\clearpage
% -----------------------------------------------------------------------

\section{Aufgabe \hfill (8 Punkte)}

\textit{Geben Sie grundsätzlich Formeln und Rechenweg zur Lösung der Teilaufgaben mit an!} \\[1Ex]
 

 
%% --------------------------------------------------------------------
\begin{minipage}[t]{0.5\textwidth}
\includegraphics[width = 1.3cm]{/Users/kruppajo/work/GitHub/exam/avatare/Mark.png}
\end{minipage}
\begin{minipage}[t]{0.5\textwidth}
\hfill
\href{https://youtu.be/aXvxGC4YLqk}{\includegraphics[width = 2cm]{img/youtube}}
\end{minipage}
\vspace{-3ex}
%% --------------------------------------------------------------------



\paragraph{Visualisierung des Histogramm für kategoriale Daten}

In seiner Hausarbeit möchte Mark gerne die Daten aus einem Feldexperiment mit Lauch in einem Histogramm darstellen. Das Histogramm erlaubt ihm dabei Rückschlüsse auf die Verteilung über das Outcome ($Y$) zu treffen. Mark schmeißt noch eine Handvoll Marzipankugeln in seinen Rachen. Im Hintergrund klirrt leise der Spiegel zum Sound von Andrea Berg. In seinem Experiment hat Mark die seltsamen Verdickungen gezählt. Es wäre einfacher, wenn da nicht noch was wäre. Wenn die Unsicherheit nicht wäre, ja dann wäre wohl vieles möglich für Mark! Aber so.. Mark streichelt liebevoll der Hamster. Der Kopf ist in seinem Schloß vergraben um den Klang von Andrea Berg zu dämpfen.

\begin{center}
Die seltsamen Verdickungen: 5, 3, 3, 5, 3, 6, 7, 6, 3, 6, 8, 4, 3, 5, 3, 6, 7, 3, 5, 5, 1, 5, 1, 8, 2, 2, 4, 5, 4, 4
\end{center}

Leider kennt sich Mark mit der Erstellung von Histogrammen überhaupt nicht aus. Deshalb braucht er bei der Erstellung Ihre Hilfe!

\begin{enumerate}
\item Zeichen Sie ein Histogramm um die Verteilung der Daten zu visualisieren! (\textbf{3 Punkte})
\item Beschriften Sie die Achsen der Abbildung! (\textbf{2 Punkte})
\item Ergänzen Sie die absoluten und relativen Häufigkeiten in der
  Abbildung! \textbf{(1 Punkt)}
\item Berechnen Sie aus den Daten die \textit{Wahrscheinlichkeit}
  mehr als die Anzahl 4 zu beobachten! \textbf{(1
    Punkt)}
\item Berechnen Sie aus den Daten die \textit{Chance} mehr
  als die Anzahl 4 zu beobachten! \textbf{(1 Punkt)}
\end{enumerate}

 
\clearpage
% -----------------------------------------------------------------------

\section{Aufgabe \hfill (8 Punkte)}

\textit{Geben Sie grundsätzlich Formeln und Rechenweg zur Lösung der Teilaufgaben mit an!} \\[1Ex]
 

 
%% --------------------------------------------------------------------
\begin{minipage}[t]{0.5\textwidth}
\includegraphics[width = 1.3cm]{/Users/kruppajo/work/GitHub/exam/avatare/Paula.png}
\end{minipage}
\begin{minipage}[t]{0.5\textwidth}
\hfill
\href{https://youtu.be/ORHSPTCdfeY}{\includegraphics[width = 2cm]{img/youtube}}
\end{minipage}
\vspace{-3ex}
%% --------------------------------------------------------------------



\paragraph{Visualisierung des Histogramm für kontinuierliche Daten}

In einem Gespräch mit ihrem Betreuer wird Paula gebeten seine Daten aus einem Leistungssteigerungsversuch mit Puten in einem Histogramm darzustellen. Aus den Boxen wummert White Lies und ihr Mund ist verklebt von Smarties. 'Herrlich', denkt Paula. In ihrem Experiment hat er die mittleren auffälligen Hautflecken erst fotographiert und dann ausgezählt. Laut ihrem Betreuer soll das Histogramm helfen, die Verteilung der die mittleren auffälligen Hautflecken zu bestimmen.  Es wäre einfacher, wenn da nicht noch was wäre. Eine echte Herausforderung für sie war schon immer der Perfektionismus gewesen. Ein leidiges Lied. Paula nickt im Takt von White Lies und bemerkt dabei gar nicht was die Ratte schon wieder anstellt.

\begin{center}
Die mittleren auffälligen Hautflecken: 8.7, 10.4, 11.3, 13.5, 11.3, 8.6, 15, 10.1, 9.9, 8.1, 8.7, 12.4, 10.1, 10.8, 10.2, 10.1, 8.5, 12.6, 10, 10.9, 6.6, 6.9
\end{center}

Leider kennt sich Paula mit der Erstellung von Histogrammen überhaupt nicht aus. Deshalb braucht sie bei der Erstellung Ihre Hilfe!

\begin{enumerate}
\item Zeichen Sie ein Histogramm um die Verteilung der Daten zu visualisieren! (\textbf{3 Punkte})
 \item Erläutern Sie Ihr Vorgehen um ein Histogramm für kontinuierliche Daten zu zeichnen!  (\textbf{2 Punkte})
\item Beschriften Sie die Achsen der Abbildung! (\textbf{2 Punkte})
\item Ergänzen Sie die relativen Häufigkeiten in der Abbildung! \textbf{(1 Punkt)}  
\end{enumerate}

 
\clearpage
% -----------------------------------------------------------------------

\section{Aufgabe \hfill (10 Punkte)}

\textit{Geben Sie grundsätzlich Formeln und Rechenweg zur Lösung der Teilaufgaben mit an!} \\[1Ex]
 

 
%% --------------------------------------------------------------------
\begin{minipage}[t]{0.5\textwidth}
\includegraphics[width = 1.3cm]{/Users/kruppajo/work/GitHub/exam/avatare/Yuki.png}
\end{minipage}
\begin{minipage}[t]{0.5\textwidth}
\hfill
\href{https://youtu.be/VAqiUdV4WQ0}{\includegraphics[width = 2cm]{img/youtube}}
\end{minipage}
\vspace{-3ex}
%% --------------------------------------------------------------------




\paragraph{Visualisierung des Scatterplots}

Wenn es nach Yuki ginge, wäre er schon längst fertig mit seiner Hausarbeit. Geht es aber nicht. Aus den Boxen wummert London Grammar und sein Mund ist verklebt von Reese's Peanut Butter Cups. 'Herrlich', denkt Yuki. In seiner Hausarbeit hatte er ein Feldexperiment im Oldenburger Land durchgeführt. Nach der Meinung seinem Betreuer sieht das jedoch etwas anders aus. Jetzt soll er doch noch eine explorative Datenanalyse für den Zusammenhang zwischen durchschnittlichen Anteil an Ton [\%/l] und Trockengewicht [kg/ha] in Kartoffeln durchführen. Wie nervig! Yuki und die Faulheit, eine unendliche Geschichte mit kniffeligen Wendungen. Da zwei kontinuierliche Variablen vorliegen, geht die explorative Datenanalyse leider nicht mit Boxplots oder Barplots. Dann was anderes. Irgendwie komisch, wenn er Matrix anmacht, dann ist das Minischwein eigentlich sofort vor dem Bildschirm und starrt hinein.

\begin{table}[!h]
\centering
\begin{tabular}{cc}
\toprule
Durchschnittlichen Anteil an Ton [\%/l] & Trockengewicht [kg/ha]\\
\midrule
18.5 & 16.1\\
23.1 & 23.3\\
20.0 & 17.7\\
21.5 & 20.8\\
19.9 & 18.4\\
\addlinespace
23.9 & 19.3\\
15.9 & 18.7\\
19.0 & 17.8\\
21.1 & 18.1\\
21.5 & 17.0\\
\bottomrule
\end{tabular}
\end{table}



Leider kennt sich Yuki mit der Erstellung einer explorativen Datenanalyse für kontinuierliche Daten überhaupt nicht aus. Deshalb braucht er bei der Erstellung Ihre Hilfe!

\begin{enumerate}
\item Erstellen Sie eine Visualisierung für die Datentabelle. Beschriften Sie
  die Achsen entsprechend! \textbf{(4 Punkte)}
\item Schätzen Sie eine Gerade durch die Punkte! \textbf{(1 Punkt)}
\item Beschriften Sie die Gerade mit den gängigen statistischen Maßzahlen! Geben Sie die numerischen Zahlenwerte mit an! \textbf{(3 Punkte)}
\item Wenn \textit{ein} Effekt von $x$ auf $y$ vorhanden wäre, wie würde die Gerade verlaufen und welche Werte würden die statistischen Maßzahlen annehmen? \textbf{(2 Punkt)}
\end{enumerate} 
\clearpage
% -----------------------------------------------------------------------

\section{Aufgabe \hfill (10 Punkte)}

\textit{Geben Sie grundsätzlich Formeln und Rechenweg zur Lösung der Teilaufgaben mit an!} \\[1Ex]
 

 
%% --------------------------------------------------------------------
\begin{minipage}[t]{0.5\textwidth}
\includegraphics[width = 1.3cm]{/Users/kruppajo/work/GitHub/exam/avatare/Mark.png}
\end{minipage}
\begin{minipage}[t]{0.5\textwidth}
\hfill
\href{https://youtu.be/t_1KL77mfmg}{\includegraphics[width = 2cm]{img/youtube}}
\end{minipage}
\vspace{-3ex}
%% --------------------------------------------------------------------



\paragraph{Visualisierung des Mosaicplots}

In einen Versuch in einer Klimakammer im Oldenburger Land hatte Mark sich zum einen die Behandlung KI-gesteuert [ja/nein] und zum anderen die Messung Proteingehalt im Zielbereich [ja/nein] im Kontext von Kartoffeln angeschaut. Nun steht Mark vor dem Problem, dass er zwei kategoriale Variablen in seinem Projektbericht gemessen hat. Dazu kommt dann noch was anderes. Mark und die Unsicherheit, eine unendliche Geschichte mit kniffeligen Wendungen. Da seine Betreuerin erstmal die langen Tabellen mit ja/nein in einer explorativen Datenanalyse zusammengefasst und präsentiert bekommen möchte bevor es überhaupt weitergeht, muss er jetzt eine Lösung finden. Was alles auch nicht einfacher macht. Am liebsten würde er ja was anderes machen. Mark liebt Geocaching. Darin kann er sich wirklich verlieren und immer wieder neu begeistern.



\vspace{1Ex}

\begin{center}
\begin{minipage}[t]{0.45\textwidth}
%\small
\begin{center}

\begin{tabular}{p{2.5cm}p{2.5cm}p{2.5cm}p{2.5cm}}
\toprule
KI-gesteuert & Proteingehalt im Zielbereich\\
\midrule
nein & nein\\
nein & nein\\
nein & ja\\
nein & nein\\
ja & ja\\
\addlinespace
nein & nein\\
nein & nein\\
nein & ja\\
nein & ja\\
ja & nein\\
\addlinespace
ja & ja\\
ja & ja\\
nein & ja\\
nein & nein\\
nein & ja\\
\addlinespace
nein & ja\\
\bottomrule
\end{tabular}


\end{center}
\end{minipage}
\begin{minipage}[t]{0.45\textwidth}
%\small
\begin{center}

\begin{tabular}{p{2.5cm}p{2.5cm}p{2.5cm}p{2.5cm}}
\toprule
KI-gesteuert & Proteingehalt im Zielbereich\\
\midrule
ja & ja\\
ja & ja\\
ja & nein\\
ja & nein\\
ja & nein\\
\addlinespace
nein & ja\\
nein & nein\\
ja & ja\\
ja & nein\\
nein & ja\\
\addlinespace
ja & nein\\
ja & ja\\
ja & ja\\
nein & nein\\
ja & ja\\
\addlinespace
ja & nein\\
\bottomrule
\end{tabular}


\end{center}
\end{minipage}
\end{center}

\vspace{2Ex}

Leider kennt sich Mark mit der Erstellung einer explorativen Datenanalyse für kategoriale Daten überhaupt nicht aus. Deshalb braucht er bei der Erstellung Ihre Hilfe!

\begin{enumerate}
\item Stellen Sie den Zusammenhang zwischen den beiden kategorialen Variablen in einer zusammenfassenden Tabelle dar! \textbf{(3 Punkte)}
\item Visualisieren Sie den Zusammenhang zwischen den beiden kategorialen Variablen! \textbf{(3 Punkte)}
\item Berechnen Sie die Verhältnisse in der Visualisierung! Welche Annahme haben Sie getroffen? \textbf{(2 Punkte)}
\item Wenn \textit{ein} Effekt von der Behandlung vorliegen würde, wie würde die Tabelle und die Visualisierung aussehen? \textbf{(2 Punkt)}
\end{enumerate} 
\clearpage
% -----------------------------------------------------------------------

\section{Aufgabe \hfill (10 Punkte)}

\textit{Geben Sie grundsätzlich Formeln und Rechenweg zur Lösung der Teilaufgaben mit an!} \\[1Ex]
 

 
%% --------------------------------------------------------------------
\begin{minipage}[t]{0.5\textwidth}
\includegraphics[width = 1.3cm]{/Users/kruppajo/work/GitHub/exam/avatare/Paula.png}\hspace{-4mm}\includegraphics[width = 1.3cm]{/Users/kruppajo/work/GitHub/exam/avatare/Tina.png}
\end{minipage}
\begin{minipage}[t]{0.5\textwidth}
\hfill
\href{https://youtu.be/Op-gjzASH9I}{\includegraphics[width = 2cm]{img/youtube}}
\end{minipage}
%% --------------------------------------------------------------------



\paragraph{Visualisierung von Verteilungen}

'Ich glaube, dass es sich hier wieder um so ein kryptisches Lernziel handelt, was nicht so gleich klar ist.', meint Paula und streichelt sanft die Ratte. Das Tier versucht dem strammen Griff zu entkommen, gibt aber auf. Tina sieht sich sehr genau die drei liegenden Boxplots an. 'Du weißt doch wie es heißt, \textit{Frei ist, wer missfallen kann.}\footnote{Oschmann, A. (2024) Mädchen stärken: Stärken fördern, Selbstwert erhöhen und liebevoll durch Krisen begleiten. Goldegg Verlag}', merkt Paula nickend an. Das Ziel ist es zu verstehen, wie eine Verteilung anhand eines Boxplots bewertet werden kann. Paula und der Perfektionismus machen die Sache nicht einfacher.



{\centering \includegraphics[width=\maxwidth]{img/desc-stat-11-1} 

}




Jetzt brauchen Paula und Tina Ihre Hilfe bei der Abschätzung einer Verteilung anhand von Boxplots um ihre Arbeit dann in diesem Semester noch abschließen zu können.

\begin{enumerate}
\item Zeichnen Sie über die Boxplots die entsprechende zugehörige Verteilung! \textbf{(3 Punkte)} 
\item Zeichnen Sie unter die Boxplots die entsprechende zugehörige Beobachtungen als Stiche! \textbf{(3 Punkte)}
\item Wie viel Prozent der Beobachtungen fallen in das IQR? Ergänzen Sie die Abbildung entsprechend um den Bereich! \textbf{(2 Punkte)}
\item Wie viel Prozent der Beobachtungen fallen in $\bar{y} \pm 1s$ und $\bar{y} \pm 2s$  unter der Annahme einer Normalverteilung? \textbf{(2 Punkte)}
\end{enumerate} 
\clearpage
% -----------------------------------------------------------------------
\part{Statistisches Testen \& statistische Testtheorie}
% -----------------------------------------------------------------------  

\section{Aufgabe \hfill (9 Punkte)}


 
%% --------------------------------------------------------------------
\begin{minipage}[t]{0.5\textwidth}
\includegraphics[width = 1.3cm]{/Users/kruppajo/work/GitHub/exam/avatare/Alex.png}\hspace{-4mm}\includegraphics[width = 1.3cm]{/Users/kruppajo/work/GitHub/exam/avatare/Steffen.png}
\end{minipage}
\begin{minipage}[t]{0.5\textwidth}
\hfill
\href{https://youtu.be/aHVYuFKTqZs}{\includegraphics[width = 2cm]{img/youtube}}
\end{minipage}
%% --------------------------------------------------------------------



\paragraph{Grundgesamtheit und experimentelle Stichprobe}

'Grundlage des statistischen Testen ist das Verständnis von der Grundgesamtheit (eng. \textit{population} oder \textit{ground truth}) und der experimentellen Stichprobe (eng. \textit{sample}). ', liest Steffen laut aus dem Skript vor. Alex war kurz eingenickt und wird mit einem Stoß geweckt. 'Reiz dich zusammen und iss noch ein paar Oreos das hilft mir immer. Alleine komme ich hier nicht weiter.', tadelt Steffen Alex etwas zu forsch. 'War ne lange Nacht', mault Alex. Beide sollen in ihrer Abschlussarbeit einen statistischen Test interpretieren und versuchen die Grundlagen zu wiederholen. Alex war auf einem Konzert von Abba.

\vspace{1ex}

Leider kennen sich Steffen und Alex mit der Grundgesamtheit und der Stuchprobe überhaupt nicht aus. Daher sind Sie gefragt!

\begin{enumerate}
\item Nennen Sie das statistische Verfahren und zwei konkrete Beispiele zur Durchführung um von einer Grundgesamtheit auf eine Stichprobe zu gelangen! \textbf{(3 Punkte)}
\item Erklären Sie den Zusammenhang zwischen Stichprobe und Grundgesamtheit an einem Schaubild! Beschriften Sie das Schaubild entsprechend!
  \textit{Nutzen Sie hierfür als Veranschaulichung ein aussagekräftiges Beispiel!}  \textbf{(3 Punkte)}
\item Erweitern Sie das Schaubild um die Entstehung von $Pr(D|H_0)$! \textit{Nutzen Sie hierfür als Veranschaulichung ein aussagekräftiges Beispiel!}  \textbf{(3 Punkte)}
\end{enumerate} 
\clearpage
% -----------------------------------------------------------------------

\section{Aufgabe \hfill (9 Punkte)}


 
%% --------------------------------------------------------------------
\begin{minipage}[t]{0.5\textwidth}
\includegraphics[width = 1.3cm]{/Users/kruppajo/work/GitHub/exam/avatare/Jessica.png}\hspace{-4mm}\includegraphics[width = 1.3cm]{/Users/kruppajo/work/GitHub/exam/avatare/Mark.png}
\end{minipage}
\begin{minipage}[t]{0.5\textwidth}
\hfill
\href{https://youtu.be/Ric8ne39DtI}{\includegraphics[width = 2cm]{img/youtube}}
\end{minipage}
%% --------------------------------------------------------------------



\paragraph{Das Nullritual - Die statistische Testtheorie}

'Das Känguruh hat mir gerade zugezwinkert. Das macht mir Angst', bemerkt Jessica. Es reicht ja schon ein Problem. Wenn der Mangel nicht wäre, ja dann wäre wohl vieles möglich für Jessica! Aber so.. Da braucht es nicht noch Wahnvorstellungen. 'Ich glaube nicht, das Känguruhs zwinkern. Aber was Wichtigeres. Wo hast du eigentlich meine Schnapspralinen hingetan? Wir haben nur noch Schokobons von dir. Bäh!', antwortet Mark. Beide sind im Zoo und wollen sich von der statistische Testheorie ablenken lassen. Eigentlich wollte ja Jessica stoppen wie lange Tiere pinkeln\footnote{Yang, P. J., et al. (2014). Duration of urination does not change with body size. Proceedings of the National Academy of Sciences, 111(33), 11932-11937.}, scheiterte aber an einer Oma mit Stock, die die beiden beschimpfte.

\vspace{1ex}

Leider kennen sich Jessica und Mark mit statistischen Testtheorie, auch Null-Ritual genannt, überhaupt nicht aus. Geschweige denn mit der Visualisierung als Kreuztabelle.  

\begin{enumerate}
\item Tragen Sie folgende statistische Fachbegriffe zur statistischen Testtheorie korrekt eine selbst erstellte Kreuztabelle ein! \textbf{(3 Punkte)}
  \begin{center}
  \begin{tabular}{cccc}
  H$_0$ falsch & (Unbekannte) Wahrheit & Testentscheidung & H$_0$ wahr \\
  \end{tabular}
  \end{center}
\item Ergänzen Sie Ihre erstellte Kreuztabelle um vier weitere, passende Fachbegriffe zur statistischen Testtheorie! \textbf{(2 Punkte)}
\end{enumerate}

Die Entscheidungsfindung durch einen statistischen Test kann auch durch die Analogie zu einem Feuermelder abgebildet werden. Dabei symbolisiert der Feuermelder den statistischen Test und es soll getestet werden, ob ein Feuer ausgebrochen ist.

\begin{enumerate}
  \setcounter{enumi}{2}    
\item In der Analogie des Feuermelders, wie lautet der $\alpha$-Fehler? \textbf{(1 Punkt)}
\item In der Analogie des Feuermelders, wie lautet der $\beta$-Fehler? \textbf{(1 Punkt)}
\item Wenn der Feuermelder einmal pro Tag messen würde, wie oft würde der Feuermelder mit einem $\alpha$ von 5\% in einem Monat Alarm schlagen? Begründen Sie Ihre Antwort! \textbf{(2 Punkte)}
\end{enumerate}



 
\clearpage
% -----------------------------------------------------------------------

\section{Aufgabe \hfill (9 Punkte)}

\textit{Geben Sie grundsätzlich Formeln und Rechenweg zur Lösung der Teilaufgaben mit an!} \\[1Ex]


 
%% --------------------------------------------------------------------
\begin{minipage}[t]{0.5\textwidth}
\includegraphics[width = 1.3cm]{/Users/kruppajo/work/GitHub/exam/avatare/Jessica.png}\hspace{-4mm}\includegraphics[width = 1.3cm]{/Users/kruppajo/work/GitHub/exam/avatare/Jonas.png}
\end{minipage}
\begin{minipage}[t]{0.5\textwidth}
\hfill
\href{https://youtu.be/32JjH1eyuTU}{\includegraphics[width = 2cm]{img/youtube}}
\end{minipage}
%% --------------------------------------------------------------------



\paragraph{Visualisierung der Teststatistik $\boldsymbol{T_D}$ und dem p-Wert}

Jessica und Jonas wollten eigentlich einen Flug nach Mallorca buchen, sind jetzt aber dann doch dazu übergegangen nochmal die Aufgaben für die Statistikklausur durchzugehen. 'Kannst du mir nochmal an einer Visualisierung erklären, wie der Zusammenhang zwischen der Teststatistik aus den Daten $T_D$ und dem p-Wert ist? Ich habe hier zig Fachbegriffe, kriege die abr nicht zusammen...', fragt Jessica. Jonas zuckt mit den Schultern. So genau hatte Jonas jetzt auch nicht aufgepasst. Da hilft dann eventuell das YouTube Video weiter. Jessica mapmft Schokobons und fragt sich, was das alles soll.

\vspace{1ex}

Leider kennen sich Jessica und Jonas mit der Visualisierung der Teststatistik $T_D$ und dem p-Wert überhaupt nicht aus und brauchen dahr Ihre Hilfe!

\vspace{1ex}

\textit{Beachten Sie, dass im Folgenden \underline{keine numerisch korrekte Darstellung} verlangt wird! Es gilt Erkennbarkeit vor Genauigkeit!}

\begin{enumerate}
\item Ergänzen Sie eine beschriftete $x$-Achse! \textbf{(1 Punkt)}
\item Ergänzen Sie "`$\bar{y}_1 = \bar{y}_2$"'! \textbf{(1 Punkt)} 
\item Ergänzen Sie "`$A = 95\%$"'! \textbf{(1 Punkt)}
\item Zeichnen Sie $T_{\alpha=5\%}$ in die Abbildung! \textbf{(1 Punkt)} 
\item Zeichnen Sie das Signifikanzniveau $\alpha$ in die Abbildung! Begründen Sie Ihre Antwort! \textbf{(2 Punkte)} 
\item Zeichnen Sie $-T_{D}$ in die Abbildung! \textbf{(1 Punkt)}
\item Zeichnen Sie einen nicht signifikant p-Wert in die Abbildung! Begründen Sie Ihre Antwort! \textbf{(2 Punkte)}   
\end{enumerate}



{\centering \includegraphics[width=\maxwidth]{img/statistisches-testen-3-1} 

}


 
\clearpage
% -----------------------------------------------------------------------

\section{Aufgabe \hfill (10 Punkte)}


 
%% --------------------------------------------------------------------
\begin{minipage}[t]{0.5\textwidth}
\includegraphics[width = 1.3cm]{/Users/kruppajo/work/GitHub/exam/avatare/Jessica.png}\hspace{-4mm}\includegraphics[width = 1.3cm]{/Users/kruppajo/work/GitHub/exam/avatare/Mark.png}
\end{minipage}
\begin{minipage}[t]{0.5\textwidth}
\hfill
\href{https://youtu.be/CN_O4fYPbhs}{\includegraphics[width = 2cm]{img/youtube}}
\end{minipage}
%% --------------------------------------------------------------------



\paragraph{Visualisierung des 95\% Konfidenzintervalls}

'Okay, für was war jetzt nochmal das 95\% Konfidenzintervall gut?', fragt Jessica und schaut in das leere Gesicht von Mark. 'Keine Ahnung. Irgendwas mit Relevanz und Effekt oder Signifikanz. Da kannst du irgendwie was verbinden. Keine Ahnung warum', entgegnet Mark. 'Wir haben doch als Messwert \textit{Energieverbrauch der Klimakammer} erhoben.', stellt Jessica fest. Jetzt haben beide das Problem, die möglichen 95\% Konfidenzintervalle zu interpretieren.

\vspace{1ex}

Leider kennen sich Jessica und Mark mit der Visualisierung des 95\% Konfidenzintervall überhaupt nicht aus. 

\begin{enumerate}
\item Beschriften Sie die untenstehende Abbildung mit der Signifikanzschwelle! Begründen Sie Ihre Antwort! \textbf{(2 Punkte)}
\item Ergänzen Sie eine \textit{in den Kontext passende} Relevanzschwelle! Begründen Sie Ihre Antwort! \textbf{(2 Punkte)} 
\item Skizieren Sie in die untenstehende Abbildung sechs einzelne Konfidenzintervalle (a-f) mit den
  jeweiligen Eigenschaften! \textbf{(6 Punkte)}
  \begin{itemize}
  \item[(a)] Ein 95\% Konfidenzintervall mit niedriger Fallzahl $n$ in der Stichprobe als der Rest 95\% der Konfidenzintervalle 	
  \item[(b)] Ein signifikantes, relevantes 99\% Konfidenzintervall. 	
  \item[(c)] Ein signifikantes, nicht relevantes 95\% Konfidenzintervall 	
  \item[(d)] Ein 95\% Konfidenzintervall mit h{"o}herer Fallzahl $n$ in der Stichprobe als der Rest der 95\% Konfidenzintervalle 
  \item[(e)] Ein signifikantes, relevantes 95\% Konfidenzintervall
  \item[(f)] Ein nicht signifikantes, nicht relevantes 95\% Konfidenzintervall
  \end{itemize}
\end{enumerate}

\begin{center}
  \includegraphics[height = 10cm]{/Users/kruppajo/work/GitHub/exam/question/img/statistisches-testen-04}
\end{center}


 
\clearpage
% -----------------------------------------------------------------------

\section{Aufgabe \hfill (10 Punkte)}

\textit{Geben Sie grundsätzlich Formeln und Rechenweg zur Lösung der Teilaufgaben mit an!} \\[1Ex]


 
%% --------------------------------------------------------------------
\begin{minipage}[t]{0.5\textwidth}
\includegraphics[width = 1.3cm]{/Users/kruppajo/work/GitHub/exam/avatare/Jonas.png}\hspace{-4mm}\includegraphics[width = 1.3cm]{/Users/kruppajo/work/GitHub/exam/avatare/Nilufar.png}
\end{minipage}
\begin{minipage}[t]{0.5\textwidth}
\hfill
\href{https://youtu.be/FgZmpnEWDag}{\includegraphics[width = 2cm]{img/youtube}}
\end{minipage}
%% --------------------------------------------------------------------



\paragraph{Zusammenhang zwischen dem Effekt, der Streuung sowie der Fallzahl}

An einem kalten Dezembermorgen haben sich Nilufar und Jonas zum Lernen verabredet. Eine große Kanne Tee und Berge von Takis Blue Heat warten darauf gegessen zu werden. Nilufar liest laut vor:\begin{quote}
                 \textit{
                 Beim statistischen Testen gibt es einen Zusammenhang zwischen dem Effekt, der Streuung sowie der Fallzahl. Gegeben sei die Formel für den Student t-Test auf den die folgenden Überlegungen basieren sollen. Welche Auswirkung hat die Änderungen der jeweiligen statistischen Maßzahl des Effekts $\Delta$, der Streuung $s$ und der Fallzahl $n$ auf die Teststistik $T_{D}$, den p-Wert $Pr(D|H_0)$ sowie dem Konfidenzintervall $KI_{1-\alpha}$?
                 }
                 \end{quote}Jonas hebt die Augenbraue. 'Irgendwie sagt mir die Aufgabe jetzt mal so gar nichts. Was soll da gemacht werden?', merkt Jonas an und ergänzt: 'Schauen wir doch erstmal zur Entspannung Mission Impossible, den Film habe ich extra nochmal mitgebracht.' Nilufar ist der Idee nicht abgeneigt und auch das Huhn kommt unter dem Sofa hervor um mitzuschauen. 

\vspace{1ex}

Leider kennen sich Nilufar und Jonas mit dem Zusammenhang zwischen dem Effekt, der Streuung sowie der Fallzahl überhaupt nicht aus. 


\begin{enumerate}
\item Visualisieren Sie den Zusammenhang zwischen der Teststatiatik $T_{D}$ und dem p-Wert $Pr(D|H_0)$ für sich verändernde $T_{D}$-Werte!\textit{Geben Sie dafür ein numerisches Beispiel in dem Sie drei $T_{D}$-Werte und deren Einfluss auf den p-Wert vergleichen!} \textbf{(3 Punkte)}  
\item  Füllen Sie die untenstehende Tabelle aus in dem Sie die Änderung der statistischen Maßzahlen auf die Teststatistik, den p-Wert sowie das Konfidenzintervall in \textit{einem} Wort oder Symbol beschreiben! \textbf{(4 Punkte)}
\begin{center}
  \large
  \begin{tabular}[c]{l|c|c|c|l|c|c|c}
    & $T_{D}$ & $Pr(D|H_0)$ & $KI_{1-\alpha}$ & & $T_{D}$ & $Pr(D|H_0)$ & $KI_{1-\alpha}$\strut\\ 
    \hline
    \textbf{$\Delta\; \uparrow$} & \hspace{1.8cm} & \hspace{1.8cm}  & \hspace{1.8cm} & \textbf{
                                                          $\Delta\; \downarrow$} &
                                                                          \hspace{1.8cm} & \hspace{1.8cm}  & \hspace{1.8cm}\strut\\
    \hline
        \textbf{$s\; \uparrow$} & \hspace{1.8cm} & \hspace{1.8cm}  & \hspace{1.8cm} & \textbf{
                                                          $s\; \downarrow$} &
                                                                          \hspace{1.8cm}
                                                & \hspace{1.8cm}  & \hspace{1.8cm}\strut\\
    \hline
        \textbf{$n\; \uparrow$} & \hspace{1.8cm} & \hspace{1.8cm}  & \hspace{1.8cm} & \textbf{
                                                          $n\; \downarrow$} &
                                                                          \hspace{1.8cm}
                                                & \hspace{1.8cm}  & \hspace{1.8cm}\strut\\
    \hline
  \end{tabular}
\end{center}
\item Visualisieren Sie ein 95\%-iges Konfidenzintervall im Vergleich zu einem 90\%-igen Konfidenzintervall! Begründen Sie Ihre Visualisierung anhand der Formel des Konfidenzintervalls des t-Tests mathematisch! \textbf{(3 Punkte)} 
\end{enumerate} 
\clearpage
% -----------------------------------------------------------------------
\part{Der Student t-Test, Welch t-Test \& gepaarter t-Test}
% -----------------------------------------------------------------------

\section{Aufgabe \hfill (9 Punkte)}

\textit{Geben Sie grundsätzlich Formeln und Rechenweg zur Lösung der Teilaufgaben mit an!} \\[1Ex]
 

 
%% --------------------------------------------------------------------
\begin{minipage}[t]{0.5\textwidth}
\includegraphics[width = 1.3cm]{/Users/kruppajo/work/GitHub/exam/avatare/Mark.png}
\end{minipage}
\begin{minipage}[t]{0.5\textwidth}
\hfill
\href{https://youtu.be/eejS2uG4o-M}{\includegraphics[width = 2cm]{img/youtube}}
\end{minipage}
\vspace{-3ex}
%% --------------------------------------------------------------------



\paragraph{Berechnung des Student t-Test \underline{oder} Welch t-Test}

Mark ist im Wendland für einen Pilotexperiment mit sehr geringer Fallzahl $(n_1 = n_2 = 3)$ mit Spargel. Allein diese Tatsache ist für ihn eine Erzählung wert. Eine echte Herausforderung für ihn war schon immer die Unsicherheit gewesen. Ein leidiges Lied.  Für seiner Hausarbeit musste er ein Feldexperiment mit Spargel durchführen und das sollte laut seiner Betreuerin an diesem Ort besonders gut gelingen, da man hier gut neue technische Anlagen und Behandlungen fernab der Bevölkerung testen könne. Zeugen gibt es hier jedenfalls keine. Gar keine.  Alleine sein hilft jetzt aber nur bedingt, denn seine Behandlung Lichtstufen ($none$ und $600lm$) und der Messwert Frischegewicht [kg/ha] sollen mit einem t-Test ausgewertet werden. Immerhin weiß er, dass sein Messwert einer Normalverteilung folgt. Hm..., was entspannendes wäre gut. Einfach mal raus um zu Reiten. Ohne Ziel und Uhr. Das ist was für Mark.

\begin{table}[!h]
\centering
\begin{tabular}{cc}
\toprule
treatment & weight\\
\midrule
ctrl & 15.5\\
dose & 9.8\\
dose & 26.0\\
dose & 16.5\\
ctrl & 17.8\\
\addlinespace
ctrl & 19.0\\
\bottomrule
\end{tabular}
\end{table}



Leider kennt sich Mark mit der Berechnung eines t-Tests überhaupt nicht aus. Deshalb braucht er bei der Berechnung Ihre Hilfe!

\begin{enumerate}
  \item Formulieren Sie das statistische Hypothesenpaar! \textbf{(1 Punkt)}
  \item Bestimmen Sie die Teststatistik $T_{D}$ eines Welch t-Tests! \textbf{(3 Punkte)}
  \item Treffen Sie mit $T_{\alpha = 5\%} = 2.68$ eine Aussage zur Nullhypothese! Begründen Sie Ihre Antwort! \textbf{(2 Punkte)}
  \item Berechnen Sie den Effekt des Welch t-Tests! \textbf{(1 Punkt)}
  \item Formulieren Sie eine Antwort an Mark über das Ergebnis Ihrer statistischen Analyse! \textbf{(2 Punkte)}
\end{enumerate} 
\clearpage
% -----------------------------------------------------------------------

\section{Aufgabe \hfill (12 Punkte)}

\textit{Geben Sie grundsätzlich Formeln und Rechenweg zur Lösung der Teilaufgaben mit an!} \\[1Ex]
 

 
%% --------------------------------------------------------------------
\begin{minipage}[t]{0.5\textwidth}
\includegraphics[width = 1.3cm]{/Users/kruppajo/work/GitHub/exam/avatare/Paula.png}
\end{minipage}
\begin{minipage}[t]{0.5\textwidth}
\hfill
\href{https://youtu.be/Cq_rF_z4xOk}{\includegraphics[width = 2cm]{img/youtube}}
\end{minipage}
\vspace{-3ex}
%% --------------------------------------------------------------------



\paragraph{Berechnung des Student t-Test}

Paula ist im Oldenburger Land für einen Versuch mit Lamas. Allein diese Tatsache ist für sie eine Erzählung wert. Wenn der Perfektionismus nicht wäre, ja dann wäre wohl vieles möglich für Paula! Aber so.. Für ihrer Hausarbeit musste sie ein Stallexperiment mit Lamas durchführen und das sollte laut ihrer Betreuerin an diesem Nichtort besonders gut gelingen. Ablenkung gibt es jedenfalls keine. Gar keine. Alleine sein hilft jetzt aber nur bedingt, denn ihre Behandlung Flüssignahrung ($ctrl$ und $flOw$) und der Messwert Gewichtszuwachs in der 1LW sollen mit einem t-Test ausgewertet werden. Immerhin weiß sie, dass ihr Messwert einer Normalverteilung folgt. Hm..., was entspannendes wäre gut. Um zu Fechten geht Paula dann später nochmal raus. Echte Entspannung.

\begin{table}[!h]
\centering
\begin{tabular}{cc}
\toprule
Flüssignahrung & Gewichtszuwachs\\
\midrule
flOw & 48.9\\
ctrl & 20.2\\
ctrl & 42.6\\
ctrl & 64.1\\
flOw & 34.0\\
\addlinespace
ctrl & 51.5\\
flOw & 35.2\\
ctrl & 46.5\\
ctrl & 40.2\\
ctrl & 32.6\\
\addlinespace
flOw & 46.3\\
ctrl & 26.2\\
flOw & 32.9\\
flOw & 54.8\\
ctrl & 51.0\\
\addlinespace
ctrl & 35.4\\
flOw & 50.8\\
flOw & 42.4\\
flOw & 42.8\\
ctrl & 39.8\\
\bottomrule
\end{tabular}
\end{table}



Leider kennt sich Paula mit der Berechnung eines t-Tests überhaupt nicht aus. Deshalb braucht sie bei der Berechnung Ihre Hilfe!

\begin{enumerate}
  \item Formulieren Sie die wissenschaftliche Fragestellung! \textbf{(1 Punkt)}
  \item Formulieren Sie das statistische Hypothesenpaar! \textbf{(1 Punkt)}
  \item Bestimmen Sie die Teststatistik $T_{D}$ eines Student t-Tests! \textbf{(3 Punkte)}
\item Treffen Sie mit $T_{\alpha = 5\%} = 2.68$ eine Aussage zur Nullhypothese! Begründen Sie Ihre Antwort! \textbf{(2 Punkte)}
\item Berechnen Sie den Effekt des Student t-Tests! \textbf{(1 Punkt)}
\item Wenn Sie \textit{einen} Unterschied zwischen den Behandlungsgruppen erwarten würden, wie groß wäre dann der Effekt? Begründen Sie Ihre Antwort! \textbf{(2 Punkte)}
\item Formulieren Sie eine Antwort an Paula über das Ergebnis Ihrer statistischen Analyse! \textbf{(2 Punkte)}
\end{enumerate} 
\clearpage
% -----------------------------------------------------------------------

\section{Aufgabe \hfill (12 Punkte)}

\textit{Geben Sie grundsätzlich Formeln und Rechenweg zur Lösung der Teilaufgaben mit an!} \\[1Ex]
 

 
%% --------------------------------------------------------------------
\begin{minipage}[t]{0.5\textwidth}
\includegraphics[width = 1.3cm]{/Users/kruppajo/work/GitHub/exam/avatare/Mark.png}
\end{minipage}
\begin{minipage}[t]{0.5\textwidth}
\hfill
\href{https://youtu.be/TbSEOMCQYl4}{\includegraphics[width = 2cm]{img/youtube}}
\end{minipage}
\vspace{-3ex}
%% --------------------------------------------------------------------



\paragraph{Berechnung des Welch t-Test}


'Der t-Test testet einen normalverteilten Messwert ($Y$).', liest Mark laut. Das hilft jetzt auch nur bedingt weiter. Wenn die Unsicherheit nicht wäre, ja dann wäre wohl vieles möglich für Mark! Aber so.. Laut seinem Betreuer ist zwar ihm Messwert Fettgehalt [\%/kg] normalverteilt, aber wie rechnet er jetzt einen t-Test? Für seinen Projektbericht musste er ein Kreuzungsexperiment mit Milchvieh in der Uckermark durchführen. Als wäre das nicht schon anstrengend genug gewesen. Jetzt soll er auch noch testen, ob die Behandlung Genotypen ($AA$ und $BB$) ein signifikantes Ergebnis liefert. Hm..., was entspannendes wäre gut. Schon dutzende Male gesehen: Columbo. Aber immer noch großartig zusammen mit Marzipankugeln.

\begin{table}[!h]
\centering
\begin{tabular}{cc}
\toprule
Genotypen & Fettgehalt\\
\midrule
BB & 12.4\\
AA & 35.7\\
AA & 29.3\\
AA & 17.0\\
BB & 22.3\\
\addlinespace
AA & 39.8\\
BB & 29.1\\
BB & 21.6\\
AA & 25.0\\
AA & 24.9\\
\addlinespace
AA & 41.3\\
AA & 26.3\\
BB & 26.9\\
BB & 33.1\\
BB & 51.7\\
\addlinespace
AA & 32.1\\
BB & 35.9\\
BB & 33.6\\
BB & 40.9\\
AA & 39.7\\
\addlinespace
BB & 21.1\\
AA & 18.4\\
\bottomrule
\end{tabular}
\end{table}



Leider kennt sich Mark mit der Berechnung eines t-Tests überhaupt nicht aus. Deshalb braucht er bei der Berechnung Ihre Hilfe!

\begin{enumerate}
  \item Formulieren Sie die wissenschaftliche Fragestellung! \textbf{(1 Punkt)}
  \item Formulieren Sie das statistische Hypothesenpaar! \textbf{(1 Punkt)}
  \item Bestimmen Sie die Teststatistik $T_{D}$ eines  Welch t-Tests! \textbf{(3 Punkte)}
  \item Treffen Sie mit $T_{\alpha = 5\%} = 2.04$ eine Aussage zur Nullhypothese! Begründen Sie Ihre Antwort! \textbf{(2 Punkte)}
\item Berechnen Sie das 95\% Konfidenzintervall. Welche Annahmen haben Sie getroffen? \textbf{(2 Punkte)}
\item Nennen Sie den statistischen Grund, warum Sie sich zwischen einem Student t-Test und einem Welch t-Test entscheiden müssen! \textbf{(1 Punkt)}
\item Formulieren Sie eine Antwort an Mark über das Ergebnis Ihrer statistischen Analyse! \textbf{(2 Punkte)}
\end{enumerate} 
\clearpage
% -----------------------------------------------------------------------

\section{Aufgabe \hfill (11 Punkte)}

\textit{Geben Sie grundsätzlich Formeln und Rechenweg zur Lösung der Teilaufgaben mit an!} \\[1Ex]
 

 
%% --------------------------------------------------------------------
\begin{minipage}[t]{0.5\textwidth}
\includegraphics[width = 1.3cm]{/Users/kruppajo/work/GitHub/exam/avatare/Mark.png}\hspace{-4mm}\includegraphics[width = 1.3cm]{/Users/kruppajo/work/GitHub/exam/avatare/Steffen.png}
\end{minipage}
\begin{minipage}[t]{0.5\textwidth}
\hfill
\href{https://youtu.be/QR90zyn0Iz8}{\includegraphics[width = 2cm]{img/youtube}}
\end{minipage}
%% --------------------------------------------------------------------



\paragraph{Berechnung des gepaarten t-Test}

Alles voll mit Spargel. Das haben Steffen und Mark gemeinsam in einer Hausarbeit gemacht! Worum ging es aber konkret? Beide haben als ein normalverteiltes Outcome ($Y$) Trockengewicht [kg/ha] von Spargel bestimmt. Die Daten haben beide zusammen in einem Gewächshausexperiment erhoben. In dem Experiment ging es um eine vorher/nachher Untersuchung an den gleichen Spargel. Als Behandlung wurde Beschattung ($7d$ und $14d$) eingesetzt. Nach der Meinung der Betreuerin muss hier ein gepaarter t-Test gerechnet werden. Steffen schaut nachdenklich zu Mark. Steffen und die Romantik, eine unendliche Geschichte mit kniffeligen Wendungen.. Steffen denkt derweil angestrengt an Marzipankugeln. Im Hintergrund wummert Taylor Swift.

\begin{table}[!h]
\centering
\begin{tabular}{ccc}
\toprule
ID & treatment & freshmatter\\
\midrule
7 & 7d & 5.5\\
3 & 14d & 31.6\\
1 & 14d & 34.5\\
1 & 7d & 3.7\\
8 & 7d & 34.0\\
\addlinespace
5 & 14d & 27.0\\
6 & 14d & 31.5\\
10 & 7d & 13.4\\
7 & 14d & 30.0\\
9 & 7d & 5.0\\
\addlinespace
6 & 7d & 22.1\\
4 & 14d & 33.9\\
2 & 7d & 8.7\\
5 & 7d & 39.6\\
11 & 7d & 21.0\\
\addlinespace
3 & 7d & 12.9\\
2 & 14d & 28.5\\
4 & 7d & 39.3\\
\bottomrule
\end{tabular}
\end{table}



Leider kennen sich Steffen und Mark mit der Berechnung eines gepaarten t-Tests überhaupt nicht aus. Deshalb brauchen sie beide bei der Berechnung Ihre Hilfe!

\begin{enumerate}
  \item Formulieren Sie die wissenschaftliche Fragestellung! \textbf{(1 Punkt)}
  \item Formulieren Sie das statistische Hypothesenpaar! \textbf{(1 Punkt)}
  \item Bestimmen Sie die Teststatistik $T_{D}$ eines gepaarten t-Tests! \textbf{(3 Punkte)}
  \item Treffen Sie mit $T_{\alpha = 5\%} = 2.68$ eine Aussage zur Nullhypothese! Begründen Sie Ihre Antwort! \textbf{(2 Punkte)}
\item Schätzen Sie den $p$-Wert des gepaarten t-Tests ab! Begründen Sie Ihre Antwort mit einer Skizze! \textbf{(2 Punkte)}
\item Formulieren Sie eine Antwort an Steffen über das Ergebnis Ihrer statistischen Analyse! \textbf{(2 Punkte)}
\end{enumerate}


 
\clearpage
% -----------------------------------------------------------------------

\section{Aufgabe \hfill (10 Punkte)}

\textit{Geben Sie grundsätzlich Formeln und Rechenweg zur Lösung der Teilaufgaben mit an!} \\[1Ex]
 

 
%% --------------------------------------------------------------------
\begin{minipage}[t]{0.5\textwidth}
\includegraphics[width = 1.3cm]{/Users/kruppajo/work/GitHub/exam/avatare/Mark.png}\hspace{-4mm}\includegraphics[width = 1.3cm]{/Users/kruppajo/work/GitHub/exam/avatare/Paula.png}\hspace{-4mm}\includegraphics[width = 1.3cm]{/Users/kruppajo/work/GitHub/exam/avatare/Tina.png}
\end{minipage}
\begin{minipage}[t]{0.5\textwidth}
\hfill
\href{https://youtu.be/exDo7AyHl4Q}{\includegraphics[width = 2cm]{img/youtube}}
\end{minipage}
%% --------------------------------------------------------------------



\paragraph{Interpretation des t-Tests in \Rlogo - die Teststatistik und der p-Wert}


Almería. Spanien. Sonne und Strand. Paula und Mark haben ihren gemeinsamen Auslandsaufenthalt sichtlich genossen. Dann hatte sich auch noch angeboten ihre Abschlussarbeit gemeinsam in Almería durchzuführen. Es hätte sogar noch bessser funktionieret, wenn Tina nicht die Wut ein paar Mal im Weg gestanden hätte und Mark nicht das Problem gehabt hätte der Perfektionismus zu händeln. Nun müssen jetzt alle Daten in \Rlogo ausgewertet werden, da \Rlogo international der Standard in der Datenauswertung ist und die Betreuer in Spanien nur \Rlogo können. Während beide Tina Oliven mit Katjes füttern, hoffen Paula und Mark mehr Informationen von Tina über die seltsame \Rlogo Ausgabe des t-Tests. Immerhin erinnern beide sich an die Behandlung Substrattypen ($torf$ und $70p30n$) und das es um Lauch ging. Im Hintergrund wummert Tocotronic und Fotos zeigen Tina mit dem Hobby Astronomie.

\begin{knitrout}
\definecolor{shadecolor}{rgb}{0.969, 0.969, 0.969}\color{fgcolor}\begin{kframe}
\begin{verbatim}
## 
## 	Two Sample t-test
## 
## data:  Chlorophyllgehalt by Substrattypen
## t = 1.4736, df = 19, p-value = 0.157
## alternative hypothesis: true  is not equal to [condensed]
## 95 percent confidence interval:
##  -2.250984 12.960075
## sample estimates:
##   mean in group torf mean in group 70p30n 
##             42.05455             36.70000
\end{verbatim}
\end{kframe}
\end{knitrout}

Helfen Sie Tina bei der Interpretation des t-Tests! Sonst geht es auch für Paula und Mark nicht weiter.
  
\begin{enumerate}
  \item Formulieren Sie die wissenschaftliche Fragestellung! \textbf{(1 Punkt)}
  \item Formulieren Sie das statistische Hypothesenpaar! \textbf{(1 Punkt)}
\item Liegt ein signifikanter Unterschied zwischen den Gruppen vor? Begründen Sie Ihre Antwort! \textbf{(2 Punkte)}
\item Skizzieren Sie eine Abbildung in der Sie $T_{D}$, $Pr(D|H_0)$, $A=0.95$, sowie $T_{\alpha=5\%} = |2.09|$ einzeichnen! \textbf{(4 Punkte)}
\item Beschriften Sie die Abbildung! \textbf{(1 Punkt)}  
\item Berechnen Sie den Effekt des t-Tests! \textbf{(1 Punkt)}
\end{enumerate} 
\clearpage
% -----------------------------------------------------------------------

\section{Aufgabe \hfill (10 Punkte)}

\textit{Geben Sie grundsätzlich Formeln und Rechenweg zur Lösung der Teilaufgaben mit an!} \\[1Ex]
 

 
%% --------------------------------------------------------------------
\begin{minipage}[t]{0.5\textwidth}
\includegraphics[width = 1.3cm]{/Users/kruppajo/work/GitHub/exam/avatare/Jessica.png}\hspace{-4mm}\includegraphics[width = 1.3cm]{/Users/kruppajo/work/GitHub/exam/avatare/Paula.png}\hspace{-4mm}\includegraphics[width = 1.3cm]{/Users/kruppajo/work/GitHub/exam/avatare/Steffen.png}
\end{minipage}
\begin{minipage}[t]{0.5\textwidth}
\hfill
\href{https://youtu.be/wJqsNV1hOW8}{\includegraphics[width = 2cm]{img/youtube}}
\end{minipage}
%% --------------------------------------------------------------------



\paragraph{Interpretation des t-Tests in \Rlogo - das 95\% Konfidenzintervall}


'Programmieren ist wie eine Sprache lernen. Man muss es nur machen, dann wird man mit der Zeit immer besser!', gibt Steffen zwinkernd zu Protokoll. Ein paar Mal hat sie schon die Romantik gehindert weiterzumachen. Das hilft jetzt Paula und Jessica nur bedingt, da beide jetzt die \Rlogo Ausgabe interpretieren müssen und nicht vor drei Wochen, wo noch Zeit gewesen wäre. Beide mampfen konzentriert Smarties und Schokobons in sich hinein. Die beiden hatten in der Uckermark einen Versuch mit Puten in einem Kreuzungsexperiment durchgeführt. Das war schon anstrengend genug! 'Wir haben Protein/Fettrate [\%/kg] gemessen, vielleicht hilft das ja...', meint Jessica leicht genervt. Alle starren auf die \Rlogo Ausgabe des t-Tests. Im Hintergrund wummert Taylor Swift und man versteht kaum sein eigenes Wort. Jessica hofft, dass die Schlange von Steffen beruhigend wirkt.

\begin{knitrout}
\definecolor{shadecolor}{rgb}{0.969, 0.969, 0.969}\color{fgcolor}\begin{kframe}
\begin{verbatim}
## 
## 	Two Sample t-test
## 
## data:  Protein/Fettrate by Lüftungssystem
## t = 0.8988, df = 17, p-value = 0.3813
## alternative hypothesis: true  is not equal to [condensed]
## 95 percent confidence interval:
##  -4.358017 10.826906
## sample estimates:
##     mean in group keins mean in group vorhanden 
##                37.84444                34.61000
\end{verbatim}
\end{kframe}
\end{knitrout}

Helfen Sie Steffen bei der Interpretation des t-Tests! Sonst geht es auch für Paula und Jessica nicht weiter.

\begin{enumerate}
  \item Formulieren Sie die wissenschaftliche Fragestellung! \textbf{(1 Punkt)}
  \item Formulieren Sie das statistische Hypothesenpaar! \textbf{(1 Punkt)}
\item Liegt ein signifikanter Unterschied zwischen den Gruppen vor? Begründen Sie Ihre Antwort! \textbf{(2 Punkte)}
\item Skizieren Sie das sich ergebende 95\% Konifidenzintervall! \textbf{(2 Punkte)}
\item Beschriften Sie die Abbildung und das 95\% Konfidenzintervall entsprechend! \textbf{(2 Punkte)}  
\item Interpretieren Sie den Effekt des 95\% Konifidenzintervalls! \textbf{(2 Punkte)}
\end{enumerate} 
\clearpage
% -----------------------------------------------------------------------

\section{Aufgabe \hfill (9 Punkte)}

\textit{Geben Sie grundsätzlich Formeln und Rechenweg zur Lösung der Teilaufgaben mit an!} \\[1Ex]
 

 
%% --------------------------------------------------------------------
\begin{minipage}[t]{0.5\textwidth}
\includegraphics[width = 1.3cm]{/Users/kruppajo/work/GitHub/exam/avatare/Jonas.png}\hspace{-4mm}\includegraphics[width = 1.3cm]{/Users/kruppajo/work/GitHub/exam/avatare/Nilufar.png}\hspace{-4mm}\includegraphics[width = 1.3cm]{/Users/kruppajo/work/GitHub/exam/avatare/Tina.png}
\end{minipage}
\begin{minipage}[t]{0.5\textwidth}
\hfill
\href{https://youtu.be/w62HJlbN28U}{\includegraphics[width = 2cm]{img/youtube}}
\end{minipage}
%% --------------------------------------------------------------------



\paragraph{Interpretation des t-Tests in \Rlogo - die Visualisierung}

Almería. Spanien. Sonne und Strand. Jonas und Tina haben ihren gemeinsamen Auslandsaufenthalt sichtlich genossen. Dann hatte sich auch noch angeboten ihre Abschlussarbeit gemeinsam in Almería durchzuführen. Es hätte sogar noch bessser funktionieret, wenn Nilufar nicht die Erwartung ein paar Mal im Weg gestanden hätte und Tina nicht das Problem gehabt hätte die Erschöpfung zu händeln. Nun müssen jetzt alle Daten in \Rlogo ausgewertet werden, da \Rlogo international der Standard in der Datenauswertung ist und die Betreuer in Spanien nur \Rlogo können. Während beide Nilufar Oliven mit Takis Blue Heat füttern, hoffen Jonas und Tina mehr Informationen von Nilufar über die seltsame \Rlogo Ausgabe des t-Tests. Immerhin erinnern beide sich an die Behandlung Genotypen ($AA$ und $BB$) und das es um Lauch ging. Im Hintergrund wummert Deichkind und Fotos zeigen Nilufar mit dem Hobby Hip Hop.

\begin{knitrout}
\definecolor{shadecolor}{rgb}{0.969, 0.969, 0.969}\color{fgcolor}\begin{kframe}
\begin{verbatim}
## 
## 	Two Sample t-test
## 
## data:  Trockengewicht by Genotypen
## t = 2.532, df = 19, p-value = 0.02032
## alternative hypothesis: true  is not equal to [condensed]
## 95 percent confidence interval:
##   1.242567 13.091978
## sample estimates:
## mean in group AA mean in group BB 
##         39.34000         32.17273
\end{verbatim}
\end{kframe}
\end{knitrout}

Helfen Sie Nilufar bei der Interpretation des t-Tests! Sonst geht es auch für Jonas und Tina nicht weiter.
  
\begin{enumerate}
  \item Formulieren Sie die wissenschaftliche Fragestellung! \textbf{(1 Punkt)}
  \item Formulieren Sie das statistische Hypothesenpaar! \textbf{(1 Punkt)}
\item Liegt ein signifikanter Unterschied zwischen den Gruppen vor? Begründen Sie Ihre Antwort! \textbf{(2 Punkte)}
\item Skizieren Sie die sich ergebenden Boxplot! Welche Annahmen an die Daten haben Sie getroffen? Begründen Sie Ihre
  Antwort! \textbf{(2 Punkte)} 
\item Skizieren Sie die sich ergebenden Barplots! \textbf{(2 Punkte)}
\item Berechnen Sie den Effekt des t-Tests! \textbf{(1 Punkt)}
\end{enumerate}
 
\clearpage
% -----------------------------------------------------------------------

\section{Aufgabe \hfill (10 Punkte)}

\textit{Geben Sie grundsätzlich Formeln und Rechenweg zur Lösung der Teilaufgaben mit an!} \\[1Ex]
 

 
%% --------------------------------------------------------------------
\begin{minipage}[t]{0.5\textwidth}
\includegraphics[width = 1.3cm]{/Users/kruppajo/work/GitHub/exam/avatare/Alex.png}\hspace{-4mm}\includegraphics[width = 1.3cm]{/Users/kruppajo/work/GitHub/exam/avatare/Nilufar.png}
\end{minipage}
\begin{minipage}[t]{0.5\textwidth}
\hfill
\href{https://youtu.be/kHmfEmU6lrk}{\includegraphics[width = 2cm]{img/youtube}}
\end{minipage}
%% --------------------------------------------------------------------



\paragraph{Interpretation des gepaarten t-Tests in \Rlogo}

Alles voll mit Kartoffeln. Aber das haben Alex und Nilufar eben gemeinsam in einer Abschlussarbeit gemacht! Worum ging es aber konkret? Beide haben als einen normalverteilten Endpunkt ($Y$) Frischegewicht [kg/ha] von Kartoffeln bestimmt. Die Daten haben beide zusammen in einem Versuch in einer Klimakammer erhoben. In dem Experiment ging es um eine vorher/nachher Untersuchung an den gleichen Kartoffeln. Als Behandlung wurde Ausgeizen ($ctrl$ und $28d$) eingesetzt. Nach der Meinung der Betreuerin muss hier ein gepaarter t-Test gerechnet werden. Leider kennen sich beide nicht sehr gut in \Rlogo aus. Alex hat einiges an Gummibärchen geholt, so dass beide die Zeit gut durchbringen werden. Dann geht Alex nochmal zum Sport. Einfach mal raus um zu Laufen. Ohne Ziel und Uhr. Das ist was für Alex.

\begin{knitrout}
\definecolor{shadecolor}{rgb}{0.969, 0.969, 0.969}\color{fgcolor}\begin{kframe}
\begin{verbatim}
## 
## 	Paired t-test
## 
## data:  Frischegewicht by Ausgeizen
## t = -2.4381, df = 8, p-value = 0.04068
## alternative hypothesis: true  is not equal to [condensed]
## 95 percent confidence interval:
##  -11.0694889  -0.3082888
## sample estimates:
## mean difference 
##       -5.688889
\end{verbatim}
\end{kframe}
\end{knitrout}

Jetzt brauchen Alex und Nilufar Ihre Hilfe bei der Berechnung eines gepaarten t-Tests in \Rlogo um ihre Arbeit dann in diesem Semester noch abschließen zu können.

\begin{enumerate}
  \item Formulieren Sie die wissenschaftliche Fragestellung! \textbf{(1 Punkt)}
  \item Formulieren Sie das statistische Hypothesenpaar! \textbf{(1 Punkt)}
\item Liegt ein signifikanter Unterschied zwischen den Gruppen vor?
  Begründen Sie Ihre Antwort! \textbf{(2 Punkte)}
\item Skizzieren Sie das sich ergebende 95\% Konfidenzintervall! \textbf{(2 Punkte)}
\item Interpretieren Sie den Effekt des gepaarten t-Tests! \textbf{(2 Punkte)}
\item Skizzieren Sie den sich ergebenden Boxplot der Differenzen! Welche Annahmen an die Daten haben Sie getroffen? Begründen Sie Ihre Antwort! \textbf{(2 Punkte)} 
\end{enumerate}
 
\clearpage
% -----------------------------------------------------------------------
\part{Die einfaktorielle \& zweifaktorielle ANOVA}
% -----------------------------------------------------------------------

\section{Aufgabe \hfill (11 Punkte)}

\textit{Geben Sie grundsätzlich Formeln und Rechenweg zur Lösung der Teilaufgaben mit an!} \\[1Ex]
 

 
%% --------------------------------------------------------------------
\begin{minipage}[t]{0.5\textwidth}
\includegraphics[width = 1.3cm]{/Users/kruppajo/work/GitHub/exam/avatare/Nilufar.png}\hspace{-4mm}\includegraphics[width = 1.3cm]{/Users/kruppajo/work/GitHub/exam/avatare/Paula.png}
\end{minipage}
\begin{minipage}[t]{0.5\textwidth}
\hfill
\href{https://youtu.be/kHmfEmU6lrk}{\includegraphics[width = 2cm]{img/youtube}}
\end{minipage}
%% --------------------------------------------------------------------



\paragraph{Visualisierung der einfaktoriellen ANOVA}

'Uff... die einfaktorielle ANOVA. Und wir können jetzt anhand der Visualisuierung sehen, ob da schon was signifikant ist?', Paula hebt die Augenbraue. 'Ja, können wir. Dafür müssen wir aber erstmal in \texttt{\{ggplot\}} uns die Daten anschauen. Oder wir zeichnen es flott mit der Hand. Geht auch.', meint Nilufar dazu. Die beiden hatten sich auf einem Konzert von Deichkind kennengelernt. Paula hatte sich in ein Stallexperiment verschiedene Zandern angeschaut. Dabei ging es herauszufinden, ob es einen Zusammenhang zwischen der Behandlung Ernährungszusatz ($ctrl$, $fedX$ und $getIt$) und dem Messwert Schlachtgewicht [kg] gibt. Später wird noch Star Trek geguckt. Nilufar befürwortet das!

\begin{knitrout}
\definecolor{shadecolor}{rgb}{0.969, 0.969, 0.969}\color{fgcolor}\begin{table}[!h]
\centering
\begin{tabular}{cc}
\toprule
Ernährungszusatz & Schlachtgewicht\\
\midrule
fedX & 44\\
getIt & 30\\
fedX & 43\\
ctrl & 45\\
ctrl & 45\\
\addlinespace
ctrl & 43\\
ctrl & 45\\
ctrl & 46\\
fedX & 44\\
getIt & 28\\
\addlinespace
ctrl & 45\\
getIt & 34\\
fedX & 46\\
getIt & 30\\
fedX & 45\\
\addlinespace
getIt & 29\\
fedX & 45\\
ctrl & 46\\
\bottomrule
\end{tabular}
\end{table}

\end{knitrout}

Leider kennen sich Paula und Nilufar mit Darstellung einer einfaktoriellen ANOVA überhaupt nicht aus. 

\begin{enumerate}
\item Erstellen  Sie  eine  Visualisierung  der  Datentabelle! Beschriften  Sie  die  Abbildung! \textbf{(2 Punkte)}
\item Benennen Sie die Visualisierung mit dem korrekten, statistischen Fachbegriff! \textbf{(1 Punkt)}
\item Zeichnen Sie folgende statistischen Maßzahlen passend ein! 
  \begin{itemize}
  \item Den globalen Mittelwert $\beta_0$ \textbf{(1 Punkt)}
  \item Die Mittelwerte der einzelnen Behandlungsstufen \textbf{(1 Punkt)}
  \item Die Mittelwertsdifferenz der einzelnen Behandlungsstufen mit $\beta_{ctrl}$, $\beta_{fedX}$ und $\beta_{getIt}$ \textbf{(1 Punkt)}
  \item Die Residuen oder Fehler mit $\epsilon$ \textbf{(1 Punkt)}
  \end{itemize}
\item Liegt ein \textit{vermutlicher} signifikanter Unterschied vor? Begründen Sie Ihre Antwort! \textbf{(2 Punkte)}
\item Schätzen Sie die Effekte der Behandlungsstufen! \textbf{(2 Punkte)}
\end{enumerate}
 
\clearpage
% -----------------------------------------------------------------------

\section{Aufgabe \hfill (9 Punkte)}

\textit{Geben Sie grundsätzlich Formeln und Rechenweg zur Lösung der Teilaufgaben mit an!} \\[1Ex]
 

 
%% --------------------------------------------------------------------
\begin{minipage}[t]{0.5\textwidth}
\includegraphics[width = 1.3cm]{/Users/kruppajo/work/GitHub/exam/avatare/Alex.png}\hspace{-4mm}\includegraphics[width = 1.3cm]{/Users/kruppajo/work/GitHub/exam/avatare/Steffen.png}
\end{minipage}
\begin{minipage}[t]{0.5\textwidth}
\hfill
\href{https://youtu.be/IhecxMcCENY}{\includegraphics[width = 2cm]{img/youtube}}
\end{minipage}
%% --------------------------------------------------------------------



\paragraph{Ergebnistabelle der einfaktoriellen ANOVA}

'Als erstes bauen wir uns aus unsere Daten die ANOVA Tabelle dann sehen wir schon, ob unser Gruppenvergleich in der ANOVA signifikant ist.', Alex schaut Steffen fragend an und hofft auf eine positive Regung im Gesicht. Wird aber enttäuscht. Da hilft die Schlange von Steffen auch nur bedingt. Steffen tut sich auch sehr schwer mit der einfaktoriellen ANOVA. Beide waren im Oldenburger Land um ein Gewächshausexperiment mit Erbsen durchzuführen. Dabei ging es herauszufinden, ob es einen Zusammenhang zwischen der Behandlung Bewässerungstypen ($ctrl$, $low$, $mid$ und $high$) und dem Messwert Proteingehalt [g/kg] gibt. Nachher wollen sich beide noch mit dem Hobby Klemmbausteine von Steffen beschäftigen. Kennt Alex noch nicht, klingt aber interessant.



\vspace{1ex}

Leider kennen sich Alex und Steffen mit Berechnung einer einfaktoriellen ANOVA überhaupt nicht aus. Deshalb brauchen beide bei der Erstellung Ihre Hilfe, die Schlange reicht als Hilfe nicht aus! 

\begin{enumerate}
  \item Formulieren Sie die wissenschaftliche Fragestellung! \textbf{(1 Punkt)}
  \item Formulieren Sie das statistische Hypothesenpaar! \textbf{(1 Punkt)}
\item Füllen Sie die unterstehende einfaktorielle ANOVA Ergebnistabelle aus! \textbf{(3 Punkte)}
\end{enumerate}

\vspace{1Ex}

\begin{center}
  \Large
  \begin{tabular}{lccccp{3cm}}
\toprule
     & \textbf{Df} & \textbf{Sum Sq} & \textbf{Mean Sq} & \textbf{F value} & \textbf{Pr(>F)} \strut\\
    \midrule
   \textbf{Bewässerungstypen}  & 3 & 351.68 &  &  &  \strut\\
   \textbf{error}  & 24 &  &  &  &  \strut\\
   \textbf{Total}  & 27 & 1272.68 &  &  &  \strut\\
\bottomrule
  \end{tabular}
\end{center}

\vspace{1Ex}

\begin{enumerate}
  \setcounter{enumi}{3}
\item Schätzen Sie den p-Wert der Tabelle mit $F_{\alpha = 5\%} = 3.01$ ab. Begründen Sie Ihre Antwort! \textbf{(2 Punkte)}
\item Berechen Sie den Effektschätzer $\eta^2$. Was sagt Ihnen der Wert von $\eta^2$ aus? \textbf{(2 Punkte)}
\end{enumerate}



 
\clearpage
% -----------------------------------------------------------------------

\section{Aufgabe \hfill (12 Punkte)}

\textit{Geben Sie grundsätzlich Formeln und Rechenweg zur Lösung der Teilaufgaben mit an!} \\[1Ex]
 

 
%% --------------------------------------------------------------------
\begin{minipage}[t]{0.5\textwidth}
\includegraphics[width = 1.3cm]{/Users/kruppajo/work/GitHub/exam/avatare/Nilufar.png}\hspace{-4mm}\includegraphics[width = 1.3cm]{/Users/kruppajo/work/GitHub/exam/avatare/Paula.png}
\end{minipage}
\begin{minipage}[t]{0.5\textwidth}
\hfill
\href{https://youtu.be/49hvImMwVyE}{\includegraphics[width = 2cm]{img/youtube}}
\end{minipage}
%% --------------------------------------------------------------------



\paragraph{Die einfaktoriellen ANOVA und der Student t-Test}

Nilufar und Paula schauen sich etwas entnervt an. Gemeinsam schreiben die beiden ihre Abschlussarbeit und sollen nun als erstes einmal die Daten mit eine einfaktoriellen ANOVA auswerten damit abgeschätzt werden kann, ob überhaupt signifikante Ergebnisse in den multipen Gruppenvergleichen zu erwarten sind. Deshalb erstmal Smarties mampfen, die Paula mitgebracht hat. Nun möchte erstmal ihre Betreuung der Arbeit eine ANOVA Tabelle sehen. Was immer da auch drin zu erkennen sein mag. Nilufar schaut Paula sehen erstmla gar nichts. Die beiden waren im Oldenburger Land um ein Feldexperiment mit Brokkoli durchzuführen. Dabei haben Nilufar und Paula den Messwert Trockengewicht [kg/ha] unter der Behandung Lüftungssystemen und Folientunneln ($ctrl$, $storm$, $thunder$ und $tornado$) ermittelt. Später wollen die beiden dann noch raus um zu Fechten.



\vspace{1ex}

Leider kennen sich Nilufar und Paula mit Berechnung einer einfaktoriellen ANOVA überhaupt nicht aus. Deshalb brauchen beide bei der Erstellung Ihre Hilfe! 

\begin{enumerate}
  \item Formulieren Sie die wissenschaftliche Fragestellung! \textbf{(1 Punkt)}
  \item Formulieren Sie das statistische Hypothesenpaar! \textbf{(1 Punkt)}
\item Füllen Sie die unterstehende einfaktorielle ANOVA Ergebnistabelle aus! \textbf{(3 Punkte)}
\end{enumerate}

\vspace{1Ex}

\begin{center}
  \Large
  \begin{tabular}{lccccp{3cm}}
\toprule
     & \textbf{Df} & \textbf{Sum Sq} & \textbf{Mean Sq} & \textbf{F value} & \textbf{Pr(>F)} \strut\\
    \midrule
   \textbf{Lüftungssystemen}  & 3 & 3906.54 &  &  &  \strut\\
   \textbf{Error}  & 26 & 471.46 &  &  &  \strut\\
\bottomrule
  \end{tabular}
\end{center}

\vspace{1Ex}

\begin{enumerate}
  \setcounter{enumi}{3}
\item Schätzen Sie den p-Wert der Tabelle mit $F_{\alpha = 5\%} = 2.98$ ab. Begründen Sie Ihre Antwort! \textbf{(2 Punkte)}
\item Was bedeutet ein signifikantes Ergebnis in einer einfaktoriellen ANOVA? \textbf{(1 Punkt)}
\item Berechnen Sie \textit{einen} Student t-Test für den \textit{vermutlich} signifikantesten Gruppenvergleich anhand der untenstehenden Tabelle mit $T_{\alpha = 5\%} = 2.03$. Begründen Sie Ihre Auswahl! \textbf{(3 Punkte)}
\end{enumerate}


\begin{knitrout}
\definecolor{shadecolor}{rgb}{0.969, 0.969, 0.969}\color{fgcolor}\begin{table}[!h]
\centering\begingroup\fontsize{11}{13}\selectfont

\begin{tabular}{cccc}
\toprule
\textbf{Lüftungssystemen} & \textbf{Fallzahl (n)} & \textbf{Mittelwert} & \textbf{Standardabweichung}\\
\midrule
ctrl & 8 & 6.88 & 4.76\\
storm & 8 & 8.88 & 1.73\\
thunder & 6 & 8.83 & 2.23\\
tornado & 8 & 33.88 & 6.17\\
\bottomrule
\end{tabular}
\endgroup{}
\end{table}

\end{knitrout}


\begin{enumerate}
  \setcounter{enumi}{6}
\item Gegebenen der ANOVA Tabelle war das Ergebnis des Student t-Tests zu erwarten? Begründen Sie Ihre Antwort! \textbf{(2 Punkte)}
\end{enumerate}

 
\clearpage
% -----------------------------------------------------------------------

\section{Aufgabe \hfill (9 Punkte)}

\textit{Geben Sie grundsätzlich Formeln und Rechenweg zur Lösung der Teilaufgaben mit an!} \\[1Ex]
 

 
%% --------------------------------------------------------------------
\begin{minipage}[t]{0.5\textwidth}
\includegraphics[width = 1.3cm]{/Users/kruppajo/work/GitHub/exam/avatare/Mark.png}
\end{minipage}
\begin{minipage}[t]{0.5\textwidth}
\hfill
\href{https://youtu.be/aXvxGC4YLqk}{\includegraphics[width = 2cm]{img/youtube}}
\end{minipage}
\vspace{-3Ex}
%% --------------------------------------------------------------------



\paragraph{Die einfaktorielle ANOVA in \Rlogo}

'Uff... die einfaktorielle ANOVA und \Rlogo. Nicht so einfach... Was sagt mir jetzt die Ausgabe der ANOVA und wo sehe ich, ob da was signifikant ist?', denkt Mark und hebt die Augenbraue. Mark hatte sich ein Gewächshausexperiment mit Erbsen angeschaut. Als wäre das nicht alles schon schwer genug. Wenn die Unsicherheit nicht wäre, ja dann wäre wohl vieles möglich für Mark! Aber so.. Dabei ging es beim Experiment herauszufinden, ob es einen Zusammenhang zwischen der Behandlung Lichtstufen ($none$, $200lm$, $400lm$ und $600lm$) und dem Messwert Frischegewicht [kg/ha] gibt. Nun möchte sein Betreuer seiner Hausarbeit erstmal eine ANOVA sehen und die Ergebnisse präsentiert bekommen. Und eigentlich will er ja was anderes... Schon dutzende Male gesehen: Columbo. Aber immer noch großartig zusammen mit Marzipankugeln.

\begin{knitrout}
\definecolor{shadecolor}{rgb}{0.969, 0.969, 0.969}\color{fgcolor}\begin{kframe}
\begin{verbatim}
## Analysis of Variance Table
## 
## Response: Frischegewicht
##             Df Sum Sq Mean Sq F value    Pr(>F)
## Lichtstufen  3 5495.1 1831.69  72.254 5.072e-13
## Residuals   27  684.5   25.35
\end{verbatim}
\end{kframe}
\end{knitrout}

\vspace{1ex}

Leider kennen sich Mark mit Berechnung einer einfaktoriellen ANOVA überhaupt nicht aus. Deshalb braucht er bei der Erstellung Ihre Hilfe! 

\begin{enumerate}
  \item Formulieren Sie die wissenschaftliche Fragestellung! \textbf{(1 Punkt)}
  \item Formulieren Sie das statistische Hypothesenpaar! \textbf{(1 Punkt)}
\item Interpretieren Sie das Ergebnis der einfaktoriellen ANOVA! \textbf{(2 Punkte)} 
\item Berechnen Sie den Effektschätzer $\eta^2$. Was sagt Ihnen der Wert von $\eta^2$ aus? \textbf{(2 Punkte)}
\item Skizzieren Sie eine Abbildung, der dem obigen Ergebnis der
  einfaktoriellen ANOVA näherungsweise entspricht! \textbf{(3 Punkte)}
\end{enumerate}

 
\clearpage
% -----------------------------------------------------------------------

\section{Aufgabe \hfill (12 Punkte)}

\textit{Geben Sie grundsätzlich Formeln und Rechenweg zur Lösung der Teilaufgaben mit an!} \\[1Ex]
 

 
%% --------------------------------------------------------------------
\begin{minipage}[t]{0.5\textwidth}
\includegraphics[width = 1.3cm]{/Users/kruppajo/work/GitHub/exam/avatare/Steffen.png}
\end{minipage}
\begin{minipage}[t]{0.5\textwidth}
\hfill
\href{https://youtu.be/8Pb2sKUIMyk}{\includegraphics[width = 2cm]{img/youtube}}
\end{minipage}
\vspace{-3Ex}
%% --------------------------------------------------------------------



\paragraph{Ergebnistabelle der zweifaktoriellen ANOVA}

Wie absolut ärgerlich. Jetzt stellt sich tatsächlich heraus, dass seinem Betreuer keine Ahnung von der zweifaktoriellen ANOVA hat. Woher soll Steffen jetzt das Wissen nehmen? Steffen mampft aus Frust noch eine Handvoll Oreos. Immerhin muss er ja noch mit seiner Abschlussarbeit dieses Jahr fertig werden. In ein Kreuzungsexperiment hatte er Schweinen mit der Behandlung Ernährungszusatz ($ctrl$, $fedX$ und $getIt$) sowie der Behandlung Genotypen ($AA$ und $BB$) im Oldenburger Land untersucht. Es wurde als Messwert Schlachtgewicht [kg] bestimmt. Jetzt muss er erstmal die zweifaktorielle ANOVA verstehen. Und eigentlich wollte Steffen doch noch zum Sport! Steffen will später nochmal raus um zu Ringen. Druck ablassen, dass muss er auch.



\vspace{1ex}

Leider kennen sich Steffen mit Berechnung einer zweifaktoriellen ANOVA überhaupt nicht aus. Deshalb braucht er bei der Erstellung Ihre Hilfe! 

\begin{enumerate}
  \item Formulieren Sie die wissenschaftliche Fragestellung! \textbf{(1 Punkt)}
  \item Formulieren Sie das statistische Hypothesenpaar! \textbf{(1 Punkt)}
\item Füllen Sie die unterstehende einfaktorielle ANOVA Ergebnistabelle aus! \textbf{(3 Punkte)}
\end{enumerate}

\vspace{1Ex}

\begin{center}
  \Large
  \begin{tabular}{lccccc}
  \toprule
     & \textbf{Df} & \textbf{Sum Sq} & \textbf{Mean Sq} & \textbf{F value} & \textbf{Pr(>F)} \strut\\
    \midrule
   \textbf{Ernährungszusatz}  & 3 & 315.8 &  &  &  \strut\\
    \textbf{Genotypen}  & 1 & 260.72 &  &  &  \strut\\
    \textbf{Ernährungszusatz:Genotypen}  & 3 & 590.25 &  &  &  \strut\\
   \textbf{Error}  & 18 & 246.42 &  &  &  \strut\\
\bottomrule
  \end{tabular}
\end{center}

\vspace{1Ex}

\begin{enumerate}
  \setcounter{enumi}{3}
\item Schätzen Sie den p-Wert der Tabelle ab. Begründen Sie Ihre
  Antwort! \textbf{(3 Punkte)}
\end{enumerate}
  
\begin{center}
    \Large
\begin{tabular}{lc}
  \toprule
     & $\boldsymbol{F_{\alpha = 5\%}}$ \\
\midrule
  \textbf{Ernährungszusatz} & $4.26$ \\
  \textbf{Genotypen} & $3.40$ \\
  \textbf{Ernährungszusatz:Genotypen} & $5.23$ \\
  \bottomrule
  \end{tabular}
\end{center}

\begin{enumerate}
  \setcounter{enumi}{4}
\item Was bedeutet ein signifikantes Ergebnis in einer zweifaktoriellen ANOVA? \textbf{(2 Punkte)}
\item Was sagt der Term \textit{Ernährungszusatz:Genotypen} aus? Interpretieren Sie das Ergebnis! \textbf{(2 Punkte)}
\end{enumerate}
 
\clearpage
% -----------------------------------------------------------------------

\section{Aufgabe \hfill (10 Punkte)}

\textit{Geben Sie grundsätzlich Formeln und Rechenweg zur Lösung der Teilaufgaben mit an!} \\[1Ex]
 

 
%% --------------------------------------------------------------------
\begin{minipage}[t]{0.5\textwidth}
\includegraphics[width = 1.3cm]{/Users/kruppajo/work/GitHub/exam/avatare/Jessica.png}
\end{minipage}
\begin{minipage}[t]{0.5\textwidth}
\hfill
\href{https://youtu.be/rWTyHXXlYjY}{\includegraphics[width = 2cm]{img/youtube}}
\end{minipage}
\vspace{-3Ex}
%% --------------------------------------------------------------------



\paragraph{Die zweifaktorielle ANOVA in \Rlogo}

Es ist schon kurz nach fünf und Jessica wird langsam nervös. Jessica wollte heute Abend noch ihre E-Sport Qualifikation schauen. Stattdessen versucht ihre Betreuerin die Ausgabe der zweifaktoriellen ANOVA zu visualieren und zu überprüfen, ob es mit der Visualisierung der Daten als Boxplots zusammenpasst. Jessica hatte im Oldenburger Land einen Versuch in einer Klimakammer mit Erbsen durchgeführt. Es gab dabei zwei Behandlungen. Einmal Bewässerungstypen ($ctrl$, $low$, $mid$ und $high$) sowie als zweite Behandlung Lichtstufen ($none$, und $600lm$). Gemessen wurde der Messwert ($Y$) Proteingehalt [g/kg]. So kompliziert kann das jetzt doch nicht sein! Eigentlich wollte Jessica nachher noch zum Sport. Jessica will später nochmal raus um Rad zu fahren. Druck ablassen, dass muss sie auch.

\begin{knitrout}
\definecolor{shadecolor}{rgb}{0.969, 0.969, 0.969}\color{fgcolor}\begin{kframe}
\begin{verbatim}
## Analysis of Variance Table
## 
## Response: Proteingehalt
##                               Df Sum Sq Mean Sq F value   Pr(>F)
## Bewässerungstypen              2 324.59 162.297 23.2371 1.03e-05
## Lichtstufen                    1   0.11   0.108  0.0154 0.902570
## Bewässerungstypen:Lichtstufen  2 216.47 108.233 15.4964 0.000122
## Residuals                     18 125.72   6.984
\end{verbatim}
\end{kframe}
\end{knitrout}

\vspace{1ex}

Leider kennt sich Jessica mit Berechnung einer zweifaktoriellen ANOVA überhaupt nicht aus. Deshalb braucht sie bei der Erstellung Ihre Hilfe! 

\begin{enumerate}
  \item Formulieren Sie die wissenschaftliche Fragestellung! \textbf{(1 Punkt)}
  \item Formulieren Sie das statistische Hypothesenpaar! \textbf{(1 Punkt)}
\item Interpretieren Sie das Ergebnis der einfaktoriellen ANOVA! \textbf{(3 Punkte)} 
\item Zeichnen Sie eine Abbildung, der dem obigen Ergebnis der
  zweifaktoriellen ANOVA näherungsweise entspricht! \textbf{(5 Punkte)}
\end{enumerate}
 
\clearpage
% -----------------------------------------------------------------------

\section{Aufgabe \hfill (12 Punkte)}

\textit{Geben Sie grundsätzlich Formeln und Rechenweg zur Lösung der Teilaufgaben mit an!} \\[1Ex]
 

 
%% --------------------------------------------------------------------
\begin{minipage}[t]{0.5\textwidth}
\includegraphics[width = 1.3cm]{/Users/kruppajo/work/GitHub/exam/avatare/Tina.png}
\end{minipage}
\begin{minipage}[t]{0.5\textwidth}
\hfill
\href{https://youtu.be/FjjJXkFJfIY}{\includegraphics[width = 2cm]{img/youtube}}
\end{minipage}
\vspace{-3Ex}
%% --------------------------------------------------------------------



\paragraph{Zusammenhang zwischen der ANOVA und dem t-Test}

Die Spinne dreht durch und verwüstet Tinas Palme zu kleinen Schnetzeln. Aber dafür hat sie jetzt keine Zeit. Tina muss verstehen wie die Formeln der ANOVA und des t-Tests miteinander zusammen hängen und was das verbindene Konzept ist. Tina dreht Tocotronic lauter, damit die Spinne sie nicht mehr stört. Die Palme leidet still. Was hat Tina eigentlich gemacht? In ein Stallexperiment wurden Lamas mit der Behandlung Flüssignahrung ($ctrl$, $superIn$ und $flOw$) sowie der Behandlung Lüftungssystem ($keins$ und $thunder$) untersucht. Das hilft der Palme auch nicht mehr. Aber das ist nicht das einzige Problem von Tina. Tina und die Wut, eine unendliche Geschichte mit kniffeligen Wendungen.

\begin{graybox}{Gegebene Formeln}
\begin{center}
  \begin{tabular}{cc}
    $F_{D} = \cfrac{MS_{treatment}}{MS_{error}}$ & $T_{D} = \cfrac{\bar{y}_1 - \bar{y}_2}{s_p \cdot \sqrt{2/n_g}}$\\
  \end{tabular}
\end{center}
\end{graybox}

Leider kennen sich Tina mit dem Zusammenhang zwischen der ANOVA und dem t-Test nicht aus. Deshalb braucht sie bei der Erstellung Ihre Hilfe! 

\begin{enumerate}
\item Welche statistische Maßzahl testet der t-Test, welche die ANOVA? Begründen Sie Ihre Antwort! \textbf{(2 Punkte)}
\item Erklären Sie den Zusammenhang zwischen der $F_{D}$ Statistik und $T_{D}$ Statistik! \textbf{(2 Punkte)}
\item Visualisieren Sie in einer 2x2 Tafel den Zusammenhang von $MS_{treatment}$ und $MS_{error}$! \textbf{(2 Punkte)}
\item Beschriften Sie die erstellte 2x2 Tafel mit \underline{signifikant} und \underline{nicht signifikant}! Begründen Sie Ihre Antwort! \textbf{(2 Punkte)}
\item Nennen Sie das numerische Minimum der F-Statistik $F_D$! Begründen Sie Ihre Antwort! \textbf{(2 Punkte)}
\item Wenn die F-Statistik $F_D$ minimal ist, welche Aussage erhalten Sie über die Nullhypothese? Begründen Sie Ihre Antwort! \textbf{(2 Punkte)}
\end{enumerate}

 
\clearpage
% -----------------------------------------------------------------------

\section{Aufgabe \hfill (11 Punkte)}

\textit{Geben Sie grundsätzlich Formeln und Rechenweg zur Lösung der Teilaufgaben mit an!} \\[1Ex]
 

 
%% --------------------------------------------------------------------
\begin{minipage}[t]{0.5\textwidth}
\includegraphics[width = 1.3cm]{/Users/kruppajo/work/GitHub/exam/avatare/Jessica.png}
\end{minipage}
\begin{minipage}[t]{0.5\textwidth}
\hfill
\href{https://youtu.be/2qG1Dws0MJo}{\includegraphics[width = 2cm]{img/youtube}}
\end{minipage}
\vspace{-3Ex}
%% --------------------------------------------------------------------



\paragraph{Interaktion in der zweifaktoriellen ANOVA}

In einen Versuch in einer Klimakammer wurden Erdbeeren mit der Behandlung Düngestufen ($ctrl$, $low$, $mid$ und $high$) sowie der Behandlung Substrattypen ($torf$, $70p30n$) untersucht. Es wurde als Messwert Chlorophyllgehalt (SPAD-502Plus) [SPAD] bestimmt. Jetzt starrt Jessica mit auf die \Rlogo Ausgabe einer zweifaktoriellen ANOVA. Leider starrt ihr Betreuer in der gleichen Art Jessica zurück an. Es liegt anscheinend eine signifikante Interaktion vor? 'Das wird ein langer Nachmmittag.', denkt sie sich und kreuselt ihren Mund. 'Und was machen wir jetzt?' entfährt es ihr überrascht entnervt. Immerhin war geht es ja um ihre Hausarbeit. Jessica hätte doch nichts mit Erdbeeren machen sollen. Erdbeeren -- was soll das auch bedeutendes sein? Eigentlich wollte Jessica nachher noch einen Film schauen. Das Verrückte ist, dass die Hündin Herr der Ringe wirklich liebt. Das ist Jessica sehr recht, denn sie braucht Entspannung.

\vspace{1ex}

Leider kennen sich Jessica und ihr Betreuer mit der zweifaktoriellen ANOVA überhaupt nicht aus. Geschweige denn mit der Interpretation einer Interaktion. Deshalb braucht sie bei der Erstellung Ihre Hilfe, sonst wird es heute Abend mit seinem Hobby Warhammer nichts mehr! 

\begin{enumerate}
\item Visualisieren Sie folgende mögliche Interaktionen zwischen den Behandlungen! Beschriften Sie die Abbildung! \textbf{(4 Punkte)}
\begin{enumerate}
\item \underline{Keine} Interaktion liegt vor.
\item Eine \underline{schwache} Interaktion liegt vor. 
\item Eine \underline{starke} Interaktion liegt vor. 
\end{enumerate}
\item Erklären Sie den Unterschied zwischen den verschiedenen Interaktionen! \textbf{(2 Punkte)}
\item Welche statistische Maßzahl betrachten Sie für die Bewertung der Interaktion? \textbf{(1 Punkt)}
\item Skizzieren Sie die notwendigen Funktionen in \Rlogo für eine Post-hoc Analyse! \textbf{(2 Punkte)} 
\item Wenn eine signifikante Interaktion in den Daten vorliegt, wie ist dann das weitere Vorgehen? Berücksichtigen Sie auch die Funktion \texttt{emmeans()}! \textbf{(2 Punkte)}
\end{enumerate}

 
\clearpage
% -----------------------------------------------------------------------

\section{Aufgabe \hfill (11 Punkte)}

\textit{Geben Sie grundsätzlich Formeln und Rechenweg zur Lösung der Teilaufgaben mit an!} \\[1Ex]
 

 
%% --------------------------------------------------------------------
\begin{minipage}[t]{0.5\textwidth}
\includegraphics[width = 1.3cm]{/Users/kruppajo/work/GitHub/exam/avatare/Yuki.png}
\end{minipage}
\begin{minipage}[t]{0.5\textwidth}
\hfill
\href{https://youtu.be/M9Uhm67ndxM}{\includegraphics[width = 2cm]{img/youtube}}
\end{minipage}
\vspace{-3Ex}
%% --------------------------------------------------------------------



\paragraph{Zusammenhang zwischen der ANOVA und dem Post-hoc-Test}

'Mit der einfaktoriellen ANOVA lassen sich flott die Gruppen in einer Behandlungen vergleichen, wenn wir normalverteilte Daten und Varianzhomogenität vorliegen haben!', ihr Betreuer scheint die einfaktoriellen ANOVA zu verstehen. Warum jetzt sie jetzt nochmal alles wiederkäuen muss, wird Yuki echt nicht so klar. Wenn es doch so klar ist? 'Wir haben jetzt bei der ANOVA einen p-Wert mit 0.071 raus sowie eine F-Statistik $F_D$ mit 1.61 berechnet. Nach den Boxplots müsste sich eigentlich ein Unterschied zwischen $mid$ und $ctrl$ ergeben. Der Unterschied ist in \texttt{\{emmeans\}} auch signifikant mit einem p-Wert von 0.045. Wie kann das sein?', fragt Yuki etwas provokant und dreht London Grammar leiser. Yuki war in der Uckermark und hatte dort einen Versuch in einer Klimakammer mit Maiss durchgeführt. Die Komune wo sie untergekommen war, war cool gewesen. Dort gab es selbstgemachte Reese's Peanut Butter Cups aus Vollkorn! Nur jetzt muss eben das Experiment fertig ausgewertet werden. Yuki hatte eine Behandlungen Bewässerungstypen ($ctrl$, $low$, $mid$ und $high$) auf Maiss angewendet. Gemessen wurde der Messwert ($Y$) Trockengewicht [kg/ha]. Dabei wurden die Daten $D$ erhoben. Jetzt muss das hier zu einem Ende kommen! Yuki hat schon genug Probleme. Wenn die Faulheit nicht wäre, dann wäre es einfacher.

\begin{graybox}{Gegebene Formeln}
\begin{center}
  \begin{tabular}{ccc}
    $MS_{treatment} = \cfrac{SS_{treatment}}{df_{treatment}}$ &
    $MS_{error} = \cfrac{SS_{error}}{df_{error}}$ &
    $F_{D} = \cfrac{MS_{treatment}}{MS_{error}}$ \\
  \end{tabular}
\end{center}
\end{graybox}

Leider kennen sich Yuki und ihr Betreuer mit der Interpretation einer ANOVA überhaupt nicht aus. Deshalb braucht sie bei der Erstellung Ihre Hilfe und die Zeit wird knapp. 

\begin{enumerate}
  \item Formulieren Sie die wissenschaftliche Fragestellung! \textbf{(1 Punkt)}
  \item Formulieren Sie das statistische Hypothesenpaar! \textbf{(1 Punkt)}
\item Was bedeutet eine signifkante ANOVA für die beobachteten Daten $D$? \textbf{(1 Punkt)}
\item Visualisieren Sie den Unterschied zwischen Varianzhomogenität und Varianzheterogenität anhand der Daten $D$! Beschriften Sie die Abbildung! \textbf{(2 Punkte)} 
\item Visualisieren Sie für die Daten $D$ die Verletzung der Annahme der Varianzhomogenität der ANOVA unter zu Hilfenahme von Boxplots! Beschriften Sie die Abbildung! \textbf{(2 Punkte)}
\item Welche Auswirkung hat die Verletzung der Annahme der Varianzhomogenität für die Teststatistik $F_D$ der ANOVA? Begründen Sie Ihre Antwort! \textbf{(2 Punkte)}
\item Erklären Sie abschließend die Diskrepanz zwischen den Ergebnis der ANOVA und dem paarweisen Gruppenvergleich in \texttt{\{emmeans\}}! \textbf{(2 Punkte)}
\end{enumerate}

 
\clearpage
% -----------------------------------------------------------------------
\part{Multiple Gruppenvergleiche}
% ----------------------------------------------------------------------- 

\section{Aufgabe \hfill (12 Punkte)}

\textit{Geben Sie grundsätzlich Formeln und Rechenweg zur Lösung der Teilaufgaben mit an!} \\[1Ex]
 

 
%% --------------------------------------------------------------------
\begin{minipage}[t]{0.5\textwidth}
\includegraphics[width = 1.3cm]{/Users/kruppajo/work/GitHub/exam/avatare/Alex.png}\hspace{-4mm}\includegraphics[width = 1.3cm]{/Users/kruppajo/work/GitHub/exam/avatare/Jessica.png}
\end{minipage}
\begin{minipage}[t]{0.5\textwidth}
\hfill
\href{https://youtu.be/kHmfEmU6lrk}{\includegraphics[width = 2cm]{img/youtube}}
\end{minipage}
%% --------------------------------------------------------------------



\paragraph{Adjustierung multipler Vergleiche}

In einen Versuch in einer Klimakammer mit Spargel wurde die Behandlung Genotypen ($00$, $AA$, $AB$ und $BB$) gegen die Ergebnisse einer früheren Studie von Almar et al. (2012) verglichen. Im Rahmen des Experiments haben Jessica und Alex verschiedene Student t-Tests für den Mittelwertsvergleich für den Messwert Frischegewicht [kg/ha] gerechnet. Es ergab sich dann die folgende Tabelle der rohen p-Werte für die Vergleiche zu Almar et al. (2012). Jetzt sollen die beiden einmal schauen, was in den Daten so drin ist.

\begin{knitrout}
\definecolor{shadecolor}{rgb}{0.969, 0.969, 0.969}\color{fgcolor}\begin{table}[!h]
\centering\begingroup\fontsize{10}{12}\selectfont

\begin{tabular}{ccc}
\toprule
\textbf{Rohen p-Werte} & \textbf{Adjustierte p-Werte} & \textbf{Nullhypothese ablehnen?}\\
\midrule
0.340 &  & \\
0.230 &  & \\
0.030 &  & \\
0.001 &  & \\
\bottomrule
\end{tabular}
\endgroup{}
\end{table}

\end{knitrout}

Leider kennen sich Jessica und Alex mit der Adjustierung von $p$-Werten und dem Signifikanzniveau $\alpha$ überhaupt nicht aus. Deshalb brauchen die beiden bei der Erstellung Ihre Hilfe!

\begin{enumerate}
  \item Formulieren Sie die wissenschaftliche Fragestellung! \textbf{(1 Punkt)}
  \item Formulieren Sie die statistischen Hypothesen! \textbf{(1 Punkt)}
\item Füllen Sie die Spalte \textit{Adjustierte p-Werte} nach der Bonferoni-Methode aus! \textbf{(2 Punkte)}
\item Entscheiden Sie, ob nach der Adjustierung die Nullhypothese abgelehnt werden kann! Begründen Sie Ihre Antwort! \textbf{(2 Punkte)}
\item Wie ist Ihr Vorgehen, wenn Sie anstatt der $p$-Werte das Signifikanzniveau $\alpha$ adjustieren? \textbf{(2 Punkte)}
\item Erklären Sie warum die $p$-Werte oder das Signifikanzniveau $\alpha$ bei multiplen Vergleichen adjustiert werden müssen! \textbf{(2 Punkte)}
\item Würden Sie die Adjustierung der $p$-Werte oder die Adjustierung des Signifikanzniveaus $\alpha$ vorziehen? Begründen Sie Ihre Antwort! \textbf{(2 Punkte)}
\end{enumerate}


 
\clearpage
% ----------------------------------------------------------------------- 

\section{Aufgabe \hfill (10 Punkte)}

\textit{Geben Sie grundsätzlich Formeln und Rechenweg zur Lösung der Teilaufgaben mit an!} \\[1Ex]
 

 
%% --------------------------------------------------------------------
\begin{minipage}[t]{0.5\textwidth}
\includegraphics[width = 1.3cm]{/Users/kruppajo/work/GitHub/exam/avatare/Paula.png}
\end{minipage}
\begin{minipage}[t]{0.5\textwidth}
\hfill
\href{https://youtu.be/xq29O8qDrg0}{\includegraphics[width = 2cm]{img/youtube}}
\end{minipage}
\vspace{-3ex}
%% --------------------------------------------------------------------



\paragraph{Visualisierung des Compact Letter Displays (CLD)}

Paula betrachtet in sich gekehrt die Poster vor dem Büro von ihrem Betreuer. Viele der explorativen Abbildungen sagen ihr etwas. Die Barplots und die Boxplots könnte sie dann schon nachbauen. Das macht sie dann zuversichtlich die Abschlussarbeit auch hinzukriegen. Etwas komischer sind die seltsamen Buchstaben über den Barplots. Paula betrachtet ein Poster das sich mit Puten beschäftigt. Genotypen ($00$, $AA$, $AB$ und $BB$) und Schlachtgewicht [kg] wurden dort bestimmt. So richtig schlau, wird sie daraus nicht.

\begin{knitrout}
\definecolor{shadecolor}{rgb}{0.969, 0.969, 0.969}\color{fgcolor}\begin{table}[!h]
\centering\begingroup\fontsize{10}{12}\selectfont

\begin{tabular}{cc}
\toprule
\textbf{Behandlung} & \textbf{Compact letter display}\\
\midrule
00 & a\\
AA & b\\
AB & a\\
BB & b\\
\bottomrule
\end{tabular}
\endgroup{}
\end{table}

\end{knitrout}

Leider kennen sich Paula mit dem \textit{Compact letter display (CLD)} überhaupt nicht aus. Deshalb braucht sie bei der Erstellung Ihre Hilfe!

\begin{enumerate}
  \item Formulieren Sie die wissenschaftliche Fragestellung! \textbf{(1 Punkt)}
  \item Formulieren Sie die statistischen Hypothesen! \textbf{(1 Punkt)}
\item Zeichnen Sie die sich anhand des \textit{Compact letter display (CLD)} ergebenden Barplots! \textbf{(2 Punkte)}
\item Ergänzen Sie das \textit{Compact letter display (CLD)} zu den Barplots! \textbf{(1 Punkt)}
\item Erklären Sie \textit{einen} Vorteil und \textit{einen} Nachteil des \textit{Compact letter display (CLD)}! \textbf{(2 Punkte)}
\item Erstellen Sie eine Matrix mit den paarweisen $p$-Werten eines Student t-Tests, die sich näherungsweise aus dem \textit{Compact letter display (CLD)} ergeben würde! Begründen Sie Ihre Antwort! \textbf{(3 Punkte)}
\end{enumerate}

 
\clearpage
% ----------------------------------------------------------------------- 

\section{Aufgabe \hfill (12 Punkte)}

\textit{Geben Sie grundsätzlich Formeln und Rechenweg zur Lösung der Teilaufgaben mit an!} \\[1Ex]
 

 
%% --------------------------------------------------------------------
\begin{minipage}[t]{0.5\textwidth}
\includegraphics[width = 1.3cm]{/Users/kruppajo/work/GitHub/exam/avatare/Jessica.png}
\end{minipage}
\begin{minipage}[t]{0.5\textwidth}
\hfill
\href{https://youtu.be/RagTFFKFbFg}{\includegraphics[width = 2cm]{img/caution}}
\end{minipage}
\vspace{-3ex}
%% --------------------------------------------------------------------



\paragraph{Berechnung des Compact Letter Displays (CLD) anhand von t-Tests}

Jessica hatte in die Projektbericht ein Kreuzungsexperiment durchgeführt. Soweit so gut. Dabei hat sie sich mit Lamas beschäftigt. Angeblich der neueste heiße Kram... aber das ist wiederum was anderes. So richtig mitgenommen hat Jessica das Thema dann doch nicht. Hat sie sich doch mit Bestandsdichte ($effizient$, $standard$, $eng$ und $kontakt$) und Protein/Fettrate [\%/kg] schon eine Menge an Daten angeschaut. Nach ihr Betreuer soll sie nun ein CLD bestimmen. Weder weiß sie was ein CLD ist, noch war ihr erster Gedanke mit Köln und die LGBTQ Community richtig... Als erstes solle sie die Gruppen nach absteigender Effektstärke sortieren. Was immer das jetzt bringen soll.

\begin{knitrout}
\definecolor{shadecolor}{rgb}{0.969, 0.969, 0.969}\color{fgcolor}\begin{table}[!h]
\centering\begingroup\fontsize{10}{12}\selectfont

\begin{tabular}{cccc}
\toprule
\textbf{Bestandsdichte} & \textbf{Fallzahl (n)} & \textbf{Mittelwert} & \textbf{Standardabweichung}\\
\midrule
effizient & 9 & 5.17 & 2.40\\
standard & 8 & 11.65 & 2.64\\
eng & 8 & 5.15 & 2.16\\
kontakt & 9 & 14.27 & 4.12\\
\bottomrule
\end{tabular}
\endgroup{}
\end{table}

\end{knitrout}

Leider kennen sich Jessica mit dem \textit{Compact letter display (CLD)} überhaupt nicht aus. Deshalb braucht sie bei der Erstellung Ihre Hilfe!

\begin{enumerate}
  \item Formulieren Sie die wissenschaftliche Fragestellung! \textbf{(1 Punkt)}
  \item Formulieren Sie die statistischen Hypothesen! \textbf{(1 Punkt)}
\item Zeichnen Sie die sich ergebenden Barplots! \textbf{(1 Punkt)}
\item Berechnen Sie die Matrix der $p$-Werte anhand von Student t-Tests! \textbf{(4 Punkte)}
\item Ergänzen Sie das \textit{Compact letter display (CLD)} zu den gezeichneten Barplots! Begründen Sie Ihre Antwort! \textbf{(4 Punkte)}
\item Interpretieren Sie das \textit{Compact letter display (CLD)} für Jessica und Alex! \textbf{(1 Punkt)} 
\end{enumerate}

 
\clearpage
% -----------------------------------------------------------------------

\section{Aufgabe \hfill (10 Punkte)}

\textit{Geben Sie grundsätzlich Formeln und Rechenweg zur Lösung der Teilaufgaben mit an!} \\[1Ex]
 

 
%% --------------------------------------------------------------------
\begin{minipage}[t]{0.5\textwidth}
\includegraphics[width = 1.3cm]{/Users/kruppajo/work/GitHub/exam/avatare/Mark.png}
\end{minipage}
\begin{minipage}[t]{0.5\textwidth}
\hfill
\href{https://youtu.be/RagTFFKFbFg}{\includegraphics[width = 2cm]{img/youtube}}
\end{minipage}
\vspace{-3ex}
%% --------------------------------------------------------------------



\paragraph{Berechnung des Compact Letter Displays (CLD) anhand der Matrix der p-Werte}

'Oh, nee!', ruft Mark aus und rollt entnervt mit seinen Augen. Mark hatte seine gesamte Analyse in SPSS gerechnet. Das war ja auch alles in Ordnung. Abbilungen haben geklappt und auch die statistischen Tests gingen dann irgendwie doch. Aber das CLD nicht. Mark findet einfach keine Möglichkeit ein CLD in SPSS zu erhalten. Aber sein Betreuer möchte unbedingt ein CLD. Sonst wird es mit der Abgabe nichts. Dabei hatte er schon wirklich eine Menge gemacht! Mark hatte sich zwei Variablen mit Genotypen ($00$, $AA$, $AB$ und $BB$) und Schlachtgewicht [kg] in einen Leistungssteigerungsversuch mit Hühnern angeschaut. Wo kriegt er jetzt ein CLD her? Dann eben per Hand aus der Matrix der $p$-Wert. Mark stöhnt...

\begin{knitrout}
\definecolor{shadecolor}{rgb}{0.969, 0.969, 0.969}\color{fgcolor}\begin{table}[!h]
\centering\begingroup\fontsize{10}{12}\selectfont

\begin{tabular}{>{}lcccc}
\toprule
\textbf{ } & \textbf{00} & \textbf{AA} & \textbf{AB} & \textbf{BB}\\
\midrule
\textbf{00} & 1.0000000 & 0.0030905 & 0.4299008 & 0.0069354\\
\textbf{AA} & 0.0030905 & 1.0000000 & 0.0186585 & 0.8142810\\
\textbf{AB} & 0.4299008 & 0.0186585 & 1.0000000 & 0.0371324\\
\textbf{BB} & 0.0069354 & 0.8142810 & 0.0371324 & 1.0000000\\
\bottomrule
\end{tabular}
\endgroup{}
\end{table}

\end{knitrout}

Leider kennen sich Mark mit dem \textit{Compact letter display (CLD)} überhaupt nicht aus. Deshalb braucht er bei der Erstellung Ihre Hilfe!

\begin{enumerate}
  \item Formulieren Sie die wissenschaftliche Fragestellung! \textbf{(1 Punkt)}
  \item Formulieren Sie die statistischen Hypothesen! \textbf{(1 Punkt)}
\item Zeichnen Sie die sich anhand der Matrix der $p$-Werte ergebenden Barplots! \textbf{(2 Punkte)}
\item Ergänzen Sie das \textit{Compact letter display (CLD)}! Begründen Sie Ihre Antwort! \textbf{(4 Punkte)}
\item Interpretieren Sie das \textit{Compact letter display (CLD)} für Mark und Alex! \textbf{(2 Punkte)} 
\end{enumerate}

 
\clearpage
% -----------------------------------------------------------------------
\part{Der Chi-Quadrat-Test \& Der diagnostische Test}
% -----------------------------------------------------------------------

\section{Aufgabe \hfill (10 Punkte)}

\textit{Geben Sie grundsätzlich Formeln und Rechenweg zur Lösung der Teilaufgaben mit an!} \\[1Ex]
 

 
%% --------------------------------------------------------------------
\begin{minipage}[t]{0.5\textwidth}
\includegraphics[width = 1.3cm]{/Users/kruppajo/work/GitHub/exam/avatare/Jessica.png}
\end{minipage}
\begin{minipage}[t]{0.5\textwidth}
\hfill
\href{https://youtu.be/-Kva5wc5Elw}{\includegraphics[width = 2cm]{img/youtube}}
\end{minipage}
\vspace{-3Ex}
%% --------------------------------------------------------------------



\paragraph{Den Chi-Quadrat-Test berechnen}

Am Ende war es für Jessica in ihrer Abschlussarbeit dann doch kein normalverteiltes Outcome. Das was jetzt etwas doof, da er sich auf eine ANOVA gefreut hatte. Dann noch schnell David Bowie auf das Ohr und los gehts. Prinzipiell ginge das auch irgendwie, aber nun möchte ihr Betreuer gerne einen $\mathcal{X}^2$-Test auf einer $2x2$-Kreuztabelle berechnet bekommen. Jessica hatte sich in ein Stallexperiment $n = 142$ Beobachtungen von Zandern angeschaut. Dabei hat sie als Behandlung \textit{Automatische Fütterung [ja/nein]} bestimmt und zum anderen die Variable \textit{Schlachtgewicht im Zielbereich [ja/nein]} ermittelt. Jetzt muss Jessica mal schauen, wie sie das jetzt rechnet. Eigentlich wollte Jessica nachher noch zum Sport. Jessica will später nochmal raus um Rad zu fahren. Druck ablassen, dass muss sie auch.

\vspace{5Ex}

\begin{center}
  \huge
  \begin{tabular}{c|l|l|c}
     & \phantom{\textbf{Erkrankt (ja)}} & \phantom{\textbf{Erkrankt (ja)}} & \phantom{\textbf{Erkrankt (ja)}} \strut\\
    \hline
    \phantom{\textbf{Pestizid (ja)}} & 56  & 11  &     \strut\\
    \hline
    \phantom{\textbf{Pestizid (ja)}} & 23  & 52  &      \strut\\
    \hline
     \phantom{100} & \phantom{100}  & \phantom{100}  &  \phantom{100}  \strut\\
  \end{tabular}
\end{center}

\vspace{5Ex}

Leider kennt sich Jessica mit der Berechnung eines $\mathcal{X}^2$-Test für kategoriale Daten überhaupt nicht aus. Deshalb braucht sie bei der Erstellung Ihre Hilfe!

\begin{enumerate}
\item Formulieren Sie die wissenschaftliche Fragestellung! \textbf{(1 Punkt)}
\item Ergänzen Sie die Tabelle um die fehlenden Informationen! \textbf{(1 Punkt)} 
\item Visualisieren Sie den Zusammenhang zwischen den beiden kategorialen Variablen! \textbf{(2 Punkte)}
\item Berechnen Sie die Teststatistik eines Chi-Quadrat-Test! \textbf{(2 Punkte)}
\item Treffen Sie eine Entscheidung im Bezug zu der Nullhypothese gegeben
  einem $\mathcal{X}^2_{\alpha = 5\%} = 4.56$! Begründen Sie Ihre Antwort!
  \textbf{(2 Punkte)}
\item Skizzieren Sie in einer Abbildung die $\mathcal{X}^2$-Verteilung, wenn die $H_0$ wahr ist! Ergänzen Sie  $\mathcal{X}^2_{\alpha = 5\%}$ und $\mathcal{X}^2_{D}$ in der Abbildung! \textit{Beachten Sie folgenden Informationen zur $\mathcal{X}^2$-Verteilung. Die $\mathcal{X}^2$-Verteilung hat ein Maxima bei $\mathcal{X}^2 = 6$ sowie ein Minima bei $\mathcal{X}^2 = 11$.} \textbf{(2 Punkte)}
\end{enumerate} 
\clearpage
% -----------------------------------------------------------------------

\section{Aufgabe \hfill (10 Punkte)}

\textit{Geben Sie grundsätzlich Formeln und Rechenweg zur Lösung der Teilaufgaben mit an!} \\[1Ex]
 

 
%% --------------------------------------------------------------------
\begin{minipage}[t]{0.5\textwidth}
\includegraphics[width = 1.3cm]{/Users/kruppajo/work/GitHub/exam/avatare/Mark.png}
\end{minipage}
\begin{minipage}[t]{0.5\textwidth}
\hfill
\href{https://youtu.be/-Kva5wc5Elw}{\includegraphics[width = 2cm]{img/caution}}
\end{minipage}
\vspace{-3Ex}
%% --------------------------------------------------------------------



\paragraph{Den Chi-Quadrat-Test in einem Fragebogen}

Mark hatte sich gleich von Beginn an in seinem Projektbericht für eine Umfrage im Marketing interessiert. Jetzt geht es um den Haupt- und Nebenerwerb von Erlebnishöfen in Norddeutschland. Viele Höfe haben angefangen auch Großkatzen zu halten, damit mehr Kunden auf die Höfe kommen. Für den Verband der Großkatzenbesitzer e.V. möchte er nun einen Fragebogen zur Zukunftsfähigkeit Schritt für Schritt auswerten. Dabei teilt er zuerst die Antwortenden in die beiden Gruppen 'Höfe mit Großkatzen [ja]' und 'Höfe mit Großkatzen [nein]' ein. Daraufhin möchte er für folgende Frage \textit{f5verband} einmal auswerten, ob es einen Unterschied zwischen den beiden Höfen mit oder ohne Großkatzen gibt. \begin{center}\textit{Sehen Sie die Notwendigkeit von genetisch veränderten Großkatzen zur Steigerung der Attraktivität des Erlebnishofes?}\end{center}Mark krazt sich an seinem Kopf. Wie soll man eine Tabelle mit so vielen Zahlen sinnvoll auswerten? Schnell noch ein paar Marzipankugeln einwerfen und los gehts!

\vspace{5Ex}

\begin{center}
  \Large
  \begin{tabular}{c|l|l|l|l|l|c}
\textit{f5verband}     & \textbf{trifft gar} & \textbf{trifft} & \textbf{weder} & \textbf{trifft} & \textbf{trifft} & \\
 \phantom{\textbf{Pestizid}}    & \textbf{nicht zu} & \textbf{nicht zu} & \textbf{noch} & \textbf{zu} & \textbf{voll zu} & \\
    \hline
   \textbf{ja}  & 3  & 8  &  12  & 21  & 11  &   \strut\\[4Ex]
    \hline
    \textbf{nein} & 1  & 11  & 25  & 9  & 4  &      \strut\\[4Ex]
    \hline
     \phantom{\textbf{Pestizid}} & \phantom{\textbf{trifft gar}}  & \phantom{\textbf{trifft gar}}  &  \phantom{\textbf{trifft gar}} &  \phantom{\textbf{trifft gar}} &  \phantom{\textbf{trifft gar}} &  \phantom{100} \strut\\
  \end{tabular}
\end{center}

\vspace{5Ex}

Leider kennt sich Mark mit der Berechnung eines $\mathcal{X}^2$-Test auf einer Frage in einem Fragebogen überhaupt nicht aus. Deshalb braucht er bei der Erstellung Ihre Hilfe!

\begin{enumerate}
\item Formulieren Sie die wissenschaftliche Fragestellung! \textbf{(1 Punkt)}
\item Ergänzen Sie die Tabelle um die fehlenden Informationen! \textbf{(1 Punkt)} 
\item Berechnen Sie die Teststatistik $\mathcal{X}^2_{D}$ eines Chi-Quadrat-Test! \textit{Ignorieren Sie Zellbelegungen kleiner gleich fünf in der Berechnung von $\mathcal{X}^2_{D}$!} \textbf{(2 Punkte)}
\item Treffen Sie eine Entscheidung im Bezug zu der Nullhypothese gegeben
  einem $\mathcal{X}^2_{\alpha = 5\%} = 23.83$! Begründen Sie Ihre Antwort!
  \textbf{(2 Punkte)}
\item Visualisieren Sie die 2x5 Kreuztabelle \textit{ohne} die Berücksichtigung der Antwortkategorie 'weder noch'! \textbf{(2 Punkte)}
\item Berechnen Sie den Effektschätzer $Cramers\; V$ auf der 2x5 Kreuztabelle! \textbf{(1 Punkt)}
\item Interpretieren Sie den berechneten Effektschätzer $Cramers\; V$ unter Berücksichtigung der Fragestellung! \textbf{(1 Punkt)}
\end{enumerate} 
\clearpage
% -----------------------------------------------------------------------

\section{Aufgabe \hfill (10 Punkte)}

\textit{Geben Sie grundsätzlich Formeln und Rechenweg zur Lösung der Teilaufgaben mit an!} \\[1Ex]
 

 
%% --------------------------------------------------------------------
\begin{minipage}[t]{0.5\textwidth}
\includegraphics[width = 1.3cm]{/Users/kruppajo/work/GitHub/exam/avatare/Jonas.png}
\end{minipage}
\begin{minipage}[t]{0.5\textwidth}
\hfill
\href{https://youtu.be/-Kva5wc5Elw}{\includegraphics[width = 2cm]{img/caution}}
\end{minipage}
\vspace{-3Ex}
%% --------------------------------------------------------------------



\paragraph{Den Chi-Quadrat-Test mit Effektmaß berechnen}

Jonas hat sich ein Herz gefasst und war für seiner Hausarbeit in die Niederlande gegangen. Das war eine super Zeit in der er viel gelernt hat. Klar gab es auch die ein oder andere Besonderheit, aber das gehört hier eher nicht hin. Dann noch schnell Iron Maiden auf das Ohr und los gehts. Jonas ist schon eine ganze Zeit im Büro, da seine Betreuerin möchte, dass er jetzt auf seinen Daten mit $n = 188$ Beobachtungen von Lamas einen $\mathcal{X}^2$-Test rechnet. Das ginge, da er als Behandlung \textit{Ökologisch [ja/nein]} bestimmt und zum anderen die Variable \textit{Protein/Fettrate im Zielbereich [ja/nein]} ermittelt hat. Wie genau, das ist jetzt eine andere Frage. Eigentlich wollte Jonas nachher noch einen Film schauen. Irgendwie komisch, wenn er Mission Impossible anmacht, dann ist das Meerschweinchen eigentlich sofort vor dem Bildschirm und starrt hinein.

\vspace{5Ex}

\begin{center}
  \huge
  \begin{tabular}{c|l|l|c}
     & \phantom{\textbf{Erkrankt (ja)}} & \phantom{\textbf{Erkrankt (ja)}} & \phantom{\textbf{Erkrankt (ja)}} \strut\\
    \hline
    \phantom{\textbf{Pestizid (ja)}} & 56  & 51  &     \strut\\
    \hline
    \phantom{\textbf{Pestizid (ja)}} & 43  & 38  &      \strut\\
    \hline
     \phantom{100} & \phantom{100}  & \phantom{100}  &  \phantom{100}  \strut\\
  \end{tabular}
\end{center}

\vspace{5Ex}

Leider kennt sich Jonas mit der Berechnung eines $\mathcal{X}^2$-Test für kategoriale Daten überhaupt nicht aus. Deshalb braucht er bei der Erstellung Ihre Hilfe!

\begin{enumerate}
\item Formulieren Sie die wissenschaftliche Fragestellung! \textbf{(1 Punkt)}
\item Ergänzen Sie die Tabelle um die fehlenden Informationen! \textbf{(1 Punkt)} 
\item Berechnen Sie die Teststatistik eines Chi-Quadrat-Test! \textbf{(2 Punkte)}
\item Treffen Sie eine Entscheidung im Bezug zu der Nullhypothese gegeben
  einem $\mathcal{X}^2_{\alpha = 5\%} = 5.61$! Begründen Sie Ihre Antwort!
  \textbf{(2 Punkte)}
\item Berechnen Sie den Effektschätzer $Cramers\; V$ auf der 2x2 Kreuztabelle! \textbf{(1 Punkt)}
\item Welchen Wertebereich kann der Effektschätzer $Cramers\; V$ annehmen? Wann liegt kein Effekt und wann ein starker Effekt vor? \textbf{(2 Punkte)}
\item Interpretieren Sie den berechneten Effektschätzer $Cramers\; V$ unter Berücksichtigung der Fragestellung! \textbf{(1 Punkt)}
\end{enumerate} 
\clearpage
% -----------------------------------------------------------------------

\section{Aufgabe \hfill (10 Punkte)}

\textit{Geben Sie grundsätzlich Formeln und Rechenweg zur Lösung der Teilaufgaben mit an!} \\[1Ex]
 

 
%% --------------------------------------------------------------------
\begin{minipage}[t]{0.5\textwidth}
\includegraphics[width = 1.3cm]{/Users/kruppajo/work/GitHub/exam/avatare/Alex.png}
\end{minipage}
\begin{minipage}[t]{0.5\textwidth}
\hfill
\href{https://youtu.be/jakM7fHyZfU}{\includegraphics[width = 2cm]{img/youtube}}
\end{minipage}
\vspace{-3Ex}
%% --------------------------------------------------------------------



\paragraph{Den Chi-Quadrat-Test konzeptionell verstehen}

Am Ende hätte Alex dann doch einen normalverteilten Endpunkt in seiner Abschlussarbeit nehmen sollen. Dann noch schnell Gummibärchen zur Stärkung und los gehts. Vor ihm liegen jetzt die Daten von zwei Variablen als Kategorien oder wie es in \Rlogo so schön heißt, als Faktoren. Aber immerhin, hofft er das was bei den Daten rausgekommen ist. Gezählt hat Alex einiges mit $n = 122$ Beobachtungen von Milchvieh. Zum einen hat er als Behandlung \textit{Automatische Fütterung [ja/nein]} bestimmt und zum anderen die Variable \textit{Schlachtgewicht im Zielbereich [ja/nein]} ermittelt. Nun möchte seine Betreuerin gerne einen $\mathcal{X}^2$-Test auf einer $2x2$-Kreuztabelle berechnet bekommen. Am Ende des Tages möchte er dann noch sein Hobby Starcraft genießen. Das muss auch mal sein!

\vspace{5Ex}

\begin{center}
  \huge
  \begin{tabular}{c|c|c|c}
     & \phantom{\textbf{Erkrankt (ja)}} & \phantom{\textbf{Erkrankt (ja)}} & \phantom{\textbf{Erkrankt (ja)}} \strut\\
    \hline
   \phantom{\textbf{Pestizid (ja)}} & \phantom{100}  & \phantom{100}  &   64  \strut\\
    \hline
    \phantom{\textbf{Pestizid (ja)}} & \phantom{100}  & \phantom{100}  &   58   \strut\\
    \hline
     &  61 &  61 &  122  \strut\\
  \end{tabular}
\end{center}

\vspace{5Ex}

Leider kennt sich Alex mit der Berechnung eines $\mathcal{X}^2$-Test für kategoriale Daten überhaupt nicht aus. Deshalb braucht er bei der Erstellung Ihre Hilfe!

\begin{enumerate}
  \item Formulieren Sie die wissenschaftliche Fragestellung! \textbf{(1 Punkt)}
\item Ergänzen Sie die Tabelle um die fehlenden Informationen! \textbf{(1 Punkt)} 
\item Ergänzen Sie die Felder innerhalb der $2x2$ Kreuztabelle, so dass \textit{kein} signifikanter Effekt zu erwarten wäre! \textbf{(2 Punkte)}
\item Begründen Sie Ihr Vorgehen an der Formel des Chi-Quadrat-Tests. Erklären Sie Ihr Vorgehen an einem Beispiel! \textbf{(2 Punkte)}
\item Visualisieren Sie den Zusammenhang zwischen den beiden kategorialen Variablen! \textbf{(2 Punkte)}
\item Was ist die Mindestanzahl an Beobachtungen je Zelle? Wenn in einer der Zellen weniger Beobachtungen auftreten, welchen Test können Sie anstatt des Standard Chi-Quadrat-Tests anwenden? \textbf{(2 Punkte)}
\end{enumerate} 
\clearpage
% -----------------------------------------------------------------------

\section{Aufgabe \hfill (10 Punkte)}

\textit{Geben Sie grundsätzlich Formeln und Rechenweg zur Lösung der Teilaufgaben mit an!} \\[1Ex]
 

 
%% --------------------------------------------------------------------
\begin{minipage}[t]{0.5\textwidth}
\includegraphics[width = 1.3cm]{/Users/kruppajo/work/GitHub/exam/avatare/Mark.png}
\end{minipage}
\begin{minipage}[t]{0.5\textwidth}
\hfill
\href{https://youtu.be/ghArbetOr_E}{\includegraphics[width = 2cm]{img/youtube}}
\end{minipage}
\vspace{-3Ex}
%% --------------------------------------------------------------------



\paragraph{Der Chi-Quadrat-Test in \Rlogo}


Mark hat sich ein Herz gefasst und war für seinem Projektbericht in die Niederlande gegangen. Das war eine super Zeit in der er viel gelernt hat. Klar gab es auch die ein oder andere Besonderheit, aber das gehört hier eher nicht hin. Dann noch schnell Columbo starten und los gehts mit der Kraft von Marzipankugeln. Mark ist schon eine ganze Zeit im Büro, da seine Betreuerin möchte, dass er jetzt auf seinen Daten mit $n = 122$ Beobachtungen von Lauch einen $\mathcal{X}^2$-Test rechnet. Das ginge, da er als Behandlung \textit{KI-gesteuert [ja/nein]} bestimmt und zum anderen die Variable \textit{Trockengewicht über Zielwert [ja/nein]} ermittelt hat. Wie genau, das ist jetzt eine andere Frage. Nach seinem Experiment erhielt er folgende $2x2$ Kreuztabelle aus seinen erhobenen Daten.

\begin{knitrout}
\definecolor{shadecolor}{rgb}{0.969, 0.969, 0.969}\color{fgcolor}\begin{kframe}
\begin{verbatim}
##                             KI-gesteuert
## Trockengewicht über Zielwert ja nein
##                         ja    6   12
##                         nein 10    7
\end{verbatim}
\end{kframe}
\end{knitrout}

Dann rechnete Mark den Fisher-Exakt-Test auf der $2x2$-Kreuztabelle in \Rlogo und erhielt folgende \Rlogo Ausgabe der Funktion \texttt{fisher.test()}.

\begin{knitrout}
\definecolor{shadecolor}{rgb}{0.969, 0.969, 0.969}\color{fgcolor}\begin{kframe}
\begin{verbatim}
## 
## 	Fisher's Exact Test for Count Data
## 
## data:  Trockengewicht über Zielwert
## p-value = 0.1811
## alternative hypothesis: true odds ratio is not equal to 1
## 95 percent confidence interval:
##  0.07075797 1.67575062
## sample estimates:
## odds ratio 
##  0.3611682
\end{verbatim}
\end{kframe}
\end{knitrout}

Leider kennt sich Mark mit der Berechnung eines $\mathcal{X}^2$-Test für kategoriale Daten überhaupt nicht aus. Deshalb braucht er bei der Erstellung Ihre Hilfe!

\begin{enumerate}
\item Formulieren Sie die wissenschaftliche Fragestellung! \textbf{(1 Punkt)}
\item Visualisieren Sie den Zusammenhang zwischen den beiden kategorialen Variablen! \textbf{(2 Punkte)}
\item Liegt ein signifikanter Unterschied zwischen den Gruppen vor? Begründen Sie Ihre Antwort! \textbf{(2 Punkte)}
\item Skizzieren Sie das sich ergebende 95\% Konfidenzintervall! \textbf{(2 Punkte)}
\item Beschriften Sie die Abbildung des 95\% Konfidenzintervalls! \textbf{(1 Punkt)} 
\item Interpretieren Sie das \textit{Odds ratio} im Kontext der wissenschaftlichen Fragestellung! \textbf{(2 Punkte)} 
\end{enumerate}
 
\clearpage
% -----------------------------------------------------------------------

\section{Aufgabe \hfill (11 Punkte)}

\textit{Geben Sie grundsätzlich Formeln und Rechenweg zur Lösung der Teilaufgaben mit an!} \\[1Ex]
 

 
%% --------------------------------------------------------------------
\begin{minipage}[t]{0.5\textwidth}
\includegraphics[width = 1.3cm]{/Users/kruppajo/work/GitHub/exam/avatare/Jessica.png}\hspace{-4mm}\includegraphics[width = 1.3cm]{/Users/kruppajo/work/GitHub/exam/avatare/Jonas.png}
\end{minipage}
\begin{minipage}[t]{0.5\textwidth}
\hfill
\href{https://youtu.be/VQlNl8hvRII}{\includegraphics[width = 2cm]{img/youtube}}
\end{minipage}
%% --------------------------------------------------------------------



\paragraph{Den diagnostische Test am Doppelbaum berechnen}

Jonas liest laut vor. 'Die Prävalenz von Klauenseuche bei Brokkoli wird mit 2\% angenommen. In 80\% der Fälle ist ein Test positiv, wenn das Pflanze erkrankt ist. In 8.0\% der Fälle ist ein Test positiv, wenn das Pflanze \underline{nicht} erkrankt ist und somit gesund ist. Wir führen einen Test auf Schorf an 2000 Brokkoli mit einem diagnostischen Test durch.' Jessica klappt die Kinnlade runter. In der Stille duddelt Iron Maiden. Jonas schaut kompetent und schmeißt sich mit offenen Mund Schokobons an den Kopf vorbei.

\begin{center}
  \includegraphics[width=17cm]{/Users/kruppajo/work/GitHub/exam/question/img/diag-doppelbaum}
\end{center}

Leider kennen sich Jonas und Jessica mit dem diagnostischen Testen überhaupt nicht aus. Deshalb brauchen beide bei der Erstellung Ihre Hilfe! 
    
\begin{enumerate}
\item Beschriften Sie die Äste des Doppelbaumes, mit denen Ihnen bekannten Informationen! \textbf{(2 Punkte)}
\item Beschriften Sie den Doppelbaum! \textbf{(2 Punkte)}
\item Füllen Sie freien Felder des Doppelbaums aus! \textbf{(4 Punkte)}
\item Berechnen Sie die Wahrscheinlichkeit $Pr(K^+|T^+)$! \textbf{(2 Punkte)}
\item Was sagt Ihnen die Wahrscheinlichkeit $Pr(K^+|T^+)$ aus? \textbf{(1 Punkt)}
\end{enumerate}






 
\clearpage
% -----------------------------------------------------------------------

\section{Aufgabe \hfill (11 Punkte)}

\textit{Geben Sie grundsätzlich Formeln und Rechenweg zur Lösung der Teilaufgaben mit an!} \\[1Ex]
 

 
%% --------------------------------------------------------------------
\begin{minipage}[t]{0.5\textwidth}
\includegraphics[width = 1.3cm]{/Users/kruppajo/work/GitHub/exam/avatare/Jessica.png}\hspace{-4mm}\includegraphics[width = 1.3cm]{/Users/kruppajo/work/GitHub/exam/avatare/Nilufar.png}
\end{minipage}
\begin{minipage}[t]{0.5\textwidth}
\hfill
\href{https://youtu.be/_7s44pbOc00}{\includegraphics[width = 2cm]{img/youtube}}
\end{minipage}
%% --------------------------------------------------------------------



\paragraph{Der diagnostische Test und statistische Maßzahlen}

'Was ist denn das?', entfährt es Jessica. 'Hm... ich glaube es handelt sich um einen Doppelbaum, den wir beim diagnostischen Testen brauchen.', meint Nilufar und dreht Herr der Ringe auf dem Second Screen etwas leiser. Was jetzt beide von einem diagnostischen Test haben, ist ihnen auch nicht klar. Es ist ja schon alles komplex genug und der Mangel von Jessica macht es heute auch nicht mehr einfacher. 'Es geht um Kräuselkrankheit an Brokkoli.', stellt Nilufar fest. Eigentlich wollte Nilufar eher los um zu Kicken. Das wird aber wohl nichts mehr.

\begin{tikzpicture}
  \node (image) at (0,0) {
    \includegraphics[width=\textwidth]{/Users/kruppajo/work/GitHub/exam/question/img/diag-doppelbaum}
  };
  \node[] at (-4.8,0) {\huge 180};
  \node[] at (-1.7,0) {\huge 60};
  \node[] at (1.7,0) {\huge 850};
  \node[] at (4.75,0) {\huge 1600};
\end{tikzpicture}

Leider kennen sich Jessica und Nilufar mit dem diagnostischen Testen überhaupt nicht aus. Deshalb brauchen beide bei der Erstellung Ihre Hilfe! 
  
\begin{enumerate}
\item Beschriften Sie den Doppelbaum! \textbf{(2 Punkte)}
\item Füllen Sie freien Felder des Doppelbaums aus! \textbf{(4 Punkte)}
\item Berechnen Sie die Wahrscheinlichkeit $Pr(K^+|T^+)$! \textbf{(2 Punkte)}
\item Berechnen Sie die Prävalenz für Klauenseuche! \textbf{(1 Punkt)}
\item Berechnen Sie die Sensifität und Spezifität des diagnostischen Tests! Erstellen Sie dafür zunächst eine 2x2 Kreuztabelle! \textbf{(2 Punkte)}
\end{enumerate}

 





 
\clearpage
% -----------------------------------------------------------------------
\part{Lineare Regression \& Korrelation}
% -----------------------------------------------------------------------

\section{Aufgabe \hfill (10 Punkte)}

\textit{Geben Sie grundsätzlich Formeln und Rechenweg zur Lösung der Teilaufgaben mit an!} \\[1Ex]
 

 
%% --------------------------------------------------------------------
\begin{minipage}[t]{0.5\textwidth}
\includegraphics[width = 1.3cm]{/Users/kruppajo/work/GitHub/exam/avatare/Jonas.png}\hspace{-4mm}\includegraphics[width = 1.3cm]{/Users/kruppajo/work/GitHub/exam/avatare/Yuki.png}
\end{minipage}
\begin{minipage}[t]{0.5\textwidth}
\hfill
\href{https://youtu.be/kHmfEmU6lrk}{\includegraphics[width = 2cm]{img/youtube}}
\end{minipage}
%% --------------------------------------------------------------------



\paragraph{Visualisierung der linearen Regression}

'Wichtig ist es, dass wir jetzt eine Gerade durch die Punkte zeichnen!', ruft Yuki. 'Ich sehe nur eine Zahlen und keine Punkte. Wie soll ich da denn jetzt eine Gerade durchzeichnen?', fragt Jonas. Yuki atmet schwer ein. Die beiden hatten ein Feldexperiment in der Uckermark mit Lauch durchgeführt. Dabei wurden die beiden folgenden Variablen gemessen: durchschnittlichen Niederschlag [ml/w] und Chlorophyllgehalt (SPAD-502Plus) [SPAD]. Jetzt will die Betreuung von den beiden einmal die Visualisierung der Daten und auch gleich noch die lineare Regression gerechnet bekommen.

\begin{table}[!h]
\centering
\begin{tabular}{cc}
\toprule
Durchschnittlichen Niederschlag [ml/w] & Chlorophyllgehalt (SPAD-502Plus) [SPAD]\\
\midrule
22.9 & 21.3\\
30.1 & 36.8\\
26.8 & 32.5\\
28.8 & 28.7\\
30.5 & 33.5\\
\addlinespace
26.3 & 29.6\\
24.7 & 27.7\\
24.9 & 32.1\\
22.4 & 25.4\\
\bottomrule
\end{tabular}
\end{table}



Leider kennen sich Yuki und Jonas mit der linearen Regression für kontinuierliche Daten überhaupt nicht aus. Deshalb brauchen beide bei der Erstellung Ihre Hilfe!

\begin{enumerate}
\item Formulieren Sie die wissenschaftliche Fragestellung! \textbf{(1 Punkt)}
\item Erstellen  Sie  eine  Visualisierung  für  die  Datentabelle.  Beschriften  Sie  die  Achsen! \textbf{(2 Punkte)}
\item Schätzen Sie die Regressionsgleichung aus der obigen Abbildung ab! \textbf{(2 Punkte)}
\item Beschriften Sie die Grade mit den statistischen Maßzahlen der linearen Regressionsgleichung! \textbf{(2 Punkte)}
\item Liegt ein Zusammenhang zwischen $x$ und $y$ vor? Begründen Sie Ihre Antwort! \textbf{(2 Punkte)}
\item Wenn kein Zusammenhang zu beobachten wäre, wie würde die Grade aussehen? \textit{Antworten Sie mit einer Skizze der Geraden!} \textbf{(1 Punkt)}
\end{enumerate} 
\clearpage
% -----------------------------------------------------------------------

\section{Aufgabe \hfill (12 Punkte)}

\textit{Geben Sie grundsätzlich Formeln und Rechenweg zur Lösung der Teilaufgaben mit an!} \\[1Ex]
 

 
%% --------------------------------------------------------------------
\begin{minipage}[t]{0.5\textwidth}
\includegraphics[width = 1.3cm]{/Users/kruppajo/work/GitHub/exam/avatare/Alex.png}\hspace{-4mm}\includegraphics[width = 1.3cm]{/Users/kruppajo/work/GitHub/exam/avatare/Mark.png}
\end{minipage}
\begin{minipage}[t]{0.5\textwidth}
\hfill
\href{https://youtu.be/lJp8rFmMnrs}{\includegraphics[width = 2cm]{img/youtube}}
\end{minipage}
%% --------------------------------------------------------------------



\paragraph{Interpretation der Ergebnisse einer linearen Regression}


'Hä? Was ist denn das? Hatten wir das als Aufgabe eine lineare Regression zu rechnen? Wir bauen aus kontinuierlichen Daten eine Abbildung und interpretieren diese dann?', fragt Mark. Alex schaut fragend zurück. 'Keine Ahnung... das ist jetzt jedenfalls keine Abbildung von irgendwas sondern Zahlen in einer Tabelle...', antwortet Alex leicht angespannt. Die beiden hatten ein Kreuzungsexperiment im Oldenburger Land mit Milchvieh durchgeführt. Dabei wurden die beiden folgenden Variablen gemessen: mittlere Anzahl an weißen Blutkörperchen [LEU/ml] und Schlachtgewicht [kg]. Jetzt haben die beiden eigentlich alles zusammen. \textit{Eigentlich...}, denn mit der \Rlogo Ausgabe haben beide jetzt ein Problem.

\begin{table}[!h]
\centering\begingroup\fontsize{11}{13}\selectfont

\begin{tabular}{ccccc}
\toprule
term & estimate & std.error & t statistic & p-value\\
\midrule
(Intercept) & 5.84 & 1.66 &  & \\
Mittlere Anzahl & -0.30 & 0.16 &  & \\
\bottomrule
\end{tabular}
\endgroup{}
\end{table}



Leider kennen sich Mark und Alex mit der linearen Regression für kontinuierliche Daten in \Rlogo überhaupt nicht aus. Deshalb brauchen beide bei der Erstellung Ihre Hilfe!

\begin{enumerate}
\item Formulieren Sie die wissenschaftliche Fragestellung! \textbf{(1 Punkt)}
\item Erstellen  Sie  eine  Visualisierung  der \texttt{lm()}-Ausgabe.  Beschriften  Sie  die  Achsen! \textbf{(2 Punkte)}
\item Beschriften Sie die Visualisierung mit den statistischen Maßzahlen der der \texttt{lm()}-Ausgabe! \textbf{(2 Punkte)}
\item Formulieren Sie die Regressionsgleichung! \textbf{(1 Punkt)}
\item Ergänzen Sie die t Statistik in der \texttt{lm()}-Ausgabe! \textbf{(2 Punkte)}
\item Ergänzen Sie den $p$-Wert in der \texttt{lm()}-Ausgabe mit $T_{\alpha = 5\%} = 1.96$!  \textbf{(2 Punkte)}
\item Interpretieren Sie den $p$-Wert im Kontext der wissenschaftlichen Fragestellung! \textbf{(1 Punkt)}  
\item Wie groß ist der Effekt im Kontext der wissenschaftlichen Fragestellung? \textbf{(1 Punkt)}
\end{enumerate} 
\clearpage
% -----------------------------------------------------------------------

\section{Aufgabe \hfill (11 Punkte)}

\textit{Geben Sie grundsätzlich Formeln und Rechenweg zur Lösung der Teilaufgaben mit an!} \\[1Ex]
 

 
%% --------------------------------------------------------------------
\begin{minipage}[t]{0.5\textwidth}
\includegraphics[width = 1.3cm]{/Users/kruppajo/work/GitHub/exam/avatare/Jonas.png}\hspace{-4mm}\includegraphics[width = 1.3cm]{/Users/kruppajo/work/GitHub/exam/avatare/Steffen.png}
\end{minipage}
\begin{minipage}[t]{0.5\textwidth}
\hfill
\href{https://youtu.be/tNNzcndrpSk}{\includegraphics[width = 2cm]{img/youtube}}
\end{minipage}
%% --------------------------------------------------------------------



\paragraph{Interpretation der Ergebnisse einer linearen Regression in \Rlogo}


'Wichtig ist es, dass wir jetzt eine Gerade durch die Punkte zeichnen!', ruft Steffen. 'Ich sehe nur Kauderwelsch und keine Punkte. Wie soll ich da denn jetzt eine Gerade durchzeichnen? Und warum überhaupt? War das unsere Fragestellung?', fragt Jonas. Steffen atmet schwer ein und starrt auf die \Rlogo Ausgabe der Funktion \texttt{lm()}. Die beiden hatten ein Feldexperiment im Emsland mit Erdbeeren durchgeführt. Dabei wurden die beiden folgenden Variablen gemessen: durchschnittlicher Anteil an Ton [\%/l] und Frischegewicht [kg/ha]. Jetzt will die Betreuung von den beiden die Interpretierung der Daten in Form einer linearen Regression gerechnet bekommen. Das haben beide in \Rlogo gemacht, aber wie soll das jetzt gehen? Das mit der Interpretation?

\begin{knitrout}
\definecolor{shadecolor}{rgb}{0.969, 0.969, 0.969}\color{fgcolor}\begin{kframe}
\begin{verbatim}
## 
## Call:
## Frischegewicht ~ Durchschnittlicher_Anteil
## 
## Residuals:
##      Min       1Q   Median       3Q      Max 
## -2.97052 -0.67736  0.07321  0.79656  2.18951 
## 
## Coefficients:
##                           Estimate Std. Error t value Pr(>|t|)
## (Intercept)                 2.2370     1.4442   1.549  0.12968
## Durchschnittlicher_Anteil   0.4666     0.1405   3.320  0.00199
## 
## Residual standard error: 1.174 on 38 degrees of freedom
## Multiple R-squared:  0.2249,	Adjusted R-squared:  0.2045 
## F-statistic: 11.02 on 1 and 38 DF,  p-value: 0.001993
\end{verbatim}
\end{kframe}
\end{knitrout}

Leider kennen sich Steffen und Jonas mit der linearen Regression für kontinuierliche Daten in \Rlogo überhaupt nicht aus. Deshalb brauchen beide bei der Erstellung Ihre Hilfe!


\begin{enumerate}
\item Formulieren Sie die wissenschaftliche Fragestellung! \textbf{(1 Punkt)}
\item Wie groß ist der Effekt im Kontext der wissenschaftlichen Fragestellung? \textbf{(2 Punkte)} 
\item Interpretieren Sie die $p$-Werte im Kontext der wissenschaftlichen Fragestellung! \textbf{(2 Punkte)}
\item Visualisieren Sie die Verteilung der Residuen! \textbf{(2 Punkte)} 
\item Ist die Annahme der Normalverteilung erfüllt? Begründen Sie die Antwort! \textbf{(2 Punkte)}
\item Erklären Sie \textit{kurz} den Begriff \texttt{R-squared}! Was sagt Ihnen der Wert aus? \textbf{(2 Punkte)}
\end{enumerate}
 
\clearpage
% -----------------------------------------------------------------------

\section{Aufgabe \hfill (10 Punkte)}

\textit{Geben Sie grundsätzlich Formeln und Rechenweg zur Lösung der Teilaufgaben mit an!} \\[1Ex]
 

 
%% --------------------------------------------------------------------
\begin{minipage}[t]{0.5\textwidth}
\includegraphics[width = 1.3cm]{/Users/kruppajo/work/GitHub/exam/avatare/Tina.png}
\end{minipage}
\begin{minipage}[t]{0.5\textwidth}
\hfill
\href{https://youtu.be/C9skfFRTHhI}{\includegraphics[width = 2cm]{img/youtube}}
\end{minipage}
\vspace{-3ex}
%% --------------------------------------------------------------------



\paragraph{Interpretation der Ergebnisse einer Korrelationsanalyse in \Rlogo}

'Ich glaube ich bringe da was durcheinander. Ich möchte eine Gerade durch die Punkte zeichnen oder doch eine Korrelation berechnen?', merkt Tina laut an. 'Ich sehe keine Punkte... das ist doch eine Ausgabe in \Rlogo. Überhaupt, darum geht es doch gar nicht in meinem Versuch. Ich wollte doch keine Gerade zeichnen?.', antwortet Tina sich sichtlich übernächtigt selber. Das Verrückte ist, dass die Spinne Indiana Jones wirklich liebt. Das ist Tina sehr recht, denn sie braucht Entspannung. Die Nacht war zu lang und überhaupt. Wenn die Wut nicht wäre, ja dann wäre wohl vieles möglich für Tina! Aber so.. Tina hatte ein Feldexperiment in der Uckermark mit Erbsen durchgeführt. Dabei wurden die beiden folgenden Variablen gemessen: durchschnittliche Regenwurmdichte [Anzahl/l] und Chlorophyllgehalt (SPAD-502Plus) [SPAD]. Jetzt will sie erstmal schauen, ob es einen Zusammenhang gibt und das soll mit der \Rlogo Ausgabe möglich sein.


\begin{knitrout}
\definecolor{shadecolor}{rgb}{0.969, 0.969, 0.969}\color{fgcolor}\begin{kframe}
\begin{verbatim}
## 
## 	Pearson's correlation
## 
## data:  Durchschnittliche Regenwurmdichte and Chlorophyllgehalt
## t = -8.8093, df = 8, p-value = 2.168e-05
## alternative hypothesis: true correlation is not equal to 0
## 95 percent confidence interval:
##  -0.9889147 -0.8052156
## sample estimates:
##        cor 
## -0.9521276
\end{verbatim}
\end{kframe}
\end{knitrout}

Leider kennt sich Tina mit der Korrelationsanalyse in \Rlogo überhaupt nicht aus. Deshalb braucht sie bei der Erstellung Ihre Hilfe!

\begin{enumerate}
  \item Formulieren Sie die wissenschaftliche Fragestellung! \textbf{(1 Punkt)}
  \item Formulieren Sie das statistische Hypothesenpaar! \textbf{(1 Punkt)}
\item Erstellen Sie eine Visualisierung für den Korrelationskoeffizienten! Beschriften Sie die Abbildung! \textbf{(2 Punkte)}
\item Nennen Sie die zwei Eigenschaften des Korrelationskoeffizienten! \textbf{(2 Punkte)}
\item Interpretieren Sie den Korrelationskoefizienten hinsichtlich des
  Effekts und der Signifikanz! Begründen Sie Ihre Antwort! \textbf{(2 Punkte)}
\item Visualisieren Sie das 95\% Konfidenzintervall! Beschriften Sie die Abbildung! \textbf{(2 Punkte)} 
\end{enumerate} 
\clearpage
% -----------------------------------------------------------------------

\section{Aufgabe \hfill (12 Punkte)}

\textit{Geben Sie grundsätzlich Formeln und Rechenweg zur Lösung der Teilaufgaben mit an!} \\[1Ex]
 

 
%% --------------------------------------------------------------------
\begin{minipage}[t]{0.5\textwidth}
\includegraphics[width = 1.3cm]{/Users/kruppajo/work/GitHub/exam/avatare/Tina.png}
\end{minipage}
\begin{minipage}[t]{0.5\textwidth}
\hfill
\href{https://youtu.be/fB6nF4dxodA}{\includegraphics[width = 2cm]{img/youtube}}
\end{minipage}
\vspace{-3ex}
%% --------------------------------------------------------------------



\paragraph{Visualisierung der Korrelation und des Bestimmtheitsmaßes}

'Hm..., drei leere Abbildungen. Was soll ich da jetzt machen?', fragt sich Tina. Tina kennt sich nur begrenzt bis gar nicht mit der linearen Regresion und Korrelation aus. Dafür mit etwas anderem. Am Ende dann doch besser Astronomie. Wunderbar. Eine echte Ablenkung für Tina. Aber das hilft hier auch nur so halb, die Aufgabe zu lösen und mehr über den Korrelationskoeffizienten zu erfahren.
\vspace{2Ex}



{\centering \includegraphics[width=\maxwidth]{img/correlation-01-1} 

}




\paragraph{Visualisierung der Korrelation und des Bestimmtheitsmaßes}

\vspace{2Ex}

Leider kennt sich Tina mit der Korrelationsanalyse und der linearen Regression überhaupt nicht aus. Deshalb braucht sie bei der Auswertung Ihre Hilfe!

\begin{enumerate}
\item Zeichnen Sie für die $\rho$-Werte eine Gerade in die entsprechende Abbildung! \textbf{(3 Punkte)}
\item Zeichnen Sie für die $R^2$-Werte die entsprechende Punktewolke um die Gerade! \textbf{(3 Punkte)}
\item Nennen Sie die zwei Eigenschaften des Korrelationskoeffizienten! \textbf{(2 Punkte)}
\item Interpretieren Sie die $R^2$-Werte für die jeweilige Gerade! \textbf{(2 Punkte)}
\item Warum müssen Sie ein $R^2$-Wert berechnen, wenn Sie die einfachere Möglichkeit der visuellen Überprüfung haben? Begründen Sie Ihre Antwort! \textbf{(2 Punkte)}
\end{enumerate}
 
\clearpage
% -----------------------------------------------------------------------

\section{Aufgabe \hfill (12 Punkte)}

\textit{Geben Sie grundsätzlich Formeln und Rechenweg zur Lösung der Teilaufgaben mit an!} \\[1Ex]
 

 
%% --------------------------------------------------------------------
\begin{minipage}[t]{0.5\textwidth}
\includegraphics[width = 1.3cm]{/Users/kruppajo/work/GitHub/exam/avatare/Nilufar.png}
\end{minipage}
\begin{minipage}[t]{0.5\textwidth}
\hfill
\href{https://youtu.be/2QJa19ZwLls}{\includegraphics[width = 2cm]{img/youtube}}
\end{minipage}
\vspace{-3ex}
%% --------------------------------------------------------------------



\paragraph{Schätzen der Korrelation und des Bestimmtheitsmaßes}

Der Bildschirm strahlt blau in das Gesicht von Nilufar. Es ist schon spät. Und das hat einen Grund. Hm, lecker Takis Blue Heat und dazu dann im Hintergrund Star Trek laufen lassen. . Nilufar überlegt, aber ihre Gedaken sind etwas zäh. 'Was soll das hier alles bedeuten?', fragt sich Nilufar. Irgendwie ist ihr nicht klar wie sie $\rho$-Werte oder $R^2$-Werte abschätzen soll. Alles nicht so einfach. Wenn die Erwartung nicht wäre, ja dann wäre wohl vieles möglich für Nilufar! Aber so.. 
\vspace{2Ex}



{\centering \includegraphics[width=\maxwidth]{img/correlation-02-1} 

}




Leider kennt sich Nilufar mit der Korrelationsanalyse und der linearen Regression überhaupt nicht aus. Deshalb braucht sie bei der Auswertung Ihre Hilfe!

\begin{enumerate}
\item Schätzen Sie die $\rho$-Werte in den Abbildungen! \textbf{(2 Punkte)}
\item Schätzen Sie die $R^2$-Werte in den Abbildungen! \textbf{(2 Punkte)}
\item Interpretieren Sie die $R^2$-Werte für die jeweilige Gerade! \textbf{(2 Punkte)}
\item Was ist der optimale $R^2$-Wert im Kontext einer wissenschaftlichen Fragestellung? Begründen Sie Ihre Antwort an einem Beispiel! \textbf{(2 Punkte)}
\item Was ist der optimale $\rho$-Wert im Kontext einer wissenschaftlichen Fragestellung? Begründen Sie Ihre Antwort an einem Beispiel! \textbf{(2 Punkte)}
\item Erklären Sie die Aussage \textit{"Correlation does not imply causation!"} an einem Beispiel! \textbf{(2 Punkte)}
\end{enumerate} 
\clearpage
% -----------------------------------------------------------------------

\section{Aufgabe \hfill (11 Punkte)}

\textit{Geben Sie grundsätzlich Formeln und Rechenweg zur Lösung der Teilaufgaben mit an!} \\[1Ex]
 

 
%% --------------------------------------------------------------------
\begin{minipage}[t]{0.5\textwidth}
\includegraphics[width = 1.3cm]{/Users/kruppajo/work/GitHub/exam/avatare/Alex.png}
\end{minipage}
\begin{minipage}[t]{0.5\textwidth}
\hfill
\href{https://youtu.be/dyQlYV9nOqY}{\includegraphics[width = 2cm]{img/youtube}}
\end{minipage}
\vspace{-3ex}
%% --------------------------------------------------------------------



\paragraph{Modellgüte der linearen Regression}

'Oh! Residuen. Die waren wichtig um zu wissen, ob eine Modellierung funktioniert hat! Da schauen wir uns dann mit der Funktion \texttt{augment()} die Werte der einzelnen Residuen an. Oder gleich den Residuenplot...da sehen wir dann... ja was eigentlich?', verkündet Alex stolz. Leider hat Alex vergessen wie der \Rlogo Code für den Residuenplot geht. Alex hatte anderes im Kopf. Schon dutzende Male gesehen: Alien. Aber immer noch großartig zusammen mit Gummibärchen. Aber sowas hilft ihm natürlich hier nicht. Da schmeißt sich Alex noch ein paar Gummibärchen in den Mund und kaut los.

\begin{knitrout}
\definecolor{shadecolor}{rgb}{0.969, 0.969, 0.969}\color{fgcolor}\begin{table}[!h]
\centering\begingroup\fontsize{12}{14}\selectfont

\begin{tabular}{cccc}
\toprule
Durchschnittlicher Bewegungsscore & Gewichtszuwachs & $\hat{y}$ & $\phantom{ttttt}\epsilon\phantom{ttttt}$\\
\midrule
23.8 & 10.4 & 24.4 & \\
24.1 & 11.0 & 25.3 & \\
30.0 & 13.1 & 28.6 & \\
29.0 & 15.4 & 32.0 & \\
29.3 & 14.4 & 30.6 & \\
\addlinespace
34.8 & 14.1 & 30.1 & \\
28.9 & 15.0 & 31.4 & \\
27.9 & 12.2 & 27.1 & \\
16.7 & 5.3 & 16.7 & \\
21.6 & 9.5 & 23.0 & \\
\addlinespace
31.4 & 12.9 & 28.3 & \\
\bottomrule
\end{tabular}
\endgroup{}
\end{table}

\end{knitrout}

Leider kennt sich Alex mit der linearen Regression überhaupt nicht aus. Deshalb braucht er bei der Auswertung Ihre Hilfe!

\begin{enumerate}
  \item Formulieren Sie die wissenschaftliche Fragestellung! \textbf{(1 Punkt)}
\item Ergänzen Sie die Werte der Residuen $\epsilon$ in der obigen Tabelle! \textbf{(2 Punkte)}
\item Zeichnen Sie den Boxplot der Residuen $\epsilon$. Beschriften Sie die Abbildung! \textbf{(2 Punkte)}
\item Zeichnen Sie den Residualplot. Beschriften Sie die Abbildung! \textbf{(2 Punkte)}
\item Gibt es auffällige Werte anhand des Residualplots? Begründen Sie Ihre Antwort! \textbf{(2 Punkte)}
\item Erklären Sie die Eigenschaft eines statistischen Modells, welche mit dem Residualplot überprüft wird! Begründen Sie Ihre Antwort anhand einer Visualisierung! \textbf{(2 Punkte)}
\end{enumerate}
 
\clearpage
% -----------------------------------------------------------------------

\section{Aufgabe \hfill (12 Punkte)}

\textit{Geben Sie grundsätzlich Formeln und Rechenweg zur Lösung der Teilaufgaben mit an!} \\[1Ex]
 

 
%% --------------------------------------------------------------------
\begin{minipage}[t]{0.5\textwidth}
\includegraphics[width = 1.3cm]{/Users/kruppajo/work/GitHub/exam/avatare/Nilufar.png}\hspace{-4mm}\includegraphics[width = 1.3cm]{/Users/kruppajo/work/GitHub/exam/avatare/Steffen.png}
\end{minipage}
\begin{minipage}[t]{0.5\textwidth}
\hfill
\href{https://youtu.be/kHmfEmU6lrk}{\includegraphics[width = 2cm]{img/youtube}}
\end{minipage}
%% --------------------------------------------------------------------



\paragraph{Visualisierung des Regressionskreuzes}

Steffen hat ein Kreuzungsexperiment mit Schweinen duchgeführt. Soweit so gut. Dann war er bei seinem Betreuer. Leider war der Schritt nicht so hilfreich.  Eine echte Herausforderung für ihn war schon immer die Romantik gewesen. Ein leidiges Lied. Aber es muss ja weitergehen. Steffen hatte dann in seiner Abschlusarbeit einfach zu viele Endpunkte gemessen und ist jetzt vollkommen durcheinander, welche Analyse er nun wie rechnen soll. Naja, dann heißt es jetzt eben Taylor Swift aufdrehen und darüber nachdenken, was hier eigentlich gemacht wurde. Steffen fängt einfach an und nimmt den ersten Endpunkt Schlachtgewicht [kg]. Dann kann er sich voran arbeiten. Später dann noch raus um zu Ringen um mal zu entspannen und vielleicht ist Nilufar auch da. Wäre toll.

\vspace{1Ex}

Leider kennt sich Steffen mit dem Kontext der linearen Regression überhaupt nicht aus. Deshalb braucht er bei der Auswertung Ihre Hilfe!

\begin{enumerate}
  \item Formulieren Sie die wissenschaftliche Fragestellung! \textbf{(1 Punkt)}
\item Zeichen Sie die Zeile des Regressionskreuzes für den Endpunkt mit \underline{drei} Feldern! Beschriften Sie die Abbildung! \textbf{(4 Punkte)}
\item Ergänzen Sie die entsprechenden statistische Methoden zur Analyse in jedem Feld! \textbf{(2 Punkte)}
\item Formulieren Sie die Nullhypothese für die statistische Methode in jedem Feld! \textbf{(2 Punkte)}
\item Ergänzen Sie die entsprechenden Funktionen in \Rlogo zur Analyse in jedem Feld! \textbf{(2 Punkte)}
\item Welchen Effekt erhalten Sie in jedem Feld? Geben Sie ein Beispiel! \textbf{(2 Punkte)}
\end{enumerate} 
\clearpage
% -----------------------------------------------------------------------
\part{Experimentelles Design}
% -----------------------------------------------------------------------

\section{Aufgabe \hfill (16 Punkte)}


 
%% --------------------------------------------------------------------
\begin{minipage}[t]{0.5\textwidth}
\includegraphics[width = 1.3cm]{/Users/kruppajo/work/GitHub/exam/avatare/Jessica.png}\hspace{-4mm}\includegraphics[width = 1.3cm]{/Users/kruppajo/work/GitHub/exam/avatare/Paula.png}\hspace{-4mm}\includegraphics[width = 1.3cm]{/Users/kruppajo/work/GitHub/exam/avatare/Steffen.png}
\end{minipage}
\begin{minipage}[t]{0.5\textwidth}
\hfill
\href{https://youtu.be/wJqsNV1hOW8}{\includegraphics[width = 2cm]{img/caution}}
\end{minipage}
%% --------------------------------------------------------------------



\paragraph{Einfache experimentelle Designs}

Die Schlange macht mal wieder Randale in Steffens Zimmer und rennt davon! Paula und Jessica sind bei Steffen in im Wendland wo der neue, bessere Versuch stattfinden soll. Dabei soll in einem Gewächshausexperiment im Wendland mit Spargel durchgeführt werden. Auf dem Tisch stapeln sich Oreos aus Vollkorndinkelmehl. Eine Spezialtät der Komune hier. Paula hasst Vollkorn wie Schokobons geliebt werden. In dem neuen Versuch geht es um den Zusammenhang zwischen der Behandlung Düngestufen ($ctrl$, $low$, $mid$ und $high$) und dem Messwert Trockengewicht [kg/ha]. Immerhin ist der Messwert normalverteilt, was einges einfacher macht. Was es nicht so einfacher macht ist, dass Jessica als zusätzliche Herausforderung noch der Mangel mitgebracht hat. Daher entscheiden sich alle drei für ein \textit{Complete randomized design (CRD)}. 'Naja, so viel einfacher ist es dann doch nicht...', merkt Jessica an und sucht die Schlange.\\

Leider kennen sich Steffen, Paula und Jessica mit dem \textit{Complete randomized design (CRD)} überhaupt nicht aus. Deshalb brauchen die Drei bei der Erstellung Ihre Hilfe!

\begin{enumerate}
  \setcounter{enumi}{0}
  \item Formulieren Sie die wissenschaftliche Fragestellung! \textbf{(1 Punkt)}
  \item Formulieren Sie das statistische Hypothesenpaar! \textbf{(1 Punkt)}
  \item Skizzieren Sie das faktorielle Versuchsdesign! \textbf{(3 Punkte)}
  \item Skizzieren Sie eine Datentabelle für das faktorielle Versuchsdesign in \Rlogo! \textbf{(2 Punkte)}
  \item Erstellen Sie das statistische Modell in der in \Rlogo üblichen Schreibweise für eine ANOVA! Skizzieren Sie die notwendige Funktionen in \Rlogo! \textbf{(3 Punkte)}
  \item Skizzieren Sie die weitere Datenanalyse hinsichtlich eines multiplen Gruppenvergleiches! \textbf{(2 Punkte)}
  \item Skizzieren Sie eine mögliche Abbildung im Kontext der wissenschaftlichen Fragestellung! Beschriften Sie die Abbildung! \textbf{(2 Punkte)}
  \item Ergänzen Sie zu der Abbildung ein mögliches Ergebnis des multiplen Gruppenvergleichs! Begründen Sie Ihre Antwort! \textbf{(2 Punkte)}
\end{enumerate}


 
\clearpage
% -----------------------------------------------------------------------

\section{Aufgabe \hfill (20 Punkte)}


 
%% --------------------------------------------------------------------
\begin{minipage}[t]{0.5\textwidth}
\includegraphics[width = 1.3cm]{/Users/kruppajo/work/GitHub/exam/avatare/Mark.png}\hspace{-4mm}\includegraphics[width = 1.3cm]{/Users/kruppajo/work/GitHub/exam/avatare/Steffen.png}\hspace{-4mm}\includegraphics[width = 1.3cm]{/Users/kruppajo/work/GitHub/exam/avatare/Yuki.png}
\end{minipage}
\begin{minipage}[t]{0.5\textwidth}
\hfill
\href{https://youtu.be/wJqsNV1hOW8}{\includegraphics[width = 2cm]{img/caution}}
\end{minipage}
%% --------------------------------------------------------------------



\paragraph{Fortgeschrittene experimentelle Designs}

Steffen und Yuki sind bei Mark um sich Hilfe für eine Versuchsplanung in \Rlogo zu holen. Im Hintergrund läuft viel zu laut Taylor Swift. Dabei geht es um den Zusammenhang zwischen der Behandlung Düngestufen ($ctrl$, $low$, $mid$ und $high$) sowie Genotypen ($AA$, und $BB$) sowie vier Blöcken und dem Messwert Frischegewicht [kg/ha] in Spargel. Der Versuch soll in einem Freilandversuch im Oldenburger Land durchgeführt werden. Nach der Dozentin ist der Messwert Frischegewicht [kg/ha] normalverteilt. Die beiden entschieden sich für ein faktorielles Versuchsdesign. Im ersten Schritt überlegt Mark ein komplexeres experimentelles Design zu probieren. Daher entscheiden sich alle drei für ein \textit{Split plot design oder auch Spaltanlage} mit Berücksichtigung einer Interaktion. Das sollte für den anfang erstmal reichen. 'Und jetzt, was machen wir jetzt?', Yuki schaut die anderen beiden mit großen Augen an. Die zucken mit der Schulter. Alle mampfen Oreos.\\

Leider kennen sich Mark, Steffen und Yuki mit dem \textit{Split plot design oder auch Spaltanlage} überhaupt nicht aus. Deshalb brauchen die Drei bei der Erstellung Ihre Hilfe!

\begin{enumerate}
  \setcounter{enumi}{0}
  \item Formulieren Sie die wissenschaftliche Fragestellung! \textbf{(1 Punkt)}
  \item Formulieren Sie die statistische Hypothesenpaare! \textbf{(2 Punkte)}
  \item Skizzieren Sie das faktorielle Versuchsdesign! \textbf{(3 Punkte)}
  \item Skizzieren Sie eine Datentabelle für das faktorielle Versuchsdesign in \Rlogo! \textbf{(2 Punkte)}
  \item Erstellen Sie das statistische Modell in der in \Rlogo üblichen Schreibweise für eine ANOVA! Skizzieren Sie die notwendige Funktionen in \Rlogo! \textbf{(4 Punkte)}
  \item Skizzieren Sie die weitere Datenanalyse hinsichtlich eines multiplen Gruppenvergleiches! \textbf{(2 Punkte)}
  \item Skizzieren Sie eine mögliche Abbildung im Kontext der wissenschaftlichen Fragestellung! Beschriften Sie die Abbildung! \textbf{(3 Punkte)}
  \item Ergänzen Sie zu der Abbildung ein mögliches Ergebnis des multiplen Gruppenvergleichs! Welche Annahme hinsichtlich der Modellierung haben Sie getroffen? Begründen Sie Ihre Antwort! \textbf{(3 Punkte)}
\end{enumerate} 
\clearpage
% -----------------------------------------------------------------------
\part{Forschendes Lernen}

\subsection*{Zerforschen einer wissenschaftlichen Veröffentlichung}

Das forschende Lernen basiert zum einen auf den folgenden wissenschaftlichen Veröffentlichungen. Für die Prüfung wird die vertiefende Kenntnis der folgenden Veröffentlichungen vorausgesetzt.\\

\begin{itemize}[noitemsep]
\item Sánchez, M., et al. (2022). Hoverfly pollination enhances yield and fruit quality in mango under protected cultivation. Scientia Horticulturae, 304, 111320. [\href{https://www.sciencedirect.com/science/article/pii/S0304423822004411}{Link}]
% \item Salinas, I., et al. (2023). Plant growth, yield, and fruit size improvements in ‘Alicia’papaya multiplied by grafting. Plants, 12(5), 1189. [\href{https://www.mdpi.com/2223-7747/12/5/1189}{Link}]
\item Petersen, F., et al. (2022). Influence of light intensity and spectrum on duckweed growth and proteins in a small-scale, re-circulating indoor vertical farm. Plants, 11(8), 1010. [\href{https://www.mdpi.com/2223-7747/11/8/1010}{Link}]
\item Selle, P. H., et al. (2010). Implications of sorghum in broiler chicken nutrition. Animal Feed Science and Technology, 156(3-4), 57-74. [\href{https://www.sciencedirect.com/science/article/pii/S0377840110000209}{Link}]
\item Wu, G., et al. (2004). Arginine nutrition in neonatal pigs. The Journal of Nutrition, 134(10), 2783S-2790S. [\href{https://www.sciencedirect.com/science/article/pii/S0022316623031279}{Link}]
\item Graham, J., E., et al. (2024) Stock assessment models overstate sustainability of the world’s fisheries. Science 385, 860-865. [\href{https://www.science.org/doi/10.1126/science.adl6282}{Link}]
\end{itemize}

\textit{In der Prüfung erhalten Sie \underline{keinen Auszug} der wissenschaftlichen Veröffentlichung! Die Veröffentlichungen werden als \underline{bekannt} in der Prüfung vorgesetzt. Sie haben sich vorab Notizen und Anmerkungen auf Ihrem Spickzettel gemacht.}

\subsection*{Zerforschen eines wissenschaftlichen Datensatzes}

Das forschende Lernen basiert zum anderen auf den folgenden wissenschaftlichen Datensätzen und deren vertiefende Analyse werden als bekannt vorausgesetzt. Die Datensätze werden über ILIAS bereitgestellt.\\

\begin{itemize}[noitemsep]
\item \texttt{salad\_fert\_weight.xlsx} % Gruppenvergleich
\item \texttt{weight\_gain\_pig.xlsx} % Regression - Prädiktion
\item \texttt{flowercolor\_data.xlsx} % Regression - Kausal
\item \texttt{chickentype\_class.xlsx} % Clusteranalyse
\end{itemize}

\textit{In der Prüfung erhalten Sie \underline{keinen Auszug} aus den wissenschaftlichen Daten. Die Datensätze werden als \underline{bekannt} in der Prüfung vorgesetzt. Sie haben sich vorab Notizen und Anmerkungen auf Ihrem Spickzettel gemacht.}


\clearpage
% -----------------------------------------------------------------------

\section{Aufgabe \hfill (20 Punkte)}

\textit{Geben Sie grundsätzlich Formeln und Rechenweg zur Lösung der Teilaufgaben mit an!} \\[1Ex]
 

 
%% --------------------------------------------------------------------
\begin{minipage}[t]{0.5\textwidth}
\includegraphics[width = 1.3cm]{/Users/kruppajo/work/GitHub/exam/avatare/Tina.png}
\end{minipage}
\begin{minipage}[t]{0.5\textwidth}
\hfill
\href{https://youtu.be/C9skfFRTHhI}{\includegraphics[width = 2cm]{img/caution}}
\end{minipage}
%% --------------------------------------------------------------------



\paragraph{Zerforschen einer wissenschaftlichen Veröffentlichung}

'Uff', denkt Tina. Das ist jetzt doch etwas umfangreicher. Tina soll die wissenschaftliche Veröffentlichung \textit{Sánchez, M., et al. (2022). Hoverfly pollination enhances yield and fruit quality in mango under protected cultivation} einmal zusammenfassen. Die Arbeit soll als eine Vorlage für ihre eigene Arbeit dienen. Daher möchte ihr Betreuer, dass sie einmal die Veröffentlichung in einer PowerPoint Präsentation zusammenfasst. 'Das ist jetzt aber doch umfangreicher als gedacht.', mault Tina in sich hinein. Schnell nochmal ein paar Katjes zur Stärkung gegessen. Das wird dann vermutlich heute Abend nichts mehr mit ihrem Hobby Astronomie. Die Spinne schaut mitleidig.\\

Leider kennt sich Tina mit dem Lesen einer wissenschaftlichen Veröffentlichung mit Fokus auf die Statistik überhaupt nicht aus. Deshalb braucht sie bei der Erstellung Ihre Hilfe! Glücklicherweise kennen Sie die wissenschaftliche Veröffentlichung schon im Detail und können sofort helfen.

\begin{enumerate}
  \setcounter{enumi}{0}
  \item Erläutern Sie die wissenschaftliche Fragestellung der wissenschaftlichen Veröffentlichung anhand des OCAR Prinzips nach Schimel (2012)\footnote{Schimel, J. (2012). Writing science: how to write papers that get cited and proposals that get funded. OUP USA.} \textbf{(4 Punkte)}
  \item Nennen Sie die untersuchten Endpunkte in der wissenschaftlichen Veröffentlichung! Wie lautet der primäre Endpunkt? \textbf{(2 Punkte)} 
\item Erstellen Sie das statistische Modell in der in \Rlogo üblichen Schreibweise! \textbf{(2 Punkte)}
  \item Nennen Sie eine Auswahl an bedeutenden statistischen Maßzahlen in der wissenschaftlichen Veröffentlichung! \textbf{(1 Punkt)}
  \item Interpretieren Sie die Hauptaussage der wissenschaftlichen Veröffentlichung hinsichtlich der Signifkanz für den primären Endpunkt! \textbf{(2 Punkte)}
  \item Interpretieren Sie die Hauptaussage der wissenschaftlichen Veröffentlichung hinsichtlich der Effektstärke für den primären Endpunkt! \textbf{(2 Punkte)}
  \item Diskutieren Sie die ökonomische Relevanz der Hauptaussage der wissenschaftlichen Veröffentlichung im Bezug auf Signifikanz und Effektstärke für den primären Endpunkt! \textbf{(1 Punkt)}
  \item Skizzieren Sie für den primären Endpunkt den sich ergebenden Datensatz in \Rlogo für eine ausgewählte Abbildung! \textbf{(2 Punkte)}
\item Skizzieren Sie einen möglichen Versuchsplan für den primären Endpunkt! \textbf{(2 Punkte)}
  \item Schätzen Sie die benötigte Fallzahl für ein zukünftiges Experiment anhand der Ergebnisse in der wisenschaftlichen Veröffentlichung für den primären Endpunkt! \textbf{(2 Punkte)}
\end{enumerate} 
\clearpage
% -----------------------------------------------------------------------

\section{Aufgabe \hfill (20 Punkte)}

\textit{Geben Sie grundsätzlich Formeln und Rechenweg zur Lösung der Teilaufgaben mit an!} \\[1Ex]
 

 
%% --------------------------------------------------------------------
\begin{minipage}[t]{0.5\textwidth}
\includegraphics[width = 1.3cm]{/Users/kruppajo/work/GitHub/exam/avatare/Yuki.png}
\end{minipage}
\begin{minipage}[t]{0.5\textwidth}
\hfill
\href{https://youtu.be/C9skfFRTHhI}{\includegraphics[width = 2cm]{img/caution}}
\end{minipage}
%% --------------------------------------------------------------------



\paragraph{Zerforschen eines wissenschaftlichen Datensatzes}

'Uff', denkt Yuki Das ist jetzt doch etwas umfangreicher. Yuki soll die wissenschaftlichen Daten \textit{data3} einmal sinnvoll analysieren und zusammenfassen. Die Daten sollen als eine Vorlage für ihre eigene Arbeit und Experiment dienen. Daher möchte ihre Betreuerin, dass sie einmal die Daten in einer PowerPoint Präsentation zusammenfasst. 'Das ist jetzt aber doch umfangreicher als gedacht.', mault Yuki in sich hinein und mampft noch ein paar Reese's Peanut Butter Cups. Dann starrt sie eine Weile in ihren Laptop. Das wird dann vermutlich heute Abend nichts mehr mit Matrix\\

Leider kennt sich Yuki mit der Analyse eines wissenschaftlichen Datensatzes überhaupt nicht aus. Deshalb braucht sie bei der Auswertung Ihre Hilfe! Glücklicherweise kennen Sie den wissenschaftlichen Datensatz aus Ihren eigenen Analysen schon im Detail und können sofort helfen.

\begin{enumerate}
  \setcounter{enumi}{0}
  \item Formulieren Sie die wissenschaftliche Fragestellung des Datensatzes in Form einer PowerPoint Folie! \textbf{(2 Punkte)}
  \item Nennen Sie zwei Besonderheiten des Datensatzes! Begründen Sie Ihre Antwort! \textbf{(2 Punkte)}
  \item Nennen Sie die untersuchten Endpunkte in dem Datensatz! Wie lautet der primäre Endpunkt für die Auswertung? \textbf{(2 Punkte)}
  \item Skizzieren Sie die großen Analysebereiche der Statistik! Beschriften Sie die Abbildungen! \textbf{(2 Punkte)}
  \item In welchen der großen Analysebereiche der Statistik fällt die Auswertung des primären Endpunktes? Begründen Sie Ihre Antwort! \textbf{(2 Punkte)}
  \item Skizzieren Sie eine ikonische Abbildung für den primären Endpunkt im Kontext der wissenschaftlichen Fragestellung! \textbf{(2 Punkte)}
  \item Erstellen Sie das statistische Modell in der in \Rlogo üblichen Schreibweise! \textbf{(2 Punkte)}
  \item Skizzieren Sie die Datenanalyse hinsichtlich der Signifkanz für den primären Endpunkt! \textbf{(2 Punkte)}
  \item Skizzieren Sie die Berechnung der Effektstärke für den primären Endpunkt! \textbf{(2 Punkte)}
  \item Skizzieren Sie einen möglichen Versuchsplan für den primären Endpunkt! \textbf{(2 Punkte)}
\end{enumerate} 
\clearpage
% -----------------------------------------------------------------------
\part{Mathematik}
% -----------------------------------------------------------------------  

\section{Aufgabe \hfill (10 Punkte)}

\textit{Geben Sie grundsätzlich Formeln und Rechenweg zur Lösung der Teilaufgaben mit an!} \\[1Ex]
 

 
%% --------------------------------------------------------------------
\begin{minipage}[t]{0.5\textwidth}
\includegraphics[width = 1.3cm]{/Users/kruppajo/work/GitHub/exam/avatare/Alex.png}\hspace{-4mm}\includegraphics[width = 1.3cm]{/Users/kruppajo/work/GitHub/exam/avatare/Jessica.png}\hspace{-4mm}\includegraphics[width = 1.3cm]{/Users/kruppajo/work/GitHub/exam/avatare/Mark.png}\hspace{-4mm}\includegraphics[width = 1.3cm]{/Users/kruppajo/work/GitHub/exam/avatare/Yuki.png}
\end{minipage}
\begin{minipage}[t]{0.5\textwidth}
\hfill
\href{https://youtu.be/Fu8kN0Uj13Y}{\includegraphics[width = 2cm]{img/youtube}}
\end{minipage}
%% --------------------------------------------------------------------



\paragraph{Herodot – der Schimmel aus Ivenack}

Die Lerngruppe \textit{Die Pantoffeltieren} bestehend aus Mark, Yuki, Alex und Jessica waren auf Exkursion in Brandenburg und haben dort Folgendes erarbeitet. Während der Besetzung Mecklenburgs durch die Franzosen kamen Napoleon die Geschichten des berühmten Apfelschimmels Herodot aus Ivenack zu Gehör. Herodot lief zwar niemals Rennen, war aber eines der berühmtesten Pferde dieser Zeit. Napoleon selbst gab den Auftrag, diesen Schimmel durch die Armee nach Frankreich zu bringen. Der Legende nach sollen Arbeiter den Schimmel im hohlen Stamm einer 1000-jährigen Eiche aus Ivenack vor den Franzosen versteckt haben. Doch Herodot verriet sein Versteck durch lautes Wiehern, woraufhin die französische Armee den Schimmel beschlagnahmte und nach Frankreich führte\footnote{Die Quelle der Inspiration  für die Aufgabe war eine Fahrt an die Ostsee und folgender Artikel:
  \href{https://www.wald-mv.de/landingpage/ivenacker-eichen/}{Entdecke das erste Nationale Naturmonument Deutschlands - Ivenacker Eichen und Hutewald}}. Jetzt wollen die vier herausfinden: \textit{"Konnten die Ivenacker den Apfelschimmel Herodot vor dem Zugriff von Napoleon in der 1000-jährigen Eiche verstecken?"} 



\vspace{1Ex}

Helfen Sie der Lerngruppe \textit{Die Pantoffeltieren} bei der Beantwortung der Forschungsfrage! Gehen Sie von einem radialen Wachstum der 1000-jährigen Eiche von $1mm$ pro Jahr aus. Es ist bekannt, dass die Eiche im Jahr 2022 einen Umfang von $12m$ in Brusthöhe hatte.

\begin{enumerate}
\item Wie groß war der Durchmesser in $m$ der Eiche im Jahr $1840$ als Herodot in der Eiche versteckt werden sollte? \textbf{(2 Punkte)}
\item Skizzieren Sie in einer Abbildung einen linearen Zusammenhang und einen exponentiellen Zusammenhang für das Wachstum der 1000-jährigen Eiche. Erklären Sie die Auswirkungen der Entscheidung für linear oder exponentiell auf Ihre Berechnungen! \textbf{(2 Punkte)}
\end{enumerate}
 
Herodot hatte eine Schulterhöhe von $195$cm, eine Breite von $90$cm sowie eine Länge von  $250$cm.

\begin{enumerate}
  \setcounter{enumi}{2}
\item Berechnen Sie das effektive Volumen von Herodot in $m^3$, welches Herodot in der 1000-jährigen Eiche einnehmen würde! \textbf{(2 Punkte)}
\end{enumerate}

Es wurde berichtet, dass sich Herodot in der 1000-jährigen Eiche $sehr bequem mit Abstand$ um die eigene Achse drehen konnte.

\begin{enumerate}
  \setcounter{enumi}{3}
\item Berechnen Sie die Dicke der Eichenwand in $cm$! Verdeutlichen Sie Ihre Berechnungen an einer aussagekräftigen Skizze für Pferd und Eiche! \textbf{(2 Punkte)} 
\item Unter einer Dicke der Eichenwand von $10cm$ bricht die Eiche zusammen. Beantworten Sie die Forschungsfrage! Begründen Sie Ihre Antwort! \textbf{(2 Punkte)} 
\end{enumerate}
 
\clearpage
% ----------------------------------------------------------------------- 

\section{Aufgabe \hfill (10 Punkte)}

\textit{Geben Sie grundsätzlich Formeln und Rechenweg zur Lösung der Teilaufgaben mit an!} \\[1Ex]
 

 
%% --------------------------------------------------------------------
\begin{minipage}[t]{0.5\textwidth}
\includegraphics[width = 1.3cm]{/Users/kruppajo/work/GitHub/exam/avatare/Jonas.png}\hspace{-4mm}\includegraphics[width = 1.3cm]{/Users/kruppajo/work/GitHub/exam/avatare/Nilufar.png}\hspace{-4mm}\includegraphics[width = 1.3cm]{/Users/kruppajo/work/GitHub/exam/avatare/Steffen.png}\hspace{-4mm}\includegraphics[width = 1.3cm]{/Users/kruppajo/work/GitHub/exam/avatare/Tina.png}
\end{minipage}
\begin{minipage}[t]{0.5\textwidth}
\hfill
\href{https://youtu.be/57B-yYoFSk0}{\includegraphics[width = 2cm]{img/youtube}}
\end{minipage}
%% --------------------------------------------------------------------




\paragraph{Von Töpfen auf Tischen}



Die Projektgruppe \textit{H} bestehend aus Jonas, Steffen, Tina und Nilufar hat sich zusammengefunden um den ersten Versuch zu planen. In einem Experiment wollen sie die Wuchshöhe von 180 Sonnenblumen bestimmen. Bevor die Vier überhaupt mit dem Experiment beginnen können, gibt es aber ein paar Abschätzungen über die Kosten und den Aufwand zu treffen. Zum einen müssen sie die Sonnenblumen einpflanzen und müssen dafür Substrat bestellen. Zum anderen muss die Projektgruppe die Sonnenblumen auch bewegen und in ein Gewächshaus platzieren. Die Töpfe für die Keimung haben
einen Durchmesser von 8.5cm und eine Höhe von 7cm. Der Kubikmeterpreis für Torf liegt bei 280 EUR.

\vspace{1Ex}

Helfen Sie der Projektgruppe \textit{H} bei der Planung des Versuches!

\begin{enumerate}
\item Skizzieren Sie den Versuchsplan auf \textit{zwei} Tischen im Gewächshaus! \textbf{(2 Punkte)}
\item Berechnen Sie die benötigte Anzahl an Pflanztöpfen, wenn Sie Randpflanzen mit berücksichtigen wollen! \textbf{(1 Punkt)}
\item Berechnen Sie die benötigte (a) $Pflanztopf$fläche in $m^2$ sowie die (b) $Tischfläche$ in $m^2$ gegeben der Anzahl an Pflanztöpfen inklusive Randpflanzen am Anfang der Keimungsphase! \textbf{(4 Punkte)}
\item Berechnen Sie die benötigte Menge an Torf in Liter $l$, die Sie für das Befüllen der Pflanztöpfe benötigen! Gehen Sie von \textit{einem Zylinder} für die Pflanztöpfe aus!  \textbf{(2 Punkte)}
\item Berechnen Sie die Kosten in EUR für Ihre Torfbestellung! \textbf{(1 Punkt)}
\end{enumerate}



 
\clearpage
% ----------------------------------------------------------------------- 

\section{Aufgabe \hfill (10 Punkte)}

\textit{Geben Sie grundsätzlich Formeln und Rechenweg zur Lösung der Teilaufgaben mit an!} \\[1Ex]
 

 
%% --------------------------------------------------------------------
\begin{minipage}[t]{0.5\textwidth}
\includegraphics[width = 1.3cm]{/Users/kruppajo/work/GitHub/exam/avatare/Mark.png}\hspace{-4mm}\includegraphics[width = 1.3cm]{/Users/kruppajo/work/GitHub/exam/avatare/Nilufar.png}\hspace{-4mm}\includegraphics[width = 1.3cm]{/Users/kruppajo/work/GitHub/exam/avatare/Tina.png}\hspace{-4mm}\includegraphics[width = 1.3cm]{/Users/kruppajo/work/GitHub/exam/avatare/Yuki.png}
\end{minipage}
\begin{minipage}[t]{0.5\textwidth}
\hfill
\href{https://youtu.be/aBxLkdF-c4M}{\includegraphics[width = 2cm]{img/youtube}}
\end{minipage}
%% --------------------------------------------------------------------





\paragraph{Solar- \& Biogasanlagen}



Tina bringt ein neues, tolles Projekt mit in die Lerngruppe \textit{Sinnlos im Studium} bestehend aus ihr, Nilufar, Yuki sowie Mark. Um die Energiekosten ihres Betriebes zu senken, will sie eine Solaranlage auf den Schweinestall montieren lassen. Dafür hat sie ihren Stall ausgemessen und findet folgende Maße wieder. Die vordere Seite des Schweinestall hat eine Höhe $h_v$ von $6.5m$. Die hintere Seite des Schweinestall hat eine Höhe $h_b$ von $11m$. Der Schweinestall hat eine Tiefe $t$ von $12m$ und eine Breite $b$ von $40m$. 'Sag mal Tina, ist das eine Matheaufgabe oder rechnen wir hier gerade für dich kostenlos als menschliche Computer Sachen für deinen Betrieb?', fragt Yuki mit erhobenenen Augenbrauen. Mark und Nilufar nicken zustimmend.

\vspace{1Ex}

Wenn die Lerngruppe nicht will, dann müssen Sie bei der Planung helfen!

\begin{enumerate}
\item Skizzieren Sie den Schweinestall auf dem die Solaranlage montiert werden soll! Ergänzen Sie die Angaben für die Höhen $h_v$, $h_b$, die Tiefe $t$ und die Breite $b$ des Stalls!  \textbf{(2 Punkte)}
\item Berechnen Sie die Fläche der schrägen, neuen Solaranlage auf dem Schweinestall! \textbf{(3 Punkte)}
\end{enumerate}

Ebenfalls plant Tina eine neue Biogasanlage für ihren Betrieb. Der neue Methantank hat einen Radius $r$ von $1.8m$. Leider gibt es ein paar bauliche Beschränkungen auf dem Grundstück. Das Fundament des zylindrischen Methantanks kann nur ein Gewicht von maximal $5t$ aushalten bevor der Tank wegbricht. Tina rechnen eine Sicherheitstoleranz von $15\%$ ein beinhaltend das Gewicht des Methantanks. In flüssiger Form hat Methan bei $-80^\circ\text{C}$ eine Dichte von $240kg/m^3$. Bei $-100^\circ\text{C}$ hat Methan eine Dichte von $270kg/m^3$. Tina betreibt ihre Anlage bei $-85^\circ\text{C}$.

\begin{enumerate}
  \setcounter{enumi}{2}
\item Extrapolieren Sie die effektive Dichte des Methans in dem Methantank! Welche Annahme haben Sie getroffen? \textbf{(1 Punkt)}
\item Berechnen Sie wie viel Kubikmeter $m^3$ in den Methantank gefüllen werden können, bevor das Fundament nachgibt! \textbf{(2 Punkte)}
\item Berechnen Sie die maximale Höhe $h_{max}$ in $m$ für den gefüllten Methantank mit dem Radius $r$, bevor das Fundament wegbricht! \textbf{(2 Punkte)}
\end{enumerate}

 
\clearpage
% -----------------------------------------------------------------------

\section{Aufgabe \hfill (10 Punkte)}

\textit{Geben Sie grundsätzlich Formeln und Rechenweg zur Lösung der Teilaufgaben mit an!} \\[1Ex]
 

 
%% --------------------------------------------------------------------
\begin{minipage}[t]{0.5\textwidth}
\includegraphics[width = 1.3cm]{/Users/kruppajo/work/GitHub/exam/avatare/Alex.png}
\end{minipage}
\begin{minipage}[t]{0.5\textwidth}
\hfill
\href{https://youtu.be/https://youtu.be/k2G52hMIfqk}{\includegraphics[width = 2cm]{img/youtube}}
\end{minipage}
%% --------------------------------------------------------------------

%% --------------------------------------------------------------------
{\tiny\textbf{Stichworte:} Riesenfaultier $\bullet$ Evolution der Avocado $\bullet$ Bluetooth $ \bullet$ Blauzahn $\bullet$ Colonia Dignidad $\bullet$ ODESSA $\bullet$ Rattenlinie $\bullet$ Adolf Eichmann}
%% --------------------------------------------------------------------



\paragraph{Aligatorenbirnen und Blaubeeren}



"'Sind Sie ein Riesenfautier oder warum kaufen Sie so viele Aligatorenbirnen?"', spricht es hinter Ihnen. Irritiert drehen Sie sich um und blicken in das puderrote Gesicht von Alex. "'Wieso?"', entfährt es Ihnen und Sie bereuen sogleich die Frage. Sofort werden Sie zu einem Whiteboard voller roter Schnüre geschliffen und müssen folgenden mathematischen untermauerten Argumenten im Edeka über sich ergehen lassen. Da kommen Sie nicht mehr raus, also können Sie auch gleich mitmachen. Das Problem liegt in Chile\footnote{Die Quelle der Inspiration für die Aufgabe waren folgende Reportagen: \href{https://www.amnesty.ch/de/ueber-amnesty/publikationen/magazin-amnesty/2021-3/bis-zum-letzten-tropfen}{"`Bis zum letzten Tropfen"' in AMNESTY – Magazin der Menschenrechte vom August 2021} und \href{https://www.welthungerhilfe.de/welternaehrung/rubriken/klima-ressourcen/wassernot-in-chile-eine-folge-der-privatisierung}{"`Wasserknappheit in Chile: Eine Folge der Privatisierung?"' in Die Welternährung dem Fachjournal der Welthungerhilfe vom April 2022.}}. Tja, die Deutschen und Südamerika.\\

Zuerst werden Ihre Fähigkeiten getestet, der Mathematik folgen zu können. Oder berechnen Sie gerade den Einkauf von Alex?\\

\begin{enumerate}
\item Wenn 4 Blaubeerschalen 7.96 Euro kosten,  wie viel kosten 5 Schalen? \textbf{(2 Punkte)}
\item Wenn Sie die 5 Blaubeerschalen gekauft haben, wie viele Aligatorbirnen zu je 2.59 EUR können Sie sich dann noch für 50 EUR leisten? \textbf{(1 Punkt)}
\end{enumerate}

Das Whiteboard beinhaltet folgende Liste mit Informationen zum Wasserverbrauch bei der Produktion von Gemüse aus Chile. Seltsam, was man so alles in einem Edeka über Gemüse erfährt.
  
\begin{itemize}[noitemsep]
\item Ein Kilo Strauchtomaten benötigt 180l Wasser. Eine Strauchtomate wiegt 90 - 130g.
\item Ein Kilo Salat benötigt 140l Wasser. Ein Salatkopf wiegt 280 - 510g.
\item Ein Kilo Avocado benötigt 980l Wasser. Eine Avocado wiegt 120 - 380g.
\item Ein Kilo Blaubeeren benötigt 830l Wasser. Eine Blaubeere wiegt 3.2 - 3.6g.
\end{itemize}

\begin{enumerate}
  \setcounter{enumi}{2}
\item Berechnen Sie den Wasserverbrauch für die Produktion für jeweils eine Strauchtomate, einem Salat, einer Avocado und einer Blaubeeren. Stellen Sie das Ergebnis als Tabelle dar! \textbf{(3 Punkte)}
\end{enumerate}

Chile exportiert im großem Ausmaß Blaubeeren und Avocados. In dem Exportjahr 2021 blieben die Erträge von Blaubeeren mit \ensuremath{7.5\times 10^{4}}t in dem prognostizierten Rahmen. Die Menge steigerte sich um 6.1\%. Die Exporte für Avocados fielen in dem gleichen Zeitraum um 22.1\% auf \ensuremath{1.9\times 10^{5}}t.

\begin{enumerate}
  \setcounter{enumi}{3}
\item Wie viele Liter Wasser hat Chile in dem Exportjahr 2020 exportiert? \textbf{(2 Punkte)}
\end{enumerate}

Chile ist eines der wenigen Länder der Welt, die ihr Wasser komplett privatisiert haben. Derzeit sind nur zwei Prozent des Wassers des Landes für den häuslichen Verbrauch vorgesehen. In den Dörfern der Anbauregionen versorgen Tankwagen die Bevölkerung jede Woche mit Wasser, es gibt etwa 48 Liter Wasser pro Kopf für den täglichen Bedarf. In \textit{Deutschland} liegt der Verbrauch bei 35 - 115 Liter pro Waschgang einer Waschmaschine und 9 - 14 Liter pro Sp{"u}lgang.

\begin{enumerate}
  \setcounter{enumi}{4}
\item Mit der rationierten Wassermenge aus Chiles Anbaugebieten können Sie in \textit{Deutschland} wie oft Ihren Bedarf stillen? \textbf{(1 Punkt)}
\end{enumerate}

Das alles hätten Sie nicht von Alex erwartet. Ganz schön viele Informationen wurden da zusammengetragen.

\begin{enumerate}
  \setcounter{enumi}{5}  
  \item Nennen Sie eine \textit{Daten}quelle im Internet, wo Sie mehr Informationen zu landwirtschaftlichen Daten oder klimatischen, wirtschaftlichen und gesellschaftlichen Daten erhalten! \textbf{(1 Punkt)}
\end{enumerate} 
\clearpage
% ----------------------------------------------------------------------- 

\section{Aufgabe \hfill (10 Punkte)}

\textit{Geben Sie grundsätzlich Formeln und Rechenweg zur Lösung der Teilaufgaben mit an!} \\[1Ex]
 

 
%% --------------------------------------------------------------------
\begin{minipage}[t]{0.5\textwidth}
\includegraphics[width = 1.3cm]{/Users/kruppajo/work/GitHub/exam/avatare/Paula.png}
\end{minipage}
\begin{minipage}[t]{0.5\textwidth}
\hfill
\href{https://youtu.be/WZSxntiNF8s}{\includegraphics[width = 2cm]{img/youtube}}
\end{minipage}
%% --------------------------------------------------------------------

%% --------------------------------------------------------------------
{\tiny\textbf{Stichworte:} Kardaschow-Skala $\bullet$ Dyson-Sphäre $\bullet$ Hohlerde $\bullet$ Entropie $\bullet$ Proton $r_P = 1.7 \times 10e-15$ $\bullet$ Wasserstoff $r_H = 5.3\times 10e-11$}
%% --------------------------------------------------------------------

\paragraph{Die Dampfnudelerde}



"'Was für einen Unsinn!"', rufen Sie. Jetzt haben Sie auf Empfehlung von von Paula kostbaren Schlaf prokrastiniert um einem Ernährungswissenschaftler auf YouTube über die Erde als Dampfnudel zu lauschen. Irgendwie passt es dann doch mit der Analogie. Übermüdet müssen Sie darüber nachdenken, warum vor 67 Millionen Jahren die Dinosaurier - so groß sie auch waren - nicht von der Schwerkraft zu Boden gerissen wurden. Hat der Dampfplauderer etwa recht und war die Schwerkraft vor Millionen von Jahren eine andere?  Sind deshalb alle Lebewesen auf der Erde \textit{heutzutage} so viel kleiner, weil die Schwerkraft größer ist als damals? War die Erde kleiner und hatte weniger Masse? Oder ist es nur ein Rechenfehler wie bei der Theorie der Hohlerde von Edmond Halley aus dem 17.–18. Jahrhundert? Müde reiben Sie sich die Augen. So wird es nichts mehr mit dem Schlafen, dann können Sie auch mal etwas rechnen\footnote{Die Quelle der Inspiration
  für die Aufgabe war folgender Artikel:
  \href{https://hpd.de/artikel/erde-dampfnudel-22236}{"Skeptische Anmerkungen --- Die Erde als Dampfnudel" in Der Humanistische Pressedienst}}.  \\

Betrachten wir die Schwerkraft oder Gewichtskraft, die auf Lebewesen damals und heute gewirkt haben soll. Nehmen Sie für die Fallbeschleunigung $g$ der Erde \textit{heutzutage} einen Wert von 9.81m/s$^2$ an. Im Weiteren hat die Erde einen ungefähren Durchmesser von \ensuremath{1.235\times 10^{4}}km und eine mittlere Dichte $\rho$ von 5.44g/cm$^3$. Das Gewicht von einem heute lebenden asiatischen Elefanten liegt bei 3t bis 5t und das Gewicht von einem Brachiosaurus bei bis zu 30t.

\begin{enumerate}
\item Welchen Durchmesser müsste die Erde vor 67 Millionen Jahren gehabt haben, wenn Dinosaurier und Elefanten die gleiche Gewichtskraft $\overrightarrow{F_G}$ damals und heute erfahren hätten? \textit{Beantworten Sie die Frage anhand der folgenden Teilaufgaben!}
\begin{enumerate}
\item Berechnen Sie die Fallbeschleunigung von vor 67 Millionen Jahren unter der obigen Annahme gleich wirkender Gewichtskraft $\overrightarrow{F_G}$ auf Elefant und Dinosaurier! \textbf{(1 Punkt)}
\item Berechnen Sie Masse der heutigen Erde! \textbf{(2 Punkte)}
\item Schließen Sie über die Masse auf den Durchmesser der Erde vor 67 Millionen Jahren! \textbf{(2 Punkte)}
\end{enumerate}
\item Beantworten Sie die Eingangsfrage mit 1-2 Antwortsätzen! \textbf{(1 Punkt)}
\end{enumerate}

Die Distanz zwischen Sonne und Erde entspricht 1.05 astronomische Einheiten ($AE$). Die Einheit 1 AE wird mit \ensuremath{1.5\times 10^{8}}km angegeben. Der \textit{massebehaftete} Sonnenwind besteht aus 87\% Wasserstoffkernen mit einer molaren Masse von 1.08g/mol, 10\% Heliumkernen mit 4.11g/mol sowie 3\% weiteren Atomkernen mit  69.18g/mol. Die Teilchendichte bei Eintritt in die Erdatmosphäre liegt zwischen 0.4 bis 100 Teilchen cm$^{-3}$ pro Sekunde mit einer mittleren Teilchendichte von 7cm$^{-3}$ pro Sekunde. \\

\textit{Lösen Sie den folgenden Aufgabenteil mit einer aussagekräftigen Skizze!}

\begin{enumerate}
  \setcounter{enumi}{2}
\item Berechnen Sie die Anzahl an massebehafteten Teilchen des Sonnenwindes, die die gesamte Erde pro Sekunde treffen! \textbf{(2 Punkte)}
%\item Berechnen Sie die Anzahl an massebehafteten Teilchen des Sonnenwindes, die die Sonne pro Sekunde in alle Richtungen aussendet! \textbf{(2 Punkte)}
\item Berechnen Sie die Masse, die die Erde pro Jahr durch die \textit{massebehafteten} Teilchen des Sonnenwind zunimmt! \textbf{(2 Punkte)}
\end{enumerate}

%\blfootnote{\tiny\textbf{Stichworte:} Kardaschow-Skala $\bullet$ Dyson-Sphäre $\bullet$ Hohlerde $\bullet$ Entropie}

% https://de.wikipedia.org/wiki/Kardaschow-Skala 
\clearpage
% ----------------------------------------------------------------------- 

\section{Aufgabe \hfill (10 Punkte)}

\textit{Geben Sie grundsätzlich Formeln und Rechenweg zur Lösung der Teilaufgaben mit an!} \\[1Ex]
 

 
%% --------------------------------------------------------------------
\begin{minipage}[t]{0.5\textwidth}
\includegraphics[width = 1.3cm]{/Users/kruppajo/work/GitHub/exam/avatare/Alex.png}\hspace{-4mm}\includegraphics[width = 1.3cm]{/Users/kruppajo/work/GitHub/exam/avatare/Paula.png}\hspace{-4mm}\includegraphics[width = 1.3cm]{/Users/kruppajo/work/GitHub/exam/avatare/Steffen.png}\hspace{-4mm}\includegraphics[width = 1.3cm]{/Users/kruppajo/work/GitHub/exam/avatare/Yuki.png}
\end{minipage}
\begin{minipage}[t]{0.5\textwidth}
\hfill
\href{https://youtu.be/n451XnhtSh4}{\includegraphics[width = 2cm]{img/youtube}}
\end{minipage}
%% --------------------------------------------------------------------




\paragraph{'Entschuldigung, ist das Ihre Feder in meinem Auge?'}



So hört man häufiger höfliche Gänse in Mastställen sagen. Das ist natürlich etwas ungünstig, den dann kommt es zu Picken und Kannibalismus. Denn wenn der Nachbar nervt, dann muss zu Maßnahmen gegriffen werden. Kennt jeder aus einer mittelmäßigen Wohngemeinschaft. Das wollen Alex, Yuki, Steffen und Paula aber als vorsorgliche Gänse-Halter:innen nicht\footnote{Die Quelle der Inspiration für die Aufgabe war der folgende wissenschaftliche Artikel: \href{https://www.efsa.europa.eu/en/efsajournal/pub/7788}{EFSA Panel on Animal Health and Welfare, et al. (2023) Welfare of broilers on farm. EFSA Journal 21.2}.}. Gemeinsam sind die Vier in einer Projektgruppe gelandet. Betrachten wir also gemeinsam einmal das Platzangebot (eng. \textit{space allowance}, abk. \textit{SA}) der Gänse für vier Tätigkeiten und versuchen die notwendige Fläche zu optimieren. Wie immer gibt es dafür eine mathematische Formel:

\begin{center}
  \begin{tabular}{cc}
    $SA = \sum^n_{i = 1} (A_i \times PB_i)$ & $A_i = \pi \times (r_i + R_i)^2$\\
  \end{tabular}
\end{center}

\vspace{-2Ex}

mit

\begin{itemize}[noitemsep]
\item $SA$ dem benötigten Platzangebot aller aufsummierten Verhalten $i$.
\item $A_i$ dem benötigten Platz für ein Verhalten $i$. 
\item $PB_i$ dem Anteil des Auftretens eines Verhaltens $i$.
\item $r_i$ dem Radius Gans plus dem benötigten Radius für das Verhalten $i$.
\item $R_i$ dem notwendigen Abstand zu den Nachbarn für das Verhalten $i$.    
\item $i$ dem Verhalten: (1) standing, (2) preening, (3)
  drinking/eating und (4) walking.
\end{itemize}

In der folgenden Tabelle 1 sind die Werte für $r_i$, $R_i$ und $PB_i$ für ein spezifisches Verhalten $i$ aus drei wissenschaftlichen Veröffentlichungen dargestellt.

\vspace{-1Ex}

{\small
\begin{knitrout}
\definecolor{shadecolor}{rgb}{0.969, 0.969, 0.969}\color{fgcolor}\begin{table}[!h]
\centering
\begin{tabular}{llll}
\toprule
  & Aldridge et al. (2021) & Baxter et al. (2022) & Jabcobs et al. (2019)\\
\midrule
standing & 11cm; 27cm; 3.5\% & 22cm; 19cm; 12.1\% & 44cm; 13cm; 10.1\%\\
preening & 50cm; 39cm; 4.3\% & 1cm; 16cm; 2.1\% & 25cm; 19cm; 4.1\%\\
drinking/eating & 23cm; 17cm; 3.6\% & 37cm; 5cm; 6.3\% & 32cm; 18cm; 6.3\%\\
walking & 45cm; 6cm; 13.2\% & 25cm; 20cm; 14.2\% & 14cm; 41cm; 12.3\%\\
\bottomrule
\end{tabular}
\end{table}

\end{knitrout}
}

Leider kennen sich die Vier nicht so gut mit der Berechnung aus! Daher brauchen die Vier Ihre Hilfe!

\begin{enumerate}
\item Skizzieren Sie die Werte $r_i$, $R_i$ und $A_i$ für zwei nebeneinander agierende Gänse für ein Verhalten $i$. Nutzen Sie hierfür vereinfachte geometrische Formen! \textbf{(2 Punkte)}
\item Erstellen Sie eine zusammenfassende Tabelle mit den mittleren Werten für $r$, $R$ und $PB$ aus der obigen Tabelle 1 für die jeweiligen Verhalten! \textbf{(3 Punkte)}
\item Ergänzen Sie eine Spalte mit dem benötigten Platz $A$ für das jeweilige Verhalten, welches sich aus den mittleren Werten ergibt! \textbf{(1 Punkt)}
\item Berechnen Sie das benötigte Platzangebot $SA$ für alle betrachteten Verhalten! \textbf{(1 Punkt)}
\item Sie entnehmen der Literatur folgende Aussage zur Verteilung der Gänse in der Fläche $A$: \textit{"`Assuming, that the animals will optimally and equally distribute in an area $A$, we observe a
    small part, which is not covered. This area is called $\omega$ and is calculated with $\omega = \tfrac{A}{0.9069}$."'} Veranschaulichen Sie die Fläche $\omega$ in einer aussagekräftigen Abbildung!  \textbf{(1 Punkt)}
\item Ein Tier braucht Platz für sich selbst. Berechnen Sie nun die Körperfläche $a$, die ein Tier einnimmt. Welche Annahmen haben Sie für die Berechnung der Körperfläche getroffen? \textbf{(2 Punkte)}
\end{enumerate}



 
\clearpage
% ----------------------------------------------------------------------- 

\section{Aufgabe \hfill (8 Punkte)}

\textit{Geben Sie grundsätzlich Formeln und Rechenweg zur Lösung der Teilaufgaben mit an!} \\[1Ex]
 

 
%% --------------------------------------------------------------------
\begin{minipage}[t]{0.5\textwidth}
\includegraphics[width = 1.3cm]{/Users/kruppajo/work/GitHub/exam/avatare/Jessica.png}\hspace{-4mm}\includegraphics[width = 1.3cm]{/Users/kruppajo/work/GitHub/exam/avatare/Yuki.png}
\end{minipage}
\begin{minipage}[t]{0.5\textwidth}
\hfill
\href{https://youtu.be/1B53cVFIU7Q}{\includegraphics[width = 2cm]{img/youtube}}
\end{minipage}
%% --------------------------------------------------------------------




\paragraph{Nelken von den Molukken}



Jessica und Yuki waren gemeinsam in Berlin und sitzen nun im IC nach Amsterdam um zurück nach Osnabrück zu fahren. 'Weißt du was ich mich frage?', entfährt es Jessica ziemlich plötzlich, so dass Yuki die Reese's Peanut Butter Cups aus dem Mund fallen. 'Nein, und ehrlich gesagt bin ich auch ziemlich müde...'. Das ist jetzt aber Jessica egal, denn sie möchte folgende Sachlage diskutieren. Und Jessica hat jetzt 3 Stunden Zeit. Plus Verspätung. In der Ausstellung \textit{Europa und das Meer} im Deutschen Historischen Museum in Berlin gab es folgendes Zitat über die Probleme der frühen Hochseeschifffahrt.

\begin{quote}
  >>Ohne ausreichende Zufuhr von Vitamin C stellen sich nach 42 Tagen die ersten Symptome ein; die ersten Toten sind nach 72 Tagen zu beklagen; nach 115 Tagen rafft die Skorbut eine ganze Schiffsbesatzung dahin<<
\end{quote}

Ferdinand Magellan stach im Jahre 1519 in See um eine Passage durch den südamerikanischen Kontinent zu finden. Zu seiner Flotte gehörten fünf Schiffe - das Flaggschiff Trinidad, die San Antonio, die Victoria, die Concepciön und die Santiago - mit einer Besatzung von insgesamt 218 Mann. 

\begin{enumerate}
\item Stellen Sie den Verlauf der Anzahl an Matrosen auf einem Schiff der Flotte in der Form einer Überlebenszeitkurve dar! Beschriften Sie die Achsen entsprechend! \textbf{(2 Punkte)} 
\item Was ist die Besonderheit der Überlebenszeitkurve? Begründen Sie Ihre Antwort! \textbf{(2 Punkte)} 
\item Schätzen Sie die Überlebenswahrscheinlichkeit nach 90 Tagen aus Ihrer Abbildung ab! \textbf{(1 Punkt)} 
\end{enumerate}

Der Chronist an Bord der Trinidad, Antonio Pigafetta, schrieb in seinem Bericht '[...] Um nicht Hungers zu sterben, aßen wir das Leder, mit dem die große Rahe zum Schutz der Taue umwunden war.' Insbesondere die Mannschaft der Concepciön erlitt große Verluste durch die Skrobut bei der Überquerung des Pazifiks, da durch Erkundungsfahrten weniger Zeit blieb, um wilden Sellerie aufzunehmen. Wilder Sellerie enthält $7000\mu g/50g$ Vitamin C. Der Bedarf liegt bei $110mg$ pro Tag für Männer.

\begin{enumerate}
  \setcounter{enumi}{2}
\item Berechnen Sie die notwendige Menge in $t$ an aufzunehmenden wilden Sellerie auf die Concepciön für die ununterbrochene Fahrt von drei Monate und 22 Tage über den Pazifik! \textbf{(3 Punkte)}
\item Skizzieren Sie die Überlebenszeitkurve für die Concepciön im Vergleich zu der Überlebenszeitkurve der Trinidad! Beschriften Sie die Achsen! \textbf{(2 Punkte)}
\end{enumerate}

 
\clearpage
% ----------------------------------------------------------------------- 

\section{Aufgabe \hfill (10 Punkte)}

\textit{Geben Sie grundsätzlich Formeln und Rechenweg zur Lösung der Teilaufgaben mit an!} \\[1Ex]
 

 
%% --------------------------------------------------------------------
\begin{minipage}[t]{0.5\textwidth}
\includegraphics[width = 1.3cm]{/Users/kruppajo/work/GitHub/exam/avatare/Alex.png}\hspace{-4mm}\includegraphics[width = 1.3cm]{/Users/kruppajo/work/GitHub/exam/avatare/Paula.png}
\end{minipage}
\begin{minipage}[t]{0.5\textwidth}
\hfill
\href{https://youtu.be/q-qYK4Chslg}{\includegraphics[width = 2cm]{img/youtube}}
\end{minipage}
%% --------------------------------------------------------------------




\paragraph{Event Horizon -- Am Rande des Universums}



Paula ist bei Alex um gemeinsam \textit{Event Horizon -- Am Rande des Universums} zu streamen. Das war jetzt nicht die beste Idee. Denn Paula kann Horror überhaupt nicht ab. Deshalb flüchtet sie sich in Logik um ihre Emotionen zu bändigen. Alex mampft ungerührt Gummibärchen. Folgenden Gedankengang nutzt Paula um dem Film zu entkommen. Die Sonne hat eine aktuelle, angenommene Masse von $\ensuremath{2\times 10^{30}}$kg. Wenn die Sonne nun am Ende ihrer Lebenszeit zu einem schwarzen Loch mit dem Radius von $3500$m kollabiert, wird die Sonne $45$\% der aktuellen Masse verloren haben. Ein Lichtteilchen mit der Masse $m_f$ und der Fluchtgeschwindigkeit $v_f$ will dem schwarzen Loch entkommen. An folgende Formeln erinnert sich Paula für die kinetische Energie des Lichtteilchens $E_{kin}$ und der Graviationsenergie des schwarzen Lochs $E_{grav}$\footnote{Die Quelle der Inspiration für die Aufgabe war ein Montagnachtfilm: \href{https://de.wikipedia.org/wiki/Event_Horizon_–_Am_Rande_des_Universums}{Event Horizon – Am Rande des Universums}}.

\begin{center}
  \begin{tabular}{cc}
    $E_{kin} = \cfrac{1}{2}m_fv_f^2$ & $E_{grav} = \cfrac{Gm_sm_f}{r_s}$\\
  \end{tabular}
\end{center}

mit

\begin{itemize}[noitemsep]
\item $m_f$, gleich der Masse [kg] des fliehenden Objektes
\item $m_s$, gleich der Masse [kg] des stationären Objekts
\item $r_s$, gleich dem Radius [m] des stationären Objekts  
\item $G$, gleich der Gravitationskonstante mit $6.165 \cdot 10^{-11} m^3(kg \cdot s^2)^{-1}$ 
\end{itemize}

Im Folgenden wollen wir Paula bei der Ablenkung helfen und uns mit der Frage beschäftigen, ob das Lichtteilchen der Gravitation des schwarzen Lochs entkommen kann.

\begin{enumerate}
\item Geben Sie die Formel für die Fluchtgeschwindigkeit $v_f$ an! \textbf{(2 Punkte)}
\item Überprüfen Sie Ihre umgestellte Formel nach $v_f$ anhand der Einheiten! \textbf{(1 Punkt)} 
\item Berechnen Sie die notwendige Fluchtgeschwindigkeit $v_f$ des Lichtteilchens mit den angegebenen Informationen! \textbf{(2 Punkte)}
\item Gehen Sie von einer Lichtgeschwindigkeit von $\ensuremath{2.7\times 10^{8}}m/s$ aus. Kann das Lichtteilchen der Gravitation des schwarzen Lochs entkommen? Begründen Sie Ihre Antwort! \textbf{(2 Punkte)}
\item Stellen Sie den Zusammenhang zwischen dem sich verringernden Radius $r$ des schwarzen Lochs bei gleichbleibender Masse $m_s$ und der notwendigen Fluchtgeschwindigkeit $v_f$ in einer Abbildung dar! \textit{Erstellen Sie dafür eine Datentabelle mit fünf Werten für den Radius $r$!} \textbf{(3 Punkte)}
\end{enumerate}

 
\clearpage
% -----------------------------------------------------------------------

\section{Aufgabe \hfill (10 Punkte)}

\textit{Geben Sie grundsätzlich Formeln und Rechenweg zur Lösung der Teilaufgaben mit an!} \\[1Ex]
 

 
%% --------------------------------------------------------------------
\begin{minipage}[t]{0.5\textwidth}
\includegraphics[width = 1.3cm]{/Users/kruppajo/work/GitHub/exam/avatare/Jessica.png}\hspace{-4mm}\includegraphics[width = 1.3cm]{/Users/kruppajo/work/GitHub/exam/avatare/Nilufar.png}
\end{minipage}
\begin{minipage}[t]{0.5\textwidth}
\hfill
\href{https://youtu.be/iCQogS6KhPM}{\includegraphics[width = 2cm]{img/youtube}}
\end{minipage}
%% --------------------------------------------------------------------

%% --------------------------------------------------------------------
{\tiny\textbf{Stichworte:} Great filter $\bullet$ SETI $\bullet$ WOW-Signal $\bullet$ 5-Sigma $\bullet$ Voyager 1 $\bullet$ Voyager 2}
%% --------------------------------------------------------------------




\paragraph{Das Fermi Paradoxon}



Jessica und Nilufar wandern durch den Teuteburgerwald um mal vom Studium runterzukommen. 'Kennst du eigentlich Enrico Fermi?', fragt Jessica und fährt ohne die Antwort abzuwarten fort, 'Er war ein berümter Kernphysiker! Enrico Fermi diskutierte 1950 auf dem Weg zum Mittagessen im Los Alamos National Laboratory mit seinen Kollegen angebliche UFO-Sichtungen und fragte schließlich: >>Where is everybody?<<. Warum seien weder Raumschiffe anderer Weltraumbewohner noch andere Spuren extraterrestrischer Technik zu beobachten?'. Nilufar schaut sie irritiert und interessiert an. Die beiden hat das Problem gepackt. Deshalb wollen Jessica und Nilufar das Paradoxon mal mathematisch untersuchen! Wie lange würde eine außerirdische Zivilisation benötigen um die gesamte Milchstraße zu besuchen, wenn das maximale Reisetempo die Geschwindigkeit der Voyager 1 Sonde wäre?\footnote{Die Quelle der Inspiration für die Aufgabe war folgender Wikipediaeintrag: \href{https://de.wikipedia.org/wiki/Fermi-Paradoxon}{Fermi-Paradoxon}}\\[-1ex]

Die beiden treffen folgende Annahmen. Eine außerirdische Zivilisation schickt $zwei$ Voyager 1 ähnliche Sonden mit der Geschwindigkeit von $\ensuremath{6.2587\times 10^{4}}km/h$ los um sich auf den erreichten Planeten selbst zu replizieren. Nach $250$ Jahren ist die Replikation abgeschlossen und wiederum $zwei$ Sonden werden ausgesendet. Gehen Sie von $3.57$ Lichtjahren als mittlerer Abstand der Sterne in der Milchstraße aus. Es gibt $\ensuremath{2\times 10^{11}}$ Sterne in der Milchstraße. Nehmen Sie eine Lichtgeschwindigkeit von $\ensuremath{2.8\times 10^{8}}m/s$ an.

\begin{enumerate}
\item Skizzieren Sie in einer Abbildung die ersten vier Schritte der Vervielfältigung der Sonden in der Galaxie! Beschriften Sie die Abbildung mit der Dauer und der Anzahl an Sonden für jeden Schritt der Vervielfältigung! \textbf{(4 Punkte)}
\item Berechnen Sie die theoretische Anzahl an Vervielfältigungsschritten die benötigt werden um mit \textit{einem einzigen Vervielfältigungsschritt} die gesamten Sterne der Milchstraße mit Sonden zu besuchen! \textbf{(2 Punkte)}
\item Berechnen Sie die Dauer, die eine außerirdische Zivilisation annährungsweise benötigt um die gesamten Sterne der Milchstraße mit Sonden zu besuchen! \textbf{(2 Punkte)}
\item Bei einem vermutetet Alter der Erde von $\ensuremath{4.1\times 10^{9}}$ Jahren, wie oft war dann eine Sonde einer außerirdischen Zivilisation schon zu Besuch? Korrigieren Sie Ihre Antwort mit dem Wissen, dass sich die Kontinentalplatten einmal alle $\ensuremath{9\times 10^{7}}$ Jahre vollständig im Erdinneren umgewandelt haben! \textbf{(2 Punkte)}
\end{enumerate}


 
\clearpage
% -----------------------------------------------------------------------

\section{Aufgabe \hfill (10 Punkte)}

\textit{Geben Sie grundsätzlich Formeln und Rechenweg zur Lösung der Teilaufgaben mit an!} \\[1Ex]
 

 
%% --------------------------------------------------------------------
\begin{minipage}[t]{0.5\textwidth}
\includegraphics[width = 1.3cm]{/Users/kruppajo/work/GitHub/exam/avatare/Alex.png}
\end{minipage}
\begin{minipage}[t]{0.5\textwidth}
\hfill
\href{https://youtu.be/tDgr6fpkkYA}{\includegraphics[width = 2cm]{img/youtube}}
\end{minipage}
%% --------------------------------------------------------------------




\paragraph{Pyramiden bauen}



Es stehen die niedersächsichen Pyramidentage an! Sie und Alex sind auf abenteuerlichen Wegen für den Bau der Pyramiden zuständig. Zu allem Überfluss handelt es sich auch noch eine \textit{Reenactment} Veranstaltung. Thema der diesjährigen Pyramidentage sind die Pyramiden von Meroe, die den Königen und Königinnen des historischen Reiches von Kusch in Nubien, dem heutigen Sudan, als Grabstätten dienten. Die Pyramiden in Meroe fallen durch ihren steilen Winkel von 70 Grad im Vergleich zu den ägyptischen Pyramiden mit 51 Grad auf. Die durchschnittliche Seitenlänge der Grundfläche einer Pyramide beträgt 44 Königsellen. Eine Königselle misst 52.4cm.\\

\textit{Lösen Sie diese Aufgabe mit Hilfe einer Skizze der Pyramide. Bezeichnen Sie Seiten und die Winkel der Pyramide entsprechend!}

\begin{enumerate}
\item Bei der Königspyramide von Meroe soll eine Seitenlänge der Grundfläche 44 Königsellen lang sein. Welche Höhe der Königspyramide in $m$ ergibt sich? \textbf{(1 Punkt)}
\item Die Außenflächen der Pyramide soll begrünt werden. Für die Bepflanzung muss eine 4cm dicke Torfschicht auf die Pyramide aufgebracht werden. Berechnen Sie die ungefähre Menge an benötigten Torf in $m^3$! \textbf{(2 Punkte)}
\end{enumerate}

Wie in jedem guten \textit{Reenactment} gibt es viel Oberschicht, aber nur 5 Sklaven, die Ihnen und Alex bei dem Befüllen der Pyramide mit Schutt zu Seite stehen. Leider haben Ihre Sklaven zu allem Überfluss auch noch chronische Rückenschmerzen entwickelt, als die Sklaven von der anstehenden Aufgabe erfahren haben. Gehen Sie daher von einer Effizienz der Sklaven von 85\% aus. In eine Schubkarre passen 110 Liter.

\begin{enumerate}
  \setcounter{enumi}{2}
\item Wie oft müssen Ihre maladen Sklaven die Rampe mit der Schubkarre zur Spitze der Pyramide hochfahren um die Pyramide mit Schutt zu füllen? \textbf{(1 Punkt)}
\item Berechnen Sie die Länge der Rampe zur Spitze der Pyramide mit einem Anstellwinkel von $10^\circ$! \textbf{(2 Punkte)}
\item Wie weit reicht Ihre Rampe vom Fuß der Pyramide in die niedersächsiche Landschaft?  \textbf{(2 Punkte)}
\end{enumerate}

Bei der Besichtigung der Pyramide teilt Ihnen der leicht übergewichtige Pharao (Nebenberuf \textit{Mittelständler}) mit, das die Pyramide zu steil sei und somit nicht in die niedersächsiche Landschaft passen würde. Sie müssen nochmal ran.

\begin{enumerate}
  \setcounter{enumi}{5}
\item Die Grundfläche der Pyramide ändert sich nicht. Berechnen Sie die Änderung der Höhe in \underline{Königsellen}, wenn sich der Anstellwinkel der Pyramide um $8^\circ$ ändert!  \textbf{(2 Punkte)}
\end{enumerate}



% https://de.wikipedia.org/wiki/Rechtwinkliges_Dreieck
% https://www.matheretter.de/wiki/pyramide
% https://de.wikipedia.org/wiki/Knickpyramide 
\clearpage
% -----------------------------------------------------------------------

\section{Aufgabe \hfill (10 Punkte)}

\textit{Geben Sie grundsätzlich Formeln und Rechenweg zur Lösung der Teilaufgaben mit an!} \\[1Ex]
 

 
%% --------------------------------------------------------------------
\begin{minipage}[t]{0.5\textwidth}
\includegraphics[width = 1.3cm]{/Users/kruppajo/work/GitHub/exam/avatare/Steffen.png}\hspace{-4mm}\includegraphics[width = 1.3cm]{/Users/kruppajo/work/GitHub/exam/avatare/Tina.png}
\end{minipage}
\begin{minipage}[t]{0.5\textwidth}
\hfill
\href{https://youtu.be/3LAq3R0rS14}{\includegraphics[width = 2cm]{img/youtube}}
\end{minipage}
%% --------------------------------------------------------------------




\paragraph{Geocaching -- Von Satelliten und Plastikdosen}



Es ist Wochenende und das Wetter ist \textit{sweet}. Tina und Steffen schwingen sich auf ihre Cachermobile um mit 16km/h, geleitet von modernster Satellietentechnologie und einem Supercompter aus dem Jahr 2000 in den Händen, Plastikdosen in der Natur und an sehenswerten Orten zu finden. Tina und Steffen wollen diesmal endlich die abwärts Terrainchallenge durchführen. Die Reihenfolge der Caches nach Terrainwertung gibt daher die von den beiden abzufahrenden Orte vor. Die Terrain- und Schwierigkeitswertungen laufen von 1 (leichteste Wertung) bis 5 (schwierigste Wertung) in 0.5 Schritten. Folgende Informationen zu den Orten und den entsprechenden Caches stehen Tina und Steffen für die Planung der Route zu Verfügung\footnote{Die Quelle der Inspiration für die Aufgabe war folgende Tätigkeit: \href{https://www.geocaching.com/play}{Geocaching -- Mach mit bei der weltweit größten Schatzsuche.}}.

\begin{center}
  \begin{tabular}{ ccc }
    \toprule
    Ort & Cache & Wertung (S|T|G) \\
    \midrule
    A & GCYCARA & 2.5 | 5.0 | Normal \\
    B & GCJLPMP & 4.0 | 2.0 | Normal \\ 
    C & GCEHQ0R & 2.0 | 4.5 | Mikro \\ 
    D & GCVPFGE & 1.5 | 2.5 | Klein \\ 
    E & GC0J7NV & 5.0 | 3.5 | Mikro \\     
 \bottomrule
\end{tabular}
\end{center}

Im Weiteren sind den beiden folgende Informationen zu den Entfernungen der Orte zugänglich. Der Entfernungsvektor $\overrightarrow{AC}$ ist $4$km. Im Weiteren ist Ihnen der Entfernungsvektor $\overrightarrow{CB}$ mit $4.5$km bekannt. Der Entfernungsvektor $\overrightarrow{BE}$ ist das $1.2$-fache des Entfernungsvektor $\overrightarrow{CB}$. Wenn Sie von dem Ort A den Ort C anpeilen, so liegt der Ort B ungefähr $25^\circ$ nördlich. Wenn Sie von dem Ort C den Ort B anpeilen, so liegt der Ort D ungefähr $50^\circ$ östlich. Vom Ort B betrachtet, bilden die Orte C und D einen rechten Winkel am Ort B. Der Ort B liegt auf gerader Linie zwischen den Orten C und E. Somit liegt der Ort E nördlich von B. Die Strecke zwischen A und E ist nicht passierbar. Sie starten an dem Ort C Ihre Cachertour. \\

Leider sind die beiden sehr schlecht im Navigieren und Entfernungen ausrechnen. Die beiden brauchen Ihre Hilfe!
  
\begin{enumerate}
\item Lösen Sie diese Aufgabe mit Hilfe einer aussagekräftigen Skizze der Orte und Caches. Bezeichnen Sie die Strecken und die Winkel Ihrer Skizze entsprechend! \textbf{(5 Punkte)}
\item Welche Strecke in $km$ legen Sie bei der Bewältigung der abwärts Terrainchallenge zurück? \textbf{(2 Punkte)}
\item Gehen Sie von einer zusätzlichen Suchzeit in Stunden für die Caches an den jeweiligen Orten zur reinen Reisezeit mit Ihrem Cachermobil aus. Die Suchzeit in Stunden für \textit{jeden einzelnen} Cache wird durch die Funktion  
  \begin{equation*}
    Suchzeit = 0.15 + 0.15 \cdot Schwierigkeit
  \end{equation*}  
  beschreiben.  Wie lange in Stunden benötigen Sie um die abwärts Terrainchallenge zu erfüllen? \textbf{(3 Punkte)}
%\item An der höchsten Schwierigkeit müssen Sie angeln. Ihre Angel ist ausgefahren l_angelm lang. Erreichen Sie einen Cache in der Höhe von h_angelm?  Berechnen Sie dazu Ihre maximale mögliche Angelhöhe! Welche Annahmen mussten Sie treffen um die Aufgabe zu lösen? \textbf{(2 Punkte)} 
\end{enumerate}

 
\clearpage
% -----------------------------------------------------------------------

\section{Aufgabe \hfill (10 Punkte)}

\textit{Geben Sie grundsätzlich Formeln und Rechenweg zur Lösung der Teilaufgaben mit an!} \\[1Ex]
 

 
%% --------------------------------------------------------------------
\begin{minipage}[t]{0.5\textwidth}
\includegraphics[width = 1.3cm]{/Users/kruppajo/work/GitHub/exam/avatare/Steffen.png}
\end{minipage}
\begin{minipage}[t]{0.5\textwidth}
\hfill
\href{https://youtu.be/4-dSaPMhK9s}{\includegraphics[width = 2cm]{img/youtube}}
\end{minipage}
%% --------------------------------------------------------------------

%% --------------------------------------------------------------------
{\tiny\textbf{Stichworte:} Brot aus Luft $\bullet$ Walöl $\bullet$ Haber-Bosch-Verfahren $\bullet$ 1. Weltkrieg $\bullet$ 40\% N im menschlichen Körper $\bullet$ Positivist}
%% --------------------------------------------------------------------




\paragraph{Die atmende Wand und Brot aus Luft}



Als Kellerkind\footnote{\href{https://www.youtube.com/watch?v=54H0HAJexVI}{Tocotronic - Electric Guitar} als passende Untermalung für diese Aufgabe.} vom Dorf will Steffen das Ausmaß der Radonbelastung in seinem Kellerzimmer bestimmen und lüften daher nicht. Passt schon. Spart dann auch Energie und lüften wird sowieso überschätzt. Während einer Messperiode von 7:00 Uhr bis 19:00 bestimmt er dreimal automatisch die Radonbelastung in seinem Kellerraum in $Bq/m^3$. Es ergibt sich folgende Abbildung\footnote{Die Quelle der Inspiration für die Aufgabe war folgender Artikel: \href{https://de.wikipedia.org/wiki/Atmende_Wand}{Atmende Wand}}. Leider helfen die Messwerte Steffen überhaupt nicht weiter. Sie müssen also helfen! 

\begin{knitrout}
\definecolor{shadecolor}{rgb}{0.969, 0.969, 0.969}\color{fgcolor}

{\centering \includegraphics[width=\maxwidth]{img/math-10-1} 

}


\end{knitrout}

\vspace{-0.75cm}

\begin{enumerate}
\item Wie lange dauert es in Stunden bis Sie eine kritische Belastung von 380$Bq/m^3$ in Ihrem ungelüfteten Kellerraum erreicht haben? \textbf{(2 Punkte)}
\end{enumerate}

Radon zerfällt mit einer Halbwertszeit von 1.8d zu Polonium. Polonium wiederum zerfällt mit einer Halbwertszeit von 143d zu Blei. Nur Radon und Polonium tragen zur radioaktiven Strahlenbelastung bei.

\begin{enumerate}
  \setcounter{enumi}{1}
\item Wie lange dauert es in Stunden bis Ihre kritische Radonbelastung von
  380$Bq/m^3$ auf unter 90$Bq/m^3$ gefallen ist?
  \textbf{(4 Punkte)}
\end{enumerate}

Folgende Tabelle enthält die Informationen zur Zusammensetzung der normalen Umgebungsluft.

\begin{center}
  \begin{tabular}{ c|c|c|c }
     & Vol-\% & M [g/mol] & ppm \\
    \hline
    Stickstoff & 77.1 & 28.4 &
                                                    \phantom{1000000000000}\strut\\
        \hline
    Sauerstoff & 19.5 & 15.8 &
                                                    \phantom{10000000}\strut\\
        \hline
    Kohlenstoffdioxid & 0.045 & 12.5 & \phantom{10000000}\strut\\     
     \hline
\end{tabular}
\end{center}

\begin{enumerate}
   \setcounter{enumi}{2}
\item Rechnen Sie die Volumenprozente (Vol-\%) der Umgebungsluft in die entsprechenden ppm-Werte um und ergänzen Sie die berechneten ppm-Werte in die Tabelle!  \textbf{(1 Punkt)}
\end{enumerate}

Während Steffen sein etwas pappiges Toastbrot mampfen kommt Steffen die Dokumentation über Brot aus Luft in den Sinn. Steffen denkt darüber ein wenig nach. Für die Umwandlung von Stickstoff $N_2$ mit Wasserstoff $H_2$ zu Ammoniak $NH_3$ gilt folgende Reaktionsgleichung\footnote{Die Quelle der Inspiration für die Aufgabe war folgender Artikel: \href{https://www.gdch.de/netzwerk-strukturen/fachstrukturen/ag-chemie-und-gesellschaft/projekte-und-veranstaltungen/cartoons/haber-bosch-verfahren-brot-aus-luft.html}{Haber-Bosch-Verfahren – Brot aus Luft}}:

\begin{equation*}
  N_2 + 3H_2 \rightarrow 2NH_3
\end{equation*}  

Ein Mol eines beliebigen Gases hat bei normalen Umweltbedingungen ein Volumen von 22.4 Liter. % Ein Mol enthält $6 \cdot 10^{23}$ Teilchen.

\begin{enumerate}
  \setcounter{enumi}{3}
\item Welche Masse an Ammoniak in Kilogramm $kg$ können Sie aus einem Kubikmeter $m^3$ Luft unter normalen Umweltbedingungen gewinnen?
  \textbf{(2 Punkte)}
\item Wie viel Ammoniak in $mol$ erhalten Sie aus einem Kubikmeter Luft? \textbf{(1 Punkt)}
\end{enumerate}

 
\clearpage
% -----------------------------------------------------------------------

\section{Aufgabe \hfill (10 Punkte)}

\textit{Geben Sie grundsätzlich Formeln und Rechenweg zur Lösung der Teilaufgaben mit an!} \\[1Ex]
 

 
%% --------------------------------------------------------------------
\begin{minipage}[t]{0.5\textwidth}
\includegraphics[width = 1.3cm]{/Users/kruppajo/work/GitHub/exam/avatare/Tina.png}
\end{minipage}
\begin{minipage}[t]{0.5\textwidth}
\hfill
\href{https://youtu.be/Bbu6n8MXxQk}{\includegraphics[width = 2cm]{img/youtube}}
\end{minipage}
%% --------------------------------------------------------------------




\paragraph{Armee der Finsternis}



Der Studentenjob von Tina war nach Ladenschluss bei IKEA die Regale einzuräumen. Dabei ist Tina in der Auslage der Sonderangebote das Necronomicon\footnote{Ein wirklich gefährliches Buch ist: \textit{Du bist genug: Vom Mut, glücklich zu sein} von Fumitake Koga und Ichiro Kishimi} in die Hände gefallen. Nun ist sie ein Magier der Zeichen geworden! Also eigentlich kann Tina nur Mathe und das dämliche Necronomicon hat sie in die Vergangenheit geschleudert... aber gut, was tut man nicht alles im Jahr 194 n. Chr. für den neuen Lehnsherren Henry dem Roten. Tina baut natürlich einen Schrottkugelturm um sich den
Horden der Finsternis mit genug Schrott erwehren zu können! Tina stehen zwei mächtige magische Formeln zur Unterstützung zu Verfügung. Leider wird das nicht reichen, deshalb müssen Sie hier auch noch durch Zeit und Raum helfen!

\begin{center}
  \begin{tabular}{cc}
    $E_{kin} = \cfrac{1}{2}\cdot m \cdot v^2$ & $E_{pot} = m \cdot g \cdot h$\\
  \end{tabular}
\end{center}

mit

\begin{itemize}[noitemsep]
\item $m$, gleich der Masse [kg] des Objekts
\item $h$, gleich der Höhe [m] des ruhenden Objekts
\item $v$, gleich der Geschwindigkeit [m/s] des Objekts
\item $g$, gleich der Erdbeschleunigung mit $9.81 \tfrac{m}{s^2}$ 
\end{itemize}

Als erstes müssen Sie die Höhe des zu bauenden Schrottkugelturmes bestimmen. Hierfür ist wichtig zu wissen, dass sich die Blei\textit{tropfen} mit einem Gewicht von $40mg$ zu gleichförmigen Blei\textit{kugeln} bei einer Geschwindigkeit von $14m/s$ bilden.

\begin{enumerate}
\item Wie hoch müssen Sie den Schrottkugelturm bauen lassen, damit sich runde Bleikugeln durch die Fallgeschwindigkeit von $14m/s$ bilden? \textbf{(3 Punkte)}
\end{enumerate}

Ihre erstellten Schrottkugeln sind leider zu gro{\ss} und somit sind zu wenige Schrottkugeln in einer Ladung. Damit können Sie die Armee der Finsternis nicht aufhalten. Die Sachlage müssen Sie einmal mathematisch untersuchen.

\begin{enumerate}
  \setcounter{enumi}{1}
\item Nennen Sie die beiden geometrischen Formen aus denen sich näherungsweise ein Tropfen zusammensetzt! Erstellen Sie eine beschriftete Skizze des Tropfens! \textbf{(2 Punkte)}
\item Sie messen eine Länge des Tropfens von 4.1mm. Die Löcher im Sieb erlauben ein Tropfendurchmesser von 1.7mm. Welchen Durchmesser in mm haben Ihre produzierten Blei\textit{kugeln}?  \textbf{(3 Punkte)}
\end{enumerate}

Sie haben jetzt die \ensuremath{1.2\times 10^{6}} Bleikugeln zusammen. Blei hat eine Dichte
von $10.32g/cm^3$.

\begin{enumerate}
  \setcounter{enumi}{3}
\item Wie schwer in Kilogramm $kg$ sind die \ensuremath{1.2\times 10^{6}} produzierten
  Bleikugeln, die Sie jetzt auf die Burgmauer transportieren müssen?
  \textbf{(1 Punkt)}
\end{enumerate}

Am Ende müssen Sie noch die Produktion von dem Bleischrott im Turm optimieren.

\begin{enumerate}
  \setcounter{enumi}{4}
\item Wie gro{\ss} in $cm^2$ ist Ihr quadratisches Sieb am oberen Ende des Turms,
  wenn Sie pro Fall ca. 900 Bleikugeln produzieren wollen und
  die Bleikugel im Fall 1.4cm Abstand haben müssen?  \textbf{(1
    Punkt)}
\end{enumerate}
 
\clearpage
% -----------------------------------------------------------------------

\section{Aufgabe \hfill (10 Punkte)}

\textit{Geben Sie grundsätzlich Formeln und Rechenweg zur Lösung der Teilaufgaben mit an!} \\[1Ex]
 

 
%% --------------------------------------------------------------------
\begin{minipage}[t]{0.5\textwidth}
\includegraphics[width = 1.3cm]{/Users/kruppajo/work/GitHub/exam/avatare/Tina.png}
\end{minipage}
\begin{minipage}[t]{0.5\textwidth}
\hfill
\href{https://youtu.be/Mr6eslls4J0}{\includegraphics[width = 2cm]{img/youtube}}
\end{minipage}
%% --------------------------------------------------------------------




\paragraph{Armee der Kaninchen}



Leider hat es bei Tina mit dem Krokodilreservat in Down Under nicht geklappt. War vielleicht auch nicht \textit{so} die beste Idee... aber dafür hat Tina eine neue Eingebung! Oder wie es Mike Tyson zugeschrieben wird: >>Ich wurde nie niedergeschlagen, ich war immer am Aufstehen!<<. Daher macht Tina jetzt einen Großhandel mit Kaninchenfleisch und damit dem teuersten Fleisch in Australien auf. Moment, hopsen hier nicht, seit Thomas Austin im Jahr 1875 ungefähr 26 Kaninchen entlassen hat, Millionen von Kaninchen rum? Wieso ist das Kaninchenfleisch dann so exklusiv? Tina wird stutzig und frag Sie, dem mal mathematisch nachzugehen!\footnote{Die Quelle der Inspiration für die Aufgabe war der folgendes YouTube Video: \href{https://youtu.be/38fuOr3tdgc?si=Li7NL_FoByML8JtT}{ Incredible Stories -- Why don't they eat wild rabbits in Australia? They have    millions of them! The reason is surprising...}} \\

Forscherinnen fand folgende Sättigungsfunktion für das jährliche Wachstum der gesamten Kaninchenpopulation im westlichen Australien.
\begin{equation*}
  f(t) = \ensuremath{1.1\times 10^{10}} - \ensuremath{10^{9}} \cdot 1.9^{-0.3 \cdot t + 2.7}
\end{equation*}

\begin{enumerate}
\item Skizzieren Sie die Sättigungsfunktion \textit{annäherungsweise} in einer Abbildung! \textbf{(1 Punkt)}
\item Wie viele Kaninchen können nach der Sättigungsfunktion maximal im westlichen Australien leben? Ergänzen Sie den Wert in Ihrer Abbildung! \textbf{(2 Punkte)}
\item Wie viele Millionen Kaninchen leben nach der Sättigungsfunktion nach 9 Jahren auf dem australischen Kontinent? \textbf{(1 Punkt)}
\end{enumerate}

Um den Kaninchen Einhalt zu gebieten wurde das Myxoma Virus und das Rabbit Haemorrhagic Disease Virus (RHDV) in 20 Kaninchen ausgebracht. Da die Kaninchen keine Maßnahmen gegen die Ausbreitung vornehmen können, verläuft die Ausbreitung mit einem wöchentlichen Wachstumsfakor von 1.4 nach folgender Formel.

\begin{equation*}
  N(t) = N(0) \cdot a^t
\end{equation*}

\begin{enumerate}
  \setcounter{enumi}{2}
\item Wie viele Wochen benötigen die Viren um theoretisch die gesamte Kaninchenpopulation nach 7 Jahren Wachstum zu durchseuchen? \textbf{(1 Punkt)}
\end{enumerate}

Das Myxoma Virus und das RHDV töten 99.7\% der Kaninchenpopulation innerhalb weniger Wochen.

\begin{enumerate}
  \setcounter{enumi}{3}  
\item Wie lange in Jahren dauert es bis eine Kaninchenpopulation nach einer Viruspandemie wieder auf 50\% der gesättigten Kaninchenpopulation angewachsen ist?  \textbf{(2 Punkte)}
\end{enumerate}

Thomas Austin entließ die Kaninchen im äußersten Süden von Australien. Australien hat eine West-Ost-Ausdehnung von 4300km und eine Nord-Süd-Ausdehnung von knapp 3500km. Die Kaninchen breiten sich radial mit einer Geschwindigkeit von 7.8km pro Jahr aus.

\begin{enumerate}
  \setcounter{enumi}{4}
\item Wie lange dauert es in Jahren bis die Kaninchen jeden Ort in Australien erreicht haben? \textit{Lösen Sie die Aufgabe unter der Verwendung einer schematischen Skizze!} \textbf{(2 Punkte)}
\end{enumerate}

Eine jährliche Impfung gegen das Myxoma Virus und das Rabbit Haemorrhagic Disease Virus (RHDV) kosten 12\$ pro Tier und der durchführende Arzt verlangt ca. 45\$ pro Tier.

\begin{enumerate}
  \setcounter{enumi}{5}
\item In Ihrem Stall leben 1000 Mastkaninchen. Mit welchen jährlichen Zusatzkosten für die Impfungen der Kaninchen müssen Sie daher kalkulieren? \textbf{(1 Punkt)}
\end{enumerate}
 
\clearpage
% -----------------------------------------------------------------------

\section{Aufgabe \hfill (12 Punkte)}

\textit{Geben Sie grundsätzlich Formeln und Rechenweg zur Lösung der Teilaufgaben mit an!} \\[1Ex]
 

 
%% --------------------------------------------------------------------
\begin{minipage}[t]{0.5\textwidth}
\includegraphics[width = 1.3cm]{/Users/kruppajo/work/GitHub/exam/avatare/Jessica.png}
\end{minipage}
\begin{minipage}[t]{0.5\textwidth}
\hfill
\href{https://youtu.be/fiWGgCX-cE4}{\includegraphics[width = 2cm]{img/youtube}}
\end{minipage}
%% --------------------------------------------------------------------






\paragraph{Lüneburger Heide. Unendliche Weiten.}



Wir schreiben das Jahr 2024. Dies sind die Abenteuer des Esels Fridolin und Jessica. Grünes Gras unter Jessicas Füßen und ein strammer Wind im Gesicht, egal wohin sie schaut. Ein schmatzendes Geräusch ertönt unter Jessica. Jessica sinniert, sollte sie ihre weiten Graslandschaften jetzt schon düngen? Dafür benötigt Jessica die \textit{Grünlandtemperatur}! Die Grünlandtemperatur (GLT) ist die Summe aller positiven Tagesmitteltemperaturen seit Jahresbeginn. Ab einer GLT von 200$^\circ$ kann mit der Stickstoffdüngung begonnen werden. Jessica sieht nicht ein, Geld für einen Agrarmetrologen zu bezahlen, wenn auch Sie mitrechnen können. Also rechnen Sie beide mit folgenden Informationen zu Monatsmultiplikatoren des GLT-Wertes: Januar mit $0.6\times$, Februar mit $0.75\times$ und März mit
$1.1\times$. Sie haben noch im letzten Jahr folgende Temperaturen gemessen.

\begin{center}
\begin{tabular}{cc}
  \toprule
  Datum & C$^\circ$ \\
  \midrule
  01. Jan 2023 & 0.1\\
  01. Feb 2023 & 1.2\\
  01. Mrz 2023 & 2.7\\
  01. Apr 2023 & 5.8\\
  \bottomrule
\end{tabular}
\end{center}

\begin{enumerate}
\item Erstellen Sie eine Skizze aus den Informationen aus der Temperaturtabelle!  \textbf{(1 Punkt)}
\item Stellen Sie die linearen Funktionen $f_1(t)$, $f_2(t)$ und $f_3(t)$ aus der obigen Temperaturtabelle auf!  \textbf{(1 Punkt)}
\item Bestimmen Sie die Stammfunktionen $F_1(t)$, $F_2(t)$ und $F_3(t)$ für Ihre linearen Funktionen aus der obigen Temperaturtabelle!  \textbf{(1 Punkt)}
\item Osterglocken beginnen ab einer GLT von 210$^\circ C$ zu blühen. An welchem Tag im 1. Quartal des Jahres 2023 war dies der Fall? \textit{Ignorieren Sie ein eventuelles Schaltjahr in Ihrer Berechnung.} \textbf{(4 Punkte)}
\end{enumerate}

Auf dem Weg zu Jessicas Pink Lady Plantage werden Sie beide auf dem Trecker von einer Gruppe elektrifizierter Renter abgedrängt. Der Trecker muss wieder aus dem Graben! Fridolin und die elektrifizierten Rentner ziehen an zwei, separaten Seilen. Dabei zieht Fridolin mit $230N$. Die elektrifizierter Renter  bringen eine Kraft von $190N$ auf.\\

\textit{Lösen Sie diese Aufgabe mit Hilfe einer aussagekräftigen Skizze der Kraftvektoren. Bezeichnen Sie die Kraftvektoren und die Winkel Ihrer Skizze entsprechend!}

\begin{enumerate}
  \setcounter{enumi}{4}  
\item Im ersten Versuch legen Sie das Seil für Fridolin lotrecht über einen Ast oberhalb des Treckers. Die Rentner ziehen in einer geraden Linie über die Böschung hinweg am anderen Seil. Welche Kraft wird aufgebracht?  \textbf{(2 Punkte)}
\item Im zweiten Versuch ziehen Fridolin und die Rentner mit einem $40^\circ$ Winkel mit ihrem Seil an dem Trecker. Welche Kraft wird aufgebracht? \textbf{(2 Punkte)}
\item Mit welcher Beschleunigung ziehen Sie den $1.2t$ schweren Trecker \textit{jeweils} aus dem Graben, wenn $F = m \cdot a$ gilt? \textbf{(1 Punkt)}
\end{enumerate}

 
\clearpage
% -----------------------------------------------------------------------

\section{Aufgabe \hfill (10 Punkte)}

\textit{Geben Sie grundsätzlich Formeln und Rechenweg zur Lösung der Teilaufgaben mit an!} \\[1Ex]
 

 
%% --------------------------------------------------------------------
\begin{minipage}[t]{0.5\textwidth}
\includegraphics[width = 1.3cm]{/Users/kruppajo/work/GitHub/exam/avatare/Jessica.png}\hspace{-4mm}\includegraphics[width = 1.3cm]{/Users/kruppajo/work/GitHub/exam/avatare/Steffen.png}\hspace{-4mm}\includegraphics[width = 1.3cm]{/Users/kruppajo/work/GitHub/exam/avatare/Tina.png}\hspace{-4mm}\includegraphics[width = 1.3cm]{/Users/kruppajo/work/GitHub/exam/avatare/Yuki.png}
\end{minipage}
\begin{minipage}[t]{0.5\textwidth}
\hfill
\href{https://youtu.be/RuzMjwvwT-4}{\includegraphics[width = 2cm]{img/youtube}}
\end{minipage}
%% --------------------------------------------------------------------




\paragraph{In der Kartonagenfabrik}



Yuki, Steffen, Jessica und Tina sitzen im Bus. Wenn man sich zu spät anmeldet, dann ist die Exkursion nicht so toll. Yuki hatte den Anderen in der Lerngruppe zu spät Bescheid gesagt. 'Was denn, bin ich eure Nanny oder was?!', entfährt es Yuki nachdem die vorwurfsvollen Blicke schon eine Weile auf ihm lasten. Also geht es eben mit Rektor Skinner und Mrs. Krabappel in die Kartonagenfabrik. Wie schon im vorherigen Semester... In der Kartonagenfabrik angekommen erfahren die Vier, dass die Kartons zum Versand von Nägeln nicht hier zusammengebautwerden sondern das sich die Endfertigung in Flint, Michigan befindet. Unter anderem wird dort der berühmte \textit{Doppelt gewellte, 5-mal-gefaltete, 0.5mm, 40-cm-Karton} durch
Falzung hergestellt. Beim letzten Mal war Rektor Skinner die Stimmung zu schlecht und deshalb geht es erst nach Hause, wenn ein paar Aufgaben gelöst sind. Martin gefällt das. An dem Vorrat an Zigaretten von Mrs. Krabappel meinen alle wenig Zuversicht zu erkennen.\\

Jetzt heißt es Kartons optimieren, wenn Sie auch nochmal nach Hause wollen. Warum jetzt \textit{Sie} mit dabei sind, lassen wir mal weg. Der nun zu optimierende, flache Karton hat eine Länge von 40cm und eine Breite von 22cm. Die Kartonagenmaschine in Flint soll dann einen quadratischen Eckenausschnitt der Länge $x$ falzen.

\begin{enumerate}
\item Erstellen Sie eine Skizze des Karton\textit{blatt}rohlings! Beschriften Sie die Skizze mit den entsprechenden Längenangaben \textbf{(1 Punkt)}
\item Berechnen Sie die Falztiefe $x$ für ein maximales Volumen des flachen Kartons! \textbf{(3 Punkte)}
\item Welches Volumen in Liter ergibt sich mit der von Ihnen berechneten Falztiefe $x$?  \textbf{(1 Punkt)}
\item Sie wollen noch einen bündig mit dem Boden abschließenden Deckel für den Karton stanzen lassen. Wie groß ist die Fläche des Kartondeckel\textit{blatt}rohlings in $cm^2$? \textbf{(2 Punkte)}
\end{enumerate}

Rektor Skinner möchte sich gerne wieder in seinem Vorgarten aufhalten und nicht die ganze Zeit von Bart mit Erdnüssen beworfen werden. Deshalb möchte er einen geräumigen Teil seines Vorgartens einzäunen. Ein Teil der Umzäunung bildet seine Vorderhauswand. Wegen Lieferschwierigkeiten stehen Rektor Skinner nur 110m Zaun zu Verfügung. Auch hier sollen Sie mal helfen, sonst fährt der Bus Sie nicht nach Hause. Sie wollen nun die maximale Fläche des abgeschirmten Vorgartens in Abhängigkeit der Seitenlängen bei der Verwendung von 110m Zaun bestimmen!

\begin{enumerate}
  \setcounter{enumi}{4}  
\item  Welche Seitenlängen für den Zaun ergeben sich für die maximale Fläche des abgeschirmten Vorgartens? \textbf{(2 Punkte)}
\item Berechnen Sie die Fläche des abgeschirmten Vorgartens! \textbf{(1 Punkt)}
\end{enumerate}

 
\clearpage
% -----------------------------------------------------------------------

\section{Aufgabe \hfill (10 Punkte)}

\textit{Geben Sie grundsätzlich Formeln und Rechenweg zur Lösung der Teilaufgaben mit an!} \\[1Ex]
 

 
%% --------------------------------------------------------------------
\begin{minipage}[t]{0.5\textwidth}
\includegraphics[width = 1.3cm]{/Users/kruppajo/work/GitHub/exam/avatare/Jonas.png}\hspace{-4mm}\includegraphics[width = 1.3cm]{/Users/kruppajo/work/GitHub/exam/avatare/Mark.png}\hspace{-4mm}\includegraphics[width = 1.3cm]{/Users/kruppajo/work/GitHub/exam/avatare/Nilufar.png}\hspace{-4mm}\includegraphics[width = 1.3cm]{/Users/kruppajo/work/GitHub/exam/avatare/Paula.png}
\end{minipage}
\begin{minipage}[t]{0.5\textwidth}
\hfill
\href{https://youtu.be/OhyuH6hzEhY}{\includegraphics[width = 2cm]{img/youtube}}
\end{minipage}
%% --------------------------------------------------------------------







\paragraph{Ein Pfund Insekten, bitte!} 'Das wird wohl häufiger gehört werden, wenn wir die Menschheit mit Proteinen ausreichend ernähren wollen\footnote{Die Quelle der Inspiration für die Aufgabe war der folgende Artikel aus dem Spiegel: \href{https://www.spiegel.de/ausland/ueberbevoelkerung-acht-milliarden-sind-wir-bald-zu-viele-menschen-auf-der-erde-a-3f20c7bc-3d60-4440-9f52-eb338db207f5}{Acht Milliarden - sind wir bald zu viele Menschen auf der Erde?}}.', merkt Paula an. Die Lerngruppe um Jonas, Nilufar und Mark sind bei Paula um mal was außergewöhnliches zu essen. Um den Sinn der Nahrungsumstellung zu verdeutlichen, vergleicht Paula einmal Deutschland mit Nigeria. Nigeria hat eine der am schnellsten wachsenden Bevölkerungen der Welt und wird vermutlich im Jahr 2100 zu den Top 5 der bevölkerungsreichsten Länder zählen. Im Jahr 2020 leben ca. \ensuremath{8\times 10^{7}} Menschen in Deutschland und ca. \ensuremath{1.84\times 10^{8}} Menschen in Nigeria. Mit den Informationen wollen Sie und Paula mit der Überzeugungsarbeit anfangen und dann eine Prognose für den Fleischkonsum im Jahr 2050 zu treffen. \\ 

Im Folgenden ist Abbildung des Fleischkonsums im Jahr 2020 in Deutschland und Nigeria in [kg] einmal dargestellt. 

\begin{knitrout}
\definecolor{shadecolor}{rgb}{0.969, 0.969, 0.969}\color{fgcolor}

{\centering \includegraphics[width=\maxwidth]{img/math-14-a-1} 

}


\end{knitrout}
  
\begin{enumerate}
\item Stellen Sie den Fleischkonsum in Deutschland und Nigeria im Jahr 2020 \textit{pro Kopf} in einer aussagekräftigen Tabelle dar! \textbf{(2 Punkte)}
\item Ergänzen Sie in der Tabelle eine Spalte in der Sie für den Fleischkonsum in Nigeria auf Deutschland normieren, daher ins Verhältnis Nigeria/Deutschland, setzen! \textbf{(1 Punkt)}
\end{enumerate}

In der nächsten Abbildung finden Sie die CO$_2$-Emission in [kg] nach Lebensmittel, die durch die Produktion entsteht, abgebildet.

\begin{knitrout}
\definecolor{shadecolor}{rgb}{0.969, 0.969, 0.969}\color{fgcolor}

{\centering \includegraphics[width=\maxwidth]{img/math-14-b-1} 

}


\end{knitrout}

\begin{enumerate}
  \setcounter{enumi}{2}  
\item Stellen Sie in einer Tabelle die Treibhausgasemissionen an $CO_2$ pro Kopf, die durch den Fleischkonsum in Deutschland und Nigeria im Jahr 2020 entstehen, dar! Ergänzen Sie auch hier das Verhältnis Nigeria zu Deutschland! \textbf{(2 Punkte)}
\end{enumerate}

\newpage

In der folgenden Abbildung sehen Sie die Bevölkerungsentwicklung [Millionen] in Nigeria von 1950 bis ins Jahr 2030 fortgeführt.

\begin{knitrout}
\definecolor{shadecolor}{rgb}{0.969, 0.969, 0.969}\color{fgcolor}

{\centering \includegraphics[width=\maxwidth]{img/math-14-c-1} 

}


\end{knitrout}

\begin{enumerate}
  \setcounter{enumi}{3}  
\item Schätzen Sie graphisch die zu erwartende Bevölkerung [Millionen] in Nigeria im Jahr 2050, die sich anhand der Informationen aus der Abbildung ergibt!
\begin{enumerate}
\item Ohne Berücksichtigung der Covid-19-Pandemie! \textbf{(1 Punkt)}
\item Unter Berücksichtigung der Covid-19-Pandemie! \textbf{(1 Punkt)}
\end{enumerate}
\item Berechnen Sie den geschätzten Fleischkonsum von Nigeria im Jahr 2050 unter der Annahme 80\%-iger Angleichung der Lebensbedingungen zu Deutschland im Jahr 2020! \textbf{(1 Punkt)}
\item Berechnen Sie die prozentuale Steigerung der Treibhausgasemissionen an $CO_2$ in Nigeria im Jahr 2050 im Vergleich zum Jahr 2020, der sich durch den angeglichenen Fleischkonsum ergibt! \textbf{(1 Punkt)}
\item Berechnen Sie die prozentuale Steigerung der Treibhausgasemissionen an $CO_2$ in Nigeria, wenn die gesamte Proteinaufnahme durch Insekten ersetzt würde! \textbf{(1 Punkt)}
\end{enumerate}







 
\clearpage
% -----------------------------------------------------------------------

\section{Aufgabe \hfill (12 Punkte)}

\textit{Geben Sie grundsätzlich Formeln und Rechenweg zur Lösung der Teilaufgaben mit an!} \\[1Ex]
 

 
%% --------------------------------------------------------------------
\begin{minipage}[t]{0.5\textwidth}
\includegraphics[width = 1.3cm]{/Users/kruppajo/work/GitHub/exam/avatare/Mark.png}\hspace{-4mm}\includegraphics[width = 1.3cm]{/Users/kruppajo/work/GitHub/exam/avatare/Nilufar.png}
\end{minipage}
\begin{minipage}[t]{0.5\textwidth}
\hfill
\href{https://youtu.be/flRBo1FWQy0}{\includegraphics[width = 2cm]{img/youtube}}
\end{minipage}
%% --------------------------------------------------------------------

%% --------------------------------------------------------------------
{\tiny\textbf{Stichworte:} Immunsystem -- Muskel vs. Interpol $\bullet$ Inzidenz $\bullet$ Prävalenz}
%% --------------------------------------------------------------------




\paragraph{Tödliche Seuche AIDS -- Die rätselhafte Krankheit}




Irritiert legt Nilufar die historische Ausgabe des Spiegels aus den 80zigern beiseite. Nilufar und Mark sind bei ihrem HNO-Arzt und wollen einen AIDS-Test machen lassen. Woanders leider keinen Termin gekriegt... Immerhin denken die beiden über Nachwuchs nach und da geht es eben nur durch ungeschützten Sex. Was wissen Nilufar und Mark nun aber über AIDS und dem diagnostischen AIDS-Test, den die beiden nun machen werden? Leider zu wenig. Da brauchen dann Nilufar und Mark mal wieder Ihre Hilfe bei der Interpretation eines diagnostischen Tests!\\

Die Prävalenz von AIDS bei einem Menschen in Europa wird mit 0.8\% angenommen. In 96\% der Fälle ist ein HIV-Test positiv, wenn der Patient erkrankt ist. In 1\% der Fälle ist ein HIV-Test positiv, wenn der Patient \textit{nicht} erkrankt ist und somit gesund ist. Sie stutzen. Wie wahrscheinlich ist es denn eigentlich an AIDS erkrankt zu sein $(K^+)$, wenn Sie einen positiven AIDS-Test vorliegen haben $(T^+)$? Gehen Sie für die folgenden Berechnungen von $n = \ensuremath{3\times 10^{4}}$ Patienten mit einem diagnostischen Test für AIDS aus. Sie nehmen sich also einen Kuli und fangen an auf der historischen Ausgabe des Spiegels zu rechnen\footnote{Die Quelle der Inspiration für die Aufgabe war der folgende wissenschaftlicher Artikel: \textit{Binder et al. (2022) Von Baumdiagrammen über Doppelbäume zu Häufigkeitsnetzen -- kognitive überlastung oder didaktische Unterstützung? Journal für Mathematik-Didaktik, 1-33}}.


\begin{enumerate}
\item Welche Wahrscheinlichkeit $Pr$ wollen Sie berechnen? \textbf{(1 Punkt)}
\item Zeichnen Sie einen Häufigkeitsdoppelbaum zur Bestimmung der gesuchten Wahrscheinlichkeit $Pr$! \textbf{(2 Punkte)} \item Beschriften Sie den Häufigkeitsdoppelbaum, mit denen Ihnen bekannten Informationen zu der AIDS Erkrankung und dem AIDS-Test! \textbf{(1 Punkt)}
\item Füllen Sie den Häufigkeitsdoppelbaum mit den sich ergebenden, absoluten Patientenzahlen $n$ aus! \textbf{(2 Punkte)}
\item Berechnen Sie die gesuchte Wahrscheinlichkeit $Pr$! \textbf{(1 Punkt)}
\end{enumerate}

Bei dem folgenden Arztgespräch erfahren Nilufar und Mark, dass beim diagnostischen Testen \textit{True Positives (TP)}, \textit{True Negatives (TN)}, \textit{False Positives (FP)} und \textit{False Negatives (FN)} auftreten. Das verstehen beiden so noch nicht und deshalb stellen Sie für Nilufar und Mark den Zusammenhang in einer 2x2 Kreuztabelle dar.

\begin{enumerate}
  \setcounter{enumi}{5}
\item Tragen Sie \textit{TP}, \textit{TN}, \textit{FP} und \textit{FN} in eine 2x2 Kreuztablle ein. Beschriften Sie die Tabelle entsprechend! \textbf{(1 Punkt)}
\item Berechnen Sie die Sensitivität und Spezifität des diagnostischen Tests für AIDS! Füllen Sie dafür die 2x2 Kreuztabelle mit den Informationen aus dem Häufigkeitsdoppelbaum aus! \textbf{(2 Punkte)}
\item Was beschreibt die Sensitivität und die Spezifität im Bezug auf die Gesunden und Kranken? Stellen Sie beide diagnostische Maßzahlen als Wahrscheinlichkeiten $Pr$ dar! \textbf{(2 Punkte)} 
\end{enumerate}



 
\clearpage
% -----------------------------------------------------------------------

\section{Aufgabe \hfill (12 Punkte)}

\textit{Geben Sie grundsätzlich Formeln und Rechenweg zur Lösung der Teilaufgaben mit an!} \\[1Ex]
 

 
%% --------------------------------------------------------------------
\begin{minipage}[t]{0.5\textwidth}
\includegraphics[width = 1.3cm]{/Users/kruppajo/work/GitHub/exam/avatare/Alex.png}\hspace{-4mm}\includegraphics[width = 1.3cm]{/Users/kruppajo/work/GitHub/exam/avatare/Jonas.png}\hspace{-4mm}\includegraphics[width = 1.3cm]{/Users/kruppajo/work/GitHub/exam/avatare/Nilufar.png}\hspace{-4mm}\includegraphics[width = 1.3cm]{/Users/kruppajo/work/GitHub/exam/avatare/Paula.png}
\end{minipage}
\begin{minipage}[t]{0.5\textwidth}
\hfill
\href{https://youtu.be/SZqp_vy3rgI}{\includegraphics[width = 2cm]{img/youtube}}
\end{minipage}
%% --------------------------------------------------------------------



\paragraph{Network-Marketing oder Schneeballschlacht!}



Nilufar, Alex und Paula sitzen bei Jonas und hören sich etwas über Network-Marketing an. Jonas ist jetzt im Network-Marketing tätig. 'Jetzt reicht es. Wir sind eine Lerngruppe und du versuchst uns hier abzuziehen!', poltert Nilufar und fährt fort, 'Ich erklär dir mal, wie falsch du liegst!'. Eine Möglichkeit, leicht Geld zu verdienen, ist es anderen Menschen für Geld zu versprechen, wie man leicht reich werden kann. Am besten natürlich ohne viel Aufwand und ortsunabhängig. Schnell ein YouTube-Werbevideo gedreht und auf geht es mit unserem Network-Marketing. Aber Moment, wie funktioniert Network-Marketing eigentlich und was hat das alles mit einer Schneeballschlacht zu tun? Nilufar und Sie wollen hier einmal in die Untiefen des >>passiven Einkommens<< abtauchen und die Lerngruppe vor Schlimmeren bewahren\footnote{Die Quellen der Inspiration für die Aufgabe waren folgendes YouTube Video: \href{https://youtu.be/UOKkZF_qK9M?si=uf4foJVFKfeQMwSw}{Simplicissimus -- Die meistgesuchte Betrügerin der Welt} und der Artikel: \href{https://www.deutschlandfunkkultur.de/netzwerk-marketing-die-illusion-schnell-reich-zu-werden-100.html}{Deutschlandfunk Kultur -- Die Illusion, schnell reich zu werden}}!\\

Das Jahr 2022 war das erfolgreichste Jahr in der Geschichte von Healthy Herbs Manufacture International (HeHeMan). Das Unternehmen steigerte den Umsatz um rund 15 Prozent von 280 Millionen Euro im Jahr 2021. Doch wie viel kommt bei den Partnern an? Laut HeHeMan habe das Unternehmen \ensuremath{3.6\times 10^{5}} aktive Partner.

\begin{enumerate}
\item Berechnen Sie zuerst den Umsatz der Firma HeHeMan im Jahr 2022! \textbf{(1 Punkt)}
\item Wie viel von dem Umsatz im Jahr 2022 wird im Durchschnitt von jedem aktiven Partner erwirtschaftet? \textbf{(1 Punkt)}
\item Welche \textit{monatlicher} Umsatz ergibt sich dadurch im Durchschnitt für jeden aktiven Partner bei einer direkten Provision von 25\%? \textbf{(1 Punkt)}
\end{enumerate}

Das von Jonas zu vermarkende Produkt, hinter dem Jonas voll steht, kostet 150EUR pro Einheit im Direktverkauf. Die direkte Provision für die erste Stufe beträgt 20\%. Für die zweite, dritte und vierte Stufe betragen die indirekten Provisionen jeweils 3\%, 2\% und 1.5\%. Jeder von Jonas angeworbener >>Partner<< wirbt wiederum vier Partner für sich selbst an. Pro Monat werden im Schnitt vier Einheiten vom Produkt verkauft. Jonas will nun 3200EUR im Monat \textit{passiv} -- also durch indirekte Provisionen -- erwirtschaften. Kann das klappen? Sie sind zusammen mit Nilufar skeptisch.

\begin{enumerate}
  \setcounter{enumi}{3}
\item Ergänzen Sie die folgende Tabelle mit den obigen Informationen! \textbf{(2 Punkte)}
\end{enumerate}

\begin{center}
\begin{tabular}{c|c|c|c}
  \toprule
  \textbf{Stufe} & \textbf{Anzahl Partner}  & \textbf{Umsatz/Stufe} & \textbf{Provision}\\
  \midrule
  1 & Sie selber  &  & \\ \midrule
  2 &   &  &  \\ \midrule
  3 &   &  &  \\ \midrule
  4 &   &  &  \\
  \bottomrule
\end{tabular}
\end{center}

\begin{enumerate}
  \setcounter{enumi}{4}
\item Wie viele Partner müssen Sie auf der 2 Stufe anwerben um Ihr passives Einkommen durch indirekte Provision zu erreichen? Wie viele Menschen arbeiten am Ende indirekt für Sie? Stellen Sie den Zusammenhang graphisch dar!  \textbf{(3 Punkte)}
\end{enumerate}

Jonas musste zum Einstieg bei HeHeMan Einheiten des Produkts für 6000EUR kaufen. Diese Einheiten kann Jonas nur direkt verkaufen. Das ganze Wohnzimmer ist voll davon. Leider musste Jonas den Kauf über einen Kredit über 6.2\% p.a. über 48 Monate finanzieren. Sie schütteln den Kopf und klären Jonas über Zinsen auf.

\begin{enumerate}
  \setcounter{enumi}{5}
\item Berechnen Sie die Gesamtsumme, die Sie als Kredit abbezahlen müssen! \textbf{(2 Punkte)}
\item Wie viele Einheiten müssen Sie pro Monat verkaufen um die anfallenden Zinsen durch die direkte Provision zu erwirtschaften? \textbf{(1 Punkt)}
\item Wie lange in Monaten benötigen Sie um den Kredit durch die direkte Provision abzubezahlen? \textbf{(1 Punkt)}
\end{enumerate} 
\clearpage
% -----------------------------------------------------------------------

\section{Aufgabe \hfill (10 Punkte)}

\textit{Geben Sie grundsätzlich Formeln und Rechenweg zur Lösung der Teilaufgaben mit an!} \\[1Ex]
 

 
%% --------------------------------------------------------------------
\begin{minipage}[t]{0.5\textwidth}
\includegraphics[width = 1.3cm]{/Users/kruppajo/work/GitHub/exam/avatare/Jessica.png}\hspace{-4mm}\includegraphics[width = 1.3cm]{/Users/kruppajo/work/GitHub/exam/avatare/Jonas.png}\hspace{-4mm}\includegraphics[width = 1.3cm]{/Users/kruppajo/work/GitHub/exam/avatare/Paula.png}\hspace{-4mm}\includegraphics[width = 1.3cm]{/Users/kruppajo/work/GitHub/exam/avatare/Steffen.png}
\end{minipage}
\begin{minipage}[t]{0.5\textwidth}
\hfill
\href{https://youtu.be/8Pb2sKUIMyk}{\includegraphics[width = 2cm]{img/youtube}}
\end{minipage}
%% --------------------------------------------------------------------





\paragraph{Höhlen \& Drachen}



Jessica, Jonas und Steffen sitzen bei Paula nachdem sich alle begeistert in der Serie \textit{Stranger Thinks} verloren haben. Alle drei wollen jetzt einmal bei Paula \textit{Höhlen \& Drachen} ausprobieren. Um Geld zu sparen, das Zeug kostet echt, wurde etwas an den Regeln gebastelt. Schnell stellen die Drei fest, dass hier ganz schön viele unterschiedliche Würfel durch die Gegend fliegen. Daher müssen Sie sich jetzt einiges an Fragen klären damit Paula nicht so alleine ist. \\%[-1ex]

In dem Spiel hat Jessica nun auf einmal 7 achtseitige Würfel (7d8) zum würfeln in der Hand. Wenn Jessica eine 8 würfelt, hat Jessica einen Erfolg.

\begin{enumerate}
\item Berechnen Sie die Wahrscheinlichkeit \textit{genau} 5 Erfolge zu erzielen!  \textbf{(2 Punkte)}
\item Berechnen Sie die Wahrscheinlichkeit keinen Erfolg zu erzielen! \textbf{(1 Punkt)}
\end{enumerate}

Jonas betrachtet nun aufmerksam die ausufernden Ausrüstungstabellen. Jonas wird aber geholfen und muss sich jetzt nur zwischen der Axt oder dem Schwert entscheiden.

\begin{enumerate}
  \setcounter{enumi}{2}
\item Würden Sie die Axt mit zwei vierseitigen Würfeln (2d4) als Schaden oder das Schwert mit einem vierseitigen Würfel plus 5 (1d4+5) als Schaden bevorzugen? Begründen Sie Ihre Antwort mathematisch! \textbf{(1 Punkt)}
\end{enumerate}

Jetzt wird es immer wilder, da Jonas und Steffen sich jetzt überlegen müssen, wie wahrscheinlich es ist, dass der Rettungswurf gegen den zaubernden Hexer funktioniert. Jonas und Steffen haben folgende Wahrscheinlichkeiten gegeben. Die Wahrscheinlichkeit für das Ereignis $A$, der Rettungswurf ist erfolgreich, ist $Pr(A) = 0.7$, die Wahrscheinlichkeit für das Ereignis $B$, der Zauberwurf des Hexers ist erfolgreich, ist $Pr(B) = 0.7$. Sie haben aber mitgezählt und festgestellt, dass in $50$ von 100 Fällen der Rettungswurf bei einem erfolgeichen Zauber funktioniert hat.  

\begin{enumerate}
  \setcounter{enumi}{3}
\item Erstellen Sie eine 2x2 Kreuztabelle mit den Ereignissen $A$ und $B$ sowie den Gegenereignissen $\bar{A}$ und $\bar{B}$ mit einen $\Omega = 100$. Beachten Sie hierbei die entsprechenden Wahrscheinlichkeiten für die Ereignisse $A$ und $B$! \textbf{(2 Punkte)}
\item Bestimmen Sie $Pr(A \cap B)$! \textbf{(1 Punkt)}
\item Erstellen Sie ein Baumdiagramm mit den passenden Informationen aus der 2x2 Kreuztabelle! \textbf{(2 Punkte)}
\item Bestimmen Sie Wahrscheinlichkeit $Pr(A|B)$, dass Ihr Rettungswurf gelingt, wenn der Hexer erfolgreich gezaubert hat! \textbf{(1 Punkt)}
\end{enumerate}

  
\clearpage
% -----------------------------------------------------------------------

\section{Aufgabe \hfill (12 Punkte)}

\textit{Geben Sie grundsätzlich Formeln und Rechenweg zur Lösung der Teilaufgaben mit an!} \\[1Ex]
 

 
%% --------------------------------------------------------------------
\begin{minipage}[t]{0.5\textwidth}
\includegraphics[width = 1.3cm]{/Users/kruppajo/work/GitHub/exam/avatare/Steffen.png}\hspace{-4mm}\includegraphics[width = 1.3cm]{/Users/kruppajo/work/GitHub/exam/avatare/Tina.png}\hspace{-4mm}\includegraphics[width = 1.3cm]{/Users/kruppajo/work/GitHub/exam/avatare/Yuki.png}
\end{minipage}
\begin{minipage}[t]{0.5\textwidth}
\hfill
\href{https://youtu.be/8Pb2sKUIMyk}{\includegraphics[width = 2cm]{img/youtube}}
\end{minipage}
%% --------------------------------------------------------------------




\paragraph{Retrocheck im TV}




Das war zu viel für Tina gestern. Die Lerngruppe mit Yuki und Steffen ging viel zu lang. Während sie wegdämmert, kommen in ihr seltsame Bilder hoch. 'Und hier ist sie wieder, die Show der fantastischen Preise. Seien Sie mit dabei, wenn es wieder heißt: Der Preis ist heiß!', ertönt es und Tina fragt sich, ob sie nicht doch lieber bezahlter Gast bei Barbara Salesch hätten sein sollten. Aber Tina braucht das Geld und jetzt heißt es Spielschows farmen! Erstmal eine Kaffemaschine von Mitropa gewinnen. Ein Kandidat gewinnt die Kaffeemaschine von Mitropa, wenn nicht alle Kandidaten überbieten (eng. \textit{outbid}). Mit Ihnen bilden Yuki und Steffen das Team der drei Kandidaten. Tina braucht dringend Ihre Hilfe in ihren Wahnträumen. Sie gehen wie in \textit{Interception} rein!\footnote{\href{https://www.youtube.com/watch?v=3RhxyHMs-w8}{South Park Inception Spoof} -- Wunderbare South Park Folge}

\begin{center}
\begin{tabular}{ccc}
  \toprule
  Name & $P(win)$ & $P(outbid)$\\
  \midrule
  Yuki & 0.2 & 0.11\\
  Steffen & 0.3 & 0.12 \\
  \bottomrule
\end{tabular}
\end{center}

\begin{enumerate}
\item Mit welcher Wahrscheinlichkeit gewinnen Sie die Kaffeemaschine von Mitropa, wenn keiner der Kandidaten überbietet? \textbf{(1 Punkt)}
\item Wenn Ihre überbietungswahrscheinlichkeit $P(outbid)$ bei 0.076 liegt, mit welcher Wahrscheinlichkeit gewinnt \textit{keiner} die Kaffeemaschine von Mitropa? \textbf{(1 Punkt)}
\end{enumerate}

Glücksrad für Arme auf der Kirmes! Leider hat es für Maren Gilzer nicht gereicht. Deshalb sind Sie jetzt mit Tina auf der Kirmes und spielen mit einem Typen in einem Tentakelkostüm um das große Geld. Das Glücksrad hat 24 Felder. Sie beide drehen das Glücksrad zweimal. Auf 10 Feldern gewinnen Tina und Sie 4000EUR sonst 2000EUR. Ganz schön viel Geld und ganz schön zwielichtig hier...

\begin{enumerate}
  \setcounter{enumi}{2}  
\item Skizzieren Sie das Glücksrad und ergänzen Sie die Wahrscheinlichkeiten! \textbf{(1 Punkt)}
\item Zeichnen Sie das zugehörige Baumdiagramm für das zweimalige Drehen! Ergänzen Sie die Wahrscheinlichkeiten und die entsprechenden Ereignisse \textbf{(2 Punkte)}
\item Mir welcher Wahrscheinlichkeit gewinnen Sie 6000EUR? \textbf{(1 Punkt)}
\end{enumerate}

Im Fiebertraum von Tina reisen sie beide im Zug nach Köln um bei >>Geh aufs Ganze!<< mitzuspielen. Tina und Sie schaffen es tatsächlich ins Finale und können als Hauptgewinn ein Auto hinter einer der drei Türen gewinnen. Und was braucht man mehr als ein Auto in einem Fiebertraum? 

\begin{enumerate}
  \setcounter{enumi}{5}  
\item Bevor die Show beginnt, wird das Auto hinter eine zufällig bestimmte Tür gestellt. Mit welcher Wahrscheinlichkeit wird jeweils eine der drei Türen ausgewählt? Zeichnen Sie ein Baumdiagramm! \textbf{(1 Punkt)}
\item Mit welcher Wahrscheinlichkeit wählen Sie sofort die Tür mit dem Auto? Erweitere Sie das Baumdiagramm entsprechend! \textbf{(1 Punkt)}
\item Der Moderator öffnet nun eine der nicht gewählten Türen, aber natürlich nicht die mit dem Auto. Mit welcher Wahrscheinlichkeit steht das Auto hinter der anderen Tür? Erweitern Sie das Baumdiagramm entsprechend! \textbf{(2 Punkte)}
\item Lösen Sie nun das >>Ziegenproblem<<! Berechne Sie dazu die Wahrscheinlichkeiten der einzelnen Pfade. Lohnt sich ein Wechsel der anfangs gewählte Tür? Begründen Sie Ihre Antwort mathematisch! \textbf{(2 Punkte)}
\end{enumerate}
 

%https://unterrichten.zum.de/wiki/Laplace-Wahrscheinlichkeit_wiederholen_und_vertiefen/Ziegen

%Buchstabenhäufigkeit im Deutschen

%%https://www.mathelike.de/abiturloesungen-mathematik-bayern/mathematik-abitur-bayern-2020/pruefungsteil-a/stochastik-2/teilaufgabe-a.html

%%https://de.wikipedia.org/wiki/Buchstabenh%C3%A4ufigkeit


 
\clearpage
% -----------------------------------------------------------------------
\part{Angewandte Nutztier- und Pflanzenwissenschaften (M.Sc.)}
% -----------------------------------------------------------------------  

\section{Aufgabe \hfill (6 Punkte)}
Vergleichen Sie die Standardabweichung mit dem Standardfehler und grenzen
Sie die beiden Kennzahlen voneinander ab.



\section{Aufgabe \hfill (8 Punkte)}
Ihnen liegt folgendes Varianzanalysemodell mit der {\"u}blichen Beschreibung
zur Auswertung des Merkmals fett- und eiwei{\ss}korrigierte Milchleistung pro
Kuh und Jahr in kg vor:

\begin{equation*}
  Y_{ijkl} = \mu + Var_i + EKA_j + VarEKA_{ij} + V_k + b(L_{ij} - L) + e_{ijkl}
\end{equation*}

mit

\begin{itemize}
\item $Y_{ijkl}$: l-te Beobachtung
\item $\mu$: Populationsmittel 
\item $Var_i$: fixer Effekt der i-ten Variante (i: Kontrolle,
  Versuchsgruppe 1, Versuchsgruppe 2)
\item $EKA_j$: fixer Effekt der j-ten Erstkalbealtergruppe (j: EKA $\leq$ 25
  Monate, EKA > 25 Monate)
\item $VarEKA_{ij}$: fixer Effekt der Interaktion Variante x
  Erstkalbealtergruppe 
\item $V_k$: zuf{\"a}lliger Effekt des Vaters 
\item $b(L_{ij} - L)$: lineare Kovariable Laktationsnummer 
\item $e_{ijkl}$: zuf{\"a}lliger
  Restfehler
\end{itemize}
Erl{\"a}utern Sie anhand dieses Beispiels die Begriffe fixer Effekt,
Interaktion, zuf{\"a}lliger Effekt und Kovariable und grenzen Sie diese
Begriffe voneinander ab.



\section{Aufgabe \hfill (6 Punkte)}
Wie bestimmen Sie die richtige Stichprobengr{\"o}{\ss}e? Welche Kennzahlen /
statistische Ma{\ss}zahlen ben{\"o}tigen Sie dabei und nennen Sie die
Voraussetzungen.

 
\end{document}
% -----------------------------------------------------------------------


  
