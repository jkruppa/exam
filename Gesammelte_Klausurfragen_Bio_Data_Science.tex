\documentclass[a4paper, 9pt]{scrartcl}\usepackage[]{graphicx}\usepackage[]{xcolor}
% maxwidth is the original width if it is less than linewidth
% otherwise use linewidth (to make sure the graphics do not exceed the margin)
\makeatletter
\def\maxwidth{ %
  \ifdim\Gin@nat@width>\linewidth
    \linewidth
  \else
    \Gin@nat@width
  \fi
}
\makeatother

\definecolor{fgcolor}{rgb}{0.345, 0.345, 0.345}
\newcommand{\hlnum}[1]{\textcolor[rgb]{0.686,0.059,0.569}{#1}}%
\newcommand{\hlstr}[1]{\textcolor[rgb]{0.192,0.494,0.8}{#1}}%
\newcommand{\hlcom}[1]{\textcolor[rgb]{0.678,0.584,0.686}{\textit{#1}}}%
\newcommand{\hlopt}[1]{\textcolor[rgb]{0,0,0}{#1}}%
\newcommand{\hlstd}[1]{\textcolor[rgb]{0.345,0.345,0.345}{#1}}%
\newcommand{\hlkwa}[1]{\textcolor[rgb]{0.161,0.373,0.58}{\textbf{#1}}}%
\newcommand{\hlkwb}[1]{\textcolor[rgb]{0.69,0.353,0.396}{#1}}%
\newcommand{\hlkwc}[1]{\textcolor[rgb]{0.333,0.667,0.333}{#1}}%
\newcommand{\hlkwd}[1]{\textcolor[rgb]{0.737,0.353,0.396}{\textbf{#1}}}%
\let\hlipl\hlkwb

\usepackage{framed}
\makeatletter
\newenvironment{kframe}{%
 \def\at@end@of@kframe{}%
 \ifinner\ifhmode%
  \def\at@end@of@kframe{\end{minipage}}%
  \begin{minipage}{\columnwidth}%
 \fi\fi%
 \def\FrameCommand##1{\hskip\@totalleftmargin \hskip-\fboxsep
 \colorbox{shadecolor}{##1}\hskip-\fboxsep
     % There is no \\@totalrightmargin, so:
     \hskip-\linewidth \hskip-\@totalleftmargin \hskip\columnwidth}%
 \MakeFramed {\advance\hsize-\width
   \@totalleftmargin\z@ \linewidth\hsize
   \@setminipage}}%
 {\par\unskip\endMakeFramed%
 \at@end@of@kframe}
\makeatother

\definecolor{shadecolor}{rgb}{.97, .97, .97}
\definecolor{messagecolor}{rgb}{0, 0, 0}
\definecolor{warningcolor}{rgb}{1, 0, 1}
\definecolor{errorcolor}{rgb}{1, 0, 0}
\newenvironment{knitrout}{}{} % an empty environment to be redefined in TeX

\usepackage{alltt}
\usepackage[ngerman]{babel}

% -----------------------------------------------------------------------

% -----------------------------------------------------------------------
%% ------------------------------------------------------------
%% by J.Kruppa on Friday, February 11, 2022 (11:31)
%% \def\mainDir{\Sexpr{exam_path}}
\def\source{/Users/jokruppa/source/tex}
\usepackage[margin=2cm, includefoot]{geometry}
\setlength{\parindent}{0cm}
\usepackage{booktabs}
\usepackage{amsmath}
\usepackage{scalerel,amssymb}
\usepackage{setspace}
\def\csquare{{\Large $\boxtimes$}}
\def\msquare{{\Large $\square$}}
\usepackage[normalem]{ulem}
\usepackage{array}
\usepackage{xcolor}
\usepackage{float}
\usepackage{currfile}
\usepackage{tikz}
\usepackage[nomessages]{fp}

%% beamer defs
\def\lecture{Klausurfragen der Bio Data Science}

%% exam defs
\def\examtitle{\lecture}
\def\exammodule{
\vspace{-1.75cm}  
\begin{graybox}{}
\vspace{2Ex}
\textbf{\large Name:} \rule[0ex]{16.75em}{.4pt}
\hfill \textnormal{\textit{Nicht bestanden:}} \msquare \\[2.5Ex]
\textbf{\large Vorname:} \rule[0ex]{15em}{.4pt} \\[2.5Ex]
\textbf{\large Matrikelnummer:} \rule[0ex]{10.8em}{.4pt}
\hfill Endnote: \rule[0ex]{7em}{.4pt} 
\end{graybox}
\vspace{3Ex}
\phantom{text}
}
\def\examsemester{Sommersemester \& Wintersemester}
\def\examdate{\today}
%% ------------------------------------------------------------
\definecolor{darkblue}{rgb}{0,0,.5}
\definecolor{darkpurple}{rgb}{0.4117, 0.2, 0.4117}
\definecolor{uni}{rgb}{0,0.3137,0.6078}
\definecolor{gray}{gray}{0.7}

\usepackage{tcolorbox}
\definecolor{logo1}{RGB}{0, 158, 227}
\definecolor{gray5}{RGB}{247, 247, 247}
\definecolor{gray2}{RGB}{102, 102, 102}

\newtcolorbox{graybox}[1]{
  colback=gray5,%%red!5!white,
  colframe=gray2,%%red!75!black,
  fonttitle=\bfseries\Large,
  %%valign=center,
  fontupper=\large,
  before skip=10pt plus 2pt,
  after skip=20pt plus 4pt,
  title=#1}

\newtcolorbox{takehomebox}[1]{
  colback=gray5,%%red!5!white,
  colframe=logo1,%%red!75!black,
  fonttitle=\bfseries\Large,
  %%valign=center,
  fontupper=\large,
  before skip=10pt plus 2pt,
  after skip=10pt plus 2pt,
  title=#1}

\def\Rlogo{\includegraphics[width = 0.5cm]{\string~/Documents/GitHub/exam/img/Rlogo}\;}

\usepackage[scaled=.90]{helvet} 
\usepackage{fancyhdr}
\usepackage{lastpage}
\usepackage{hyperref}
\hypersetup{
    colorlinks=true,       % false: boxed links; true: colored links
    linkcolor=black,          % color of internal links 
    urlcolor=magenta           % color of external links
}
\renewcommand{\familydefault}{\sfdefault}

\title{
\large \exammodule \\[5Ex]
\Huge \examtitle \\[2Ex] 
\Large Hochschule Osnabr{\"u}ck
}
\author{Pr{\"u}fer: Prof. Dr. Jochen Kruppa \\
Fakult{\"a}t f{\"u}r Agrarwissenschaften und Landschaftsarchitektur \\ 
j.kruppa@hs-osnabrueck.de}
\date{Version vom \examdate}

%% ------------------------------------------------------------
%% by J.Kruppa on Tuesday, September 23, 2014 (12:50)
%% Header
\renewcommand{\headrulewidth}{0pt}
\renewcommand{\footrulewidth}{0pt}
\pagestyle{fancy}

\fancyhf{}
\fancyhead[L]{}
\fancyhead[R]{}
\fancyfoot[R]{\thepage}
\fancyfoot[L]{\footnotesize \examtitle}

\fancypagestyle{empty}{
 \fancyhf{}
 \fancyhead[L]{}
 \fancyhead[R]{}
 \fancyfoot[R]{\thepage}
 \fancyfoot[L]{\footnotesize \examtitle}
}

\usepackage{arevtext,arevmath}

\newcommand\Tstrut{\rule{0pt}{2.6ex}}         % = `top' strut
\newcommand\Bstrut{\rule[-0.9ex]{0pt}{0pt}}   % = `bottom' strut
\def\strut{\Tstrut\Bstrut}

% -----------------------------------------------------------------------
\IfFileExists{upquote.sty}{\usepackage{upquote}}{}
\begin{document}
\date{Wintersemester 2024/25 
\vfill
\begin{center}
\includegraphics[width = 1.9cm]{avatare/Alex}\hspace{-8mm}
\includegraphics[width = 1.9cm]{avatare/Jessica}\hspace{-8mm}
\includegraphics[width = 1.9cm]{avatare/Jonas}\hspace{-8mm}
\includegraphics[width = 1.9cm]{avatare/Mark}\hspace{-8mm}
\includegraphics[width = 1.9cm]{avatare/Nilufar}\hspace{-8mm}
\includegraphics[width = 1.9cm]{avatare/Paula}\hspace{-8mm}
\includegraphics[width = 1.9cm]{avatare/Steffen}\hspace{-8mm}
\includegraphics[width = 1.9cm]{avatare/Tina}\hspace{-8mm}
\includegraphics[width = 1.9cm]{avatare/Yuki}\\
\small
\vspace{1.5Ex}
\textit{"`The test of a student is not how much he knows,\\ but how much he wants to know."'\\ --- Alice W. Rollins}
\end{center}}
% -----------------------------------------------------------------------
\maketitle
\fancypagestyle{empty}{
  \fancyfoot[L]{\tiny $\square\!\square\!\square\!\square\!\blacksquare\!\blacksquare\!\square\!\blacksquare\!\blacksquare\!\blacksquare\!\blacksquare\!\square\!\blacksquare\!\square\!\blacksquare\!\blacksquare\!\blacksquare\!\blacksquare\!\square\!\blacksquare$}
}
\thispagestyle{empty}
\clearpage
% -----------------------------------------------------------------------
\begin{minipage}[c]{0.125\textwidth}
\includegraphics[width = 1.9cm]{avatare/Alex}
\end{minipage}
\begin{minipage}[c]{0.875\textwidth}
\textit{Alex studiert im 3. Semester und wiederholt das Modul, da er im ersten Jahr andere Prioritäten für sich gesetzt hat. Das musste sein, da er sich ziemlich im Abitur verausgabt hat. Darüber hinaus war die WG auch eher auf Party angelegt. Alex hofft jetzt über Pünktlichkeit wieder in die Bahn zu kommen. Dafür steht er jetzt immer um 5 Uhr auf! Freunde von ihm beschreiben ihn eher als extrovertiert. Er kennt Paula noch aus der Schulzeit und er überlegt, ob nicht beide Mal nach Mallorca sollten.} 
\end{minipage}\\[2.75Ex]
% -----------------------------------------------------------------------
\begin{minipage}[c]{0.875\textwidth}
\textit{Nach zwei Semestern Studium an der Universität Osnabrück war es dann Jessica doch viel zu theoretisch. Etwas angewandtes sollte es sein, wo sie auch eine Fähigkeit lernt, die frau nutzen kann. Deshalb hat sich Jessica an der Hochschule eingeschrieben. Hoffentlich lernt sie etwas nützliches, wo andere für Geld geben würden. Wer nützlich ist, ist wertvoll. Ihr Traum ist ja eine Hundeschule aufzumachen. Die großen Parties hat sie immer gemieden. Sie ist lieber mit ihrer Hündin im Teuteburgerwald.}
\end{minipage}
\begin{minipage}[c]{0.125\textwidth}
\includegraphics[width = 1.9cm]{avatare/Jessica}
\end{minipage}\\[2.75Ex]
% -----------------------------------------------------------------------
\begin{minipage}[c]{0.125\textwidth}
\includegraphics[width = 1.9cm]{avatare/Jonas}
\end{minipage}
\begin{minipage}[c]{0.875\textwidth}
\textit{Das ist jetzt der letzte Versuch mit einem Studium. Wenn es nicht klappt dann überlegt Jonas das \href{https://www.ihk.de/osnabrueck/aus-und-weiterbildung/ausbildung/ausbildungsbetriebe/projekt-neustart-1087206}{Programm der IHK zu Ausbildungsvermittlung} zu nutzen. Aber eine Runde gibt er sich noch. Struktur ist eigentlich das Wichtigste und diesmal hat er sich alle Altklausuren der Fachschaft besorgt. Dann ist er auch noch gleich der Fachschaft beigetreten um mehr Informationen abzugreifen. Und er versucht nicht seine Zeit mit Alex zu verdaddeln oder in der Fachschaft bei einem Bier oder so...}
\end{minipage}\\[2.75Ex]
% -----------------------------------------------------------------------
\begin{minipage}[c]{0.875\textwidth}
\textit{Nächstes Semester geht es nach Kanada davon hat er schon auf der Berufsschule geträumt. Deshalb konzentriert er sich sehr auf die Prüfungen. Ein Schiff ist im Hafen sicher, aber dafür ist es nicht gebaut worden. Das \href{https://www.hs-osnabrueck.de/wir/fakultaeten/aul/international/}{International Faculty Office} der Fakultät Agrarwissenschaften und Landschaftsarchitektur hat super geholfen, aber es waren einiges an Unterlagen. Jetzt hofft er, dass Tina dann doch noch mitkommt. Aber sonst macht er das eben alleine. Obwohl das eher nicht so seine Art ist. Vielleicht sollte er sich mal einen Tipp bei Tina holen, sie wirkt sehr entschlossen.} 
\end{minipage}
\begin{minipage}[c]{0.125\textwidth}
\includegraphics[width = 1.9cm]{avatare/Mark}
\end{minipage}\\[2.75Ex]
% -----------------------------------------------------------------------
\begin{minipage}[c]{0.125\textwidth}
\includegraphics[width = 1.9cm]{avatare/Nilufar}
\end{minipage}
\begin{minipage}[c]{0.875\textwidth}
\textit{Nach der Ausbildung wollte Nilufar eigentlich gleich anfangen zu arbeiten, aber nach einem Praktikum und der Probezeit stellte sie fest, dass es ohne einen Hochschulabschluss schwer wird Führungsverantwortung zu übernehmen. Mit Menschen kann sie schon immer und dann auch eigene Projekte mit anderen verwirklichen, dass ist doch was. Mit dem notwendigen Abschluss sollte der Start um so einfacher sein. Dann ist sie keine Befehlsempfängerin mehr sondern gibt die Marschrichtung vor. Schon jetzt koordiniert Nilufar das Studium von anderen.}
\end{minipage}\\[2.75Ex]
% -----------------------------------------------------------------------
\begin{minipage}[c]{0.875\textwidth}
\textit{Paula möchte die Welt zu einem besseren Ort machen. Wenn da nicht die anderen Mitmenschen wären. Paula muss das Modul nochmal wiederholen, da es dann am Ende des Semesters zu viel für sie wurde. Eine Lerngruppe hätte geholfen, aber das ist dann gar nicht so einfach eine zu finden. Zwar kennt sie schon Nilufar, aber Nilufar ist ihr manchmal zu forsch. Ihr schwant aber, dass alleine das Studium sehr schwer werden wird. Das Abitur war schon so ein Lernhorror, das möchte sie nicht nochmal. Alex sieht sie da als Vorbild.}
\end{minipage}
\begin{minipage}[c]{0.125\textwidth}
\includegraphics[width = 1.9cm]{avatare/Paula}
\end{minipage}\\[2.75Ex]
% -----------------------------------------------------------------------
\begin{minipage}[c]{0.125\textwidth}
\includegraphics[width = 1.9cm]{avatare/Steffen}
\end{minipage}
\begin{minipage}[c]{0.875\textwidth}
\textit{Sommer, Sonne, Natur. Das ist es was Steffen mag. Raus in die Komune und die Natur genießen. Leider hat Steffen noch andere Bedürfnisse, die ein Einkommen benötigen. Da Studierte mehr verdienen, würde dann in Teilzeit auch mehr rausspringen. Wenn er dann privat was anbauen kann, dann spart er gleich doppelt. Leider sind viele seiner Kommilitonen total verkrampfte Karrieristen. Es geht nur ums Äußere. Dabei verliert sich Steffen gerne im Prozess. Das hat auch seinen Schulabschluss etwas verzögert. Steffen lässt sich eben Zeit.}
\end{minipage}\\[2.75Ex]
% -----------------------------------------------------------------------
\begin{minipage}[c]{0.875\textwidth}
\textit{Wille  war es, die es Tina hierher gebracht hat und Wille wird es sein, die Tina dann auch zum Abschluß treibt. Nach einem Rückschlag muss Tina jetzt einige Module wiederholen, damit sie dann auch fertig wird. Ab und zu ist sie schwach gewesen und das hat dann Zeit gekostet. Das Tina es dann manchmal übertreibt, weiß sie nur zu gut, aber irgendwie muss die Kontrolle ja erhalten bleiben? Insbesondere, wenn sie mal wieder die Nacht durchgefeiert hat, verachtet Tina sich. Dann baut Nilufar sie dann bei einem Tee wieder auf.}
\end{minipage}
\begin{minipage}[c]{0.125\textwidth}
\includegraphics[width = 1.9cm]{avatare/Tina}
\end{minipage}\\[2.75Ex]
% -----------------------------------------------------------------------
\begin{minipage}[c]{0.125\textwidth}
\includegraphics[width = 1.9cm]{avatare/Yuki}
\end{minipage}
\begin{minipage}[c]{0.875\textwidth}
\textit{Für Yuki war es nicht einfach. Teilweise war die Krankheit sehr hinderlich, dann war es Yuki selber. Dann muss man auch wieder auf die Beine kommen und es dauert eben seine Zeit. Aber immerhin hat Yuki es jetzt den Abschluss gekriegt und hat einen Studienplatz. Jetzt heißt es in den Rhythmus kommen und schauen, was noch so passiert. Immerhin hat Yuki schon eine kleine Gruppe gefunden, in der Yuki dann Hilfe findet. Ist aber auch sehr unübersichtlich so ein Studium. Steffen ist immer super entspannt.}
\end{minipage}
\clearpage
% -----------------------------------------------------------------------


\begin{graybox}{Erlaubte Hilfsmittel}
  \vspace{1Ex}
  \begin{itemize}
  \item Normaler Taschenrechner ohne Möglichkeit der Kommunikation mit anderen
    Geräten! Ausdrücklich kein Handy!
  \item Eine DIN A4-Seite als beidseitig, selbstgeschriebene,
    handschriftliche Formelsammlung. Keine digitalen Ausdrucke! 
  \item \textbf{\textcolor{red}{Die Verwendung eines roten Farbstiftes ist nicht gestattet! Korrekturfarbe!}}
  \item \textit{You can answer the questions in English without any consequences.}  
  \end{itemize}
\end{graybox}
\vfill

\begin{graybox}{Endnote}
  \vspace{1Ex}
  \begin{itemize}
  \item[] \rule[0ex]{3em}{.4pt}\, von 20\, Punkten sind aus den Multiple
    Choice Aufgaben erreicht.
  \item[] \rule[0ex]{3em}{.4pt}\, von 76 Punkten sind aus den Rechen- und
    Textaufgaben erreicht. 
  \item[] \rule[0ex]{3em}{.4pt}\, von 96 Punkten in Summe.
  \item[] Es wird folgender Notenschlüssel angewendet.   
  \end{itemize}
  \vspace{1ex}
\begin{center}
  \begin{tabular}[c]{cc}
    \toprule
    \textbf{Punkte}	&	\textbf{Note}	\\
    \midrule
    91.5 - 96.0	&	1,0	\\
    87.0 - 91.0	&	1,3	\\
    82.0 - 86.5	&	1,7	\\
    77.5 - 81.5	&	2,0	\\
    72.5 - 77.0	&	2,3	\\
    67.5 - 72.0	&	2,7	\\
    63.0 - 67.0	&	3,0	\\
    58.0 - 62.5	&	3,3	\\
    53.5 - 57.5	&	3,7	\\
    48.0 - 53.0	&	4,0	\\
    \bottomrule
  \end{tabular}
\end{center}
  \vspace{1ex}
\begin{itemize}
\item[] Es ergibt sich eine Endnote von \rule[0ex]{4em}{.4pt}.
\end{itemize}
  \vspace{1Ex}
\end{graybox}

% -----------------------------------------------------------------------
\newpage
% -----------------------------------------------------------------------

\begin{graybox}{Multiple Choice Aufgaben}
  \begin{itemize}
  \item Pro Multipe Choice Frage ist \emph{genau} eine Antwort richtig.
  \item \textbf{Übertragen Sie Ihre Kreuze in die Tabelle auf
      dieser Seite.}
  \end{itemize}

\begin{center}
  \large
  \begin{tabular}{|l|c|c|c|c|c?c|}
    \hline
    & \textbf{A} & \textbf{B} & \textbf{C} & \textbf{D} & \textbf{E} & $\boldsymbol{\checkmark}$\strut\\
    \hline
    \textbf{Aufgabe 1} &   &   &   &   &   & \strut\\
    \hline
    \textbf{Aufgabe 2} &   &   &   &   &   & \strut\\
    \hline
    \textbf{Aufgabe 3} &   &   &   &   &   & \strut\\
    \hline
    \textbf{Aufgabe 4} &   &   &   &   &   & \strut\\
    \hline
    \textbf{Aufgabe 5} &   &   &   &   &   & \strut\\
    \hline
    \textbf{Aufgabe 6} &   &   &   &   &   & \strut\\
    \hline
    \textbf{Aufgabe 7} &   &   &   &   &   & \strut\\
    \hline
    \textbf{Aufgabe 8} &   &   &   &   &   & \strut\\
    \hline
    \textbf{Aufgabe 9} &   &   &   &   &   & \strut\\
    \hline
    \textbf{Aufgabe 10} &   &   &   &   &   & \strut\\
    \hline
  \end{tabular}
\end{center}

\begin{itemize}
\item Es sind \rule[0ex]{2em}{.4pt}\, von 20 Punkten erreicht worden.
\end{itemize}
\end{graybox}

\vfill

\begin{graybox}{Rechen- und Textaufgaben}
  \begin{center}
    \large
    \begin{tabular}{|l|c|c|c|c|c|c|c|}
      \hline
      \textbf{Aufgabe} & \textbf{11} & \textbf{12} & \textbf{13} & \textbf{14} & \textbf{15} & \textbf{16} & \textbf{17} \strut\\
      \hline
      \textbf{Punkte} & 
      \hspace{1Ex}\Large\textcolor{gray!70}{10}\hspace{1Ex}  & 
      \hspace{1Ex}\Large\textcolor{gray!70}{11}\hspace{1Ex}  & 
      \hspace{1Ex}\Large\textcolor{gray!70}{12}\hspace{1Ex}  & 
      \hspace{1Ex}\Large\textcolor{gray!70}{11}\hspace{1Ex}  & 
      \hspace{1Ex}\Large\textcolor{gray!70}{10}\hspace{1Ex}  & 
      \hspace{1Ex}\Large\textcolor{gray!70}{12}\hspace{1Ex}  & 
      \hspace{1Ex}\Large\textcolor{gray!70}{10}\hspace{1Ex} \strut\\
      \hline
  \end{tabular}
\end{center}
\begin{itemize}
\item Es sind \rule[0ex]{2em}{.4pt}\, von 76 Punkten erreicht worden.
\end{itemize}
\end{graybox}

% -----------------------------------------------------------------------
\clearpage
% -----------------------------------------------------------------------
\begin{graybox}{Multiple Choice Aufgaben}
Die Multiple Choice Aufgaben \textcolor{red}{unterliegen dem Zufall}. Die Reihenfolge der Antworten ist zufällig. Die Fragen und Antworten sind semantisch zufällig und haben somit \textcolor{red}{verschiedene Textvarianten}. Insbesondere die reinen Textaufgaben haben verschiedene Textvarianten. Die Semeantik mag sich unterscheiden, die Inhalte sind aber gleich.
\end{graybox}
\section*{ANOVA}

\section{Aufgabe \hfill (2 Punkte)}





Nach der Berechnung einer einfaktoriellen ANOVA ergibt sich ein $\eta^2 = 0.12$. Welche Aussage ist richtig?



\begin{enumerate}
\item [\textbf{A} \msquare] Das $\eta^2$ ist die Korrelation der ANOVA. Mit der Ausnahme, dass 0 der beste Wert ist.
\item [\textbf{B} \msquare] Das $\eta^2$ ist ein Wert f{"u}r die G{"u}te der ANOVA. Je kleiner desto besser. Ein $\eta^2$ von 0 bedeutet ein perfektes Modell mit keiner Abweichung. Die Varianz ist null.
\item [\textbf{C} \msquare] Das $\eta^2$ ist damit mit dem $R^2$ aus der linearen Regression zu vergleichen und beschreibt den Anteil der Varianz, der von den Behandlungsbedingungen erkl{"a}rt wird.
\item [\textbf{D} \msquare] Das $\eta^2$ beschreibt den Anteil der Varianz, der von den Behandlungsbedingungen nicht erkl{"a}rt wird. Somit der Rest an nicht erkl{"a}rbarer Varianz.
\item [\textbf{E} \msquare] Die Berechnung von $\eta^2$ ist ein Wert f{"u}r die Interaktion.
\end{enumerate} 

\section{Aufgabe \hfill (2 Punkte)}



Sie führen ein Feldexperiment durch um das Gewicht von Brokoli zu
steigern. Die Pflanzen wachsen unter einer Kontrolle und zwei verschiedenen
Behandlungsbedingungen. Nach der Berechnung einer einfaktoriellen ANOVA
ergibt sich ein $\eta^2 = 0.18$. Welche Aussage ist richtig?



\begin{enumerate}
\item [\textbf{A} \msquare] Das $\eta^2$ beschreibt den Anteil der Varianz, der durch den Forschenden entsteht. Es gilt die Regel, dass ca. 70\% der Varianz eines Versuches durch die Versuchsdurchführung entstehen sollen.
\item [\textbf{B} \msquare] Das $\eta^2$ beschreibt den Anteil der Varianz, der von den Behandlungsbedingungen erklärt wird. Daher werden 18\% der Varianz erklärt.
\item [\textbf{C} \msquare] Es werden 18\% der Varianz durch den Versuch erklärt. Das $\eta^2$ beschreibt den Anteil der Varianz, der durch Fehler in der Versuchsdurchführung entsteht.
\item [\textbf{D} \msquare] Mit dem $\eta^2$ lässt sich auf die Qualität der Randomisierung und damit der Strukturgleichheit zwischen der Grundgesamtheit und der Stichprobe schließen. Es gilt dabei die Regel, dass ein $\eta^2$-Wert von 1 zu bevorzugen ist.
\item [\textbf{E} \msquare] Es werden 82\% der Varianz durch die Behandlung erklärt. Das $\eta^2$ beschreibt den Anteil der Varianz, der von den unterschiedlichen Behandlungsbedingungen nicht erklärt wird.
\end{enumerate} 

\section{Aufgabe \hfill (2 Punkte)}



Die einfaktorielle ANOVA ist ein Standardverfahren in der agrawissenschaftlichen Forschung wenn es um den Vergleich von Behandlungsgruppen geht. Welche der folgenden Aussage zu der Berechnung der Teststatistik der einfaktoriellen ANOVA ist richtig?



\begin{enumerate}
\item [\textbf{A} \msquare] Wenn die F-Statistik höher ist als der kritische Wert kann die Nullhypothese nicht abgelehnt werden. Die F-Statistik ist die Differenz der MS der Behandlung durch die MS des Fehlers.
\item [\textbf{B} \msquare] Die F-Statistik wird berechnet indem die MS der Behandlung durch die MS des Fehlers geteilt werden. Wenn die F-Statistik sich kaum von der Null unterscheidet kann die Nullhypothese nicht abgelehnt werden.
\item [\textbf{C} \msquare] Die ANOVA berechnet die T-Statistik indem den Mittelwertsunterschied der Gruppen simultan durch die Standardabweichung der Gruppen teilt. Wenn die T-Statistik h{"o}her als 1.96 ist, kann die Nullhypothese abgelehnt werden.
\item [\textbf{D} \msquare] Die ANOVA berechnet die F-Statistik indem die MS des Fehlers durch die MS der Behandlung geteilt werden. Wenn die F-Statistik sich der 0 ann{"a}hert kann die Nullhypothese abgelehnt werden.
\item [\textbf{E} \msquare] Die ANOVA berechnt die F-Statistik aus den SS Behandlung geteilt durch die SS Fehler.
\end{enumerate} 

\section{Aufgabe \hfill (2 Punkte)}




Viele statistische Verfahren nutzen eine Teststatistik um eine Aussage über den Zusammenhang zwischen der Grundgesamthat und der Stichprobe abzubilden. Ein statistisches Testwerkzeug ist hierbei die ANOVA. Die ANOVA rechnet dabei...



\begin{enumerate}
\item [\textbf{A} \msquare] ... den Unterschied zwischen der Varianz über alle Behandlungsgruppen oder der Varianz aus verschiedenen Behandlungsguppen. Wenn die ANOVA signifikant ist, muss sich zwischen einem der beiden Varianzquellen entschieden werden.
\item [\textbf{B} \msquare] ... den Unterschied zwischen der globalen Varianz und der Varianz aus verschiedenen Behandlungsguppen. Wenn die ANOVA signifikant ist, ist nicht bekannt welcher Vergleich konkret unterschiedlich ist.
\item [\textbf{C} \msquare] ... den Unterschied zwischen der Varianz ausgelöst durch alle Behandlungsgruppen und der Varianz aus globalen Behandlungsguppen der Kontrollen. Wenn die ANOVA nicht signifikant ist, muss ein Posthoc-Test ausgeschlossen werden.
\item [\textbf{D} \msquare] ... den Unterschied zwischen der Mittelwerte und der Varianz aus verschiedenen Behandlungsguppen. Wenn die ANOVA signifikant ist, ist bekannt welcher Vergleich konkret unterschiedlich ist.
\item [\textbf{E} \msquare] ... den Unterschied zwischen der Varianz in den verschiedenen Behandlungsguppen und der Varianz in einer der Behandlungsgruppen. Wenn die ANOVA signifikant ist, muss über einen Posthoc-Test nachgedacht werden um den signifikanten Unterschied in einer der Gruppen exakt zu bestimmen.
\end{enumerate} 

\section{Aufgabe \hfill (2 Punkte)}



Ein Versuch wurde an 63 Tieren durchgeführt, wobei jedes Tier eine von drei Vitamin-C-Dosen (0.5, 1 und 1.5 mg/Tag) über eine von zwei Verabreichungsmethoden erhielt. Die folgende Abbildung enthält die Daten aus diesem Versuch zur Bewertung der Wirkung von Vitamin D auf das Zahnwachstum bei Igeln.  Welche Aussage ist richtig, wenn Sie eine zweifaktorielle ANOVA rechnen?



{\centering \includegraphics[width=\maxwidth]{img/mc-anova-02-a-1} 

}







\begin{enumerate}
\item [\textbf{A} \msquare] Eine mittlere bis starke Interaktion liegt vor $(p \leq 0.05)$
\item [\textbf{B} \msquare] Die Koeffizienten sind positiv $(\beta_0 > 0; \beta_1 > 0)$.
\item [\textbf{C} \msquare] Die Koeffizienten sind negativ $(\beta_0 < 0; \beta_1 < 0)$.
\item [\textbf{D} \msquare] Keine Interaktion liegt vor $(p \leq 0.05)$.
\item [\textbf{E} \msquare] Eine Korrelation liegt vor $(p \leq 0.05)$.
\end{enumerate} 
\section*{Deskriptive Statistik \& Explorative Datenanalyse}

\section{Aufgabe \hfill (2 Punkte)}




Wie lautet der Mittelwert und Standardabweichung von $y$ mit 3, 14, 13, 12 und 9.



\begin{enumerate}
\item [\textbf{A} \msquare] Sie erhalten 10.2 +/- 4.44
\item [\textbf{B} \msquare] Es berechnet sich 11.2 +/- 19.7
\item [\textbf{C} \msquare] Es ergibt sich 11.2 +/- 2.22
\item [\textbf{D} \msquare] Sie erhalten 10.2 +/- 2.11
\item [\textbf{E} \msquare] Es berechnet sich 10.2 +/- 19.7
\end{enumerate} 

\section{Aufgabe \hfill (2 Punkte)}




Wie lautet der Median, das $1^{st}$ Quartile sowie das $3^{rd}$ Quartile von $y$ mit 30, 23, 20, 25, 23, 14 und 42.




\begin{enumerate}
\item [\textbf{A} \msquare] Sie erhalten 23 [20; 30]
\item [\textbf{B} \msquare] Es ergibt sich 25 +/- 20
\item [\textbf{C} \msquare] Sie erhalten 23 [18; 28]
\item [\textbf{D} \msquare] Es berechnet sich 25 [21; 31]
\item [\textbf{E} \msquare] Sie erhalten 23 +/- 30
\end{enumerate} 

\section{Aufgabe \hfill (2 Punkte)}



Mit einem Boxplot  können Sie sehr gut die Verteilung von Daten visualisieren. Die empfohlene Mindestanzahl an Beobachtungen ist dabei?



\begin{enumerate}
\item [\textbf{A} \msquare] Die opimale Anzahl ist größer als hundert Beobachtungen, wobei es gerne sehr viel mehr sein können.
\item [\textbf{B} \msquare] 1 Beobachtung.
\item [\textbf{C} \msquare] Die untere Grenze liegt bei einer Beobachtung.
\item [\textbf{D} \msquare] Wir brauchen fünf oder mehr Beobachtungen.
\item [\textbf{E} \msquare] Mindestens 20 Beobachtungen.
\end{enumerate}

\section{Aufgabe \hfill (2 Punkte)}



Um die Varianz zu berechnen müssen wir folgende Rechenoperationen durchführen.



\begin{enumerate}
\item [\textbf{A} \msquare] Als erstes berechnen wir den Mittelwert. Dann bilden wir die Summe der quadratischen Abstände zu dem Mittelwert. Abschließend teilen wir durch die Fallzahl.
\item [\textbf{B} \msquare] Den Mittelwert berechnen und die Abstände quadrieren. Die Summe mit der Fallzahl multiplizieren.
\item [\textbf{C} \msquare] Wir berechnen erst den Mittelwert und dann die absoluten Abstände zu dem Mittelwert. Diese quadratischen Abstände summieren wir auf und teilen am Ende durch die Fallzahl.
\item [\textbf{D} \msquare] Den Median berechen, dann die quadratischen Abstände zum Median aufsummieren, dann die Wurzel ziehen.
\item [\textbf{E} \msquare] Den Mittelwert berechen, dann die quadratischen Abstände zum Mittelwert aufsummieren und durch die Fallzahl teilen, dann die Wurzel ziehen.
\end{enumerate} 

\section{Aufgabe \hfill (2 Punkte)}



Nachdem Sie eine ANOVA und die paarweisen t-Tests über das \Rlogo Paket \{emmeans\} durchgeführt haben, müssen Sie Ihre Daten nochmal zur Überprüfung visualisieren. Sie entscheiden sich für den Boxplot. Welche statistischen Maßzahlen stellt der Boxplot dar?

 



\begin{enumerate}
\item [\textbf{A} \msquare] Der Boxplot stellt die Mittelwerte und die Standardabweichung dar.
\item [\textbf{B} \msquare] Der Boxplot stellt den Median und die Quartile dar.
\item [\textbf{C} \msquare] Der Boxplot stellt den Median und die Streuung dar.
\item [\textbf{D} \msquare] Durch die Abbildung des Boxplot erhalten wir die Informationen über den Median und die Standardabweichung.
\item [\textbf{E} \msquare] Den Mittelwert sowie den Median und die Streuung.
\end{enumerate}

\section{Aufgabe \hfill (2 Punkte)}



In Ihrer Abschlussarbeit zuMaiss finden Sie aufeinmal seltsame Daten. Jedenfalls kommt Ihnen das so vor. Daher berechnen Sie den Mittelwert und den Median. Der Mittelwert $\bar{y}$ und der Median $\tilde{y}$ unterscheiden sich. Welche Aussage ist richtig?



\begin{enumerate}
\item [\textbf{A} \msquare] Da sich der Mittelwert und der Median unterscheiden, liegen vermutlich keine Outlier in den Daten vor. Wir verweden den Datensatz so wie er ist.
\item [\textbf{B} \msquare] Der Mittelwert und der Median sollten gleich sein, wenn Outlier in den Daten vorliegen. 
\item [\textbf{C} \msquare] Da sich der Mittelwert und der Median nicht unterscheiden, liegen vermutlich Outlier in den Daten vor. Wir untersuchen den Datensatz nach auffälligen Beobachtungen.
\item [\textbf{D} \msquare] Wenn sich der Mittelwert und der Median unterscheiden, liegen vermutlich Outlier in den Daten vor.
\item [\textbf{E} \msquare] Der Mittelwert und der Median sollten sich unterscheiden sein, wenn Outlier in den Daten vorliegen. 
\end{enumerate}

\section{Aufgabe \hfill (2 Punkte)}



Ihre Betreuung der Abschlussarbeit fragt überraschend in der letzten Besprechung, ob Ihre Messwerte einer Normalverteilung genügen. Sonst könnten Sie ja gar nicht einen t-Test rechnen. Da Ihnen die Zeit wegrennt, entscheiden Sie sich für eine schnelle Visualisierung im Anhang. Welche Visualisierung nutzen Sie und welche Regel kommt zur Abschätzung einer Normalverteilung zur Anwendung?



\begin{enumerate}
\item [\textbf{A} \msquare] Nach dem Einlesen der Daten nutzen wir einen Boxplot um zu schauen, ob alle Boxen über alle Behandlungen in etwa gleich groß sind. Damit ist dann auch das IQR in allen Behandlungen in etwa gleich.
\item [\textbf{B} \msquare] Einen Violinplot. Der Bauch der Violine muss hierbei einen höhren Wert annehmen als der Steg der Violine. Dann kann die Annahme einer Normalverteilung angenommen werden.
\item [\textbf{C} \msquare] In einer explorativen Datanalyse nutzen wir den Violinplot. Dabei sollte der Bauch am Rand liegen. Dann können wir von einer Normalverteilung ausgehen.
\item [\textbf{D} \msquare] In einer explorativen Datanalyse nutzen wir den Boxplot. Dabei sollte der Median als dicke Linie in der Mitte der Box liegen. Dann können wir von einer Normalverteilung ausgehen.
\item [\textbf{E} \msquare] Einen Dotplot. Die Punkte müssen sich wie an einer Perlenschnurr audreihen. Eine Abweichung führt zur Ablehnung der Annahme einer Normalverteilung.
\end{enumerate}

\section{Aufgabe \hfill (2 Punkte)}




In der Statistik müssen wir häufig überprüfen, ob unser Outcome einer bestimmten Verteilung folgt. Meistens überprüfen wir, ob eine
Normalverteilung vorliegt. Folgende drei Abbildungen eigenen sich im Besonderen für die Überprüfung einer Verteilungsannahme an eine Variable.





\begin{enumerate}
\item [\textbf{A} \msquare] Histogramm, Densityplot, Dotplot
\item [\textbf{B} \msquare] Violinplot, Boxplot, Densityplot
\item [\textbf{C} \msquare] Histogramm, Scatterplot, Boxplot
\item [\textbf{D} \msquare] Boxplot, Violinplot, Mosaicplot
\item [\textbf{E} \msquare] Barplot, Mosaicplot, Violinplot
\end{enumerate} 

\section{Aufgabe \hfill (2 Punkte)}



Sie haben $n = 195$ Pflanzen geerntet und wollen sich nun die Verteilung der Pflanzen einmal in einem Histogramm anschauen. Welche Verteilung ist dargestellt?



{\centering \includegraphics[width=\maxwidth]{img/mc-distribution-02-a-1} 

}







\begin{enumerate}
\item [\textbf{A} \msquare] Wir haben eine Normalverteilung vorliegen.
\item [\textbf{B} \msquare] Wir haben eine Gammaverteilung vorliegen.
\item [\textbf{C} \msquare] Eine multivariate Normalverteilung.
\item [\textbf{D} \msquare] Es handelt sich um eine Poisson-Verteilung.
\item [\textbf{E} \msquare] In dem Histogramm ist eine Ordinalverteilung dargestellt.
\end{enumerate} 
\section*{Lineare Regression \& Korrelation}

\section{Aufgabe \hfill (2 Punkte)}



Im Allgemeinen gibt es zwei mögliche Ziele für ein Regressionsmodell. Wir können eine Vorhersagemodell oder ein kausales Modell rechnen. Welche Aussage ist für ein kausales Modell richtig?



\begin{enumerate}
\item [\textbf{A} \msquare] Es wird ein Trainingsdatensatz zum Modellieren des Trainingsmodells benötigt. Der Testdatensatz dient rein zur Visualisierung. Dies gilt vor allem für ein kausales Modell.
\item [\textbf{B} \msquare] Ein kausales Modell wird auf einem Trainingsdatensatz trainiert und anschliessend über eine explorative Datenanalyse validiert. Signifikanzen über $\beta_i$ können hier nicht festgestellt werden.
\item [\textbf{C} \msquare] Wenn ein kausales Modell gerechnet werden soll dann kann dies auf dem gesamten Datensatz geschehen. Das Ziel ist es einen Zusammenhang von $X$ auf $Y$ zu modellieren. Wie wirken sich die Einflussvariablen $Y$ auf die gemessenen Endpunkte $X = x_1, ..., x_p$ aus?
\item [\textbf{D} \msquare] Wir modellieren den Zusammenhang zwischen $X$ und $Y$ wenn ein kausales Modell rerechnet wird. Dabei kann der gesamte Datensatz genutzt werden. Eine Aufteilung wie in einem prädiktiven Modell ist nicht notwendig.
\item [\textbf{E} \msquare] Ein kausales Modell basiert auf einem Traingsdatensatz und einem Testdatensatz. Auf dem Trainingsdatensatz wird das Modell trainiert und auf dem Testdatensatz validiert.
\end{enumerate}

\section{Aufgabe \hfill (2 Punkte)}



Nach einer Regressions sollten die Residuen normalverteilt sein. Was bei einer simplen Regression noch relativ einfach visuell in einem Scatterplot zu überprüfen ist. Für komplexere Modell liefert der QQ-Plot die notwendigen Informationen über die Normalverteilung. Welche Aussage ist richtig?



{\centering \includegraphics[width=\maxwidth]{img/mc-regression-05-a-1} 

}







\begin{enumerate}
\item [\textbf{A} \msquare] Die Annahme der normalverteilten Residuen ist erfüllt. Die Punkte liegen zum überwiegenden Teil nicht auf der Geraden und Korrelation ist negativ.
\item [\textbf{B} \msquare] Wir betrachten die Gerade und dabei insbesondere die beiden Enden der Gerade in dem IQR, also dem ersten und dritten Quartile. Hier sollten die Punkte auf der Geraden liegen, dann ist die Annahme an die Normalverteilung der Residuen erfüllt.
\item [\textbf{C} \msquare] Wir betrachten die Gerade, die durch die einzelnen Punkte laufen sollte. Wenn die 95\% der Punkte von der Geraden getroffen werden, dann gehen wir von normalverteilten Residuen aus.
\item [\textbf{D} \msquare] Wir betrachten die Punkte. Wenn die Punkte einigermaßen gleichmäßig verteilt liegen, dann gehen wir von normalen Residuen aus.
\item [\textbf{E} \msquare] Die Annahme der normalverteilten Residuen ist erfüllt. Die Punkte liegen zum überwiegenden Teil auf der Geraden.
\end{enumerate}

\section{Aufgabe \hfill (2 Punkte)}



Sie rechnen eine linearen Regression und erhalten folgende Abbildung der Residuen (\texttt{.resid}). Welche Aussage ist richtig?



{\centering \includegraphics[width=\maxwidth]{img/mc-regression-06-a-1} 

}







\begin{enumerate}
\item [\textbf{A} \msquare] Die Annahme der normalverteilten Residuen ist nicht erfüllt. Es ist kein Muster zu erkennen.
\item [\textbf{B} \msquare] Wenn wir die Nulllinie betrachten so müssen die Punkte gleichmäßig unter der Nulllinie liegen. Unser Modell erfüllt somit nicht die Annahme von normalverteilten Residuen mit einem Mittelwert von $>0$ und einer Streuung von $s$.
\item [\textbf{C} \msquare] Wir betrachten die Nulllinie und alle Punkte sollten ohne Muster gleichmäßig um die Nulllinie liegen. Da dies der Fal ist, gehen wir von keinen Ausreißern aus.
\item [\textbf{D} \msquare] Die Annahme der normalverteilten Residuen ist erfüllt. Die Punkte liegen zum überwiegenden Teil auf der Diagonalen. Damit ist das Modell erfolgreich geschätzt worden.
\item [\textbf{E} \msquare] Wenn wir die Nulllinie betrachten so liegen die Punkte nicht gleichmäßig über und unter der Nulllinie. Unser Modell erfüllt nicht die Annahme von normalverteilten Residuen mit einem Mittelwert von 0 und einer Streuung von $s^2$.
\end{enumerate}

\section{Aufgabe \hfill (2 Punkte)}




Sie berechnen in Ihgrer Abschlussarbeit den Korrelationskoeffizienten $\rho$. Welche Aussage über den Korrelationskoeffizienten $\rho$ ist richtig?




\begin{enumerate}
\item [\textbf{A} \msquare] Der Korrelationskoeffizienten $\rho$ ist eine standardisierte, statistische Maßzahl, die zwischen -1 und 1 liegt. Dabei ist Korrelationskoeffizienten $\rho$ einheitslos.
\item [\textbf{B} \msquare] Der Korrelationskoeffizienten $\rho$ zeigt keinen Zusammenhang zwischen zwei Variablen $x$ und $y$ bei einem Wert von 0. Einen negativen Zusammenhang Richtung -1 und somit auch einen positiven Zusammenhang Richtung 1. Je größer die Zahl allgemein, desto stärker der Effekt.
\item [\textbf{C} \msquare] Der Korrelationskoeffizienten $\rho$ ist eine veraltete Darstellungsform von Effekten in der linearen Regression und wird wie das $\eta^2$ aus der ANOVA interpretiert. Der Korrelationskoeffizienten $\rho$ beschreibt den Anteil an erklärter Varianz durch die Regression.
\item [\textbf{D} \msquare] Der Korrelationskoeffizienten $\rho$ wird wie das $\eta^2$ aus der ANOVA interpretiert. Der Korrelationskoeffizienten $\rho$ beschreibt den Anteil an erklärter Varianz durch die Regression. Dabei gibt er jedoch eine Richtung an und kann auch negativ werden.
\item [\textbf{E} \msquare] Korrelationskoeffizienten $\rho$ liegt zwischen 0 und 1. Darüber hinaus ist der Korrelationskoeffizienten $\rho$ einheitslos und kann als Standardisierung verstanden werden.
\end{enumerate}

\section{Aufgabe \hfill (2 Punkte)}



Nach einer simplen linearen Regression zur Untersuchung vom Einfluss der $CO_2$-Konzentration in [$\mu g$] im Wasser auf das Trockengewicht von Wasserlinsen in [$kg$] erhalten Sie einen $\beta_{CO_2}$ Koeffizienten von $6.9\times 10^{-7}$ und einen hoch signifikanten $p$-Wert mit $0.00051$. Warum sehen Sie so einen kleinen Effekt bei einer so deutlichen Signifikanz?




\begin{enumerate}
\item [\textbf{A} \msquare] Manchmal ist die Einheit der Einflussvariable $X$ zu groß gewählt, so dass der Ansteig von 1 Einheit in $X$ zu einer zu großen Änderung in $y$ führt. Daher kann der Effekt $\beta_{CO_2}$ sehr klein wirken, da der p-Wert wird auf einer einheitslosen Teststatistik bestimmt wird.
\item [\textbf{B} \msquare] Manchmal ist die Einheit der Einflussvariable $X$ zu klein gewählt, so dass der Ansteig von 1 Einheit in $X$ zu einer zu kleinen Änderung in $y$ führt. Daher kann der Effekt $\beta_{CO_2}$ sehr klein wirken, aber auf einer anderen Einheit sehr viel größer sein. Der p-Wert wird auf einer einheitslosen Teststatistik bestimmt.
\item [\textbf{C} \msquare] Die Fallzahl ist zu klein angesetzt. Je kleiner die Fallzahl ist, desto höher ist die Teststatsitik und damit auch der $p$-Wert kleiner. Wir brauchen also mehr Fallzahl um den geringen Effekt noch signifikant zu krigen.
\item [\textbf{D} \msquare] Die Fallzahl ist zu hoch angesetzt. Je höher die Fallzahl ist, desto kleiner ist die Teststatistik und damit ist dann auch der $p$-Wert sehr klein. Es sollte über eine Reduzierung der Fallzahl nachgedacht werden. Dann sollte der Effekt zum p-Wert passen.
\item [\textbf{E} \msquare] Wenn der Effekt $\beta_{CO_2}$ winzig ist, dann kann es an einer falsch gewählten Einheit liegen. Der Anstieg von einer Einheit in $X$ führt ja zu einer Änderung von $\beta_{CO_2}$ in $x$. Wir müssen daher die Einheit von $y$ entsprechend anpassen.
\end{enumerate}

\section{Aufgabe \hfill (2 Punkte)}



Sie wollen nach der explorativen Datenanalyse (EDA) Ihre Daten in der Abschlussarbeit auswerten. Nach einiger Rechereche finden Sie heraus, dass Sie zuerst die Daten mit der Funktion \texttt{lm()} in \Rlogo modellieren müssen. Welche Anwendung folgt drauf?





\begin{enumerate}
\item [\textbf{A} \msquare] Neben der klassichen Verwendung der Funktion \texttt{lm()} in der linearen Regression kann auch ein Gruppenvergleich gerechnet werden. Dafür müssen aber alle Faktoren aus den Daten entfernt und numerishc umgewandelt werden. Dann kann das R Paket \{emmeans\} genutzt werden um die Korrelation zu berechnen. Eine Adjustierung ist dann nicht mehr notwendig.
\item [\textbf{B} \msquare] Ist die Einflussvariable $X$ ein Faktor so werden die Gruppenmittelwerte geschätzt und eine anschließende ANOVA sowie multipler Gruppenvergleich mit \{emmeans\} ist möglich. Dennoch muss zuerst ein lineares Modell mit der Funktion \texttt{lm()} in \Rlogo gerechnet werden.
\item [\textbf{C} \msquare] Ist die Einflussvariable $X$ numerisch so werden die Gruppenmittelwerte geschätzt und eine anschließende ANOVA sowie multipler Gruppenvergleich mit \{emmeans\} ist möglich.
\item [\textbf{D} \msquare] Die Funktion \texttt{lm()} berechnet die Varianzstruktur für eine ANOVA. Dannach kann dann über eine explorative Datenalayse nochmal eine Signifikanz berechnet werden. Sollte vor der Verwendung der Funktion \texttt{lm()} schon eine EDA gerechnet worden sein, so ist die Analyse wertlos.
\item [\textbf{E} \msquare] Die Funktion \texttt{lm()} in \Rlogo wird klassischerweise für die nicht-lineare Regression genutzt. Ist die Einflussvariable $X$ numerisch so werden die Gruppenmittelwerte geschätzt.
\end{enumerate}

\section{Aufgabe \hfill (2 Punkte)}



Wenn Ihr gemessener Endpunkt nicht einer Normalverteilung folgt, so können Sie dennoch Ihre Daten modellieren. Hierzu nutzen Sie dann das \textit{generalisierte lineare Modell (GLM)}. Welche Aussage ist richtig?




\begin{enumerate}
\item [\textbf{A} \msquare] In \Rlogo ist mit dem \textit{generalisierten linearen Modell (GLM)} eine Modellierung implementiert, die die Poissonverteilung für Zähldaten oder die Binomialverteilung für 0/1-Daten modellieren kann. Weitere Modellierungen sind in \Rlogo auch mit zusätzlich geladenen Paketen nicht möglich.
\item [\textbf{B} \msquare] Das GLM ist ein faktisch maschineller Lernalgorithmus, der selstständig die Verteilungsfamilie für Y wählt.
\item [\textbf{C} \msquare] Das \textit{generalisierte lineare Modell (GLM)} erlaubt auch weitere Verteilungsfamilien für das $Y$ bzw. das Outcome in einer linearen Regression zu wählen.
\item [\textbf{D} \msquare] Das GLM ist eine Vereinfachung des LM in R. Mit dem GLM lassen sich polygonale Regressionen rechnen. Somit stehen neben der Normalverteilung noch weitere Verteilungen zu Verfügung.
\item [\textbf{E} \msquare] Das GLM erlaubt auch nicht normalverteilte Residuen in der Schätzung der Regressionsgrade.
\end{enumerate}
\section*{Vermischte Themen}  

\section{Aufgabe \hfill (2 Punkte)}

Die Randomisierung von Beobachtungen zu den Versuchseinheiten
ist bedeutend in der Versuchsplanung. Welche der folgenden Aussagen ist richtig?



\begin{enumerate}
\item [\textbf{A} \msquare] Randomisierung erlaubt erst die Mittelwerte zu schätzen. Ohne Randomisierung keine Mittelwerte. Ohne Mittelwerte keine Varianz und somit auch kein statistischer Test.
\item [\textbf{B} \msquare] Randomisierung ist die direkte Folge von Strukturgleichheit. Die Strukturgleichheit erlaubt es erst von der Stichprobe auf die Grundgesamtheit zurückzuschliessen.
\item [\textbf{C} \msquare] Randomisierung erlaubt erst die Varianzen zu schätzen. Ohne eine Randomisierung ist die Berechnung von Mittelwerten und Varianzen nicht möglich. Dadurch lässt sich erst ein Experiment auswerten.
\item [\textbf{D} \msquare] Strukturgleichheit ist durch Randomisierung gegeben. Leider hilft die Randomisierung noch nicht um von der Stichprobe auf die Grundgesamtheit zu schließen. Deshalb wurde das Falsifikationsprinzip entwickelt.
\item [\textbf{E} \msquare] Randomisierung sorgt für Strukturgleichheit und erlaubt erst von der Stichprobe auf die Grundgesamtheit zurückzuschliessen.
\end{enumerate}

\section{Aufgabe \hfill (2 Punkte)}



Viele wissenschaftliche Orginalquellen sind in Englisch verfasst. Jetzt finden Sie heraus, dass auch \Rlogo nur in englischer Sprache funktioniert. Warum ist das so?



\begin{enumerate}
\item [\textbf{A} \msquare] Programmiersprachen können nur englische Begriffe verarbeiten. Zusätzliche Pakete können zwar geladen werden, aber meist funktionieren diese Pakete nicht richtig. Deutsch ist International nicht bedeutend genug.
\item [\textbf{B} \msquare] Im Allgemeinen haben Programmiersprachen Probleme mit Umlauten und Sonderzeichen, die in der deutschen Sprache vorkommen. Eine Nutzung der englischen Sprache umgeht dieses Problem auf einfache Art.
\item [\textbf{C} \msquare] Es gibt keinen Grund nicht auch deutsche Wörter zu verwenden. Es ist ein Stilmittel.
\item [\textbf{D} \msquare] Die Spracherkennung von \Rlogo ist nicht in der Lage Deutsch zu verstehen.
\item [\textbf{E} \msquare] Alle Funktionen und auch Anwendungen sind in \Rlogo in englischer Sprache. Die Nutzung von deutschen Wörtern ist nicht schick und das ist zu vermeiden.
\end{enumerate}

\section{Aufgabe \hfill (2 Punkte)}



Nachdem Sie Ihr Feldexperiment als Vorversuch für Ihre Abschlussarbeit abgeschlossen haben, wollen Sie in einer explorativen Datenanalyse (EDA) in \Rlogo einmal schauen, ob Sie überhaupt Effekte der Behandlung vorliegen haben. Welche Reihenfolge von Schritten müssen Sie in \Rlogo durchführen, damit Sie eine EDA rechnen können?



\begin{enumerate}
\item [\textbf{A} \msquare] Für eine explorativen Datenanalyse (EDA) in \Rlogo müssen wir als erstes die Daten über \texttt{read\_excel()} einlesen. Danach müssen wir schauen, dass wir die Spalten richtig über \texttt{mutate()} transformiert haben. Insbesondere müssen Variablen mit Kategorien in einen Faktor umgewandelt werden. Am Ende nutzen wir die Funktion \text{ggplot()} für die eigentlich EDA.
\item [\textbf{B} \msquare] Für eine explorativen Datenanalyse (EDA) in \Rlogo müssen wir als erstes die Daten über \texttt{read\_excel()} einlesen. Danach müssen wir schauen, dass wir die Zeilen richtig über \texttt{mutate()} transformiert haben. Insbesondere müssen Variablen mit kontinuierlichen Werten in einen Faktor umgewandelt werden. Am Ende nutzen wir die Funktion \text{ggplot()} für die eigentlich EDA.
\item [\textbf{C} \msquare] Wir lesen die Daten ein und mutieren die Daten. Dabei ist wichtig, dass wir nicht das Paket \texttt{tidyverse} nutzen, da dieses Paket veraltet ist. über die Funktion \texttt{library(tidyverse)} entfernen wir das Paket von der Analyse.
\item [\textbf{D} \msquare] Wir transformieren die Spalten über \texttt{mutate()} in ein \texttt{tibble} und können dann über \text{ggplot()} uns die Abbildungen erstellen lassen. Dabei beachten wir das wir keine Faktoren in den Daten haben.
\item [\textbf{E} \msquare] Wir lesen die Daten über eine generische Funktion \texttt{read()} ein und müssen dann die Funktion \texttt{ggplot()} nur noch installieren. Dann haben wir die Abbildungen als \texttt{*.png} vorliegen.
\end{enumerate}

\section{Aufgabe \hfill (2 Punkte)}



Gegeben ist das Modell $Y \sim X$. Welche Aussage über $n_1 = n_2$ ist richtig?



\begin{enumerate}
\item [\textbf{A} \msquare] Es liegt Varianzhetrogenität vor.
\item [\textbf{B} \msquare] Es handelt sich um ein balanciertes Design.
\item [\textbf{C} \msquare] Es handelt sich um abhängige Beobachtungen.
\item [\textbf{D} \msquare] Es handelt sich um ein unbalanciertes Design.
\item [\textbf{E} \msquare] Es liegt Varianzhomogenität vor.
\end{enumerate}

\section{Aufgabe \hfill (2 Punkte)}



Im Rahmen Ihrer Abschlussarbeit werten Sie ein Experiment mit Ferkel aus. Es geht um die Leistungssteigerung der Ferkelproduktion. Sie messen jeweils die Gewichtszunahme der Ferkel. Die Ferkel einer Muttersau sind dabei...



\begin{enumerate}
\item [\textbf{A} \msquare] Untereinander unabhängig. Sollten die Mütter verwandt sein, so ist die Varianzstruktur ähnlich und muss modelliert werden.
\item [\textbf{B} \msquare] Untereinander abhängig, wenn die Mütter ebenfalls miteinander verwandt sind. Erst die Abhängigkeit 2. Grades wird in der Statistik modelliert.
\item [\textbf{C} \msquare] Die Ferkel stammen vom gleichen Muttertier und haben vermutlich eine ähnlichere Varianzstruktur als die Ferkel von anderen Sauen. Die Ferkel sind untereinander über die Mutter abhängig.
\item [\textbf{D} \msquare] Die Ferkel stammen von der gleichen Sau und sind somit untereinander unabhängig.
\item [\textbf{E} \msquare] Je nach Stallanlage kommt eine andere Analyse in Betracht. Eine allgemeine Aussage über Ferkel und Sauen lässt sich statistisch nicht treffen.
\end{enumerate}

\section{Aufgabe \hfill (2 Punkte)}



Neben der Mittelwertsdifferenz als Effektschätzer bei normalverteilten Endpunkten wird auch häufig der Effektschätzer Risk ratio bei binären Endpunkten verwendet. Welche Aussage über den Effektschätzer Risk ratio ist im folgenden Beispiel zur Behandlung von Klaueninfektionen bei Schweinen richtig? Dabei sind 5 Tiere krank und 12 Tiere sind gesund.



\begin{enumerate}
\item [\textbf{A} \msquare] Es ergibt sich ein Risk ratio von 0.42, da es sich um ein Anteil handelt.
\item [\textbf{B} \msquare] Da es sich um ein Chancenverhältnis handelt ergibt sich ein Risk ratio von 3.4.
\item [\textbf{C} \msquare] Der Anteil der Kranken wird berechnet. Da es sich um ein Anteil handelt ergibt sich ein Risk ratio von 0.29.
\item [\textbf{D} \msquare] Es ergibt sich ein Risk ratio von 2.4, da es sich um ein Anteil handelt.
\item [\textbf{E} \msquare] Da es sich um ein Chancenverhältnis handelt ergibt sich ein Risk ratio von 0.42.
\end{enumerate}

\section{Aufgabe \hfill (2 Punkte)}



Sie werten in Ihrer Abschlussarbeit einen sehr großen Datensatz aus einer öffentlichen Datenbank aus. Nun stellen Sie fest, dass Sie ein Problem mit der Bewertung Ihrer Ergbnisse anhand der Signifikanz bekommen. Wie Sie herausfinden, scheint dies ein häufiges Problem in der Bio Data Science zu sein. Welche Aussage ist richtig?




\begin{enumerate}
\item [\textbf{A} \msquare] Aktuell werden zu grosse Datensätze für die gänigige Statistik gemessen. Daher wendet man maschinelle Lernverfahren für kausale Modelle an. Hier ist die Relevanz gleich Signifikanz.
\item [\textbf{B} \msquare] Eine große Fallzahl führt zu mehr signifikanten Ergebnissen auch bei kleinen Effekten. Daher werden fast alle Vergleich esignifikant, wenn die Fallzahl nur groß genug wird.
\item [\textbf{C} \msquare] Big Data ist ein Problem der parametrischen Statistik. Parameter lassen sich nur auf kleinen Datensätzen berechnen, da es sich sonst nicht mehr um eine Stichprobe im engen Sinne der Statistik handelt.
\item [\textbf{D} \msquare] Aktuell werden immer größere Datensätze erhoben. Dadurch wird auch die Varianz immer höher was automatisch zu mehr signifikanten Ergebnissen führt.
\item [\textbf{E} \msquare] Relevanz und Signifikanz haben nichts miteinander zu tun. Daher gibt es auch keinen Zusammenhang zwischen hoher Fahlzahl (n > 10000) und einem signifikanten Test. Ein Effekt ist immer relevant und somit signifikant.
\end{enumerate}
\section*{Multiple Gruppenvergleiche}    

\section{Aufgabe \hfill (2 Punkte)}



Sie haben folgende unadjustierten p-Werte gegeben: 0.21, 0.01, 0.42, 0.02 und 0.89. Sie adjustieren die p-Werte nach
Bonferroni. Welche Aussage ist richtig?



\begin{enumerate}
\item [\textbf{A} \msquare] Nach der Bonferroni-Adjustierung ergeben sich die adjustierten p-Werte von 0.042, 0.002, 0.084, 0.004 und 0.178. Die adjustierten p-Werte werden zu einem $\alpha$-Niveau von 1\% verglichen.
\item [\textbf{B} \msquare] Nach der Bonferroni-Adjustierung ergeben sich die adjustierten p-Werte von 0.042, 0.002, 0.084, 0.004 und 0.178. Die adjustierten p-Werte werden zu einem $\alpha$-Niveau von 5\% verglichen.
\item [\textbf{C} \msquare] Nach der Bonferroni-Adjustierung ergeben sich die adjustierten p-Werte von 1, 0.05, 1, 0.1 und 1. Die adjustierten p-Werte werden zu einem $\alpha$-Niveau von 1\% verglichen.
\item [\textbf{D} \msquare] Nach der Bonferroni-Adjustierung ergeben sich die adjustierten p-Werte von 1, 0.05, 1, 0.1 und 1. Die adjustierten p-Werte werden zu einem $\alpha$-Niveau von 5\% verglichen.
\item [\textbf{E} \msquare] Nach der Bonferroni-Adjustierung ergeben sich die adjustierten p-Werte von 1.05, 0.05, 2.1, 0.1 und 4.45. Die adjustierten p-Werte werden zu einem $\alpha$-Niveau von 5\% verglichen.
\end{enumerate}

\section{Aufgabe \hfill (2 Punkte)}



Auf wissenschaftlichen Postern finden Sie unter Abbildungen häufig die Abbkürzung \textit{CLD}. Für welchen statistischen Fachbegriff steht die Abbkürzung und wie interpretieren Sie ein \textit{CLD}?



\begin{enumerate}
\item [\textbf{A} \msquare] Compact letter display. Das CLD ist umstritten, da es die Gleichheit der Behandlungen durch gleiche Buchstaben darstellt. Dadurch ist das CLD nicht mehr sauber auf einer Linie mit dem statistischen Testen. Wir lehnen die Nullhypothese ab und zeigen keine Gleichheit im statistischen Testen.
\item [\textbf{B} \msquare] Compact line display. Gleichheit in den Behandlungen wird durch den gleichen Buchstaben oder Symbol dargestellt. Früher wurden keine Buchstaben sondern eine durchgezogene Linie verwendet. Bei mehr als drei Gruppen funktioniert die Linie aber graphisch nicht mehr.
\item [\textbf{C} \msquare] Contrast letter display. Unterschiede in den Behandlungen werden durch den gleichen Buchstaben oder Symbol dargestellt. Die Interpretation des CLD führt häufig in die Irre.
\item [\textbf{D} \msquare] Compact letter display. Gleiche Buchstaben bedeuten, dass sich die Behandlungen unterscheiden. Daher ist das CLD sehr unintuitiv. Es wäre besser, wenn gleiche Buchstaben Gleichheit anzeigen würden. Dies ist aber leider in der statistischen Testtheorie nicht möglich.
\item [\textbf{E} \msquare] Compound letter display. Gleichheit in dem Outcomes wird durch den gleichen Buchstaben oder Symbol dargestellt. Teilweise ist die Interpretation des Verbunds (eng. compound) herausfordernd, da wir ja nach dem Unterschied suchen.
\end{enumerate}

\section{Aufgabe \hfill (2 Punkte)}




In Ihrer Bachelorarbeit müssen Sie einen Feldversuch auswerten. Nachdem Sie die zweifaktorielle ANOVA gerechnet haben und keine signifikante Interaktion vorliegt, wollen Sie jetzt einen Posthoc-Test rechnen. Welches R Paket nutzen Sie dafür am besten?



\begin{enumerate}
\item [\textbf{A} \msquare] Sie nutzen das R Paket \{emmeans\} für die Berechnung des multipnen Gruppenvergleichs. Die Ausgabe der Funktion emmeans() erlaubt zügig über \{ggplot\} einen Barplot zu erstellen und dann auch das CLD zu berechnen. Sie haben alles sofort zusammen.
\item [\textbf{B} \msquare] Das R Paket \{lm\}. Das Paket \{lm\} erstellt selbstständig Konfidenzintervalle und entsprechende p-Werte. Da wir in dem Paket nicht adjustieren müssen, ist es bei Anwendern sehr beliebt.
\item [\textbf{C} \msquare] Das R Paket \{hmisc\} erlaubt die Durchführung eines multiplen Gruppenvergleichs aus verschiedenen Modellen heraus. Aus einem hmisc Objekt lässt sich recht einfach das CLD erstellen und so über Barplots eine schnelle Interpration der statistischen Auswertung durchführen.
\item [\textbf{D} \msquare] Das R Paket \{ggplot\}. Wir erhalten hier sofort eine Visualisierung der Daten. Anhand der Visualisierung lässt sich eine explorative Datenanalyse durchführen, die gleichwertig zu einem Posthoc-Test ist.
\item [\textbf{E} \msquare] Das R Paket \{emmeans\} erlaubt die Durchführung eines multiplen Gruppenvergleichs. Aus einem emmeans Objekt lässt sich leider kein CLD erstellen. Dennoch ist das Paket einfach zu bedienen und wird deshalb genutzt. Die Interpretation der statistischen Auswertung wird über einen Barplot abgebildet.
\end{enumerate}

\section{Aufgabe \hfill (2 Punkte)}



Bei einem Posthoc-Test kann es zu einer überraschenden Besonderheit beim statistischen Testen kommen. Wie lautet der Fachbegriff und wie kann mit der überraschenden Besonderheit umgegangen werden?



\begin{enumerate}
\item [\textbf{A} \msquare] Das globale Signifikanzniveau liegt nicht mehr bei $5\%$ sondern sehr viel niedriger, bei ca. $1\%$. Es kommt zu einer $\alpha$-Hyperinflation. Dagegen kann mit der Adjustierung der p-Werte nach Bonferroni vorgegangen werden.
\item [\textbf{B} \msquare] Das globale Signifikanzniveau explodiert und erreicht Werte größer als Eins. Es kommt zu einer $\alpha$-Inflation. Dagegen kann mit der Adjustierung der $\alpha$-Werte nach Bonferroni vorgegangen werden.
\item [\textbf{C} \msquare] Beim multiplen Testen kann es zu einer $\alpha$-Inflation kommen. Das globale Signifikanzniveau liegt nicht mehr bei $5\%$ sondern sehr viel höher. Daher müssen die p-Werte entsprechend adjustiert werden. Hierfür gibt es verschiedene Verfahren, wobei das Verfahren zur Adjustierung der p-Werte nach Bonferroni das bekanneste Verfahren ist.
\item [\textbf{D} \msquare] Beim multiplen Testen kann es zu einer $\alpha$-Inflation kommen. Das globale Signifikanzniveau liegt nicht mehr bei $5\%$ sondern weit darunter. Daher müssen die p-Werte entsprechend adjustiert werden. Hierfür gibt es verschiedene Verfahren, wobei das Verfahren zur Adjustierung der p-Werte nach Welch das bekanneste Verfahren ist.
\item [\textbf{E} \msquare] Beim multiplen Testen kann es zu Varianzheterogenität kommen. Das globale Signifikanzniveau liegt nicht mehr bei $5\%$. Daher müssen die p-Werte entsprechend adjustiert werden. Das Verfahren nach Welch, bekannt aus dem t-Test, ist hier häufig anzuwenden.
\end{enumerate}

\section{Aufgabe \hfill (2 Punkte)}




In Ihrer Bachelorarbeit werten Sie einen einfaktoriellen Versuch aus. Dafür rechnen Sie in \Rlogo zunächst eine ANOVA und schließen dann dann einen multiplen vergleich mit t-Tests an. Welche Aussage über die Effekte in Ihrem versuch ist richtig?



\begin{enumerate}
\item [\textbf{A} \msquare] Wenn ein multipler Test gerechnet wird, dann muss der Effekt $\Delta$ nicht adjustiert werden. Bei einem Effekt im multiplen Testen handelt es sich um eine Wahrscheinlichkeit für das Auftreten der Nullhypothese.
\item [\textbf{B} \msquare] Beim multiplen Testen kann es zu einer $\Delta$-Inflation kommen. Das globale Effektniveau liegt nicht mehr bei $20\%$. Daher müssen die Effekte entsprechend adjustiert werden. Hierfür gibt es verschiedene Verfahren, wobei das Verfahren zur Adjustierung der Effekte nach Bonferroni das bekanneste Verfahren ist.
\item [\textbf{C} \msquare] Wenn ein multipler Test gerechnet wird, dann muss der Effekt $\Delta$ nach Bonferroni adjustiert werden. Dafür wird der Effekt mit der Anzahl an Vergleichen $k$ multipliziert. Dies geschiet analog zu den p-Werten.
\item [\textbf{D} \msquare] Beim multiplen Testen kann es zu einer $\Delta$-Deflation kommen. Das globale Relevanzniveau liegt nicht mehr bei $5\%$ sondern weit darunter. Daher müssen die $\Delta$-Werte entsprechend adjustiert werden. Hierfür gibt es verschiedene Verfahren, wobei das Verfahren zur Adjustierung der $\Delta$-Werte nach Bonferroni das bekanneste Verfahren ist. Die $\Delta$-Werte werden durch die Anzahl an Vergleichen geteilt.
\item [\textbf{E} \msquare] Wenn ein multipler Test gerechnet wird, dann muss der Effekt $\Delta$ nicht adjustiert werden im Gegensatz zu den p-Werten.
\end{enumerate}
\section*{Statistische Testtheorie}  

\section{Aufgabe \hfill (2 Punkte)}




Sie haben den mathematischen Ausdruck $Pr(D|H_0)$ vorliegen, welche Aussage ist richtig?



\begin{enumerate}
\item [\textbf{A} \msquare] $Pr(D|H_0)$ ist die Wahrscheinlichkeit nicht die Daten $D$ zu beobachten sondern die Nullhypothese, wenn diese wahr ist.
\item [\textbf{B} \msquare] Die Inverse der Wahrscheinlichkeit unter der die Nullhypothese nicht mehr die Alternativehypothese überdeckt.
\item [\textbf{C} \msquare] $Pr(D|H_0)$ ist die Wahrscheinlichkeit die Daten $D$ zu beobachten, wenn die Nullhypothese wahr ist.
\item [\textbf{D} \msquare] $Pr(D|H_0)$ ist die Wahrscheinlichkeit der Alternativehypothese und somit $1 - Pr(H_A)$
\item [\textbf{E} \msquare] $Pr(D|H_0)$ stellt die Wahrscheinlichkeit die Teststatistik $T$ zu beobachten dar, wenn die Nullhypothese falsch ist.
\end{enumerate}

\section{Aufgabe \hfill (2 Punkte)}



Die Testtheorie hat mehrere Säulen. Einer der Säulen ist das Falsifikationsprinzip. Das Falsifikationsprinzip besagt,



\begin{enumerate}
\item [\textbf{A} \msquare] ... dass ein schlechtes Modell durch ein weniger schlechtes Modell ersetzt wird. Die Wissenschaft lehnt ab und verifiziert nicht.
\item [\textbf{B} \msquare] ... dass ein schlechtes Modell durch ein schlechteres Modell ersetzt wird. Die Wissenschaft lehnt ab und verifiziert nicht.
\item [\textbf{C} \msquare] ... dass ein schlechtes Modell durch das Falsifikationsprinzip durch ein noch schlechteres Modell ersetzt wird. Die Wissenschaft lehnt ab und verifiziert nicht.
\item [\textbf{D} \msquare] ... dass Modelle meist falsch sind und selten richtig.
\item [\textbf{E} \msquare] ... dass in der Wissenschaft immer etwas falsch sein muss. Sonst gebe es keinen Fortschritt.
\end{enumerate}

\section{Aufgabe \hfill (2 Punkte)}



Das Signifikanzniveau $\alpha$ wird auch Fehler 1. Art genannt und liegt bei 5\%. Warum wurde der Grenzwert von 5\% als Signifikanzschwelle gewählt?



\begin{enumerate}
\item [\textbf{A} \msquare] Da Wissenschaftler eine Schwelle für die statistische Testentscheidung benötigen wurde $\alpha$ in einer großen Konferenz 1945 gewählt. Damit ist $\alpha = 5\%$ eine Kulturkonstante mit einem Rank einer Naturkonstante.
\item [\textbf{B} \msquare] Als Kulturkonstante hat $\alpha = 5\%$ den Rang einer Naturkonstante und wurde nach langer Diskussion in der UN im Jahre 1983 festgesetzt. Damals auch schon mit der Zustimmung der UdSSR.
\item [\textbf{C} \msquare] In der Wissenschaft gibt es neben der Naturkonstante, die sich aus der Beobachtung der Welt ergibt, noch die Kulturkonstante, die von einer Gruppe Menschen selbstgewählt wird. Dabei ist $\alpha = 5\%$ eine Kulturkonstante und wurde somit eher zufällig gewählt.
\item [\textbf{D} \msquare] Der Begründer der modernen Statistik, R. Fischer, hat die Grenze simuliert und berechnet. Dadurch ergibt sich dieser optimale Cut-Off.
\item [\textbf{E} \msquare] Der Wert ergab sich aus einer Auswertung von 1042 wissenschaftlichen Veröffentlichungen zwischen 1914 und 1948. Der Wert $5\%$ wurde in $28\%$ der Veröffentlichungen genutzt. Daher legte man sich auf diese Zahl fest.
\end{enumerate}

\section{Aufgabe \hfill (2 Punkte)}

Betrachten wir die Teststatistik aus einem abstrakteren Blickwinkel. Beim
statistischen Testen wird das \textit{"`signal"'} mit dem
\textit{"`noise"'} aus den Daten $D$ zu einer Teststatistik $T_D$ verrechnet. Welche der Formel
berechnet korrekt die Teststatistik $T_D$?



\begin{enumerate}
\item [\textbf{A} \msquare] Es gilt $T_D = \cfrac{noise}{signal}$
\item [\textbf{B} \msquare] Es gilt $T_D = \cfrac{signal}{noise}$
\item [\textbf{C} \msquare] Es gilt $T_D = signal \cdot noise$
\item [\textbf{D} \msquare] Es gilt $T_D = (signal \cdot noise)^2$
\item [\textbf{E} \msquare] Es gilt $T_D = \cfrac{signal}{noise^2}$
\end{enumerate}

%% ------------------------------------------------------------

\section{Aufgabe \hfill (2 Punkte)}



Eine Analogie kann helfen einen Sachverhalt besser zu verstehen. Wie kann folgende Aussage richtig in die Analogie der statistischen Testtheorie gesetzt werden?

\begin{center}
\textit{$H_0$ beibehalten obwohl die $H_0$ falsch ist}
\end{center}



\begin{enumerate}
\item [\textbf{A} \msquare] In die Analogie eines Rauchmelders: \textit{Alarm without fire police}, dem $\alpha$-Fehler.
\item [\textbf{B} \msquare] \textit{Fire without alarm}, dem $\beta$-Fehler als Analogie eines Rauchmelders.
\item [\textbf{C} \msquare] Dem $\beta$-Fehler mit der Analogie eines brennenden Hauses: \textit{Fire without alarm}.
\item [\textbf{D} \msquare] In die Analogie eines Rauchmelders: \textit{Alarm without fire}, dem $\alpha$-Fehler.
\item [\textbf{E} \msquare] \textit{Fire without alarm}, dem $\beta$-Fehler als Analogie von Rauch im Haus.
\end{enumerate}

\section{Aufgabe \hfill (2 Punkte)}



Welche statistische Maßzahl erlaubt es Relevanz mit Signifikanz zu verbinden? Welche Aussage ist richtig?



\begin{enumerate}
\item [\textbf{A} \msquare] Über das Konfidenzintervall. Das Konfidenzinterval inkludiert eine Entscheidung über die Relevanz und zusätzlich kann über die Visualizierung des Konfidenzintervals eine Signifikanzschwelle vom Forschenden definiert werden.
\item [\textbf{B} \msquare] Das OR. Als Chancenverhältnis gibt es das Verhältnis von Relevanz und Signifikanz wieder.
\item [\textbf{C} \msquare] Einem Konfidenzintervall. Das Konfidenzinterval bringt durch eine Visualisierung und zwei Intervallgrenzen die Möglichkeit mit, eine Relevanzschwelle neben der definierten Signifikanzschwelle zu definieren.
\item [\textbf{D} \msquare] Die Teststatistik. Durch den Vergleich von $T_c$ zu $T_k$ ist es m{"o}glich die $H_0$ abzulehnen. Die Relevanz ergibt sich aus der Fläche rechts vom dem $T_c$-Wert.
\item [\textbf{E} \msquare] Der p-Wert. Durch den Vergleich mit $\alpha$ lässt sich über die Signifikanz entscheiden und der $\beta$-Fehler erlaubt über die Power eine Einschätzung der Relevanz.
\end{enumerate}

\section{Aufgabe \hfill (2 Punkte)}



Ein statistischer Test produziert für einen Gruppenvergleich einen $p$-Wert. Welche Aussage zusammen mit dem Signifikanzniveau $\alpha$ gleich 5\% stimmt?



\begin{enumerate}
\item [\textbf{A} \msquare] Wir machen ein Aussage über die Flächen und zwischen den Kurve der Teststatistiken der Hypothesen $H_0$ und $H_A$, wenn die $H_0$ gilt. Dabei werden Wahrscheinlichkeiten vergleichen, die durch die Flächen unter der Kurve repräsentiert werden.
\item [\textbf{B} \msquare] Wir vergleichen mit dem $p$-Wert und dem Signifikanzniveau $\alpha$ absolute Werte auf einem Zahlenstrahl und damit den Unterschied der Teststatistiken, wenn die $H_0$ gilt.
\item [\textbf{C} \msquare] Wir schauen, ob der $p$-Wert größer ist als das Signifikanzniveau $\alpha$ und vergleichen somit Wahrscheinlichkeiten. Die Wahrscheinlichkeiten werden als Flächen unter der Kurve der Teststaistik dargestellt, wenn die $H_A$ gilt.
\item [\textbf{D} \msquare] Wir machen eine Aussage über die indivduelle Wahrscheinlichkeit des Eintretens der Nullhypothese $H_0$. Der $p$-Wert wird mit dem Signifikanzniveau verglichen und bewertet.
\item [\textbf{E} \msquare] Wir vergleichen mit dem $p$-Wert und dem Signifikanzniveau $\alpha$ Wahrscheinlichkeiten und damit die Flächen unter der Kurve der Teststatistik, wenn die $H_0$ gilt.
\end{enumerate}

\section{Aufgabe \hfill (2 Punkte)}



Um die Ergebnisse eines statistischen Tests und die damit verbundene Theorie besser zu verstehen, kann eine Analogie zur Wettervorhersage genutzt werden. Welche Analogie zu der Testtheorie trifft am meisten zu?



\begin{enumerate}
\item [\textbf{A} \msquare] In der Analogie des Niederschlags oder Regenmenge: ein statistischer Test gibt die Stärke eines Effektes wieder. Zum Beispiel, wie hoch ist der Mittelwertsunterschied.
\item [\textbf{B} \msquare] In der Analogie der Regenwahrscheinlichkeit in einem bestimmten Gebiet: ein statistischer Test gibt die Wahrscheinlichkeit für ein Ereignis in einem Experiment mit den Daten $D$ wieder und lässt sich kaum verallgemeinern.
\item [\textbf{C} \msquare] In der Analogie der Durchschnittstemperatur: Wie oft tritt ein Effekt durchschnittlich ein? Wir erhalten eine Wahrscheinlichkeit für die Effekte. Zum Beispiel, wie hoch ist die Wahrscheinlichkeit für einen Mittelwert als Durchschnitt.
\item [\textbf{D} \msquare] In der Analogie der Maximaltemperatur: Was ist der maximale Unterschied zwischen zwei Gruppen. Wir erhalten hier eine Aussage über die Spannweite und den maximalen Effekt.
\item [\textbf{E} \msquare] Die Analogie der Regenwahrscheinlichkeit: der statistische Test erlaubt es die Wahrscheinlichkeit für Regen abzuschätzen jedoch nicht die Menge und somit den Effekt.
\end{enumerate}

\section{Aufgabe \hfill (2 Punkte)}



In Ihrer Forschungsarbeit wollen Sie eine Aussage über ein untersuchtes Individuum treffen. Dazu nutzen Sie eine ANOVA als statistischen Test. Erhalten Sie eine valide Aussage aus einem statistischen Test?



\begin{enumerate}
\item [\textbf{A} \msquare] Weder eine Ausssage über die Population noch über das Individuum ist mit einem statistischen Test möglich. Wir erhalten eine Aussage über ein Experiment.
\item [\textbf{B} \msquare] Nein, wir können ein untersuchtes Individuum nicht mit einer ANOVA auswerten. Wir erhalten keine Aussage zum Individuum. Wir können aber den Test adjustieren und so die Auswertung ermöglichen.
\item [\textbf{C} \msquare] Nein, wir können ein untersuchtes Individuum nicht mit einer ANOVA auswerten. Wir erhalten keine Aussage zum Individuum.
\item [\textbf{D} \msquare] Nein, wir erhalten nur eine Aussage zu zwei Individuen. Ein statistischer Test liefert Informationen zu einem Individuum im Vergleich zu einem anderen Individuum.
\item [\textbf{E} \msquare] Ja, es ist möglich ein untersuchtes Individuum mit einem t-Test auszuwerten. Wir erhalten dann eine Aussage zum Individuum.
\end{enumerate}

\section{Aufgabe \hfill (2 Punkte)}



In der statistischen Testtheorie gibt es den Begriff \textit{Power}. Was sagt der statistische Begriff \textit{Power} aus?



\begin{enumerate}
\item [\textbf{A} \msquare] Die Power $1-\beta$ wird auf 80\% gesetzt. Damit liegt die Wahrscheinlichkeit für die $H_0$ bei 20\%.
\item [\textbf{B} \msquare] Es gilt $\alpha + \beta = 1$ und somit liegt $\beta$ meist bei 95\%.
\item [\textbf{C} \msquare] Die Power beschreibt die Wahrscheinlichkeit die $H_A$ abzulehnen. Wir testen die Power jedoch nicht.
\item [\textbf{D} \msquare] Die Power wird nicht berechnet sondern ist eine Eigenschaft des Tests. Die Power wird auf $80\%$ gesetzt und beschreibt mit welcher Wahrscheinlichkeit $H_A$ \textit{bewiesen wird}
\item [\textbf{E} \msquare] Alle statistischen Tests sind so konstruiert, dass die $H_A$ mit 20\% \textit{bewiesen wird}. Die Power ist $1-\beta$ mit $\beta$ gleich 80\% gesetzt.
\end{enumerate}

\section{Aufgabe \hfill (2 Punkte)}



Welche Aussage über den Effekt eines statistischen Tests ist richtig?



\begin{enumerate}
\item [\textbf{A} \msquare] Der Effekt eines statistischen Tests beschreibt die biologisch interpretierbare Ausgabe eines Tests. Moderen Algorithmen liefern keine Effekte mehr sondern nur noch bedingte Wahrscheinlichkeiten. Der Effekt spielt in der modernen Statistik keine Rollen mehr.
\item [\textbf{B} \msquare] Der Effekt eines statistischen Tests beschreibt den Output oder die Wiedergabe eines Tests in einem Computer.
\item [\textbf{C} \msquare] Durch den Effekt erfahren wir die statistische interpretierbare Ausgabe eines statistischen Tests. Zum Beispiel das $\eta^2$ aus einer ANOVA. Damit können wir die Signifikanz direkt mit dem Effekt verbinden. Am Ende muss der Forschende aber entscheiden, ob der Effekt entsprechend seinen Erwartungen als bedeutet zu bewerten ist.
\item [\textbf{D} \msquare] Der Effekt eines statistischen Tests beschreibt die mathematisch interpretierbare Ausgabe eines Tests. Damit ist der Effekt direkt mit dem Begriff der Signifikanz verbunden. Die Entscheidung über die Signifikanz trifft der Forschende unabhängig von der Relevanz eines statistsichen Tests.
\item [\textbf{E} \msquare] Durch den Effekt erfahren wir die biologisch interpretierbare Ausgabe eines statistischen Tests. Zum Beispiel das $\eta^2$ aus einer ANOVA. Damit können wir die Relevanz direkt mit dem Effekt verbinden. Am Ende muss der Forschende aber entscheiden, ob der Effekt entsprechend seinen Erwartungen als bedeutet zu bewerten ist.
\end{enumerate}

\section{Aufgabe \hfill (2 Punkte)}



Welche Aussage über die Entscheidung anhand des 95\%-Konfidenzintervalls gegen die
Nullhypothese ist richtig?



\begin{enumerate}
\item [\textbf{A} \msquare] Anhand des 95\%-Konfidenzintervalls lässt sich wie folgt eine Entscheidung treffen. Liegt der Wert in dem Signifikanzniveauintervall $\alpha$ dann kann die Nullhypothese abgelehnt werden.
\item [\textbf{B} \msquare] Ist in dem 95\%-Konfidenzintervall nicht die Null enthalten dann wird die Nullhypothese $H_0$ abgelehnt.
\item [\textbf{C} \msquare] Ist $T_{D}$ h{"o}her als der kritische Wert $T_{\alpha = 5\%}$ dann wird die Nullhypothese $H_0$ abgelehnt.
\item [\textbf{D} \msquare] Anhand des 95\%-Konfidenzintervalls lässt sich wie folgt eine Entscheidung treffen. Liegt der Wert über oder gleich dem Signifikanzniveau $\alpha$ dann kann die Nullhypothese abgelehnt werden.
\item [\textbf{E} \msquare] Ist $Pr(D|H_0)$ kleiner als das Signifikanzniveau $\alpha$ gleich $5\%$ dann wird die Nullhypothese $H_0$ abgelehnt.
\end{enumerate}

\section{Aufgabe \hfill (2 Punkte)}



In Ihrer Abschlussarbeit müssen Sie für die statistischen Tests im Anhang Ihrer Arbeit die Hypothesen $H$ formulieren. Welche Aussage über Hypothesen $H$ ist richtig



\begin{enumerate}
\item [\textbf{A} \msquare] Es gibt - bedingt durch das das Falsifikationsprinzip - ein Set von $k$ Nullhypothesen, die iterative gegen $k-1$ Alternativhypothesen getestet werden.
\item [\textbf{B} \msquare] Es gibt ein statistisches Hypothesenpaar mit der Nullhypothese $H_0$ und der Alternativehypothese $H_A$ oder $H_1$.
\item [\textbf{C} \msquare] Die Hypothesen $H_0$ und $H_A$ sind rein prosarischer Natur und bilden keinen mathematischen Hintergrund ab. In der Statistik wird die wissenschaftliche Fragestellung getestet. Daher stehen auch die verständlichen Hypothesen im Mittelpunkt der biologischen Interpretation.
\item [\textbf{D} \msquare] Es gibt ein Hypothesenset bestehend aus $k$ Hypothesen. Meistens wird die Nullhypothese $H_0$ und die Alternativhypothese $H_A$ verwendet. Wegen des Falsifikationsprinzips ist es wichtig, die bekannte falsche und unbekannte richtige Hypothese mit in das Set zu nehmen.
\item [\textbf{E} \msquare] Es gibt ein statistisches Hypothesenpaar mit der Hypothese für und gegen die wissenschaftliche Fragestellung. Die Hypothesen werden $H_{pro}$ und $H_{contra}$ bezeichnet.
\end{enumerate}
\section*{Statistische Tests für Gruppenvergleiche} 

\section{Aufgabe \hfill (2 Punkte)}



In Ihrer Abschlussarbeit rechnen Sie einen Student t-Test. Welche Aussage ist auch für den Welch t-Test richtig?



\begin{enumerate}
\item [\textbf{A} \msquare] Der t-Test vergleicht die Varianzen von mindestens zwei oder mehr Gruppen
\item [\textbf{B} \msquare] Der t-Test berechnet die Differenz von zwei Mittelwerten als Effekt und gibt eine Entscheidung, ob sich die beiden Mittelwerte in den Gruppen signifikant unterscheiden.
\item [\textbf{C} \msquare] Der t-Test vergleicht zwei oder mehr Gruppen indem die Mittelwerte miteinander verglichen werden.
\item [\textbf{D} \msquare] Der t-Test vergleicht die Mittelwerte von zwei Gruppen unter der strikten Annahme von Varianzhomogenität. Sollte keine Varianzhomogenität vorliegen, so gibt es keine Möglichkeit den t-Test in einer Variante anzuwenden.
\item [\textbf{E} \msquare] Der t-Test testet generell zu einem erhöhten $\alpha$-Niveau von 20\%.
\end{enumerate}

\section{Aufgabe \hfill (2 Punkte)}



In einer Studie zur Bewertung der Wirkung des Mikronährstoff Nitrat auf den Ertrag in t/ha  von Mais im Vergleich zu einer Kontrolle entstand folgende Abbildung. Der Versuch wurde in 9 Parzellen pro Gruppe durchgeführt. Welche Aussage ist im Bezug auf einen t-Test ist richtig?



{\centering \includegraphics[width=\maxwidth]{img/mc-testing-ttest-02-1} 

}







\begin{enumerate}
\item [\textbf{A} \msquare] Nach Betrachtung des Barplots liegt kein signifikanter Unterschied vor. Der Effekt kann nicht bei einem t-Test aus Barplots bestimmt werden.
\item [\textbf{B} \msquare] Der Test deutet auf keinen signifikanten Unterschied hin. Der Effekt liegt vermutlich bei 7.
\item [\textbf{C} \msquare] Es liegt ein signifikanter Unterschied vor. Der Effekt liegt bei 0.7.
\item [\textbf{D} \msquare] Die Barplots deuten auf ein signifikanten Unterschied. Der Effekt liegt vermutlich bei 7.
\item [\textbf{E} \msquare] Die Barplots deuten auf einen signifikanten Unterschied. Der Effekt liegt vermutlich bei 7. Wir müssen aber einen Posthoc-Test rechnen um den Effekt wirklich bestimmen zu können.
\end{enumerate}

\section{Aufgabe \hfill (2 Punkte)}




Sie rechnen einen gepaarten t-Test, da Ihre Beobachtungen verbunden sind. Welche der folgenden Aussagen ist richtig?



\begin{enumerate}
\item [\textbf{A} \msquare] Beim gepaarten t-Test kombinieren wir die Vorteile des Student t-Test für Varianzhomogenität mit den Vorteilen des Welch t-Test für Varianzheterogenität. Wir bilden dafür die Differenz der Einzelbeobachtungen.
\item [\textbf{B} \msquare] Wenn die Beobachtungen nicht unabhängig voneinander sind, rechnen wir einen gepaarten t-Test. Messen wir wiederholt an dem gleichen Tier oder Pflanze dann bilden wir die Differenz zwischen den zwei Messpunkten.
\item [\textbf{C} \msquare] Der gepaarte t-Test wird gerechnet, wenn die Beobachtungen abhängig voneinander sind. Wir messen jede Beobachtung nur einmal und berechnen dann die Differenz zu dem Mittel der anderen Beobachtungen.
\item [\textbf{D} \msquare] Der gepaarte t-Test nutzt die Varianz der Beobachtungen jeweils paarweise und bildet dafür eine verbundene Stichprobe. Dieser Datensatz $d$ dient dann zur Differenzbildung.
\item [\textbf{E} \msquare] Der gepaarte t-Test wird genutzt, wenn die Differenzen der Beobachtungen verbunden sind und wir dadurch die Unabhäängigkeit nicht mehr vorliegen haben.
\end{enumerate}

\section{Aufgabe \hfill (2 Punkte)}



In Ihrer Abschlussarbeit passen die Ergebnisse einer ANOVA und eines multiplen Vergleiches nicht zusammen. Nach einem Experiment mit fünf Maissorten ergibt eine ANOVA ($p = 0.045$). Sie führen anschließend die paarweisen t-Tests für alle Vergleiche durch. Nach der Adjustierung für multiples Testen ist kein p-Wert unter der $\alpha$-Schwelle. Sie schauen sich auch die rohen, unadjustierten p-Werte an und finden hier als niedrigsten p-Wert $p_{3-2} = 0.052$. Welche Aussage ist richtig?




\begin{enumerate}
\item [\textbf{A} \msquare] Es gibt einen Fehler in der Varianzstruktur. Daher kann die ANOVA nicht richtig sein und paarweise t-Tests liefern das richtige Ergebnis.
\item [\textbf{B} \msquare] Der Fehler liegt in den t-Tests. Wenn eine ANOVA signifikant ist, dann muss zwangsweise auch ein t-Test signifikant sein.
\item [\textbf{C} \msquare] Das Beispiel kann so nicht auftreten, da die ANOVA und die t-Tests algorithmisch miteinander verschränkt sind.
\item [\textbf{D} \msquare] Die ANOVA testet auf der gesamten Fallzahl. Die einzelnen t-Tests immer nur auf einer kleineren Subgruppe. Da mit weniger Fallzahl weniger signifikante Ergebnisse zu erwarten sind, kann eine Diskrepenz zwischen der ANOVA und den paarweisen t-Tests auftreten.
\item [\textbf{E} \msquare] Die adjustierten p-Werte deuten in die richtige Richtung. Zusammen mit den nicht signifikanten rohen p-Werten ist von einem Fehler in der ANOVA auszugehen.
\end{enumerate}
    
% -----------------------------------------------------------------------
\clearpage
% -----------------------------------------------------------------------
\part{Deskriptive Statistik \& Explorative Datenanalyse}
% -----------------------------------------------------------------------

\section{Aufgabe \hfill (8 Punkte)}

\textit{Geben Sie grundsätzlich Formeln und Rechenweg zur Lösung der Teilaufgaben mit an!} \\[1Ex]
 

 
%% --------------------------------------------------------------------
\begin{minipage}[t]{0.5\textwidth}
\includegraphics[width = 1.3cm]{/Users/kruppajo/work/GitHub/exam/avatare/Paula.png}
\end{minipage}
\begin{minipage}[t]{0.5\textwidth}
\hfill
\href{https://youtu.be/t0WYa_LVc5o}{\includegraphics[width = 2cm]{img/youtube}}
\end{minipage}
\vspace{-3ex}
%% --------------------------------------------------------------------



\paragraph{Zerforschen des Barplots}

Paula steht vor einem ersten Problem, denn wenn es nach ihrer Betreuerin geht, soll sie in einem einem Freilandversuch Kartoffeln auswertet. Soweit eigentlich alles passend. Besser wäre was anderes gewesen. Harry Potter. Ein wunderbares Hobby um sich drin zu verlieren und Abstand zu bekommen. Paula denkt gerne über Harry Potter nach. Das heißt erstmal überlegen für Paula. Paula schmeißt noch eine Handvoll Smarties in ihren Rachen. Im Hintergrund klirrt leise der Spiegel zum Sound von White Lies. Die Behandlung werden verschiedene Lichtstufen ($none$, $200lm$ und $600lm$) sein. In ihrer Exceldatei wird sie den Messwert ($Y$) \textit{Proteingehalt} als \textit{protein} aufnehmen. Vorab soll Paula aber eimal die folgenden Barplots ihrer Betreuerin nachbauen, damit sie den \Rlogo Code schonmal für später vorliegen hat. Damit geht das Problem schon los. Eine echte Herausforderung für sie war schon immer der Perfektionismus gewesen. Ein leidiges Lied.



{\centering \includegraphics[width=\maxwidth]{img/barplot-02-1} 

}




Leider kennt sich Paula mit der Erstellung von Barplots in \Rlogo nicht aus. Deshalb braucht sie bei der Visualisierung Ihre Hilfe!

\begin{enumerate}
\item Formulieren Sie die wissenschaftliche Fragestellung! \textbf{(1 Punkt)}
\item Erstellen Sie eine Tabelle mit den statistischen Maßzahlen der drei Barplots! \textit{Beachten Sie die korrekte Darstellungsform der statistischen Maßzahlen!} \textbf{(3 Punkte)}
\item Erstellen Sie einen beispielhaften Datensatz im \Rlogo üblichen Format, aus dem die drei Barplots \textit{möglicherweise} erstellt wurden! \textbf{(2 Punkte)}
\item Kann Paula einen Unterschied zwischen den Behandlungen erwarten? Begründen Sie Ihre Antwort! \textbf{(2 Punkte)}
\end{enumerate} 
\clearpage
% -----------------------------------------------------------------------

\section{Aufgabe \hfill (8 Punkte)}

\textit{Geben Sie grundsätzlich Formeln und Rechenweg zur Lösung der Teilaufgaben mit an!} \\[1Ex]
 

 
%% --------------------------------------------------------------------
\begin{minipage}[t]{0.5\textwidth}
\includegraphics[width = 1.3cm]{/Users/kruppajo/work/GitHub/exam/avatare/Yuki.png}
\end{minipage}
\begin{minipage}[t]{0.5\textwidth}
\hfill
\href{https://youtu.be/vXnLttRL_VI}{\includegraphics[width = 2cm]{img/youtube}}
\end{minipage}
\vspace{-3ex}
%% --------------------------------------------------------------------



\paragraph{Visualisierung des Barplots}


Yuki steht vor einem ersten Problem, denn wenn es nach seiner Betreuerin geht, soll er in einem einer Klimakammer Lauch auswertet. Soweit eigentlich alles passend. Besser wäre was anderes gewesen. Am Ende dann doch besser Orchideen. Wunderbar. Eine echte Ablenkung für Yuki. Die Behandlung waren verschiedene Lüftungssysteme und Folientunnel ($ctrl$, $storm$ und $tornado$). In seiner Exceldatei hat er den Messwert ($Y$) \textit{Frischegewicht} als \textit{freshmatter} aufgenommen. Nun soll Yuki die Daten eimal als Barplots in einer Präsentation visualisieren, damit seiner Betreuerin wieder klar wird, was er eigentlich nochmal gemacht hat und was für ein Ergbnis in einem statistischen Test zu erwarten wäre. Wäre da nicht noch etwas. Eine echte Herausforderung für ihn war schon immer die Faulheit gewesen. Ein leidiges Lied. Aber egal. Um zu Boldern geht Yuki dann später nochmal raus. Echte Entspannung.

\begin{table}[!h]
\centering
\begin{tabular}{cc}
\toprule
treatment & freshmatter\\
\midrule
tornado & 29.2\\
ctrl & 39.7\\
ctrl & 31.4\\
tornado & 40.5\\
storm & 53.2\\
\addlinespace
ctrl & 34.8\\
ctrl & 40.1\\
storm & 43.9\\
tornado & 46.5\\
storm & 39.0\\
\addlinespace
tornado & 42.4\\
storm & 38.5\\
\bottomrule
\end{tabular}
\end{table}



Leider kennt sich Yuki mit der Erstellung von Barplots nicht aus. Deshalb braucht er bei der Visualisierung Ihre Hilfe!

\begin{enumerate}
\item Formulieren Sie die wissenschaftliche Fragestellung! \textbf{(1 Punkt)}
\item Zeichnen Sie in \textit{einer} Abbildung die Barplots für die Behandlung von Lauch! Beschriften Sie die Achsen entsprechend!\textbf{(4 Punkte)}
\item Beschriften Sie \textit{einen} Barplot mit den gängigen statistischen Maßzahlen! \textbf{(2 Punkte)}
\item Wenn Yuki \textit{keinen Effekt} zwischen den Behandlungen von Lauch erwarten würde, wie sehen dann die Barplots aus? \textit{Antworten Sie mit einer Skizze der Barplots!}
  \textbf{(1 Punkt)}
\end{enumerate} 
\clearpage
% -----------------------------------------------------------------------

\section{Aufgabe \hfill (9 Punkte)}

\textit{Geben Sie grundsätzlich Formeln und Rechenweg zur Lösung der Teilaufgaben mit an!} \\[1Ex]
 

 
%% --------------------------------------------------------------------
\begin{minipage}[t]{0.5\textwidth}
\includegraphics[width = 1.3cm]{/Users/kruppajo/work/GitHub/exam/avatare/Yuki.png}
\end{minipage}
\begin{minipage}[t]{0.5\textwidth}
\hfill
\href{https://youtu.be/Xf0yE-o7bEU}{\includegraphics[width = 2cm]{img/youtube}}
\end{minipage}
\vspace{-3ex}
%% --------------------------------------------------------------------



\paragraph{Zerforschen des Boxplots}

Eine echte Herausforderung für sie war schon immer die Faulheit gewesen. Ein leidiges Lied. Deshalb gilt anschauen, was andere vor einem gemacht haben. Für Yuki ist es eine Möglichkeit schneller ans Ziel zu gelangen. Yuki soll in ihrem Projektbericht Lauch untersuchen. Die Behandlung in ihrem Projektbericht werden verschiedene Genotypen ($AA$, $AB$ und $BB$) sein. Erheben wird Yuki als Endpunkt ($Y$) \textit{Trockengewicht} benannt als \textit{drymatter} in ihrer Exceldatei. Von ihrer Betreuerin erhält sie nun folgende Abbildung von Boxplots, die sie erstmal zur Übung nachbauen soll, bevor sie mit dem eigentlichen Versuch beginnt. Aber nur in passender Atmospäre! Schon dutzende Male gesehen: Matrix. Aber immer noch großartig zusammen mit Reese's Peanut Butter Cups.



{\centering \includegraphics[width=\maxwidth]{img/boxplot-02-zer-1} 

}




Leider kennt sich Yuki mit der Erstellung von Boxplots in \Rlogo nicht aus. Deshalb braucht sie bei der Visualisierung Ihre Hilfe!

\begin{enumerate}
\item Erstellen Sie eine Tabelle mit den statistischen Maßzahlen aus der obigen Abbildung der drei Boxplots! \textit{Beachten Sie die korrekte Darstellungsform der statistischen Maßzahlen!} \textbf{(3 Punkte)}
\item Beschriften Sie \textit{einen} der Boxplots mit den gängigen statistischen Maßzahlen! \textbf{(2 Punkte)}
\item Erstellen Sie einen beispielhaften Datensatz, aus dem die drei Boxplots \textit{möglicherweise} erstellt wurden, im \Rlogo üblichen Format! \textbf{(2 Punkte)}
\item Kann Yuki einen Unterschied zwischen den Behandlungen erwarten? Begründen Sie Ihre Antwort! \textbf{(2 Punkte)}
\end{enumerate} 
\clearpage
% -----------------------------------------------------------------------

\section{Aufgabe \hfill (9 Punkte)}

\textit{Geben Sie grundsätzlich Formeln und Rechenweg zur Lösung der Teilaufgaben mit an!} \\[1Ex]
 

 
%% --------------------------------------------------------------------
\begin{minipage}[t]{0.5\textwidth}
\includegraphics[width = 1.3cm]{/Users/kruppajo/work/GitHub/exam/avatare/Steffen.png}
\end{minipage}
\begin{minipage}[t]{0.5\textwidth}
\hfill
\href{https://youtu.be/0xc0jIPeiyw}{\includegraphics[width = 2cm]{img/youtube}}
\end{minipage}
\vspace{-3ex}
%% --------------------------------------------------------------------



\paragraph{Visualisierung des Boxplots}

Boxplots sind bedeutend in der Darstellung von wissenschaftlichen Ergebnissen. Leider hat sich Steffen nicht gemerkt, welche statistischen Maßzahlen für einen Boxplot erhoben werden müssen. Besser wäre was anderes gewesen. Steffen liebt Klemmbausteine. Darin kann er sich wirklich verlieren und immer wieder neu begeistern. Das ist in soweit doof, da nach seinem Betreuer nun Boxplots aus seinen Daten gebaut werden sollen, bevor es mit dem statistischen Testen weitergeht. Anhand von Boxplots lässt sich eine Aussage über die Normalverteilung von $Y$ treffen. Die Behandlung für Brokoli waren verschiedene Substrattypen ($torf$ und $70p30n$). Erfasst wurde von Steffen als Endpunkt ($Y$) \textit{Trockengewicht}. Steffen hat dann \textit{drymatter} in seiner Exceldatei eintragen. Aber nur in passender Atmospäre! Hm, lecker Oreos und dazu dann im Hintergrund Harry Potter laufen lassen.

\begin{table}[!h]
\centering
\begin{tabular}{cc}
\toprule
treatment & drymatter\\
\midrule
torf & 25.9\\
torf & 22.8\\
torf & 31.4\\
70p30n & 26.6\\
torf & 30.5\\
\addlinespace
torf & 22.2\\
torf & 16.2\\
70p30n & 25.0\\
torf & 25.5\\
70p30n & 32.7\\
\addlinespace
torf & 24.4\\
70p30n & 27.7\\
70p30n & 27.7\\
70p30n & 24.3\\
torf & 28.6\\
\addlinespace
torf & 21.3\\
70p30n & 43.4\\
\bottomrule
\end{tabular}
\end{table}



Leider kennt sich Steffen mit der Erstellung von Boxplots nicht aus. Deshalb braucht er bei der Visualisierung Ihre Hilfe!

\begin{enumerate}
\item Zeichnen Sie in \textit{einer} Abbildung die beiden Boxplots für die zwei Behandlungen von Brokoli! Beschriften Sie die Achsen entsprechend! \textbf{(5 Punkte)} 
\item Wie ist Ihr Vorgehen, wenn Sie eine \textit{gerade} Anzahl an
  Beobachtungen pro Gruppe haben? \textbf{(1 Punkt)}
\item Beschriften Sie \textit{einen} der beiden Boxplots mit den gängigen
  statistischen Maßzahlen! \textbf{(2 Punkte)}
\item Wenn Sie \textit{keinen Effekt} zwischen den Behandlungen von
  Brokoli erwarten würden, wie sehen dann die beiden Boxplots aus?
  \textit{Antworten Sie mit einer Skizze der Boxplots!}
  \textbf{(1 Punkt)}
\end{enumerate} 
\clearpage
% -----------------------------------------------------------------------

\section{Aufgabe \hfill (8 Punkte)}

\textit{Geben Sie grundsätzlich Formeln und Rechenweg zur Lösung der Teilaufgaben mit an!} \\[1Ex]
 

 
%% --------------------------------------------------------------------
\begin{minipage}[t]{0.5\textwidth}
\includegraphics[width = 1.3cm]{/Users/kruppajo/work/GitHub/exam/avatare/Yuki.png}
\end{minipage}
\begin{minipage}[t]{0.5\textwidth}
\hfill
\href{https://youtu.be/aXvxGC4YLqk}{\includegraphics[width = 2cm]{img/youtube}}
\end{minipage}
\vspace{-3ex}
%% --------------------------------------------------------------------



\paragraph{Visualisierung des Histogramm für kategoriale Daten}

In ihrer Abschlussarbeit möchte Yuki gerne die Daten aus einem Versuch in einer Klimakammer mit Kartoffeln in einem Histogramm darstellen. Das Histogramm erlaubt ihr dabei Rückschlüsse auf die Verteilung über den Endpunkt ($Y$) zu treffen. Yuki schmeißt noch eine Handvoll Reese's Peanut Butter Cups in ihren Rachen. Im Hintergrund klirrt leise der Spiegel zum Sound von London Grammar. In seinem Experiment hat Yuki die Läsionen auf den Blättern gezählt. Es wäre einfacher, wenn da nicht noch was wäre. Yuki und die Faulheit, eine unendliche Geschichte mit kniffeligen Wendungen. Wenn London Grammar ertönt, dann sucht das Minischwein schleunigst Schutz unter dem Sofa. Yuki schüttelt den Kopf.

\begin{center}
Die Läsionen auf den Blättern: 9, 2, 4, 5, 5, 3, 1, 3, 6, 3, 3, 2, 1, 4, 6, 2, 5, 6, 3, 3, 1, 1, 4, 3, 4, 2, 3, 3, 6, 6, 3, 2, 6
\end{center}

Leider kennt sich Yuki mit der Erstellung von Histogrammen überhaupt nicht aus. Deshalb braucht sie bei der Erstellung Ihre Hilfe!

\begin{enumerate}
\item Zeichen Sie ein Histogramm um die Verteilung der Daten zu visualisieren! (\textbf{3 Punkte})
\item Beschriften Sie die Achsen der Abbildung! (\textbf{2 Punkte})
\item Ergänzen Sie die absoluten und relativen Häufigkeiten in der
  Abbildung! \textbf{(1 Punkt)}
\item Berechnen Sie aus den Daten die \textit{Wahrscheinlichkeit}
  mehr als die Anzahl 4 zu beobachten! \textbf{(1
    Punkt)}
\item Berechnen Sie aus den Daten die \textit{Chance} mehr
  als die Anzahl 4 zu beobachten! \textbf{(1 Punkt)}
\end{enumerate}

 
\clearpage
% -----------------------------------------------------------------------

\section{Aufgabe \hfill (8 Punkte)}

\textit{Geben Sie grundsätzlich Formeln und Rechenweg zur Lösung der Teilaufgaben mit an!} \\[1Ex]
 

 
%% --------------------------------------------------------------------
\begin{minipage}[t]{0.5\textwidth}
\includegraphics[width = 1.3cm]{/Users/kruppajo/work/GitHub/exam/avatare/Steffen.png}
\end{minipage}
\begin{minipage}[t]{0.5\textwidth}
\hfill
\href{https://youtu.be/ORHSPTCdfeY}{\includegraphics[width = 2cm]{img/youtube}}
\end{minipage}
\vspace{-3ex}
%% --------------------------------------------------------------------



\paragraph{Visualisierung des Histogramm für kontinuierliche Daten}

In einem Gespräch mit seinem Betreuer wird Steffen gebeten seine Daten aus einem Freilandversuch mit Spargel in einem Histogramm darzustellen. Aus den Boxen wummert Taylor Swift und sein Mund ist verklebt von Oreos. 'Herrlich', denkt Steffen. In seinem Experiment hat er die mittleren Läsionen auf den Blättern erst fotographiert und dann ausgezählt. Laut seinem Betreuer soll das Histogramm helfen, die Verteilung der die mittleren Läsionen auf den Blättern zu bestimmen.  Es wäre einfacher, wenn da nicht noch was wäre. Steffen und die Romantik, eine unendliche Geschichte mit kniffeligen Wendungen. Steffen streichelt liebevoll die Schlange. Der Kopf ist in seinem Schloß vergraben um den Klang von Taylor Swift zu dämpfen.

\begin{center}
Die mittleren Läsionen auf den Blättern: 6.2, 10.6, 14, 12.1, 9.8, 11.3, 8.2, 12.1, 10.4, 7.6, 11.3, 11.8, 9.6, 9.1, 8.2, 8.5, 10.9, 10.8, 10.3, 11.3, 7.9, 11, 8.7, 13.3, 7.9, 9, 8, 10.5
\end{center}

Leider kennt sich Steffen mit der Erstellung von Histogrammen überhaupt nicht aus. Deshalb braucht er bei der Erstellung Ihre Hilfe!

\begin{enumerate}
\item Zeichen Sie ein Histogramm um die Verteilung der Daten zu visualisieren! (\textbf{3 Punkte})
 \item Erläutern Sie Ihr Vorgehen um ein Histogramm für kontinuierliche Daten zu zeichnen!  (\textbf{2 Punkte})
\item Beschriften Sie die Achsen der Abbildung! (\textbf{2 Punkte})
\item Ergänzen Sie die relativen Häufigkeiten in der Abbildung! \textbf{(1 Punkt)}  
\end{enumerate}

 
\clearpage
% -----------------------------------------------------------------------

\section{Aufgabe \hfill (10 Punkte)}

\textit{Geben Sie grundsätzlich Formeln und Rechenweg zur Lösung der Teilaufgaben mit an!} \\[1Ex]
 

 
%% --------------------------------------------------------------------
\begin{minipage}[t]{0.5\textwidth}
\includegraphics[width = 1.3cm]{/Users/kruppajo/work/GitHub/exam/avatare/Jonas.png}
\end{minipage}
\begin{minipage}[t]{0.5\textwidth}
\hfill
\href{https://youtu.be/VAqiUdV4WQ0}{\includegraphics[width = 2cm]{img/youtube}}
\end{minipage}
\vspace{-3ex}
%% --------------------------------------------------------------------




\paragraph{Visualisierung des Scatterplots}

'Hm...', Snickers und Iron Maiden. Das ist und bleibt die beste Kombination zum Nachdenken für Jonas. Jetzt heißt es aber erstmal auf was anderes konzentrieren. Jonas möchte gerne den Zusammenhang zwischen durschnittlichen Niederschlag [ml/w] und Trockengewicht [kg/ha] im Kontext von Erbsen herausfinden. Hierfür hat Jonas ein Gewächshausexperiment im Emsland durchgeführt. Nach einigen unvorgesehenen Ereignissen hat er es geschafft folgende Datentabelle zu erstellen. Jonas und die Erschöpfung, eine unendliche Geschichte mit kniffeligen Wendungen. Aber das steht auch nicht im Zentrum. Nun stellt sich die Frage für ihn, ob es überhaupt einen Zusammenhang zwischen den gemessenen Variablen gibt. Deshalb möchte Jonas als erstes eine explorative Datenanalyse durchführen. Dann was anderes. Irgendwie komisch, wenn er Mission Impossible anmacht, dann ist das Meerschweinchen eigentlich sofort vor dem Bildschirm und starrt hinein.

\begin{table}[!h]
\centering
\begin{tabular}{cc}
\toprule
Durschnittlichen Niederschlag [ml/w] & Trockengewicht [kg/ha]\\
\midrule
8.3 & 17.9\\
14.4 & 25.6\\
23.2 & 34.4\\
21.4 & 31.7\\
22.1 & 35.6\\
\addlinespace
21.2 & 30.7\\
18.0 & 23.3\\
23.5 & 36.6\\
21.4 & 34.8\\
\bottomrule
\end{tabular}
\end{table}



Leider kennt sich Jonas mit der Erstellung einer explorativen Datenanalyse für kontinuierliche Daten überhaupt nicht aus. Deshalb braucht er bei der Erstellung Ihre Hilfe!

\begin{enumerate}
\item Erstellen Sie eine Visualisierung für die Datentabelle. Beschriften Sie
  die Achsen entsprechend! \textbf{(4 Punkte)}
\item Schätzen Sie eine Gerade durch die Punkte! \textbf{(1 Punkt)}
\item Beschriften Sie die Gerade mit den gängigen statistischen Maßzahlen! Geben Sie die numerischen Zahlenwerte mit an! \textbf{(3 Punkte)}
\item Wenn \textit{ein} Effekt von $x$ auf $y$ vorhanden wäre, wie würde die Gerade verlaufen und welche Werte würden die statistischen Maßzahlen annehmen? \textbf{(2 Punkt)}
\end{enumerate} 
\clearpage
% -----------------------------------------------------------------------

\section{Aufgabe \hfill (10 Punkte)}

\textit{Geben Sie grundsätzlich Formeln und Rechenweg zur Lösung der Teilaufgaben mit an!} \\[1Ex]
 

 
%% --------------------------------------------------------------------
\begin{minipage}[t]{0.5\textwidth}
\includegraphics[width = 1.3cm]{/Users/kruppajo/work/GitHub/exam/avatare/Jessica.png}
\end{minipage}
\begin{minipage}[t]{0.5\textwidth}
\hfill
\href{https://youtu.be/t_1KL77mfmg}{\includegraphics[width = 2cm]{img/youtube}}
\end{minipage}
\vspace{-3ex}
%% --------------------------------------------------------------------



\paragraph{Visualisierung des Mosaicplots}

Zwei kategoriale Variablen darzustellen ist nicht so einfach. Jessica hatte erst über einen Mittelwert nachgedacht, dann aber die Idee verworfen. Wäre da nicht noch was anderes. Wenn der Mangel nicht wäre, ja dann wäre wohl vieles möglich für Jessica! Aber so.. Dabei hatte sie sich in einen Leistungssteigerungsversuch im Wendland zum einen die Behandlung Außenklimakontakt [ja/nein] und zum anderen die Messung Fettgehalt erreicht [ja/nein] im Kontext von Puten angeschaut. Jetzt möchte ihre Betreuerin erstmal die langen Tabellen mit ja/nein in einer explorativen Datenanalyse zusammengefasst bekommen. Sonst geht es bei ihrem Projektbericht nicht weiter. Was super nervig ist. Einfach mal raus um Rad zu fahren. Ohne Ziel und Uhr. Das ist was für Jessica.



\vspace{1Ex}

\begin{center}
\begin{minipage}[t]{0.45\textwidth}
%\small
\begin{center}

\begin{tabular}{p{2.5cm}p{2.5cm}p{2.5cm}p{2.5cm}}
\toprule
Außenklimakontakt & Fettgehalt erreicht\\
\midrule
ja & ja\\
nein & nein\\
ja & ja\\
ja & nein\\
nein & nein\\
\addlinespace
nein & nein\\
ja & ja\\
nein & nein\\
nein & ja\\
ja & ja\\
\addlinespace
nein & ja\\
ja & ja\\
ja & ja\\
ja & ja\\
nein & ja\\
\bottomrule
\end{tabular}


\end{center}
\end{minipage}
\begin{minipage}[t]{0.45\textwidth}
%\small
\begin{center}

\begin{tabular}{p{2.5cm}p{2.5cm}p{2.5cm}p{2.5cm}}
\toprule
Außenklimakontakt & Fettgehalt erreicht\\
\midrule
nein & ja\\
ja & ja\\
ja & nein\\
ja & ja\\
nein & nein\\
\addlinespace
ja & ja\\
nein & nein\\
nein & ja\\
ja & ja\\
nein & nein\\
\addlinespace
ja & nein\\
ja & ja\\
nein & nein\\
nein & ja\\
ja & nein\\
\bottomrule
\end{tabular}


\end{center}
\end{minipage}
\end{center}

\vspace{2Ex}

Leider kennt sich Jessica mit der Erstellung einer explorativen Datenanalyse für kategoriale Daten überhaupt nicht aus. Deshalb braucht sie bei der Erstellung Ihre Hilfe!

\begin{enumerate}
\item Stellen Sie den Zusammenhang zwischen den beiden kategorialen Variablen in einer zusammenfassenden Tabelle dar! \textbf{(3 Punkte)}
\item Visualisieren Sie den Zusammenhang zwischen den beiden kategorialen Variablen! \textbf{(3 Punkte)}
\item Berechnen Sie die Verhältnisse in der Visualisierung! Welche Annahme haben Sie getroffen? \textbf{(2 Punkte)}
\item Wenn \textit{ein} Effekt von der Behandlung vorliegen würde, wie würde die Tabelle und die Visualisierung aussehen? \textbf{(2 Punkt)}
\end{enumerate} 
\clearpage
% -----------------------------------------------------------------------

\section{Aufgabe \hfill (10 Punkte)}

\textit{Geben Sie grundsätzlich Formeln und Rechenweg zur Lösung der Teilaufgaben mit an!} \\[1Ex]
 

 
%% --------------------------------------------------------------------
\begin{minipage}[t]{0.5\textwidth}
\includegraphics[width = 1.3cm]{/Users/kruppajo/work/GitHub/exam/avatare/Jessica.png}\hspace{-4mm}\includegraphics[width = 1.3cm]{/Users/kruppajo/work/GitHub/exam/avatare/Nilufar.png}
\end{minipage}
\begin{minipage}[t]{0.5\textwidth}
\hfill
\href{https://youtu.be/Op-gjzASH9I}{\includegraphics[width = 2cm]{img/youtube}}
\end{minipage}
%% --------------------------------------------------------------------



\paragraph{Visualisierung von Verteilungen}

'Was soll das denn jetzt schon wieder sein? Drei Boxplot, die auf der Seite liegen?', entfährt es Jessica und schaut dabei Nilufar an. 'Keine Ahnung. Es ist bestimmt wieder so ein Lernziel mit der Verteilung und so.', meint Nilufar sichtlich genervt und mampft noch ein paar Takis Blue Heat. 'Du weißt doch wie es heißt, \textit{Frei ist, wer missfallen kann.}\footnote{Oschmann, A. (2024) Mädchen stärken: Stärken fördern, Selbstwert erhöhen und liebevoll durch Krisen begleiten. Goldegg Verlag}', merkt Jessica nickend an. Die beiden schauen angestrengt auf die drei Boxplots. Das Ziel ist es zu verstehen, wie eine Verteilung anhand eines Boxplots bewertet werden kann. Nilufar und die Erwartung machen die Sache nicht einfacher.



{\centering \includegraphics[width=\maxwidth]{img/desc-stat-11-1} 

}




Jetzt brauchen Jessica und Nilufar Ihre Hilfe bei der Abschätzung einer Verteilung anhand von Boxplots um ihre Arbeit dann in diesem Semester noch abschließen zu können.

\begin{enumerate}
\item Zeichnen Sie über die Boxplots die entsprechende zugehörige Verteilung! \textbf{(3 Punkte)} 
\item Zeichnen Sie unter die Boxplots die entsprechende zugehörige Beobachtungen als Stiche! \textbf{(3 Punkte)}
\item Wie viel Prozent der Beobachtungen fallen in das IQR? Ergänzen Sie die Abbildung entsprechend um den Bereich! \textbf{(2 Punkte)}
\item Wie viel Prozent der Beobachtungen fallen in $\bar{y} \pm 1s$ und $\bar{y} \pm 2s$  unter der Annahme einer Normalverteilung? \textbf{(2 Punkte)}
\end{enumerate} 
\clearpage
% -----------------------------------------------------------------------

\section{Aufgabe \hfill (10 Punkte)}

\textit{Geben Sie grundsätzlich Formeln und Rechenweg zur Lösung der Teilaufgaben mit an!} \\[1Ex]
 

 
%% --------------------------------------------------------------------
\begin{minipage}[t]{0.5\textwidth}
\includegraphics[width = 1.3cm]{/Users/kruppajo/work/GitHub/exam/avatare/Mark.png}\hspace{-4mm}\includegraphics[width = 1.3cm]{/Users/kruppajo/work/GitHub/exam/avatare/Steffen.png}
\end{minipage}
\begin{minipage}[t]{0.5\textwidth}
\hfill
\href{https://youtu.be/ZrJhn2wPbq4}{\includegraphics[width = 2cm]{img/youtube}}
\end{minipage}
%% --------------------------------------------------------------------



\paragraph{Visualisierung der Normalverteilung}

'Jetzt haben wir schon überall geschaut und ich finde die verdammte der Hamster nicht. Wo ist die den normalerweise? Und wenn du jetzt einen blöden Witz über die Aufgabe und normal machst, dann gehe ich.', faucht Mark Steffen an. 'Ui, alles gut. Vielleicht ein paar Oreos zur Entspannung?', entgegnet Steffen. Manchmal macht die Unsicherheit Mark zu einem anderen Menschen, der er nicht sein will. Da rennt dann auch der Hamster vor ihm weg. Jetzt sollen die beiden diese Aufgabe zur Normalverteilung lösen. Es geht um verschiedene Normalverteilungen udn zu verstehen, wie die Parameter der Normalverteilung funktionieren. Anscheinend hängen Normalverteilungen vom Mittelwert $\bar{y}$ und der Standardabweichung $s$ ab.\\



Jetzt brauchen Mark und Steffen Ihre Hilfe bei der Abschätzung einer Verteilung um ihre Arbeit dann in diesem Semester noch abschließen zu können.

\begin{enumerate}
\item Skizzieren Sie vier Normalverteilungen mit $\bar{y}_1 \neq \bar{y}_2 \neq \bar{y}_3 \neq \bar{y}_4$ und $s_1 = s_2 = s_3 = s_4$! \textbf{(3 Punkte)}
\item Beschriften Sie die Normalverteilungen mit den statistischen Maßzahlen! \textbf{(2 Punkte)}
\item Liegt Varianzhomogenität oder Varianzheterogenität vor? Begründen Sie Ihre Antwort! \textbf{(2 Punkte)}
\item In welchen Bereich fallen 68\% bzw. 95\% der Beobachtungen in einer Normalverteilung? Ergänzen Sie die Bereiche in \underline{einer} Normalverteilung! \textbf{(2 Punkte)}
\item Ergänzen Sie unter \underline{einer} der Normalverteilungen den entsprechenden Boxplot! \textbf{(1 Punkt)}
\end{enumerate}

 
\clearpage
% -----------------------------------------------------------------------

\section{Aufgabe \hfill (10 Punkte)}

\textit{Geben Sie grundsätzlich Formeln und Rechenweg zur Lösung der Teilaufgaben mit an!} \\[1Ex]
 

 
%% --------------------------------------------------------------------
\begin{minipage}[t]{0.5\textwidth}
\includegraphics[width = 1.3cm]{/Users/kruppajo/work/GitHub/exam/avatare/Jonas.png}\hspace{-4mm}\includegraphics[width = 1.3cm]{/Users/kruppajo/work/GitHub/exam/avatare/Tina.png}
\end{minipage}
\begin{minipage}[t]{0.5\textwidth}
\hfill
\href{https://youtu.be/MiD42k4l5Ag}{\includegraphics[width = 2cm]{img/youtube}}
\end{minipage}
%% --------------------------------------------------------------------



\paragraph{Visualisierung der Normalverteilung und der Poissonverteilung}

'Wer hat sich denn sowas ausgedacht? Verteilungen?', entfährt es Tina und schaut dabei Jonas an. In ihrer Hand zerdrückt sie Snickers von Jonas. 'Wir sollen eine Normalverteilung mit einem Mittelwert von $\bar{y}_1 = 3$ und einer Standardabweichung von $s_1 = 1$ zeichnen. Sowie eine weitere Normalverteilung mit einem Mittelwert von $\bar{y}_2 = 1$ und einer Standardabweichung von $s_2 = 1$. Keine Ahnung wie das geht. Darunter sollen dann noch eine Poissonverteilung mit einem Mittelwert von $\lambda_1 = 1$ sowie einer weiteren Poissonverteilung mit einem Mittelwert von $\lambda_2 = 20$ gezeichnet werden.', meint Jonas sichtlich eingeschüchtert und rettet noch ein paar Snickers in seinem Mund. Tina und die Wut machen die Suche nach der Lösung nicht einfacher. Im Hintergrund spielt viel zu laut Iron Maiden, die diesmal Tina ausgewählt hat und die Spinne schon in die Küche vertrieben hat, wo es den Biomüll mampft.\\




{\centering \includegraphics[width=\maxwidth]{img/histogram-01-1} 

}




Jetzt brauchen Tina und Jonas Ihre Hilfe bei der Abschätzung einer Verteilung um ihre Arbeit dann in diesem Semester noch abschließen zu können.


\begin{enumerate}
\item Skizzieren Sie die zwei Normalverteilungen und zwei Poissonverteilungen! \textbf{(4 Punkte)}
\item Achten Sie auf die entsprechende Skalierung in den jeweiligen Abbildungen! \textbf{(2 Punkte)}
\item Ergänzen Sie unter \underline{einer} Normalverteilung den entsprechenden Boxplot! \textbf{(1 Punkt)}
\item Ergänzen Sie unter \underline{einer} Poissonverteilung den entsprechenden Boxplot! \textbf{(1 Punkt)}
\item Geben Sie ein Beispiel für ein Outcome $y$, welches einer Normalverteilung folgt! \textbf{(1 Punkt)}
\item Geben Sie ein Beispiel für ein Outcome $y$, welches einer Poissonverteilung folgt! \textbf{(1 Punkt)}
\end{enumerate} 
\clearpage
% -----------------------------------------------------------------------
\part{Statistisches Testen \& statistische Testtheorie}
% -----------------------------------------------------------------------  

\section{Aufgabe \hfill (9 Punkte)}


 
%% --------------------------------------------------------------------
\begin{minipage}[t]{0.5\textwidth}
\includegraphics[width = 1.3cm]{/Users/kruppajo/work/GitHub/exam/avatare/Alex.png}\hspace{-4mm}\includegraphics[width = 1.3cm]{/Users/kruppajo/work/GitHub/exam/avatare/Mark.png}
\end{minipage}
\begin{minipage}[t]{0.5\textwidth}
\hfill
\href{https://youtu.be/aHVYuFKTqZs}{\includegraphics[width = 2cm]{img/youtube}}
\end{minipage}
%% --------------------------------------------------------------------



\paragraph{Grundgesamtheit und experimentelle Stichprobe}

'Grundlage des statistischen Testen ist das Verständnis von der Grundgesamtheit (eng. \textit{population} oder \textit{ground truth}) und der experimentellen Stichprobe (eng. \textit{sample}). ', liest Alex laut aus dem Skript vor. Mark war kurz eingenickt und wird mit einem Stoß geweckt. 'Reiz dich zusammen und iss noch ein paar Gummibärchen das hilft mir immer. Alleine komme ich hier nicht weiter.', tadelt Alex Mark etwas zu forsch. 'War ne lange Nacht', mault Mark. Beide sollen in ihrer Abschlussarbeit einen statistischen Test interpretieren und versuchen die Grundlagen zu wiederholen. Mark war auf einem Konzert von Andrea Berg.

\vspace{1ex}

Leider kennen sich Alex und Mark mit der Grundgesamtheit und der Stuchprobe überhaupt nicht aus. Daher sind Sie gefragt!

\begin{enumerate}
\item Nennen Sie das statistische Verfahren und zwei konkrete Beispiele zur Durchführung um von einer Grundgesamtheit auf eine Stichprobe zu gelangen! \textbf{(3 Punkte)}
\item Erklären Sie den Zusammenhang zwischen Stichprobe und Grundgesamtheit an einem Schaubild! Beschriften Sie das Schaubild entsprechend!
  \textit{Nutzen Sie hierfür als Veranschaulichung die Körpergröße von Männern oder Frauen aus den Gummibärchendaten!}  \textbf{(3 Punkte)}
\item Erweitern Sie das Schaubild um die Entstehung von $Pr(D|H_0)$! \textit{Nutzen Sie hierfür als Veranschaulichung zusätzlich die Gruppierungsvariable "`Modul"' aus den Gummibärchendaten!}  \textbf{(3 Punkte)}
\end{enumerate} 
\clearpage
% -----------------------------------------------------------------------

\section{Aufgabe \hfill (9 Punkte)}


 
%% --------------------------------------------------------------------
\begin{minipage}[t]{0.5\textwidth}
\includegraphics[width = 1.3cm]{/Users/kruppajo/work/GitHub/exam/avatare/Jonas.png}\hspace{-4mm}\includegraphics[width = 1.3cm]{/Users/kruppajo/work/GitHub/exam/avatare/Yuki.png}
\end{minipage}
\begin{minipage}[t]{0.5\textwidth}
\hfill
\href{https://youtu.be/Ric8ne39DtI}{\includegraphics[width = 2cm]{img/youtube}}
\end{minipage}
%% --------------------------------------------------------------------



\paragraph{Das Nullritual - Die statistische Testtheorie}

'Matrix ist der beste Film, den es gibt.', meint Yuki. Jonas entgegnet, ' Ich empfehle ja immer allen Mission Impossible.'Die beiden sind im Zoo und diskutieren, ob Pinguine Knie haben. Eigentlich wollten beide nochmal die statistische Testheorie durchgehen, sind dann aber auf abenteuerlichen Wege im Zoo gelandet. Yuki starrt wie hypnotisiert auf einen strullenden Elefanten und stopt die Zeit.\footnote{Yang, P. J., et al. (2014). Duration of urination does not change with body size. Proceedings of the National Academy of Sciences, 111(33), 11932-11937.} 'Du bist so peinlich.', entfährt es Jonas und schmeißt sich noch ein paar überteuerte Snickers rein.

\vspace{1ex}

Leider kennen sich Yuki und Jonas mit statistischen Testtheorie, auch Null-Ritual genannt, überhaupt nicht aus. Geschweige denn mit der Visualisierung als Kreuztabelle.  

\begin{enumerate}
\item Tragen Sie folgende statistische Fachbegriffe zur statistischen Testtheorie korrekt eine selbst erstellte Kreuztabelle ein! \textbf{(3 Punkte)}
  \begin{center}
  \begin{tabular}{cccc}
  H$_0$ abgelehnt & Richtige Entscheidung & 5\% & H$_0$ beibehalten \\
  \end{tabular}
  \end{center}
\item Ergänzen Sie Ihre erstellte Kreuztabelle um vier weitere, passende Fachbegriffe zur statistischen Testtheorie! \textbf{(2 Punkte)}
\end{enumerate}

Die Entscheidungsfindung durch einen statistischen Test kann auch durch die Analogie zu einem Feuermelder abgebildet werden. Dabei symbolisiert der Feuermelder den statistischen Test und es soll getestet werden, ob ein Feuer ausgebrochen ist.

\begin{enumerate}
  \setcounter{enumi}{2}    
\item In der Analogie des Feuermelders, wie lautet der $\alpha$-Fehler? \textbf{(1 Punkt)}
\item In der Analogie des Feuermelders, wie lautet der $\beta$-Fehler? \textbf{(1 Punkt)}
\item Wenn der Feuermelder einmal pro Tag messen würde, wie oft würde der Feuermelder mit einem $\alpha$ von 5\% in einem Monat Alarm schlagen? Begründen Sie Ihre Antwort! \textbf{(2 Punkte)}
\end{enumerate}



 
\clearpage
% -----------------------------------------------------------------------

\section{Aufgabe \hfill (9 Punkte)}

\textit{Geben Sie grundsätzlich Formeln und Rechenweg zur Lösung der Teilaufgaben mit an!} \\[1Ex]


 
%% --------------------------------------------------------------------
\begin{minipage}[t]{0.5\textwidth}
\includegraphics[width = 1.3cm]{/Users/kruppajo/work/GitHub/exam/avatare/Alex.png}\hspace{-4mm}\includegraphics[width = 1.3cm]{/Users/kruppajo/work/GitHub/exam/avatare/Jonas.png}
\end{minipage}
\begin{minipage}[t]{0.5\textwidth}
\hfill
\href{https://youtu.be/32JjH1eyuTU}{\includegraphics[width = 2cm]{img/youtube}}
\end{minipage}
%% --------------------------------------------------------------------



\paragraph{Visualisierung der Teststatistik $\boldsymbol{T_D}$ und dem p-Wert}

'Wir sollen die Teststatistik $T_D$ umd dem p-Wert visualisieren, da mit einer Visualisierung vieles verständlicher wird!', ruft Alex um Abba zu übertönen. 'Ich weiß nicht, was das jetzt helfen soll. Können wir nicht einfach schauen, ob der p-Wert kleiner als das Signifikanzniveau  $\alpha$ gleich 5\% ist? Und gut ist?', merkt Jonas an, was aber im Refrain von Abba untergeht. Alex nickt im Beat. 'Wir haben hier eine t-verteilung unter der Annahme der Nullhypothese!', singt er.

\vspace{1ex}

Leider kennen sich Alex und Jonas mit der Visualisierung der Teststatistik $T_D$ und dem p-Wert überhaupt nicht aus und brauchen dahr Ihre Hilfe!

\vspace{1ex}

\textit{Beachten Sie, dass im Folgenden \underline{keine numerisch korrekte Darstellung} verlangt wird! Es gilt Erkennbarkeit vor Genauigkeit!}

\begin{enumerate}
\item Ergänzen Sie eine beschriftete $x$-Achse! \textbf{(1 Punkt)}
\item Ergänzen Sie "`$\bar{y}_1 = \bar{y}_2$"'! \textbf{(1 Punkt)} 
\item Ergänzen Sie "`$A = 0.95$"'! \textbf{(1 Punkt)}
\item Zeichnen Sie $T_{\alpha=5\%}$ in die Abbildung! \textbf{(1 Punkt)} 
\item Zeichnen Sie das Signifikanzniveau $\alpha$ in die Abbildung! Begründen Sie Ihre Antwort! \textbf{(2 Punkte)} 
\item Zeichnen Sie $+T_{D}$ in die Abbildung! \textbf{(1 Punkt)}
\item Zeichnen Sie einen nicht signifikant p-Wert in die Abbildung! Begründen Sie Ihre Antwort! \textbf{(2 Punkte)}   
\end{enumerate}



{\centering \includegraphics[width=\maxwidth]{img/statistisches-testen-3-1} 

}


 
\clearpage
% -----------------------------------------------------------------------

\section{Aufgabe \hfill (10 Punkte)}


 
%% --------------------------------------------------------------------
\begin{minipage}[t]{0.5\textwidth}
\includegraphics[width = 1.3cm]{/Users/kruppajo/work/GitHub/exam/avatare/Jonas.png}\hspace{-4mm}\includegraphics[width = 1.3cm]{/Users/kruppajo/work/GitHub/exam/avatare/Yuki.png}
\end{minipage}
\begin{minipage}[t]{0.5\textwidth}
\hfill
\href{https://youtu.be/CN_O4fYPbhs}{\includegraphics[width = 2cm]{img/youtube}}
\end{minipage}
%% --------------------------------------------------------------------



\paragraph{Visualisierung des 95\% Konfidenzintervalls}

'So, was haben wir gemacht? Wir haben einen t-test für den Vergleich der Mittelwerte gerechnet.', meint Yuki. Jonas schaut fragend. 'Hatten wir nicht alles zu einer Kontrolle verglichen? Das war doch so!', ruft Jonas laut aus. 'Wir haben doch Messwert \textit{Energieverbrauch der Klimakammer} erhoben.', stellt Yuki fest. Jetzt haben beide das Problem, die möglichen 95\% Konfidenzintervalle zu interpretieren.

\vspace{1ex}

Leider kennen sich Yuki und Jonas mit der Visualisierung des 95\% Konfidenzintervall überhaupt nicht aus. 

\begin{enumerate}
\item Beschriften Sie die untenstehende Abbildung mit der Signifikanzschwelle! Begründen Sie Ihre Antwort! \textbf{(2 Punkte)}
\item Ergänzen Sie eine \textit{in den Kontext passende} Relevanzschwelle! Begründen Sie Ihre Antwort! \textbf{(2 Punkte)} 
\item Skizieren Sie in die untenstehende Abbildung sechs einzelne Konfidenzintervalle (a-f) mit den
  jeweiligen Eigenschaften! \textbf{(6 Punkte)}
  \begin{itemize}
  \item[(a)] Ein signifikantes, nicht relevantes 95\% Konfidenzintervall 	
  \item[(b)] Ein 95\% Konfidenzintervall mit niedriger Fallzahl $n$ in der Stichprobe als der Rest 95\% der Konfidenzintervalle 	
  \item[(c)] Ein signifikantes, relevantes 95\% Konfidenzintervall 	
  \item[(d)] Ein nicht signifikantes, nicht relevantes 95\% Konfidenzintervall 
  \item[(e)] Ein 95\% Konfidenzintervall mit h{"o}herer Fallzahl $n$ in der Stichprobe als der Rest der 95\% Konfidenzintervalle
  \item[(f)] Ein signifikantes, relevantes 99\% Konfidenzintervall.
  \end{itemize}
\end{enumerate}

\begin{center}
  \includegraphics[height = 10cm]{/Users/kruppajo/work/GitHub/exam/question/img/statistisches-testen-04}
\end{center}


 
\clearpage
% -----------------------------------------------------------------------

\section{Aufgabe \hfill (10 Punkte)}

\textit{Geben Sie grundsätzlich Formeln und Rechenweg zur Lösung der Teilaufgaben mit an!} \\[1Ex]


 
%% --------------------------------------------------------------------
\begin{minipage}[t]{0.5\textwidth}
\includegraphics[width = 1.3cm]{/Users/kruppajo/work/GitHub/exam/avatare/Nilufar.png}\hspace{-4mm}\includegraphics[width = 1.3cm]{/Users/kruppajo/work/GitHub/exam/avatare/Yuki.png}
\end{minipage}
\begin{minipage}[t]{0.5\textwidth}
\hfill
\href{https://youtu.be/FgZmpnEWDag}{\includegraphics[width = 2cm]{img/youtube}}
\end{minipage}
%% --------------------------------------------------------------------



\paragraph{Zusammenhang zwischen dem Effekt, der Streuung sowie der Fallzahl}

An einem kalten Dezembermorgen haben sich Yuki und Nilufar zum Lernen verabredet. Eine große Kanne Tee und Berge von Reese's Peanut Butter Cups warten darauf gegessen zu werden. Yuki liest laut vor:\begin{quote}
                 \textit{
                 Beim statistischen Testen gibt es einen Zusammenhang zwischen dem Effekt, der Streuung sowie der Fallzahl. Gegeben sei die Formel für den Student t-Test auf den die folgenden Überlegungen basieren sollen. Welche Auswirkung hat die Änderungen der jeweiligen statistischen Maßzahl des Effekts $\Delta$, der Streuung $s$ und der Fallzahl $n$ auf die Teststistik $T_{D}$, den p-Wert $Pr(D|H_0)$ sowie dem Konfidenzintervall $KI_{1-\alpha}$?
                 }
                 \end{quote}Nilufar hebt die Augenbraue. 'Irgendwie sagt mir die Aufgabe jetzt mal so gar nichts. Was soll da gemacht werden?', merkt Nilufar an und ergänzt: 'Schauen wir doch erstmal zur Entspannung Star Trek, den Film habe ich extra nochmal mitgebracht.' Yuki ist der Idee nicht abgeneigt und auch das Minischwein kommt unter dem Sofa hervor um mitzuschauen. 

\vspace{1ex}

Leider kennen sich Yuki und Nilufar mit dem Zusammenhang zwischen dem Effekt, der Streuung sowie der Fallzahl überhaupt nicht aus. 


\begin{enumerate}
\item Visualisieren Sie den Zusammenhang zwischen der Teststatiatik $T_{D}$ und dem p-Wert $Pr(D|H_0)$ für sich verändernde $T_{D}$-Werte!\textit{Geben Sie dafür ein numerisches Beispiel in dem Sie drei $T_{D}$-Werte und deren Einfluss auf den p-Wert vergleichen!} \textbf{(3 Punkte)}  
\item  Füllen Sie die untenstehende Tabelle aus in dem Sie die Änderung der statistischen Maßzahlen auf die Teststatistik, den p-Wert sowie das Konfidenzintervall in \textit{einem} Wort oder Symbol beschreiben! \textbf{(4 Punkte)}
\begin{center}
  \large
  \begin{tabular}[c]{l|c|c|c|l|c|c|c}
    & $T_{D}$ & $Pr(D|H_0)$ & $KI_{1-\alpha}$ & & $T_{D}$ & $Pr(D|H_0)$ & $KI_{1-\alpha}$\strut\\ 
    \hline
    \textbf{$\Delta\; \uparrow$} & \hspace{1.8cm} & \hspace{1.8cm}  & \hspace{1.8cm} & \textbf{
                                                          $\Delta\; \downarrow$} &
                                                                          \hspace{1.8cm} & \hspace{1.8cm}  & \hspace{1.8cm}\strut\\
    \hline
        \textbf{$s\; \uparrow$} & \hspace{1.8cm} & \hspace{1.8cm}  & \hspace{1.8cm} & \textbf{
                                                          $s\; \downarrow$} &
                                                                          \hspace{1.8cm}
                                                & \hspace{1.8cm}  & \hspace{1.8cm}\strut\\
    \hline
        \textbf{$n\; \uparrow$} & \hspace{1.8cm} & \hspace{1.8cm}  & \hspace{1.8cm} & \textbf{
                                                          $n\; \downarrow$} &
                                                                          \hspace{1.8cm}
                                                & \hspace{1.8cm}  & \hspace{1.8cm}\strut\\
    \hline
  \end{tabular}
\end{center}
\item Visualisieren Sie ein 95\%-iges Konfidenzintervall im Vergleich zu einem 90\%-igen Konfidenzintervall! Begründen Sie Ihre Visualisierung anhand der Formel des Konfidenzintervalls des t-Tests mathematisch! \textbf{(3 Punkte)} 
\end{enumerate} 
\clearpage
% -----------------------------------------------------------------------
\part{Der Student t-Test, Welch t-Test \& gepaarter t-Test}
% -----------------------------------------------------------------------

\section{Aufgabe \hfill (9 Punkte)}

\textit{Geben Sie grundsätzlich Formeln und Rechenweg zur Lösung der Teilaufgaben mit an!} \\[1Ex]
 

 
%% --------------------------------------------------------------------
\begin{minipage}[t]{0.5\textwidth}
\includegraphics[width = 1.3cm]{/Users/kruppajo/work/GitHub/exam/avatare/Yuki.png}
\end{minipage}
\begin{minipage}[t]{0.5\textwidth}
\hfill
\href{https://youtu.be/eejS2uG4o-M}{\includegraphics[width = 2cm]{img/youtube}}
\end{minipage}
\vspace{-3ex}
%% --------------------------------------------------------------------



\paragraph{Berechnung des Student t-Test \underline{oder} Welch t-Test}

'Der t-Test testet einen normalverteilten Messwert ($Y$).', liest Yuki laut. Das hilft jetzt auch nur bedingt weiter. Yuki und die Faulheit, eine unendliche Geschichte mit kniffeligen Wendungen. Laut ihrem Betreuer ist zwar ihr Messwert Gewichtszuwachs in der 1LW normalverteilt, aber wie rechnet sie jetzt einen t-Test? Für ihrer Hausarbeit zum Testen einer neuen technischen Anlage musste sie ein Kreuzungsexperiment mit Puten im Oldenburger Land durchführen. Als wäre das nicht schon anstrengend genug gewesen bei dem anspruchsvollen Pilotprojekt mit sehr geringer Fallzahl $(n_1 = n_2 = 3)$. Jetzt soll sie auch noch testen, ob die Behandlung Genotypen ($AA$ und $BB$) ein signifikantes Ergebnis liefert. Hm, lecker Reese's Peanut Butter Cups und dazu dann im Hintergrund Matrix laufen lassen.

\begin{table}[!h]
\centering
\begin{tabular}{cc}
\toprule
treatment & weight\\
\midrule
ctrl & 14.7\\
dose & 15.9\\
dose & 25.7\\
dose & 29.0\\
ctrl & 19.9\\
\addlinespace
ctrl & 19.7\\
\bottomrule
\end{tabular}
\end{table}



Leider kennt sich Yuki mit der Berechnung eines t-Tests überhaupt nicht aus. Deshalb braucht sie bei der Berechnung Ihre Hilfe!

\begin{enumerate}
  \item Formulieren Sie das statistische Hypothesenpaar! \textbf{(1 Punkt)}
  \item Bestimmen Sie die Teststatistik $T_{D}$ eines Welch t-Tests! \textbf{(3 Punkte)}
  \item Treffen Sie mit $T_{\alpha = 5\%} = 1.64$ eine Aussage zur Nullhypothese! Begründen Sie Ihre Antwort! \textbf{(2 Punkte)}
  \item Berechnen Sie den Effekt des Welch t-Tests! \textbf{(1 Punkt)}
  \item Formulieren Sie eine Antwort an Yuki über das Ergebnis Ihrer statistischen Analyse! \textbf{(2 Punkte)}
\end{enumerate} 
\clearpage
% -----------------------------------------------------------------------

\section{Aufgabe \hfill (12 Punkte)}

\textit{Geben Sie grundsätzlich Formeln und Rechenweg zur Lösung der Teilaufgaben mit an!} \\[1Ex]
 

 
%% --------------------------------------------------------------------
\begin{minipage}[t]{0.5\textwidth}
\includegraphics[width = 1.3cm]{/Users/kruppajo/work/GitHub/exam/avatare/Tina.png}
\end{minipage}
\begin{minipage}[t]{0.5\textwidth}
\hfill
\href{https://youtu.be/Cq_rF_z4xOk}{\includegraphics[width = 2cm]{img/youtube}}
\end{minipage}
\vspace{-3ex}
%% --------------------------------------------------------------------



\paragraph{Berechnung des Student t-Test}

Tina ist in der Uckermark für einen Versuch mit Lauch. Allein diese Tatsache ist für sie eine Erzählung wert. Tina und die Wut, eine unendliche Geschichte mit kniffeligen Wendungen. Für ihrer Hausarbeit musste sie ein Gewächshausexperiment mit Lauch durchführen und das sollte laut ihrem Betreuer an diesem Nichtort besonders gut gelingen. Ablenkung gibt es jedenfalls keine. Gar keine. Alleine sein hilft jetzt aber nur bedingt, denn ihre Behandlung Genotypen ($AA$ und $BB$) und der Messwert Trockengewicht [kg/ha] sollen mit einem t-Test ausgewertet werden. Immerhin weiß sie, dass ihr Messwert einer Normalverteilung folgt. Hm..., was entspannendes wäre gut. Einfach mal raus um zu Boxen. Ohne Ziel und Uhr. Das ist was für Tina.

\begin{table}[!h]
\centering
\begin{tabular}{cc}
\toprule
Genotypen & Trockengewicht\\
\midrule
AA & 28.5\\
BB & 27.7\\
AA & 23.9\\
BB & 32.4\\
AA & 22.4\\
\addlinespace
AA & 18.2\\
AA & 18.9\\
BB & 36.6\\
AA & 33.2\\
BB & 34.5\\
\addlinespace
BB & 39.6\\
BB & 29.7\\
BB & 41.6\\
AA & 31.1\\
AA & 22.3\\
\addlinespace
AA & 22.8\\
BB & 29.9\\
AA & 26.4\\
\bottomrule
\end{tabular}
\end{table}



Leider kennt sich Tina mit der Berechnung eines t-Tests überhaupt nicht aus. Deshalb braucht sie bei der Berechnung Ihre Hilfe!

\begin{enumerate}
  \item Formulieren Sie die wissenschaftliche Fragestellung! \textbf{(1 Punkt)}
  \item Formulieren Sie das statistische Hypothesenpaar! \textbf{(1 Punkt)}
  \item Bestimmen Sie die Teststatistik $T_{D}$ eines Student t-Tests! \textbf{(3 Punkte)}
\item Treffen Sie mit $T_{\alpha = 5\%} = 1.84$ eine Aussage zur Nullhypothese! Begründen Sie Ihre Antwort! \textbf{(2 Punkte)}
\item Berechnen Sie den Effekt des Student t-Tests! \textbf{(1 Punkt)}
\item Wenn Sie \textit{keinen} Unterschied zwischen den Behandlungsgruppen erwarten würden, wie groß wäre dann der Effekt? Begründen Sie Ihre Antwort! \textbf{(2 Punkte)}
\item Formulieren Sie eine Antwort an Tina über das Ergebnis Ihrer statistischen Analyse! \textbf{(2 Punkte)}
\end{enumerate} 
\clearpage
% -----------------------------------------------------------------------

\section{Aufgabe \hfill (12 Punkte)}

\textit{Geben Sie grundsätzlich Formeln und Rechenweg zur Lösung der Teilaufgaben mit an!} \\[1Ex]
 

 
%% --------------------------------------------------------------------
\begin{minipage}[t]{0.5\textwidth}
\includegraphics[width = 1.3cm]{/Users/kruppajo/work/GitHub/exam/avatare/Jessica.png}
\end{minipage}
\begin{minipage}[t]{0.5\textwidth}
\hfill
\href{https://youtu.be/TbSEOMCQYl4}{\includegraphics[width = 2cm]{img/youtube}}
\end{minipage}
\vspace{-3ex}
%% --------------------------------------------------------------------



\paragraph{Berechnung des Welch t-Test}


Der t-Test. Jessica erschaudert. Jessica und der Mangel, eine unendliche Geschichte mit kniffeligen Wendungen. Ein mächtiges Werkzeug ist der t-Test in den Händen desjenigen, der einen normalverteilten Endpunkt ($Y$) hat. Aber erstmal überhaupt den t-Test rechnen können. Wie sah das Experiment von Jessica überhaupt aus? Jessica schmeißt noch eine Handvoll Schokobons in ihren Rachen. Im Hintergrund klirrt leise der Spiegel zum Sound von David Bowie. Jessica hat ein Kreuzungsexperiment mit Lamas durchgeführt. Dabei wurde die Behandlung Ernährungszusatz ($ctrl$ und $fedX$) an den Lamas getestet. Gemessen hat Jessica dann als Messwert Fettgehalt [\%/kg]. Warum der Versuch im Wendland für ihre Abschlussarbeit stattfinden musste, ist ihr bis heute ein Rätsel. Egal. Gibt es jetzt einen Zusammenhang zwischen der Behandlung und Fettgehalt [\%/kg]?

\begin{table}[!h]
\centering
\begin{tabular}{cc}
\toprule
Ernährungszusatz & Fettgehalt\\
\midrule
ctrl & 40.7\\
fedX & 43.6\\
fedX & 45.2\\
fedX & 45.7\\
fedX & 45.6\\
\addlinespace
ctrl & 44.7\\
ctrl & 49.1\\
ctrl & 28.5\\
ctrl & 39.4\\
fedX & 34.2\\
\addlinespace
fedX & 44.2\\
fedX & 36.7\\
ctrl & 49.1\\
ctrl & 39.1\\
fedX & 49.0\\
\addlinespace
fedX & 49.9\\
\bottomrule
\end{tabular}
\end{table}



Leider kennt sich Jessica mit der Berechnung eines t-Tests überhaupt nicht aus. Deshalb braucht sie bei der Berechnung Ihre Hilfe!

\begin{enumerate}
  \item Formulieren Sie die wissenschaftliche Fragestellung! \textbf{(1 Punkt)}
  \item Formulieren Sie das statistische Hypothesenpaar! \textbf{(1 Punkt)}
  \item Bestimmen Sie die Teststatistik $T_{D}$ eines  Welch t-Tests! \textbf{(3 Punkte)}
  \item Treffen Sie mit $T_{\alpha = 5\%} = 1.84$ eine Aussage zur Nullhypothese! Begründen Sie Ihre Antwort! \textbf{(2 Punkte)}
\item Berechnen Sie das 95\% Konfidenzintervall. Welche Annahmen haben Sie getroffen? \textbf{(2 Punkte)}
\item Nennen Sie den statistischen Grund, warum Sie sich zwischen einem Student t-Test und einem Welch t-Test entscheiden müssen! \textbf{(1 Punkt)}
\item Formulieren Sie eine Antwort an Jessica über das Ergebnis Ihrer statistischen Analyse! \textbf{(2 Punkte)}
\end{enumerate} 
\clearpage
% -----------------------------------------------------------------------

\section{Aufgabe \hfill (11 Punkte)}

\textit{Geben Sie grundsätzlich Formeln und Rechenweg zur Lösung der Teilaufgaben mit an!} \\[1Ex]
 

 
%% --------------------------------------------------------------------
\begin{minipage}[t]{0.5\textwidth}
\includegraphics[width = 1.3cm]{/Users/kruppajo/work/GitHub/exam/avatare/Alex.png}\hspace{-4mm}\includegraphics[width = 1.3cm]{/Users/kruppajo/work/GitHub/exam/avatare/Tina.png}
\end{minipage}
\begin{minipage}[t]{0.5\textwidth}
\hfill
\href{https://youtu.be/QR90zyn0Iz8}{\includegraphics[width = 2cm]{img/youtube}}
\end{minipage}
%% --------------------------------------------------------------------



\paragraph{Berechnung des gepaarten t-Test}

Es gibt ja immer die Möglichkeit sich Hilfe zu holen. Das geht natürlich auch immer in einer Abschlussarbeit. Deshalb arbeiten Tina und Alex gemeinsam an einer Abschlussarbeit. Das macht dann auch die Analyse ihres Hauptversuches einfacher. Zwar hat jeder von ihnen noch ein Subthema, aber auch da kann man sich ja helfen. In dem Hauptversuch wurde Folgendes von den beiden gemacht. Tina und Alex haben sich Lamas angeschaut. Dabei geht um Zusammenhang zwischen Flüssignahrung ($1l/d$ und $5l/d$) und Schlachtgewicht [kg]. Jetzt sollen beide einen gepaarten t-Test rechnen. Es würde auch besser funktionieren, wenn Tina nicht die Wut im Weg stehen würde und Alex nicht das Problem hätte die Gefälligkeit zu händeln. Gott sei Dank haben beide genug Katjes und Gummibärchen auf dem Tisch aufgetürmt.

\begin{table}[!h]
\centering
\begin{tabular}{ccc}
\toprule
ID & treatment & freshmatter\\
\midrule
9 & 1l/d & 41.0\\
3 & 1l/d & 23.0\\
7 & 5l/d & 21.7\\
1 & 5l/d & 24.3\\
5 & 5l/d & 29.9\\
\addlinespace
6 & 1l/d & 31.1\\
2 & 5l/d & 18.5\\
4 & 1l/d & 46.3\\
7 & 1l/d & 17.1\\
6 & 5l/d & 22.2\\
\addlinespace
1 & 1l/d & 26.9\\
8 & 5l/d & 23.2\\
4 & 5l/d & 21.9\\
8 & 1l/d & 30.4\\
2 & 1l/d & 43.1\\
\addlinespace
3 & 5l/d & 18.5\\
5 & 1l/d & 35.3\\
11 & 1l/d & 18.4\\
10 & 1l/d & 40.7\\
\bottomrule
\end{tabular}
\end{table}



Leider kennen sich Tina und Alex mit der Berechnung eines gepaarten t-Tests überhaupt nicht aus. Deshalb brauchen sie beide bei der Berechnung Ihre Hilfe!

\begin{enumerate}
  \item Formulieren Sie die wissenschaftliche Fragestellung! \textbf{(1 Punkt)}
  \item Formulieren Sie das statistische Hypothesenpaar! \textbf{(1 Punkt)}
  \item Bestimmen Sie die Teststatistik $T_{D}$ eines gepaarten t-Tests! \textbf{(3 Punkte)}
  \item Treffen Sie mit $T_{\alpha = 5\%} = 1.96$ eine Aussage zur Nullhypothese! Begründen Sie Ihre Antwort! \textbf{(2 Punkte)}
\item Schätzen Sie den $p$-Wert des gepaarten t-Tests ab! Begründen Sie Ihre Antwort mit einer Skizze! \textbf{(2 Punkte)}
\item Formulieren Sie eine Antwort an Tina über das Ergebnis Ihrer statistischen Analyse! \textbf{(2 Punkte)}
\end{enumerate}


 
\clearpage
% -----------------------------------------------------------------------

\section{Aufgabe \hfill (10 Punkte)}

\textit{Geben Sie grundsätzlich Formeln und Rechenweg zur Lösung der Teilaufgaben mit an!} \\[1Ex]
 

 
%% --------------------------------------------------------------------
\begin{minipage}[t]{0.5\textwidth}
\includegraphics[width = 1.3cm]{/Users/kruppajo/work/GitHub/exam/avatare/Jonas.png}\hspace{-4mm}\includegraphics[width = 1.3cm]{/Users/kruppajo/work/GitHub/exam/avatare/Nilufar.png}\hspace{-4mm}\includegraphics[width = 1.3cm]{/Users/kruppajo/work/GitHub/exam/avatare/Tina.png}
\end{minipage}
\begin{minipage}[t]{0.5\textwidth}
\hfill
\href{https://youtu.be/exDo7AyHl4Q}{\includegraphics[width = 2cm]{img/youtube}}
\end{minipage}
%% --------------------------------------------------------------------



\paragraph{Interpretation des t-Tests in \Rlogo - die Teststatistik und der p-Wert}


Almería. Spanien. Sonne und Strand. Jonas und Tina haben ihren gemeinsamen Auslandsaufenthalt sichtlich genossen. Dann hatte sich auch noch angeboten ihre Abschlussarbeit gemeinsam in Almería durchzuführen. Es hätte sogar noch bessser funktionieret, wenn Nilufar nicht die Erwartung ein paar Mal im Weg gestanden hätte und Tina nicht das Problem gehabt hätte die Gefälligkeit zu händeln. Nun müssen jetzt alle Daten in \Rlogo ausgewertet werden, da \Rlogo international der Standard in der Datenauswertung ist und die Betreuer in Spanien nur \Rlogo können. Während beide Nilufar Oliven mit Takis Blue Heat füttern, hoffen Jonas und Tina mehr Informationen von Nilufar über die seltsame \Rlogo Ausgabe des t-Tests. Immerhin erinnern beide sich an die Behandlung Düngestufen ($ctrl$ und $high$) und das es um Kartoffeln ging. Im Hintergrund wummert Deichkind und Fotos zeigen Nilufar mit dem Hobby Hip Hop.

\begin{knitrout}
\definecolor{shadecolor}{rgb}{0.969, 0.969, 0.969}\color{fgcolor}\begin{kframe}
\begin{verbatim}
## 
## 	Two Sample t-test
## 
## data:  Frischegewicht by Düngestufen
## t = 1.9848, df = 17, p-value = 0.06354
## alternative hypothesis: true  is not equal to [condensed]
## 95 percent confidence interval:
##  -0.326432 10.694614
## sample estimates:
## mean in group ctrl mean in group high 
##           38.10909           32.92500
\end{verbatim}
\end{kframe}
\end{knitrout}

Helfen Sie Nilufar bei der Interpretation des t-Tests! Sonst geht es auch für Jonas und Tina nicht weiter.
  
\begin{enumerate}
  \item Formulieren Sie die wissenschaftliche Fragestellung! \textbf{(1 Punkt)}
  \item Formulieren Sie das statistische Hypothesenpaar! \textbf{(1 Punkt)}
\item Liegt ein signifikanter Unterschied zwischen den Gruppen vor? Begründen Sie Ihre Antwort! \textbf{(2 Punkte)}
\item Skizzieren Sie eine Abbildung in der Sie $T_{D}$, $Pr(D|H_0)$, $A=0.95$, sowie $T_{\alpha=5\%} = |2.11|$ einzeichnen! \textbf{(4 Punkte)}
\item Beschriften Sie die Abbildung! \textbf{(1 Punkt)}  
\item Berechnen Sie den Effekt des t-Tests! \textbf{(1 Punkt)}
\end{enumerate} 
\clearpage
% -----------------------------------------------------------------------

\section{Aufgabe \hfill (10 Punkte)}

\textit{Geben Sie grundsätzlich Formeln und Rechenweg zur Lösung der Teilaufgaben mit an!} \\[1Ex]
 

 
%% --------------------------------------------------------------------
\begin{minipage}[t]{0.5\textwidth}
\includegraphics[width = 1.3cm]{/Users/kruppajo/work/GitHub/exam/avatare/Jonas.png}\hspace{-4mm}\includegraphics[width = 1.3cm]{/Users/kruppajo/work/GitHub/exam/avatare/Nilufar.png}\hspace{-4mm}\includegraphics[width = 1.3cm]{/Users/kruppajo/work/GitHub/exam/avatare/Yuki.png}
\end{minipage}
\begin{minipage}[t]{0.5\textwidth}
\hfill
\href{https://youtu.be/wJqsNV1hOW8}{\includegraphics[width = 2cm]{img/youtube}}
\end{minipage}
%% --------------------------------------------------------------------



\paragraph{Interpretation des t-Tests in \Rlogo - das 95\% Konifidenzintervall}


Almería. Spanien. Sonne und Strand. Yuki und Jonas haben ihren gemeinsamen Auslandsaufenthalt sichtlich genossen. Dann hatte sich auch noch angeboten ihre Abschlussarbeit gemeinsam in Almería durchzuführen. Es hätte sogar noch bessser funktionieret, wenn Nilufar nicht die Erwartung ein paar Mal im Weg gestanden hätte und Yuki nicht das Problem gehabt hätte die Gefälligkeit zu händeln. Nun müssen jetzt alle Daten in \Rlogo ausgewertet werden, da \Rlogo international der Standard in der Datenauswertung ist und die Betreuer in Spanien nur \Rlogo können. Während beide Nilufar Oliven mit Takis Blue Heat füttern, hoffen Yuki und Jonas mehr Informationen von Nilufar über die seltsame \Rlogo Ausgabe des t-Tests. Immerhin erinnern beide sich an die Behandlung Flüssignahrung ($ctrl$ und $flOw$) und das es um Puten ging. Im Hintergrund wummert Deichkind und Fotos zeigen Nilufar mit dem Hobby Hip Hop.

\begin{knitrout}
\definecolor{shadecolor}{rgb}{0.969, 0.969, 0.969}\color{fgcolor}\begin{kframe}
\begin{verbatim}
## 
## 	Two Sample t-test
## 
## data:  Gewichtszuwachs by Flüssignahrung
## t = -4.8161, df = 15, p-value = 0.0002266
## alternative hypothesis: true  is not equal to [condensed]
## 95 percent confidence interval:
##  -24.426715  -9.438999
## sample estimates:
## mean in group ctrl mean in group flOw 
##           22.75714           39.69000
\end{verbatim}
\end{kframe}
\end{knitrout}

Helfen Sie Nilufar bei der Interpretation des t-Tests! Sonst geht es auch für Yuki und Jonas nicht weiter.

\begin{enumerate}
  \item Formulieren Sie die wissenschaftliche Fragestellung! \textbf{(1 Punkt)}
  \item Formulieren Sie das statistische Hypothesenpaar! \textbf{(1 Punkt)}
\item Liegt ein signifikanter Unterschied zwischen den Gruppen vor? Begründen Sie Ihre Antwort! \textbf{(2 Punkte)}
\item Skizieren Sie das sich ergebende 95\% Konifidenzintervall! \textbf{(2 Punkte)}
\item Beschriften Sie die Abbildung und das 95\% Konfidenzintervall entsprechend! \textbf{(2 Punkte)}  
\item Interpretieren Sie den Effekt des 95\% Konifidenzintervalls! \textbf{(2 Punkte)}
\end{enumerate} 
\clearpage
% -----------------------------------------------------------------------

\section{Aufgabe \hfill (9 Punkte)}

\textit{Geben Sie grundsätzlich Formeln und Rechenweg zur Lösung der Teilaufgaben mit an!} \\[1Ex]
 

 
%% --------------------------------------------------------------------
\begin{minipage}[t]{0.5\textwidth}
\includegraphics[width = 1.3cm]{/Users/kruppajo/work/GitHub/exam/avatare/Jessica.png}\hspace{-4mm}\includegraphics[width = 1.3cm]{/Users/kruppajo/work/GitHub/exam/avatare/Paula.png}\hspace{-4mm}\includegraphics[width = 1.3cm]{/Users/kruppajo/work/GitHub/exam/avatare/Yuki.png}
\end{minipage}
\begin{minipage}[t]{0.5\textwidth}
\hfill
\href{https://youtu.be/w62HJlbN28U}{\includegraphics[width = 2cm]{img/youtube}}
\end{minipage}
%% --------------------------------------------------------------------



\paragraph{Interpretation des t-Tests in \Rlogo - die Visualisierung}

'Mit dem R Paket \texttt{\{emmeans\}} können wir gleich die Gruppenvergleiche rechnen und uns das \textit{compact letter displac}' ausgeben lassen!', verkündet Jessica sichtlich stolz. Ein paar Mal hat sie schon der Mangel gehindert weiterzumachen. 'Nach Meinung der Betreuerin soll es aber nur erstmal ein t-Test sein. Und die Ausgabe ist schon wirr genug.', merkt Yuki an. Yuki und Paula sind bei Jessica um sich in \Rlogo helfen zu lassen. Im Hintergrund wummert David Bowie. Paula streichelt zur Beruhigung die Hündin von Jessica. Die beiden waren 3 Monate in der Uckermark um einen Versuch mit Hühnern in einem Kreuzungsexperiment durchzuführen. Ziel war es das Outcome ($Y$) Protein/Fettrate [\%/kg] zu bestimmen. Jessica überlegt, ob sie die beiden nicht noch auf den Film \textit{Herr der Ringe} einlädt oder dann doch lieber raus geht um Rad zu fahren? Vielleicht will ja Paula mit. Besser als der Film.

\begin{knitrout}
\definecolor{shadecolor}{rgb}{0.969, 0.969, 0.969}\color{fgcolor}\begin{kframe}
\begin{verbatim}
## 
## 	Two Sample t-test
## 
## data:  Protein/Fettrate by Lüftungssystem
## t = 2.3967, df = 16, p-value = 0.02911
## alternative hypothesis: true  is not equal to [condensed]
## 95 percent confidence interval:
##   0.9535361 15.5575751
## sample estimates:
##     mean in group keins mean in group vorhanden 
##                37.57778                29.32222
\end{verbatim}
\end{kframe}
\end{knitrout}

Helfen Sie Jessica bei der Interpretation des t-Tests! Sonst geht es auch für Yuki und Paula nicht weiter.
  
\begin{enumerate}
  \item Formulieren Sie die wissenschaftliche Fragestellung! \textbf{(1 Punkt)}
  \item Formulieren Sie das statistische Hypothesenpaar! \textbf{(1 Punkt)}
\item Liegt ein signifikanter Unterschied zwischen den Gruppen vor? Begründen Sie Ihre Antwort! \textbf{(2 Punkte)}
\item Skizieren Sie die sich ergebenden Boxplot! Welche Annahmen an die Daten haben Sie getroffen? Begründen Sie Ihre
  Antwort! \textbf{(2 Punkte)} 
\item Skizieren Sie die sich ergebenden Barplots! \textbf{(2 Punkte)}
\item Berechnen Sie den Effekt des t-Tests! \textbf{(1 Punkt)}
\end{enumerate}
 
\clearpage
% -----------------------------------------------------------------------

\section{Aufgabe \hfill (10 Punkte)}

\textit{Geben Sie grundsätzlich Formeln und Rechenweg zur Lösung der Teilaufgaben mit an!} \\[1Ex]
 

 
%% --------------------------------------------------------------------
\begin{minipage}[t]{0.5\textwidth}
\includegraphics[width = 1.3cm]{/Users/kruppajo/work/GitHub/exam/avatare/Jonas.png}\hspace{-4mm}\includegraphics[width = 1.3cm]{/Users/kruppajo/work/GitHub/exam/avatare/Mark.png}
\end{minipage}
\begin{minipage}[t]{0.5\textwidth}
\hfill
\href{https://youtu.be/kHmfEmU6lrk}{\includegraphics[width = 2cm]{img/youtube}}
\end{minipage}
%% --------------------------------------------------------------------



\paragraph{Interpretation des gepaarten t-Tests in \Rlogo}

Mark und Jonas haben sich dazu entschieden zusammenzuarbeiten. Das sollte alles etwas einfacher machen. Jeder hat zwar ein getrenntes Themenfeld aber den Hauptversuch machen beide gemeinsam. Das hat sich schonmal als gut Idee soweit herausgestellt. Eine echte Herausforderung für ihn war schon immer die Unsicherheit gewesen. Ein leidiges Lied. In einer Abschlussarbeit sollen beide herausfinden, ob es einen Zusammenhang zwischen Genotypisierung ($0d$ und $14d$) und Protein/Fettrate [\%/kg] gibt. Die Besonderheit ist hierbei, dass die Messungen an der gleichen Beobachtung stattfinden. Beide messen also zweimal an den gleichen Fleischrindern. Hier muss dann wohl auf ein normalverteiltes Outcome ($Y$) ein gepaarter t-Test gerechnet werden. Leider kennen sich beide nicht sehr gut in \Rlogo aus. Aber wenigtens haben beide eine Menge an Marzipankugeln.

\begin{knitrout}
\definecolor{shadecolor}{rgb}{0.969, 0.969, 0.969}\color{fgcolor}\begin{kframe}
\begin{verbatim}
## 
## 	Paired t-test
## 
## data:  Protein/Fettrate by Genotypisierung
## t = -1.5124, df = 8, p-value = 0.1689
## alternative hypothesis: true  is not equal to [condensed]
## 95 percent confidence interval:
##  -14.194834   2.950389
## sample estimates:
## mean difference 
##       -5.622222
\end{verbatim}
\end{kframe}
\end{knitrout}

Jetzt brauchen Mark und Jonas Ihre Hilfe bei der Berechnung eines gepaarten t-Tests in \Rlogo um ihre Arbeit dann in diesem Semester noch abschließen zu können.

\begin{enumerate}
  \item Formulieren Sie die wissenschaftliche Fragestellung! \textbf{(1 Punkt)}
  \item Formulieren Sie das statistische Hypothesenpaar! \textbf{(1 Punkt)}
\item Liegt ein signifikanter Unterschied zwischen den Gruppen vor?
  Begründen Sie Ihre Antwort! \textbf{(2 Punkte)}
\item Skizzieren Sie das sich ergebende 95\% Konfidenzintervall! \textbf{(2 Punkte)}
\item Interpretieren Sie den Effekt des gepaarten t-Tests! \textbf{(2 Punkte)}
\item Skizzieren Sie den sich ergebenden Boxplot der Differenzen! Welche Annahmen an die Daten haben Sie getroffen? Begründen Sie Ihre Antwort! \textbf{(2 Punkte)} 
\end{enumerate}
 
\clearpage
% -----------------------------------------------------------------------
\part{Die einfaktorielle \& zweifaktorielle ANOVA}
% -----------------------------------------------------------------------

\section{Aufgabe \hfill (11 Punkte)}

\textit{Geben Sie grundsätzlich Formeln und Rechenweg zur Lösung der Teilaufgaben mit an!} \\[1Ex]
 

 
%% --------------------------------------------------------------------
\begin{minipage}[t]{0.5\textwidth}
\includegraphics[width = 1.3cm]{/Users/kruppajo/work/GitHub/exam/avatare/Nilufar.png}\hspace{-4mm}\includegraphics[width = 1.3cm]{/Users/kruppajo/work/GitHub/exam/avatare/Yuki.png}
\end{minipage}
\begin{minipage}[t]{0.5\textwidth}
\hfill
\href{https://youtu.be/kHmfEmU6lrk}{\includegraphics[width = 2cm]{img/youtube}}
\end{minipage}
%% --------------------------------------------------------------------



\paragraph{Visualisierung der einfaktoriellen ANOVA}

'Als erstes visualiseren wir unsere Daten und dann können wir schon abschätzen, ob unser Gruppenvergleich in der ANOVA signifikant werden würde?', Nilufar schaut Yuki fragend an und hofft auf eine positive Regung im Gesicht. Wird aber enttäuscht. Die beiden hatten sich auf einem Konzert von London Grammar kennengelernt. Yuki tut sich auch sehr schwer mit der einfaktoriellen ANOVA. Beide waren im Teuteburgerwald um ein Stallexperiment mit Milchvieh durchzuführen. Dabei ging es herauszufinden, ob es einen Zusammenhang zwischen der Behandlung Elterlinie ($Standard$, $Yray$ und $Xray$) und dem Messwert Fettgehalt [\%/kg] gibt. Später wird noch Matrix geguckt. Yuki befürwortet das!

\begin{knitrout}
\definecolor{shadecolor}{rgb}{0.969, 0.969, 0.969}\color{fgcolor}\begin{table}[!h]
\centering
\begin{tabular}{cc}
\toprule
Elterlinie & Fettgehalt\\
\midrule
Yray & 44\\
Yray & 44\\
Standard & 35\\
Xray & 22\\
Yray & 45\\
\addlinespace
Standard & 34\\
Xray & 24\\
Standard & 35\\
Standard & 36\\
Standard & 35\\
\addlinespace
Standard & 35\\
Yray & 45\\
Yray & 45\\
Xray & 26\\
Xray & 23\\
\addlinespace
Xray & 29\\
Yray & 45\\
Standard & 35\\
Xray & 26\\
\bottomrule
\end{tabular}
\end{table}

\end{knitrout}

Leider kennen sich Nilufar und Yuki mit Darstellung einer einfaktoriellen ANOVA überhaupt nicht aus. 

\begin{enumerate}
\item Erstellen  Sie  eine  Visualisierung  der  Datentabelle! Beschriften  Sie  die  Abbildung! \textbf{(2 Punkte)}
\item Benennen Sie die Visualisierung mit dem korrekten, statistischen Fachbegriff! \textbf{(1 Punkt)}
\item Zeichnen Sie folgende statistischen Maßzahlen passend ein! 
  \begin{itemize}
  \item Globale Mittelwert: $\beta_0$ \textbf{(1 Punkt)}
  \item Mittelwerte der einzelnen Behandlungsstufen: $\bar{y}_{0.5}$, $\bar{y}_{1.5}$ und $\bar{y}_{2.5}$ \textbf{(1 Punkt)}
  \item Mittelwertsdifferenz der einzelnen Behandlungsstufen: $\beta_{0.5}$, $\beta_{1.5}$ und $\beta_{2.5}$ \textbf{(1 Punkt)}
  \item Residuen oder Fehler: $\epsilon$ \textbf{(1 Punkt)}
  \end{itemize}
\item Liegt ein \textit{vermutlicher} signifikanter Unterschied vor? Begründen Sie Ihre Antwort! \textbf{(2 Punkte)}
\item Schätzen Sie die Effekte der Behandlungsstufen! \textbf{(2 Punkte)}
\end{enumerate}
 
\clearpage
% -----------------------------------------------------------------------

\section{Aufgabe \hfill (9 Punkte)}

\textit{Geben Sie grundsätzlich Formeln und Rechenweg zur Lösung der Teilaufgaben mit an!} \\[1Ex]
 

 
%% --------------------------------------------------------------------
\begin{minipage}[t]{0.5\textwidth}
\includegraphics[width = 1.3cm]{/Users/kruppajo/work/GitHub/exam/avatare/Alex.png}\hspace{-4mm}\includegraphics[width = 1.3cm]{/Users/kruppajo/work/GitHub/exam/avatare/Mark.png}
\end{minipage}
\begin{minipage}[t]{0.5\textwidth}
\hfill
\href{https://youtu.be/IhecxMcCENY}{\includegraphics[width = 2cm]{img/youtube}}
\end{minipage}
%% --------------------------------------------------------------------



\paragraph{Ergebnistabelle der einfaktoriellen ANOVA}

Alex und Mark schauen sich etwas entnervt an. Gemeinsam schreiben die beiden ihre Abschlussarbeit und sollen nun als erstes einmal die Daten mit eine einfaktoriellen ANOVA auswerten damit abgeschätzt werden kann, ob überhaupt signifikante Ergebnisse in den multipen Gruppenvergleichen zu erwarten sind. Da hilft der Hamster von Mark auch nur bedingt. Die beiden waren im Teuteburgerwald um ein Kreuzungsexperiment mit Zandern durchzuführen. Dabei haben Alex und Mark den Messwert Schlachtgewicht [kg] unter der Behandung Lüftungssystem ($keins$, $storm$, $tornado$ und $thunder$) ermittelt. Nachher wollen sich beide noch mit dem Hobby Geocaching von Mark beschäftigen. Kennt Alex noch nicht, klingt aber interessant.



\vspace{1ex}

Leider kennen sich Alex und Mark mit Berechnung einer einfaktoriellen ANOVA überhaupt nicht aus. Deshalb brauchen beide bei der Erstellung Ihre Hilfe, der Hamster reicht als Hilfe nicht aus! 

\begin{enumerate}
  \item Formulieren Sie die wissenschaftliche Fragestellung! \textbf{(1 Punkt)}
  \item Formulieren Sie das statistische Hypothesenpaar! \textbf{(1 Punkt)}
\item Füllen Sie die unterstehende einfaktorielle ANOVA Ergebnistabelle aus! \textbf{(3 Punkte)}
\end{enumerate}

\vspace{1Ex}

\begin{center}
  \Large
  \begin{tabular}{lccccp{3cm}}
\toprule
     & \textbf{Df} & \textbf{Sum Sq} & \textbf{Mean Sq} & \textbf{F value} & \textbf{Pr(>F)} \strut\\
    \midrule
   \textbf{Lüftungssystem}  & 3 &  &  &  &  \strut\\
   \textbf{error}  & 27 & 375.48 &  &  &  \strut\\
   \textbf{Total}  & 30 & 547.1 &  &  &  \strut\\
\bottomrule
  \end{tabular}
\end{center}

\vspace{1Ex}

\begin{enumerate}
  \setcounter{enumi}{3}
\item Schätzen Sie den p-Wert der Tabelle mit $F_{\alpha = 5\%} = 2.96$ ab. Begründen Sie Ihre Antwort! \textbf{(2 Punkte)}
\item Berechen Sie den Effektschätzer $\eta^2$. Was sagt Ihnen der Wert von $\eta^2$ aus? \textbf{(2 Punkte)}
\end{enumerate}



 
\clearpage
% -----------------------------------------------------------------------

\section{Aufgabe \hfill (12 Punkte)}

\textit{Geben Sie grundsätzlich Formeln und Rechenweg zur Lösung der Teilaufgaben mit an!} \\[1Ex]
 

 
%% --------------------------------------------------------------------
\begin{minipage}[t]{0.5\textwidth}
\includegraphics[width = 1.3cm]{/Users/kruppajo/work/GitHub/exam/avatare/Jonas.png}\hspace{-4mm}\includegraphics[width = 1.3cm]{/Users/kruppajo/work/GitHub/exam/avatare/Nilufar.png}
\end{minipage}
\begin{minipage}[t]{0.5\textwidth}
\hfill
\href{https://youtu.be/49hvImMwVyE}{\includegraphics[width = 2cm]{img/youtube}}
\end{minipage}
%% --------------------------------------------------------------------



\paragraph{Die einfaktoriellen ANOVA und der Student t-Test}

'Uff... die einfaktorielle ANOVA. Und wie füllen wir jetzt 	extit{genau} die Tabelle der ANOVA aus und schauen, ob da was signifikant ist?', Nilufar hebt die Augenbraue. 'Das ist eine sehr gute Frage. Ich glaube man kann alles in der Tabelle relativ einfach mit wenigen Informationen berechnen.', meint Jonas dazu und schmiß sich noch ein paar Snickers in den Rachen. Nilufar hatte sich in ein Gewächshausexperiment verschiedene Spargel angeschaut. Dabei ging es herauszufinden, ob es einen Zusammenhang zwischen der Behandlung Substrattypen ($torf$, $40p60n$, $30p20n$ und $70p30n$) und dem Messwert Chlorophyllgehalt (SPAD-502Plus) [SPAD] gibt. Nun möchte erstmal ihr Betreuer eine ANOVA Tabelle sehen. Was immer da auch drin zu erkennen sein mag. Später wollen die beiden dann noch raus um zu Schwimmen.



\vspace{1ex}

Leider kennen sich Nilufar und Jonas mit Berechnung einer einfaktoriellen ANOVA überhaupt nicht aus. Deshalb brauchen beide bei der Erstellung Ihre Hilfe! 

\begin{enumerate}
  \item Formulieren Sie die wissenschaftliche Fragestellung! \textbf{(1 Punkt)}
  \item Formulieren Sie das statistische Hypothesenpaar! \textbf{(1 Punkt)}
\item Füllen Sie die unterstehende einfaktorielle ANOVA Ergebnistabelle aus! \textbf{(3 Punkte)}
\end{enumerate}

\vspace{1Ex}

\begin{center}
  \Large
  \begin{tabular}{lccccp{3cm}}
\toprule
     & \textbf{Df} & \textbf{Sum Sq} & \textbf{Mean Sq} & \textbf{F value} & \textbf{Pr(>F)} \strut\\
    \midrule
   \textbf{Substrattypen}  & 3 & 294.96 &  &  &  \strut\\
   \textbf{Error}  & 30 & 695.3 &  &  &  \strut\\
\bottomrule
  \end{tabular}
\end{center}

\vspace{1Ex}

\begin{enumerate}
  \setcounter{enumi}{3}
\item Schätzen Sie den p-Wert der Tabelle mit $F_{\alpha = 5\%} = 2.92$ ab. Begründen Sie Ihre Antwort! \textbf{(2 Punkte)}
\item Was bedeutet ein signifikantes Ergebnis in einer einfaktoriellen ANOVA? \textbf{(1 Punkt)}
\item Berechnen Sie \textit{einen} Student t-Test für den \textit{vermutlich} signifikantesten Gruppenvergleich anhand der untenstehenden Tabelle mit $T_{\alpha = 5\%} = 2.03$. Begründen Sie Ihre Auswahl! \textbf{(3 Punkte)}
\end{enumerate}


\begin{knitrout}
\definecolor{shadecolor}{rgb}{0.969, 0.969, 0.969}\color{fgcolor}\begin{table}[!h]
\centering\begingroup\fontsize{11}{13}\selectfont

\begin{tabular}{cccc}
\toprule
\textbf{Substrattypen} & \textbf{Fallzahl (n)} & \textbf{Mittelwert} & \textbf{Standardabweichung}\\
\midrule
torf & 5 & 14.00 & 7.75\\
40p60n & 9 & 5.33 & 3.57\\
30p20n & 10 & 7.10 & 4.38\\
70p30n & 10 & 5.40 & 4.48\\
\bottomrule
\end{tabular}
\endgroup{}
\end{table}

\end{knitrout}


\begin{enumerate}
  \setcounter{enumi}{6}
\item Gegebenen der ANOVA Tabelle war das Ergebnis des Student t-Tests zu erwarten? Begründen Sie Ihre Antwort! \textbf{(2 Punkte)}
\end{enumerate}

 
\clearpage
% -----------------------------------------------------------------------

\section{Aufgabe \hfill (9 Punkte)}

\textit{Geben Sie grundsätzlich Formeln und Rechenweg zur Lösung der Teilaufgaben mit an!} \\[1Ex]
 

 
%% --------------------------------------------------------------------
\begin{minipage}[t]{0.5\textwidth}
\includegraphics[width = 1.3cm]{/Users/kruppajo/work/GitHub/exam/avatare/Alex.png}
\end{minipage}
\begin{minipage}[t]{0.5\textwidth}
\hfill
\href{https://youtu.be/aXvxGC4YLqk}{\includegraphics[width = 2cm]{img/youtube}}
\end{minipage}
\vspace{-3Ex}
%% --------------------------------------------------------------------



\paragraph{Die einfaktorielle ANOVA in \Rlogo}

'Uff... die einfaktorielle ANOVA und \Rlogo. Nicht so einfach... Was sagt mir jetzt die Ausgabe der ANOVA und wo sehe ich, ob da was signifikant ist?', denkt Alex und hebt die Augenbraue. Alex hatte sich ein Stallexperiment mit Lamas angeschaut. Als wäre das nicht alles schon schwer genug. Alex und die Gefälligkeit, eine unendliche Geschichte mit kniffeligen Wendungen. Dabei ging es beim Experiment herauszufinden, ob es einen Zusammenhang zwischen der Behandlung Ernährungszusatz ($ctrl$, $fedX$ und $getIt$) und dem Messwert Gewichtszuwachs in der 1LW gibt. Nun möchte seine Betreuerin seinem Projektbericht erstmal eine ANOVA sehen und die Ergebnisse präsentiert bekommen. Und eigentlich will er ja was anderes... Hm, lecker Gummibärchen und dazu dann im Hintergrund Alien laufen lassen.

\begin{knitrout}
\definecolor{shadecolor}{rgb}{0.969, 0.969, 0.969}\color{fgcolor}\begin{kframe}
\begin{verbatim}
## Analysis of Variance Table
## 
## Response: Gewichtszuwachs
##                  Df Sum Sq Mean Sq F value Pr(>F)
## Ernährungszusatz  2 104.43  52.215  1.5854 0.2373
## Residuals        15 494.01  32.934
\end{verbatim}
\end{kframe}
\end{knitrout}

\vspace{1ex}

Leider kennen sich Alex mit Berechnung einer einfaktoriellen ANOVA überhaupt nicht aus. Deshalb braucht er bei der Erstellung Ihre Hilfe! 

\begin{enumerate}
  \item Formulieren Sie die wissenschaftliche Fragestellung! \textbf{(1 Punkt)}
  \item Formulieren Sie das statistische Hypothesenpaar! \textbf{(1 Punkt)}
\item Interpretieren Sie das Ergebnis der einfaktoriellen ANOVA! \textbf{(2 Punkte)} 
\item Berechnen Sie den Effektschätzer $\eta^2$. Was sagt Ihnen der Wert von $\eta^2$ aus? \textbf{(2 Punkte)}
\item Skizzieren Sie eine Abbildung, der dem obigen Ergebnis der
  einfaktoriellen ANOVA näherungsweise entspricht! \textbf{(3 Punkte)}
\end{enumerate}

 
\clearpage
% -----------------------------------------------------------------------

\section{Aufgabe \hfill (12 Punkte)}

\textit{Geben Sie grundsätzlich Formeln und Rechenweg zur Lösung der Teilaufgaben mit an!} \\[1Ex]
 

 
%% --------------------------------------------------------------------
\begin{minipage}[t]{0.5\textwidth}
\includegraphics[width = 1.3cm]{/Users/kruppajo/work/GitHub/exam/avatare/Paula.png}
\end{minipage}
\begin{minipage}[t]{0.5\textwidth}
\hfill
\href{https://youtu.be/8Pb2sKUIMyk}{\includegraphics[width = 2cm]{img/youtube}}
\end{minipage}
\vspace{-3Ex}
%% --------------------------------------------------------------------



\paragraph{Ergebnistabelle der zweifaktoriellen ANOVA}

Wie absolut ärgerlich. Jetzt stellt sich tatsächlich heraus, dass ihrer Betreuerin keine Ahnung von der zweifaktoriellen ANOVA hat. Woher soll Paula jetzt das Wissen nehmen? Paula mampft aus Frust noch eine Handvoll Smarties. Immerhin muss sie ja noch mit ihrer Abschlussarbeit dieses Jahr fertig werden. In einen Leistungssteigerungsversuch hatte sie Puten mit der Behandlung Lüftungssystem ($keins$, $storm$, $tornado$ und $thunder$) sowie der Behandlung Bestandsdichte ($standard$ und $kontakt$) im Teuteburgerwald untersucht. Es wurde als Messwert Fettgehalt [\%/kg] bestimmt. Jetzt muss sie erstmal die zweifaktorielle ANOVA verstehen. Und eigentlich wollte Paula doch noch zum Sport! Um zu Fechten geht Paula dann später nochmal raus. Echte Entspannung.



\vspace{1ex}

Leider kennen sich Paula mit Berechnung einer zweifaktoriellen ANOVA überhaupt nicht aus. Deshalb braucht sie bei der Erstellung Ihre Hilfe! 

\begin{enumerate}
  \item Formulieren Sie die wissenschaftliche Fragestellung! \textbf{(1 Punkt)}
  \item Formulieren Sie das statistische Hypothesenpaar! \textbf{(1 Punkt)}
\item Füllen Sie die unterstehende einfaktorielle ANOVA Ergebnistabelle aus! \textbf{(3 Punkte)}
\end{enumerate}

\vspace{1Ex}

\begin{center}
  \Large
  \begin{tabular}{lccccc}
  \toprule
     & \textbf{Df} & \textbf{Sum Sq} & \textbf{Mean Sq} & \textbf{F value} & \textbf{Pr(>F)} \strut\\
    \midrule
   \textbf{Lüftungssystem}  & 3 & 110.59 &  &  &  \strut\\
    \textbf{Bestandsdichte}  & 1 & 92.36 &  &  &  \strut\\
    \textbf{Lüftungssystem:Bestandsdichte}  & 3 & 29.2 &  &  &  \strut\\
   \textbf{Error}  & 18 & 291.4 &  &  &  \strut\\
\bottomrule
  \end{tabular}
\end{center}

\vspace{1Ex}

\begin{enumerate}
  \setcounter{enumi}{3}
\item Schätzen Sie den p-Wert der Tabelle ab. Begründen Sie Ihre
  Antwort! \textbf{(3 Punkte)}
\end{enumerate}
  
\begin{center}
    \Large
\begin{tabular}{lc}
  \toprule
     & $\boldsymbol{F_{\alpha = 5\%}}$ \\
\midrule
  \textbf{Lüftungssystem} & $4.26$ \\
  \textbf{Bestandsdichte} & $3.40$ \\
  \textbf{Lüftungssystem:Bestandsdichte} & $5.23$ \\
  \bottomrule
  \end{tabular}
\end{center}

\begin{enumerate}
  \setcounter{enumi}{4}
\item Was bedeutet ein signifikantes Ergebnis in einer zweifaktoriellen ANOVA? \textbf{(2 Punkte)}
\item Was sagt der Term \textit{Lüftungssystem:Bestandsdichte} aus? Interpretieren Sie das Ergebnis! \textbf{(2 Punkte)}
\end{enumerate}
 
\clearpage
% -----------------------------------------------------------------------

\section{Aufgabe \hfill (10 Punkte)}

\textit{Geben Sie grundsätzlich Formeln und Rechenweg zur Lösung der Teilaufgaben mit an!} \\[1Ex]
 

 
%% --------------------------------------------------------------------
\begin{minipage}[t]{0.5\textwidth}
\includegraphics[width = 1.3cm]{/Users/kruppajo/work/GitHub/exam/avatare/Jessica.png}
\end{minipage}
\begin{minipage}[t]{0.5\textwidth}
\hfill
\href{https://youtu.be/rWTyHXXlYjY}{\includegraphics[width = 2cm]{img/youtube}}
\end{minipage}
\vspace{-3Ex}
%% --------------------------------------------------------------------



\paragraph{Die zweifaktorielle ANOVA in \Rlogo}

Es ist schon kurz nach fünf und Jessica wird langsam nervös. Jessica wollte heute Abend noch ihre E-Sport Qualifikation schauen. Stattdessen versucht ihre Betreuerin die Ausgabe der zweifaktoriellen ANOVA zu visualieren und zu überprüfen, ob es mit der Visualisierung der Daten als Boxplots zusammenpasst. Jessica hatte im Teuteburgerwald einen Leistungssteigerungsversuch mit Puten durchgeführt. Es gab dabei zwei Behandlungen. Einmal Elterlinie ($ctrl$, $Standard$, $Yray$ und $Xray$) sowie als zweite Behandlung Bestandsdichte ($standard$ und $kontakt$). Gemessen wurde der Messwert ($Y$) Fettgehalt [\%/kg]. So kompliziert kann das jetzt doch nicht sein! Eigentlich wollte Jessica nachher noch zum Sport. Einfach mal raus um Rad zu fahren. Ohne Ziel und Uhr. Das ist was für Jessica.

\begin{knitrout}
\definecolor{shadecolor}{rgb}{0.969, 0.969, 0.969}\color{fgcolor}\begin{kframe}
\begin{verbatim}
## Analysis of Variance Table
## 
## Response: Fettgehalt
##                           Df Sum Sq Mean Sq F value  Pr(>F)
## Elterlinie                 2 184.50  92.251  5.2501 0.01599
## Bestandsdichte             1 134.16 134.162  7.6352 0.01281
## Elterlinie:Bestandsdichte  2 143.97  71.984  4.0966 0.03418
## Residuals                 18 316.29  17.572
\end{verbatim}
\end{kframe}
\end{knitrout}

\vspace{1ex}

Leider kennt sich Jessica mit Berechnung einer zweifaktoriellen ANOVA überhaupt nicht aus. Deshalb braucht sie bei der Erstellung Ihre Hilfe! 

\begin{enumerate}
  \item Formulieren Sie die wissenschaftliche Fragestellung! \textbf{(1 Punkt)}
  \item Formulieren Sie das statistische Hypothesenpaar! \textbf{(1 Punkt)}
\item Interpretieren Sie das Ergebnis der einfaktoriellen ANOVA! \textbf{(3 Punkte)} 
\item Zeichnen Sie eine Abbildung, der dem obigen Ergebnis der
  zweifaktoriellen ANOVA näherungsweise entspricht! \textbf{(5 Punkte)}
\end{enumerate}
 
\clearpage
% -----------------------------------------------------------------------

\section{Aufgabe \hfill (12 Punkte)}

\textit{Geben Sie grundsätzlich Formeln und Rechenweg zur Lösung der Teilaufgaben mit an!} \\[1Ex]
 

 
%% --------------------------------------------------------------------
\begin{minipage}[t]{0.5\textwidth}
\includegraphics[width = 1.3cm]{/Users/kruppajo/work/GitHub/exam/avatare/Steffen.png}
\end{minipage}
\begin{minipage}[t]{0.5\textwidth}
\hfill
\href{https://youtu.be/FjjJXkFJfIY}{\includegraphics[width = 2cm]{img/youtube}}
\end{minipage}
\vspace{-3Ex}
%% --------------------------------------------------------------------



\paragraph{Zusammenhang zwischen der ANOVA und dem t-Test}

Steffen schaut konzentriert auf die Formeln der ANOVA und des t-Tests. In seinem Experiment wurde als Messwert Schlachtgewicht [kg] bestimmt. Steffen und die Romantik, eine unendliche Geschichte mit kniffeligen Wendungen. Dann wäre es nicht noch komplizierter. In ein Stallexperiment wurden Milchvieh mit der Behandlung Ernährungszusatz ($ctrl$, $fedX$ und $getIt$) sowie der Behandlung Lüftungssystem ($keins$ und $thunder$) untersucht. Beide Verfahren müssen etwas miteinander zu tun haben und sein Betreuer möchte das jetzt auch noch verstehen. Im Hintergrund läuft leise Harry Potter auf seinem Second Screen. Immerhin hat er die beiden Formeln vorliegen.

\begin{graybox}{Gegebene Formeln}
\begin{center}
  \begin{tabular}{cc}
    $F_{D} = \cfrac{MS_{treatment}}{MS_{error}}$ & $T_{D} = \cfrac{\bar{y}_1 - \bar{y}_2}{s_p \cdot \sqrt{2/n_g}}$\\
  \end{tabular}
\end{center}
\end{graybox}

Leider kennen sich Steffen mit dem Zusammenhang zwischen der ANOVA und dem t-Test nicht aus. Deshalb braucht er bei der Erstellung Ihre Hilfe! 

\begin{enumerate}
\item Welche statistische Maßzahl testet der t-Test, welche die ANOVA? Begründen Sie Ihre Antwort! \textbf{(2 Punkte)}
\item Erklären Sie den Zusammenhang zwischen der $F_{D}$ Statistik und $T_{D}$ Statistik! \textbf{(2 Punkte)}
\item Visualisieren Sie in einer 2x2 Tafel den Zusammenhang von $MS_{treatment}$ und $MS_{error}$! \textbf{(2 Punkte)}
\item Beschriften Sie die erstellte 2x2 Tafel mit \underline{signifikant} und \underline{nicht signifikant}! Begründen Sie Ihre Antwort! \textbf{(2 Punkte)}
\item Nennen Sie das numerische Minimum der F-Statistik $F_D$! Begründen Sie Ihre Antwort! \textbf{(2 Punkte)}
\item Wenn die F-Statistik $F_D$ minimal ist, welche Aussage erhalten Sie über die Nullhypothese? Begründen Sie Ihre Antwort! \textbf{(2 Punkte)}
\end{enumerate}

 
\clearpage
% -----------------------------------------------------------------------

\section{Aufgabe \hfill (11 Punkte)}

\textit{Geben Sie grundsätzlich Formeln und Rechenweg zur Lösung der Teilaufgaben mit an!} \\[1Ex]
 

 
%% --------------------------------------------------------------------
\begin{minipage}[t]{0.5\textwidth}
\includegraphics[width = 1.3cm]{/Users/kruppajo/work/GitHub/exam/avatare/Steffen.png}
\end{minipage}
\begin{minipage}[t]{0.5\textwidth}
\hfill
\href{https://youtu.be/2qG1Dws0MJo}{\includegraphics[width = 2cm]{img/youtube}}
\end{minipage}
\vspace{-3Ex}
%% --------------------------------------------------------------------



\paragraph{Interaktion in der zweifaktoriellen ANOVA}

Es ist schon kurz nach fünf und Steffen wird langsam nervös. Steffen wollte heute Abend noch seine E-Sport Qualifikation schauen und dann zum Sport. Stattdessen versucht sein Betreuer die Ausgabe der zweifaktoriellen ANOVA zu visualieren und zu überprüfen, ob es mit der Visualisierung der Daten als Boxplots zusammenpasst. Es liegt anscheinend eine signifikante Interaktion vor? Steffen hatte im Teuteburgerwald einen Leistungssteigerungsversuch mit Lamas durchgeführt. Es gab dabei zwei Behandlungen. Einmal Genotypen ($AA$, $AB$ und $BB$) sowie als zweite Behandlung Lüftungssystem ($keins$ und $thunder$). Gemessen wurde der Messwert ($Y$) Schlachtgewicht [kg]. So kompliziert kann das jetzt doch nicht sein! Eigentlich wollte Steffen nachher noch zum Sport. Steffen will später nochmal raus um zu Ringen. Druck ablassen, dass muss er auch.

\vspace{1ex}

Leider kennen sich Steffen und sein Betreuer mit der zweifaktoriellen ANOVA überhaupt nicht aus. Geschweige denn mit der Interpretation einer Interaktion. Deshalb braucht er bei der Erstellung Ihre Hilfe, sonst wird es heute Abend mit seinem Hobby Klemmbausteine nichts mehr! 

\begin{enumerate}
\item Visualisieren Sie folgende mögliche Interaktionen zwischen den Behandlungen! Beschriften Sie die Abbildung! \textbf{(4 Punkte)}
\begin{enumerate}
\item \underline{Keine} Interaktion liegt vor.
\item Eine \underline{schwache} Interaktion liegt vor. 
\item Eine \underline{starke} Interaktion liegt vor. 
\end{enumerate}
\item Erklären Sie den Unterschied zwischen den verschiedenen Interaktionen! \textbf{(2 Punkte)}
\item Welche statistische Maßzahl betrachten Sie für die Bewertung der Interaktion? \textbf{(1 Punkt)}
\item Skizzieren Sie die notwendigen Funktionen in \Rlogo für eine Post-hoc Analyse! \textbf{(2 Punkte)} 
\item Wenn eine signifikante Interaktion in den Daten vorliegt, wie ist dann das weitere Vorgehen? Berücksichtigen Sie auch die Funktion \texttt{emmeans()}! \textbf{(2 Punkte)}
\end{enumerate}

 
\clearpage
% -----------------------------------------------------------------------

\section{Aufgabe \hfill (11 Punkte)}

\textit{Geben Sie grundsätzlich Formeln und Rechenweg zur Lösung der Teilaufgaben mit an!} \\[1Ex]
 

 
%% --------------------------------------------------------------------
\begin{minipage}[t]{0.5\textwidth}
\includegraphics[width = 1.3cm]{/Users/kruppajo/work/GitHub/exam/avatare/Jonas.png}
\end{minipage}
\begin{minipage}[t]{0.5\textwidth}
\hfill
\href{https://youtu.be/M9Uhm67ndxM}{\includegraphics[width = 2cm]{img/youtube}}
\end{minipage}
\vspace{-3Ex}
%% --------------------------------------------------------------------



\paragraph{Zusammenhang zwischen der ANOVA und dem Post-hoc-Test}

Es ist schon kurz nach fünf und Jonas wird langsam nervös. Jonas wollte heute Abend noch seine E-Sport Qualifikation schauen. Hoffentlich kommt er noch rechtzeitig zum Streamen. Angestrengend krault er das Meerschweinchen. Stattdessen versucht seine Betreuerin die Ausgabe der einfaktoriellen ANOVA zu visualieren und zu überprüfen, ob es mit der Visualisierung der Daten als Boxplots zusammenpasst. Anscheinend gibt es ein Problem mit der Annahme der Normalverteilung und der Varianzhomogenität der ANOVA in den Daten. 'Wir haben jetzt bei der ANOVA einen p-Wert mit 0.071 raus sowie eine F-Statistik $F_D$ mit 1.78 berechnet. Nach den Boxplots müsste sich eigentlich ein Unterschied zwischen $flOw$ und $ctrl$ ergeben. Der Unterschied ist in \texttt{\{emmeans\}} auch signifikant mit einem p-Wert von 0.045. Wie kann das sein?', grummelt seine Betreuerin. Jonas hatte im Teuteburgerwald ein Stallexperiment mit Fleischrindern durchgeführt. Dabei wurden die Daten $D$ erhoben. Es gab dabei eine Behandlungen Flüssignahrung ($ctrl$, $superIn$ und $flOw$). Gemessen wurde der Messwert ($Y$) Protein/Fettrate [\%/kg]. So kompliziert kann das jetzt doch nicht sein! Jonas hat schon genug Probleme. Wenn die Erschöpfung nicht wäre, dann wäre es einfacher.

\begin{graybox}{Gegebene Formeln}
\begin{center}
  \begin{tabular}{ccc}
    $MS_{treatment} = \cfrac{SS_{treatment}}{df_{treatment}}$ &
    $MS_{error} = \cfrac{SS_{error}}{df_{error}}$ &
    $F_{D} = \cfrac{MS_{treatment}}{MS_{error}}$ \\
  \end{tabular}
\end{center}
\end{graybox}

Leider kennen sich Jonas und seine Betreuerin mit der Interpretation einer ANOVA überhaupt nicht aus. Deshalb braucht er bei der Erstellung Ihre Hilfe und die Zeit wird knapp. 

\begin{enumerate}
  \item Formulieren Sie die wissenschaftliche Fragestellung! \textbf{(1 Punkt)}
  \item Formulieren Sie das statistische Hypothesenpaar! \textbf{(1 Punkt)}
\item Was bedeutet eine signifkante ANOVA für die beobachteten Daten $D$? \textbf{(1 Punkt)}
\item Visualisieren Sie den Unterschied zwischen Varianzhomogenität und Varianzheterogenität anhand der Daten $D$! Beschriften Sie die Abbildung! \textbf{(2 Punkte)} 
\item Visualisieren Sie für die Daten $D$ die Verletzung der Annahme der Varianzhomogenität der ANOVA unter zu Hilfenahme von Boxplots! Beschriften Sie die Abbildung! \textbf{(2 Punkte)}
\item Welche Auswirkung hat die Verletzung der Annahme der Varianzhomogenität für die Teststatistik $F_D$ der ANOVA? Begründen Sie Ihre Antwort! \textbf{(2 Punkte)}
\item Erklären Sie abschließend die Diskrepanz zwischen den Ergebnis der ANOVA und dem paarweisen Gruppenvergleich in \texttt{\{emmeans\}}! \textbf{(2 Punkte)}
\end{enumerate}

 
\clearpage
% -----------------------------------------------------------------------
\part{Multiple Gruppenvergleiche}
% ----------------------------------------------------------------------- 

\section{Aufgabe \hfill (12 Punkte)}

\textit{Geben Sie grundsätzlich Formeln und Rechenweg zur Lösung der Teilaufgaben mit an!} \\[1Ex]
 

 
%% --------------------------------------------------------------------
\begin{minipage}[t]{0.5\textwidth}
\includegraphics[width = 1.3cm]{/Users/kruppajo/work/GitHub/exam/avatare/Paula.png}\hspace{-4mm}\includegraphics[width = 1.3cm]{/Users/kruppajo/work/GitHub/exam/avatare/Steffen.png}
\end{minipage}
\begin{minipage}[t]{0.5\textwidth}
\hfill
\href{https://youtu.be/kHmfEmU6lrk}{\includegraphics[width = 2cm]{img/youtube}}
\end{minipage}
%% --------------------------------------------------------------------



\paragraph{Adjustierung multipler Vergleiche}

Paula und Steffen untersuchen gemeinsam in ihrer Abschlussarbeit den Messwert Proteingehalt [g/kg] in Erdbeeren. Es ist einiges schiefgelaufen, wie es immer so passiert. Hauptsächlich waren es Katzen, auch wenn man erstmal bei dem Messwert nicht unbedingt an Katzen denken würde. Aber das ist eine andere Geschichte. Jetzt wollen Paula und Steffen ihre Ergebnisse nochmal mit einer Studie von Almar et al. (2012) vergleichen und schauen, ob was ähnliches rauskommt. Angeschaut wurde sich als Behandlung Lichtstufen ($none$, $200lm$, $400lm$, $600lm$, $700lm$ und $800lm$). Es ergab sich dann die folgende Tabelle der rohen p-Werte für die Vergleiche zu Almar et al. (2012).

\begin{knitrout}
\definecolor{shadecolor}{rgb}{0.969, 0.969, 0.969}\color{fgcolor}\begin{table}[!h]
\centering\begingroup\fontsize{10}{12}\selectfont

\begin{tabular}{ccc}
\toprule
\textbf{Rohen p-Werte} & \textbf{Adjustierte p-Werte} & \textbf{Nullhypothese ablehnen?}\\
\midrule
0.0600 &  & \\
0.0012 &  & \\
0.0120 &  & \\
0.0200 &  & \\
0.0020 &  & \\
\addlinespace
0.0800 &  & \\
\bottomrule
\end{tabular}
\endgroup{}
\end{table}

\end{knitrout}

Leider kennen sich Paula und Steffen mit der Adjustierung von $p$-Werten und dem Signifikanzniveau $\alpha$ überhaupt nicht aus. Deshalb brauchen die beiden bei der Erstellung Ihre Hilfe!

\begin{enumerate}
  \item Formulieren Sie die wissenschaftliche Fragestellung! \textbf{(1 Punkt)}
  \item Formulieren Sie die statistischen Hypothesen! \textbf{(1 Punkt)}
\item Füllen Sie die Spalte \textit{Adjustierte p-Werte} nach der Bonferoni-Methode aus! \textbf{(2 Punkte)}
\item Entscheiden Sie, ob nach der Adjustierung die Nullhypothese abgelehnt werden kann! Begründen Sie Ihre Antwort! \textbf{(2 Punkte)}
\item Wie ist Ihr Vorgehen, wenn Sie anstatt der $p$-Werte das Signifikanzniveau $\alpha$ adjustieren? \textbf{(2 Punkte)}
\item Erklären Sie warum die $p$-Werte oder das Signifikanzniveau $\alpha$ bei multiplen Vergleichen adjustiert werden müssen! \textbf{(2 Punkte)}
\item Würden Sie die Adjustierung der $p$-Werte oder die Adjustierung des Signifikanzniveaus $\alpha$ vorziehen? Begründen Sie Ihre Antwort! \textbf{(2 Punkte)}
\end{enumerate}


 
\clearpage
% ----------------------------------------------------------------------- 

\section{Aufgabe \hfill (10 Punkte)}

\textit{Geben Sie grundsätzlich Formeln und Rechenweg zur Lösung der Teilaufgaben mit an!} \\[1Ex]
 

 
%% --------------------------------------------------------------------
\begin{minipage}[t]{0.5\textwidth}
\includegraphics[width = 1.3cm]{/Users/kruppajo/work/GitHub/exam/avatare/Jessica.png}
\end{minipage}
\begin{minipage}[t]{0.5\textwidth}
\hfill
\href{https://youtu.be/xq29O8qDrg0}{\includegraphics[width = 2cm]{img/youtube}}
\end{minipage}
\vspace{-3ex}
%% --------------------------------------------------------------------



\paragraph{Visualisierung des Compact Letter Displays (CLD)}

Jessica betrachtet in sich gekehrt die Poster vor dem Büro von ihrer Betreuerin. Viele der explorativen Abbildungen sagen ihr etwas. Die Barplots und die Boxplots könnte sie dann schon nachbauen. Das macht sie dann zuversichtlich die Abschlussarbeit auch hinzukriegen. Etwas komischer sind die seltsamen Buchstaben über den Barplots. Jessica betrachtet ein Poster das sich mit Erbsen beschäftigt. Lichtstufen ($none$, $200lm$, $400lm$, $600lm$, $700lm$ und $800lm$) und Chlorophyllgehalt (SPAD-502Plus) [SPAD] wurden dort bestimmt. So richtig schlau, wird sie daraus nicht.

\begin{knitrout}
\definecolor{shadecolor}{rgb}{0.969, 0.969, 0.969}\color{fgcolor}\begin{table}[!h]
\centering\begingroup\fontsize{10}{12}\selectfont

\begin{tabular}{cc}
\toprule
\textbf{Behandlung} & \textbf{Compact letter display}\\
\midrule
none & a\\
200lm & b\\
400lm & b\\
600lm & b\\
700lm & b\\
\addlinespace
800lm & b\\
\bottomrule
\end{tabular}
\endgroup{}
\end{table}

\end{knitrout}

Leider kennen sich Jessica mit dem \textit{Compact letter display (CLD)} überhaupt nicht aus. Deshalb braucht sie bei der Erstellung Ihre Hilfe!

\begin{enumerate}
  \item Formulieren Sie die wissenschaftliche Fragestellung! \textbf{(1 Punkt)}
  \item Formulieren Sie die statistischen Hypothesen! \textbf{(1 Punkt)}
\item Zeichnen Sie die sich anhand des \textit{Compact letter display (CLD)} ergebenden Barplots! \textbf{(2 Punkte)}
\item Ergänzen Sie das \textit{Compact letter display (CLD)} zu den Barplots! \textbf{(1 Punkt)}
\item Erklären Sie \textit{einen} Vorteil und \textit{einen} Nachteil des \textit{Compact letter display (CLD)}! \textbf{(2 Punkte)}
\item Erstellen Sie eine Matrix mit den paarweisen $p$-Werten eines Student t-Tests, die sich näherungsweise aus dem \textit{Compact letter display (CLD)} ergeben würde! Begründen Sie Ihre Antwort! \textbf{(3 Punkte)}
\end{enumerate}

 
\clearpage
% ----------------------------------------------------------------------- 

\section{Aufgabe \hfill (12 Punkte)}

\textit{Geben Sie grundsätzlich Formeln und Rechenweg zur Lösung der Teilaufgaben mit an!} \\[1Ex]
 

 
%% --------------------------------------------------------------------
\begin{minipage}[t]{0.5\textwidth}
\includegraphics[width = 1.3cm]{/Users/kruppajo/work/GitHub/exam/avatare/Nilufar.png}
\end{minipage}
\begin{minipage}[t]{0.5\textwidth}
\hfill
\href{https://youtu.be/RagTFFKFbFg}{\includegraphics[width = 2cm]{img/caution}}
\end{minipage}
\vspace{-3ex}
%% --------------------------------------------------------------------



\paragraph{Berechnung des Compact Letter Displays (CLD) anhand von t-Tests}

Nilufar hatte in ihre Hausarbeit ein Stallexperiment durchgeführt. Soweit so gut. Dabei hat sie sich mit Hühnern beschäftigt. Angeblich der neueste heiße Kram... aber das ist wiederum was anderes. So richtig mitgenommen hat Nilufar das Thema dann doch nicht. Hat sie sich doch mit Flüssignahrung ($ctrl$, $superIn$, $compostIn$ und $flOw$) und Fettgehalt [\%/kg] schon eine Menge an Daten angeschaut. Nach ihre Betreuerin soll sie nun ein CLD bestimmen. Weder weiß sie was ein CLD ist, noch war ihr erster Gedanke mit Köln und die LGBTQ Community richtig... Als erstes solle sie die Gruppen nach absteigender Effektstärke sortieren. Was immer das jetzt bringen soll.

\begin{knitrout}
\definecolor{shadecolor}{rgb}{0.969, 0.969, 0.969}\color{fgcolor}\begin{table}[!h]
\centering\begingroup\fontsize{10}{12}\selectfont

\begin{tabular}{cccc}
\toprule
\textbf{Flüssignahrung} & \textbf{Fallzahl (n)} & \textbf{Mittelwert} & \textbf{Standardabweichung}\\
\midrule
ctrl & 8 & 10.47 & 2.16\\
superIn & 9 & 16.58 & 2.27\\
compostIn & 9 & 14.54 & 2.80\\
flOw & 9 & 5.77 & 1.70\\
\bottomrule
\end{tabular}
\endgroup{}
\end{table}

\end{knitrout}

Leider kennen sich Nilufar mit dem \textit{Compact letter display (CLD)} überhaupt nicht aus. Deshalb braucht sie bei der Erstellung Ihre Hilfe!

\begin{enumerate}
  \item Formulieren Sie die wissenschaftliche Fragestellung! \textbf{(1 Punkt)}
  \item Formulieren Sie die statistischen Hypothesen! \textbf{(1 Punkt)}
\item Zeichnen Sie die sich ergebenden Barplots! \textbf{(1 Punkt)}
\item Berechnen Sie die Matrix der $p$-Werte anhand von Student t-Tests! \textbf{(4 Punkte)}
\item Ergänzen Sie das \textit{Compact letter display (CLD)} zu den gezeichneten Barplots! Begründen Sie Ihre Antwort! \textbf{(4 Punkte)}
\item Interpretieren Sie das \textit{Compact letter display (CLD)} für Nilufar und Steffen! \textbf{(1 Punkt)} 
\end{enumerate}

 
\clearpage
% -----------------------------------------------------------------------

\section{Aufgabe \hfill (10 Punkte)}

\textit{Geben Sie grundsätzlich Formeln und Rechenweg zur Lösung der Teilaufgaben mit an!} \\[1Ex]
 

 
%% --------------------------------------------------------------------
\begin{minipage}[t]{0.5\textwidth}
\includegraphics[width = 1.3cm]{/Users/kruppajo/work/GitHub/exam/avatare/Mark.png}
\end{minipage}
\begin{minipage}[t]{0.5\textwidth}
\hfill
\href{https://youtu.be/RagTFFKFbFg}{\includegraphics[width = 2cm]{img/youtube}}
\end{minipage}
\vspace{-3ex}
%% --------------------------------------------------------------------



\paragraph{Berechnung des Compact Letter Displays (CLD) anhand der Matrix der p-Werte}

'Okay, dann nochmal für mich. Ich habe jetzt alles in Excel gemacht, aber das Wichtigste, was gemacht werden soll, nämlich das CLD, das kann ich nicht in Excel machen?', Mark muss sich echt beherrschen. Immerhin betreut seine Betreuerin ja erst nicht seit gestern Abschlussarbeiten und wusste ja was gemacht werden soll! Mark hatte sich zwei Variablen mit Lüftungssysteme ($ctrl$, $storm$, $thunder$ und $tornado$) und Trockengewicht [kg/ha] in einen Versuch in einer Klimakammer mit Erdbeeren angeschaut. Jetzt möchte er eigentlich fertig werden und nicht nochmal alles neu in \Rlogo und \texttt\{emmeans\} machen. Deshalb soll jetzt das CLD per Hand aus der Matrix der $p$-Wert abgeleitet werden. 'Ich glaube ich wechsel nochmal das Thema...', denkt Mark, verwirft dann aber den Gedanken.

\begin{knitrout}
\definecolor{shadecolor}{rgb}{0.969, 0.969, 0.969}\color{fgcolor}\begin{table}[!h]
\centering\begingroup\fontsize{10}{12}\selectfont

\begin{tabular}{>{}lcccc}
\toprule
\textbf{ } & \textbf{ctrl} & \textbf{storm} & \textbf{thunder} & \textbf{tornado}\\
\midrule
\textbf{ctrl} & 1.0000000 & 0.3613622 & 0.4538125 & 0.1813155\\
\textbf{storm} & 0.3613622 & 1.0000000 & 0.8921473 & 0.0329686\\
\textbf{thunder} & 0.4538125 & 0.8921473 & 1.0000000 & 0.0498533\\
\textbf{tornado} & 0.1813155 & 0.0329686 & 0.0498533 & 1.0000000\\
\bottomrule
\end{tabular}
\endgroup{}
\end{table}

\end{knitrout}

Leider kennen sich Mark mit dem \textit{Compact letter display (CLD)} überhaupt nicht aus. Deshalb braucht er bei der Erstellung Ihre Hilfe!

\begin{enumerate}
  \item Formulieren Sie die wissenschaftliche Fragestellung! \textbf{(1 Punkt)}
  \item Formulieren Sie die statistischen Hypothesen! \textbf{(1 Punkt)}
\item Zeichnen Sie die sich anhand der Matrix der $p$-Werte ergebenden Barplots! \textbf{(2 Punkte)}
\item Ergänzen Sie das \textit{Compact letter display (CLD)}! Begründen Sie Ihre Antwort! \textbf{(4 Punkte)}
\item Interpretieren Sie das \textit{Compact letter display (CLD)} für Mark und Steffen! \textbf{(2 Punkte)} 
\end{enumerate}

 
\clearpage
% -----------------------------------------------------------------------
\part{Der Chi-Quadrat-Test \& Der diagnostische Test}
% -----------------------------------------------------------------------

\section{Aufgabe \hfill (12 Punkte)}

\textit{Geben Sie grundsätzlich Formeln und Rechenweg zur Lösung der Teilaufgaben mit an!} \\[1Ex]
 

 
%% --------------------------------------------------------------------
\begin{minipage}[t]{0.5\textwidth}
\includegraphics[width = 1.3cm]{/Users/kruppajo/work/GitHub/exam/avatare/Alex.png}
\end{minipage}
\begin{minipage}[t]{0.5\textwidth}
\hfill
\href{https://youtu.be/-Kva5wc5Elw}{\includegraphics[width = 2cm]{img/youtube}}
\end{minipage}
\vspace{-3Ex}
%% --------------------------------------------------------------------



\paragraph{Den Chi-Quadrat-Test berechnen}

Alex hat sich ein Herz gefasst und war für seiner Abschlussarbeit in die Niederlande gegangen. Das war eine super Zeit in der er viel gelernt hat. Klar gab es auch die ein oder andere Besonderheit, aber das gehört hier eher nicht hin. Dann noch schnell Abba auf das Ohr und los gehts. Alex ist schon eine ganze Zeit im Büro, da sein Betreuer möchte, dass er jetzt auf seinen Daten mit $n = 132$ Beobachtungen von Lauch einen $\mathcal{X}^2$-Test rechnet. Das ginge, da er als Behandlung \textit{Pestizideinsatz [ja/nein]} bestimmt und zum anderen die Variable \textit{Trockengewicht über Zielwert [ja/nein]} ermittelt hat. Wie genau, das ist jetzt eine andere Frage. Eigentlich wollte Alex nachher noch einen Film schauen. Das Verrückte ist, dass die Katze Alien wirklich liebt. Das ist Alex sehr recht, denn er braucht Entspannung.

\vspace{5Ex}

\begin{center}
  \huge
  \begin{tabular}{c|c|c|c}
     & \phantom{\textbf{Erkrankt (ja)}} & \phantom{\textbf{Erkrankt (ja)}} & \phantom{\textbf{Erkrankt (ja)}} \strut\\
    \hline
    \phantom{\textbf{Pestizid (ja)}} & 38  & 19  &     \strut\\
    \hline
    \phantom{\textbf{Pestizid (ja)}} & 23  & 52  &      \strut\\
    \hline
     \phantom{100} & \phantom{100}  & \phantom{100}  &  \phantom{100}  \strut\\
  \end{tabular}
\end{center}

\vspace{5Ex}

Leider kennt sich Alex mit der Berechnung eines $\mathcal{X}^2$-Test für kategoriale Daten überhaupt nicht aus. Deshalb braucht er bei der Erstellung Ihre Hilfe!

\begin{enumerate}
\item Formulieren Sie die wissenschaftliche Fragestellung! \textbf{(1 Punkt)}
\item Ergänzen Sie die Tabelle um die fehlenden Informationen! \textbf{(1 Punkt)} 
\item Visualisieren Sie den Zusammenhang zwischen den beiden kategorialen Variablen! \textbf{(2 Punkte)}
\item Berechnen Sie die Teststatistik eines Chi-Quadrat-Test! \textbf{(2 Punkte)}
\item Treffen Sie eine Entscheidung im Bezug zu der Nullhypothese gegeben
  einem $\mathcal{X}^2_{\alpha = 5\%} = 3.841$! Begründen Sie Ihre Antwort!
  \textbf{(2 Punkte)}
\item Skizzieren Sie die $\mathcal{X}^2$-Verteilung, wenn die $H_0$ wahr ist! Ergänzen Sie  $\mathcal{X}^2_{\alpha = 5\%}$ und $\mathcal{X}^2_{D}$ in der Abbildung! \textbf{(2 Punkte)}
\item Berechnen Sie den Effektschätzer $Cramers\; V$! Interpretieren Sie den
  Effektschätzer! \textbf{(2 Punkte)}
\end{enumerate} 
\clearpage
% -----------------------------------------------------------------------

\section{Aufgabe \hfill (10 Punkte)}

\textit{Geben Sie grundsätzlich Formeln und Rechenweg zur Lösung der Teilaufgaben mit an!} \\[1Ex]
 

 
%% --------------------------------------------------------------------
\begin{minipage}[t]{0.5\textwidth}
\includegraphics[width = 1.3cm]{/Users/kruppajo/work/GitHub/exam/avatare/Nilufar.png}
\end{minipage}
\begin{minipage}[t]{0.5\textwidth}
\hfill
\href{https://youtu.be/jakM7fHyZfU}{\includegraphics[width = 2cm]{img/youtube}}
\end{minipage}
\vspace{-3Ex}
%% --------------------------------------------------------------------



\paragraph{Der Chi-Quadrat-Test konzeptionell verstehen}

Am Ende hätte Nilufar dann doch einen normalverteilten Endpunkt in ihrer Hausarbeit nehmen sollen. Dann noch schnell Takis Blue Heat zur Stärkung und los gehts. Vor ihr liegen jetzt die Daten von zwei Variablen als Kategorien oder wie es in \Rlogo so schön heißt, als Faktoren. Aber immerhin, hofft sie das was bei den Daten rausgekommen ist. Gezählt hat Nilufar einiges mit $n = 121$ Beobachtungen von Puten. Zum einen hat sie als Behandlung \textit{Automatische Fütterung [ja/nein]} bestimmt und zum anderen die Variable \textit{Fettgehalt erreicht [ja/nein]} ermittelt. Nun möchte ihr Betreuer gerne einen $\mathcal{X}^2$-Test auf einer $2x2$-Kreuztabelle berechnet bekommen. Am Ende des Tages möchte sie dann noch ihr Hobby Hip Hop genießen. Das muss auch mal sein!

\vspace{5Ex}

\begin{center}
  \huge
  \begin{tabular}{c|c|c|c}
     & \phantom{\textbf{Erkrankt (ja)}} & \phantom{\textbf{Erkrankt (ja)}} & \phantom{\textbf{Erkrankt (ja)}} \strut\\
    \hline
   \phantom{\textbf{Pestizid (ja)}} & \phantom{100}  & \phantom{100}  &   46  \strut\\
    \hline
    \phantom{\textbf{Pestizid (ja)}} & \phantom{100}  & \phantom{100}  &   75   \strut\\
    \hline
     &  57 &  64 &  121  \strut\\
  \end{tabular}
\end{center}

\vspace{5Ex}

Leider kennt sich Nilufar mit der Berechnung eines $\mathcal{X}^2$-Test für kategoriale Daten überhaupt nicht aus. Deshalb braucht sie bei der Erstellung Ihre Hilfe!

\begin{enumerate}
  \item Formulieren Sie die wissenschaftliche Fragestellung! \textbf{(1 Punkt)}
\item Ergänzen Sie die Tabelle um die fehlenden Informationen! \textbf{(1 Punkt)} 
\item Ergänzen Sie die Felder innerhalb der $2x2$ Kreuztabelle, so dass \textit{kein} signifikanter Effekt zu erwarten wäre! \textbf{(2 Punkte)}
\item Begründen Sie Ihr Vorgehen an der Formel des Chi-Quadrat-Tests. Erklären Sie Ihr Vorgehen an einem Beispiel! \textbf{(2 Punkte)}
\item Visualisieren Sie den Zusammenhang zwischen den beiden kategorialen Variablen! \textbf{(2 Punkte)}
\item Was ist die Mindestanzahl an Beobachtungen je Zelle? Wenn in einer der Zellen weniger Beobachtungen auftreten, welchen Test können Sie anstatt des Standard Chi-Quadrat-Tests anwenden? \textbf{(2 Punkte)}
\end{enumerate} 
\clearpage
% -----------------------------------------------------------------------

\section{Aufgabe \hfill (10 Punkte)}

\textit{Geben Sie grundsätzlich Formeln und Rechenweg zur Lösung der Teilaufgaben mit an!} \\[1Ex]
 

 
%% --------------------------------------------------------------------
\begin{minipage}[t]{0.5\textwidth}
\includegraphics[width = 1.3cm]{/Users/kruppajo/work/GitHub/exam/avatare/Jonas.png}
\end{minipage}
\begin{minipage}[t]{0.5\textwidth}
\hfill
\href{https://youtu.be/ghArbetOr_E}{\includegraphics[width = 2cm]{img/youtube}}
\end{minipage}
\vspace{-3Ex}
%% --------------------------------------------------------------------



\paragraph{Der Chi-Quadrat-Test in \Rlogo}


Am Ende war es für Jonas in seiner Abschlussarbeit dann doch kein normalverteiltes Outcome. Das was jetzt etwas doof, da er sich auf eine ANOVA gefreut hatte. Dann noch schnell Mission Impossible starten und los gehts mit der Kraft von Snickers. Prinzipiell ginge das auch irgendwie, aber nun möchte seine Betreuerin gerne einen $\mathcal{X}^2$-Test auf einer $2x2$-Kreuztabelle berechnet bekommen. Jonas hatte sich in ein Gewächshausexperiment $n = 121$ Beobachtungen von Erdbeeren angeschaut. Dabei hat er als Behandlung \textit{KI-gesteuert [ja/nein]} bestimmt und zum anderen die Variable \textit{Proteingehalt im Zielbereich [ja/nein]} ermittelt. Jetzt muss Jonas mal schauen, wie er das jetzt rechnet. Nach seinem Experiment erhielt er folgende $2x2$ Kreuztabelle aus seinen erhobenen Daten.

\begin{knitrout}
\definecolor{shadecolor}{rgb}{0.969, 0.969, 0.969}\color{fgcolor}\begin{kframe}
\begin{verbatim}
##                             KI-gesteuert
## Proteingehalt im Zielbereich ja nein
##                         ja    6    8
##                         nein  6    4
\end{verbatim}
\end{kframe}
\end{knitrout}

Dann rechnete Jonas den Fisher-Exakt-Test auf der $2x2$-Kreuztabelle in \Rlogo und erhielt folgende \Rlogo Ausgabe der Funktion \texttt{fisher.test()}.

\begin{knitrout}
\definecolor{shadecolor}{rgb}{0.969, 0.969, 0.969}\color{fgcolor}\begin{kframe}
\begin{verbatim}
## 
## 	Fisher's Exact Test for Count Data
## 
## data:  Proteingehalt im Zielbereich
## p-value = 0.6802
## alternative hypothesis: true odds ratio is not equal to 1
## 95 percent confidence interval:
##  0.06993249 3.40089546
## sample estimates:
## odds ratio 
##  0.5149174
\end{verbatim}
\end{kframe}
\end{knitrout}

Leider kennt sich Jonas mit der Berechnung eines $\mathcal{X}^2$-Test für kategoriale Daten überhaupt nicht aus. Deshalb braucht er bei der Erstellung Ihre Hilfe!

\begin{enumerate}
\item Formulieren Sie die wissenschaftliche Fragestellung! \textbf{(1 Punkt)}
\item Visualisieren Sie den Zusammenhang zwischen den beiden kategorialen Variablen! \textbf{(2 Punkte)}
\item Liegt ein signifikanter Unterschied zwischen den Gruppen vor? Begründen Sie Ihre Antwort! \textbf{(2 Punkte)}
\item Skizzieren Sie das sich ergebende 95\% Konfidenzintervall! \textbf{(2 Punkte)}
\item Beschriften Sie die Abbildung des 95\% Konfidenzintervalls! \textbf{(1 Punkt)} 
\item Interpretieren Sie das \textit{Odds ratio} im Kontext der wissenschaftlichen Fragestellung! \textbf{(2 Punkte)} 
\end{enumerate}
 
\clearpage
% -----------------------------------------------------------------------

\section{Aufgabe \hfill (11 Punkte)}

\textit{Geben Sie grundsätzlich Formeln und Rechenweg zur Lösung der Teilaufgaben mit an!} \\[1Ex]
 

 
%% --------------------------------------------------------------------
\begin{minipage}[t]{0.5\textwidth}
\includegraphics[width = 1.3cm]{/Users/kruppajo/work/GitHub/exam/avatare/Nilufar.png}\hspace{-4mm}\includegraphics[width = 1.3cm]{/Users/kruppajo/work/GitHub/exam/avatare/Tina.png}
\end{minipage}
\begin{minipage}[t]{0.5\textwidth}
\hfill
\href{https://youtu.be/VQlNl8hvRII}{\includegraphics[width = 2cm]{img/youtube}}
\end{minipage}
%% --------------------------------------------------------------------



\paragraph{Den diagnostische Test am Doppelbaum berechnen}

Nilufar liest laut vor. 'Die Prävalenz von Klauenseuche bei Hühnern wird mit 3\% angenommen. In 75\% der Fälle ist ein Test positiv, wenn das Tier erkrankt ist. In 8.0\% der Fälle ist ein Test positiv, wenn das Tier \underline{nicht} erkrankt ist und somit gesund ist. Wir führen einen Test auf Tuberkulose an 1000 Hühnern mit einem diagnostischen Test durch.' Tina klappt die Kinnlade runter. In der Stille duddelt Deichkind. Nilufar schaut kompetent und schmeißt sich mit offenen Mund Katjes an den Kopf vorbei.

\begin{center}
  \includegraphics[width=17cm]{/Users/kruppajo/work/GitHub/exam/question/img/diag-doppelbaum}
\end{center}

Leider kennen sich Nilufar und Tina mit dem diagnostischen Testen überhaupt nicht aus. Deshalb brauchen beide bei der Erstellung Ihre Hilfe! 
    
\begin{enumerate}
\item Beschriften Sie die Äste des Doppelbaumes, mit denen Ihnen bekannten Informationen! \textbf{(2 Punkte)}
\item Beschriften Sie den Doppelbaum! \textbf{(2 Punkte)}
\item Füllen Sie freien Felder des Doppelbaums aus! \textbf{(4 Punkte)}
\item Berechnen Sie die Wahrscheinlichkeit $Pr(K^+|T^+)$! \textbf{(2 Punkte)}
\item Was sagt Ihnen die Wahrscheinlichkeit $Pr(K^+|T^+)$ aus? \textbf{(1 Punkt)}
\end{enumerate}






 
\clearpage
% -----------------------------------------------------------------------

\section{Aufgabe \hfill (11 Punkte)}

\textit{Geben Sie grundsätzlich Formeln und Rechenweg zur Lösung der Teilaufgaben mit an!} \\[1Ex]
 

 
%% --------------------------------------------------------------------
\begin{minipage}[t]{0.5\textwidth}
\includegraphics[width = 1.3cm]{/Users/kruppajo/work/GitHub/exam/avatare/Steffen.png}\hspace{-4mm}\includegraphics[width = 1.3cm]{/Users/kruppajo/work/GitHub/exam/avatare/Tina.png}
\end{minipage}
\begin{minipage}[t]{0.5\textwidth}
\hfill
\href{https://youtu.be/_7s44pbOc00}{\includegraphics[width = 2cm]{img/youtube}}
\end{minipage}
%% --------------------------------------------------------------------



\paragraph{Der diagnostische Test und statistische Maßzahlen}

'Was ist denn das?', entfährt es Steffen. 'Hm... ich glaube es handelt sich um einen Doppelbaum, den wir beim diagnostischen Testen brauchen.', meint Tina und dreht Harry Potter auf dem Second Screen etwas leiser. Was jetzt beide von einem diagnostischen Test haben, ist ihnen auch nicht klar. Es ist ja schon alles komplex genug und die Romantik von Steffen macht es heute auch nicht mehr einfacher. 'Es geht um Escherichia coli (E. coli) an Zandern.', stellt Tina fest. Eigentlich wollte Tina eher los um zu Boxen. Das wird aber wohl nichts mehr.

\begin{tikzpicture}
  \node (image) at (0,0) {
    \includegraphics[width=\textwidth]{/Users/kruppajo/work/GitHub/exam/question/img/diag-doppelbaum}
  };
  \node[] at (-4.8,0) {\huge 160};
  \node[] at (-1.7,0) {\huge 20};
  \node[] at (1.7,0) {\huge 560};
  \node[] at (4.75,0) {\huge 1800};
\end{tikzpicture}

Leider kennen sich Steffen und Tina mit dem diagnostischen Testen überhaupt nicht aus. Deshalb brauchen beide bei der Erstellung Ihre Hilfe! 
  
\begin{enumerate}
\item Beschriften Sie den Doppelbaum! \textbf{(2 Punkte)}
\item Füllen Sie freien Felder des Doppelbaums aus! \textbf{(4 Punkte)}
\item Berechnen Sie die Wahrscheinlichkeit $Pr(K^+|T^+)$! \textbf{(2 Punkte)}
\item Berechnen Sie die Prävalenz für Klauenseuche! \textbf{(1 Punkt)}
\item Berechnen Sie die Sensifität und Spezifität des diagnostischen Tests! Erstellen Sie dafür zunächst eine 2x2 Kreuztabelle! \textbf{(2 Punkte)}
\end{enumerate}

 





 
\clearpage
% -----------------------------------------------------------------------
\part{Lineare Regression \& Korrelation}
% -----------------------------------------------------------------------

\section{Aufgabe \hfill (10 Punkte)}

\textit{Geben Sie grundsätzlich Formeln und Rechenweg zur Lösung der Teilaufgaben mit an!} \\[1Ex]
 

 
%% --------------------------------------------------------------------
\begin{minipage}[t]{0.5\textwidth}
\includegraphics[width = 1.3cm]{/Users/kruppajo/work/GitHub/exam/avatare/Mark.png}\hspace{-4mm}\includegraphics[width = 1.3cm]{/Users/kruppajo/work/GitHub/exam/avatare/Paula.png}
\end{minipage}
\begin{minipage}[t]{0.5\textwidth}
\hfill
\href{https://youtu.be/kHmfEmU6lrk}{\includegraphics[width = 2cm]{img/youtube}}
\end{minipage}
%% --------------------------------------------------------------------



\paragraph{Visualisierung der linearen Regression}

'Wichtig ist es, dass wir jetzt eine Gerade durch die Punkte zeichnen!', ruft Paula. 'Ich sehe nur eine Zahlen und keine Punkte. Wie soll ich da denn jetzt eine Gerade durchzeichnen?', fragt Mark. Paula atmet schwer ein. Die beiden hatten ein Kreuzungsexperiment im Teuteburgerwald mit Fleischrindern durchgeführt. Dabei wurden die beiden folgenden Variablen gemessen: mittlerer Anzahl an weißen Blutkörperchen [LEU/ml] und Fettgehalt [\%/kg]. Jetzt will die Betreuung von den beiden einmal die Visualisierung der Daten und auch gleich noch die lineare Regression gerechnet bekommen.

\begin{table}[!h]
\centering
\begin{tabular}{cc}
\toprule
Fettgehalt [\%/kg] & Mittlerer Anzahl an weißen Blutkörperchen [LEU/ml]\\
\midrule
14.2 & 23.1\\
18.7 & 28.1\\
11.9 & 21.2\\
16.5 & 27.9\\
11.1 & 21.0\\
\addlinespace
13.8 & 23.5\\
17.1 & 27.3\\
13.2 & 21.1\\
16.4 & 27.0\\
15.4 & 24.2\\
\addlinespace
23.1 & 30.0\\
\bottomrule
\end{tabular}
\end{table}



Leider kennen sich Paula und Mark mit der linearen Regression für kontinuierliche Daten überhaupt nicht aus. Deshalb brauchen beide bei der Erstellung Ihre Hilfe!

\begin{enumerate}
\item Formulieren Sie die wissenschaftliche Fragestellung! \textbf{(1 Punkt)}
\item Erstellen  Sie  eine  Visualisierung  für  die  Datentabelle.  Beschriften  Sie  die  Achsen! \textbf{(2 Punkte)}
\item Schätzen Sie die Regressionsgleichung aus der obigen Abbildung ab! \textbf{(2 Punkte)}
\item Beschriften Sie die Grade mit den statistischen Maßzahlen der linearen Regressionsgleichung! \textbf{(2 Punkte)}
\item Liegt ein Zusammenhang zwischen $x$ und $y$ vor? Begründen Sie Ihre Antwort! \textbf{(2 Punkte)}
\item Wenn kein Zusammenhang zu beobachten wäre, wie würde die Grade aussehen? \textit{Antworten Sie mit einer Skizze der Geraden!} \textbf{(1 Punkt)}
\end{enumerate} 
\clearpage
% -----------------------------------------------------------------------

\section{Aufgabe \hfill (12 Punkte)}

\textit{Geben Sie grundsätzlich Formeln und Rechenweg zur Lösung der Teilaufgaben mit an!} \\[1Ex]
 

 
%% --------------------------------------------------------------------
\begin{minipage}[t]{0.5\textwidth}
\includegraphics[width = 1.3cm]{/Users/kruppajo/work/GitHub/exam/avatare/Jessica.png}\hspace{-4mm}\includegraphics[width = 1.3cm]{/Users/kruppajo/work/GitHub/exam/avatare/Nilufar.png}
\end{minipage}
\begin{minipage}[t]{0.5\textwidth}
\hfill
\href{https://youtu.be/lJp8rFmMnrs}{\includegraphics[width = 2cm]{img/youtube}}
\end{minipage}
%% --------------------------------------------------------------------



\paragraph{Interpretation der Ergebnisse einer linearen Regression}


'Hä? Was ist denn das? Hatten wir das als Aufgabe eine lineare Regression zu rechnen? Wir bauen aus kontinuierlichen Daten eine Abbildung und interpretieren diese dann?', fragt Nilufar. Jessica schaut fragend zurück. 'Keine Ahnung... das ist jetzt jedenfalls keine Abbildung von irgendwas sondern Zahlen in einer Tabelle...', antwortet Jessica leicht angespannt. Die beiden hatten ein Kreuzungsexperiment im Emsland mit Zandern durchgeführt. Dabei wurden die beiden folgenden Variablen gemessen: durchschnittlicher Bewegungsscore [Movement/h] und Gewichtszuwachs in der 1LW. Jetzt haben die beiden eigentlich alles zusammen. \textit{Eigentlich...}, denn mit der \Rlogo Ausgabe haben beide jetzt ein Problem.

\begin{table}[!h]
\centering\begingroup\fontsize{12}{14}\selectfont

\begin{tabular}{ccccc}
\toprule
term & estimate & std.error & t statistic & p-value\\
\midrule
(Intercept) & 0.05 & 2.80 &  & \\
Durchschnittlicher Bewegungsscore & 0.66 & 0.27 &  & \\
\bottomrule
\end{tabular}
\endgroup{}
\end{table}



Leider kennen sich Nilufar und Jessica mit der linearen Regression für kontinuierliche Daten in \Rlogo überhaupt nicht aus. Deshalb brauchen beide bei der Erstellung Ihre Hilfe!

\begin{enumerate}
\item Formulieren Sie die wissenschaftliche Fragestellung! \textbf{(1 Punkt)}
\item Erstellen  Sie  eine  Visualisierung  der \texttt{lm()}-Ausgabe.  Beschriften  Sie  die  Achsen! \textbf{(2 Punkte)}
\item Beschriften Sie die Visualisierung mit den statistischen Maßzahlen der der \texttt{lm()}-Ausgabe! \textbf{(2 Punkte)}
\item Formulieren Sie die Regressionsgleichung! \textbf{(1 Punkt)}
\item Ergänzen Sie die t Statistik in der \texttt{lm()}-Ausgabe! \textbf{(2 Punkte)}
\item Ergänzen Sie den $p$-Wert in der \texttt{lm()}-Ausgabe mit $T_{\alpha = 5\%} = 1.96$!  \textbf{(2 Punkte)}
\item Interpretieren Sie den $p$-Wert im Kontext der wissenschaftlichen Fragestellung! \textbf{(1 Punkt)}  
\item Wie groß ist der Effekt im Kontext der wissenschaftlichen Fragestellung? \textbf{(1 Punkt)}
\end{enumerate} 
\clearpage
% -----------------------------------------------------------------------

\section{Aufgabe \hfill (11 Punkte)}

\textit{Geben Sie grundsätzlich Formeln und Rechenweg zur Lösung der Teilaufgaben mit an!} \\[1Ex]
 

 
%% --------------------------------------------------------------------
\begin{minipage}[t]{0.5\textwidth}
\includegraphics[width = 1.3cm]{/Users/kruppajo/work/GitHub/exam/avatare/Nilufar.png}\hspace{-4mm}\includegraphics[width = 1.3cm]{/Users/kruppajo/work/GitHub/exam/avatare/Paula.png}
\end{minipage}
\begin{minipage}[t]{0.5\textwidth}
\hfill
\href{https://youtu.be/tNNzcndrpSk}{\includegraphics[width = 2cm]{img/youtube}}
\end{minipage}
%% --------------------------------------------------------------------



\paragraph{Interpretation der Ergebnisse einer linearen Regression in \Rlogo}


'Ich glaube du bringst da was durcheinander. Wir nutzen zwar auch für die ANOVA die Funktion \texttt{lm()} aber hier wollen wir, glaube ich, eine Gerade durch die Punkte zeichnen.', merkt Nilufar an. 'Ich sehe keine Punkte... das ist doch eine Ausgabe in \Rlogo. Überhaupt, darum geht es doch gar nicht in unserem Versuch. Wir wollen doch keine Gerade zeichnen?.', antwortet Paula sichtlich übernächtigt. 'Doch wir müssen nur die Koeffizienten der linearen Regression erst richtig interpretieren und vor unserem geistigen Auge erscheint eine Gerade!', spricht Nilufar sehr deutlich und langsam. Die beiden hatten einen Versuch in einer Klimakammer im Teuteburgerwald mit Brokkoli durchgeführt. Dabei wurden die beiden folgenden Variablen gemessen: durchschnittlicher Anteil an Ton [\%/l] und Trockengewicht [kg/ha]. Jetzt wollen sie erstmal schauen, ob es einen Zusammenhang gibt und das soll mit der \Rlogo Ausgabe möglich sein.

\begin{knitrout}
\definecolor{shadecolor}{rgb}{0.969, 0.969, 0.969}\color{fgcolor}\begin{kframe}
\begin{verbatim}
## 
## Call:
## Trockengewicht ~ Durchschnittlicher_Anteil
## 
## Residuals:
##      Min       1Q   Median       3Q      Max 
## -2.52011 -1.00497 -0.00971  1.06279  2.48227 
## 
## Coefficients:
##                           Estimate Std. Error t value Pr(>|t|)
## (Intercept)                 1.8290     1.2720   1.438    0.158
## Durchschnittlicher_Anteil   1.5338     0.1258  12.195 3.19e-15
## 
## Residual standard error: 1.271 on 41 degrees of freedom
## Multiple R-squared:  0.7839,	Adjusted R-squared:  0.7786 
## F-statistic: 148.7 on 1 and 41 DF,  p-value: 3.188e-15
\end{verbatim}
\end{kframe}
\end{knitrout}

Leider kennen sich Nilufar und Paula mit der linearen Regression für kontinuierliche Daten in \Rlogo überhaupt nicht aus. Deshalb brauchen beide bei der Erstellung Ihre Hilfe!


\begin{enumerate}
\item Formulieren Sie die wissenschaftliche Fragestellung! \textbf{(1 Punkt)}
\item Wie groß ist der Effekt im Kontext der wissenschaftlichen Fragestellung? \textbf{(2 Punkte)} 
\item Interpretieren Sie die $p$-Werte im Kontext der wissenschaftlichen Fragestellung! \textbf{(2 Punkte)}
\item Visualisieren Sie die Verteilung der Residuen! \textbf{(2 Punkte)} 
\item Ist die Annahme der Normalverteilung erfüllt? Begründen Sie die Antwort! \textbf{(2 Punkte)}
\item Erklären Sie \textit{kurz} den Begriff \texttt{R-squared}! Was sagt Ihnen der Wert aus? \textbf{(2 Punkte)}
\end{enumerate}
 
\clearpage
% -----------------------------------------------------------------------

\section{Aufgabe \hfill (10 Punkte)}

\textit{Geben Sie grundsätzlich Formeln und Rechenweg zur Lösung der Teilaufgaben mit an!} \\[1Ex]
 

 
%% --------------------------------------------------------------------
\begin{minipage}[t]{0.5\textwidth}
\includegraphics[width = 1.3cm]{/Users/kruppajo/work/GitHub/exam/avatare/Jessica.png}
\end{minipage}
\begin{minipage}[t]{0.5\textwidth}
\hfill
\href{https://youtu.be/C9skfFRTHhI}{\includegraphics[width = 2cm]{img/youtube}}
\end{minipage}
\vspace{-3ex}
%% --------------------------------------------------------------------



\paragraph{Interpretation der Ergebnisse einer Korrelationsanalyse in \Rlogo}

'Wichtig ist es, dass wir jetzt eine Gerade durch die Punkte zeichnen...', denkt Jessica. 'Ich sehe nur Kauderwelsch und keine Punkte. Ich glaube das war jetzt doch eine Korrelation, die ich rechnen sollte. Und warum überhaupt? War das unsere Fragestellung?', denkt sich Jessica. Jessica atmet schwer ein und starrt auf die \Rlogo Ausgabe der Funktion \texttt{cor.test()}. Das hilft alles nur begrenzt. Das Verrückte ist, dass die Hündin Herr der Ringe wirklich liebt. Das ist Jessica sehr recht, denn sie braucht Entspannung. Jessica hatte ein Stallexperiment im Wendland mit Milchvieh durchgeführt. Dabei wurden die beiden folgenden Variablen gemessen: durchschnittlicher Bewegungsscore [Movement/h] und Proteianteil [\%/kg]. Jetzt will die Betreuung von ihr die Interpretierung der Daten in Form einer Korrelation berechnet bekommen. Das hat Jessica in \Rlogo gemacht, aber wie soll das jetzt gehen? Das mit der Interpretation?  Eine echte Herausforderung für sie war schon immer der Mangel gewesen. Ein leidiges Lied. 


\begin{knitrout}
\definecolor{shadecolor}{rgb}{0.969, 0.969, 0.969}\color{fgcolor}\begin{kframe}
\begin{verbatim}
## 
## 	Pearson's correlation
## 
## data:  Durchschnittlicher Bewegungsscore and Proteianteil
## t = -5.1097, df = 8, p-value = 0.0009188
## alternative hypothesis: true correlation is not equal to 0
## 95 percent confidence interval:
##  -0.9701246 -0.5461058
## sample estimates:
##        cor 
## -0.8749031
\end{verbatim}
\end{kframe}
\end{knitrout}

Leider kennt sich Jessica mit der Korrelationsanalyse in \Rlogo überhaupt nicht aus. Deshalb braucht sie bei der Erstellung Ihre Hilfe!

\begin{enumerate}
  \item Formulieren Sie die wissenschaftliche Fragestellung! \textbf{(1 Punkt)}
  \item Formulieren Sie das statistische Hypothesenpaar! \textbf{(1 Punkt)}
\item Erstellen Sie eine Visualisierung für den Korrelationskoeffizienten! Beschriften Sie die Abbildung! \textbf{(2 Punkte)}
\item Nennen Sie die zwei Eigenschaften des Korrelationskoeffizienten! \textbf{(2 Punkte)}
\item Interpretieren Sie den Korrelationskoefizienten hinsichtlich des
  Effekts und der Signifikanz! Begründen Sie Ihre Antwort! \textbf{(2 Punkte)}
\item Visualisieren Sie das 95\% Konfidenzintervall! Beschriften Sie die Abbildung! \textbf{(2 Punkte)} 
\end{enumerate} 
\clearpage
% -----------------------------------------------------------------------

\section{Aufgabe \hfill (12 Punkte)}

\textit{Geben Sie grundsätzlich Formeln und Rechenweg zur Lösung der Teilaufgaben mit an!} \\[1Ex]
 

 
%% --------------------------------------------------------------------
\begin{minipage}[t]{0.5\textwidth}
\includegraphics[width = 1.3cm]{/Users/kruppajo/work/GitHub/exam/avatare/Steffen.png}
\end{minipage}
\begin{minipage}[t]{0.5\textwidth}
\hfill
\href{https://youtu.be/fB6nF4dxodA}{\includegraphics[width = 2cm]{img/youtube}}
\end{minipage}
\vspace{-3ex}
%% --------------------------------------------------------------------



Das Verrückte ist, dass die Schlange Harry Potter wirklich liebt. Das ist Steffen sehr recht, denn er braucht Entspannung. Da hilft dann die Aufgabe auch nur bedingt. 'Hm..., drei leere Abbildungen. Was soll ich da jetzt machen?', fragt sich Steffen und mampft noch ein paar Oreos in sich hinein. Steffen kennt sich nur begrenzt bis gar nicht mit der linearen Regresion und Korrelation aus.
\vspace{2Ex}



{\centering \includegraphics[width=\maxwidth]{img/correlation-01-1} 

}




\paragraph{Visualisierung der Korrelation und des Bestimmtheitsmaßes}

\vspace{2Ex}

Leider kennt sich Steffen mit der Korrelationsanalyse und der linearen Regression überhaupt nicht aus. Deshalb braucht er bei der Auswertung Ihre Hilfe!

\begin{enumerate}
\item Zeichnen Sie für die $\rho$-Werte eine Gerade in die entsprechende Abbildung! \textbf{(3 Punkte)}
\item Zeichnen Sie für die $R^2$-Werte die entsprechende Punktewolke um die Gerade! \textbf{(3 Punkte)}
\item Nennen Sie die zwei Eigenschaften des Korrelationskoeffizienten! \textbf{(2 Punkte)}
\item Interpretieren Sie die $R^2$-Werte für die jeweilige Gerade! \textbf{(2 Punkte)}
\item Warum müssen Sie ein $R^2$-Wert berechnen, wenn Sie die einfachere Möglichkeit der visuellen Überprüfung haben? Begründen Sie Ihre Antwort! \textbf{(2 Punkte)}
\end{enumerate}
 
\clearpage
% -----------------------------------------------------------------------

\section{Aufgabe \hfill (12 Punkte)}

\textit{Geben Sie grundsätzlich Formeln und Rechenweg zur Lösung der Teilaufgaben mit an!} \\[1Ex]
 

 
%% --------------------------------------------------------------------
\begin{minipage}[t]{0.5\textwidth}
\includegraphics[width = 1.3cm]{/Users/kruppajo/work/GitHub/exam/avatare/Jessica.png}
\end{minipage}
\begin{minipage}[t]{0.5\textwidth}
\hfill
\href{https://youtu.be/2QJa19ZwLls}{\includegraphics[width = 2cm]{img/youtube}}
\end{minipage}
\vspace{-3ex}
%% --------------------------------------------------------------------



\paragraph{Schätzen der Korrelation und des Bestimmtheitsmaßes}

Der Bildschirm strahlt blau in das Gesicht von Jessica. Es ist schon spät. Und das hat einen Grund. Hm, lecker Schokobons und dazu dann im Hintergrund Herr der Ringe laufen lassen. . Jessica überlegt, aber ihre Gedaken sind etwas zäh. 'Was soll das hier alles bedeuten?', fragt sich Jessica. Irgendwie ist ihr nicht klar wie sie $\rho$-Werte oder $R^2$-Werte abschätzen soll. Alles nicht so einfach. Wenn der Mangel nicht wäre, ja dann wäre wohl vieles möglich für Jessica! Aber so.. 
\vspace{2Ex}



{\centering \includegraphics[width=\maxwidth]{img/correlation-02-1} 

}




Leider kennt sich Jessica mit der Korrelationsanalyse und der linearen Regression überhaupt nicht aus. Deshalb braucht sie bei der Auswertung Ihre Hilfe!

\begin{enumerate}
\item Schätzen Sie die $\rho$-Werte in den Abbildungen! \textbf{(2 Punkte)}
\item Schätzen Sie die $R^2$-Werte in den Abbildungen! \textbf{(2 Punkte)}
\item Interpretieren Sie die $R^2$-Werte für die jeweilige Gerade! \textbf{(2 Punkte)}
\item Was ist der optimale $R^2$-Wert im Kontext einer wissenschaftlichen Fragestellung? Begründen Sie Ihre Antwort an einem Beispiel! \textbf{(2 Punkte)}
\item Was ist der optimale $\rho$-Wert im Kontext einer wissenschaftlichen Fragestellung? Begründen Sie Ihre Antwort an einem Beispiel! \textbf{(2 Punkte)}
\item Erklären Sie die Aussage \textit{"Correlation does not imply causation!"} an einem Beispiel! \textbf{(2 Punkte)}
\end{enumerate} 
\clearpage
% -----------------------------------------------------------------------

\section{Aufgabe \hfill (11 Punkte)}

\textit{Geben Sie grundsätzlich Formeln und Rechenweg zur Lösung der Teilaufgaben mit an!} \\[1Ex]
 

 
%% --------------------------------------------------------------------
\begin{minipage}[t]{0.5\textwidth}
\includegraphics[width = 1.3cm]{/Users/kruppajo/work/GitHub/exam/avatare/Paula.png}
\end{minipage}
\begin{minipage}[t]{0.5\textwidth}
\hfill
\href{https://youtu.be/dyQlYV9nOqY}{\includegraphics[width = 2cm]{img/youtube}}
\end{minipage}
\vspace{-3ex}
%% --------------------------------------------------------------------



\paragraph{Modellgüte der linearen Regression}

'Oh! Residuen. Die waren wichtig um zu wissen, ob eine Modellierung funktioniert hat! Da schauen wir uns dann mit der Funktion \texttt{augment()} die Werte der einzelnen Residuen an. Oder gleich den Residuenplot...da sehen wir dann... ja was eigentlich?', verkündet Paula stolz. Leider hat Paula vergessen wie der \Rlogo Code für den Residuenplot geht. Paula hatte anderes im Kopf. Hm, lecker Smarties und dazu dann im Hintergrund Jagd auf roter Oktober laufen lassen. Aber sowas hilft ihr natürlich hier nicht. Da schmeißt sich Paula noch ein paar Smarties in den Mund und kaut los.

\begin{knitrout}
\definecolor{shadecolor}{rgb}{0.969, 0.969, 0.969}\color{fgcolor}\begin{table}[!h]
\centering\begingroup\fontsize{12}{14}\selectfont

\begin{tabular}{cccc}
\toprule
Proteinanteil & Durchschnittlicher Bewegungsscore & $\hat{y}$ & $\phantom{ttttt}\epsilon\phantom{ttttt}$\\
\midrule
22.7 & 9.5 & 23.8 & \\
19.1 & 7.0 & 19.6 & \\
21.5 & 7.7 & 20.8 & \\
31.4 & 12.8 & 29.5 & \\
37.6 & 16.6 & 36.0 & \\
\addlinespace
30.7 & 14.6 & 32.5 & \\
33.5 & 16.0 & 34.9 & \\
23.5 & 9.1 & 23.1 & \\
\bottomrule
\end{tabular}
\endgroup{}
\end{table}

\end{knitrout}

Leider kennt sich Paula mit der linearen Regression überhaupt nicht aus. Deshalb braucht sie bei der Auswertung Ihre Hilfe!

\begin{enumerate}
  \item Formulieren Sie die wissenschaftliche Fragestellung! \textbf{(1 Punkt)}
\item Ergänzen Sie die Werte der Residuen $\epsilon$ in der obigen Tabelle! \textbf{(2 Punkte)}
\item Zeichnen Sie den Boxplot der Residuen $\epsilon$. Beschriften Sie die Abbildung! \textbf{(2 Punkte)}
\item Zeichnen Sie den Residualplot. Beschriften Sie die Abbildung! \textbf{(2 Punkte)}
\item Gibt es auffällige Werte anhand des Residualplots? Begründen Sie Ihre Antwort! \textbf{(2 Punkte)}
\item Erklären Sie die Eigenschaft eines statistischen Modells, welche mit dem Residualplot überprüft wird! Begründen Sie Ihre Antwort anhand einer Visualisierung! \textbf{(2 Punkte)}
\end{enumerate}
 
\clearpage
% -----------------------------------------------------------------------

\section{Aufgabe \hfill (12 Punkte)}

\textit{Geben Sie grundsätzlich Formeln und Rechenweg zur Lösung der Teilaufgaben mit an!} \\[1Ex]
 

 
%% --------------------------------------------------------------------
\begin{minipage}[t]{0.5\textwidth}
\includegraphics[width = 1.3cm]{/Users/kruppajo/work/GitHub/exam/avatare/Alex.png}\hspace{-4mm}\includegraphics[width = 1.3cm]{/Users/kruppajo/work/GitHub/exam/avatare/Jessica.png}
\end{minipage}
\begin{minipage}[t]{0.5\textwidth}
\hfill
\href{https://youtu.be/kHmfEmU6lrk}{\includegraphics[width = 2cm]{img/youtube}}
\end{minipage}
%% --------------------------------------------------------------------



\paragraph{Visualisierung des Regressionskreuzes}

Alex hat ein Kreuzungsexperiment mit Zandern duchgeführt. Soweit so gut. Dann war er bei seiner Betreuerin. Leider war der Schritt nicht so hilfreich.  Alex und die Gefälligkeit, eine unendliche Geschichte mit kniffeligen Wendungen. Aber es muss ja weitergehen. Alex hatte dann in seiner Abschlusarbeit einfach zu viele Endpunkte gemessen und ist jetzt vollkommen durcheinander, welche Analyse er nun wie rechnen soll. Naja, dann heißt es jetzt eben Abba aufdrehen und darüber nachdenken, was hier eigentlich gemacht wurde. Alex fängt einfach an und nimmt den ersten Endpunkt Anzahl dunkle Pigmentstörungen. Dann kann er sich voran arbeiten. Später dann noch raus um zu Laufen um mal zu entspannen und vielleicht ist Jessica auch da. Wäre toll.

\vspace{1Ex}

Leider kennt sich Alex mit dem Kontext der linearen Regression überhaupt nicht aus. Deshalb braucht er bei der Auswertung Ihre Hilfe!

\begin{enumerate}
  \item Formulieren Sie die wissenschaftliche Fragestellung! \textbf{(1 Punkt)}
\item Zeichen Sie die Zeile des Regressionskreuzes für den Endpunkt mit \underline{drei} Feldern! Beschriften Sie die Abbildung! \textbf{(4 Punkte)}
\item Ergänzen Sie die entsprechenden statistische Methoden zur Analyse in jedem Feld! \textbf{(2 Punkte)}
\item Formulieren Sie die Nullhypothese für die statistische Methode in jedem Feld! \textbf{(2 Punkte)}
\item Ergänzen Sie die entsprechenden Funktionen in \Rlogo zur Analyse in jedem Feld! \textbf{(2 Punkte)}
\item Welchen Effekt erhalten Sie in jedem Feld? Geben Sie ein Beispiel! \textbf{(2 Punkte)}
\end{enumerate} 
\clearpage
% -----------------------------------------------------------------------
\part{Experimentelles Design}
% -----------------------------------------------------------------------

\section{Aufgabe \hfill (16 Punkte)}


 
%% --------------------------------------------------------------------
\begin{minipage}[t]{0.5\textwidth}
\includegraphics[width = 1.3cm]{/Users/kruppajo/work/GitHub/exam/avatare/Alex.png}\hspace{-4mm}\includegraphics[width = 1.3cm]{/Users/kruppajo/work/GitHub/exam/avatare/Tina.png}\hspace{-4mm}\includegraphics[width = 1.3cm]{/Users/kruppajo/work/GitHub/exam/avatare/Yuki.png}
\end{minipage}
\begin{minipage}[t]{0.5\textwidth}
\hfill
\href{https://youtu.be/wJqsNV1hOW8}{\includegraphics[width = 2cm]{img/caution}}
\end{minipage}
%% --------------------------------------------------------------------



\paragraph{Einfache experimentelle Designs}

Neuer Versuch neues Glück! Alex und Tina sind bei Yuki um sich Hilfe für eine Versuchsplanung in \Rlogo zu holen. Im Hintergrund läuft viel zu laut Matrix. Daher hat das Minischwein schon lange reißaus genommen. In dem neuen Versuch geht es um den Zusammenhang zwischen der Behandlung Flüssignahrung ($ctrl$, $superIn$ und $flOw$) und dem Messwert Protein/Fettrate [\%/kg] in Fleischrindern. Der Versuch soll in einem Leistungssteigerungsversuch in der Uckermark durchgeführt werden. Immerhin ist der Messswert normalverteilt, was einges einfacher macht. Was es nicht so einfacher macht ist, dass Alex noch als zusätzliche Herausforderung etwas anderes umtreibt: die Gefälligkeit. Im ersten Schritt überlegt Yuki ein einfaches experimentelles Design zu probieren. Daher entscheiden sich alle drei für ein \textit{Randomized complete block design (RCBD) mit zwei Blöcken}. Ob es das jetzt einfacher macht?\\

Leider kennen sich Yuki, Alex und Tina mit dem \textit{Randomized complete block design (RCBD) mit zwei Blöcken} überhaupt nicht aus. Deshalb brauchen die Drei bei der Erstellung Ihre Hilfe!

\begin{enumerate}
  \setcounter{enumi}{0}
  \item Formulieren Sie die wissenschaftliche Fragestellung! \textbf{(1 Punkt)}
  \item Formulieren Sie das statistische Hypothesenpaar! \textbf{(1 Punkt)}
  \item Skizzieren Sie das faktorielle Versuchsdesign! \textbf{(3 Punkte)}
  \item Skizzieren Sie eine Datentabelle für das faktorielle Versuchsdesign in \Rlogo! \textbf{(2 Punkte)}
  \item Erstellen Sie das statistische Modell in der in \Rlogo üblichen Schreibweise für eine ANOVA! Skizzieren Sie die notwendige Funktionen in \Rlogo! \textbf{(3 Punkte)}
  \item Skizzieren Sie die weitere Datenanalyse hinsichtlich eines multiplen Gruppenvergleiches! \textbf{(2 Punkte)}
  \item Skizzieren Sie eine mögliche Abbildung im Kontext der wissenschaftlichen Fragestellung! Beschriften Sie die Abbildung! \textbf{(2 Punkte)}
  \item Ergänzen Sie zu der Abbildung ein mögliches Ergebnis des multiplen Gruppenvergleichs! Begründen Sie Ihre Antwort! \textbf{(2 Punkte)}
\end{enumerate}


 
\clearpage
% -----------------------------------------------------------------------

\section{Aufgabe \hfill (20 Punkte)}


 
%% --------------------------------------------------------------------
\begin{minipage}[t]{0.5\textwidth}
\includegraphics[width = 1.3cm]{/Users/kruppajo/work/GitHub/exam/avatare/Paula.png}\hspace{-4mm}\includegraphics[width = 1.3cm]{/Users/kruppajo/work/GitHub/exam/avatare/Steffen.png}\hspace{-4mm}\includegraphics[width = 1.3cm]{/Users/kruppajo/work/GitHub/exam/avatare/Tina.png}
\end{minipage}
\begin{minipage}[t]{0.5\textwidth}
\hfill
\href{https://youtu.be/wJqsNV1hOW8}{\includegraphics[width = 2cm]{img/caution}}
\end{minipage}
%% --------------------------------------------------------------------



\paragraph{Fortgeschrittene experimentelle Designs}

Das Minischwein macht mal wieder Randale in Steffens Zimmer und rennt davon! Tina und Paula sind bei Steffen in im Emsland wo der neue, bessere Versuch stattfinden soll. Dabei soll in einem Versuch in einer Klimakammer im Emsland mit Maiss durchgeführt werden. Auf dem Tisch stapeln sich Reese's Peanut Butter Cups aus Vollkorndinkelmehl. Eine Spezialtät der Komune hier. Tina hasst Vollkorn wie Katjes geliebt werden. In dem neuen Versuch geht es um den Zusammenhang zwischen der Behandlung Lichtstufen ($none$, $200lm$, $400lm$ und $600lm$) sowie Lüftungssystemen und Folientunneln ($ctrl$, und $tornado$) sowie vier Blöcken und dem Messwert Proteingehalt [g/kg]. Immerhin ist der Messswert normalverteilt, was einges einfacher macht. Was es nicht so einfacher macht ist, dass Paula als zusätzliche Herausforderung noch die Wut mitgebracht hat. Daher entscheiden sich alle drei für ein \textit{Randomized complete block design}. 'Naja, so viel einfacher ist es dann doch nicht...', merkt Paula an und sucht das Minischwein.\\

Leider kennen sich Steffen, Tina und Paula mit dem \textit{Randomized complete block design} überhaupt nicht aus. Deshalb brauchen die Drei bei der Erstellung Ihre Hilfe!

\begin{enumerate}
  \setcounter{enumi}{0}
  \item Formulieren Sie die wissenschaftliche Fragestellung! \textbf{(1 Punkt)}
  \item Formulieren Sie die statistische Hypothesenpaare! \textbf{(2 Punkte)}
  \item Skizzieren Sie das faktorielle Versuchsdesign! \textbf{(3 Punkte)}
  \item Skizzieren Sie eine Datentabelle für das faktorielle Versuchsdesign in \Rlogo! \textbf{(2 Punkte)}
  \item Erstellen Sie das statistische Modell in der in \Rlogo üblichen Schreibweise für eine ANOVA! Skizzieren Sie die notwendige Funktionen in \Rlogo! \textbf{(4 Punkte)}
  \item Skizzieren Sie die weitere Datenanalyse hinsichtlich eines multiplen Gruppenvergleiches! \textbf{(2 Punkte)}
  \item Skizzieren Sie eine mögliche Abbildung im Kontext der wissenschaftlichen Fragestellung! Beschriften Sie die Abbildung! \textbf{(3 Punkte)}
  \item Ergänzen Sie zu der Abbildung ein mögliches Ergebnis des multiplen Gruppenvergleichs! Welche Annahme hinsichtlich der Modellierung haben Sie getroffen? Begründen Sie Ihre Antwort! \textbf{(3 Punkte)}
\end{enumerate} 
\clearpage
% -----------------------------------------------------------------------
\part{Programmieren in R}
% -----------------------------------------------------------------------

\section{Aufgabe \hfill (9 Punkte)}



 
%% --------------------------------------------------------------------
\begin{minipage}[t]{0.5\textwidth}
\includegraphics[width = 1.3cm]{/Users/kruppajo/work/GitHub/exam/avatare/Jonas.png}
\end{minipage}
\begin{minipage}[t]{0.5\textwidth}
\hfill
\href{https://www.youtube.com/playlist?list=PLe51bCp9JvEFUnFqaJG5aRmON9i1ZbOYC}{\includegraphics[width = 2cm]{img/youtube}}
\end{minipage}
\vspace{-3ex}
%% --------------------------------------------------------------------



\paragraph{Grundlegende Kenntnisse der Programierung in \Rlogo}

'Hm. \Rlogo ist eigentlich gar nicht so schwer, wenn man die Grundlagen kann.', meint  Jonas ganz zuversichtlich. Nur leider kennt er sich überhaupt nicht mit \Rlogo aus! Das heißt, Sie müssen hier einmal Rede und Antwort stehen und helfen.\\[1Ex]

Jonas: \textit{Wie sieht der Pipe-Operator aus und was ist seine Funktion? Gerne mit Beispiel!} \textbf{(1 Punkt)}\\[1ex]
Sie antworten:\\[2Ex]

Jonas: \textit{Ich habe doch die Spalte mutiert und geändert. Warum sehe ich das in R aber mein Datensatz ändert sich nicht?} \textbf{(1 Punkt)}\\[1ex]
Sie antworten:\\[2Ex]

Jonas: \textit{Teilweise brauche ich das Konzept des Faktors in \Rlogo. Was ist ein Faktor?} \textbf{(1 Punkt)}\\[1ex]
Sie antworten:\\[2Ex]

Jonas: \textit{Was macht überhaupt dieses \texttt{c()} hier überall im \Rlogo Code?} \textbf{(1 Punkt)}\\[1ex]
Sie antworten:\\[2Ex]

Jonas: \textit{Wie speichern wir in R intern Daten ab? Ich brauche nochmal den Namen der Funktion. Was sind den die Vorteile von dieser Art der Speicherung?} \textbf{(1 Punkt)}\\[1ex]
Sie antworten:\\[2Ex]

Jonas: \textit{Wir brauchen recht häufig die Tilde ($\sim$) in R. Wo wird die nochmal angewandt und genutzt?} \textbf{(1 Punkt)}\\[1ex]
Sie antworten:\\[2Ex]

Jonas: \textit{Warum gibt es eigentlich das RStudio und R? Wie unterscheiden sich beide voneinander?} \textbf{(1 Punkt)}\\[1ex]
Sie antworten:\\[2Ex]

Jonas: \textit{Ich verstehe den Unterschied zwischen \texttt{library()} und \texttt{Packages} nicht. Warum gibt es die?} \textbf{(1 Punkt)}\\[1ex]
Sie antworten:\\[2Ex]

Jonas: \textit{Wenn ich was in \Rlogo machen möchte, dann lade ich nochmal welche zwei \Rlogo Pakete sehr häufig?} \textbf{(1 Punkt)}\\[1ex]
Sie antworten:\\[2Ex] 
\clearpage
% -----------------------------------------------------------------------

\section{Aufgabe \hfill (9 Punkte)}



 
%% --------------------------------------------------------------------
\begin{minipage}[t]{0.5\textwidth}
\includegraphics[width = 1.3cm]{/Users/kruppajo/work/GitHub/exam/avatare/Nilufar.png}
\end{minipage}
\begin{minipage}[t]{0.5\textwidth}
\hfill
\href{https://www.youtube.com/playlist?list=PLe51bCp9JvEFUnFqaJG5aRmON9i1ZbOYC}{\includegraphics[width = 2cm]{img/youtube}}
\end{minipage}
\vspace{1ex}
%% --------------------------------------------------------------------



\paragraph{Fortgeschrittene Kenntnisse der Programierung in \Rlogo}

Nilufar muss ihrer Hausarbeit mit \Rlogo arbeiten. Leider ist die Analyse etwas komplexer, so dass es eben in Excel dann nicht mehr geht. Deshalb also gleich alles in \Rlogo. Das ist auch der Grund warum sie jetzt mit Ihnen in der Küche sitzt und einige vertiefende Fragen zu \Rlogo an Sie hat! Na dann wollen Sie mal helfen. Immerhin will ihr Betreuer, dass \Rlogo genutzt wird und die Abgabe ist dann auch schon in gut einem Monat.\\[1Ex]

Nilufar fragt: \textit{Wozu war nochmal die Funktion \texttt{mutate()} gut?  \textbf{(1 Punkt)}}\\[1ex]
Sie antworten:\\[2Ex]

Nilufar fragt: \textit{Ich baue mir ja meinen Datensatz in Excel. Was muss ich da im Bezug auf die Namen der Spalten beachten? \textbf{(1 Punkt)}}\\[1ex]
Sie antworten:\\[2Ex]

Nilufar fragt: \textit{Es gibt ja nur ein richtiges Format für die Eingabe eines Datums. Wie lautet das Format? \textbf{(1 Punkt)}}\\[1ex]
Sie antworten:\\[2Ex]

Nilufar fragt: \textit{Wenn ich Daten in R mit Gruppen eingelesen habe, welche Funktion nutze ich dann meistens als erstes und warum muss ich das machen? Was muss ich da machen? \textbf{(2 Punkte)}}\\[1ex]
Sie antworten:\\[2Ex]

Nilufar fragt: \textit{Welche Funktionen brauche ich in welcher Reihenfolge um eine ANOVA zu rechnen? \textbf{(1 Punkt)}}\\[1ex]
Sie antworten:\\[2Ex]

Nilufar fragt: \textit{Das Dateiformat in R hat einen Namen. Wie heißt der und gerne mit Beispiel! \textbf{(1 Punkt)}}\\[1ex]
Sie antworten:\\[2Ex]

Nilufar fragt: \textit{Warum wurden jeweils die R Pakete \texttt{\{emmeans\}}, \texttt{\{ggplot\}} und \texttt{\{readxl\}} geladen? \textbf{(2 Punkte)}}\\[1ex]
Sie antworten:\\[2Ex]



 
\clearpage
% -----------------------------------------------------------------------
\part{Forschendes Lernen}

Das forschende Lernen basiert zum einen auf den folgenden wissenschaftlichen Veröffentlichungen. Für die Prüfung wird die vertiefende Kenntnis der folgenden Veröffentlichungen vorausgesetzt.\\

\textit{In der Prüfung erhalten Sie einen Auszug der wissenschaftlichen Veröffentlichung. Für die Einarbeitung in die Veröffentlichung ist in der Prüfung \underline{ausdrücklich} keine Zeit vorgesehen.}

\begin{itemize}[noitemsep]
\item Sánchez, M., Velásquez, Y., González, M., \& Cuevas, J. (2022). Hoverfly pollination enhances yield and fruit quality in mango under protected cultivation. Scientia Horticulturae, 304, 111320. [\href{https://www.sciencedirect.com/science/article/pii/S0304423822004411}{Link}]
% \item Salinas, I., Hueso, J. J., Força Baroni, D., \& Cuevas, J. (2023). Plant growth, yield, and fruit size improvements in ‘Alicia’papaya multiplied by grafting. Plants, 12(5), 1189. [\href{https://www.mdpi.com/2223-7747/12/5/1189}{Link}]
\item Petersen, F., Demann, J., Restemeyer, D., Olfs, H. W., Westendarp, H., Appenroth, K. J., \& Ulbrich, A. (2022). Influence of light intensity and spectrum on duckweed growth and proteins in a small-scale, re-circulating indoor vertical farm. Plants, 11(8), 1010. [\href{https://www.mdpi.com/2223-7747/11/8/1010}{Link}]
\item Selle, P. H., Cadogan, D. J., Li, X., \& Bryden, W. L. (2010). Implications of sorghum in broiler chicken nutrition. Animal Feed Science and Technology, 156(3-4), 57-74. [\href{https://www.sciencedirect.com/science/article/pii/S0377840110000209}{Link}]
\item Wu, G., Knabe, D. A., \& Kim, S. W. (2004). Arginine nutrition in neonatal pigs. The Journal of Nutrition, 134(10), 2783S-2790S. [\href{https://www.sciencedirect.com/science/article/pii/S0022316623031279}{Link}]
\end{itemize}

Das forschende Lernen basiert zum anderen auf den folgenden wissenschaftlichen Datensätzen und deren vertiefende Analyse werden als bekannt vorausgesetzt. Die Teilaufgaben der Aufgaben stellen nur eine zufällige Auswahl an möglichen Fragen dar. Die Datensätze werden über ILIAS bereitgestellt.\\

\textit{In der Prüfung erhalten Sie \underline{keinen Auszug} aus den wissenschaftlichen Daten. Die Datensätze werden als \underline{bekannt} in der Prüfung vorgesetzt. Sie haben sich vorab Notizen und Anmerkungen gemacht.}

\begin{itemize}[noitemsep]
\item bar
\end{itemize}

\clearpage
% -----------------------------------------------------------------------

\section{Aufgabe \hfill (20 Punkte)}

\textit{Geben Sie grundsätzlich Formeln und Rechenweg zur Lösung der Teilaufgaben mit an!} \\[1Ex]
 

 
%% --------------------------------------------------------------------
\begin{minipage}[t]{0.5\textwidth}
\includegraphics[width = 1.3cm]{/Users/kruppajo/work/GitHub/exam/avatare/Mark.png}
\end{minipage}
\begin{minipage}[t]{0.5\textwidth}
\hfill
\href{https://youtu.be/C9skfFRTHhI}{\includegraphics[width = 2cm]{img/caution}}
\end{minipage}
%% --------------------------------------------------------------------



\paragraph{Zerforschen einer wissenschaftlichen Veröffentlichung}

'Uff', denkt Mark. Das ist jetzt doch etwas umfangreicher. Mark soll die wissenschaftliche Veröffentlichung \textit{Sánchez, M., et al. (2022). Hoverfly pollination enhances yield and fruit quality in mango under protected cultivation} einmal zusammenfassen. Die Arbeit soll als eine Vorlage für seine eigene Arbeit dienen. Daher möchte sein Betreuer, dass er einmal die Veröffentlichung in einer PowerPoint Präsentation zusammenfasst. 'Das ist jetzt aber doch umfangreicher als gedacht.', mault Mark in sich hinein. Schnell nochmal ein paar Marzipankugeln zur Stärkung gegessen. Das wird dann vermutlich heute Abend nichts mehr mit seinem Hobby Geocaching. Der Hamster schaut mitleidig.\\

Leider kennt sich Mark mit dem Lesen einer wissenschaftlichen Veröffentlichung mit Fokus auf die Statistik überhaupt nicht aus. Deshalb braucht er bei der Erstellung Ihre Hilfe! Glücklicherweise kennen Sie die wissenschaftliche Veröffentlichung schon im Detail und können sofort helfen.

\begin{enumerate}
  \setcounter{enumi}{0}
  \item Erläutern Sie die wissenschaftliche Fragestellung der wissenschaftlichen Veröffentlichung anhand des OCAR Prinzips nach Schimel (2012)\footnote{Schimel, J. (2012). Writing science: how to write papers that get cited and proposals that get funded. OUP USA.} \textbf{(4 Punkte)}
  \item Nennen Sie die untersuchten Endpunkte in der wissenschaftlichen Veröffentlichung! Wie lautet der primäre Endpunkt? \textbf{(2 Punkte)} 
\item Erstellen Sie das statistische Modell in der in \Rlogo üblichen Schreibweise! \textbf{(2 Punkte)}
  \item Nennen Sie eine Auswahl an bedeutenden statistischen Maßzahlen in der wissenschaftlichen Veröffentlichung! \textbf{(1 Punkt)}
  \item Interpretieren Sie die Hauptaussage der wissenschaftlichen Veröffentlichung hinsichtlich der Signifkanz für den primären Endpunkt! \textbf{(2 Punkte)}
  \item Interpretieren Sie die Hauptaussage der wissenschaftlichen Veröffentlichung hinsichtlich der Effektstärke für den primären Endpunkt! \textbf{(2 Punkte)}
  \item Diskutieren Sie die ökonomische Relevanz der Hauptaussage der wissenschaftlichen Veröffentlichung im Bezug auf Signifikanz und Effektstärke für den primären Endpunkt! \textbf{(1 Punkt)}
  \item Skizzieren Sie für den primären Endpunkt den sich ergebenden Datensatz in \Rlogo für eine ausgewählte Abbildung! \textbf{(2 Punkte)}
\item Skizzieren Sie einen möglichen Versuchsplan für den primären Endpunkt! \textbf{(2 Punkte)}
  \item Schätzen Sie die benötigte Fallzahl für ein zukünftiges Experiment anhand der Ergebnisse in der wisenschaftlichen Veröffentlichung für den primären Endpunkt! \textbf{(2 Punkte)}
\end{enumerate} 
\clearpage
% -----------------------------------------------------------------------

\section{Aufgabe \hfill (20 Punkte)}

\textit{Geben Sie grundsätzlich Formeln und Rechenweg zur Lösung der Teilaufgaben mit an!} \\[1Ex]
 

 
%% --------------------------------------------------------------------
\begin{minipage}[t]{0.5\textwidth}
\includegraphics[width = 1.3cm]{/Users/kruppajo/work/GitHub/exam/avatare/Jonas.png}
\end{minipage}
\begin{minipage}[t]{0.5\textwidth}
\hfill
\href{https://youtu.be/C9skfFRTHhI}{\includegraphics[width = 2cm]{img/caution}}
\end{minipage}
%% --------------------------------------------------------------------



\paragraph{Zerforschen eines wissenschaftlichen Datensatzes}

Vor dem Start der eigenen Arbeit möchte sein Betreuer, dass Jonas einmal die wissenschaftlichen Daten \textit{data3} sinnvoll zusammenfasst. Dann würde die eigene Arbeit auch leichter von der Hand gehen und Jonas hätte dann schon eine Vorlage um die eigenen erhobenen Daten in eine Tabelle eintragen zu können. 'Das ist jetzt aber umfangreicher als gedacht!', schnauft er und runzelt die Stirn als er in seinen Laptop starrt. Dabei isst er noch ein paar Snickers. Das wird dann vermutlich heute Abend nichts mehr mit Mission Impossible\\

Leider kennt sich Jonas mit der Analyse eines wissenschaftlichen Datensatzes überhaupt nicht aus. Deshalb braucht er bei der Auswertung Ihre Hilfe! Glücklicherweise kennen Sie den wissenschaftlichen Datensatz aus Ihren eigenen Analysen schon im Detail und können sofort helfen.

\begin{enumerate}
  \setcounter{enumi}{0}
  \item Formulieren Sie die wissenschaftliche Fragestellung des Datensatzes in Form einer PowerPoint Folie! \textbf{(2 Punkte)}
  \item Nennen Sie zwei Besonderheiten des Datensatzes! Begründen Sie Ihre Antwort! \textbf{(2 Punkte)}
  \item Nennen Sie die untersuchten Endpunkte in dem Datensatz! Wie lautet der primäre Endpunkt für die Auswertung? \textbf{(2 Punkte)}
  \item Skizzieren Sie die großen Analysebereiche der Statistik! Beschriften Sie die Abbildungen! \textbf{(2 Punkte)}
  \item In welchen der großen Analysebereiche der Statistik fällt die Auswertung des primären Endpunktes? Begründen Sie Ihre Antwort! \textbf{(2 Punkte)}
  \item Skizzieren Sie eine ikonische Abbildung für den primären Endpunkt im Kontext der wissenschaftlichen Fragestellung! \textbf{(2 Punkte)}
  \item Erstellen Sie das statistische Modell in der in \Rlogo üblichen Schreibweise! \textbf{(2 Punkte)}
  \item Skizzieren Sie die Datenanalyse hinsichtlich der Signifkanz für den primären Endpunkt! \textbf{(2 Punkte)}
  \item Skizzieren Sie die Berechnung der Effektstärke für den primären Endpunkt! \textbf{(2 Punkte)}
  \item Skizzieren Sie einen möglichen Versuchsplan für den primären Endpunkt! \textbf{(2 Punkte)}
\end{enumerate} 
\clearpage
% -----------------------------------------------------------------------
\part{Mathematik}
% -----------------------------------------------------------------------  

\section{Aufgabe \hfill (10 Punkte)}

\textit{Geben Sie grundsätzlich Formeln und Rechenweg zur Lösung der Teilaufgaben mit an!} \\[1Ex]
 

 
%% --------------------------------------------------------------------
\begin{minipage}[t]{0.5\textwidth}
\includegraphics[width = 1.3cm]{/Users/kruppajo/work/GitHub/exam/avatare/Alex.png}\hspace{-4mm}\includegraphics[width = 1.3cm]{/Users/kruppajo/work/GitHub/exam/avatare/Jonas.png}\hspace{-4mm}\includegraphics[width = 1.3cm]{/Users/kruppajo/work/GitHub/exam/avatare/Steffen.png}\hspace{-4mm}\includegraphics[width = 1.3cm]{/Users/kruppajo/work/GitHub/exam/avatare/Tina.png}
\end{minipage}
\begin{minipage}[t]{0.5\textwidth}
\hfill
\href{https://youtu.be/Fu8kN0Uj13Y}{\includegraphics[width = 2cm]{img/youtube}}
\end{minipage}
%% --------------------------------------------------------------------



\paragraph{Herodot – der Schimmel aus Ivenack}

Die Lerngruppe \textit{Die Kühe auf dem Deich} bestehend aus Steffen, Alex, Tina und Jonas waren auf Exkursion in Mecklenburg-Vorpommern und haben dort Folgendes erarbeitet. Während der Besetzung Mecklenburgs durch die Franzosen kamen Napoleon die Geschichten des berühmten Apfelschimmels Herodot aus Ivenack zu Gehör. Herodot lief zwar niemals Rennen, war aber eines der berühmtesten Pferde dieser Zeit. Napoleon selbst gab den Auftrag, diesen Schimmel durch die Armee nach Frankreich zu bringen. Der Legende nach sollen Arbeiter den Schimmel im hohlen Stamm einer 1000-jährigen Eiche aus Ivenack vor den Franzosen versteckt haben. Doch Herodot verriet sein Versteck durch lautes Wiehern, woraufhin die französische Armee den Schimmel beschlagnahmte und nach Frankreich führte\footnote{Die Quelle der Inspiration  für die Aufgabe war eine Fahrt an die Ostsee und folgender Artikel:
  \href{https://www.wald-mv.de/landingpage/ivenacker-eichen/}{Entdecke das erste Nationale Naturmonument Deutschlands - Ivenacker Eichen und Hutewald}}. Jetzt wollen die vier herausfinden: \textit{"Konnten die Ivenacker den Apfelschimmel Herodot vor dem Zugriff von Napoleon in der 1000-jährigen Eiche verstecken?"} 



\vspace{1Ex}

Helfen Sie der Lerngruppe \textit{Die Kühe auf dem Deich} bei der Beantwortung der Forschungsfrage! Gehen Sie von einem radialen Wachstum der 1000-jährigen Eiche von $0.8mm$ pro Jahr aus. Es ist bekannt, dass die Eiche im Jahr 2022 einen Umfang von $12.5m$ in Brusthöhe hatte.

\begin{enumerate}
\item Wie groß war der Durchmesser in $m$ der Eiche im Jahr $1820$ als Herodot in der Eiche versteckt werden sollte? \textbf{(2 Punkte)}
\item Skizzieren Sie in einer Abbildung einen linearen Zusammenhang und einen exponentiellen Zusammenhang für das Wachstum der 1000-jährigen Eiche. Erklären Sie die Auswirkungen der Entscheidung für linear oder exponentiell auf Ihre Berechnungen! \textbf{(2 Punkte)}
\end{enumerate}
 
Herodot hatte eine Schulterhöhe von $185$cm, eine Breite von $80$cm sowie eine Länge von  $250$cm.

\begin{enumerate}
  \setcounter{enumi}{2}
\item Berechnen Sie das effektive Volumen von Herodot in $m^3$, welches Herodot in der 1000-jährigen Eiche einnehmen würde! \textbf{(2 Punkte)}
\end{enumerate}

Es wurde berichtet, dass sich Herodot in der 1000-jährigen Eiche $bequem$ um die eigene Achse drehen konnte.

\begin{enumerate}
  \setcounter{enumi}{3}
\item Berechnen Sie die Dicke der Eichenwand in $cm$! Verdeutlichen Sie Ihre Berechnungen an einer aussagekräftigen Skizze für Pferd und Eiche! \textbf{(2 Punkte)} 
\item Unter einer Dicke der Eichenwand von $15cm$ bricht die Eiche zusammen. Beantworten Sie die Forschungsfrage! Begründen Sie Ihre Antwort! \textbf{(2 Punkte)} 
\end{enumerate}
 
\clearpage
% ----------------------------------------------------------------------- 

\section{Aufgabe \hfill (10 Punkte)}

\textit{Geben Sie grundsätzlich Formeln und Rechenweg zur Lösung der Teilaufgaben mit an!} \\[1Ex]
 

 
%% --------------------------------------------------------------------
\begin{minipage}[t]{0.5\textwidth}
\includegraphics[width = 1.3cm]{/Users/kruppajo/work/GitHub/exam/avatare/Jessica.png}\hspace{-4mm}\includegraphics[width = 1.3cm]{/Users/kruppajo/work/GitHub/exam/avatare/Nilufar.png}\hspace{-4mm}\includegraphics[width = 1.3cm]{/Users/kruppajo/work/GitHub/exam/avatare/Paula.png}\hspace{-4mm}\includegraphics[width = 1.3cm]{/Users/kruppajo/work/GitHub/exam/avatare/Steffen.png}
\end{minipage}
\begin{minipage}[t]{0.5\textwidth}
\hfill
\href{https://youtu.be/57B-yYoFSk0}{\includegraphics[width = 2cm]{img/youtube}}
\end{minipage}
%% --------------------------------------------------------------------




\paragraph{Von Töpfen auf Tischen}



Die Projektgruppe \textit{B} bestehend aus Nilufar, Jessica, Steffen und Paula hat sich zusammengefunden um den ersten Versuch zu planen. In einem Experiment wollen sie die Wuchshöhe von 160 Sonnenblumen bestimmen. Bevor die Vier überhaupt mit dem Experiment beginnen können, gibt es aber ein paar Abschätzungen über die Kosten und den Aufwand zu treffen. Zum einen müssen sie die Sonnenblumen einpflanzen und müssen dafür Substrat bestellen. Zum anderen muss die Projektgruppe die Sonnenblumen auch bewegen und in ein Gewächshaus platzieren. Die Töpfe für die Keimung haben
einen Durchmesser von 10cm und eine Höhe von 8cm. Der Kubikmeterpreis für Torf liegt bei 290 EUR.

\vspace{1Ex}

Helfen Sie der Projektgruppe \textit{B} bei der Planung des Versuches!

\begin{enumerate}
\item Skizzieren Sie den Versuchsplan auf \textit{zwei} Tischen im Gewächshaus! \textbf{(2 Punkte)}
\item Berechnen Sie die benötigte Anzahl an Pflanztöpfen, wenn Sie Randpflanzen mit berücksichtigen wollen! \textbf{(1 Punkt)}
\item Welche $Tisch$fläche in $m^2$ gegeben der Anzahl an Pflanztöpfen inklusive Randpflanzen benötigen Sie im Gewächshaus am Anfang der Keimungsphase?  \textbf{(3 Punkte)}
\item Berechnen Sie die benötigte Menge an Torf in Liter $l$, die Sie für das Befüllen der Pflanztöpfe benötigen! Gehen Sie von \textit{einem Zylinder} für die Pflanztöpfe aus!  \textbf{(3 Punkte)}
\item Berechnen Sie die Kosten in EUR für Ihre Torfbestellung! \textbf{(1 Punkt)}
\end{enumerate}



 
\clearpage
% ----------------------------------------------------------------------- 

\section{Aufgabe \hfill (10 Punkte)}

\textit{Geben Sie grundsätzlich Formeln und Rechenweg zur Lösung der Teilaufgaben mit an!} \\[1Ex]
 

 
%% --------------------------------------------------------------------
\begin{minipage}[t]{0.5\textwidth}
\includegraphics[width = 1.3cm]{/Users/kruppajo/work/GitHub/exam/avatare/Jessica.png}\hspace{-4mm}\includegraphics[width = 1.3cm]{/Users/kruppajo/work/GitHub/exam/avatare/Mark.png}\hspace{-4mm}\includegraphics[width = 1.3cm]{/Users/kruppajo/work/GitHub/exam/avatare/Paula.png}\hspace{-4mm}\includegraphics[width = 1.3cm]{/Users/kruppajo/work/GitHub/exam/avatare/Steffen.png}
\end{minipage}
\begin{minipage}[t]{0.5\textwidth}
\hfill
\href{https://youtu.be/aBxLkdF-c4M}{\includegraphics[width = 2cm]{img/youtube}}
\end{minipage}
%% --------------------------------------------------------------------





\paragraph{Solar- \& Biogasanlagen}



Paula bringt ein neues, tolles Projekt mit in die Lerngruppe \textit{Die Blattläuse} bestehend aus ihr, Mark, Jessica sowie Steffen. Um die Energiekosten ihres Betriebes zu senken, will sie eine Solaranlage auf den Schweinestall montieren lassen. Dafür hat sie ihren Stall ausgemessen und findet folgende Maße wieder. Die vordere Seite des Schweinestall hat eine Höhe $h_v$ von $5m$. Die hintere Seite des Schweinestall hat eine Höhe $h_b$ von $11m$. Der Schweinestall hat eine Tiefe $t$ von $13m$ und eine Breite $b$ von $30m$. 'Sag mal Paula, ist das eine Matheaufgabe oder rechnen wir hier gerade für dich kostenlos als menschliche Computer Sachen für deinen Betrieb?', fragt Jessica mit erhobenenen Augenbrauen. Steffen und Mark nicken zustimmend.

\vspace{1Ex}

Wenn die Lerngruppe nicht will, dann müssen Sie bei der Planung helfen!

\begin{enumerate}
\item Skizzieren Sie den Schweinestall auf dem die Solaranlage montiert werden soll! Ergänzen Sie die Angaben für die Höhen $h_v$, $h_b$, die Tiefe $t$ und die Breite $b$ des Stalls!  \textbf{(2 Punkte)}
\item Berechnen Sie die Fläche der schrägen, neuen Solaranlage auf dem Schweinestall! \textbf{(3 Punkte)}
\end{enumerate}

Ebenfalls plant Paula eine neue Biogasanlage für ihren Betrieb. Der neue Methantank hat einen Radius $r$ von $1m$. Leider gibt es ein paar bauliche Beschränkungen auf dem Grundstück. Ihr Fundament des
zylindrischen Methantanks kann nur ein Gewicht von maximal $5t$ aushalten bevor der Tank wegbricht. Paula rechnen eine Sicherheitstoleranz von $10\%$ ein beinhaltend das Gewicht des
Methantanks. In flüssiger Form hat Methan bei $-80^\circ\text{C}$ eine Dichte von $235kg/m^3$. Bei $-100^\circ\text{C}$ hat Methan eine Dichte von $290kg/m^3$. Paula betreibt ihre Anlage bei $-95^\circ\text{C}$.

\begin{enumerate}
  \setcounter{enumi}{2}
\item Extrapolieren Sie die effektive Dichte des Methans in Ihrem Methantank! Welche Annahme haben Sie getroffen? \textbf{(1 Punkt)}
\item Berechnen Sie wie viel Kubikmeter $m^3$ Sie in den Methantank füllen können, bevor das Fundament nachgibt! \textbf{(2 Punkte)}
\item Berechnen Sie die maximale Höhe $h_{max}$ in $m$ für den gefüllten Methantank mit dem Radius $r$, bevor das Fundament wegbricht! \textbf{(2 Punkte)}
\end{enumerate}

 
\clearpage
% -----------------------------------------------------------------------

\section{Aufgabe \hfill (10 Punkte)}

\textit{Geben Sie grundsätzlich Formeln und Rechenweg zur Lösung der Teilaufgaben mit an!} \\[1Ex]
 

 
%% --------------------------------------------------------------------
\begin{minipage}[t]{0.5\textwidth}
\includegraphics[width = 1.3cm]{/Users/kruppajo/work/GitHub/exam/avatare/Yuki.png}
\end{minipage}
\begin{minipage}[t]{0.5\textwidth}
\hfill
\href{https://youtu.be/https://youtu.be/k2G52hMIfqk}{\includegraphics[width = 2cm]{img/youtube}}
\end{minipage}
%% --------------------------------------------------------------------

%% --------------------------------------------------------------------
{\tiny\textbf{Stichworte:} Riesenfaultier $\bullet$ Evolution der Avocado $\bullet$ Bluetooth $ \bullet$ Blauzahn $\bullet$ Colonia Dignidad $\bullet$ ODESSA $\bullet$ Rattenlinie $\bullet$ Adolf Eichmann}
%% --------------------------------------------------------------------



\paragraph{Aligatorenbirnen und Blaubeeren}



"'Sind Sie ein Riesenfautier oder warum kaufen Sie so viele Aligatorenbirnen?"', spricht es hinter Ihnen. Irritiert drehen Sie sich um und blicken in das puderrote Gesicht von Yuki. "'Wieso?"', entfährt es Ihnen und Sie bereuen sogleich die Frage. Sofort werden Sie zu einem Whiteboard voller roter Schnüre geschliffen und müssen folgenden mathematischen untermauerten Argumenten im Lidl über sich ergehen lassen. Da kommen Sie nicht mehr raus, also können Sie auch gleich mitmachen. Das Problem liegt in Chile\footnote{Die Quelle der Inspiration für die Aufgabe waren folgende Reportagen: \href{https://www.amnesty.ch/de/ueber-amnesty/publikationen/magazin-amnesty/2021-3/bis-zum-letzten-tropfen}{"`Bis zum letzten Tropfen"' in AMNESTY – Magazin der Menschenrechte vom August 2021} und \href{https://www.welthungerhilfe.de/welternaehrung/rubriken/klima-ressourcen/wassernot-in-chile-eine-folge-der-privatisierung}{"`Wasserknappheit in Chile: Eine Folge der Privatisierung?"' in Die Welternährung dem Fachjournal der Welthungerhilfe vom April 2022.}}. Tja, die Deutschen und Südamerika.\\

Zuerst werden Ihre Fähigkeiten getestet, der Mathematik folgen zu können. Oder berechnen Sie gerade den Einkauf von Yuki?\\

\begin{enumerate}
\item Wenn 3 Blaubeerschalen 5.37 Euro kosten,  wie viel kosten 5 Schalen? \textbf{(2 Punkte)}
\item Wenn Sie die 5 Blaubeerschalen gekauft haben, wie viele Aligatorbirnen zu je 2.89 EUR können Sie sich dann noch für 200 EUR leisten? \textbf{(1 Punkt)}
\end{enumerate}

Das Whiteboard beinhaltet folgende Liste mit Informationen zum Wasserverbrauch bei der Produktion von Gemüse aus Chile. Seltsam, was man so alles in einem Lidl über Gemüse erfährt.
  
\begin{itemize}[noitemsep]
\item Ein Kilo Strauchtomaten benötigt 190l Wasser. Eine Strauchtomate wiegt 100 - 120g.
\item Ein Kilo Salat benötigt 140l Wasser. Ein Salatkopf wiegt 300 - 490g.
\item Ein Kilo Avocado benötigt 950l Wasser. Eine Avocado wiegt 150 - 410g.
\item Ein Kilo Blaubeeren benötigt 830l Wasser. Eine Blaubeere wiegt 3 - 3.6g.
\end{itemize}

\begin{enumerate}
  \setcounter{enumi}{2}
\item Berechnen Sie den Wasserverbrauch für die Produktion für jeweils eine Strauchtomate, einem Salat, einer Avocado und einer Blaubeeren. Stellen Sie das Ergebnis als Tabelle dar! \textbf{(3 Punkte)}
\end{enumerate}

Chile exportiert im großem Ausmaß Blaubeeren und Avocados. In dem Exportjahr 2022 blieben die Erträge von Blaubeeren mit \ensuremath{7.9\times 10^{4}}t in dem prognostizierten Rahmen. Die Menge steigerte sich um 6.8\%. Die Exporte für Avocados stiegen in dem gleichen Zeitraum um 22.1\% auf \ensuremath{1.9\times 10^{5}}t.

\begin{enumerate}
  \setcounter{enumi}{3}
\item Wie viele Liter Wasser hat Chile in dem Exportjahr 2021 exportiert? \textbf{(2 Punkte)}
\end{enumerate}

Chile ist eines der wenigen Länder der Welt, die ihr Wasser komplett privatisiert haben. Derzeit sind nur zwei Prozent des Wassers des Landes für den häuslichen Verbrauch vorgesehen. In den Dörfern der Anbauregionen versorgen Tankwagen die Bevölkerung jede Woche mit Wasser, es gibt etwa 55 Liter Wasser pro Kopf für den täglichen Bedarf. In \textit{Deutschland} liegt der Verbrauch bei 35 - 115 Liter pro Waschgang einer Waschmaschine und 8 - 17 Liter pro Sp{"u}lmaschinenlauf.

\begin{enumerate}
  \setcounter{enumi}{4}
\item Mit der rationierten Wassermenge aus Chiles Anbaugebieten können Sie in \textit{Deutschland} wie oft Ihren Bedarf stillen? \textbf{(1 Punkt)}
\end{enumerate}

Das alles hätten Sie nicht von Yuki erwartet. Ganz schön viele Informationen wurden da zusammengetragen.

\begin{enumerate}
  \setcounter{enumi}{5}  
  \item Nennen Sie eine \textit{Daten}quelle im Internet, wo Sie mehr Informationen zu landwirtschaftlichen Daten oder klimatischen, wirtschaftlichen und gesellschaftlichen Daten erhalten! \textbf{(1 Punkt)}
\end{enumerate} 
\clearpage
% ----------------------------------------------------------------------- 

\section{Aufgabe \hfill (12 Punkte)}

\textit{Geben Sie grundsätzlich Formeln und Rechenweg zur Lösung der Teilaufgaben mit an!} \\[1Ex]
 

 
%% --------------------------------------------------------------------
\begin{minipage}[t]{0.5\textwidth}
\includegraphics[width = 1.3cm]{/Users/kruppajo/work/GitHub/exam/avatare/Jonas.png}
\end{minipage}
\begin{minipage}[t]{0.5\textwidth}
\hfill
\href{https://youtu.be/WZSxntiNF8s}{\includegraphics[width = 2cm]{img/youtube}}
\end{minipage}
%% --------------------------------------------------------------------

%% --------------------------------------------------------------------
{\tiny\textbf{Stichworte:} Kardaschow-Skala $\bullet$ Dyson-Sphäre $\bullet$ Hohlerde $\bullet$ Entropie $\bullet$ Proton $r_P = 1.7 \times 10e-15$ $\bullet$ Wasserstoff $r_H = 5.3\times 10e-11$}
%% --------------------------------------------------------------------

\paragraph{Die Dampfnudelerde}



"'Was für einen Unsinn!"', rufen Sie. Jetzt haben Sie auf Empfehlung von von Jonas kostbaren Schlaf prokrastiniert um einem Ernährungswissenschaftler auf YouTube über die Erde als Dampfnudel zu lauschen. Irgendwie passt es dann doch mit der Analogie. Übermüdet müssen Sie darüber nachdenken, warum vor 65 Millionen Jahren die Dinosaurier - so groß sie auch waren - nicht von der Schwerkraft zu Boden gerissen wurden. Hat der Dampfplauderer etwa recht und war die Schwerkraft vor Millionen von Jahren eine andere?  Sind deshalb alle Lebewesen auf der Erde \textit{heutzutage} so viel kleiner, weil die Schwerkraft größer ist als damals? War die Erde kleiner und hatte weniger Masse? Oder ist es nur ein Rechenfehler wie bei der Theorie der Hohlerde von Edmond Halley aus dem 17.–18. Jahrhundert? Müde reiben Sie sich die Augen. So wird es nichts mehr mit dem Schlafen, dann können Sie auch mal etwas rechnen\footnote{Die Quelle der Inspiration
  für die Aufgabe war folgender Artikel:
  \href{https://hpd.de/artikel/erde-dampfnudel-22236}{"Skeptische Anmerkungen --- Die Erde als Dampfnudel" in Der Humanistische Pressedienst}}.  \\

Betrachten wir die Schwerkraft oder Gewichtskraft, die auf Lebewesen damals und heute gewirkt haben soll. Nehmen Sie für die Fallbeschleunigung $g$ der Erde \textit{heutzutage} einen Wert von 9.65m/s$^2$ an. Im Weiteren hat die Erde einen ungefähren Durchmesser von \ensuremath{1.235\times 10^{4}}km und eine mittlere Dichte $\rho$ von 5.86g/cm$^3$. Das Gewicht von einem heute lebenden afrikanischen Elefanten liegt bei 5t bis 7t und das Gewicht von einem Brachiosaurus bei bis zu 30t.

\begin{enumerate}
\item Welchen Durchmesser müsste die Erde vor 65 Millionen Jahren gehabt haben, wenn Dinosaurier und Elefanten die gleiche Gewichtskraft $\overrightarrow{F_G}$ damals und heute erfahren hätten? \textit{Beantworten Sie die Frage anhand der folgenden Teilaufgaben!}
\begin{enumerate}
\item Berechnen Sie die Fallbeschleunigung von vor 65 Millionen Jahren unter der obigen Annahme gleich wirkender Gewichtskraft $\overrightarrow{F_G}$ auf Elefant und Dinosaurier! \textbf{(1 Punkt)}
\item Berechnen Sie Masse der heutigen Erde! \textbf{(2 Punkte)}
\item Schließen Sie über die Masse auf den Durchmesser der Erde vor 65 Millionen Jahren! \textbf{(2 Punkte)}
\end{enumerate}
\item Beantworten Sie die Eingangsfrage mit 1-2 Antwortsätzen! \textbf{(1 Punkt)}
\end{enumerate}

Die Distanz zwischen Sonne und Erde entspricht 1.03 astronomische Einheiten ($AE$). Die Einheit 1 AE wird mit \ensuremath{1.55\times 10^{8}}km angegeben. Der \textit{massebehaftete} Sonnenwind besteht aus 84\% Wasserstoffkernen mit einer molaren Masse von 1.05g/mol, 12\% Heliumkernen mit 4.01g/mol sowie 4\% weiteren Atomkernen mit  69.18g/mol. Die Teilchendichte bei Eintritt in die Erdatmosphäre liegt zwischen 0.4 bis 100 Teilchen cm$^{-3}$ pro Sekunde mit einer mittleren Teilchendichte von 8cm$^{-3}$ pro Sekunde. \\

\textit{Lösen Sie den folgenden Aufgabenteil mit einer aussagekräftigen Skizze!}

\begin{enumerate}
  \setcounter{enumi}{3}
\item Berechnen Sie die Anzahl an massebehafteten Teilchen des Sonnenwindes, die die gesamte Erde pro Sekunde treffen! \textbf{(2 Punkte)}
\item Berechnen Sie die Anzahl an massebehafteten Teilchen des Sonnenwindes, die die Sonne pro Sekunde in alle Richtungen aussendet! \textbf{(2 Punkte)}
\item Berechnen Sie die Masse, die die Erde pro Jahr durch die \textit{massebehafteten} Teilchen des Sonnenwind zunimmt! \textbf{(2 Punkte)}
\end{enumerate}

%\blfootnote{\tiny\textbf{Stichworte:} Kardaschow-Skala $\bullet$ Dyson-Sphäre $\bullet$ Hohlerde $\bullet$ Entropie}

% https://de.wikipedia.org/wiki/Kardaschow-Skala 
\clearpage
% ----------------------------------------------------------------------- 

\section{Aufgabe \hfill (10 Punkte)}

\textit{Geben Sie grundsätzlich Formeln und Rechenweg zur Lösung der Teilaufgaben mit an!} \\[1Ex]
 

 
%% --------------------------------------------------------------------
\begin{minipage}[t]{0.5\textwidth}
\includegraphics[width = 1.3cm]{/Users/kruppajo/work/GitHub/exam/avatare/Jonas.png}\hspace{-4mm}\includegraphics[width = 1.3cm]{/Users/kruppajo/work/GitHub/exam/avatare/Nilufar.png}\hspace{-4mm}\includegraphics[width = 1.3cm]{/Users/kruppajo/work/GitHub/exam/avatare/Paula.png}\hspace{-4mm}\includegraphics[width = 1.3cm]{/Users/kruppajo/work/GitHub/exam/avatare/Yuki.png}
\end{minipage}
\begin{minipage}[t]{0.5\textwidth}
\hfill
\href{https://youtu.be/n451XnhtSh4}{\includegraphics[width = 2cm]{img/youtube}}
\end{minipage}
%% --------------------------------------------------------------------




\paragraph{'Entschuldigung, ist das Ihre Feder in meinem Auge?'}



So hört man häufiger höfliche Gänse in Mastställen sagen. Das ist natürlich etwas ungünstig, den dann kommt es zu Picken und Kannibalismus. Denn wenn der Nachbar nervt, dann muss zu Maßnahmen gegriffen werden. Kennt jeder aus einer mittelmäßigen Wohngemeinschaft. Das wollen Yuki, Nilufar, Paula und Jonas aber als vorsorgliche Gänse-Halter:innen nicht\footnote{Die Quelle der Inspiration für die Aufgabe war der folgende wissenschaftliche Artikel: \href{https://www.efsa.europa.eu/en/efsajournal/pub/7788}{EFSA Panel on Animal Health and Welfare, et al. (2023) Welfare of broilers on farm. EFSA Journal 21.2}.}. Gemeinsam sind die Vier in einer Projektgruppe gelandet. Betrachten wir also gemeinsam einmal das Platzangebot (eng. \textit{space allowance}, abk. \textit{SA}) der Gänse für vier Tätigkeiten und versuchen die notwendige Fläche zu optimieren. Wie immer gibt es dafür eine mathematische Formel:

\begin{center}
  \begin{tabular}{cc}
    $SA = \sum^n_{i = 1} (A_i \times PB_i)$ & $A_i = \pi \times (r_i + R_i)^2$\\
  \end{tabular}
\end{center}

\vspace{-2Ex}

mit

\begin{itemize}[noitemsep]
\item $SA$ dem benötigten Platzangebot aller aufsummierten Verhalten $i$.
\item $A_i$ dem benötigten Platz für ein Verhalten $i$. 
\item $PB_i$ dem Anteil des Auftretens eines Verhaltens $i$.
\item $r_i$ dem Radius Gans plus dem benötigten Radius für das Verhalten $i$.
\item $R_i$ dem notwendigen Abstand zu den Nachbarn für das Verhalten $i$.    
\item $i$ dem Verhalten: (1) sitting, (2) drinking/eating, (3)
  dustbathing und (4) standing.
\end{itemize}

In der folgenden Tabelle 1 sind die Werte für $r_i$, $R_i$ und $PB_i$ für ein spezifisches Verhalten $i$ aus drei wissenschaftlichen Veröffentlichungen dargestellt.

\vspace{-1Ex}

{\small
\begin{knitrout}
\definecolor{shadecolor}{rgb}{0.969, 0.969, 0.969}\color{fgcolor}\begin{table}[!h]
\centering
\begin{tabular}{llll}
\toprule
  & Aldridge et al. (2021) & Baxter et al. (2022) & Jabcobs et al. (2019)\\
\midrule
sitting & 40cm; 19cm; 7.2\% & 31cm; 27cm; 12.1\% & 25cm; 19cm; 8.9\%\\
drinking/eating & 33cm; 15cm; 50.1\% & 34cm; 24cm; 64.1\% & 29cm; 21cm; 50.1\%\\
dustbathing & 32cm; 27cm; 1.6\% & 41cm; 22cm; 2.4\% & 33cm; 24cm; 0.5\%\\
standing & 37cm; 24cm; 13.2\% & 33cm; 22cm; 16.4\% & 38cm; 24cm; 18.1\%\\
\bottomrule
\end{tabular}
\end{table}

\end{knitrout}
}

Leider kennen sich die Vier nicht so gut mit der Berechnung aus! Daher brauchen die Vier Ihre Hilfe!

\begin{enumerate}
\item Erstellen Sie eine zusammenfassende Tabelle mit den mittleren Werten für $r$, $R$ und $PB$ aus der obigen Tabelle 1 für die jeweiligen Verhalten! \textbf{(3 Punkte)}
\item Ergänzen Sie eine Spalte mit dem benötigten Platz $A$ für das jeweilige Verhalten, welches sich aus den mittleren Werten ergibt! \textbf{(1 Punkt)}
\item Berechnen Sie das benötigte Platzangebot $SA$ für alle betrachteten Verhalten! \textbf{(1 Punkt)}
\item Skizzieren Sie die Werte $r_i$, $R_i$ und $A_i$ für zwei nebeneinander agierende Gänse für ein Verhalten $i$. Nutzen Sie hierfür vereinfachte geometrische Formen! \textbf{(2 Punkte)}
\item Sie entnehmen der Literatur folgende Aussage zur Verteilung der Gänse in der Fläche $A$: \textit{"`Assuming, that the animals will optimally and equally distribute in an area $A$, we observe a
    small part, which is not covered. This area is called $\omega$ and is calculated with $\omega = \tfrac{A}{0.9069}$."'} Veranschaulichen Sie die Fläche $\omega$ in einer aussagekräftigen Abbildung!  \textbf{(1 Punkt)}
\item Ein Tier braucht Platz für sich selbst. Berechnen Sie nun die Körperfläche $a$, die ein Tier einnimmt. Welche Annahmen haben Sie für die Berechnung der Körperfläche getroffen? \textbf{(2 Punkte)}
\end{enumerate}



 
\clearpage
% ----------------------------------------------------------------------- 

\section{Aufgabe \hfill (8 Punkte)}

\textit{Geben Sie grundsätzlich Formeln und Rechenweg zur Lösung der Teilaufgaben mit an!} \\[1Ex]
 

 
%% --------------------------------------------------------------------
\begin{minipage}[t]{0.5\textwidth}
\includegraphics[width = 1.3cm]{/Users/kruppajo/work/GitHub/exam/avatare/Jessica.png}\hspace{-4mm}\includegraphics[width = 1.3cm]{/Users/kruppajo/work/GitHub/exam/avatare/Yuki.png}
\end{minipage}
\begin{minipage}[t]{0.5\textwidth}
\hfill
\href{https://youtu.be/1B53cVFIU7Q}{\includegraphics[width = 2cm]{img/youtube}}
\end{minipage}
%% --------------------------------------------------------------------




\paragraph{Nelken von den Molukken}



Jessica und Yuki waren gemeinsam in Berlin und sitzen nun im IC nach Amsterdam um zurück nach Osnabrück zu fahren. 'Weißt du was ich mich frage?', entfährt es Jessica ziemlich plötzlich, so dass Yuki die Reese's Peanut Butter Cups aus dem Mund fallen. 'Nein, und ehrlich gesagt bin ich auch ziemlich müde...'. Das ist jetzt aber Jessica egal, denn sie möchte folgende Sachlage diskutieren. Und Jessica hat jetzt 3 Stunden Zeit. Plus Verspätung. In der Ausstellung \textit{Europa und das Meer} im Deutschen Historischen Museum in Berlin gab es folgendes Zitat über die Probleme der frühen Hochseeschifffahrt.

\begin{quote}
  >>Ohne ausreichende Zufuhr von Vitamin C stellen sich nach 50 Tagen die ersten Symptome ein; die ersten Toten sind nach 72 Tagen zu beklagen; nach 110 Tagen rafft die Skorbut eine ganze Schiffsbesatzung dahin<<
\end{quote}

Ferdinand Magellan stach im Jahre 1519 in See um eine Passage durch den südamerikanischen Kontinent zu finden. Zu seiner Flotte gehörten fünf Schiffe - das Flaggschiff Trinidad, die San Antonio, die Victoria, die Concepciön und die Santiago - mit einer Besatzung von insgesamt 222 Mann. 

\begin{enumerate}
\item Stellen Sie den Verlauf der Anzahl an Matrosen auf einem Schiff der Flotte in der Form einer überlebenszeitkurve dar! Beschriften Sie die Achsen entsprechend! \textbf{(2 Punkte)} 
\item Was ist die Besonderheit der Überlebenszeitkurve? Begründen Sie Ihre Antwort! \textbf{(2 Punkte)} 
\item Schätzen Sie die überlebenswahrscheinlichkeit nach 100 Tagen aus Ihrer Abbildung ab! \textbf{(1 Punkt)} 
\end{enumerate}

Der Chronist an Bord der Trinidad, Antonio Pigafetta, schrieb in seinem Bericht '[...] Um nicht Hungers zu sterben, aßen wir das Leder, mit dem die große Rahe zum Schutz der Taue umwunden war.' Insbesondere die Mannschaft der Concepciön erlitt große Verluste durch die Skrobut bei der überquerung des Pazifiks, da durch Erkundungsfahrten weniger Zeit blieb, um wilden Sellerie aufzunehmen. Wilder Sellerie enthält $5000\mu g/10mg$ Vitamin C. Der Bedarf liegt bei $110mg$ pro Tag für Männer.

\begin{enumerate}
  \setcounter{enumi}{2}
\item Berechnen Sie die notwendige Menge in $t$ an aufzunehmenden wilden Sellerie auf die Concepciön für die ununterbrochene Fahrt von drei Monate und 20 Tage über den Pazifik! \textbf{(3 Punkte)}
\item Skizzieren Sie die überlebenszeitkurve für die Concepciön im Vergleich zu der überlebenszeitkurve der Trinidad! Beschriften Sie die Achsen! \textbf{(2 Punkte)}
\end{enumerate}

 
\clearpage
% ----------------------------------------------------------------------- 

\section{Aufgabe \hfill (10 Punkte)}

\textit{Geben Sie grundsätzlich Formeln und Rechenweg zur Lösung der Teilaufgaben mit an!} \\[1Ex]
 

 
%% --------------------------------------------------------------------
\begin{minipage}[t]{0.5\textwidth}
\includegraphics[width = 1.3cm]{/Users/kruppajo/work/GitHub/exam/avatare/Paula.png}\hspace{-4mm}\includegraphics[width = 1.3cm]{/Users/kruppajo/work/GitHub/exam/avatare/Steffen.png}
\end{minipage}
\begin{minipage}[t]{0.5\textwidth}
\hfill
\href{https://youtu.be/q-qYK4Chslg}{\includegraphics[width = 2cm]{img/youtube}}
\end{minipage}
%% --------------------------------------------------------------------




\paragraph{Event Horizon -- Am Rande des Universums}



Paula ist bei Steffen um gemeinsam \textit{Event Horizon -- Am Rande des Universums} zu streamen. Das war jetzt nicht die beste Idee. Denn Paula kann Horror überhaupt nicht ab. Deshalb flüchtet sie sich in Logik um ihre Emotionen zu bändigen. Steffen mampft ungerührt Oreos. Folgenden Gedankengang nutzt Paula um dem Film zu entkommen. Die Sonne hat eine aktuelle, angenommene Masse von $\ensuremath{2\times 10^{29}}$kg. Wenn die Sonne nun am Ende ihrer Lebenszeit zu einem schwarzen Loch mit dem Radius von $3500$m kollabiert, wird die Sonne $40$\% der aktuellen Masse verloren haben. Ein Lichtteilchen mit der Masse $m_f$ und der Fluchtgeschwindigkeit $v_f$ will dem schwarzen Loch entkommen. An folgende Formeln erinnert sich Paula für die kinetische Energie des Lichtteilchens $E_{kin}$ und der Graviationsenergie des schwarzen Lochs $E_{grav}$\footnote{Die Quelle der Inspiration für die Aufgabe war ein Montagnachtfilm: \href{https://de.wikipedia.org/wiki/Event_Horizon_–_Am_Rande_des_Universums}{Event Horizon – Am Rande des Universums}}.

\begin{center}
  \begin{tabular}{cc}
    $E_{kin} = \cfrac{1}{2}m_fv_f^2$ & $E_{grav} = \cfrac{Gm_sm_f}{r_s}$\\
  \end{tabular}
\end{center}

mit

\begin{itemize}[noitemsep]
\item $m_f$, gleich der Masse [kg] des fliehenden Objektes
\item $m_s$, gleich der Masse [kg] des stationären Objekts
\item $r_s$, gleich dem Radius [m] des stationären Objekts  
\item $G$, gleich der Gravitationskonstante mit $6.274 \cdot 10^{-11} m^3(kg \cdot s^2)^{-1}$ 
\end{itemize}

Im Folgenden wollen wir Paula bei der Ablenkung helfen und uns mit der Frage beschäftigen, ob das Lichtteilchen der Gravitation des schwarzen Lochs entkommen kann.

\begin{enumerate}
\item Geben Sie die Formel für die Fluchtgeschwindigkeit $v_f$ an! \textbf{(1 Punkt)}
\item überprüfen Sie Ihre umgestellte Formel nach $v_f$ anhand der Einheiten! \textbf{(1 Punkt)} 
\item Berechnen Sie die notwendige Fluchtgeschwindigkeit $v_f$ des Lichtteilchens mit den angegebenen Informationen! \textbf{(2 Punkte)}
\item Gehen Sie von einer Lichtgeschwindigkeit von $\ensuremath{2.7\times 10^{8}}m/s$ aus. Kann das Lichtteilchen der Gravitation des schwarzen Lochs entkommen? Begründen Sie Ihre Antwort! \textbf{(2 Punkte)}
\item Stellen Sie den Zusammenhang zwischen dem sich verringernden Radius $r$ des schwarzen Lochs bei gleichbleibender Masse $m_s$ und der notwendigen Fluchtgeschwindigkeit $v_f$ in einer Abbildung dar!\textbf{(2 Punkte)}
 \item Ein Flugzeug und eine Feder stürzen aus großer und gleicher Höhe in ein schwarzes Loch. Welches der beiden Objekte überschreitet zuerst den Ereignishorizont des schwarzes Loches? Begründen Sie Ihre Antwort mathematisch! \textbf{(2 Punkte)}  
\end{enumerate}

 
\clearpage
% -----------------------------------------------------------------------

\section{Aufgabe \hfill (10 Punkte)}

\textit{Geben Sie grundsätzlich Formeln und Rechenweg zur Lösung der Teilaufgaben mit an!} \\[1Ex]
 

 
%% --------------------------------------------------------------------
\begin{minipage}[t]{0.5\textwidth}
\includegraphics[width = 1.3cm]{/Users/kruppajo/work/GitHub/exam/avatare/Nilufar.png}\hspace{-4mm}\includegraphics[width = 1.3cm]{/Users/kruppajo/work/GitHub/exam/avatare/Paula.png}
\end{minipage}
\begin{minipage}[t]{0.5\textwidth}
\hfill
\href{https://youtu.be/iCQogS6KhPM}{\includegraphics[width = 2cm]{img/youtube}}
\end{minipage}
%% --------------------------------------------------------------------

%% --------------------------------------------------------------------
{\tiny\textbf{Stichworte:} Great filter $\bullet$ SETI $\bullet$ WOW-Signal $\bullet$ 5-Sigma $\bullet$ Voyager 1 $\bullet$ Voyager 2}
%% --------------------------------------------------------------------




\paragraph{Das Fermi Paradoxon}



Nilufar und Paula wandern durch den Teuteburgerwald um mal vom Studium runterzukommen. 'Kennst du eigentlich Enrico Fermi?', fragt Nilufar und fährt ohne die Antwort abzuwarten fort, 'Er war ein berümter Kernphysiker! Enrico Fermi diskutierte 1950 auf dem Weg zum Mittagessen im Los Alamos National Laboratory mit seinen Kollegen angebliche UFO-Sichtungen und fragte schließlich: >>Where is everybody?<<. Warum seien weder Raumschiffe anderer Weltraumbewohner noch andere Spuren extraterrestrischer Technik zu beobachten?'. Paula schaut sie irritiert und interessiert an. Die beiden hat das Problem gepackt. Deshalb wollen Nilufar und Paula das Paradoxon mal mathematisch untersuchen! Wie lange würde eine außerirdische Zivilisation benötigen um die gesamte Milchstraße zu besuchen, wenn das maximale Reisetempo die Geschwindigkeit der Voyager 1 Sonde wäre?\footnote{Die Quelle der Inspiration für die Aufgabe war folgender Wikipediaeintrag: \href{https://de.wikipedia.org/wiki/Fermi-Paradoxon}{Fermi-Paradoxon}}\\[-1ex]

Die beiden treffen folgende Annahmen. Eine außerirdische Zivilisation schickt $zwei$ Voyager 1 ähnliche Sonden mit der Geschwindigkeit von $\ensuremath{5.9256\times 10^{4}}km/h$ los um sich auf den erreichten Planeten selbst zu replizieren. Nach $500$ Jahren ist die Replikation abgeschlossen und wiederum $zwei$ Sonden werden ausgesendet. Gehen Sie von $6.23$ Lichtjahren als mittlerer Abstand der Sterne in der Milchstraße aus. Es gibt $\ensuremath{2\times 10^{11}}$ Sterne in der Milchstraße. Nehmen Sie eine Lichtgeschwindigkeit von $\ensuremath{2.7\times 10^{8}}m/s$ an.

\begin{enumerate}
\item Skizzieren Sie in einer Abbildung die ersten vier Schritte der Vervielfältigung der Sonden in der Galaxie! Beschriften Sie die Abbildung mit der Dauer und der Anzahl an Sonden für jeden Schritt der Vervielfältigung! \textbf{(4 Punkte)}
\item Berechnen Sie die theoretische Anzahl an Vervielfältigungsschritten die benötigt werden um mit \textit{einem einzigen Vervielfältigungsschritt} die gesamten Sterne der Milchstraße mit Sonden zu besuchen! \textbf{(2 Punkte)}
\item Berechnen Sie die Dauer, die eine außerirdische Zivilisation annährungsweise benötigt um die gesamten Sterne der Milchstraße mit Sonden zu besuchen! \textbf{(2 Punkte)}
\item Bei einem vermutetet Alter der Erde von $\ensuremath{4.3\times 10^{9}}$ Jahren, wie oft war dann eine Sonde einer außerirdischen Zivilisation schon zu Besuch? Korrigieren Sie Ihre Antwort mit dem Wissen, dass sich die Kontinentalplatten einmal alle $\ensuremath{9\times 10^{7}}$ Jahre vollständig im Erdinneren umgewandelt haben! \textbf{(2 Punkte)}
\end{enumerate}


 
\clearpage
% -----------------------------------------------------------------------

\section{Aufgabe \hfill (10 Punkte)}

\textit{Geben Sie grundsätzlich Formeln und Rechenweg zur Lösung der Teilaufgaben mit an!} \\[1Ex]
 

 
%% --------------------------------------------------------------------
\begin{minipage}[t]{0.5\textwidth}
\includegraphics[width = 1.3cm]{/Users/kruppajo/work/GitHub/exam/avatare/Tina.png}
\end{minipage}
\begin{minipage}[t]{0.5\textwidth}
\hfill
\href{https://youtu.be/tDgr6fpkkYA}{\includegraphics[width = 2cm]{img/youtube}}
\end{minipage}
%% --------------------------------------------------------------------




\paragraph{Pyramiden bauen}



Es stehen die mecklemburgischen Pyramidentage! Sie und Tina sind auf abenteuerlichen Wegen für den Bau der Pyramiden zuständig. Zu allem Überfluss handelt es sich auch noch eine \textit{Reenactment} Veranstaltung. Thema der diesjährigen Pyramidentage sind die Pyramiden von Meroe, die den Königen und Königinnen des historischen Reiches von Kusch in Nubien, dem heutigen Sudan, als Grabstätten dienten. Die Pyramiden in Meroe fallen durch ihren steilen Winkel von 73 Grad im Vergleich zu den ägyptischen Pyramiden mit 53 Grad auf. Die durchschnittliche Seitenlänge der Grundfläche einer Pyramide beträgt 33 Königsellen. Eine Königselle misst 52.2cm.\\

\textit{Lösen Sie diese Aufgabe mit Hilfe einer Skizze der Pyramide. Bezeichnen Sie Seiten und die Winkel der Pyramide entsprechend!}

\begin{enumerate}
\item Bei der Königspyramide von Meroe soll eine Seitenlänge der Grundfläche 33 Königsellen lang sein. Welche Höhe der Königspyramide in $m$ ergibt sich? \textbf{(1 Punkt)}
\item Die Außenflächen der Pyramide soll begrünt werden. Für die Bepflanzung muss eine 5cm dicke Torfschicht auf die Pyramide aufgebracht werden. Berechnen Sie die ungefähre Menge an benötigten Torf in $m^3$! \textbf{(2 Punkte)}
\end{enumerate}

Wie in jedem guten \textit{Reenactment} gibt es viel Oberschicht, aber nur 2 Sklaven, die Ihnen und Tina bei dem Befüllen der Pyramide mit Schutt zu Seite stehen. Leider haben Ihre Sklaven zu allem Überfluss auch noch chronische Knieschmerzen entwickelt, als die Sklaven von der anstehenden Aufgabe erfahren haben. Gehen Sie daher von einer Effizienz der Sklaven von 75\% aus. In eine Schubkarre passen 95 Liter.

\begin{enumerate}
  \setcounter{enumi}{2}
\item Wie oft müssen Ihre maladen Sklaven die Rampe mit der Schubkarre zur Spitze der Pyramide hochfahren um die Pyramide mit Schutt zu füllen? \textbf{(1 Punkt)}
\item Berechnen Sie die Länge der Rampe zur Spitze der Pyramide mit einem Anstellwinkel von $14^\circ$! \textbf{(2 Punkte)}
\item Wie weit reicht Ihre Rampe vom Fuß der Pyramide in die mecklemburgische Landschaft?  \textbf{(2 Punkte)}
\end{enumerate}

Bei der Besichtigung der Pyramide teilt Ihnen der leicht übergewichtige Pharao (Nebenberuf \textit{Chefredaktuer}) mit, das die Pyramide zu steil sei und somit nicht in die mecklemburgische Landschaft passen würde. Sie müssen nochmal ran.

\begin{enumerate}
  \setcounter{enumi}{5}
\item Die Grundfläche der Pyramide ändert sich nicht. Berechnen Sie die Änderung der Höhe in \underline{Königsellen}, wenn sich der Anstellwinkel der Pyramide um $5^\circ$ ändert!  \textbf{(2 Punkte)}
\end{enumerate}



% https://de.wikipedia.org/wiki/Rechtwinkliges_Dreieck
% https://www.matheretter.de/wiki/pyramide
% https://de.wikipedia.org/wiki/Knickpyramide 
\clearpage
% -----------------------------------------------------------------------

\section{Aufgabe \hfill (12 Punkte)}

\textit{Geben Sie grundsätzlich Formeln und Rechenweg zur Lösung der Teilaufgaben mit an!} \\[1Ex]
 

 
%% --------------------------------------------------------------------
\begin{minipage}[t]{0.5\textwidth}
\includegraphics[width = 1.3cm]{/Users/kruppajo/work/GitHub/exam/avatare/Mark.png}\hspace{-4mm}\includegraphics[width = 1.3cm]{/Users/kruppajo/work/GitHub/exam/avatare/Nilufar.png}
\end{minipage}
\begin{minipage}[t]{0.5\textwidth}
\hfill
\href{https://youtu.be/3LAq3R0rS14}{\includegraphics[width = 2cm]{img/youtube}}
\end{minipage}
%% --------------------------------------------------------------------




\paragraph{Geocaching -- Von Satelliten und Plastikdosen}



Es ist Wochenende und das Wetter ist \textit{sweet}. Nilufar und Mark schwingen sich auf ihre Cachermobile um mit 18km/h, geleitet von modernster Satellietentechnologie und einem Supercompter aus dem Jahr 2000 in den Händen, Plastikdosen in der Natur und an sehenswerten Orten zu finden. Nilufar und Mark wollen diesmal endlich die aufwärts Schwierigkeitschallenge durchführen. Die Reihenfolge der Caches nach Schwierigkeitswertung gibt daher die von den beiden abzufahrenden Orte vor. Die Terrain- und Schwierigkeitswertungen laufen von 1 (leichteste Wertung) bis 5 (schwierigste Wertung) in 0.5 Schritten. Folgende Informationen zu den Orten und den entsprechenden Caches stehen Nilufar und Mark für die Planung der Route zu Verfügung\footnote{Die Quelle der Inspiration für die Aufgabe war folgende Tätigkeit: \href{https://www.geocaching.com/play}{Geocaching -- Mach mit bei der weltweit größten Schatzsuche.}}.

\begin{center}
  \begin{tabular}{ ccc }
    \toprule
    Ort & Cache & Wertung (S|T|G) \\
    \midrule
    A & GCPD9RC & 1.0 | 2.5 | Mikro \\
    B & GCZQJAD & 5.0 | 1.5 | Normal \\ 
    C & GCQLORC & 3.5 | 1.0 | Normal \\ 
    D & GCO9NWV & 4.5 | 4.0 | Mikro \\ 
    E & GCV5SDP & 1.5 | 5.0 | Mikro \\     
 \bottomrule
\end{tabular}
\end{center}

Im Weiteren sind den beiden folgende Informationen zu den Entfernungen der Orte zugänglich. Der Entfernungsvektor $\overrightarrow{AC}$ ist $5$km. Im Weiteren ist Ihnen der Entfernungsvektor $\overrightarrow{CB}$ mit $4.5$km bekannt. Der Entfernungsvektor $\overrightarrow{BE}$ ist das $2.1$-fache des Entfernungsvektor $\overrightarrow{CB}$. Wenn Sie von dem Ort A den Ort C anpeilen, so liegt der Ort B ungefähr $25^\circ$ südlich. Wenn Sie von dem Ort C den Ort B anpeilen, so liegt der Ort D ungefähr $35^\circ$ östlich. Vom Ort B betrachtet, bilden die Orte C und D einen rechten Winkel am Ort B. Der Ort B liegt auf gerader Linie zwischen den Orten C und E. Somit liegt der Ort E südlich von B. Die Strecke zwischen A und E ist nicht passierbar. Sie starten an dem Ort E Ihre Cachertour. \\

Leider sind die beiden sehr schlecht im Navigieren und Entfernungen ausrechnen. Die beiden brauchen Ihre Hilfe!
  
\begin{enumerate}
\item Lösen Sie diese Aufgabe mit Hilfe einer aussagekräftigen Skizze der Orte und Caches. Bezeichnen Sie die Strecken und die Winkel Ihrer Skizze entsprechend! \textbf{(2 Punkte)}
\item Welche Strecke in $km$ legen Sie bei der Bewältigung der aufwärts Schwierigkeitschallenge zurück? \textbf{(5    Punkte)}
\item Gehen Sie von einer zusätzlichen Suchzeit in Stunden für die Caches an den jeweiligen Orten zur reinen Reisezeit mit Ihrem Cachermobil aus. Die Suchzeit in Stunden für \textit{jeden einzelnen} Cache wird durch die Funktion  
  \begin{equation*}
    Suchzeit = 0.2 + 0.18 \cdot Schwierigkeit
  \end{equation*}  
  beschreiben.  Wie lange in Stunden benötigen Sie um die aufwärts Schwierigkeitschallenge zu erfüllen? \textbf{(3 Punkte)}
\item An der höchsten Schwierigkeit müssen Sie angeln. Ihre Angel ist ausgefahren 6m lang. Erreichen Sie einen Cache in der Höhe von 8.1m?  Berechnen Sie dazu Ihre maximale mögliche Angelhöhe! Welche Annahmen mussten Sie treffen um die Aufgabe zu lösen? \textbf{(2 Punkte)} 
\end{enumerate}

 
\clearpage
% -----------------------------------------------------------------------

\section{Aufgabe \hfill (10 Punkte)}

\textit{Geben Sie grundsätzlich Formeln und Rechenweg zur Lösung der Teilaufgaben mit an!} \\[1Ex]
 

 
%% --------------------------------------------------------------------
\begin{minipage}[t]{0.5\textwidth}
\includegraphics[width = 1.3cm]{/Users/kruppajo/work/GitHub/exam/avatare/Paula.png}
\end{minipage}
\begin{minipage}[t]{0.5\textwidth}
\hfill
\href{https://youtu.be/4-dSaPMhK9s}{\includegraphics[width = 2cm]{img/youtube}}
\end{minipage}
%% --------------------------------------------------------------------

%% --------------------------------------------------------------------
{\tiny\textbf{Stichworte:} Brot aus Luft $\bullet$ Walöl $\bullet$ Haber-Bosch-Verfahren $\bullet$ 1. Weltkrieg $\bullet$ 40\% N im menschlichen Körper $\bullet$ Positivist}
%% --------------------------------------------------------------------




\paragraph{Die atmende Wand und Brot aus Luft}



Als Kellerkind\footnote{\href{https://www.youtube.com/watch?v=54H0HAJexVI}{Tocotronic - Electric Guitar} als passende Untermalung für diese Aufgabe.} vom Dorf will Paula das Ausmaß der Radonbelastung in ihrem Kellerzimmer bestimmen und lüften daher nicht. Passt schon. Spart dann auch Energie und lüften wird sowieso überschätzt. Während einer Messperiode von 7:00 Uhr bis 19:00 bestimmt sie dreimal automatisch die Radonbelastung in ihrem Kellerraum in $Bq/m^3$. Es ergibt sich folgende Abbildung\footnote{Die Quelle der Inspiration für die Aufgabe war folgender Artikel: \href{https://de.wikipedia.org/wiki/Atmende_Wand}{Atmende Wand}}. Leider helfen die Messwerte Paula überhaupt nicht weiter. Sie müssen also helfen! 

\begin{knitrout}
\definecolor{shadecolor}{rgb}{0.969, 0.969, 0.969}\color{fgcolor}

{\centering \includegraphics[width=\maxwidth]{img/math-10-1} 

}


\end{knitrout}

\vspace{-0.75cm}

\begin{enumerate}
\item Wie lange dauert es in Stunden bis Sie eine kritische Belastung von 280$Bq/m^3$ in Ihrem ungelüfteten Kellerraum erreicht haben? \textbf{(2 Punkte)}
\end{enumerate}

Radon zerfällt mit einer Halbwertszeit von 3.7d zu Polonium. Polonium wiederum zerfällt mit einer Halbwertszeit von 160d zu Blei. Nur Radon und Polonium tragen zur radioaktiven Strahlenbelastung bei.

\begin{enumerate}
  \setcounter{enumi}{1}
\item Wie lange dauert es in Stunden bis Ihre kritische Radonbelastung von
  280$Bq/m^3$ auf unter 90$Bq/m^3$ gefallen ist?
  \textbf{(4 Punkte)}
\end{enumerate}

Folgende Tabelle enthält die Informationen zur Zusammensetzung der normalen Umgebungsluft.

\begin{center}
  \begin{tabular}{ c|c|c|c }
     & Vol-\% & M [g/mol] & ppm \\
    \hline
    Stickstoff & 79.7 & 28.4 &
                                                    \phantom{1000000000000}\strut\\
        \hline
    Sauerstoff & 20.45 & 16.2 &
                                                    \phantom{10000000}\strut\\
        \hline
    Kohlenstoffdioxid & 0.045 & 12.5 & \phantom{10000000}\strut\\     
     \hline
\end{tabular}
\end{center}

\begin{enumerate}
   \setcounter{enumi}{2}
\item Rechnen Sie die Volumenprozente (Vol-\%) der Umgebungsluft in die entsprechenden ppm-Werte um und ergänzen Sie die berechneten ppm-Werte in die Tabelle!  \textbf{(1 Punkt)}
\end{enumerate}

Während Paula ihr etwas pappiges Toastbrot mampfen kommt Paula die Dokumentation über Brot aus Luft in den Sinn. Paula denkt darüber ein wenig nach. Für die Umwandlung von Stickstoff $N_2$ mit Wasserstoff $H_2$ zu Ammoniak $NH_3$ gilt folgende Reaktionsgleichung\footnote{Die Quelle der Inspiration für die Aufgabe war folgender Artikel: \href{https://www.gdch.de/netzwerk-strukturen/fachstrukturen/ag-chemie-und-gesellschaft/projekte-und-veranstaltungen/cartoons/haber-bosch-verfahren-brot-aus-luft.html}{Haber-Bosch-Verfahren – Brot aus Luft}}:

\begin{equation*}
  N_2 + 3H_2 \rightarrow 2NH_3
\end{equation*}  

Ein Mol eines beliebigen Gases hat bei normalen Umweltbedingungen ein Volumen von 22.4 Liter. % Ein Mol enthält $6 \cdot 10^{23}$ Teilchen.

\begin{enumerate}
  \setcounter{enumi}{3}
\item Welche Masse an Ammoniak in Kilogramm $kg$ können Sie aus einem Kubikmeter $m^3$ Luft unter normalen Umweltbedingungen gewinnen?
  \textbf{(2 Punkte)}
\item Wieviel Ammoniak in $mol$ erhalten Sie aus einem Kubikmeter Luft? \textbf{(1 Punkt)}
\end{enumerate}

 
\clearpage
% -----------------------------------------------------------------------

\section{Aufgabe \hfill (10 Punkte)}

\textit{Geben Sie grundsätzlich Formeln und Rechenweg zur Lösung der Teilaufgaben mit an!} \\[1Ex]
 

 
%% --------------------------------------------------------------------
\begin{minipage}[t]{0.5\textwidth}
\includegraphics[width = 1.3cm]{/Users/kruppajo/work/GitHub/exam/avatare/Mark.png}
\end{minipage}
\begin{minipage}[t]{0.5\textwidth}
\hfill
\href{https://youtu.be/Bbu6n8MXxQk}{\includegraphics[width = 2cm]{img/youtube}}
\end{minipage}
%% --------------------------------------------------------------------




\paragraph{Armee der Finsternis}



Der Studentenjob von Mark war nach Ladenschluss bei IKEA die Regale einzuräumen. Dabei ist Mark in der Auslage der Sonderangebote das Necronomicon\footnote{Ein wirklich gefährliches Buch ist: \textit{Du bist genug: Vom Mut, glücklich zu sein} von Fumitake Koga und Ichiro Kishimi} in die Hände gefallen. Nun ist er eine Magierin der Zeichen geworden! Also eigentlich kann Mark nur Mathe und das dämliche Necronomicon hat ihn in die Vergangenheit geschleudert... aber gut, was tut man nicht alles im Jahr 685 n. Chr. für den neuen Lehnsherren Fürsten Arthur. Mark baut natürlich einen Schrottkugelturm um sich den
Horden der Finsternis mit genug Schrott erwehren zu können! Mark stehen zwei mächtige magische Formeln zur Unterstützung zu Verfügung. Leider wird das nicht reichen, deshalb müssen Sie hier auch noch durch Zeit und Raum helfen!

\begin{center}
  \begin{tabular}{cc}
    $E_{kin} = \cfrac{1}{2}\cdot m \cdot v^2$ & $E_{pot} = m \cdot g \cdot h$\\
  \end{tabular}
\end{center}

mit

\begin{itemize}[noitemsep]
\item $m$, gleich der Masse [kg] des Objekts
\item $h$, gleich der Höhe [m] des ruhenden Objekts
\item $v$, gleich der Geschwindigkeit [m/s] des Objekts
\item $g$, gleich der Erdbeschleunigung mit $9.81 \tfrac{m}{s^2}$ 
\end{itemize}

Als erstes müssen Sie die Höhe des zu bauenden Schrottkugelturmes bestimmen. Hierfür ist wichtig zu wissen, dass sich die Blei\textit{tropfen} mit einem Gewicht von $20mg$ zu gleichförmigen Blei\textit{kugeln} bei einer Geschwindigkeit von $11m/s$ bilden.

\begin{enumerate}
\item Wie hoch müssen Sie den Schrottkugelturm bauen lassen, damit sich runde Bleikugeln durch die Fallgeschwindigkeit von $11m/s$ bilden? \textbf{(3 Punkte)}
\end{enumerate}

Ihre erstellten Schrottkugeln sind leider zu gro{\ss} und somit sind zu wenige Schrottkugeln in einer Ladung. Damit können Sie die Armee der Finsternis nicht aufhalten. Die Sachlage müssen Sie einmal mathematisch untersuchen.

\begin{enumerate}
  \setcounter{enumi}{1}
\item Nennen Sie die beiden geometrischen Formen aus denen sich näherungsweise ein Tropfen zusammensetzt! Erstellen Sie eine beschriftete Skizze des Tropfens! \textbf{(2 Punkte)}
\item Sie messen eine Länge des Tropfens von 2.8mm. Die Löcher im Sieb erlauben ein Tropfendurchmesser von 1.8mm. Welchen Durchmesser in mm haben Ihre produzierten Blei\textit{kugeln}?  \textbf{(3 Punkte)}
\end{enumerate}

Sie haben jetzt die \ensuremath{2.3\times 10^{5}} Bleikugeln zusammen. Blei hat eine Dichte
von $12.63g/cm^3$.

\begin{enumerate}
  \setcounter{enumi}{3}
\item Wie schwer in Kilogramm $kg$ sind die \ensuremath{2.3\times 10^{5}} produzierten
  Bleikugeln, die Sie jetzt auf die Burgmauer transportieren müssen?
  \textbf{(1 Punkt)}
\end{enumerate}

Am Ende müssen Sie noch die Produktion von dem Bleischrott im Turm optimieren.

\begin{enumerate}
  \setcounter{enumi}{4}
\item Wie gro{\ss} in $cm^2$ ist Ihr quadratisches Sieb am oberen Ende des Turms,
  wenn Sie pro Fall ca. 1200 Bleikugeln produzieren wollen und
  die Bleikugel im Fall 0.8cm Abstand haben müssen?  \textbf{(1
    Punkt)}
\end{enumerate}
 
\clearpage
% -----------------------------------------------------------------------

\section{Aufgabe \hfill (10 Punkte)}

\textit{Geben Sie grundsätzlich Formeln und Rechenweg zur Lösung der Teilaufgaben mit an!} \\[1Ex]
 

 
%% --------------------------------------------------------------------
\begin{minipage}[t]{0.5\textwidth}
\includegraphics[width = 1.3cm]{/Users/kruppajo/work/GitHub/exam/avatare/Alex.png}
\end{minipage}
\begin{minipage}[t]{0.5\textwidth}
\hfill
\href{https://youtu.be/Mr6eslls4J0}{\includegraphics[width = 2cm]{img/youtube}}
\end{minipage}
%% --------------------------------------------------------------------




\paragraph{Armee der Kaninchen}



Leider hat es bei Alex mit der Faultierpension in Down Under nicht geklappt. War vielleicht auch nicht \textit{so} die beste Idee... aber dafür hat Alex eine neue Eingebung! Oder wie es Mike Tyson zugeschrieben wird: >>Ich wurde nie niedergeschlagen, ich war immer am Aufstehen!<<. Daher macht Alex jetzt einen Großhandel mit Kaninchenfleisch und damit dem teuersten Fleisch in Australien auf. Moment, hopsen hier nicht, seit Thomas Austin im Jahr 1859 ungefähr 30 Kaninchen entlassen hat, Millionen von Kaninchen rum? Wieso ist das Kaninchenfleisch dann so exklusiv? Alex wird stutzig und frag Sie, dem mal mathematisch nachzugehen!\footnote{Die Quelle der Inspiration für die Aufgabe war der folgendes YouTube Video: \href{https://youtu.be/38fuOr3tdgc?si=Li7NL_FoByML8JtT}{ Incredible Stories -- Why don't they eat wild rabbits in Australia? They have    millions of them! The reason is surprising...}} \\

Forscherinnen fand folgende Sättigungsfunktion für das jährliche Wachstum der gesamten Kaninchenpopulation im westlichen Australien.
\begin{equation*}
  f(t) = \ensuremath{1.2\times 10^{10}} - \ensuremath{9\times 10^{8}} \cdot 1.9^{-0.15 \cdot t + 4.1}
\end{equation*}

\begin{enumerate}
\item Skizzieren Sie die Sättigungsfunktion \textit{annäherungsweise} in einer Abbildung! \textbf{(1 Punkt)}
\item Wie viele Kaninchen können nach der Sättigungsfunktion maximal im westlichen Australien leben? Ergänzen Sie den Wert in Ihrer Abbildung! \textbf{(2 Punkte)}
\item Wie viele Millionen Kaninchen leben nach der Sättigungsfunktion nach 9 Jahren auf dem australischen Kontinent? \textbf{(1 Punkt)}
\end{enumerate}

Um den Kaninchen Einhalt zu gebieten wurde das Myxoma Virus und das Rabbit Haemorrhagic Disease Virus (RHDV) in 14 Kaninchen ausgebracht. Da die Kaninchen keine Maßnahmen gegen die Ausbreitung vornehmen können, verläuft die Ausbreitung mit einem wöchentlichen Wachstumsfakor von 1.8 nach folgender Formel.

\begin{equation*}
  N(t) = N(0) \cdot a^t
\end{equation*}

\begin{enumerate}
  \setcounter{enumi}{2}
\item Wie viele Wochen benötigen die Viren um theoretisch die gesamte Kaninchenpopulation nach 7 Jahren Wachstum zu durchseuchen? \textbf{(1 Punkt)}
\end{enumerate}

Das Myxoma Virus und das RHDV töten 99.7\% der Kaninchenpopulation innerhalb weniger Wochen.

\begin{enumerate}
  \setcounter{enumi}{3}  
\item Wie lange in Jahren dauert es bis eine Kaninchenpopulation nach einer Viruspandemie wieder auf 70\% der gesättigten Kaninchenpopulation angewachsen ist?  \textbf{(2 Punkte)}
\end{enumerate}

Thomas Austin entließ die Kaninchen im äußersten Süden von Australien. Australien hat eine West-Ost-Ausdehnung von 4400km und eine Nord-Süd-Ausdehnung von knapp 3700km. Die Kaninchen breiten sich radial mit einer Geschwindigkeit von 11.5km pro Jahr aus.

\begin{enumerate}
  \setcounter{enumi}{4}
\item Wie lange dauert es in Jahren bis die Kaninchen jeden Ort in Australien erreicht haben? \textit{Lösen Sie die Aufgabe unter der Verwendung einer schematischen Skizze!} \textbf{(2 Punkte)}
\end{enumerate}

Eine jährliche Impfung gegen das Myxoma Virus und das Rabbit Haemorrhagic Disease Virus (RHDV) kosten 13\$ pro Tier und der durchführende Arzt verlangt ca. 45\$ pro Tier.

\begin{enumerate}
  \setcounter{enumi}{5}
\item In Ihrem Stall leben 1100 Mastkaninchen. Mit welchen jährlichen Zusatzkosten für die Impfungen der Kaninchen müssen Sie daher kalkulieren? \textbf{(1 Punkt)}
\end{enumerate}
 
\clearpage
% -----------------------------------------------------------------------

\section{Aufgabe \hfill (12 Punkte)}

\textit{Geben Sie grundsätzlich Formeln und Rechenweg zur Lösung der Teilaufgaben mit an!} \\[1Ex]
 

 
%% --------------------------------------------------------------------
\begin{minipage}[t]{0.5\textwidth}
\includegraphics[width = 1.3cm]{/Users/kruppajo/work/GitHub/exam/avatare/Alex.png}
\end{minipage}
\begin{minipage}[t]{0.5\textwidth}
\hfill
\href{https://youtu.be/fiWGgCX-cE4}{\includegraphics[width = 2cm]{img/youtube}}
\end{minipage}
%% --------------------------------------------------------------------






\paragraph{Uckermark. Unendliche Weiten.}



Wir schreiben das Jahr 2024. Dies sind die Abenteuer  des Schafs Fridolin und Alex. Grünes Gras unter Alexs Füßen und ein strammer Wind im Gesicht, egal wohin er schaut. Ein schmatzendes Geräusch ertönt unter Alex. Alex sinniert, sollte er seine weiten Graslandschaften jetzt schon düngen? Dafür benötigt Alex die \textit{Grünlandtemperatur}! Die Grünlandtemperatur (GLT) ist die Summe aller positiven Tagesmitteltemperaturen seit Jahresbeginn. Ab einer GLT von 200$^\circ$ kann mit der Stickstoffdüngung begonnen werden. Alex sieht nicht ein, Geld für einen Agrarmetrologen zu bezahlen, wenn auch Sie mitrechnen können. Also rechnen Sie beide mit folgenden Informationen zu Monatsmultiplikatoren des GLT-Wertes: Januar mit $0.6\times$, Februar mit $0.8\times$ und März mit
$1.2\times$. Sie haben noch im letzten Jahr folgende Temperaturen gemessen.

\begin{center}
\begin{tabular}{cc}
  \toprule
  Datum & C$^\circ$ \\
  \midrule
  01. Jan 2023 & 0.3\\
  01. Feb 2023 & 1.1\\
  01. Mrz 2023 & 3.5\\
  01. Apr 2023 & 6.1\\
  \bottomrule
\end{tabular}
\end{center}

\begin{enumerate}
\item Erstellen Sie eine Skizze aus den Informationen aus der Temperaturtabelle!  \textbf{(1 Punkt)}
\item Stellen Sie die linearen Funktionen $f_1(t)$, $f_2(t)$ und $f_3(t)$ aus der obigen Temperaturtabelle auf!  \textbf{(1 Punkt)}
\item Bestimmen Sie die Stammfunktionen $F_1(t)$, $F_2(t)$ und $F_3(t)$ für Ihre linearen Funktionen aus der obigen Temperaturtabelle!  \textbf{(1 Punkt)}
\item Osterglocken beginnen ab einer GLT von 210$^\circ C$ zu blühen. An welchem Tag im 1. Quartal des Jahres 2023 war dies der Fall? \textit{Ignorieren Sie ein eventuelles Schaltjahr in Ihrer Berechnung.} \textbf{(4 Punkte)}
\end{enumerate}

Auf dem Weg zu Alexs Boskoopplantage werden Sie beide auf dem Trecker von einer Gruppe elektrifizierter Renter abgedrängt. Der Trecker muss wieder aus dem Graben! Fridolin und die elektrifizierten Rentner ziehen an zwei, separaten Seilen. Dabei zieht Fridolin mit $230N$. Die elektrifizierter Renter  bringen eine Kraft von $190N$ auf.\\

\textit{Lösen Sie diese Aufgabe mit Hilfe einer aussagekräftigen Skizze der Kraftvektoren. Bezeichnen Sie die Kraftvektoren und die Winkel Ihrer Skizze entsprechend!}

\begin{enumerate}
  \setcounter{enumi}{4}  
\item Im ersten Versuch legen Sie das Seil für Fridolin lotrecht über einen Ast oberhalb des Treckers. Die Rentner ziehen in einer geraden Linie über die Böschung hinweg am anderen Seil. Welche Kraft wird aufgebracht?  \textbf{(2 Punkte)}
\item Im zweiten Versuch ziehen Fridolin und die Rentner mit einem $40^\circ$ Winkel mit ihrem Seil an dem Trecker. Welche Kraft wird aufgebracht? \textbf{(2 Punkte)}
\item Mit welcher Beschleunigung ziehen Sie den $1.3t$ schweren Trecker \textit{jeweils} aus dem Graben, wenn $F = m \cdot a$ gilt? \textbf{(1 Punkt)}
\end{enumerate}

 
\clearpage
% -----------------------------------------------------------------------

\section{Aufgabe \hfill (10 Punkte)}

\textit{Geben Sie grundsätzlich Formeln und Rechenweg zur Lösung der Teilaufgaben mit an!} \\[1Ex]
 

 
%% --------------------------------------------------------------------
\begin{minipage}[t]{0.5\textwidth}
\includegraphics[width = 1.3cm]{/Users/kruppajo/work/GitHub/exam/avatare/Jonas.png}\hspace{-4mm}\includegraphics[width = 1.3cm]{/Users/kruppajo/work/GitHub/exam/avatare/Mark.png}\hspace{-4mm}\includegraphics[width = 1.3cm]{/Users/kruppajo/work/GitHub/exam/avatare/Nilufar.png}\hspace{-4mm}\includegraphics[width = 1.3cm]{/Users/kruppajo/work/GitHub/exam/avatare/Steffen.png}
\end{minipage}
\begin{minipage}[t]{0.5\textwidth}
\hfill
\href{https://youtu.be/RuzMjwvwT-4}{\includegraphics[width = 2cm]{img/youtube}}
\end{minipage}
%% --------------------------------------------------------------------




\paragraph{In der Kartonagenfabrik}



Mark, Jonas, Nilufar und Steffen sitzen im Bus. Wenn man sich zu spät anmeldet, dann ist die Exkursion nicht so toll. Mark hatte den Anderen in der Lerngruppe zu spät Bescheid gesagt. 'Was denn, bin ich eure Nanny oder was?!', entfährt es Mark nachdem die vorwurfsvollen Blicke schon eine Weile auf ihm lasten. Also geht es eben mit Rektor Skinner und Mrs. Krabappel in die Kartonagenfabrik. Wie schon im vorherigen Semester... In der Kartonagenfabrik angekommen erfahren die Vier, dass die Kartons zum Versand von Nägeln nicht hier zusammengebautwerden sondern das sich die Endfertigung in Flint, Michigan befindet. Unter anderem wird dort der berühmte \textit{Doppelt gewellte, 8-mal-gefaltete, 0.6mm, 60-cm-Karton} durch
Falzung hergestellt. Beim letzten Mal war Rektor Skinner die Stimmung zu schlecht und deshalb geht es erst nach Hause, wenn ein paar Aufgaben gelöst sind. Martin gefällt das. An dem Vorrat an Zigaretten von Mrs. Krabappel meinen alle wenig Zuversicht zu erkennen.\\

Jetzt heißt es Kartons optimieren, wenn Sie auch nochmal nach Hause wollen. Warum jetzt \textit{Sie} mit dabei sind, lassen wir mal weg. Der nun zu optimierende, flache Karton hat eine Länge von 60cm und eine Breite von 23cm. Die Kartonagenmaschine in Flint soll dann einen quadratischen Eckenausschnitt der Länge $x$ falzen.

\begin{enumerate}
\item Erstellen Sie eine Skizze des Karton\textit{blatt}rohlings! Beschriften Sie die Skizze mit den entsprechenden Längenangaben \textbf{(1 Punkt)}
\item Berechnen Sie die Falztiefe $x$ für ein maximales Volumen des flachen Kartons! \textbf{(3 Punkte)}
\item Welches Volumen in Liter ergibt sich mit der von Ihnen berechneten Falztiefe $x$?  \textbf{(1 Punkt)}
\item Sie wollen noch einen bündig mit dem Boden abschließenden Deckel für den Karton stanzen lassen. Wie groß ist die Fläche des Kartondeckel\textit{blatt}rohlings in $cm^2$? \textbf{(2 Punkte)}
\end{enumerate}

Rektor Skinner möchte sich gerne wieder in seinem Vorgarten aufhalten und nicht die ganze Zeit von Bart mit Erdnüssen beworfen werden. Deshalb möchte er einen geräumigen Teil seines Vorgartens einzäunen. Ein Teil der Umzäunung bildet seine Vorderhauswand. Wegen Lieferschwierigkeiten stehen Rektor Skinner nur 120m Zaun zu Verfügung. Auch hier sollen Sie mal helfen, sonst fährt der Bus Sie nicht nach Hause. Sie wollen nun die maximale Fläche des abgeschirmten Vorgartens in Abhängigkeit der Seitenlängen bei der Verwendung von 120m Zaun bestimmen!

\begin{enumerate}
  \setcounter{enumi}{4}  
\item  Welche Seitenlängen für den Zaun ergeben sich für die maximale Fläche des abgeschirmten Vorgartens? \textbf{(2 Punkte)}
\item Berechnen Sie die Fläche des abgeschirmten Vorgartens! \textbf{(1 Punkt)}
\end{enumerate}

 
\clearpage
% -----------------------------------------------------------------------

\section{Aufgabe \hfill (10 Punkte)}

\textit{Geben Sie grunds{\"a}tzlich Formeln und Rechenweg zur L{\"o}sung der
  Teilaufgaben mit an!} \\[1Ex]

%% --------------------------------------------------------------------
\hfill\href{https://youtu.be/OhyuH6hzEhY}{\includegraphics[width = 2cm]{img/youtube}} %%youtube
\hspace{2Ex}
%% --------------------------------------------------------------------




\paragraph{Ein Pfund Insekten, bitte!} Das wird wohl h{\"a}ufiger geh{\"o}rt
werden, wenn wir die Menschheit mit Proteinen ausreichend ern{\"a}hren
wollen\footnote{Die Quelle der Inspiration f{\"u}r die Aufgabe war der folgende
  Artikel aus dem Spiegel:
  \href{https://www.spiegel.de/ausland/ueberbevoelkerung-acht-milliarden-sind-wir-bald-zu-viele-menschen-auf-der-erde-a-3f20c7bc-3d60-4440-9f52-eb338db207f5}{Acht
    Milliarden - sind wir bald zu viele Menschen auf der Erde?}}. Schauen
wir uns dazu einmal den Vergleich Deutschland zu Nigeria an. Nigeria hat
eine der am schnellsten wachsenden Bev{\"o}lkerungen der Welt und wird
vermutlich im Jahr 2100 zu den Top 5 der bev{\"o}lkerungsreichsten L{\"a}nder
z{\"a}hlen. Im Jahr 2023 leben ca. \ensuremath{8.2\times 10^{7}} Menschen in
Deutschland und ca. \ensuremath{1.84\times 10^{8}} Menschen in Nigeria. Mit den
Informationen wollen wir anfangen und dann eine Prognose f{\"u}r den
Fleischkonsum im Jahr 2050 zu treffen. \\ 


Im folgenden ist Abbildung des Fleischkonsums im Jahr 2023 in
Deutschland und Nigeria in [kg] einmal dargestellt. 

\begin{knitrout}
\definecolor{shadecolor}{rgb}{0.969, 0.969, 0.969}\color{fgcolor}

{\centering \includegraphics[width=\maxwidth]{img/math-14-a-1} 

}


\end{knitrout}
  
\begin{enumerate}
\item Stellen Sie den Fleischkonsum in Deutschland und Nigeria im Jahr
  2023 \textit{pro Kopf} in einer aussagekr{\"a}ftigen Tabelle dar! 
  \textbf{(2 Punkte)}
\item Erg{\"a}nzen Sie in der Tabelle eine Spalte in der Sie f{\"u}r den
  Fleischkonsum in Nigeria auf Deutschland normieren, daher ins Verh{\"a}ltnis
  Nigeria/Deutschland, setzen! \textbf{(1 Punkt)}
\end{enumerate}

In der n{\"a}chsten Abbildung finden Sie die CO$_2$-Emission in [kg] nach
Lebensmittel, die durch die Produktion entsteht, abgebildet.

\begin{knitrout}
\definecolor{shadecolor}{rgb}{0.969, 0.969, 0.969}\color{fgcolor}

{\centering \includegraphics[width=\maxwidth]{img/math-14-b-1} 

}


\end{knitrout}

\begin{enumerate}
  \setcounter{enumi}{2}  
\item Stellen Sie in einer Tabelle die Treibhausgasemissionen an $CO_2$ pro
  Kopf, die durch den Fleischkonsum in Deutschland und Nigeria im Jahr
  2023 entstehen, dar! Erg{\"a}nzen Sie auch hier das Verh{\"a}ltnis
  Nigeria zu Deutschland! \textbf{(2 Punkte)}
\end{enumerate}

\newpage

In der folgenden Abbildung sehen Sie die
Bev{\"o}lkerungsentwicklung [Millionen] in Nigeria von 1950 bis ins Jahr 2030 fortgef{\"u}hrt.

\begin{knitrout}
\definecolor{shadecolor}{rgb}{0.969, 0.969, 0.969}\color{fgcolor}

{\centering \includegraphics[width=\maxwidth]{img/math-14-c-1} 

}


\end{knitrout}

\begin{enumerate}
  \setcounter{enumi}{3}  
\item Sch{\"a}tzen Sie graphisch die zu erwartende Bev{\"o}lkerung [Millionen] in Nigeria im
  Jahr 2050, die sich anhand der Informationen aus der Abbildung ergibt!
\begin{enumerate}
\item Ohne Ber{\"u}cksichtigung der Covid-19-Pandemie! \textbf{(1
    Punkt)}
\item Unter Ber{\"u}cksichtigung der Covid-19-Pandemie! \textbf{(1
    Punkt)}
\end{enumerate}
\item Berechnen Sie den gesch{\"a}tzten Fleischkonsum von Nigeria im Jahr
  2050 unter der Annahme 80\%-iger Angleichung der Lebensbedingungen zu
  Deutschland im Jahr 2023! \textbf{(1 Punkt)}
\item Berechnen Sie die prozentuale Steigerung der Treibhausgasemissionen
  an $CO_2$ in Nigeria im Jahr 2050 im Vergleich zum Jahr 2023, der
  sich durch den angeglichenen Fleischkonsum ergibt! \textbf{(1 Punkt)}
\item Berechnen Sie die prozentuale Steigerung der Treibhausgasemissionen
  an $CO_2$ in Nigeria, wenn die gesamte Proteinaufnahme 
  durch Insekten ersetzt w{\"u}rde! \textbf{(1
    Punkt)}
\end{enumerate}







 
\clearpage
% -----------------------------------------------------------------------

\section{Aufgabe \hfill (12 Punkte)}

\textit{Geben Sie grunds{\"a}tzlich Formeln und Rechenweg zur L{\"o}sung der
  Teilaufgaben mit an!} \\[1Ex]

%% --------------------------------------------------------------------
\hfill\href{https://youtu.be/flRBo1FWQy0}{\includegraphics[width = 2cm]{img/youtube}} %%youtube
\hspace{2Ex}
%% --------------------------------------------------------------------

\paragraph{T{\"o}dliche Seuche AIDS -- Die r{\"a}tselhafte Krankheit}




Irritiert legen Sie die historische Ausgabe des Spiegels aus den 80zigern
beiseite. Sie sind bei Ihrem HNO-Arzt und wollen einen AIDS-Test machen
lassen. Woanders leider keinen Termin gekriegt... Immerhin denken Sie und
Ihr Partner {\"u}ber Nachwuchs nach und da geht es eben nur durch
ungesch{\"u}tzten Sex. Was wissen Sie nun aber {\"u}ber AIDS und dem diagnostischen
AIDS-Test, den Sie nun machen werden?\\

Die Pr{\"a}valenz von AIDS bei einem Menschen in Europa wird mit
0.1\% angenommen. In 90\% der F{\"a}lle ist ein
HIV-Test positiv, wenn der Patient erkrankt ist. In 3\%
der F{\"a}lle ist ein HIV-Test positiv, wenn der Patient \textit{nicht}
erkrankt ist und somit gesund ist. Sie stutzen. Wie wahrscheinlich ist es
denn eigentlich an AIDS erkrankt zu sein $(K^+)$, wenn Sie einen positiven
AIDS-Test vorliegen haben $(T^+)$? Gehen Sie f{\"u}r die folgenden Berechnungen
von $n = \ensuremath{10^{4}}$ Patienten mit einem diagnostischen Test f{\"u}r AIDS
aus. Sie nehmen sich also einen Kuli und fangen an auf der historischen
Ausgabe des Spiegels zu rechnen\footnote{Die Quelle der Inspiration f{\"u}r die
  Aufgabe war der folgende wissenschaftlicher Artikel: \textit{Binder et
    al. (2022) Von Baumdiagrammen {\"u}ber Doppelb{\"a}ume zu H{\"a}ufigkeitsnetzen --
    kognitive {\"U}berlastung oder didaktische Unterst{\"u}tzung? Journal f{\"u}r
    Mathematik-Didaktik, 1-33}}.

\begin{enumerate}
\item Welche Wahrscheinlichkeit $Pr$ wollen Sie berechnen? \textbf{(1 Punkt)}
\item Zeichnen Sie einen H{\"a}ufigkeitsdoppelbaum zur Bestimmung der gesuchten
  Wahrscheinlichkeit $Pr$! \textbf{(2 Punkte)} 
\item Beschriften Sie den H{\"a}ufigkeitsdoppelbaum, mit denen Ihnen bekannten
  Informationen zu der AIDS Erkrankung und dem AIDS-Test! \textbf{(1 Punkt)}
\item F{\"u}llen Sie den H{\"a}ufigkeitsdoppelbaum mit den sich ergebenden,
  absoluten Patientenzahlen $n$ aus! \textbf{(2 Punkte)}
\item Berechnen Sie die gesuchte Wahrscheinlichkeit $Pr$! \textbf{(1 Punkt)}
\end{enumerate}

Bei dem folgenden Arztgespr{\"a}ch erfahren Sie, dass beim diagnostischen
Testen \textit{True Positives (TP)}, \textit{True Negatives (TN)},
\textit{False Positives (FP)} und \textit{False Negatives (FN)}
auftreten. Das verstehen Sie so noch nicht und deshalb stellen Sie f{\"u}r sich
den Zusammenhang in einer 2x2 Kreuztabelle dar.

\begin{enumerate}
  \setcounter{enumi}{5}
\item Tragen Sie \textit{TP}, \textit{TN}, \textit{FP} und \textit{FN} in
  eine 2x2 Kreuztablle ein. Beschriften Sie die Tabelle entsprechend!
  \textbf{(1 Punkt)}
\item Berechnen Sie die Sensitivit{\"a}t und Spezifit{\"a}t des diagnostischen Tests
  f{\"u}r AIDS! F{\"u}llen Sie daf{\"u}r die 2x2 Kreuztabelle mit den Informationen aus
  dem H{\"a}ufigkeitsdoppelbaum aus! \textbf{(2 Punkte)}
\item Was beschreibt die Sensitivit{\"a}t und die Spezifit{\"a}t im Bezug auf die
  Gesunden und Kranken? Stellen Sie beide diagnostische Ma{\ss}zahlen als
  Wahrscheinlichkeiten $Pr$ dar! \textbf{(2 Punkte)} 
\end{enumerate}



 
\clearpage
% -----------------------------------------------------------------------

\section{Aufgabe \hfill (12 Punkte)}

\textit{Geben Sie grunds{\"a}tzlich Formeln und Rechenweg zur L{\"o}sung der
  Teilaufgaben mit an!} \\[1Ex]

%% --------------------------------------------------------------------
\hfill\href{https://youtu.be/SZqp_vy3rgI}{\includegraphics[width = 2cm]{img/youtube}} %%youtube
\hspace{2Ex}
%% --------------------------------------------------------------------

\paragraph{Network-Marketing oder Schneeballschlacht!}




Eine M{\"o}glichkeit, leicht Geld zu verdienen, ist es anderen Menschen f{\"u}r Geld
zu versprechen, wie man leicht reich werden kann. Am besten nat{\"u}rlich ohne
viel Aufwand und ortsunabh{\"a}ngig. Dann wollen wir mal loslegen. Schnell
ein YouTube-Werbevideo gedreht und auf geht es mit unserem
Network-Marketing. Aber Moment, wie funktioniert Network-Marketing
eigentlich und was hat das alles mit einer Schneeballschlacht zu tun? Wir
wollen hier einmal in die Untiefen des "`passiven Einkommens"'
abtauchen\footnote{Die Quellen der Inspiration f{\"u}r die Aufgabe waren
  folgendes YouTube Video:
  \href{https://youtu.be/UOKkZF_qK9M?si=uf4foJVFKfeQMwSw}{Simplicissimus --
    Die meistgesuchte Betr{\"u}gerin der Welt} und der Artikel:
  \href{https://www.deutschlandfunkkultur.de/netzwerk-marketing-die-illusion-schnell-reich-zu-werden-100.html}{
    Deutschlandfunk Kultur -- Die Illusion, schnell reich zu werden}}.\\

Das Jahr 2022 war das erfolgreichste Jahr in der Geschichte von
Direct Finanzanlagen Left/Right (D-FL/R). Das Unternehmen steigerte den Umsatz um rund
24 Prozent von 275 Millionen Euro im Jahr
2021. Doch wie viel kommt bei den Partnern an? Laut
D-FL/R habe das Unternehmen \ensuremath{3.8\times 10^{5}} aktive Partner.

\begin{enumerate}
\item Berechnen Sie zuerst den Umsatz der Firma D-FL/R im
  Jahr 2022! \textbf{(1 Punkt)}
\item Wie viel von dem Umsatz im Jahr 2022 wird im Durchschnitt von jedem
  aktiven Partner erwirtschaftet? \textbf{(1 Punkt)}
\item Welche \textit{monatlicher} Umsatz ergibt sich dadurch im
  Durchschnitt f{\"u}r jeden aktiven Partner bei einer direkten Provision von
  20\%? \textbf{(1 Punkt)}
\end{enumerate}

Ihr zu vermarkendes Produkt, hinter dem Sie voll stehen, kostet
100EUR pro Einheit im Direktverkauf. Die direkte Provision
f{\"u}r die erste Stufe betr{\"a}gt 35\%. F{\"u}r die zweite, dritte und
vierte Stufe betragen die indirekten Provisionen jeweils 3.25\%,
2.25\% und 1.75\%. Jeder Ihrer angeworbenen "`Partner"'
wirbt wiederum f{"u}nf Partner f{\"u}r sich selbst an. Pro Monat
werden im Schnitt vier Einheiten vom Produkt verkauft. Sie wollen nun
1800EUR im Monat \textit{passiv} -- also durch indirekte
Provisionen -- erwirtschaften.

\begin{enumerate}
  \setcounter{enumi}{3}
\item Erg{\"a}nzen Sie die folgende Tabelle mit den obigen Informationen! \textbf{(2 Punkte)}
\end{enumerate}

\begin{center}
\begin{tabular}{c|c|c|c}
  \toprule
  \textbf{Stufe} & \textbf{Anzahl Partner}  & \textbf{Umsatz/Stufe} & \textbf{Provision}\\
  \midrule
  1 & Sie selber  &  & \\ \midrule
  2 &   &  &  \\ \midrule
  3 &   &  &  \\ \midrule
  4 &   &  &  \\
  \bottomrule
\end{tabular}
\end{center}

\begin{enumerate}
  \setcounter{enumi}{4}
\item Wie viele Partner m{\"u}ssen Sie auf der 2 Stufe anwerben um Ihr passives
  Einkommen durch indirekte Provision zu erreichen? Wie viele Menschen
  arbeiten am Ende indirekt f{\"u}r Sie? Stellen Sie den
  Zusammenhang graphisch dar!  \textbf{(3 Punkte)}
\end{enumerate}

Sie mussten zum Einstieg bei D-FL/R Einheiten des Produkts
f{\"u}r 3500EUR kaufen. Diese Einheiten k{\"o}nnen Sie nur direkt
verkaufen. Leider mussten Sie den Kauf {\"u}ber einen Kredit {\"u}ber
4.7\% p.a. {\"u}ber 36 Monate finanzieren.

\begin{enumerate}
  \setcounter{enumi}{5}
\item Berechnen Sie die Gesamtsumme, die Sie als Kredit abbezahlen m{\"u}ssen! \textbf{(2 Punkte)}
\item Wie viele Einheiten m{\"u}ssen Sie pro Monat verkaufen um die anfallenden Zinsen
  durch die direkte Provision zu erwirtschaften? \textbf{(1 Punkt)}
\item Wie lange in
  Monaten ben{\"o}tigen Sie um den Kredit durch die direkte
  Provision abzubezahlen? \textbf{(1 Punkt)}
\end{enumerate} 
\clearpage
% -----------------------------------------------------------------------

\section{Aufgabe \hfill (10 Punkte)}

\textit{Geben Sie grunds{\"a}tzlich Formeln und Rechenweg zur L{\"o}sung der
  Teilaufgaben mit an!} \\[1Ex]

%% --------------------------------------------------------------------
\hfill\href{https://youtu.be/8Pb2sKUIMyk}{\includegraphics[width =
  2cm]{img/youtube}} %%youtube
\hspace{2Ex}
%% --------------------------------------------------------------------

\paragraph{H{\"o}hlen \& Drachen}



Nachdem Sie sich begeistert in der Serie \textit{Stranger Thinks} verloren
haben, wollen Sie bei einem Ihrer Freunde einmal \textit{H{\"o}hlen \& Drachen}
ausprobieren. Um Geld zu sparen, das Zeug kostet echt, wurde etwas an den
Regeln gebastelt. Schnell stellen Sie fest, dass hier ganz sch{\"o}n viele
unterschiedliche W{\"u}rfel durch die Gegend fliegen. Daher m{\"u}ssen Sie sich
jetzt einiges an Fragen stellen. \\%[-1ex]

In dem Spiel haben Sie nun auf einmal 4 achtseitige W{"u}rfel (4d8) zum w{\"u}rfeln in der Hand. Wenn Sie eine 8 w{\"u}rfeln,
haben Sie einen Erfolg.

\begin{enumerate}
\item Berechnen Sie die Wahrscheinlichkeit \textit{genau}
  2 Erfolge zu erzielen!  \textbf{(2 Punkte)}
\item Berechnen Sie die Wahrscheinlichkeit keinen Erfolg zu erzielen!
  \textbf{(1 Punkt)}
\end{enumerate}

Sie betrachten nun aufmerksam die ausufernden Ausr{\"u}stungstabellen. Ihnen
wird aber geholfen und Sie m{\"u}ssen sich jetzt nur zwischen der Axt oder dem
Schwert entscheiden.

\begin{enumerate}
  \setcounter{enumi}{2}
\item W{\"u}rden Sie die Axt mit zwei achtseitigen W{"u}rfeln (2d8) als Schaden oder
  das Schwert mit einem vierseitigen W{"u}rfel plus 5 (1d4+5) als Schaden bevorzugen?
  Begr{\"u}nden Sie Ihre Antwort mathematisch! \textbf{(1 Punkt)}
\end{enumerate}

Jetzt wird es immer wilder, da Sie sich jetzt {\"u}berlegen m{\"u}ssen, wie
wahrscheinlich es ist, dass Ihr Rettungswurf gegen den zaubernden Hexer
funktioniert. Sie haben folgende Wahrscheinlichkeiten gegeben. Die
Wahrscheinlichkeit f{\"u}r das Ereignis $A$, der Rettungswurf ist erfolgreich,
ist $Pr(A) = 0.65$, die Wahrscheinlichkeit f{\"u}r das Ereignis $B$,
der Zauberwurf des Hexers ist erfolgreich, ist $Pr(B) = 0.75$. Sie
haben mitgez{\"a}hlt und festgestellt, dass in $40$ von 100 F{\"a}llen
Ihr Rettungswurf bei einem erfolgeichen Zauber funktioniert hat.  

\begin{enumerate}
  \setcounter{enumi}{3}
\item Erstellen Sie eine 2x2 Kreuztabelle mit den Ereignissen $A$ und $B$
  sowie den Gegenereignissen $\bar{A}$ und $\bar{B}$ mit einen
  $\Omega = 100$. Beachten Sie hierbei die entsprechenden
  Wahrscheinlichkeiten f{\"u}r die Ereignisse $A$ und $B$! \textbf{(2 Punkte)}
\item Bestimmen Sie $Pr(A \cap B)$! \textbf{(1 Punkt)}
\item Erstellen Sie ein Baumdiagramm mit den passenden Informationen aus der 2x2
  Kreuztabelle! \textbf{(2 Punkte)}
\item Bestimmen Sie Wahrscheinlichkeit $Pr(A|B)$, dass Ihr Rettungswurf gelingt, wenn
  der Hexer erfolgreich gezaubert hat! \textbf{(1 Punkt)}
\end{enumerate}

  
\clearpage
% -----------------------------------------------------------------------

\section{Aufgabe \hfill (12 Punkte)}

\textit{Geben Sie grunds{\"a}tzlich Formeln und Rechenweg zur L{\"o}sung der
  Teilaufgaben mit an!} \\[1Ex]

%% --------------------------------------------------------------------
\hfill\href{https://youtu.be/9DQKaXdxT_g}{\includegraphics[width =
  2cm]{img/youtube}} %%youtube
\hspace{2Ex}
%% --------------------------------------------------------------------

\paragraph{Retrocheck im TV}



"`Und hier ist sie wieder, die Show der fantastischen Preise. Seien Sie mit
dabei, wenn es wieder hei{\ss}t: Der Preis ist hei{\ss}!"', ert{\"o}nt es und Sie
fragen sich, ob Sie nicht doch lieber bezahlter Gast bei Barbara Salesch
h{\"a}tten sein sollten. Aber Sie brauchen das Geld und jetzt hei{\ss}t es
Spielschows farmen! Erstmal eine Kaffemaschine von Mitropa
gewinnen. Ein Kandidat gewinnt die Kaffeemaschine von Mitropa, wenn nicht alle
Kandidaten {\"u}berbieten (eng. \textit{outbid}). Mit Ihnen bilden
Annegret und G{"u}nther das Team der drei Kandidaten.

\begin{center}
\begin{tabular}{ccc}
  \toprule
  Name & $P(win)$ & $P(outbid)$\\
  \midrule
  Annegret & 0.2 & 0.076\\
  G{"u}nther & 0.1 & 0.12 \\
  \bottomrule
\end{tabular}
\end{center}

\begin{enumerate}
\item Mit welcher Wahrscheinlichkeit gewinnen Sie die Kaffeemaschine von
  Mitropa, wenn keiner der Kandidaten {\"u}berbietet? \textbf{(1 Punkt)}
\item Wenn Ihre {\"U}berbietungswahrscheinlichkeit $P(outbid)$ bei
  0.02 liegt, mit welcher Wahrscheinlichkeit gewinnt
  \textit{keiner} die Kaffeemaschine von Mitropa? \textbf{(1 Punkt)}
\end{enumerate}

Gl{\"u}cksrad f{\"u}r Arme auf der Kirmes! Leider hat es f{\"u}r Maren Gilzer nicht
gereicht. Deshalb sind Sie jetzt auf der Kirmes und spielen mit
Catwoman um das gro{\ss}e Geld. Das Gl{\"u}cksrad hat 22
Felder. Sie drehen das Gl{\"u}cksrad zweimal. Auf 8 Feldern
gewinnen Sie 4000EUR sonst 1000EUR. Ganz sch{\"o}n viel Geld
und ganz sch{\"o}n zwielichtig hier...

\begin{enumerate}
  \setcounter{enumi}{2}  
\item Skizzieren Sie das Gl{\"u}cksrad und erg{\"a}nzen Sie die
  Wahrscheinlichkeiten! \textbf{(1 Punkt)}
\item Zeichnen Sie das zugeh{\"o}rige Baumdiagramm f{\"u}r das zweimalige Drehen!
  Erg{\"a}nzen Sie die Wahrscheinlichkeiten und die entsprechenden Ereignisse
  \textbf{(2 Punkte)}
\item Mir welcher Wahrscheinlichkeit gewinnen Sie 5000EUR? \textbf{(1
    Punkt)}
\end{enumerate}

Nach Ihrem Fiebertraum reisen Sie im Zug nach K{\"o}ln um bei "`Geh aufs
Ganze!"' mitzuspielen. Sie schaffen es tats{\"a}chlich ins Finale und k{\"o}nnen
als Hauptgewinn ein Auto hinter einer der drei T{\"u}ren gewinnen. 

\begin{enumerate}
  \setcounter{enumi}{5}  
\item Bevor die Show beginnt, wird das Auto hinter eine zuf{\"a}llig bestimmte
  T{\"u}r gestellt. Mit welcher Wahrscheinlichkeit wird jeweils eine der drei
  T{\"u}ren ausgew{\"a}hlt? Zeichnen Sie ein Baumdiagramm! \textbf{(1 Punkt)}
\item Mit welcher Wahrscheinlichkeit w{\"a}hlen Sie sofort die T{\"u}r mit
  dem Auto? Erweitere Sie das Baumdiagramm entsprechend! \textbf{(1
    Punkt)}
\item Der Moderator {\"o}ffnet nun eine der nicht gew{\"a}hlten T{\"u}ren, aber
  nat{\"u}rlich nicht die mit dem Auto. Mit welcher Wahrscheinlichkeit steht
  das Auto hinter der anderen T{\"u}r? Erweitern Sie das Baumdiagramm
  entsprechend! \textbf{(2 Punkte)}
\item L{\"o}sen Sie nun das "`Ziegenproblem"'! Berechne Sie dazu die
  Wahrscheinlichkeiten der einzelnen Pfade. Lohnt sich ein Wechsel der
  anfangs gew{\"a}hlte T{\"u}r? Begr{\"u}nden Sie Ihre Antwort mathematisch!
  \textbf{(2 Punkte)}
\end{enumerate}
 

%https://unterrichten.zum.de/wiki/Laplace-Wahrscheinlichkeit_wiederholen_und_vertiefen/Ziegen

%Buchstabenh{\"a}ufigkeit im Deutschen

%%https://www.mathelike.de/abiturloesungen-mathematik-bayern/mathematik-abitur-bayern-2020/pruefungsteil-a/stochastik-2/teilaufgabe-a.html

%%https://de.wikipedia.org/wiki/Buchstabenh%C3%A4ufigkeit


 
\clearpage
% -----------------------------------------------------------------------
\part{Angewandte Nutztier- und Pflanzenwissenschaften (M.Sc.)}
% -----------------------------------------------------------------------  

\section{Aufgabe \hfill (6 Punkte)}
Vergleichen Sie die Standardabweichung mit dem Standardfehler und grenzen
Sie die beiden Kennzahlen voneinander ab.



\section{Aufgabe \hfill (8 Punkte)}
Ihnen liegt folgendes Varianzanalysemodell mit der {\"u}blichen Beschreibung
zur Auswertung des Merkmals fett- und eiwei{\ss}korrigierte Milchleistung pro
Kuh und Jahr in kg vor:

\begin{equation*}
  Y_{ijkl} = \mu + Var_i + EKA_j + VarEKA_{ij} + V_k + b(L_{ij} - L) + e_{ijkl}
\end{equation*}

mit

\begin{itemize}
\item $Y_{ijkl}$: l-te Beobachtung
\item $\mu$: Populationsmittel 
\item $Var_i$: fixer Effekt der i-ten Variante (i: Kontrolle,
  Versuchsgruppe 1, Versuchsgruppe 2)
\item $EKA_j$: fixer Effekt der j-ten Erstkalbealtergruppe (j: EKA $\leq$ 25
  Monate, EKA > 25 Monate)
\item $VarEKA_{ij}$: fixer Effekt der Interaktion Variante x
  Erstkalbealtergruppe 
\item $V_k$: zuf{\"a}lliger Effekt des Vaters 
\item $b(L_{ij} - L)$: lineare Kovariable Laktationsnummer 
\item $e_{ijkl}$: zuf{\"a}lliger
  Restfehler
\end{itemize}
Erl{\"a}utern Sie anhand dieses Beispiels die Begriffe fixer Effekt,
Interaktion, zuf{\"a}lliger Effekt und Kovariable und grenzen Sie diese
Begriffe voneinander ab.



\section{Aufgabe \hfill (6 Punkte)}
Wie bestimmen Sie die richtige Stichprobengr{\"o}{\ss}e? Welche Kennzahlen /
statistische Ma{\ss}zahlen ben{\"o}tigen Sie dabei und nennen Sie die
Voraussetzungen.

 
\end{document}
% -----------------------------------------------------------------------


  
