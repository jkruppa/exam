\documentclass[a4paper, 9pt]{scrartcl}\usepackage[]{graphicx}\usepackage[]{xcolor}
% maxwidth is the original width if it is less than linewidth
% otherwise use linewidth (to make sure the graphics do not exceed the margin)
\makeatletter
\def\maxwidth{ %
  \ifdim\Gin@nat@width>\linewidth
    \linewidth
  \else
    \Gin@nat@width
  \fi
}
\makeatother

\definecolor{fgcolor}{rgb}{0.345, 0.345, 0.345}
\newcommand{\hlnum}[1]{\textcolor[rgb]{0.686,0.059,0.569}{#1}}%
\newcommand{\hlstr}[1]{\textcolor[rgb]{0.192,0.494,0.8}{#1}}%
\newcommand{\hlcom}[1]{\textcolor[rgb]{0.678,0.584,0.686}{\textit{#1}}}%
\newcommand{\hlopt}[1]{\textcolor[rgb]{0,0,0}{#1}}%
\newcommand{\hlstd}[1]{\textcolor[rgb]{0.345,0.345,0.345}{#1}}%
\newcommand{\hlkwa}[1]{\textcolor[rgb]{0.161,0.373,0.58}{\textbf{#1}}}%
\newcommand{\hlkwb}[1]{\textcolor[rgb]{0.69,0.353,0.396}{#1}}%
\newcommand{\hlkwc}[1]{\textcolor[rgb]{0.333,0.667,0.333}{#1}}%
\newcommand{\hlkwd}[1]{\textcolor[rgb]{0.737,0.353,0.396}{\textbf{#1}}}%
\let\hlipl\hlkwb

\usepackage{framed}
\makeatletter
\newenvironment{kframe}{%
 \def\at@end@of@kframe{}%
 \ifinner\ifhmode%
  \def\at@end@of@kframe{\end{minipage}}%
  \begin{minipage}{\columnwidth}%
 \fi\fi%
 \def\FrameCommand##1{\hskip\@totalleftmargin \hskip-\fboxsep
 \colorbox{shadecolor}{##1}\hskip-\fboxsep
     % There is no \\@totalrightmargin, so:
     \hskip-\linewidth \hskip-\@totalleftmargin \hskip\columnwidth}%
 \MakeFramed {\advance\hsize-\width
   \@totalleftmargin\z@ \linewidth\hsize
   \@setminipage}}%
 {\par\unskip\endMakeFramed%
 \at@end@of@kframe}
\makeatother

\definecolor{shadecolor}{rgb}{.97, .97, .97}
\definecolor{messagecolor}{rgb}{0, 0, 0}
\definecolor{warningcolor}{rgb}{1, 0, 1}
\definecolor{errorcolor}{rgb}{1, 0, 0}
\newenvironment{knitrout}{}{} % an empty environment to be redefined in TeX

\usepackage{alltt}
\usepackage[ngerman]{babel}

% -----------------------------------------------------------------------

% -----------------------------------------------------------------------
%% ------------------------------------------------------------
%% by J.Kruppa on Friday, February 11, 2022 (11:31)
%% \def\mainDir{\Sexpr{exam_path}}
\def\source{/Users/jokruppa/source/tex}
\usepackage[margin=2cm, includefoot]{geometry}
\setlength{\parindent}{0cm}
\usepackage{booktabs}
\usepackage{amsmath}
\usepackage{scalerel,amssymb}
\usepackage{setspace}
\def\csquare{{\Large $\boxtimes$}}
\def\msquare{{\Large $\square$}}
\usepackage[normalem]{ulem}
\usepackage{array}
\usepackage{xcolor}
\usepackage{float}
\usepackage{currfile}
\usepackage{tikz}
\usepackage[nomessages]{fp}

%% beamer defs
\def\lecture{Klausurfragen der Bio Data Science}

%% exam defs
\def\examtitle{\lecture}
\def\exammodule{
\vspace{-1.75cm}  
\begin{graybox}{}
\vspace{2Ex}
\textbf{\large Name:} \rule[0ex]{16.75em}{.4pt}
\hfill \textnormal{\textit{Nicht bestanden:}} \msquare \\[2.5Ex]
\textbf{\large Vorname:} \rule[0ex]{15em}{.4pt} \\[2.5Ex]
\textbf{\large Matrikelnummer:} \rule[0ex]{10.8em}{.4pt}
\hfill Endnote: \rule[0ex]{7em}{.4pt} 
\end{graybox}
\vspace{3Ex}
\phantom{text}
}
\def\examsemester{Sommersemester \& Wintersemester}
\def\examdate{\today}
%% ------------------------------------------------------------
\definecolor{darkblue}{rgb}{0,0,.5}
\definecolor{darkpurple}{rgb}{0.4117, 0.2, 0.4117}
\definecolor{uni}{rgb}{0,0.3137,0.6078}
\definecolor{gray}{gray}{0.7}

\usepackage{tcolorbox}
\definecolor{logo1}{RGB}{0, 158, 227}
\definecolor{gray5}{RGB}{247, 247, 247}
\definecolor{gray2}{RGB}{102, 102, 102}

\newtcolorbox{graybox}[1]{
  colback=gray5,%%red!5!white,
  colframe=gray2,%%red!75!black,
  fonttitle=\bfseries\Large,
  %%valign=center,
  fontupper=\large,
  before skip=10pt plus 2pt,
  after skip=20pt plus 4pt,
  title=#1}

\newtcolorbox{takehomebox}[1]{
  colback=gray5,%%red!5!white,
  colframe=logo1,%%red!75!black,
  fonttitle=\bfseries\Large,
  %%valign=center,
  fontupper=\large,
  before skip=10pt plus 2pt,
  after skip=10pt plus 2pt,
  title=#1}

\def\Rlogo{\includegraphics[width = 0.5cm]{\string~/Documents/GitHub/exam/img/Rlogo}\;}

\usepackage[scaled=.90]{helvet} 
\usepackage{fancyhdr}
\usepackage{lastpage}
\usepackage{hyperref}
\hypersetup{
    colorlinks=true,       % false: boxed links; true: colored links
    linkcolor=black,          % color of internal links 
    urlcolor=magenta           % color of external links
}
\renewcommand{\familydefault}{\sfdefault}

\title{
\large \exammodule \\[5Ex]
\Huge \examtitle \\[2Ex] 
\Large Hochschule Osnabr{\"u}ck
}
\author{Pr{\"u}fer: Prof. Dr. Jochen Kruppa \\
Fakult{\"a}t f{\"u}r Agrarwissenschaften und Landschaftsarchitektur \\ 
j.kruppa@hs-osnabrueck.de}
\date{Version vom \examdate}

%% ------------------------------------------------------------
%% by J.Kruppa on Tuesday, September 23, 2014 (12:50)
%% Header
\renewcommand{\headrulewidth}{0pt}
\renewcommand{\footrulewidth}{0pt}
\pagestyle{fancy}

\fancyhf{}
\fancyhead[L]{}
\fancyhead[R]{}
\fancyfoot[R]{\thepage}
\fancyfoot[L]{\footnotesize \examtitle}

\fancypagestyle{empty}{
 \fancyhf{}
 \fancyhead[L]{}
 \fancyhead[R]{}
 \fancyfoot[R]{\thepage}
 \fancyfoot[L]{\footnotesize \examtitle}
}

\usepackage{arevtext,arevmath}

\newcommand\Tstrut{\rule{0pt}{2.6ex}}         % = `top' strut
\newcommand\Bstrut{\rule[-0.9ex]{0pt}{0pt}}   % = `bottom' strut
\def\strut{\Tstrut\Bstrut}

% -----------------------------------------------------------------------
\IfFileExists{upquote.sty}{\usepackage{upquote}}{}
\begin{document}
\date{Wintersemester 2024/25 
\vfill
\begin{center}
\includegraphics[width = 1.9cm]{avatare/Alex}\hspace{-8mm}
\includegraphics[width = 1.9cm]{avatare/Jessica}\hspace{-8mm}
\includegraphics[width = 1.9cm]{avatare/Jonas}\hspace{-8mm}
\includegraphics[width = 1.9cm]{avatare/Mark}\hspace{-8mm}
\includegraphics[width = 1.9cm]{avatare/Nilufar}\hspace{-8mm}
\includegraphics[width = 1.9cm]{avatare/Paula}\hspace{-8mm}
\includegraphics[width = 1.9cm]{avatare/Steffen}\hspace{-8mm}
\includegraphics[width = 1.9cm]{avatare/Tina}\hspace{-8mm}
\includegraphics[width = 1.9cm]{avatare/Yuki}\\
\small
\vspace{1.5Ex}
\textit{"`The test of a student is not how much he knows,\\ but how much he wants to know."'\\ --- Alice W. Rollins}
\end{center}}
% -----------------------------------------------------------------------
\maketitle
\fancypagestyle{empty}{
  \fancyfoot[L]{\tiny $\blacksquare\!\square\!\blacksquare\!\blacksquare\!\blacksquare\!\square\!\square\!\blacksquare\!\blacksquare\!\blacksquare\!\square\!\blacksquare\!\blacksquare\!\square\!\blacksquare\!\blacksquare\!\blacksquare\!\blacksquare\!\square\!\blacksquare$}
}
\thispagestyle{empty}
\clearpage
% -----------------------------------------------------------------------
\begin{minipage}[c]{0.125\textwidth}
\includegraphics[width = 1.9cm]{avatare/Alex}
\end{minipage}
\begin{minipage}[c]{0.875\textwidth}
\textit{Alex studiert im 3. Semester und wiederholt das Modul, da er im ersten Jahr andere Prioritäten für sich gesetzt hat. Das musste sein, da er sich ziemlich im Abitur verausgabt hat. Darüber hinaus war die WG auch eher auf Party angelegt. Alex hofft jetzt über Pünktlichkeit wieder in die Bahn zu kommen. Dafür steht er jetzt immer um 5 Uhr auf! Freunde von ihm beschreiben ihn eher als extrovertiert. Er kennt Paula noch aus der Schulzeit und er überlegt, ob nicht beide Mal nach Mallorca sollten.} 
\end{minipage}\\[2.75Ex]
% -----------------------------------------------------------------------
\begin{minipage}[c]{0.875\textwidth}
\textit{Nach zwei Semestern Studium an der Universität Osnabrück war es dann Jessica doch viel zu theoretisch. Etwas angewandtes sollte es sein, wo sie auch eine Fähigkeit lernt, die frau nutzen kann. Deshalb hat sich Jessica an der Hochschule eingeschrieben. Hoffentlich lernt sie etwas nützliches, wo andere für Geld geben würden. Wer nützlich ist, ist wertvoll. Ihr Traum ist ja eine Hundeschule aufzumachen. Die großen Parties hat sie immer gemieden. Sie ist lieber mit ihrer Hündin im Teuteburgerwald.}
\end{minipage}
\begin{minipage}[c]{0.125\textwidth}
\includegraphics[width = 1.9cm]{avatare/Jessica}
\end{minipage}\\[2.75Ex]
% -----------------------------------------------------------------------
\begin{minipage}[c]{0.125\textwidth}
\includegraphics[width = 1.9cm]{avatare/Jonas}
\end{minipage}
\begin{minipage}[c]{0.875\textwidth}
\textit{Das ist jetzt der letzte Versuch mit einem Studium. Wenn es nicht klappt dann überlegt Jonas das \href{https://www.ihk.de/osnabrueck/aus-und-weiterbildung/ausbildung/ausbildungsbetriebe/projekt-neustart-1087206}{Programm der IHK zu Ausbildungsvermittlung} zu nutzen. Aber eine Runde gibt er sich noch. Struktur ist eigentlich das Wichtigste und diesmal hat er sich alle Altklausuren der Fachschaft besorgt. Dann ist er auch noch gleich der Fachschaft beigetreten um mehr Informationen abzugreifen. Und er versucht nicht seine Zeit mit Alex zu verdaddeln oder in der Fachschaft bei einem Bier oder so...}
\end{minipage}\\[2.75Ex]
% -----------------------------------------------------------------------
\begin{minipage}[c]{0.875\textwidth}
\textit{Nächstes Semester geht es nach Kanada davon hat er schon auf der Berufsschule geträumt. Deshalb konzentriert er sich sehr auf die Prüfungen. Ein Schiff ist im Hafen sicher, aber dafür ist es nicht gebaut worden. Das \href{https://www.hs-osnabrueck.de/wir/fakultaeten/aul/international/}{International Faculty Office} der Fakultät Agrarwissenschaften und Landschaftsarchitektur hat super geholfen, aber es waren einiges an Unterlagen. Jetzt hofft er, dass Tina dann doch noch mitkommt. Aber sonst macht er das eben alleine. Obwohl das eher nicht so seine Art ist. Vielleicht sollte er sich mal einen Tipp bei Tina holen, sie wirkt sehr entschlossen.} 
\end{minipage}
\begin{minipage}[c]{0.125\textwidth}
\includegraphics[width = 1.9cm]{avatare/Mark}
\end{minipage}\\[2.75Ex]
% -----------------------------------------------------------------------
\begin{minipage}[c]{0.125\textwidth}
\includegraphics[width = 1.9cm]{avatare/Nilufar}
\end{minipage}
\begin{minipage}[c]{0.875\textwidth}
\textit{Nach der Ausbildung wollte Nilufar eigentlich gleich anfangen zu arbeiten, aber nach einem Praktikum und der Probezeit stellte sie fest, dass es ohne einen Hochschulabschluss schwer wird Führungsverantwortung zu übernehmen. Mit Menschen kann sie schon immer und dann auch eigene Projekte mit anderen verwirklichen, dass ist doch was. Mit dem notwendigen Abschluss sollte der Start um so einfacher sein. Dann ist sie keine Befehlsempfängerin mehr sondern gibt die Marschrichtung vor. Schon jetzt koordiniert Nilufar das Studium von anderen.}
\end{minipage}\\[2.75Ex]
% -----------------------------------------------------------------------
\begin{minipage}[c]{0.875\textwidth}
\textit{Paula möchte die Welt zu einem besseren Ort machen. Wenn da nicht die anderen Mitmenschen wären. Paula muss das Modul nochmal wiederholen, da es dann am Ende des Semesters zu viel für sie wurde. Eine Lerngruppe hätte geholfen, aber das ist dann gar nicht so einfach eine zu finden. Zwar kennt sie schon Nilufar, aber Nilufar ist ihr manchmal zu forsch. Ihr schwant aber, dass alleine das Studium sehr schwer werden wird. Das Abitur war schon so ein Lernhorror, das möchte sie nicht nochmal. Alex sieht sie da als Vorbild.}
\end{minipage}
\begin{minipage}[c]{0.125\textwidth}
\includegraphics[width = 1.9cm]{avatare/Paula}
\end{minipage}\\[2.75Ex]
% -----------------------------------------------------------------------
\begin{minipage}[c]{0.125\textwidth}
\includegraphics[width = 1.9cm]{avatare/Steffen}
\end{minipage}
\begin{minipage}[c]{0.875\textwidth}
\textit{Sommer, Sonne, Natur. Das ist es was Steffen mag. Raus in die Komune und die Natur genießen. Leider hat Steffen noch andere Bedürfnisse, die ein Einkommen benötigen. Da Studierte mehr verdienen, würde dann in Teilzeit auch mehr rausspringen. Wenn er dann privat was anbauen kann, dann spart er gleich doppelt. Leider sind viele seiner Kommilitonen total verkrampfte Karrieristen. Es geht nur ums Äußere. Dabei verliert sich Steffen gerne im Prozess. Das hat auch seinen Schulabschluss etwas verzögert. Steffen lässt sich eben Zeit.}
\end{minipage}\\[2.75Ex]
% -----------------------------------------------------------------------
\begin{minipage}[c]{0.875\textwidth}
\textit{Wille  war es, die es Tina hierher gebracht hat und Wille wird es sein, die Tina dann auch zum Abschluß treibt. Nach einem Rückschlag muss Tina jetzt einige Module wiederholen, damit sie dann auch fertig wird. Ab und zu ist sie schwach gewesen und das hat dann Zeit gekostet. Das Tina es dann manchmal übertreibt, weiß sie nur zu gut, aber irgendwie muss die Kontrolle ja erhalten bleiben? Insbesondere, wenn sie mal wieder die Nacht durchgefeiert hat, verachtet Tina sich. Dann baut Nilufar sie dann bei einem Tee wieder auf.}
\end{minipage}
\begin{minipage}[c]{0.125\textwidth}
\includegraphics[width = 1.9cm]{avatare/Tina}
\end{minipage}\\[2.75Ex]
% -----------------------------------------------------------------------
\begin{minipage}[c]{0.125\textwidth}
\includegraphics[width = 1.9cm]{avatare/Yuki}
\end{minipage}
\begin{minipage}[c]{0.875\textwidth}
\textit{Für Yuki war es nicht einfach. Teilweise war die Krankheit sehr hinderlich, dann war es Yuki selber. Dann muss man auch wieder auf die Beine kommen und es dauert eben seine Zeit. Aber immerhin hat Yuki es jetzt den Abschluss gekriegt und hat einen Studienplatz. Jetzt heißt es in den Rhythmus kommen und schauen, was noch so passiert. Immerhin hat Yuki schon eine kleine Gruppe gefunden, in der Yuki dann Hilfe findet. Ist aber auch sehr unübersichtlich so ein Studium. Steffen ist immer super entspannt.}
\end{minipage}
\clearpage
% -----------------------------------------------------------------------


\begin{graybox}{Erlaubte Hilfsmittel}
  \vspace{1Ex}
  \begin{itemize}
  \item Normaler Taschenrechner ohne Möglichkeit der Kommunikation mit anderen
    Geräten! Ausdrücklich kein Handy!
  \item Eine DIN A4-Seite als beidseitig, selbstgeschriebene,
    handschriftliche Formelsammlung. Keine digitalen Ausdrucke! 
  \item \textbf{\textcolor{red}{Die Verwendung eines roten Farbstiftes ist nicht gestattet! Korrekturfarbe!}}
  \item \textit{You can answer the questions in English without any consequences.}  
  \end{itemize}
\end{graybox}
\vfill

\begin{graybox}{Endnote}
  \vspace{1Ex}
  \begin{itemize}
  \item[] \rule[0ex]{3em}{.4pt}\, von 20\, Punkten sind aus den Multiple
    Choice Aufgaben erreicht.
  \item[] \rule[0ex]{3em}{.4pt}\, von 81 Punkten sind aus den Rechen- und
    Textaufgaben erreicht. 
  \item[] \rule[0ex]{3em}{.4pt}\, von 101 Punkten in Summe.
  \item[] Es wird folgender Notenschlüssel angewendet.   
  \end{itemize}
  \vspace{1ex}
\begin{center}
  \begin{tabular}[c]{cc}
    \toprule
    \textbf{Punkte}	&	\textbf{Note}	\\
    \midrule
    96.5 - 101	&	1,0	\\
    91.5 - 96.0	&	1,3	\\
    86.5 - 91.0	&	1,7	\\
    81.5 - 86.0	&	2,0	\\
    76.5 - 81.0	&	2,3	\\
    71.5 - 76.0	&	2,7	\\
    66.5 - 71.0	&	3,0	\\
    61.5 - 66.0	&	3,3	\\
    56.5 - 61.0	&	3,7	\\
    50.5 - 56.0	&	4,0	\\
    \bottomrule
  \end{tabular}
\end{center}
  \vspace{1ex}
\begin{itemize}
\item[] Es ergibt sich eine Endnote von \rule[0ex]{4em}{.4pt}.
\end{itemize}
  \vspace{1Ex}
\end{graybox}

% -----------------------------------------------------------------------
\newpage
% -----------------------------------------------------------------------

\begin{graybox}{Multiple Choice Aufgaben}
  \begin{itemize}
  \item Pro Multipe Choice Frage ist \emph{genau} eine Antwort richtig.
  \item \textbf{Übertragen Sie Ihre Kreuze in die Tabelle auf
      dieser Seite.}
  \end{itemize}

\begin{center}
  \large
  \begin{tabular}{|l|c|c|c|c|c?c|}
    \hline
    & \textbf{A} & \textbf{B} & \textbf{C} & \textbf{D} & \textbf{E} & $\boldsymbol{\checkmark}$\strut\\
    \hline
    \textbf{Aufgabe 1} &   &   &   &   &   & \strut\\
    \hline
    \textbf{Aufgabe 2} &   &   &   &   &   & \strut\\
    \hline
    \textbf{Aufgabe 3} &   &   &   &   &   & \strut\\
    \hline
    \textbf{Aufgabe 4} &   &   &   &   &   & \strut\\
    \hline
    \textbf{Aufgabe 5} &   &   &   &   &   & \strut\\
    \hline
    \textbf{Aufgabe 6} &   &   &   &   &   & \strut\\
    \hline
    \textbf{Aufgabe 7} &   &   &   &   &   & \strut\\
    \hline
    \textbf{Aufgabe 8} &   &   &   &   &   & \strut\\
    \hline
    \textbf{Aufgabe 9} &   &   &   &   &   & \strut\\
    \hline
    \textbf{Aufgabe 10} &   &   &   &   &   & \strut\\
    \hline
  \end{tabular}
\end{center}

\begin{itemize}
\item Es sind \rule[0ex]{2em}{.4pt}\, von 20 Punkten erreicht worden.
\end{itemize}
\end{graybox}

\vfill

\begin{graybox}{Rechen- und Textaufgaben}
  \begin{center}
    \large
    \begin{tabular}{|l|c|c|c|c|c|c|c|}
      \hline
      \textbf{Aufgabe} & \textbf{11} & \textbf{12} & \textbf{13} & \textbf{14} & \textbf{15} & \textbf{16} & \textbf{17} \strut\\
      \hline
      \textbf{Punkte} & 
      \hspace{1Ex}\Large\textcolor{gray!70}{10}\hspace{1Ex}  & 
      \hspace{1Ex}\Large\textcolor{gray!70}{9}\hspace{1Ex}  & 
      \hspace{1Ex}\Large\textcolor{gray!70}{20}\hspace{1Ex}  & 
      \hspace{1Ex}\Large\textcolor{gray!70}{10}\hspace{1Ex}  & 
      \hspace{1Ex}\Large\textcolor{gray!70}{10}\hspace{1Ex}  & 
      \hspace{1Ex}\Large\textcolor{gray!70}{10}\hspace{1Ex}  & 
      \hspace{1Ex}\Large\textcolor{gray!70}{12}\hspace{1Ex} \strut\\
      \hline
  \end{tabular}
\end{center}
\begin{itemize}
\item Es sind \rule[0ex]{2em}{.4pt}\, von 81 Punkten erreicht worden.
\end{itemize}
\end{graybox}

% -----------------------------------------------------------------------
\clearpage
% -----------------------------------------------------------------------
\begin{graybox}{Multiple Choice Aufgaben}
Die Multiple Choice Aufgaben \textcolor{red}{unterliegen dem Zufall}. Die Reihenfolge der Antworten ist zufällig. Die Fragen und Antworten sind semantisch zufällig und haben somit \textcolor{red}{verschiedene Textvarianten}. Insbesondere die reinen Textaufgaben haben verschiedene Textvarianten. Die Semeantik mag sich unterscheiden, die Inhalte sind aber gleich.
\end{graybox}
\section*{ANOVA}

\section{Aufgabe \hfill (2 Punkte)}





Nach der Berechnung einer einfaktoriellen ANOVA ergibt sich ein $\eta^2 = 0.52$. Welche Aussage ist richtig?



\begin{enumerate}
\item [\textbf{A} \msquare] Das $\eta^2$ ist ein Wert f{"u}r die G{"u}te der ANOVA. Je kleiner desto besser. Ein $\eta^2$ von 0 bedeutet ein perfektes Modell mit keiner Abweichung. Die Varianz ist null.
\item [\textbf{B} \msquare] Die Berechnung von $\eta^2$ ist ein Wert f{"u}r die Interaktion.
\item [\textbf{C} \msquare] Das $\eta^2$ ist die Korrelation der ANOVA. Mit der Ausnahme, dass 0 der beste Wert ist.
\item [\textbf{D} \msquare] Das $\eta^2$ beschreibt den Anteil der Varianz, der von den Behandlungsbedingungen nicht erkl{"a}rt wird. Somit der Rest an nicht erkl{"a}rbarer Varianz.
\item [\textbf{E} \msquare] Das $\eta^2$ beschreibt den Anteil der Varianz, der von den Behandlungsbedingungen erkl{"a}rt wird. Das $\eta^2$ ist damit mit dem $R^2$ aus der linearen Regression zu vergleichen.
\end{enumerate} 

\section{Aufgabe \hfill (2 Punkte)}



Sie führen ein Feldexperiment durch um das Gewicht von Lauch zu
steigern. Die Pflanzen wachsen unter einer Kontrolle und zwei verschiedenen
Behandlungsbedingungen. Nach der Berechnung einer einfaktoriellen ANOVA
ergibt sich ein $\eta^2 = 0.17$. Welche Aussage ist richtig?



\begin{enumerate}
\item [\textbf{A} \msquare] Es werden 83\% der Varianz durch die Behandlung erklärt. Das $\eta^2$ beschreibt den Anteil der Varianz, der von den unterschiedlichen Behandlungsbedingungen nicht erklärt wird.
\item [\textbf{B} \msquare] Das $\eta^2$ beschreibt den Anteil der Varianz, der von den Umweltbedingungen erklärt wird. Daher werden 17\% der Varianz durch die Umweltbedingungen erklärt. Der Anteil der Varianz durch die Behandlungsgruppen ist dann 83\%.
\item [\textbf{C} \msquare] Das $\eta^2$ beschreibt den Anteil der Varianz, der von den Behandlungsbedingungen erklärt wird. Daher werden 17\% der Varianz erklärt.
\item [\textbf{D} \msquare] Mit dem $\eta^2$ lässt sich auf die Qualität der Randomisierung und damit der Strukturgleichheit zwischen der Grundgesamtheit und der Stichprobe schließen. Es gilt dabei die Regel, dass ein $\eta^2$-Wert von 1 zu bevorzugen ist.
\item [\textbf{E} \msquare] Es werden 17\% der Varianz durch den Versuch erklärt. Das $\eta^2$ beschreibt den Anteil der Varianz, der durch Fehler in der Versuchsdurchführung entsteht.
\end{enumerate} 

\section{Aufgabe \hfill (2 Punkte)}



Eine einfaktorielle ANOVA berechnet eine Teststatistik um zu die Nullhypothese abzulehnen. Welche Aussage über die Teststatistik der ANOVA ist richtig?



\begin{enumerate}
\item [\textbf{A} \msquare] Die ANOVA berechnet die T-Statistik indem den Mittelwertsunterschied der Gruppen simultan durch die Standardabweichung der Gruppen teilt. Wenn die T-Statistik h{"o}her als 1.96 ist, kann die Nullhypothese abgelehnt werden.
\item [\textbf{B} \msquare] Wenn die F-Statistik höher ist als der kritische Wert kann die Nullhypothese nicht abgelehnt werden. Die F-Statistik ist die Differenz der MS der Behandlung durch die MS des Fehlers.
\item [\textbf{C} \msquare] Die ANOVA berechnet die F-Statistik indem die MS des Fehlers durch die MS der Behandlung geteilt werden. Wenn die F-Statistik sich der 0 ann{"a}hert kann die Nullhypothese abgelehnt werden.
\item [\textbf{D} \msquare] Die ANOVA berechnet die F-Statistik indem die MS des Fehlers durch die MS der Behandlung geteilt werden. Wenn die F-Statistik sich der 1 ann{"a}hert kann die Nullhypothese nicht abgelehnt werden.
\item [\textbf{E} \msquare] Wenn die F-Statistik kleiner als der kritische Wert ist kann die Nullhypothese nicht abgelehnt werden. Die F-Statistik ist der Quotient der MS der Behandlung durch die MS des Fehlers.
\end{enumerate} 

\section{Aufgabe \hfill (2 Punkte)}




Viele statistische Verfahren nutzen eine Teststatistik um eine Aussage über den Zusammenhang zwischen der Grundgesamthat und der Stichprobe abzubilden. Ein statistisches Testwerkzeug ist hierbei die ANOVA. Die ANOVA rechnet dabei...



\begin{enumerate}
\item [\textbf{A} \msquare] ... den Unterschied zwischen mehreren Varianzen aus verschiedenen Behandlungsguppen. Wenn die ANOVA signifikant ist, ist nicht bekannt welcher Vergleich konkret unterschiedlich ist.
\item [\textbf{B} \msquare] ... den Unterschied zwischen der Varianz ausgelöst durch alle Behandlungsgruppen und der Varianz aus globalen Behandlungsguppen der Kontrollen. Wenn die ANOVA nicht signifikant ist, muss ein Posthoc-Test ausgeschlossen werden.
\item [\textbf{C} \msquare] ... den Unterschied zwischen der Varianz aus verschiedenen Behandlungsguppen und der Varianz über alle Behandlungsgruppen. Wenn die ANOVA signifikant ist, muss über einen Posthoc-Test nachgedacht werden um den signifikanten Unterschied in den Gruppen exakt zu bestimmen.
\item [\textbf{D} \msquare] ... den Unterschied zwischen der F-Statistik anhand der Varianz der Gruppen. Wenn die F-Statistik exakt 0 ist, kann die Nullhypothese abgelehnt werden.
\item [\textbf{E} \msquare] ... den Unterschied zwischen der Varianz über alle Behandlungsgruppen oder der Varianz aus verschiedenen Behandlungsguppen. Wenn die ANOVA signifikant ist, muss sich zwischen einem der beiden Varianzquellen entschieden werden.
\end{enumerate} 

\section{Aufgabe \hfill (2 Punkte)}



Die folgende Abbildung enthält die Daten aus einer Studie zur Bewertung der Wirkung von Vitamin E auf das Zahnwachstum bei Igeln. Der Versuch wurde an 67 Tieren durchgeführt, wobei jedes Tier eine von  drei Vitamin-C-Dosen (0.5, 1 und 1.5 mg/Tag) über eine von zwei Verabreichungsmethoden erhielt. Welche Aussage ist richtig im Bezug auf eine zweifaktorielle ANOVA?



{\centering \includegraphics[width=\maxwidth]{img/mc-anova-02-a-1} 

}







\begin{enumerate}
\item [\textbf{A} \msquare] Die Koeffizienten sind positiv $(\beta_0 > 0; \beta_1 > 0)$.
\item [\textbf{B} \msquare] Eine mittlere bis starke Interaktion liegt vor $(p \leq 0.05)$
\item [\textbf{C} \msquare] Eine positive Interaktion liegt vor $(\rho \leq -0.5)$ 
\item [\textbf{D} \msquare] Die Koeffizienten sind negativ $(\beta_0 < 0; \beta_1 < 0)$.
\item [\textbf{E} \msquare] Keine Interaktion liegt vor $(p \leq 0.05)$.
\end{enumerate} 
\section*{Deskriptive Statistik \& Explorative Datenanalyse}

\section{Aufgabe \hfill (2 Punkte)}




Berechnen Sie den Mittelwert und Standardabweichung von $y$ mit 12, 11, 10, 10 und 13.



\begin{enumerate}
\item [\textbf{A} \msquare] Sie erhalten 11.2 +/- 1.14
\item [\textbf{B} \msquare] Es berechnet sich 11.2 +/- 1.3
\item [\textbf{C} \msquare] Es ergibt sich 10.2 +/- 0.85
\item [\textbf{D} \msquare] Es ergibt sich 12.2 +/- 0.65
\item [\textbf{E} \msquare] Es berechnet sich 11.2 +/- 1.7
\end{enumerate} 

\section{Aufgabe \hfill (2 Punkte)}




Gegeben ist $y$ mit 28, 23, 22, 14, 28, 15, 15, 8, 28, 15 und 63. Berechnen Sie den Median, das $1^{st}$ Quartile sowie das $3^{rd}$ Quartile.




\begin{enumerate}
\item [\textbf{A} \msquare] Es ergibt sich 24 +/- 15
\item [\textbf{B} \msquare] Es berechnet sich 24 [16; 29]
\item [\textbf{C} \msquare] Sie erhalten 22 [15; 28]
\item [\textbf{D} \msquare] Sie erhalten 22 [13; 26]
\item [\textbf{E} \msquare] Sie erhalten 22 +/- 28
\end{enumerate} 

\section{Aufgabe \hfill (2 Punkte)}



Sie überlegen Ihre Daten mit einem Boxplot zu visualisieren. Was ist die minimale Anzahl an Beobachtungen pro Gruppe ?



\begin{enumerate}
\item [\textbf{A} \msquare] Die opimale Anzahl ist größer als hundert Beobachtungen, wobei es gerne sehr viel mehr sein können.
\item [\textbf{B} \msquare] Die untere Grenze liegt bei einer Beobachtung.
\item [\textbf{C} \msquare] Wir sollten zwei bis fünf Beobachtungen mindestens pro Gruppe vorliegen haben.
\item [\textbf{D} \msquare] Mindestens 20 Beobachtungen.
\item [\textbf{E} \msquare] 1 Beobachtung.
\end{enumerate}

\section{Aufgabe \hfill (2 Punkte)}



Die Varianz ist eine bedeutende deskriptive Statistik für die Analyse von Daten. Wie müssen Sie vorgehen um die Varianz zu berechnen?



\begin{enumerate}
\item [\textbf{A} \msquare] Den Median berechen, dann die quadratischen Abstände zum Median aufsummieren, dann die Wurzel ziehen.
\item [\textbf{B} \msquare] Den Mittelwert berechnen und die Abstände quadrieren. Die Summe mit der Fallzahl multiplizieren.
\item [\textbf{C} \msquare] Wir berechnen erst den Mittelwert und dann die quadratischen Abstände zu dem Mittelwert. Diese quadratischen Abstände summieren wir auf und teilen am Ende durch die Fallzahl. Als letzten Schritt ziehen wir die quadratische Wurzel.
\item [\textbf{D} \msquare] Den Mittelwert berechen, dann die quadratischen Abstände zum Mittelwert aufsummieren und durch die Fallzahl teilen.
\item [\textbf{E} \msquare] Wir berechnen erst den Mittelwert und dann die absoluten Abstände zu dem Mittelwert. Diese quadratischen Abstände summieren wir auf und teilen am Ende durch die Fallzahl.
\end{enumerate} 

\section{Aufgabe \hfill (2 Punkte)}



Der Boxplot stellt folgende statistische Maßzahlen in einer Abbildung dar. Damit gehört der Boxplot zu einem der am meisten genutzten statistischen Verfahren zur Visualisierung von Daten.

 



\begin{enumerate}
\item [\textbf{A} \msquare] Der Boxplot stellt die Mittelwerte und die Standardabweichung dar.
\item [\textbf{B} \msquare] Der Boxplot stellt die Mittelwerte und die Varianz dar.
\item [\textbf{C} \msquare] Der Boxplot stellt den Median und die Streuung dar.
\item [\textbf{D} \msquare] Den Mittelwert und die Varianz.
\item [\textbf{E} \msquare] Der Boxplot stellt den Median und die Quartile dar.
\end{enumerate}

\section{Aufgabe \hfill (2 Punkte)}



Nachdem Sie in einem Feldexperiment zu Leistungssteigerung von Maiss durchgeführt haben, berechnen Sie den Mittelwert und den Median. Der Mittelwert $\bar{y}$ und der Median $\tilde{y}$ unterscheiden sich. Welche Aussage ist richtig?



\begin{enumerate}
\item [\textbf{A} \msquare] Der Mittelwert und der Median sollten gleich sein, wenn Outlier in den Daten vorliegen. 
\item [\textbf{B} \msquare] Da sich der Mittelwert und der Median unterscheiden, ist der Datensatz nicht zu verwenden. Mittelwert und Median müssen gleich sein.
\item [\textbf{C} \msquare] Da sich der Mittelwert und der Median nicht unterscheiden, liegen vermutlich keine Outlier in den Daten vor. Wir verweden den Datensatz so wie er ist.
\item [\textbf{D} \msquare] Wenn sich der Mittelwert und der Median unterscheiden, liegen vermutlich Outlier in den Daten vor.
\item [\textbf{E} \msquare] Wenn sich der Mittelwert und der Median unterscheiden, liegen vermutlich keine Outlier in den Daten vor.
\end{enumerate}

\section{Aufgabe \hfill (2 Punkte)}



Sie wollen eine ANOVA im Anschluss an Ihr Feldexperiment rechnen. Dafür muss Ihr gemessener Endpunkt die Annahme einer Varianzhomogenität genügen. Zur Überprüfung können Sie folgende Visualisierung nutzen. Welche entsprechende Regel zur Abschätzung der Annahme einer Varianzhomogenität kommt zur Anwendung?



\begin{enumerate}
\item [\textbf{A} \msquare] Einen Barplot. Die Mittelwerte müssen alle auf einer Höhe liegen. Die Fehlerbalken haben hier keine Informationen.
\item [\textbf{B} \msquare] Nach der Erstellung eines Boxplots schauen wir, ob der Median in der Mitte der Box liegt. Dabei ist der Median als dicke Linie dargestellt und die Box ist das IQR.
\item [\textbf{C} \msquare] Einen Violinplot. Der Bauch der Violine muss hierbei einen höhren Wert annehmen als der Steg der Violine. Dann kann die Annahme einer Varianzhomogenität angenommen werden.
\item [\textbf{D} \msquare] Einen Dotplot. Die Punkte müssen sich wie an einer Perlenschnurr audreihen. Eine Abweichung führt zur Ablehnung der Annahme einer Varianzhomogenität.
\item [\textbf{E} \msquare] Einen Boxplot. Das IQR muss über alle Behandlungen zusammen mit den Whiskers ungefähr gleich aussehen.
\end{enumerate}

\section{Aufgabe \hfill (2 Punkte)}




In der Statistik müssen wir häufig überprüfen, ob unser Outcome einer bestimmten Verteilung folgt. Meistens überprüfen wir, ob eine
Normalverteilung vorliegt. Folgende drei Abbildungen eigenen sich im Besonderen für die Überprüfung einer Verteilungsannahme an eine Variable.





\begin{enumerate}
\item [\textbf{A} \msquare] Boxplot, Violinplot, Mosaicplot
\item [\textbf{B} \msquare] Histogramm, Densityplot, Dotplot
\item [\textbf{C} \msquare] Barplot, Mosaicplot, Violinplot
\item [\textbf{D} \msquare] Histogramm, Scatterplot, Boxplot
\item [\textbf{E} \msquare] Violinplot, Boxplot, Densityplot
\end{enumerate} 

\section{Aufgabe \hfill (2 Punkte)}



In dem folgenden Histogramm von $n = 194$ Pflanzen ist welche Verteilung abgebildet?



{\centering \includegraphics[width=\maxwidth]{img/mc-distribution-02-a-1} 

}







\begin{enumerate}
\item [\textbf{A} \msquare] Wir haben eine Poisson-Verteilung vorliegen.
\item [\textbf{B} \msquare] Eine multivariate Normalverteilung.
\item [\textbf{C} \msquare] Wir haben eine Gammaverteilung vorliegen.
\item [\textbf{D} \msquare] Es handelt sich um eine Binomial-Verteilung.
\item [\textbf{E} \msquare] In dem Histogramm ist eine Normalverteilung dargestellt.
\end{enumerate} 
\section*{Lineare Regression \& Korrelation}

\section{Aufgabe \hfill (2 Punkte)}



Sie haben das Modell $Y \sim X$ vorliegen und wollen nun ein kausales Modell rechnen. Welche Aussage ist richtig?



\begin{enumerate}
\item [\textbf{A} \msquare] Ein kausales Modell wird auf einem Trainingsdatensatz trainiert und anschliessend über eine explorative Datenanalyse validiert. Signifikanzen über $\beta_i$ können hier nicht festgestellt werden.
\item [\textbf{B} \msquare] Ein kausales Modell benötigt mindestens eine Fallzahl von über 100 Beobachtungen und darf keine fehlenden Werte beinhalten. Die Varianzkomponenten müssen homogen sein.
\item [\textbf{C} \msquare] Wenn ein kausales Modell gerechnet werden soll, dann muss zum einen ein Traingsdatensatz sowie ein Testdatensatz definiert werden. Dabei ist der Trainingsdatensatz meist 2/3 und der Testdatensatz 1/3 der Fallzahl groß. Der Testdatensatz dient zur Validierung.
\item [\textbf{D} \msquare] Wir modellieren den Zusammenhang zwischen $X$ und $Y$ wenn ein kausales Modell rerechnet wird. Dabei kann nicht der gesamte Datensatz genutzt werden. Es wird ein Trainingsdatensatz zum Trainieren des Modells benötigt.
\item [\textbf{E} \msquare] Wenn ein kausales Modell gerechnet werden soll dann kann dies auf dem gesamten Datensatz geschehen. Das Ziel ist es einen Zusammenhang von $X$ auf $Y$ zu modellieren. Wie wirken sich die Einflussvariablen $X$ auf den gemessenen Endpunkt $Y$ aus?
\end{enumerate}

\section{Aufgabe \hfill (2 Punkte)}



Nach der Modellierung einer Regression stellt sich die Frage, ob die Residuen approximativ einer Normalverteilung folgen. Sie können einen QQ-Plot für die visuelle Überprüfung der Annahme an die Residuen nutzen. Welche Aussage ist richtig?



{\centering \includegraphics[width=\maxwidth]{img/mc-regression-05-a-1} 

}







\begin{enumerate}
\item [\textbf{A} \msquare] Die Annahme der normalverteilten Residuen ist erfüllt. Die Punkte liegen zum überwiegenden Teil auf der Geraden.
\item [\textbf{B} \msquare] Wir betrachten die Gerade, die durch die einzelnen Punkte laufen sollte. Wenn die 95\% der Punkte von der Geraden getroffen werden, dann gehen wir von normalverteilten Residuen aus.
\item [\textbf{C} \msquare] Wir betrachten die Punkte. Wenn die Punkte einigermaßen gleichmäßig verteilt liegen, dann gehen wir von normalen Residuen aus.
\item [\textbf{D} \msquare] Wir betrachten insbesondere die beiden Enden der Gerade. Der Rest ist mehr oder minder egal, dann ist die Annahme an die Normalverteilung der Residuen erfüllt.
\item [\textbf{E} \msquare] Die Annahme der normalverteilten Residuen ist erfüllt. Die Punkte liegen zum überwiegenden Teil nicht auf der Geraden.
\end{enumerate}

\section{Aufgabe \hfill (2 Punkte)}



Nach einer Regressions sollten die Residuen (\texttt{.resid}) gleichmäßig um die Gerade verortet sein. Was bei einer simplen Regression noch relativ einfach visuell in einem Scatterplot zu überprüfen ist. Für komplexere Modell liefert der Residual Plot die notwendigen Informationen. Welche Aussage ist richtig?



{\centering \includegraphics[width=\maxwidth]{img/mc-regression-06-a-1} 

}







\begin{enumerate}
\item [\textbf{A} \msquare] Die Annahme der normalverteilten Residuen ist erfüllt. Es ist ein Muster zu erkennen und wir können damit auf die Signifkanz von $x_1, ..., x_p$ schließen.
\item [\textbf{B} \msquare] Die Annahme der normalverteilten Residuen ist nicht erfüllt. Vereinzelte Punkte liegen oberhalb bzw. unterhalb der Geraden um die 0 Linie weiter entfernt. Ein klares Muster ist zu erkennen.
\item [\textbf{C} \msquare] Wir betrachten die Nulllinie und alle Punkte sollten ohne Muster gleichmäßig um die Nulllinie liegen. Da dies der Fal ist, gehen wir von keinen Ausreißern aus.
\item [\textbf{D} \msquare] Wenn wir die Nulllinie betrachten so müssen die Punkte gleichmäßig unter der Nulllinie liegen. Unser Modell erfüllt somit nicht die Annahme von normalverteilten Residuen mit einem Mittelwert von $>0$ und einer Streuung von $s$.
\item [\textbf{E} \msquare] Die Punkte müssen gleichmäßig, mit ähnlichen Abständen, in dem positiven wie auch negativen Bereich liegen. Dies ist hier klar nicht der Fall. Einzelne Ausreißer können beobachtet werden. Wir können mit dem Model so nicht rechnen und müssen erst die auffälligen Werte gesondert betrachten.
\end{enumerate}

\section{Aufgabe \hfill (2 Punkte)}




Sie berechnen in Ihgrer Abschlussarbeit den Korrelationskoeffizienten $\rho$. Welche Aussage über den Korrelationskoeffizienten $\rho$ ist richtig?




\begin{enumerate}
\item [\textbf{A} \msquare] Der Korrelationskoeffizienten $\rho$ zeigt keinen Zusammenhang zwischen zwei Variablen $x$ und $y$ bei einem Wert von 0. Einen maximalen negativen Zusammenhang bei -1 und somit auch einen maximalen positiven Zusammenhang bei 1. Korrelationskoeffizienten $\rho$ ist einheitslos.
\item [\textbf{B} \msquare] Der Korrelationskoeffizienten $\rho$ liegt zwischen -1 und 1. Darüber hinaus ist der Korrelationskoeffizienten $\rho$ als standardisierte Steigung zu verstehen, wenn eine Standardisierung durchgeführt wurde. Diese Adjustierung nach Fischer muss am Anschluß der Berechnung der Korrelation durchgeführt werden.
\item [\textbf{C} \msquare] Der Korrelationskoeffizienten $\rho$ ist eine standardisierte, statistische Maßzahl, die zwischen 0 und 1 liegt. Dabei ist Korrelationskoeffizienten $\rho$ einheitslos. Eine Signifikanz kann nicht nachgewiesen werden.
\item [\textbf{D} \msquare] Der Korrelationskoeffizienten $\rho$ zeigt keinen Zusammenhang zwischen zwei Variablen $x$ und $y$ bei einem Wert von 0. Einen negativen Zusammenhang Richtung -1 und somit auch einen positiven Zusammenhang Richtung 1. Je größer die Zahl allgemein, desto stärker der Effekt.
\item [\textbf{E} \msquare] Der Korrelationskoeffizienten $\rho$ wird wie das $\eta^2$ aus der ANOVA interpretiert. Der Korrelationskoeffizienten $\rho$ beschreibt den Anteil an erklärter Varianz durch die Regression. Dabei gibt er jedoch eine Richtung an und kann auch negativ werden.
\end{enumerate}

\section{Aufgabe \hfill (2 Punkte)}



In einer lineren Regression kann es vorkommen, dass der Effekt repräsentiert durch den $\beta$ Koeffizienten nicht so richtig von der Größenordnung zu dem p-Wert passen will. So liefert eine Untersuchung des Einflusses von der $PO_2$-Konzentration in [$\mu g$] im Wasser auf das Wachstum in [$kg$] an Brokkoli folgende Effekte und p-Werte: $0.00051$ als p-Wert und einen $\beta_{PO_2}$ Koeffizienten von $2.3\times 10^{-9}$. Welche Aussage ist richtig?




\begin{enumerate}
\item [\textbf{A} \msquare] Die Fallzahl ist zu hoch angesetzt. Je höher die Fallzahl ist, desto kleiner ist die Teststatistik und damit ist dann auch der $p$-Wert sehr klein. Es sollte über eine Reduzierung der Fallzahl nachgedacht werden. Dann sollte der Effekt zum p-Wert passen.
\item [\textbf{B} \msquare] Manchmal ist die Einheit der Einflussvariable $X$ zu groß gewählt, so dass der Ansteig von 1 Einheit in $X$ zu einer zu großen Änderung in $y$ führt. Daher kann der Effekt $\beta_{PO_2}$ sehr klein wirken, da der p-Wert wird auf einer einheitslosen Teststatistik bestimmt wird.
\item [\textbf{C} \msquare] Die Fallzahl ist zu klein angesetzt. Je kleiner die Fallzahl ist, desto höher ist die Teststatsitik und damit auch der $p$-Wert kleiner. Wir brauchen also mehr Fallzahl um den geringen Effekt noch signifikant zu krigen.
\item [\textbf{D} \msquare] Manchmal ist die Einheit der Einflussvariable $X$ zu klein gewählt, so dass der Ansteig von 1 Einheit in $X$ zu einer zu kleinen Änderung in $y$ führt. Daher kann der Effekt $\beta_{PO_2}$ sehr klein wirken, aber auf einer anderen Einheit sehr viel größer sein. Der p-Wert wird auf einer einheitslosen Teststatistik bestimmt.
\item [\textbf{E} \msquare] Wenn der Effekt $\beta_{PO_2}$ winzig ist, dann kann es an einer falsch gewählten Einheit liegen. Der Anstieg von einer Einheit in $X$ führt ja zu einer Änderung von $\beta_{PO_2}$ in $x$. Wir müssen daher die Einheit von $y$ entsprechend anpassen.
\end{enumerate}

\section{Aufgabe \hfill (2 Punkte)}



Nachdem Sie Ihr Experiment abgeschlossen haben, stehen Sie vor der Frage wie Sie Ihre Daten modellieren sollen. In der Beispielauswertung von Ihrem Betreuenden finden Sie die Funktion \texttt{lm()} in \Rlogo. Welche Aussage ist richtig?





\begin{enumerate}
\item [\textbf{A} \msquare] Die Funktion \texttt{lm()} in \Rlogo wird klassischerweise für die nicht-lineare Regression genutzt. Ist die Einflussvariable $X$ numerisch so werden die Gruppenmittelwerte geschätzt.
\item [\textbf{B} \msquare] Ist die Einflussvariable $X$ numerisch so werden die Gruppenmittelwerte geschätzt und eine anschließende ANOVA sowie multipler Gruppenvergleich mit \{emmeans\} ist möglich.
\item [\textbf{C} \msquare] Neben der klassichen Verwendung der Funktion \texttt{lm()} in der linearen Regression kann auch ein Gruppenvergleich gerechnet werden. Dafür müssen aber alle Faktoren aus den Daten entfernt und numerishc umgewandelt werden. Dann kann das R Paket \{emmeans\} genutzt werden um die Korrelation zu berechnen. Eine Adjustierung ist dann nicht mehr notwendig.
\item [\textbf{D} \msquare] Die Funktion \texttt{lm()} in \Rlogo ist der erste Schritt für einen Gruppenvergleich. Danach kann eine ANOVA oder aber ein multipler Vergleich in \{emmeans\} gerechnet werden. In der Funktion  \texttt{lm()} werden die Gruppenmittelwerte bestimmt.
\item [\textbf{E} \msquare] Die Funktion \texttt{lm()} berechnet die Varianzstruktur für eine ANOVA. Dannach kann dann über eine explorative Datenalayse nochmal eine Signifikanz berechnet werden. Sollte vor der Verwendung der Funktion \texttt{lm()} schon eine EDA gerechnet worden sein, so ist die Analyse wertlos.
\end{enumerate}

\section{Aufgabe \hfill (2 Punkte)}



Welche Aussage über das \textit{generalisierte lineare Modell (GLM)} ist richtig?




\begin{enumerate}
\item [\textbf{A} \msquare] Dank dem \textit{generalisierten linearen Modell (GLM)} können auch andere Verteilungsfamilien als die Normalverteilung mit einer linearen Regression modelliert werden.
\item [\textbf{B} \msquare] Das GLM ist ein faktisch maschineller Lernalgorithmus, der selstständig die Verteilungsfamilie für Y wählt.
\item [\textbf{C} \msquare] Das GLM ist eine allgemeine Erweiterung der linearen Regression auf die Normalverteilung.
\item [\textbf{D} \msquare] Das GLM ist eine Vereinfachung des LM in R. Mit dem GLM lassen sich polygonale Regressionen rechnen. Somit stehen neben der Normalverteilung noch weitere Verteilungen zu Verfügung.
\item [\textbf{E} \msquare] In \Rlogo ist mit dem \textit{generalisierten linearen Modell (GLM)} eine Modellierung implementiert, die die Poissonverteilung für Zähldaten oder die Binomialverteilung für 0/1-Daten modellieren kann. Weitere Modellierungen sind in \Rlogo auch mit zusätzlich geladenen Paketen nicht möglich.
\end{enumerate}
\section*{Vermischte Themen}  

\section{Aufgabe \hfill (2 Punkte)}

Die Randomisierung von Beobachtungen zu den Versuchseinheiten
ist bedeutend in der Versuchsplanung. Welche der folgenden Aussagen ist richtig?



\begin{enumerate}
\item [\textbf{A} \msquare] Randomisierung ist die direkte Folge von Strukturgleichheit. Die Strukturgleichheit erlaubt es erst von der Stichprobe auf die Grundgesamtheit zurückzuschliessen.
\item [\textbf{B} \msquare] Durch eine Randomisierung können wir nicht von Strukturgleichheit zwischen der Stichprobe und der Grundgesamtheit ausgehen.
\item [\textbf{C} \msquare] Randomisierung erlaubt erst die Varianzen zu schätzen. Ohne eine Randomisierung ist die Berechnung von Mittelwerten und Varianzen nicht möglich. Dadurch lässt sich erst ein Experiment auswerten.
\item [\textbf{D} \msquare] Strukturgleichheit ist durch Randomisierung gegeben. Somit kann von der Stichprobe auf die Grundgesamtheit geschlossen werden
\item [\textbf{E} \msquare] Randomisierung erlaubt erst die Mittelwerte zu schätzen. Ohne Randomisierung keine Mittelwerte. Ohne Mittelwerte keine Varianz und somit auch kein statistischer Test.
\end{enumerate}

\section{Aufgabe \hfill (2 Punkte)}



Sie wollen Ihren Datensatz in \Rlogo einlesen und stehen nun vor einem Problem. Sie stellen fest, dass die Hilfeseiten alle in englischer Sprache verfasst sind. Warum mag die Nutzung von Deutsch problematisch sein?



\begin{enumerate}
\item [\textbf{A} \msquare] Programmiersprachen können nur englische Begriffe verarbeiten. Zusätzliche Pakete können zwar geladen werden, aber meist funktionieren diese Pakete nicht richtig. Deutsch ist International nicht bedeutend genug.
\item [\textbf{B} \msquare] \Rlogo Pakete sind nur in englischer Sprache verfasst. Es macht keinen Sinn \Rlogo daher in Deutsch zu bedienen.
\item [\textbf{C} \msquare] Programmiersprachen haben Probleme mit Umlauten und Sonderzeichen der deutschen Sprache. Daher ist die Nutzung in Deutsch in den AGBs von \Rlogo untersagt.
\item [\textbf{D} \msquare] Die \Rlogo Pakete sind nur in englischer Sprache verfasst. Das ist aber nicht der Hauptgrund, denn \Rlogo hat wie alle Programmiersprachen Probelem mit Umlauten und Sonderzeichen.
\item [\textbf{E} \msquare] Alle Funktionen und auch Anwendungen sind in \Rlogo in englischer Sprache. Die Nutzung von deutschen Wörtern ist nicht schick und das ist zu vermeiden.
\end{enumerate}

\section{Aufgabe \hfill (2 Punkte)}



Nachdem Sie Ihr Feldexperiment als Vorversuch für Ihre Abschlussarbeit abgeschlossen haben, wollen Sie in einer explorativen Datenanalyse (EDA) in \Rlogo einmal schauen, ob Sie überhaupt Effekte der Behandlung vorliegen haben. Welche Reihenfolge von Schritten müssen Sie in \Rlogo durchführen, damit Sie eine EDA rechnen können?



\begin{enumerate}
\item [\textbf{A} \msquare] Für eine explorativen Datenanalyse (EDA) in \Rlogo müssen wir als erstes die Daten über \texttt{read\_excel()} einlesen. Danach müssen wir schauen, dass wir die Zeilen richtig über \texttt{mutate()} transformiert haben. Insbesondere müssen Variablen mit kontinuierlichen Werten in einen Faktor umgewandelt werden. Am Ende nutzen wir die Funktion \text{ggplot()} für die eigentlich EDA.
\item [\textbf{B} \msquare] Die Funktionsreihenfolge ist wie folgt: \texttt{read\_excel()} ->  \texttt{mutate()} -> \text{ggplot()}. Dabei ist bei der Transformation der Daten darauf zu achten, dass keine Faktoren erstellt werden.
\item [\textbf{C} \msquare] Wir lesen als erstes die Daten über \texttt{read\_excel()} ein, transformieren die Spalten über \texttt{mutate()} in die richtige Form und können dann  über \text{ggplot()} uns die Abbildungen erstellen lassen. Wichtig ist, dass wir keine Faktoren sondern nur numerische Variablen vorliegen haben.
\item [\textbf{D} \msquare] Die Funktionsreihenfolge ist wie folgt: \texttt{read\_excel()} ->  \texttt{mutate()} -> \text{ggplot()}. Dabei ist bei der Transformation der Daten darauf zu achten, dass die Faktoren richtig erstellt werden.
\item [\textbf{E} \msquare] Wir lesen die Daten über eine generische Funktion \texttt{read()} ein und müssen dann die Funktion \texttt{ggplot()} nur noch installieren. Dann haben wir die Abbildungen als \texttt{*.png} vorliegen.
\end{enumerate}

\section{Aufgabe \hfill (2 Punkte)}



Es sei $n_1 = n_2$ in dem Modell $Y \sim X$. Welche Aussage ist richtig?



\begin{enumerate}
\item [\textbf{A} \msquare] Es handelt sich um ein balanciertes Design.
\item [\textbf{B} \msquare] Es handelt sich um unabhängige Beobachtungen.
\item [\textbf{C} \msquare] Es liegt Varianzhetrogenität vor.
\item [\textbf{D} \msquare] Es liegt Varianzhomogenität vor.
\item [\textbf{E} \msquare] Es handelt sich um ein unbalanciertes Design.
\end{enumerate}

\section{Aufgabe \hfill (2 Punkte)}



Im Rahmen Ihrer Abschlussarbeit werten Sie ein Experiment mit Ferkel aus. Es geht um die Leistungssteigerung der Ferkelproduktion. Sie messen jeweils die Gewichtszunahme der Ferkel. Die Ferkel einer Muttersau sind dabei...



\begin{enumerate}
\item [\textbf{A} \msquare] Abhängig von der Stallanlage und des Experiments können die Ferkel abhängig oder unabhängig sein. Allgmein gilt, dass Ferkel von unterschiedlichen Sauen näher miteinander verwandt sind als Ferkel von gleichen Sauen. Das Fisher-Axiom.
\item [\textbf{B} \msquare] Untereinander unabhängig. Die Ferkel sind eigenständig und benötigen keine zusätzliche Behandlung.
\item [\textbf{C} \msquare] Die Ferkel stammen vom gleichen Muttertier und haben vermutlich eine ähnlichere Varianzstruktur als die Ferkel von anderen Sauen. Die Ferkel sind untereinander über die Mutter abhängig.
\item [\textbf{D} \msquare] Untereinander abhängig, wenn die Mütter ebenfalls miteinander verwandt sind. Erst die Abhängigkeit 2. Grades wird in der Statistik modelliert.
\item [\textbf{E} \msquare] Je nach Stallanlage kommt eine andere Analyse in Betracht. Eine allgemeine Aussage über Ferkel und Sauen lässt sich statistisch nicht treffen.
\end{enumerate}

\section{Aufgabe \hfill (2 Punkte)}



In einer Studie wollen Sie den Effektschätzer Odds ratio berechnen. Sie finden in Ihrem Experiment zur Behandlung von Klaueninfektionen bei Kühen in 4 Tieren Erkrankung der Klauen vor. 8 Tiere sind gesund. Welche Aussage ist richtig?



\begin{enumerate}
\item [\textbf{A} \msquare] Es ergibt sich ein Odds ratio von 0.33, da es sich um ein Anteil handelt. Wir berechnen den Anteil der Kranken.
\item [\textbf{B} \msquare] Es ergibt sich ein Odds ratio von 0.5, da es sich um ein Anteil handelt.
\item [\textbf{C} \msquare] Es ergibt sich ein Odds ratio von 2, da es sich um ein Anteil handelt.
\item [\textbf{D} \msquare] Das Verhältnis der Chancen Odds ratio ergibt ein Chancenverhältnis von 0.33. Wir sind an der Chance krank zu sein interessiert.
\item [\textbf{E} \msquare] Das Verhältnis von Chancen Odds ratio ergibt ein Chancenverhältnis von 0.5.
\end{enumerate}

\section{Aufgabe \hfill (2 Punkte)}



In der Bio Data Science wird häufig mit sehr großen Datensätzen gerechnet. Historisch ergibt sich nun ein Problem bei der Auswertung der Daten und deren Bewertung hinsichtlich der Signifikanz. Welche Aussage ist richtig?





\begin{enumerate}
\item [\textbf{A} \msquare] Aktuell werden zu grosse Datensätze für die gänigige Statistik gemessen. Daher wendet man maschinelle Lernverfahren für kausale Modelle an. Hier ist die Relevanz gleich Signifikanz.
\item [\textbf{B} \msquare] Aktuell werden immer größere Datensätze erhoben. Dadurch wird auch die Varianz immer höher was automatisch zu mehr signifikanten Ergebnissen führt.
\item [\textbf{C} \msquare] Eine große Fallzahl führt zu mehr signifikanten Ergebnissen auch bei kleinen Effekten. Daher werden fast alle Vergleich esignifikant, wenn die Fallzahl nur groß genug wird.
\item [\textbf{D} \msquare] Riesige Datensätz haben mehr Fallzahl was zur $\alpha$-Inflation führt. Durch eine Adjustoerung kann dem Problem entgegengewirkt werden.
\item [\textbf{E} \msquare] Big Data ist ein Problem der parametrischen Statistik. Parameter lassen sich nur auf kleinen Datensätzen berechnen, da es sich sonst nicht mehr um eine Stichprobe im engen Sinne der Statistik handelt.
\end{enumerate}
\section*{Multiple Gruppenvergleiche}    

\section{Aufgabe \hfill (2 Punkte)}



Sie haben folgende unadjustierten p-Werte gegeben: 0.89, 0.001, 0.34 und 0.02. Sie adjustieren die p-Werte nach
Bonferroni. Welche Aussage ist richtig?



\begin{enumerate}
\item [\textbf{A} \msquare] Nach der Bonferroni-Adjustierung ergeben sich die adjustierten p-Werte von 1, 0.004, 1 und 0.08. Die adjustierten p-Werte werden zu einem $\alpha$-Niveau von 5\% verglichen.
\item [\textbf{B} \msquare] Nach der Bonferroni-Adjustierung ergeben sich die adjustierten p-Werte von 3.56, 0.004, 1.36 und 0.08. Die adjustierten p-Werte werden zu einem $\alpha$-Niveau von 5\% verglichen.
\item [\textbf{C} \msquare] Nach der Bonferroni-Adjustierung ergeben sich die adjustierten p-Werte von 1, 0.004, 1 und 0.08. Die adjustierten p-Werte werden zu einem $\alpha$-Niveau von 1.25\% verglichen.
\item [\textbf{D} \msquare] Nach der Bonferroni-Adjustierung ergeben sich die adjustierten p-Werte von 0.2225, 3e-04, 0.085 und 0.005. Die adjustierten p-Werte werden zu einem $\alpha$-Niveau von 5\% verglichen.
\item [\textbf{E} \msquare] Nach der Bonferroni-Adjustierung ergeben sich die adjustierten p-Werte von 0.2225, 3e-04, 0.085 und 0.005. Die adjustierten p-Werte werden zu einem $\alpha$-Niveau von 1.25\% verglichen.
\end{enumerate}

\section{Aufgabe \hfill (2 Punkte)}



Sie rechnen einen PostHoc-Test. Nun sollen Sie ein \textit{CLD} erstellen. Was bedeutet dieser Fachbegriff und welche folgende Beschreibung der Interpretation ist korrekt?



\begin{enumerate}
\item [\textbf{A} \msquare] Compact letter display. Das CLD ist umstritten, da es die Gleichheit der Behandlungen durch gleiche Buchstaben darstellt. Dadurch ist das CLD nicht mehr sauber auf einer Linie mit dem statistischen Testen. Wir lehnen die Nullhypothese ab und zeigen keine Gleichheit im statistischen Testen.
\item [\textbf{B} \msquare] Compound letter display. Gleichheit in dem Outcomes wird durch den gleichen Buchstaben oder Symbol dargestellt. Teilweise ist die Interpretation des Verbunds (eng. compound) herausfordernd, da wir ja nach dem Unterschied suchen.
\item [\textbf{C} \msquare] Compact letter detection. Gleichheit in den Behandlungen wird durch den gleichen Buchstaben oder Symbol dargestellt.
\item [\textbf{D} \msquare] Compact line display. Gleichheit in den Behandlungen wird durch den gleichen Buchstaben oder Symbol dargestellt. Früher wurden keine Buchstaben sondern eine durchgezogene Linie verwendet. Bei mehr als drei Gruppen funktioniert die Linie aber graphisch nicht mehr.
\item [\textbf{E} \msquare] Contrast letter display. Unterschiede in den Behandlungen werden durch den gleichen Buchstaben oder Symbol dargestellt. Die Interpretation des CLD führt häufig in die Irre.
\end{enumerate}

\section{Aufgabe \hfill (2 Punkte)}




Der multiple Vergleich als Posthoc-Test nach einer ANOVA ist in den Agrarwissenschaften heutzutage Standard. Welches R Paket wird häufig für den multiplen Vergleich genutzt? Welche Beschreibung der Eigenschaften ist korrekt?



\begin{enumerate}
\item [\textbf{A} \msquare] Das R Paket \{ggplot\}. Wir erhalten hier sofort eine Visualisierung der Daten. Anhand der Visualisierung lässt sich eine explorative Datenanalyse durchführen, die gleichwertig zu einem Posthoc-Test ist.
\item [\textbf{B} \msquare] Da Sie für Ihre Bachelorarbeit einen Barplot mit CLD brauchen nutzen Sie das R Paket \{emmeans\} welches Ihnen schnell die notwenidigen Informationen liefert um einen Barplot zu erstelen. Die Berechnung eines CLD ist hierbei auch einfach.
\item [\textbf{C} \msquare] Das R Paket \{hmisc\} erlaubt die Durchführung eines multiplen Gruppenvergleichs aus verschiedenen Modellen heraus. Aus einem hmisc Objekt lässt sich recht einfach das CLD erstellen und so über Barplots eine schnelle Interpration der statistischen Auswertung durchführen.
\item [\textbf{D} \msquare] Das R Paket \{emmeans\} erlaubt die Durchführung eines multiplen Gruppenvergleichs. Aus einem emmeans Objekt lässt sich leider kein CLD erstellen. Dennoch ist das Paket einfach zu bedienen und wird deshalb genutzt. Die Interpretation der statistischen Auswertung wird über einen Barplot abgebildet.
\item [\textbf{E} \msquare] Das R Paket \{lm\}. Das Paket \{lm\} erstellt selbstständig Konfidenzintervalle und entsprechende p-Werte. Da wir in dem Paket nicht adjustieren müssen, ist es bei Anwendern sehr beliebt.
\end{enumerate}

\section{Aufgabe \hfill (2 Punkte)}



In den Humanwissenschaften werden multiple Vergleiche häufig anders behandelt als in den Agrarwissenschaften. In beiden Bereichen tritt jedoch das gleiche Phänomen bei multiplen Testen auf. Wie muss mit dem Phänomen umgegangen werden und wie ist es benannt?



\begin{enumerate}
\item [\textbf{A} \msquare] Die Adjustierung der p-Werte nach Bonferroni erlaubt es gegen die $\beta$-Inflation vorzugehen, die häufig beim multiplen Testen auftritt. Das globale Powerniveau liegt nicht mehr bei $80\%$ sondern sehr viel niedriger.
\item [\textbf{B} \msquare] Beim multiplen Testen kann es zu einer $\alpha$-Inflation kommen. Das globale Signifikanzniveau liegt nicht mehr bei $5\%$ sondern sehr viel höher. Daher müssen die p-Werte entsprechend adjustiert werden. Hierfür gibt es verschiedene Verfahren, wobei das Verfahren zur Adjustierung der p-Werte nach Bonferroni das bekanneste Verfahren ist.
\item [\textbf{C} \msquare] Das globale Signifikanzniveau liegt nicht mehr bei $5\%$ sondern sehr viel niedriger, bei ca. $1\%$. Es kommt zu einer $\alpha$-Hyperinflation. Dagegen kann mit der Adjustierung der p-Werte nach Bonferroni vorgegangen werden.
\item [\textbf{D} \msquare] Beim multiplen Testen kann es zu einer $\alpha$-Inflation kommen. Das globale Signifikanzniveau liegt nicht mehr bei $5\%$ sondern weit darunter. Daher müssen die p-Werte entsprechend adjustiert werden. Hierfür gibt es verschiedene Verfahren, wobei das Verfahren zur Adjustierung der p-Werte nach Welch das bekanneste Verfahren ist.
\item [\textbf{E} \msquare] Beim multiplen Testen kann es zu einer $\beta$-Inflation kommen. Das globale Signifikanzniveau liegt nicht mehr bei $20\%$. Daher müssen die p-Werte entsprechend adjustiert werden. Hierfür gibt es verschiedene Verfahren, wobei das Verfahren zur Adjustierung der p-Werte nach Bonferroni das bekanneste Verfahren ist.
\end{enumerate}

\section{Aufgabe \hfill (2 Punkte)}




In einem Feldversuch haben Sie einen Behandlungsfaktor mit mehreren Leveln vorliegen. Sie rechnen einen multiplen Vergleich. Vorher hatten Sie eine einfaktorielle ANOVA mit einem signifikanten Ergebnis vorliegen. Welche Aussage ist richtig?



\begin{enumerate}
\item [\textbf{A} \msquare] Beim multiplen Testen werden die Effekte der paarweisen Vergleiche ignoriert. Der Nachteil des multiplen Testens ist ja auch, dass wir am Ende keine Effekte mehr vorliegen haben. Eine ANOVA liefert hier bessere Informationen.
\item [\textbf{B} \msquare] Wenn ein multipler Test gerechnet wird, dann muss der Effekt $\Delta$ nicht adjustiert werden. Bei einem Effekt im multiplen Testen handelt es sich um eine Wahrscheinlichkeit für das Auftreten der Nullhypothese.
\item [\textbf{C} \msquare] Beim multiplen Testen muss der Effekt, wie der Mittelwertsunterschied $\Delta$ aus einem t-Test, nicht adjusiert werden.
\item [\textbf{D} \msquare] Beim multiplen Testen kann es zu einer Effektüberschätzung ($\Delta$-Inflation) kommen. Daher müssen die Effekte angepasst werden. Dies geschieht nicht händisch sondern intern in den angewendeten Algorithmen.
\item [\textbf{E} \msquare] Beim multiplen Testen kann es zu einer $\Delta$-Deflation kommen. Das globale Relevanzniveau liegt nicht mehr bei $5\%$ sondern weit darunter. Daher müssen die $\Delta$-Werte entsprechend adjustiert werden. Hierfür gibt es verschiedene Verfahren, wobei das Verfahren zur Adjustierung der $\Delta$-Werte nach Bonferroni das bekanneste Verfahren ist. Die $\Delta$-Werte werden durch die Anzahl an Vergleichen geteilt.
\end{enumerate}
\section*{Statistische Testtheorie}  

\section{Aufgabe \hfill (2 Punkte)}




Geben ist $Pr(D|H_0)$ als mathematischer Ausdruck, welche Aussage ist richtig?



\begin{enumerate}
\item [\textbf{A} \msquare] $Pr(D|H_0)$ ist die Wahrscheinlichkeit der Alternativehypothese und somit $1 - Pr(H_A)$
\item [\textbf{B} \msquare] $Pr(D|H_0)$ beschreibt die Wahrscheinlichkeit die Teststatistik $T_D$ aus den Daten $D$ zu beobachten, wenn die Nullhypothese wahr ist.
\item [\textbf{C} \msquare] $Pr(D|H_0)$ ist die Wahrscheinlichkeit nicht die Daten $D$ zu beobachten sondern die Nullhypothese, wenn diese wahr ist.
\item [\textbf{D} \msquare] Die Inverse der Wahrscheinlichkeit unter der die Nullhypothese nicht mehr die Alternativehypothese überdeckt.
\item [\textbf{E} \msquare] Die Wahrscheinlichkeit der Daten unter der Nullhypothese in der Grundgesamtheit.
\end{enumerate}

\section{Aufgabe \hfill (2 Punkte)}



Die Testtheorie hat mehrere Säulen. Einer der Säulen ist das Falsifikationsprinzip. Das Falsifikationsprinzip besagt,



\begin{enumerate}
\item [\textbf{A} \msquare] ... dass ein schlechtes Modell durch das Falsifikationsprinzip durch ein noch schlechteres Modell ersetzt wird. Die Wissenschaft lehnt ab und verifiziert nicht.
\item [\textbf{B} \msquare] ... dass ein schlechtes Modell durch ein schlechteres Modell ersetzt wird. Die Wissenschaft lehnt ab und verifiziert nicht.
\item [\textbf{C} \msquare] ... dass ein schlechtes Modell durch ein weniger schlechtes Modell ersetzt wird. Die Wissenschaft lehnt ab und verifiziert nicht.
\item [\textbf{D} \msquare] ... dass ein minderwertes Modell durch ein minderwertiges Modell ersetzt wird. Es gilt das Verifikationsprinzip nach Karl Popper.
\item [\textbf{E} \msquare] ... dass Annahmen an statistische Modelle meist falsch sind.
\end{enumerate}

\section{Aufgabe \hfill (2 Punkte)}



Der Fehler 1. Art oder auch Signifikanzniveau $\alpha$ genannt, liegt bei
5\%. Welcher der folgenden Gründe für diese Festlegeung auf 5\% als Signifikanzschwelle ist richtig?



\begin{enumerate}
\item [\textbf{A} \msquare] Da Wissenschaftler eine Schwelle für die statistische Testentscheidung benötigen wurde $\alpha$ historisch gewählt. Damit ist $\alpha = 5\%$ eine Kulturkonstante.
\item [\textbf{B} \msquare] Der Begründer der modernen Statistik, R. Fischer, hat die Grenze simuliert und berechnet. Dadurch ergibt sich dieser optimale Cut-Off.
\item [\textbf{C} \msquare] Da Wissenschaftler eine Schwelle für die statistische Testentscheidung benötigen wurde $\alpha$ in einer großen Konferenz 1945 gewählt. Damit ist $\alpha = 5\%$ eine Kulturkonstante mit einem Rank einer Naturkonstante.
\item [\textbf{D} \msquare] Der Wert ergab sich aus einer Auswertung von 1042 wissenschaftlichen Veröffentlichungen zwischen 1914 und 1948. Der Wert $5\%$ wurde in $28\%$ der Veröffentlichungen genutzt. Daher legte man sich auf diese Zahl fest.
\item [\textbf{E} \msquare] Im Rahmen eines langen Disputs zwischen Neyman und Fischer wurde $\alpha = 5\%$ festgelegt. Leider werden die Randbedingungen und Voraussetzungen an statistsiche Modelle heute immer wieder ignoriert.
\end{enumerate}

\section{Aufgabe \hfill (2 Punkte)}

Betrachten wir die Teststatistik aus einem abstrakteren Blickwinkel. Beim
statistischen Testen wird das \textit{"`signal"'} mit dem
\textit{"`noise"'} aus den Daten $D$ zu einer Teststatistik $T_D$ verrechnet. Welche der Formel
berechnet korrekt die Teststatistik $T_D$?



\begin{enumerate}
\item [\textbf{A} \msquare] Es gilt $T_D = (signal \cdot noise)^2$
\item [\textbf{B} \msquare] Es gilt $T_D = \cfrac{signal}{noise}$
\item [\textbf{C} \msquare] Es gilt $T_D = signal \cdot noise$
\item [\textbf{D} \msquare] Es gilt $T_D = \cfrac{signal}{noise^2}$
\item [\textbf{E} \msquare] Es gilt $T_D = \cfrac{noise}{signal}$
\end{enumerate}

%% ------------------------------------------------------------

\section{Aufgabe \hfill (2 Punkte)}



Eine Analogie kann helfen einen Sachverhalt besser zu verstehen. Wie kann folgende Aussage richtig in die Analogie der statistischen Testtheorie gesetzt werden?

\begin{center}
\textit{$H_0$ ablehnen obwohl die $H_0$ gilt}
\end{center}



\begin{enumerate}
\item [\textbf{A} \msquare] In die Analogie eines Rauchmelders: \textit{Alarm with fire}.
\item [\textbf{B} \msquare] In die Analogie eines Rauchmelders: \textit{Fire without alarm}, dem $\beta$-Fehler.
\item [\textbf{C} \msquare] Dem $\alpha$-Fehler in der Analogie eines Rauchmelder: \textit{Alarm without fire}.
\item [\textbf{D} \msquare] In die Analogie eines Feuerwehrautos: \textit{Car without noise}.
\item [\textbf{E} \msquare] In die Analogie eines brennenden Hauses ohne Rauchmelder: \textit{House without noise}.
\end{enumerate}

\section{Aufgabe \hfill (2 Punkte)}



Sie lesen eine wissenschaftliche Arbeit, die damit wirbt, dass Effekte und Signifikanz nicht separat dargestellt sind, sondern in einer statistischen Maßzahl zusammen. Welche Aussage ist richtig?



\begin{enumerate}
\item [\textbf{A} \msquare] Das OR. Als Chancenverhältnis gibt es das Verhältnis von Relevanz und Signifikanz wieder.
\item [\textbf{B} \msquare] Der p-Wert. Durch den Vergleich mit $\alpha$ lässt sich über die Signifikanz entscheiden und der $\beta$-Fehler erlaubt über die Power eine Einschätzung der Relevanz.
\item [\textbf{C} \msquare] Das $\Delta$. Durch die Effektstärke haben wir einen Wert für die Relevanz, die vom Anwender bewertet werden muss. Da $\Delta$ antiproportional zum p-Wert ist, bedeutet auch ein hohes $\Delta$ ein sehr kleinen p-Wert.
\item [\textbf{D} \msquare] Einem Konfidenzintervall. Das Konfidenzinterval bringt durch eine Visualisierung und drei Intervallgrenzen die Möglichkeit mit, eine Relevanzschwelle neben der Signifikanzschwelle und der $\alpha$-Schwelle zu definieren.
\item [\textbf{E} \msquare] Über das Konfidenzintervall. Das Konfidenzinterval beitet eine Entscheidung über die Signifikanz und zusätzlich kann über die Visualizierung des Konfidenzintervals eine Relevanzschwelle definiert werden.
\end{enumerate}

\section{Aufgabe \hfill (2 Punkte)}



Sie haben ein Signifikanzniveau $\alpha$ gleich 5\% vorliegen. Welche Aussage zusammen mit dem $p$-Wert ist richtig?



\begin{enumerate}
\item [\textbf{A} \msquare] Wir schauen, ob der $p$-Wert kleiner ist als das Signifikanzniveau $\alpha$ und vergleichen somit Wahrscheinlichkeiten. Die Wahrscheinlichkeiten werden als Flächen unter der Kurve der Teststaistik dargestellt, wenn die $H_0$ gilt.
\item [\textbf{B} \msquare] Wir vergleichen die Effekte des $p$-Wertes mit den Effekten der Signifikanzschwelle unter der Annahme der Nullhypothese. Dabei gilt, dass wir die Nullhypothese nur ablehnen können anhand des Falsifikationsprinzips.
\item [\textbf{C} \msquare] Wir machen eine Aussage über die indivduelle Wahrscheinlichkeit des Eintretens der Nullhypothese $H_0$. Der $p$-Wert wird mit dem Signifikanzniveau verglichen und bewertet.
\item [\textbf{D} \msquare] Wir vergleichen mit dem $p$-Wert und dem Signifikanzniveau $\alpha$ Wahrscheinlichkeiten und damit die absoluten Werte auf einem Zahlenstrahl, wenn die $H_0$ gilt.
\item [\textbf{E} \msquare] Wir vergleichen mit dem $p$-Wert und dem Signifikanzniveau $\alpha$ absolute Werte auf einem Zahlenstrahl und damit den Unterschied der Teststatistiken, wenn die $H_0$ gilt.
\end{enumerate}

\section{Aufgabe \hfill (2 Punkte)}



Um die Ergebnisse eines statistischen Tests und die damit verbundene Theorie besser zu verstehen, kann eine Analogie zur Wettervorhersage genutzt werden. Welche Analogie zu der Testtheorie trifft am meisten zu?



\begin{enumerate}
\item [\textbf{A} \msquare] In der Analogie der Durchschnittstemperatur: Wie oft tritt ein Effekt durchschnittlich ein? Wir erhalten eine Wahrscheinlichkeit für die Effekte. Zum Beispiel, wie hoch ist die Wahrscheinlichkeit für einen Mittelwert als Durchschnitt.
\item [\textbf{B} \msquare] In der Analogie der Maximaltemperatur: Was ist der maximale Unterschied zwischen zwei Gruppen. Wir erhalten hier eine Aussage über die Spannweite und den maximalen Effekt.
\item [\textbf{C} \msquare] In der Analogie der Sonnenscheindauer: Wie lange kann mit einem entsprechenden Effekt gerechnet werden? Die Wahrscheinlichkeit für den Effekt gibt der statistische Test wieder.
\item [\textbf{D} \msquare] In der Analogie der Regenwahrscheinlichkeit in einem bestimmten Gebiet: ein statistischer Test gibt die Wahrscheinlichkeit für ein Ereignis in einem Experiment mit den Daten $D$ wieder und lässt sich kaum verallgemeinern.
\item [\textbf{E} \msquare] In der Analogie der Wahrscheinlichkeit für Regen: ein statistischer Test erlaubt die Wahrscheinlichkeit für ein Ereignis abzuschätzen. Die Stärke des Effektes können wir nicht bestimmen.
\end{enumerate}

\section{Aufgabe \hfill (2 Punkte)}



Sie wollen eine Aussage über ein untersuchtes Individuum treffen. Dazu nutzen Sie einen statistischen Test. Können Sie eine valide Aussage aus einem statistischen Test erhalten?



\begin{enumerate}
\item [\textbf{A} \msquare] Weder eine Ausssage über die Population noch über das Individuum ist mit einem statistischen Test möglich. Wir erhalten eine Aussage über ein Experiment.
\item [\textbf{B} \msquare] Ja, wir können ein untersuchtes Individuum nicht mit einem t-Test auswerten. Wir erhalten keine Aussage zum Individuum. Wir können aber den Effekt als Quelle der Relevanz nutzen.
\item [\textbf{C} \msquare] Ja, wir können ein untersuchtes Individuum mit einer ANOVA auswerten. Wir erhalten eine Aussage zum Individuum.
\item [\textbf{D} \msquare] Nein, ein untersuchtes Individuum können wir mit einem statistischen Test nicht auswerten. Wir erhalten keine Aussage zum Individuum.
\item [\textbf{E} \msquare] Nein, wir können ein untersuchtes Individuum nicht mit einer ANOVA auswerten. Wir erhalten keine Aussage zum Individuum. Wir können aber den Test adjustieren und so die Auswertung ermöglichen.
\end{enumerate}

\section{Aufgabe \hfill (2 Punkte)}



In der statistischen Testtheorie gibt es den Begriff \textit{Power}. Was sagt der statistische Begriff \textit{Power} aus?



\begin{enumerate}
\item [\textbf{A} \msquare] Die Power $1-\beta$ wird auf 80\% gesetzt. Damit liegt die Wahrscheinlichkeit für die $H_0$ bei 20\%.
\item [\textbf{B} \msquare] Die Power $1-\beta$ wird auf 80\% gesetzt. Alle statistischen Tests sind so konstruiert, dass die $H_A$ mit 80\% \textit{bewiesen wird}.
\item [\textbf{C} \msquare] Die Power ist nicht in der aktuellen Testthorie mehr vertreten. Wir rechnen nur noch mit dem Fehler 1. Art.
\item [\textbf{D} \msquare] Alle statistischen Tests sind so konstruiert, dass die $H_A$ mit 20\% \textit{bewiesen wird}. Die Power ist $1-\beta$ mit $\beta$ gleich 80\% gesetzt.
\item [\textbf{E} \msquare] Es gilt $\alpha + \beta = 1$ und somit liegt $\beta$ meist bei 95\%.
\end{enumerate}

\section{Aufgabe \hfill (2 Punkte)}



Sie rechnen einen statistischen Test und erhalten neben dem p-Wert noch einen Effekt wiedergegeben. Welche Aussage zum Effekt ist richtig?



\begin{enumerate}
\item [\textbf{A} \msquare] Durch den Effekt erfahren wir die biologisch interpretierbare Ausgabe eines statistischen Tests. Zum Beispiel das $\eta^2$ aus einer ANOVA. Damit können wir die Relevanz direkt mit dem Effekt verbinden. Am Ende muss der Forschende aber entscheiden, ob der Effekt entsprechend seinen Erwartungen als bedeutet zu bewerten ist.
\item [\textbf{B} \msquare] Der Effekt eines statistischen Tests beschreibt die mathematisch interpretierbare Ausgabe eines Tests. Damit ist der Effekt direkt mit dem Begriff der Signifikanz verbunden. Die Entscheidung über die Signifikanz trifft der Forschende unabhängig von der Relevanz eines statistsichen Tests.
\item [\textbf{C} \msquare] Der Effekt eines statistischen Tests beschreibt die biologisch interpretierbare Ausgabe eines Tests. Moderen Algorithmen liefern keine Effekte mehr sondern nur noch bedingte Wahrscheinlichkeiten. Der Effekt spielt in der modernen Statistik keine Rollen mehr.
\item [\textbf{D} \msquare] Der Effekt eines statistischen Tests beschreibt die biologisch interpretierbare Ausgabe eines Tests. Damit ist der Effekt direkt mit dem Begriff der Signifikanz verbunden. Die Entscheidung über die Signifikanz trifft der Forschende unabhängig von der Relevanz eines statistsichen Tests.
\item [\textbf{E} \msquare] Der Effekt eines statistischen Tests beschreibt den Output oder die Wiedergabe eines Tests in einem Computer.
\end{enumerate}

\section{Aufgabe \hfill (2 Punkte)}



Welche Aussage über die Entscheidung anhand der berechneten Teststatistik gegen die
Nullhypothese ist richtig?



\begin{enumerate}
\item [\textbf{A} \msquare] Anhand der berechneten Teststatistik lässt sich wie folgt eine Entscheidung treffen. Liegt der Wert über oder gleich dem Signifikanzniveau $\alpha$ dann kann die Nullhypothese abgelehnt werden.
\item [\textbf{B} \msquare] Ist $Pr(D|H_0)$ kleiner als das Signifikanzniveau $\alpha$ gleich $5\%$ dann wird die Nullhypothese $H_0$ abgelehnt.
\item [\textbf{C} \msquare] Anhand der berechneten Teststatistik lässt sich wie folgt eine Entscheidung treffen. Liegt der Wert in dem Signifikanzniveauintervall $\alpha$ dann kann die Nullhypothese abgelehnt werden.
\item [\textbf{D} \msquare] Ist in dem 95\%-Konfidenzintervall nicht die Null enthalten dann wird die Nullhypothese $H_0$ abgelehnt.
\item [\textbf{E} \msquare] Ist $T_{D}$ h{"o}her als der kritische Wert $T_{\alpha = 5\%}$ dann wird die Nullhypothese $H_0$ abgelehnt.
\end{enumerate}

\section{Aufgabe \hfill (2 Punkte)}



In Ihrer Abschlussarbeit müssen Sie für die statistischen Tests im Anhang Ihrer Arbeit die Hypothesen $H$ formulieren. Welche Aussage über Hypothesen $H$ ist richtig



\begin{enumerate}
\item [\textbf{A} \msquare] Mit der Nullhypothese $H_A$ und der Alternativehypothese $H_0$ gibt es zwei Hypothesen, die aber selten genutzt werden.
\item [\textbf{B} \msquare] Ein statistisches Hypothesenpaare gibt es. Zum einen die Nullhypothese und zum anderen die Alternativehypothese. Es ist aber nur notwendig die Alternative anzugeben, da die Nullhypothese nicht beim Testen benötigt wird.
\item [\textbf{C} \msquare] Es gibt ein statistisches Hypothesenpaar mit der Hypothese für und gegen die wissenschaftliche Fragestellung. Die Hypothesen werden $H_{pro}$ und $H_{contra}$ bezeichnet.
\item [\textbf{D} \msquare] Es gibt ein Hypothesenset bestehend aus $k$ Hypothesen. Meistens wird die Nullhypothese $H_0$ und die Alternativhypothese $H_A$ verwendet. Wegen des Falsifikationsprinzips ist es wichtig, die bekannte falsche und unbekannte richtige Hypothese mit in das Set zu nehmen.
\item [\textbf{E} \msquare] Ein statistisches Hypothesenpaare gibt es. Zum einen die Nullhypothese $H_0$ und zum anderen die Alternativehypothese $H_A$ oder $H_1$
\end{enumerate}
\section*{Statistische Tests für Gruppenvergleiche} 

\section{Aufgabe \hfill (2 Punkte)}



Nach einem Feldexperiment wollen Sie zwei Gruppen mit einem Welch t-Test vergleichen. Welche Aussage ist auch für den Student t-Test richtig?



\begin{enumerate}
\item [\textbf{A} \msquare] Der t-Test vergleicht zwei oder mehr Gruppen indem die Mittelwerte miteinander verglichen werden.
\item [\textbf{B} \msquare] Der t-Test vergleicht die Varianzen von mindestens zwei oder mehr Gruppen
\item [\textbf{C} \msquare] Der t-Test vergleicht zwei Gruppen indem die Mittelwerte miteinander verglichen werden.
\item [\textbf{D} \msquare] Der t-Test testet generell zu einem erhöhten $\alpha$-Niveau von 20\%.
\item [\textbf{E} \msquare] Der t-Test ist ein Vortest der ANOVA und basiert daher auf dem Vergleich von Streuungsparametern
\end{enumerate}

\section{Aufgabe \hfill (2 Punkte)}



Ein Versuch wurde in 11 Parzellen pro Gruppe durchgeführt. Die folgende Abbildung enthält die Daten aus diesem Versuch zur Bewertung der Wirkung des Mikronährstoff Eisen auf den Ertrag in t/ha von Weizen im Vergleich zu einer Kontrolle.  Welche Aussage ist richtig, wenn Sie einen t-Test rechnen?



{\centering \includegraphics[width=\maxwidth]{img/mc-testing-ttest-02-1} 

}







\begin{enumerate}
\item [\textbf{A} \msquare] Es liegt ein signifikanter Unterschied vor. Der Effekt liegt bei 0.2.
\item [\textbf{B} \msquare] Es liegt ein signifikanter Unterschied vor. Der Effekt liegt bei 2.
\item [\textbf{C} \msquare] Die Barplots deuten auf einen signifikanten Unterschied. Der Effekt liegt vermutlich bei 2. Wir müssen aber einen Posthoc-Test rechnen um den Effekt wirklich bestimmen zu können.
\item [\textbf{D} \msquare] Die Barplots deuten auf keinen signifikanten Unterschied. Der Effekt liegt vermutlich bei 2 unter einer groben Abschätzung. Wir müssen aber eine ANOVA rechnen um den Effekt wirklich bestimmen zu können.
\item [\textbf{E} \msquare] Der Test deutet auf keinen signifikanten Unterschied hin. Der Effekt liegt vermutlich bei 2.
\end{enumerate}

\section{Aufgabe \hfill (2 Punkte)}




Sie rechnen einen gepaarten t-Test, da Ihre Beobachtungen verbunden sind. Welche der folgenden Aussagen ist richtig?



\begin{enumerate}
\item [\textbf{A} \msquare] Beim gepaarten t-Test kombinieren wir die Vorteile des Student t-Test für Varianzhomogenität mit den Vorteilen des Welch t-Test für Varianzheterogenität. Wir bilden dafür die Differenz der Einzelbeobachtungen.
\item [\textbf{B} \msquare] Abhängige Beobachtungen müssen gesondert in einem gepaarten t-Test modelliert werden. Wenn wiederholt an dem gleichen Tier oder Pflanze gemessen wird, dann bilden wir den Quotienten zwischen den beiden Zeitpunkten. Auf den Quotienten rechnen wir den gepaarten t-Test.
\item [\textbf{C} \msquare] Wenn die Beobachtungen unabhängig voneinander sind, rechnen wir einen gepaarten t-Test. Messen wir wiederholt an dem gleichen Tier oder Pflanze dann bilden wir das Produkt zwischen den zwei Messpunkten.
\item [\textbf{D} \msquare] Abhängige Beobachtungen müssen gesondert in einem gepaarten t-Test modelliert werden. Wenn wiederholt an dem gleichen Tier oder Pflanze gemessen wird, dann bilden wir die Differenz zwischen den beiden Zeitpunkten. Auf den Differenzen rechnen wir den gepaarten t-Test.
\item [\textbf{E} \msquare] Der gepaarte t-Test wird gerechnet, wenn die Beobachtungen abhängig voneinander sind. Wir messen jede Beobachtung nur einmal und berechnen dann die Differenz zu dem Mittel der anderen Beobachtungen.
\end{enumerate}

\section{Aufgabe \hfill (2 Punkte)}



Sie führen paarweise t-Tests für alle Vergleiche der verschiedenen Rapssorten in Ihrem Experiment durch. Nach der Adjustierung für multiples Testen ist kein p-Wert unter der $\alpha$-Schwelle. Ihr Experiment beinhaltet fünf Rapssorten und eine ANOVA ergibt $p = 0.049$ für den Ertrag. Sie schauen sich auch die rohen, unadjustierten p-Werte an und finden hier als niedrigsten p-Wert $p_{3-2} = 0.053$. Welche Aussage ist richtig?




\begin{enumerate}
\item [\textbf{A} \msquare] Das ist kein Wunder. Die ANOVA testet auf der gesamten Fallzahl und die paarweisen t-Tests verlieren immer eine oder mehr Gruppen als Fallzahl. Mit steigender Fallzahl sind mehr signifikante Unterschiede zu erwarten. Die p-Werte unterscheiden sich numerisch auch kaum.
\item [\textbf{B} \msquare] Es gibt einen Fehler in der Varianzstruktur. Daher kann die ANOVA nicht richtig sein und paarweise t-Tests liefern das richtige Ergebnis.
\item [\textbf{C} \msquare] Hier kommt der Effekt der stiegenden Fallzahl auf die Anzahl an signifikante Ergebnisse zu tragen. Da die ANOVA auf weniger Fallzahl testet als die paarweisen t-Tests, kann die ANOVA schwerer einen signifikanten Unterscheid nachweisen.
\item [\textbf{D} \msquare] Das ist kein Wunder. Die ANOVA testet nicht auf der gesamten Fallzahl und die paarweisen t-Tests gewinnen immer eine oder mehr Gruppen als Fallzahl dazu. Mit steigender Fallzahl sind mehr signifikante Unterschiede zu erwarten. Die p-Werte unterscheiden sich numerisch auch kaum.
\item [\textbf{E} \msquare] Die ANOVA testet auf der gesamten Fallzahl. Es wäre besser die ANOVA auf der gleichen Fallzahl wie die einzelnen t-Tests zu rechnen.
\end{enumerate}
    
% -----------------------------------------------------------------------
\clearpage
% -----------------------------------------------------------------------
\part{Deskriptive Statistik \& Explorative Datenanalyse}
% -----------------------------------------------------------------------

\section{Aufgabe \hfill (8 Punkte)}

\textit{Geben Sie grundsätzlich Formeln und Rechenweg zur Lösung der Teilaufgaben mit an!} \\[1Ex]
 

 
%% --------------------------------------------------------------------
\begin{minipage}[t]{0.5\textwidth}
\includegraphics[width = 1.3cm]{/Users/kruppajo/work/GitHub/exam/avatare/Mark.png}
\end{minipage}
\begin{minipage}[t]{0.5\textwidth}
\hfill
\href{https://youtu.be/t0WYa_LVc5o}{\includegraphics[width = 2cm]{img/youtube}}
\end{minipage}
\vspace{-3ex}
%% --------------------------------------------------------------------



\paragraph{Zerforschen des Barplots}

Wenn die Unsicherheit nicht wäre, ja dann wäre wohl vieles möglich für Mark! Aber so.. Deshalb gilt anschauen, was andere vor einem gemacht haben. Für Mark ist es eine Möglichkeit schneller ans Ziel zu gelangen. Mark soll in seiner Abschlussarbeit Erdbeeren untersuchen. Die Behandlung in seiner Abschlussarbeit werden verschiedene Düngestufen ($ctrl$, $low$ und $high$) sein. Erheben wird Mark als Messwert ($Y$) \textit{Proteingehalt} benannt als \textit{protein} in seiner Exceldatei. Von seiner Betreuerin erhält er nun folgende Abbildung von Barplots, die er erstmal zur Übung nachbauen soll, bevor er mit dem eigentlichen Versuch beginnt.



{\centering \includegraphics[width=\maxwidth]{img/barplot-02-1} 

}




Leider kennt sich Mark mit der Erstellung von Barplots in \Rlogo nicht aus. Deshalb braucht er bei der Visualisierung Ihre Hilfe!

\begin{enumerate}
\item Formulieren Sie die wissenschaftliche Fragestellung! \textbf{(1 Punkt)}
\item Erstellen Sie eine Tabelle mit den statistischen Maßzahlen der drei Barplots! \textit{Beachten Sie die korrekte Darstellungsform der statistischen Maßzahlen!} \textbf{(3 Punkte)}
\item Erstellen Sie einen beispielhaften Datensatz im \Rlogo üblichen Format, aus dem die drei Barplots \textit{möglicherweise} erstellt wurden! \textbf{(2 Punkte)}
\item Kann Mark einen Unterschied zwischen den Behandlungen erwarten? Begründen Sie Ihre Antwort! \textbf{(2 Punkte)}
\end{enumerate} 
\clearpage
% -----------------------------------------------------------------------

\section{Aufgabe \hfill (8 Punkte)}

\textit{Geben Sie grundsätzlich Formeln und Rechenweg zur Lösung der Teilaufgaben mit an!} \\[1Ex]
 

 
%% --------------------------------------------------------------------
\begin{minipage}[t]{0.5\textwidth}
\includegraphics[width = 1.3cm]{/Users/kruppajo/work/GitHub/exam/avatare/Alex.png}
\end{minipage}
\begin{minipage}[t]{0.5\textwidth}
\hfill
\href{https://youtu.be/vXnLttRL_VI}{\includegraphics[width = 2cm]{img/youtube}}
\end{minipage}
\vspace{-3ex}
%% --------------------------------------------------------------------



\paragraph{Visualisierung des Barplots}


Barplots sind bedeutend in der Darstellung von wissenschaftlichen Ergebnissen. Leider hat sich Alex nicht gemerkt, welche statistischen Maßzahlen für einen Barplot erhoben werden müssen. Besser wäre was anderes gewesen. Starcraft. Ein wunderbares Hobby um sich drin zu verlieren und Abstand zu bekommen. Alex denkt gerne über Starcraft nach. Das ist in soweit doof, da nach seinem Betreuer nun Barplots aus seinen Daten gebaut werden sollen, bevor es mit dem statistischen Testen weitergeht. Na dann mal los. Alex schafft sich die nötige Stimmung. Alex streichelt liebevoll die Katze. Der Kopf ist in seinem Schloß vergraben um den Klang von Abba zu dämpfen. Die Behandlung für Brokoli waren verschiedene Lichtstufen ($none$, $200lm$ und $600lm$). Erfasst wurde von Alex als Messwert ($Y$) \textit{Proteingehalt}. Alex hat dann \textit{protein} in seiner Exceldatei eintragen. Aber auch irgendwie egal. Alex will später nochmal raus um zu Laufen. Druck ablassen, dass muss er auch.

\begin{table}[!h]
\centering
\begin{tabular}{cc}
\toprule
treatment & protein\\
\midrule
200lm & 33.3\\
600lm & 34.7\\
200lm & 29.0\\
none & 44.3\\
600lm & 13.5\\
\addlinespace
200lm & 12.3\\
600lm & 17.8\\
none & 39.5\\
200lm & 30.2\\
200lm & 31.2\\
\addlinespace
none & 35.1\\
none & 47.5\\
\bottomrule
\end{tabular}
\end{table}



Leider kennt sich Alex mit der Erstellung von Barplots nicht aus. Deshalb braucht er bei der Visualisierung Ihre Hilfe!

\begin{enumerate}
\item Formulieren Sie die wissenschaftliche Fragestellung! \textbf{(1 Punkt)}
\item Zeichnen Sie in \textit{einer} Abbildung die Barplots für die Behandlung von Brokoli! Beschriften Sie die Achsen entsprechend!\textbf{(4 Punkte)}
\item Beschriften Sie \textit{einen} Barplot mit den gängigen statistischen Maßzahlen! \textbf{(2 Punkte)}
\item Wenn Alex \textit{keinen Effekt} zwischen den Behandlungen von Brokoli erwarten würde, wie sehen dann die Barplots aus? \textit{Antworten Sie mit einer Skizze der Barplots!}
  \textbf{(1 Punkt)}
\end{enumerate} 
\clearpage
% -----------------------------------------------------------------------

\section{Aufgabe \hfill (9 Punkte)}

\textit{Geben Sie grundsätzlich Formeln und Rechenweg zur Lösung der Teilaufgaben mit an!} \\[1Ex]
 

 
%% --------------------------------------------------------------------
\begin{minipage}[t]{0.5\textwidth}
\includegraphics[width = 1.3cm]{/Users/kruppajo/work/GitHub/exam/avatare/Paula.png}
\end{minipage}
\begin{minipage}[t]{0.5\textwidth}
\hfill
\href{https://youtu.be/Xf0yE-o7bEU}{\includegraphics[width = 2cm]{img/youtube}}
\end{minipage}
\vspace{-3ex}
%% --------------------------------------------------------------------



\paragraph{Zerforschen des Boxplots}

Boxplots sind bedeutend in der Darstellung von wissenschaftlichen Ergebnissen. Leider hat sich Paula nicht gemerkt, welche statistischen Maßzahlen für einen Barplot erhoben werden müssen. Besser wäre was anderes gewesen. Harry Potter. Ein wunderbares Hobby um sich drin zu verlieren und Abstand zu bekommen. Paula denkt gerne über Harry Potter nach. Das ist in soweit doof, da nach ihrer Betreuerin erstmal ein Barplot nachgebaut werden soll, bevor es mit ihrem Projektbericht losgeht. Dann hat sie schonmal den \Rlogo Code vorliegen und nachher geht dann alles schneller. Na dann mal los. Paula schafft sich die nötige Stimmung. Paula streichelt liebevoll die Ratte. Der Kopf ist in ihrem Schloß vergraben um den Klang von White Lies zu dämpfen. In der Behandlung für Lauch werden verschiedene Lüftungssystemen und Folientunneln ($ctrl$, $storm$ und $tornado$) sein. Erfasst wird als Messwert ($Y$) \textit{Frischegewicht}. Paula soll dann \textit{freshmatter} in ihrer Exceldatei eintragen. Aber nur in passender Atmospäre! Auf seinem Second Screen läuft Jagd auf roter Oktober und Paula schaufelt Smarties. Nicht effizient, aber gut.



{\centering \includegraphics[width=\maxwidth]{img/boxplot-02-zer-1} 

}




Leider kennt sich Paula mit der Erstellung von Boxplots in \Rlogo nicht aus. Deshalb braucht sie bei der Visualisierung Ihre Hilfe!

\begin{enumerate}
\item Erstellen Sie eine Tabelle mit den statistischen Maßzahlen aus der obigen Abbildung der drei Boxplots! \textit{Beachten Sie die korrekte Darstellungsform der statistischen Maßzahlen!} \textbf{(3 Punkte)}
\item Beschriften Sie \textit{einen} der Boxplots mit den gängigen statistischen Maßzahlen! \textbf{(2 Punkte)}
\item Erstellen Sie einen beispielhaften Datensatz, aus dem die drei Boxplots \textit{möglicherweise} erstellt wurden, im \Rlogo üblichen Format! \textbf{(2 Punkte)}
\item Kann Paula einen Unterschied zwischen den Behandlungen erwarten? Begründen Sie Ihre Antwort! \textbf{(2 Punkte)}
\end{enumerate} 
\clearpage
% -----------------------------------------------------------------------

\section{Aufgabe \hfill (9 Punkte)}

\textit{Geben Sie grundsätzlich Formeln und Rechenweg zur Lösung der Teilaufgaben mit an!} \\[1Ex]
 

 
%% --------------------------------------------------------------------
\begin{minipage}[t]{0.5\textwidth}
\includegraphics[width = 1.3cm]{/Users/kruppajo/work/GitHub/exam/avatare/Yuki.png}
\end{minipage}
\begin{minipage}[t]{0.5\textwidth}
\hfill
\href{https://youtu.be/0xc0jIPeiyw}{\includegraphics[width = 2cm]{img/youtube}}
\end{minipage}
\vspace{-3ex}
%% --------------------------------------------------------------------



\paragraph{Visualisierung des Boxplots}

Yuki und die Faulheit, eine unendliche Geschichte mit kniffeligen Wendungen. Deshalb gilt anschauen, was andere vor einem gemacht haben. Für Yuki ist es eine Möglichkeit schneller ans Ziel zu gelangen. Deshalb hat sich Yuki viele Poster in der Fakultät angeschaut und ist zum Schluß gekommen, dass Boxplots eine häufig genutzte Abbildung sind. Yuki soll nun in ihrem Projektbericht Erbsen untersuchen. Die Behandlung in ihrem Projektbericht sind verschiedene Lichtstufen ($none$ und $600lm$). Erhoben wurden von Yuki als Messwert ($Y$) \textit{Frischegewicht} benannt als \textit{freshmatter} in ihrer Exceldatei. Erwartungsgemäß erhält sie von ihrer Betreuerin den Auftrag die erhobenen Daten als Boxplots darzustellen. Dann kann Yuki auch schonmal abschätzen, was bei einem statistischen Test rauskommen könnte. Darüber hinaus kann Yuki anhand Boxplots eine Aussage über die Varianzhomogenität über die Behandlungsgruppen treffen. Na dann mal los. Yuki schafft sich die nötige Stimmung. Wenn London Grammar ertönt, dann sucht das Minischwein schleunigst Schutz unter dem Sofa. Yuki schüttelt den Kopf.

\begin{table}[!h]
\centering
\begin{tabular}{cc}
\toprule
treatment & drymatter\\
\midrule
600lm & 30.9\\
none & 27.7\\
600lm & 22.1\\
600lm & 40.8\\
none & 33.7\\
\addlinespace
600lm & 21.9\\
600lm & 28.7\\
none & 31.2\\
none & 46.5\\
600lm & 16.2\\
\addlinespace
none & 36.5\\
none & 50.2\\
600lm & 7.2\\
600lm & 18.4\\
none & 48.2\\
\addlinespace
none & 40.0\\
600lm & 33.5\\
\bottomrule
\end{tabular}
\end{table}



Leider kennt sich Yuki mit der Erstellung von Boxplots nicht aus. Deshalb braucht sie bei der Visualisierung Ihre Hilfe!

\begin{enumerate}
\item Zeichnen Sie in \textit{einer} Abbildung die beiden Boxplots für die zwei Behandlungen von Erbsen! Beschriften Sie die Achsen entsprechend! \textbf{(5 Punkte)} 
\item Wie ist Ihr Vorgehen, wenn Sie eine \textit{gerade} Anzahl an
  Beobachtungen pro Gruppe haben? \textbf{(1 Punkt)}
\item Beschriften Sie \textit{einen} der beiden Boxplots mit den gängigen
  statistischen Maßzahlen! \textbf{(2 Punkte)}
\item Wenn Sie \textit{keinen Effekt} zwischen den Behandlungen von
  Erbsen erwarten würden, wie sehen dann die beiden Boxplots aus?
  \textit{Antworten Sie mit einer Skizze der Boxplots!}
  \textbf{(1 Punkt)}
\end{enumerate} 
\clearpage
% -----------------------------------------------------------------------

\section{Aufgabe \hfill (8 Punkte)}

\textit{Geben Sie grundsätzlich Formeln und Rechenweg zur Lösung der Teilaufgaben mit an!} \\[1Ex]
 

 
%% --------------------------------------------------------------------
\begin{minipage}[t]{0.5\textwidth}
\includegraphics[width = 1.3cm]{/Users/kruppajo/work/GitHub/exam/avatare/Yuki.png}
\end{minipage}
\begin{minipage}[t]{0.5\textwidth}
\hfill
\href{https://youtu.be/aXvxGC4YLqk}{\includegraphics[width = 2cm]{img/youtube}}
\end{minipage}
\vspace{-3ex}
%% --------------------------------------------------------------------



\paragraph{Visualisierung des Histogramm für kategoriale Daten}

Yuki schmeißt noch eine Handvoll Reese's Peanut Butter Cups in ihren Rachen. Im Hintergrund klirrt leise der Spiegel zum Sound von London Grammar. Yuki betrachtet die folgenden Daten nach einem Stallexperiment mit Milchvieh. In dem Experiment wurden die dunklen Pigmentstörungen gezählt. Nach der Meinung ihrer Betreuerin muss als erstes geschaut werden, wie diese verteilt sind. Also welcher statistischen Verteilung die dunklen Pigmentstörungen folgen. Dazu soll Yuki ein Histogramm verwenden. Dann hätte man auch einen guten Überblick über das Outcome ($Y$). Es wäre einfacher, wenn da nicht noch was wäre. Wenn die Faulheit nicht wäre, ja dann wäre wohl vieles möglich für Yuki! Aber so.. Wenn London Grammar ertönt, dann sucht das Minischwein schleunigst Schutz unter dem Sofa. Yuki schüttelt den Kopf.

\begin{center}
Die dunklen Pigmentstörungen: 4, 6, 7, 1, 5, 2, 4, 2, 6, 4, 3, 1, 4, 1, 2, 1, 3, 5, 7, 5, 7, 5, 6, 3, 6, 6, 7, 3, 1, 5
\end{center}

Leider kennt sich Yuki mit der Erstellung von Histogrammen überhaupt nicht aus. Deshalb braucht sie bei der Erstellung Ihre Hilfe!

\begin{enumerate}
\item Zeichen Sie ein Histogramm um die Verteilung der Daten zu visualisieren! (\textbf{3 Punkte})
\item Beschriften Sie die Achsen der Abbildung! (\textbf{2 Punkte})
\item Ergänzen Sie die absoluten und relativen Häufigkeiten in der
  Abbildung! \textbf{(1 Punkt)}
\item Berechnen Sie aus den Daten die \textit{Wahrscheinlichkeit}
  mehr als die Anzahl 5 zu beobachten! \textbf{(1
    Punkt)}
\item Berechnen Sie aus den Daten die \textit{Chance} mehr
  als die Anzahl 5 zu beobachten! \textbf{(1 Punkt)}
\end{enumerate}

 
\clearpage
% -----------------------------------------------------------------------

\section{Aufgabe \hfill (8 Punkte)}

\textit{Geben Sie grundsätzlich Formeln und Rechenweg zur Lösung der Teilaufgaben mit an!} \\[1Ex]
 

 
%% --------------------------------------------------------------------
\begin{minipage}[t]{0.5\textwidth}
\includegraphics[width = 1.3cm]{/Users/kruppajo/work/GitHub/exam/avatare/Paula.png}
\end{minipage}
\begin{minipage}[t]{0.5\textwidth}
\hfill
\href{https://youtu.be/ORHSPTCdfeY}{\includegraphics[width = 2cm]{img/youtube}}
\end{minipage}
\vspace{-3ex}
%% --------------------------------------------------------------------



\paragraph{Visualisierung des Histogramm für kontinuierliche Daten}

In einem Gespräch mit ihrer Betreuerin wird Paula gebeten seine Daten aus einem Versuch in einer Klimakammer mit Kartoffeln in einem Histogramm darzustellen. 'Hm...', Smarties und White Lies. Das ist und bleibt die beste Kombination zum Nachdenken für Paula. In ihrem Experiment hat er die mittleren Läsionen auf den Blättern erst fotographiert und dann ausgezählt. Laut ihrer Betreuerin soll das Histogramm helfen, die Verteilung der die mittleren Läsionen auf den Blättern zu bestimmen.  Es wäre einfacher, wenn da nicht noch was wäre. Wenn der Perfektionismus nicht wäre, ja dann wäre wohl vieles möglich für Paula! Aber so.. Paula streichelt liebevoll die Ratte. Der Kopf ist in ihrem Schloß vergraben um den Klang von White Lies zu dämpfen.

\begin{center}
Die mittleren Läsionen auf den Blättern: 13.3, 11.6, 9.7, 11, 7.8, 9.6, 11, 8.4, 11.1, 12.5, 9.9, 11.8, 9.2, 9.6, 11.9, 9, 11.2, 9.6, 12.3, 15, 9.8, 12
\end{center}

Leider kennt sich Paula mit der Erstellung von Histogrammen überhaupt nicht aus. Deshalb braucht sie bei der Erstellung Ihre Hilfe!

\begin{enumerate}
\item Zeichen Sie ein Histogramm um die Verteilung der Daten zu visualisieren! (\textbf{3 Punkte})
 \item Erläutern Sie Ihr Vorgehen um ein Histogramm für kontinuierliche Daten zu zeichnen!  (\textbf{2 Punkte})
\item Beschriften Sie die Achsen der Abbildung! (\textbf{2 Punkte})
\item Ergänzen Sie die relativen Häufigkeiten in der Abbildung! \textbf{(1 Punkt)}  
\end{enumerate}

 
\clearpage
% -----------------------------------------------------------------------

\section{Aufgabe \hfill (10 Punkte)}

\textit{Geben Sie grundsätzlich Formeln und Rechenweg zur Lösung der Teilaufgaben mit an!} \\[1Ex]
 

 
%% --------------------------------------------------------------------
\begin{minipage}[t]{0.5\textwidth}
\includegraphics[width = 1.3cm]{/Users/kruppajo/work/GitHub/exam/avatare/Alex.png}
\end{minipage}
\begin{minipage}[t]{0.5\textwidth}
\hfill
\href{https://youtu.be/VAqiUdV4WQ0}{\includegraphics[width = 2cm]{img/youtube}}
\end{minipage}
\vspace{-3ex}
%% --------------------------------------------------------------------




\paragraph{Visualisierung des Scatterplots}

Alex liest laut: 'Wenn zwei kontinuierliche Variablen vorliegen, können diese in einer exploartiven Datenanalyse...'. Alex stoppt. Alex schmeißt noch eine Handvoll Gummibärchen in seinen Rachen. Im Hintergrund klirrt leise der Spiegel zum Sound von Abba. Was waren noch gleich kontinuierliche Variablen? In seiner Hausarbeit hatte er ein Freilandversuch in der Uckermark durchgeführt. Dabei ging es um den Zusammenhang zwischen Trockengewicht [kg/ha] und durchschnittlicher Regenwurmdichte [Anzahl/l] im groben Kontext von Spargel. Nun stellt sich die Frage für ihn, ob es überhaupt einen Zusammenhang zwischen den gemessenen Variablen gibt. Dafür war eine explorative Datenanalyse gut! Alex und die Gefälligkeit, eine unendliche Geschichte mit kniffeligen Wendungen. Dann was anderes. Irgendwie komisch, wenn er Alien anmacht, dann ist die Katze eigentlich sofort vor dem Bildschirm und starrt hinein.

\begin{table}[!h]
\centering
\begin{tabular}{cc}
\toprule
Durchschnittlicher Regenwurmdichte [Anzahl/l] & Trockengewicht [kg/ha]\\
\midrule
14.6 & 19.0\\
17.0 & 27.8\\
18.3 & 24.3\\
19.7 & 22.7\\
19.3 & 23.2\\
\addlinespace
11.2 & 14.6\\
16.7 & 25.0\\
14.4 & 21.9\\
15.1 & 19.1\\
14.9 & 20.0\\
\addlinespace
18.6 & 25.7\\
6.1 & 9.7\\
\bottomrule
\end{tabular}
\end{table}



Leider kennt sich Alex mit der Erstellung einer explorativen Datenanalyse für kontinuierliche Daten überhaupt nicht aus. Deshalb braucht er bei der Erstellung Ihre Hilfe!

\begin{enumerate}
\item Erstellen Sie eine Visualisierung für die Datentabelle. Beschriften Sie
  die Achsen entsprechend! \textbf{(4 Punkte)}
\item Schätzen Sie eine Gerade durch die Punkte! \textbf{(1 Punkt)}
\item Beschriften Sie die Gerade mit den gängigen statistischen Maßzahlen! Geben Sie die numerischen Zahlenwerte mit an! \textbf{(3 Punkte)}
\item Wenn \textit{kein} Effekt von $x$ auf $y$ vorhanden wäre, wie würde die Gerade verlaufen und welche Werte würden die statistischen Maßzahlen annehmen? \textbf{(2 Punkt)}
\end{enumerate} 
\clearpage
% -----------------------------------------------------------------------

\section{Aufgabe \hfill (10 Punkte)}

\textit{Geben Sie grundsätzlich Formeln und Rechenweg zur Lösung der Teilaufgaben mit an!} \\[1Ex]
 

 
%% --------------------------------------------------------------------
\begin{minipage}[t]{0.5\textwidth}
\includegraphics[width = 1.3cm]{/Users/kruppajo/work/GitHub/exam/avatare/Mark.png}
\end{minipage}
\begin{minipage}[t]{0.5\textwidth}
\hfill
\href{https://youtu.be/t_1KL77mfmg}{\includegraphics[width = 2cm]{img/youtube}}
\end{minipage}
\vspace{-3ex}
%% --------------------------------------------------------------------



\paragraph{Visualisierung des Mosaicplots}

Wenn Columbo läuft, dann ist der Hamster nicht mehr da. Aber jetzt braucht er mal Entspannung! Aber Ablenkung hilft nur begrenzt. 'Uff!', denkt sich Mark. Jetzt hat er doch tatsächlich zwei kategoriale Variablen in seiner Abschlussarbeit gemessen. Zum einen die Behandlung Herbizideinsatz [ja/nein] und zum anderen die Messung Trockengewicht über Zielwert [ja/nein] im Kontext von Spargel. Hierfür hat er einen Versuch in einer Klimakammer im Teuteburgerwald durchgeführt. Jetzt möchte Mark die Daten einmal in einer explorativen Datenanalyse darstellen. Danach kann er dann über den passenden statistischen Test nachdenken. Dabei unterstützt seine Betreuerin diesen Ansatz bevor es in der Datenanalyse weiter geht. So schön wie so gut. Eine echte Herausforderung für ihn war schon immer die Unsicherheit gewesen. Ein leidiges Lied.



\vspace{1Ex}

\begin{center}
\begin{minipage}[t]{0.45\textwidth}
%\small
\begin{center}

\begin{tabular}{p{2.5cm}p{2.5cm}p{2.5cm}p{2.5cm}}
\toprule
Trockengewicht über Zielwert & Herbizideinsatz\\
\midrule
ja & nein\\
nein & ja\\
nein & ja\\
ja & ja\\
ja & nein\\
\addlinespace
ja & ja\\
ja & nein\\
nein & nein\\
ja & ja\\
ja & nein\\
\addlinespace
nein & ja\\
ja & nein\\
nein & nein\\
nein & ja\\
nein & nein\\
\addlinespace
ja & nein\\
nein & ja\\
\bottomrule
\end{tabular}


\end{center}
\end{minipage}
\begin{minipage}[t]{0.45\textwidth}
%\small
\begin{center}

\begin{tabular}{p{2.5cm}p{2.5cm}p{2.5cm}p{2.5cm}}
\toprule
Trockengewicht über Zielwert & Herbizideinsatz\\
\midrule
nein & ja\\
nein & ja\\
nein & ja\\
nein & nein\\
ja & nein\\
\addlinespace
ja & ja\\
ja & nein\\
ja & ja\\
nein & nein\\
nein & nein\\
\addlinespace
ja & ja\\
nein & ja\\
nein & ja\\
ja & nein\\
ja & nein\\
\addlinespace
ja & nein\\
ja & nein\\
\bottomrule
\end{tabular}


\end{center}
\end{minipage}
\end{center}

\vspace{2Ex}

Leider kennt sich Mark mit der Erstellung einer explorativen Datenanalyse für kategoriale Daten überhaupt nicht aus. Deshalb braucht er bei der Erstellung Ihre Hilfe!

\begin{enumerate}
\item Stellen Sie den Zusammenhang zwischen den beiden kategorialen Variablen in einer zusammenfassenden Tabelle dar! \textbf{(3 Punkte)}
\item Visualisieren Sie den Zusammenhang zwischen den beiden kategorialen Variablen! \textbf{(3 Punkte)}
\item Berechnen Sie die Verhältnisse in der Visualisierung! Welche Annahme haben Sie getroffen? \textbf{(2 Punkte)}
\item Wenn \textit{ein} Effekt von der Behandlung vorliegen würde, wie würde die Tabelle und die Visualisierung aussehen? \textbf{(2 Punkt)}
\end{enumerate} 
\clearpage
% -----------------------------------------------------------------------

\section{Aufgabe \hfill (10 Punkte)}

\textit{Geben Sie grundsätzlich Formeln und Rechenweg zur Lösung der Teilaufgaben mit an!} \\[1Ex]
 

 
%% --------------------------------------------------------------------
\begin{minipage}[t]{0.5\textwidth}
\includegraphics[width = 1.3cm]{/Users/kruppajo/work/GitHub/exam/avatare/Paula.png}\hspace{-4mm}\includegraphics[width = 1.3cm]{/Users/kruppajo/work/GitHub/exam/avatare/Steffen.png}
\end{minipage}
\begin{minipage}[t]{0.5\textwidth}
\hfill
\href{https://youtu.be/Op-gjzASH9I}{\includegraphics[width = 2cm]{img/youtube}}
\end{minipage}
%% --------------------------------------------------------------------



\paragraph{Visualisierung von Verteilungen}

'Was soll das denn jetzt schon wieder sein? Drei Boxplot, die auf der Seite liegen?', entfährt es Steffen und schaut dabei Paula an. 'Keine Ahnung. Es ist bestimmt wieder so ein Lernziel mit der Verteilung und so.', meint Paula sichtlich genervt und mampft noch ein paar Smarties. 'Du weißt doch wie es heißt, \textit{Frei ist, wer missfallen kann.}\footnote{Oschmann, A. (2024) Mädchen stärken: Stärken fördern, Selbstwert erhöhen und liebevoll durch Krisen begleiten. Goldegg Verlag}', merkt Steffen nickend an. Die beiden schauen angestrengt auf die drei Boxplots. Das Ziel ist es zu verstehen, wie eine Verteilung anhand eines Boxplots bewertet werden kann. Paula und die Romantik machen die Sache nicht einfacher.



{\centering \includegraphics[width=\maxwidth]{img/desc-stat-11-1} 

}




Jetzt brauchen Steffen und Paula Ihre Hilfe bei der Abschätzung einer Verteilung anhand von Boxplots um ihre Arbeit dann in diesem Semester noch abschließen zu können.

\begin{enumerate}
\item Zeichnen Sie über die Boxplots die entsprechende zugehörige Verteilung! \textbf{(3 Punkte)} 
\item Zeichnen Sie unter die Boxplots die entsprechende zugehörige Beobachtungen als Stiche! \textbf{(3 Punkte)}
\item Wie viel Prozent der Beobachtungen fallen in das IQR? Ergänzen Sie die Abbildung entsprechend um den Bereich! \textbf{(2 Punkte)}
\item Wie viel Prozent der Beobachtungen fallen in $\bar{y} \pm 1s$ und $\bar{y} \pm 2s$  unter der Annahme einer Normalverteilung? \textbf{(2 Punkte)}
\end{enumerate} 
\clearpage
% -----------------------------------------------------------------------

\section{Aufgabe \hfill (10 Punkte)}

\textit{Geben Sie grundsätzlich Formeln und Rechenweg zur Lösung der Teilaufgaben mit an!} \\[1Ex]
 

 
%% --------------------------------------------------------------------
\begin{minipage}[t]{0.5\textwidth}
\includegraphics[width = 1.3cm]{/Users/kruppajo/work/GitHub/exam/avatare/Steffen.png}\hspace{-4mm}\includegraphics[width = 1.3cm]{/Users/kruppajo/work/GitHub/exam/avatare/Yuki.png}
\end{minipage}
\begin{minipage}[t]{0.5\textwidth}
\hfill
\href{https://youtu.be/ZrJhn2wPbq4}{\includegraphics[width = 2cm]{img/youtube}}
\end{minipage}
%% --------------------------------------------------------------------



\paragraph{Visualisierung der Normalverteilung}

Yuki und die Faulheit machen die Sache mit dem Studium nicht einfacher. Immerhin ist noch Steffen zur Hilfe mit dabei. Steffen hat Reese's Peanut Butter Cups mitgebracht und London Grammar aufgedreht. Das ist immerhin eine Ablenkung. Nicht so gut wie Orchideen, aber immerhin etwas. Jetzt sollen die beiden diese komische Aufgabe lösen. Es geht um verschiedene Normalverteilungen. Anscheinend hängen Normalverteilungen vom Mittelwert $\bar{y}$ und der Standardabweichung $s$ ab. 'Wozu brauchen wir nochmal Normalverteilungen?', entfährt es Yuki. Durch das Mampfen von Steffen versteht sie kein Wort der Antwort. Steffen lächelt.\\



Jetzt brauchen Yuki und Steffen Ihre Hilfe bei der Abschätzung einer Verteilung um ihre Arbeit dann in diesem Semester noch abschließen zu können.

\begin{enumerate}
\item Skizzieren Sie zwei Normalverteilungen mit $\bar{y}_1 \neq \bar{y}_2$ und $s_1 = s_2$! \textbf{(3 Punkte)}
\item Beschriften Sie die Normalverteilungen mit den statistischen Maßzahlen! \textbf{(2 Punkte)}
\item Liegt Varianzhomogenität oder Varianzheterogenität vor? Begründen Sie Ihre Antwort! \textbf{(2 Punkte)}
\item In welchen Bereich fallen 68\% bzw. 95\% der Beobachtungen in einer Normalverteilung? Ergänzen Sie die Bereiche in \underline{einer} Normalverteilung! \textbf{(2 Punkte)}
\item Ergänzen Sie unter \underline{einer} der Normalverteilungen den entsprechenden Boxplot! \textbf{(1 Punkt)}
\end{enumerate}

 
\clearpage
% -----------------------------------------------------------------------

\section{Aufgabe \hfill (10 Punkte)}

\textit{Geben Sie grundsätzlich Formeln und Rechenweg zur Lösung der Teilaufgaben mit an!} \\[1Ex]
 

 
%% --------------------------------------------------------------------
\begin{minipage}[t]{0.5\textwidth}
\includegraphics[width = 1.3cm]{/Users/kruppajo/work/GitHub/exam/avatare/Steffen.png}\hspace{-4mm}\includegraphics[width = 1.3cm]{/Users/kruppajo/work/GitHub/exam/avatare/Tina.png}
\end{minipage}
\begin{minipage}[t]{0.5\textwidth}
\hfill
\href{https://youtu.be/MiD42k4l5Ag}{\includegraphics[width = 2cm]{img/youtube}}
\end{minipage}
%% --------------------------------------------------------------------



\paragraph{Visualisierung der Normalverteilung und der Poissonverteilung}

'Was sollen wir hier dann noch zeichnen?!', entfährt es Tina und schaut dabei Steffen an. 'Wir sollen eine Normalverteilung mit einem Mittelwert von $\bar{y}_1 = 4$ und einer Standardabweichung von $s_1 = 0.25$ zeichnen. Sowie eine weitere Normalverteilung mit einem Mittelwert von $\bar{y}_2 = 0$ und einer Standardabweichung von $s_2 = 0.25$. Keine Ahnung wie das geht. Darunter sollen dann noch eine Poissonverteilung mit einem Mittelwert von $\lambda_1 = 20$ sowie einer weiteren Poissonverteilung mit einem Mittelwert von $\lambda_2 = 3$ gezeichnet werden.', meint Steffen sichtlich genervt und mampft noch ein paar Oreos. Im Hintergrund spielt leise Taylor Swift. 'Wirre Geschichte...', merkt Tina nickend an. Die beiden schauen angestrengt auf die leeren Flächen für die Abbildungen. Steffen und die Wut machen die Suche nach der Lösung nicht einfacher.\\




{\centering \includegraphics[width=\maxwidth]{img/histogram-01-1} 

}




Jetzt brauchen Tina und Steffen Ihre Hilfe bei der Abschätzung einer Verteilung um ihre Arbeit dann in diesem Semester noch abschließen zu können.


\begin{enumerate}
\item Skizzieren Sie die zwei Normalverteilungen und zwei Poissonverteilungen! \textbf{(4 Punkte)}
\item Achten Sie auf die entsprechende Skalierung in den jeweiligen Abbildungen! \textbf{(2 Punkte)}
\item Ergänzen Sie unter \underline{einer} Normalverteilung den entsprechenden Boxplot! \textbf{(1 Punkt)}
\item Ergänzen Sie unter \underline{einer} Poissonverteilung den entsprechenden Boxplot! \textbf{(1 Punkt)}
\item Geben Sie ein Beispiel für ein Outcome $y$, welches einer Normalverteilung folgt! \textbf{(1 Punkt)}
\item Geben Sie ein Beispiel für ein Outcome $y$, welches einer Poissonverteilung folgt! \textbf{(1 Punkt)}
\end{enumerate} 
\clearpage
% -----------------------------------------------------------------------
\part{Statistisches Testen \& statistische Testtheorie}
% -----------------------------------------------------------------------  

\section{Aufgabe \hfill (9 Punkte)}


 
%% --------------------------------------------------------------------
\begin{minipage}[t]{0.5\textwidth}
\includegraphics[width = 1.3cm]{/Users/kruppajo/work/GitHub/exam/avatare/Alex.png}\hspace{-4mm}\includegraphics[width = 1.3cm]{/Users/kruppajo/work/GitHub/exam/avatare/Mark.png}
\end{minipage}
\begin{minipage}[t]{0.5\textwidth}
\hfill
\href{https://youtu.be/aHVYuFKTqZs}{\includegraphics[width = 2cm]{img/youtube}}
\end{minipage}
%% --------------------------------------------------------------------



\paragraph{Grundgesamtheit und experimentelle Stichprobe}

'Schnapspraline?', räuspert sich das Känguruh und schaukelt in der Wippe. Mark und Alex schauen erschrocken auf. 'Ähm, das ist hier eine Klausuraufgabe...', merkt Mark mit leicht schrägen Blick an. 'Ich mache hier ein Praktikum und schreibe Teile der Aufgaben.', gähnt das Känguruh. 'Also, ich glaube das ist so nicht gedacht. Und das sind die Schnapspralinen für den Geburtstag meiner Oma!', ruft Alex. 'Pillepalle! Meins, deins, das sind doch alles bürgerliche Kategorien!', entgegnet das Känguruh und liest von einem zerknitterten Stück Papier ab: 'Was ist der Unterschied zwischen dem Einen und dem Anderen. Steht alles unten in den Fragen. Einfach selber lesen...'.

\vspace{1ex}

Leider kennen sich Mark und Alex mit der Grundgesamtheit und der Stuchprobe überhaupt nicht aus. Daher sind Sie gefragt!

\begin{enumerate}
\item Nennen Sie das statistische Verfahren und zwei konkrete Beispiele zur Durchführung um von einer Grundgesamtheit auf eine Stichprobe zu gelangen! \textbf{(3 Punkte)}
\item Erklären Sie den Zusammenhang zwischen Stichprobe und Grundgesamtheit an einem Schaubild! Beschriften Sie das Schaubild entsprechend!
  \textit{Nutzen Sie hierfür als Veranschaulichung die Körpergröße von Männern oder Frauen aus den Gummibärchendaten!}  \textbf{(3 Punkte)}
\item Erweitern Sie das Schaubild um die Entstehung von $Pr(D|H_0)$! \textit{Nutzen Sie hierfür als Veranschaulichung zusätzlich die Gruppierungsvariable "`Modul"' aus den Gummibärchendaten!}  \textbf{(3 Punkte)}
\end{enumerate} 
\clearpage
% -----------------------------------------------------------------------

\section{Aufgabe \hfill (9 Punkte)}


 
%% --------------------------------------------------------------------
\begin{minipage}[t]{0.5\textwidth}
\includegraphics[width = 1.3cm]{/Users/kruppajo/work/GitHub/exam/avatare/Alex.png}\hspace{-4mm}\includegraphics[width = 1.3cm]{/Users/kruppajo/work/GitHub/exam/avatare/Jonas.png}
\end{minipage}
\begin{minipage}[t]{0.5\textwidth}
\hfill
\href{https://youtu.be/Ric8ne39DtI}{\includegraphics[width = 2cm]{img/youtube}}
\end{minipage}
%% --------------------------------------------------------------------



\paragraph{Das Nullritual - Die statistische Testtheorie}

NA

\vspace{1ex}

Leider kennen sich Jonas und Alex mit statistischen Testtheorie, auch Null-Ritual genannt, überhaupt nicht aus. Geschweige denn mit der Visualisierung als Kreuztabelle.  

\begin{enumerate}
\item Tragen Sie folgende statistische Fachbegriffe zur statistischen Testtheorie korrekt eine selbst erstellte Kreuztabelle ein! \textbf{(3 Punkte)}
  \begin{center}
  \begin{tabular}{cccc}
  (Unbekannte) Wahrheit & 5\% & Richtige Entscheidung & H$_0$ abgelehnt \\
  \end{tabular}
  \end{center}
\item Ergänzen Sie Ihre erstellte Kreuztabelle um vier weitere, passende Fachbegriffe zur statistischen Testtheorie! \textbf{(2 Punkte)}
\end{enumerate}

Die Entscheidungsfindung durch einen statistischen Test kann auch durch die Analogie zu einem Feuermelder abgebildet werden. Dabei symbolisiert der Feuermelder den statistischen Test und es soll getestet werden, ob ein Feuer ausgebrochen ist.

\begin{enumerate}
  \setcounter{enumi}{2}    
\item In der Analogie des Feuermelders, wie lautet der $\alpha$-Fehler? \textbf{(1 Punkt)}
\item In der Analogie des Feuermelders, wie lautet der $\beta$-Fehler? \textbf{(1 Punkt)}
\item Wenn der Feuermelder einmal pro Tag messen würde, wie oft würde der Feuermelder mit einem $\alpha$ von 5\% in einem Jahr Alarm schlagen? Begründen Sie Ihre Antwort! \textbf{(2 Punkte)}
\end{enumerate}



 
\clearpage
% -----------------------------------------------------------------------

\section{Aufgabe \hfill (9 Punkte)}

\textit{Geben Sie grundsätzlich Formeln und Rechenweg zur Lösung der Teilaufgaben mit an!} \\[1Ex]


 
%% --------------------------------------------------------------------
\begin{minipage}[t]{0.5\textwidth}
\includegraphics[width = 1.3cm]{/Users/kruppajo/work/GitHub/exam/avatare/Jonas.png}\hspace{-4mm}\includegraphics[width = 1.3cm]{/Users/kruppajo/work/GitHub/exam/avatare/Paula.png}
\end{minipage}
\begin{minipage}[t]{0.5\textwidth}
\hfill
\href{https://youtu.be/32JjH1eyuTU}{\includegraphics[width = 2cm]{img/youtube}}
\end{minipage}
%% --------------------------------------------------------------------



\paragraph{Visualisierung der Teststatistik $\boldsymbol{T_D}$ und dem p-Wert}

'Wir sollen die Teststatistik $T_D$ umd dem p-Wert visualisieren, da mit einer Visualisierung vieles verständlicher wird!', ruft Paula um White Lies zu übertönen. 'Ich weiß nicht, was das jetzt helfen soll. Können wir nicht einfach schauen, ob der p-Wert kleiner als das Signifikanzniveau  $\alpha$ gleich 5\% ist? Und gut ist?', merkt Jonas an, was aber im Refrain von White Lies untergeht. Paula nickt im Beat. 'Wir haben hier eine t-verteilung unter der Annahme der Nullhypothese!', singt sie.

\vspace{1ex}

Leider kennen sich Paula und Jonas mit der Visualisierung der Teststatistik $T_D$ und dem p-Wert überhaupt nicht aus und brauchen dahr Ihre Hilfe!

\vspace{1ex}

\textit{Beachten Sie, dass im Folgenden \underline{keine numerisch korrekte Darstellung} verlangt wird! Es gilt Erkennbarkeit vor Genauigkeit!}

\begin{enumerate}
\item Ergänzen Sie eine beschriftete $x$-Achse! \textbf{(1 Punkt)}
\item Ergänzen Sie "`$\bar{y}_1 = \bar{y}_2$"'! \textbf{(1 Punkt)} 
\item Ergänzen Sie "`$A = 95\%$"'! \textbf{(1 Punkt)}
\item Zeichnen Sie $T_{\alpha=5\%}$ in die Abbildung! \textbf{(1 Punkt)} 
\item Zeichnen Sie das Signifikanzniveau $\alpha$ in die Abbildung! Begründen Sie Ihre Antwort! \textbf{(2 Punkte)} 
\item Zeichnen Sie $+T_{D}$ in die Abbildung! \textbf{(1 Punkt)}
\item Zeichnen Sie einen nicht signifikant p-Wert in die Abbildung! Begründen Sie Ihre Antwort! \textbf{(2 Punkte)}   
\end{enumerate}



{\centering \includegraphics[width=\maxwidth]{img/statistisches-testen-3-1} 

}


 
\clearpage
% -----------------------------------------------------------------------

\section{Aufgabe \hfill (10 Punkte)}


 
%% --------------------------------------------------------------------
\begin{minipage}[t]{0.5\textwidth}
\includegraphics[width = 1.3cm]{/Users/kruppajo/work/GitHub/exam/avatare/Jessica.png}\hspace{-4mm}\includegraphics[width = 1.3cm]{/Users/kruppajo/work/GitHub/exam/avatare/Steffen.png}
\end{minipage}
\begin{minipage}[t]{0.5\textwidth}
\hfill
\href{https://youtu.be/CN_O4fYPbhs}{\includegraphics[width = 2cm]{img/youtube}}
\end{minipage}
%% --------------------------------------------------------------------



\paragraph{Visualisierung des 95\% Konfidenzintervalls}

'So, was haben wir gemacht? Wir haben einen t-test für den Vergleich der Mittelwerte gerechnet.', meint Jessica. Steffen schaut fragend. 'Hatten wir nicht alles zu einer Kontrolle verglichen? Das war doch so!', ruft Steffen laut aus. 'Wir haben doch Messwert \textit{Trockengewicht nach Düngergabe} erhoben.', stellt Jessica fest. Jetzt haben beide das Problem, die möglichen 95\% Konfidenzintervalle zu interpretieren.

\vspace{1ex}

Leider kennen sich Jessica und Steffen mit der Visualisierung des 95\% Konfidenzintervall überhaupt nicht aus. 

\begin{enumerate}
\item Beschriften Sie die untenstehende Abbildung mit der Signifikanzschwelle! Begründen Sie Ihre Antwort! \textbf{(2 Punkte)}
\item Ergänzen Sie eine \textit{in den Kontext passende} Relevanzschwelle! Begründen Sie Ihre Antwort! \textbf{(2 Punkte)} 
\item Skizieren Sie in die untenstehende Abbildung sechs einzelne Konfidenzintervalle (a-f) mit den
  jeweiligen Eigenschaften! \textbf{(6 Punkte)}
  \begin{itemize}
  \item[(a)] Ein nicht signifikantes, nicht relevantes 95\% Konfidenzintervall 	
  \item[(b)] Ein signifikantes, nicht relevantes 95\% Konfidenzintervall 	
  \item[(c)] Ein 95\% Konfidenzintervall mit h{"o}herer Fallzahl $n$ in der Stichprobe als der Rest der 95\% Konfidenzintervalle 	
  \item[(d)] Ein signifikantes, relevantes 95\% Konfidenzintervall 
  \item[(e)] Ein signifikantes, relevantes 99\% Konfidenzintervall.
  \item[(f)] Ein 95\% Konfidenzintervall mit niedriger Fallzahl $n$ in der Stichprobe als der Rest 95\% der Konfidenzintervalle
  \end{itemize}
\end{enumerate}

\begin{center}
  \includegraphics[height = 10cm]{/Users/kruppajo/work/GitHub/exam/question/img/statistisches-testen-04}
\end{center}


 
\clearpage
% -----------------------------------------------------------------------

\section{Aufgabe \hfill (10 Punkte)}

\textit{Geben Sie grundsätzlich Formeln und Rechenweg zur Lösung der Teilaufgaben mit an!} \\[1Ex]


 
%% --------------------------------------------------------------------
\begin{minipage}[t]{0.5\textwidth}
\includegraphics[width = 1.3cm]{/Users/kruppajo/work/GitHub/exam/avatare/Jessica.png}\hspace{-4mm}\includegraphics[width = 1.3cm]{/Users/kruppajo/work/GitHub/exam/avatare/Tina.png}
\end{minipage}
\begin{minipage}[t]{0.5\textwidth}
\hfill
\href{https://youtu.be/FgZmpnEWDag}{\includegraphics[width = 2cm]{img/youtube}}
\end{minipage}
%% --------------------------------------------------------------------



\paragraph{Zusammenhang zwischen dem Effekt, der Streuung sowie der Fallzahl}

An einem kalten Dezembermorgen haben sich Jessica und Tina zum Lernen verabredet. Eine große Kanne Tee und Berge von Schokobons warten darauf gegessen zu werden. Jessica liest laut vor:\begin{quote}
                 \textit{
                 Beim statistischen Testen gibt es einen Zusammenhang zwischen dem Effekt, der Streuung sowie der Fallzahl. Gegeben sei die Formel für den Student t-Test auf den die folgenden Überlegungen basieren sollen. Welche Auswirkung hat die Änderungen der jeweiligen statistischen Maßzahl des Effekts $\Delta$, der Streuung $s$ und der Fallzahl $n$ auf die Teststistik $T_{D}$, den p-Wert $Pr(D|H_0)$ sowie dem Konfidenzintervall $KI_{1-\alpha}$?
                 }
                 \end{quote}Tina hebt die Augenbraue. 'Irgendwie sagt mir die Aufgabe jetzt mal so gar nichts. Was soll da gemacht werden?', merkt Tina an und ergänzt: 'Schauen wir doch erstmal zur Entspannung Indiana Jones, den Film habe ich extra nochmal mitgebracht.' Jessica ist der Idee nicht abgeneigt und auch die Hündin kommt unter dem Sofa hervor um mitzuschauen. 

\vspace{1ex}

Leider kennen sich Jessica und Tina mit dem Zusammenhang zwischen dem Effekt, der Streuung sowie der Fallzahl überhaupt nicht aus. 


\begin{enumerate}
\item Visualisieren Sie den Zusammenhang zwischen der Teststatiatik $T_{D}$ und dem p-Wert $Pr(D|H_0)$ für sich verändernde $T_{D}$-Werte!\textit{Geben Sie dafür ein numerisches Beispiel in dem Sie drei $T_{D}$-Werte und deren Einfluss auf den p-Wert vergleichen!} \textbf{(3 Punkte)}  
\item  Füllen Sie die untenstehende Tabelle aus in dem Sie die Änderung der statistischen Maßzahlen auf die Teststatistik, den p-Wert sowie das Konfidenzintervall in \textit{einem} Wort oder Symbol beschreiben! \textbf{(4 Punkte)}
\begin{center}
  \large
  \begin{tabular}[c]{l|c|c|c|l|c|c|c}
    & $T_{D}$ & $Pr(D|H_0)$ & $KI_{1-\alpha}$ & & $T_{D}$ & $Pr(D|H_0)$ & $KI_{1-\alpha}$\strut\\ 
    \hline
    \textbf{$\Delta\; \uparrow$} & \hspace{1.8cm} & \hspace{1.8cm}  & \hspace{1.8cm} & \textbf{
                                                          $\Delta\; \downarrow$} &
                                                                          \hspace{1.8cm} & \hspace{1.8cm}  & \hspace{1.8cm}\strut\\
    \hline
        \textbf{$s\; \uparrow$} & \hspace{1.8cm} & \hspace{1.8cm}  & \hspace{1.8cm} & \textbf{
                                                          $s\; \downarrow$} &
                                                                          \hspace{1.8cm}
                                                & \hspace{1.8cm}  & \hspace{1.8cm}\strut\\
    \hline
        \textbf{$n\; \uparrow$} & \hspace{1.8cm} & \hspace{1.8cm}  & \hspace{1.8cm} & \textbf{
                                                          $n\; \downarrow$} &
                                                                          \hspace{1.8cm}
                                                & \hspace{1.8cm}  & \hspace{1.8cm}\strut\\
    \hline
  \end{tabular}
\end{center}
\item Visualisieren Sie ein 95\%-iges Konfidenzintervall im Vergleich zu einem 90\%-igen Konfidenzintervall! Begründen Sie Ihre Visualisierung anhand der Formel des Konfidenzintervalls des t-Tests mathematisch! \textbf{(3 Punkte)} 
\end{enumerate} 
\clearpage
% -----------------------------------------------------------------------
\part{Der Student t-Test, Welch t-Test \& gepaarter t-Test}
% -----------------------------------------------------------------------

\section{Aufgabe \hfill (9 Punkte)}

\textit{Geben Sie grundsätzlich Formeln und Rechenweg zur Lösung der Teilaufgaben mit an!} \\[1Ex]
 

 
%% --------------------------------------------------------------------
\begin{minipage}[t]{0.5\textwidth}
\includegraphics[width = 1.3cm]{/Users/kruppajo/work/GitHub/exam/avatare/Nilufar.png}
\end{minipage}
\begin{minipage}[t]{0.5\textwidth}
\hfill
\href{https://youtu.be/eejS2uG4o-M}{\includegraphics[width = 2cm]{img/youtube}}
\end{minipage}
\vspace{-3ex}
%% --------------------------------------------------------------------



\paragraph{Berechnung des Student t-Test \underline{oder} Welch t-Test}

Nilufar ist im Oldenburger Land für einen Pilotexperiment mit sehr geringer Fallzahl $(n_1 = n_2 = 3)$ mit Erdbeeren. Allein diese Tatsache ist für sie eine Erzählung wert. Eine echte Herausforderung für sie war schon immer die Erwartung gewesen. Ein leidiges Lied.  Für ihrer Hausarbeit musste sie ein Gewächshausexperiment mit Erdbeeren durchführen und das sollte laut ihrer Betreuerin an diesem Ort besonders gut gelingen, da man hier gut neue technische Anlagen und Behandlungen fernab der Bevölkerung testen könne. Zeugen gibt es hier jedenfalls keine. Gar keine.  Alleine sein hilft jetzt aber nur bedingt, denn ihre Behandlung Lüftungssystemen und Folientunneln ($ctrl$ und $tornado$) und der Messwert Chlorophyllgehalt (SPAD-502Plus) [SPAD] sollen mit einem t-Test ausgewertet werden. Immerhin weiß sie, dass ihr Messwert einer Normalverteilung folgt. Hm..., was entspannendes wäre gut. Einfach mal raus um zu Kicken. Ohne Ziel und Uhr. Das ist was für Nilufar.

\begin{table}[!h]
\centering
\begin{tabular}{cc}
\toprule
treatment & weight\\
\midrule
dose & 5.8\\
ctrl & 14.5\\
ctrl & 20.9\\
dose & 13.9\\
dose & 13.9\\
\addlinespace
ctrl & 17.8\\
\bottomrule
\end{tabular}
\end{table}



Leider kennt sich Nilufar mit der Berechnung eines t-Tests überhaupt nicht aus. Deshalb braucht sie bei der Berechnung Ihre Hilfe!

\begin{enumerate}
  \item Formulieren Sie die wissenschaftliche Fragestellung! \textbf{(1 Punkt)}
  \item Bestimmen Sie die Teststatistik $T_{D}$ eines Student t-Tests! \textbf{(3 Punkte)}
  \item Treffen Sie mit $T_{\alpha = 5\%} = 1.84$ eine Aussage zur Nullhypothese! Begründen Sie Ihre Antwort! \textbf{(2 Punkte)}
  \item Berechnen Sie den Effekt des Student t-Tests! \textbf{(1 Punkt)}
  \item Formulieren Sie eine Antwort an Nilufar über das Ergebnis Ihrer statistischen Analyse! \textbf{(2 Punkte)}
\end{enumerate} 
\clearpage
% -----------------------------------------------------------------------

\section{Aufgabe \hfill (12 Punkte)}

\textit{Geben Sie grundsätzlich Formeln und Rechenweg zur Lösung der Teilaufgaben mit an!} \\[1Ex]
 

 
%% --------------------------------------------------------------------
\begin{minipage}[t]{0.5\textwidth}
\includegraphics[width = 1.3cm]{/Users/kruppajo/work/GitHub/exam/avatare/Steffen.png}
\end{minipage}
\begin{minipage}[t]{0.5\textwidth}
\hfill
\href{https://youtu.be/Cq_rF_z4xOk}{\includegraphics[width = 2cm]{img/youtube}}
\end{minipage}
\vspace{-3ex}
%% --------------------------------------------------------------------



\paragraph{Berechnung des Student t-Test}

'Der t-Test testet ein normalverteiltes Outcome ($Y$).', liest Steffen laut. Das hilft jetzt auch nur bedingt weiter. Steffen und die Romantik, eine unendliche Geschichte mit kniffeligen Wendungen. Laut seiner Betreuerin ist zwar ihm Messwert Schlachtgewicht [kg] normalverteilt, aber wie rechnet er jetzt einen t-Test? Für seine Abschlussarbeit musste er ein Stallexperiment mit Hühnern im Oldenburger Land durchführen. Als wäre das nicht schon anstrengend genug gewesen. Jetzt soll er auch noch testen, ob die Behandlung Lüftungssystem ($keins$ und $vorhanden$) ein signifikantes Ergebnis liefert. Hm..., was entspannendes wäre gut. Schon dutzende Male gesehen: Harry Potter. Aber immer noch großartig zusammen mit Oreos.

\begin{table}[!h]
\centering
\begin{tabular}{cc}
\toprule
Lüftungssystem & Schlachtgewicht\\
\midrule
keins & 36.9\\
vorhanden & 31.2\\
keins & 30.2\\
keins & 30.6\\
vorhanden & 27.8\\
\addlinespace
vorhanden & 32.0\\
keins & 40.0\\
vorhanden & 35.6\\
keins & 45.4\\
keins & 32.2\\
\addlinespace
vorhanden & 17.3\\
keins & 47.8\\
vorhanden & 29.2\\
vorhanden & 25.9\\
vorhanden & 30.1\\
\addlinespace
keins & 37.5\\
\bottomrule
\end{tabular}
\end{table}



Leider kennt sich Steffen mit der Berechnung eines t-Tests überhaupt nicht aus. Deshalb braucht er bei der Berechnung Ihre Hilfe!

\begin{enumerate}
  \item Formulieren Sie die wissenschaftliche Fragestellung! \textbf{(1 Punkt)}
  \item Formulieren Sie das statistische Hypothesenpaar! \textbf{(1 Punkt)}
  \item Bestimmen Sie die Teststatistik $T_{D}$ eines Student t-Tests! \textbf{(3 Punkte)}
\item Treffen Sie mit $T_{\alpha = 5\%} = 2.04$ eine Aussage zur Nullhypothese! Begründen Sie Ihre Antwort! \textbf{(2 Punkte)}
\item Berechnen Sie den Effekt des Student t-Tests! \textbf{(1 Punkt)}
\item Wenn Sie \textit{keinen} Unterschied zwischen den Behandlungsgruppen erwarten würden, wie groß wäre dann der Effekt? Begründen Sie Ihre Antwort! \textbf{(2 Punkte)}
\item Formulieren Sie eine Antwort an Steffen über das Ergebnis Ihrer statistischen Analyse! \textbf{(2 Punkte)}
\end{enumerate} 
\clearpage
% -----------------------------------------------------------------------

\section{Aufgabe \hfill (12 Punkte)}

\textit{Geben Sie grundsätzlich Formeln und Rechenweg zur Lösung der Teilaufgaben mit an!} \\[1Ex]
 

 
%% --------------------------------------------------------------------
\begin{minipage}[t]{0.5\textwidth}
\includegraphics[width = 1.3cm]{/Users/kruppajo/work/GitHub/exam/avatare/Steffen.png}
\end{minipage}
\begin{minipage}[t]{0.5\textwidth}
\hfill
\href{https://youtu.be/TbSEOMCQYl4}{\includegraphics[width = 2cm]{img/youtube}}
\end{minipage}
\vspace{-3ex}
%% --------------------------------------------------------------------



\paragraph{Berechnung des Welch t-Test}


Das Wendland, unendliche Weiten. Wir schreiben das Jahr 2024. Dies sind die Abenteuer von Steffen, der mit seiner 1 Mann starken Besatzung 12 Wochen lang unterwegs ist, um neue Welten zu erforschen, neues Leben und neue Zivilisationen. 'Oder nennen wir es Ödnis und Verzweiflung', denkt Steffen. Für seine Abschlussarbeit ist Steffen ins Nichts gezogen. Wenn die Romantik nicht wäre, ja dann wäre wohl vieles möglich für Steffen! Aber so.. Was macht er nun? Steffen hat einen Leistungssteigerungsversuch mit Fleischrindern durchgeführt. Die Behandlung Ernährungszusatz ($ctrl$ und $fedX$) wurde an Fleischrindern getestet. Gemessen hat er dann als ein normalverteiltes Outcome ($Y$) Fettgehalt [\%/kg]. Jetzt soll er seiner Betreuerin nach testen, ob die Behandlung Ernährungszusatz ($ctrl$ und $fedX$) ein signifikantes Ergebnis liefert. Hm..., was entspannendes wäre gut. Aus den Boxen wummert Taylor Swift und sein Mund ist verklebt von Oreos. 'Herrlich', denkt Steffen.

\begin{table}[!h]
\centering
\begin{tabular}{cc}
\toprule
Ernährungszusatz & Fettgehalt\\
\midrule
fedX & 37.0\\
fedX & 48.4\\
ctrl & 24.5\\
fedX & 54.5\\
ctrl & 21.6\\
\addlinespace
fedX & 48.9\\
fedX & 52.5\\
fedX & 37.1\\
ctrl & 31.6\\
fedX & 39.1\\
\addlinespace
ctrl & 14.5\\
ctrl & 24.9\\
ctrl & 31.2\\
fedX & 51.0\\
fedX & 46.5\\
\addlinespace
ctrl & 20.3\\
\bottomrule
\end{tabular}
\end{table}



Leider kennt sich Steffen mit der Berechnung eines t-Tests überhaupt nicht aus. Deshalb braucht er bei der Berechnung Ihre Hilfe!

\begin{enumerate}
  \item Formulieren Sie die wissenschaftliche Fragestellung! \textbf{(1 Punkt)}
  \item Formulieren Sie das statistische Hypothesenpaar! \textbf{(1 Punkt)}
  \item Bestimmen Sie die Teststatistik $T_{D}$ eines  Welch t-Tests! \textbf{(3 Punkte)}
  \item Treffen Sie mit $T_{\alpha = 5\%} = 2.04$ eine Aussage zur Nullhypothese! Begründen Sie Ihre Antwort! \textbf{(2 Punkte)}
\item Berechnen Sie das 95\% Konfidenzintervall. Welche Annahmen haben Sie getroffen? \textbf{(2 Punkte)}
\item Nennen Sie den statistischen Grund, warum Sie sich zwischen einem Student t-Test und einem Welch t-Test entscheiden müssen! \textbf{(1 Punkt)}
\item Formulieren Sie eine Antwort an Steffen über das Ergebnis Ihrer statistischen Analyse! \textbf{(2 Punkte)}
\end{enumerate} 
\clearpage
% -----------------------------------------------------------------------

\section{Aufgabe \hfill (11 Punkte)}

\textit{Geben Sie grundsätzlich Formeln und Rechenweg zur Lösung der Teilaufgaben mit an!} \\[1Ex]
 

 
%% --------------------------------------------------------------------
\begin{minipage}[t]{0.5\textwidth}
\includegraphics[width = 1.3cm]{/Users/kruppajo/work/GitHub/exam/avatare/Steffen.png}\hspace{-4mm}\includegraphics[width = 1.3cm]{/Users/kruppajo/work/GitHub/exam/avatare/Tina.png}
\end{minipage}
\begin{minipage}[t]{0.5\textwidth}
\hfill
\href{https://youtu.be/QR90zyn0Iz8}{\includegraphics[width = 2cm]{img/youtube}}
\end{minipage}
%% --------------------------------------------------------------------



\paragraph{Berechnung des gepaarten t-Test}

Tina und Steffen haben sich dazu entschieden zusammenzuarbeiten. Das sollte alles etwas einfacher machen. Jeder hat zwar ein getrenntes Themenfeld aber den Hauptversuch machen beide gemeinsam. Das hat sich schonmal als gut Idee soweit herausgestellt. In einer Hausarbeit sollen beide herausfinden, ob es einen Zusammenhang zwischen Bestandsdichte ($hoch$ und $niedrig$) und Protein/Fettrate [\%/kg] gibt. Die Besonderheit ist hierbei, dass die Messungen an der gleichen Beobachtung stattfinden. Beide messen also zweimal an den gleichen Zandern. Hier muss dann wohl auf einen normalverteilten Messwert ($Y$) ein gepaarter t-Test gerechnet werden. Tina schaut etwas flehentlich zu Steffen. Eine echte Herausforderung für sie war schon immer die Wut gewesen. Ein leidiges Lied.. Steffen denkt derweil angestrengt an Taylor Swift und wippt leicht mit dem Fuß.

\begin{table}[!h]
\centering
\begin{tabular}{ccc}
\toprule
ID & treatment & freshmatter\\
\midrule
7 & hoch & 12.7\\
7 & niedrig & 24.4\\
3 & niedrig & 29.6\\
2 & hoch & 31.4\\
8 & niedrig & 25.7\\
\addlinespace
5 & niedrig & 22.2\\
5 & hoch & 30.1\\
6 & niedrig & 19.4\\
2 & niedrig & 26.5\\
4 & hoch & 43.2\\
\addlinespace
1 & niedrig & 29.5\\
6 & hoch & 54.8\\
3 & hoch & 34.3\\
1 & hoch & 50.6\\
9 & niedrig & 31.2\\
\addlinespace
8 & hoch & 28.0\\
9 & hoch & 31.0\\
11 & hoch & 27.9\\
10 & hoch & 36.4\\
4 & niedrig & 19.4\\
\bottomrule
\end{tabular}
\end{table}



Leider kennen sich Tina und Steffen mit der Berechnung eines gepaarten t-Tests überhaupt nicht aus. Deshalb brauchen sie beide bei der Berechnung Ihre Hilfe!

\begin{enumerate}
  \item Formulieren Sie die wissenschaftliche Fragestellung! \textbf{(1 Punkt)}
  \item Formulieren Sie das statistische Hypothesenpaar! \textbf{(1 Punkt)}
  \item Bestimmen Sie die Teststatistik $T_{D}$ eines gepaarten t-Tests! \textbf{(3 Punkte)}
  \item Treffen Sie mit $T_{\alpha = 5\%} = 1.64$ eine Aussage zur Nullhypothese! Begründen Sie Ihre Antwort! \textbf{(2 Punkte)}
\item Schätzen Sie den $p$-Wert des gepaarten t-Tests ab! Begründen Sie Ihre Antwort mit einer Skizze! \textbf{(2 Punkte)}
\item Formulieren Sie eine Antwort an Tina über das Ergebnis Ihrer statistischen Analyse! \textbf{(2 Punkte)}
\end{enumerate}


 
\clearpage
% -----------------------------------------------------------------------

\section{Aufgabe \hfill (10 Punkte)}

\textit{Geben Sie grundsätzlich Formeln und Rechenweg zur Lösung der Teilaufgaben mit an!} \\[1Ex]
 

 
%% --------------------------------------------------------------------
\begin{minipage}[t]{0.5\textwidth}
\includegraphics[width = 1.3cm]{/Users/kruppajo/work/GitHub/exam/avatare/Jessica.png}\hspace{-4mm}\includegraphics[width = 1.3cm]{/Users/kruppajo/work/GitHub/exam/avatare/Nilufar.png}\hspace{-4mm}\includegraphics[width = 1.3cm]{/Users/kruppajo/work/GitHub/exam/avatare/Yuki.png}
\end{minipage}
\begin{minipage}[t]{0.5\textwidth}
\hfill
\href{https://youtu.be/exDo7AyHl4Q}{\includegraphics[width = 2cm]{img/youtube}}
\end{minipage}
%% --------------------------------------------------------------------



\paragraph{Interpretation des t-Tests in \Rlogo - die Teststatistik und der p-Wert}


Yuki und Jessica sind bei Nilufar um sich Hilfe in \Rlogo zu holen.  Im Hintergrund wummert Deichkind. Die beiden hatten zwar schon erste Kontakte mit \Rlogo sind sich aber unsicher bei der Interpetierung der Ausgabe eines t-Tests für ihren gemeinsamen Versuch. Es würde auch besser funktionieren, wenn Nilufar nicht die Erwartung im Weg stehen würde und Yuki nicht das Problem hätte die Faulheit zu händeln. In einer Abschlussarbeit haben beide zusammen Brokkoli untersucht. Dabei ging es um den Zusammenhang zwischen der Behandlung Bewässerungstypen ($low$ und $high$) und dem Messwert Proteingehalt [g/kg]. Der Versuch wurde in einem Feldexperiment im Teuteburgerwald durchgeführt. Nach der Betreuerin ist der Messwert Proteingehalt [g/kg] normalverteilt und ein t-Test passt daher. Das wird jetzt nicht mehr angezweifel...Nilufar überlegt, ob sie die beiden nicht noch auf den Film \textit{Star Trek} einlädt.

\begin{knitrout}
\definecolor{shadecolor}{rgb}{0.969, 0.969, 0.969}\color{fgcolor}\begin{kframe}
\begin{verbatim}
## 
## 	Two Sample t-test
## 
## data:  Proteingehalt by Bewässerungstypen
## t = -2.9055, df = 16, p-value = 0.01032
## alternative hypothesis: true  is not equal to [condensed]
## 95 percent confidence interval:
##  -24.033014  -3.756986
## sample estimates:
##  mean in group low mean in group high 
##             32.830             46.725
\end{verbatim}
\end{kframe}
\end{knitrout}

Helfen Sie Nilufar bei der Interpretation des t-Tests! Sonst geht es auch für Yuki und Jessica nicht weiter.
  
\begin{enumerate}
  \item Formulieren Sie die wissenschaftliche Fragestellung! \textbf{(1 Punkt)}
  \item Formulieren Sie das statistische Hypothesenpaar! \textbf{(1 Punkt)}
\item Liegt ein signifikanter Unterschied zwischen den Gruppen vor? Begründen Sie Ihre Antwort! \textbf{(2 Punkte)}
\item Skizzieren Sie eine Abbildung in der Sie $T_{D}$, $Pr(D|H_0)$, $A=0.95$, sowie $T_{\alpha=5\%} = |2.12|$ einzeichnen! \textbf{(4 Punkte)}
\item Beschriften Sie die Abbildung! \textbf{(1 Punkt)}  
\item Berechnen Sie den Effekt des t-Tests! \textbf{(1 Punkt)}
\end{enumerate} 
\clearpage
% -----------------------------------------------------------------------

\section{Aufgabe \hfill (10 Punkte)}

\textit{Geben Sie grundsätzlich Formeln und Rechenweg zur Lösung der Teilaufgaben mit an!} \\[1Ex]
 

 
%% --------------------------------------------------------------------
\begin{minipage}[t]{0.5\textwidth}
\includegraphics[width = 1.3cm]{/Users/kruppajo/work/GitHub/exam/avatare/Jessica.png}\hspace{-4mm}\includegraphics[width = 1.3cm]{/Users/kruppajo/work/GitHub/exam/avatare/Jonas.png}\hspace{-4mm}\includegraphics[width = 1.3cm]{/Users/kruppajo/work/GitHub/exam/avatare/Tina.png}
\end{minipage}
\begin{minipage}[t]{0.5\textwidth}
\hfill
\href{https://youtu.be/wJqsNV1hOW8}{\includegraphics[width = 2cm]{img/youtube}}
\end{minipage}
%% --------------------------------------------------------------------



\paragraph{Interpretation des t-Tests in \Rlogo - das 95\% Konifidenzintervall}


'Programmieren ist wie eine Sprache lernen. Man muss es nur machen, dann wird man mit der Zeit immer besser!', gibt Tina zwinkernd zu Protokoll. Ein paar Mal hat sie schon die Wut gehindert weiterzumachen. Das hilft jetzt Jessica und Jonas nur bedingt, da beide jetzt die \Rlogo Ausgabe interpretieren müssen und nicht vor drei Wochen, wo noch Zeit gewesen wäre. Beide mampfen konzentriert Schokobons und Snickers in sich hinein. Die beiden hatten in der Uckermark einen Versuch mit Schweinen in einem Kreuzungsexperiment durchgeführt. Das war schon anstrengend genug! 'Wir haben Protein/Fettrate [\%/kg] gemessen, vielleicht hilft das ja...', meint Jessica leicht genervt. Alle starren auf die \Rlogo Ausgabe des t-Tests. Im Hintergrund wummert Tocotronic und man versteht kaum sein eigenes Wort. Jonas hofft, dass die Spinne von Tina beruhigend wirkt.

\begin{knitrout}
\definecolor{shadecolor}{rgb}{0.969, 0.969, 0.969}\color{fgcolor}\begin{kframe}
\begin{verbatim}
## 
## 	Two Sample t-test
## 
## data:  Protein/Fettrate by Flüssignahrung
## t = -2.1832, df = 15, p-value = 0.04532
## alternative hypothesis: true  is not equal to [condensed]
## 95 percent confidence interval:
##  -17.8316690  -0.2140452
## sample estimates:
## mean in group ctrl mean in group flOw 
##           23.85714           32.88000
\end{verbatim}
\end{kframe}
\end{knitrout}

Helfen Sie Tina bei der Interpretation des t-Tests! Sonst geht es auch für Jessica und Jonas nicht weiter.

\begin{enumerate}
  \item Formulieren Sie die wissenschaftliche Fragestellung! \textbf{(1 Punkt)}
  \item Formulieren Sie das statistische Hypothesenpaar! \textbf{(1 Punkt)}
\item Liegt ein signifikanter Unterschied zwischen den Gruppen vor? Begründen Sie Ihre Antwort! \textbf{(2 Punkte)}
\item Skizieren Sie das sich ergebende 95\% Konifidenzintervall! \textbf{(2 Punkte)}
\item Beschriften Sie die Abbildung und das 95\% Konfidenzintervall entsprechend! \textbf{(2 Punkte)}  
\item Interpretieren Sie den Effekt des 95\% Konifidenzintervalls! \textbf{(2 Punkte)}
\end{enumerate} 
\clearpage
% -----------------------------------------------------------------------

\section{Aufgabe \hfill (9 Punkte)}

\textit{Geben Sie grundsätzlich Formeln und Rechenweg zur Lösung der Teilaufgaben mit an!} \\[1Ex]
 

 
%% --------------------------------------------------------------------
\begin{minipage}[t]{0.5\textwidth}
\includegraphics[width = 1.3cm]{/Users/kruppajo/work/GitHub/exam/avatare/Mark.png}\hspace{-4mm}\includegraphics[width = 1.3cm]{/Users/kruppajo/work/GitHub/exam/avatare/Steffen.png}\hspace{-4mm}\includegraphics[width = 1.3cm]{/Users/kruppajo/work/GitHub/exam/avatare/Tina.png}
\end{minipage}
\begin{minipage}[t]{0.5\textwidth}
\hfill
\href{https://youtu.be/w62HJlbN28U}{\includegraphics[width = 2cm]{img/youtube}}
\end{minipage}
%% --------------------------------------------------------------------



\paragraph{Interpretation des t-Tests in \Rlogo - die Visualisierung}

'Mit dem R Paket \texttt{\{emmeans\}} können wir gleich die Gruppenvergleiche rechnen und uns das \textit{compact letter displac}' ausgeben lassen!', verkündet Steffen sichtlich stolz. Ein paar Mal hat sie schon die Romantik gehindert weiterzumachen. 'Nach Meinung des Betreuers soll es aber nur erstmal ein t-Test sein. Und die Ausgabe ist schon wirr genug.', merkt Mark an. Mark und Tina sind bei Steffen um sich in \Rlogo helfen zu lassen. Im Hintergrund wummert Taylor Swift. Tina streichelt zur Beruhigung die Schlange von Steffen. Die beiden waren 3 Monate in der Uckermark um einen Versuch mit Erbsen in einem Feldexperiment durchzuführen. Ziel war es das Outcome ($Y$) Proteingehalt [g/kg] zu bestimmen. Steffen überlegt, ob er die beiden nicht noch auf den Film \textit{Harry Potter} einlädt oder dann doch lieber raus geht um zu Ringen? Vielleicht will ja Tina mit. Besser als der Film.

\begin{knitrout}
\definecolor{shadecolor}{rgb}{0.969, 0.969, 0.969}\color{fgcolor}\begin{kframe}
\begin{verbatim}
## 
## 	Two Sample t-test
## 
## data:  Proteingehalt by Lichtstufen
## t = 0.79471, df = 13, p-value = 0.441
## alternative hypothesis: true  is not equal to [condensed]
## 95 percent confidence interval:
##  -5.222789 11.301361
## sample estimates:
##  mean in group none mean in group 600lm 
##            45.42500            42.38571
\end{verbatim}
\end{kframe}
\end{knitrout}

Helfen Sie Steffen bei der Interpretation des t-Tests! Sonst geht es auch für Mark und Tina nicht weiter.
  
\begin{enumerate}
  \item Formulieren Sie die wissenschaftliche Fragestellung! \textbf{(1 Punkt)}
  \item Formulieren Sie das statistische Hypothesenpaar! \textbf{(1 Punkt)}
\item Liegt ein signifikanter Unterschied zwischen den Gruppen vor? Begründen Sie Ihre Antwort! \textbf{(2 Punkte)}
\item Skizieren Sie die sich ergebenden Boxplot! Welche Annahmen an die Daten haben Sie getroffen? Begründen Sie Ihre
  Antwort! \textbf{(2 Punkte)} 
\item Skizieren Sie die sich ergebenden Barplots! \textbf{(2 Punkte)}
\item Berechnen Sie den Effekt des t-Tests! \textbf{(1 Punkt)}
\end{enumerate}
 
\clearpage
% -----------------------------------------------------------------------

\section{Aufgabe \hfill (10 Punkte)}

\textit{Geben Sie grundsätzlich Formeln und Rechenweg zur Lösung der Teilaufgaben mit an!} \\[1Ex]
 

 
%% --------------------------------------------------------------------
\begin{minipage}[t]{0.5\textwidth}
\includegraphics[width = 1.3cm]{/Users/kruppajo/work/GitHub/exam/avatare/Mark.png}\hspace{-4mm}\includegraphics[width = 1.3cm]{/Users/kruppajo/work/GitHub/exam/avatare/Tina.png}
\end{minipage}
\begin{minipage}[t]{0.5\textwidth}
\hfill
\href{https://youtu.be/kHmfEmU6lrk}{\includegraphics[width = 2cm]{img/youtube}}
\end{minipage}
%% --------------------------------------------------------------------



\paragraph{Interpretation des gepaarten t-Tests in \Rlogo}

Es gibt ja immer die Möglichkeit sich Hilfe zu holen. Das geht natürlich auch immer in einer Abschlussarbeit. Deshalb arbeiten Mark und Tina gemeinsam an einer Abschlussarbeit. Das macht dann auch die Analyse ihres Hauptversuches einfacher. Zwar hat jeder von ihnen noch ein Subthema, aber auch da kann man sich ja helfen. Das hilft dann teilweise nur bedingt. Mark und die Unsicherheit, eine unendliche Geschichte mit kniffeligen Wendungen. In dem Hauptversuch wurde Folgendes von den beiden gemacht. Mark und Tina haben sich Zandern angeschaut. Dabei geht um Zusammenhang zwischen Genotypisierung ($0d$ und $14d$) und Gewichtszuwachs in der 1LW. Jetzt sollen beide einen gepaarten t-Test rechnen. Leider kennen sich beide nicht sehr gut in \Rlogo aus. Aber wenigtens haben beide eine Menge an Marzipankugeln und in der Wohnung wummert Andrea Berg.

\begin{knitrout}
\definecolor{shadecolor}{rgb}{0.969, 0.969, 0.969}\color{fgcolor}\begin{kframe}
\begin{verbatim}
## 
## 	Paired t-test
## 
## data:  Gewichtszuwachs by Genotypisierung
## t = 9.578, df = 8, p-value = 1.17e-05
## alternative hypothesis: true  is not equal to [condensed]
## 95 percent confidence interval:
##  13.91095 22.73350
## sample estimates:
## mean difference 
##        18.32222
\end{verbatim}
\end{kframe}
\end{knitrout}

Jetzt brauchen Mark und Tina Ihre Hilfe bei der Berechnung eines gepaarten t-Tests in \Rlogo um ihre Arbeit dann in diesem Semester noch abschließen zu können.

\begin{enumerate}
  \item Formulieren Sie die wissenschaftliche Fragestellung! \textbf{(1 Punkt)}
  \item Formulieren Sie das statistische Hypothesenpaar! \textbf{(1 Punkt)}
\item Liegt ein signifikanter Unterschied zwischen den Gruppen vor?
  Begründen Sie Ihre Antwort! \textbf{(2 Punkte)}
\item Skizzieren Sie das sich ergebende 95\% Konfidenzintervall! \textbf{(2 Punkte)}
\item Interpretieren Sie den Effekt des gepaarten t-Tests! \textbf{(2 Punkte)}
\item Skizzieren Sie den sich ergebenden Boxplot der Differenzen! Welche Annahmen an die Daten haben Sie getroffen? Begründen Sie Ihre Antwort! \textbf{(2 Punkte)} 
\end{enumerate}
 
\clearpage
% -----------------------------------------------------------------------
\part{Die einfaktorielle \& zweifaktorielle ANOVA}
% -----------------------------------------------------------------------

\section{Aufgabe \hfill (11 Punkte)}

\textit{Geben Sie grundsätzlich Formeln und Rechenweg zur Lösung der Teilaufgaben mit an!} \\[1Ex]
 

 
%% --------------------------------------------------------------------
\begin{minipage}[t]{0.5\textwidth}
\includegraphics[width = 1.3cm]{/Users/kruppajo/work/GitHub/exam/avatare/Jessica.png}\hspace{-4mm}\includegraphics[width = 1.3cm]{/Users/kruppajo/work/GitHub/exam/avatare/Nilufar.png}
\end{minipage}
\begin{minipage}[t]{0.5\textwidth}
\hfill
\href{https://youtu.be/kHmfEmU6lrk}{\includegraphics[width = 2cm]{img/youtube}}
\end{minipage}
%% --------------------------------------------------------------------



\paragraph{Visualisierung der einfaktoriellen ANOVA}

Jessica und Nilufar schauen sich etwas entnervt an. Gemeinsam schreiben die beiden ihre Abschlussarbeit und sollen nun als erstes einmal die Daten visualisieren damit abgeschätzt werden kann, ob überhaupt signifikante Ergebnisse zu erwarten sind. Die beiden waren in der Uckermark um ein Stallexperiment mit Puten durchzuführen. Dabei haben Jessica und Nilufar den Messwert Schlachtgewicht [kg] unter der Behandung Ernährungszusatz ($ctrl$, $fedX$ und $getIt$) ermittelt. Kennengelernt haben sich die beiden auf einem Konzert von Deichkind. Später wird noch Star Trek geguckt. Nilufar befürwortet das!

\begin{knitrout}
\definecolor{shadecolor}{rgb}{0.969, 0.969, 0.969}\color{fgcolor}\begin{table}[!h]
\centering
\begin{tabular}{cc}
\toprule
Ernährungszusatz & Schlachtgewicht\\
\midrule
fedX & 30\\
ctrl & 27\\
ctrl & 24\\
ctrl & 26\\
getIt & 45\\
\addlinespace
getIt & 43\\
ctrl & 28\\
fedX & 31\\
getIt & 46\\
getIt & 47\\
\addlinespace
fedX & 30\\
getIt & 45\\
fedX & 30\\
fedX & 30\\
getIt & 47\\
\addlinespace
getIt & 43\\
ctrl & 26\\
fedX & 29\\
\bottomrule
\end{tabular}
\end{table}

\end{knitrout}

Leider kennen sich Jessica und Nilufar mit Darstellung einer einfaktoriellen ANOVA überhaupt nicht aus. 

\begin{enumerate}
\item Erstellen  Sie  eine  Visualisierung  der  Datentabelle! Beschriften  Sie  die  Abbildung! \textbf{(2 Punkte)}
\item Benennen Sie die Visualisierung mit dem korrekten, statistischen Fachbegriff! \textbf{(1 Punkt)}
\item Zeichnen Sie folgende statistischen Maßzahlen passend ein! 
  \begin{itemize}
  \item Globale Mittelwert: $\beta_0$ \textbf{(1 Punkt)}
  \item Mittelwerte der einzelnen Behandlungsstufen: $\bar{y}_{0.5}$, $\bar{y}_{1.5}$ und $\bar{y}_{2.5}$ \textbf{(1 Punkt)}
  \item Mittelwertsdifferenz der einzelnen Behandlungsstufen: $\beta_{0.5}$, $\beta_{1.5}$ und $\beta_{2.5}$ \textbf{(1 Punkt)}
  \item Residuen oder Fehler: $\epsilon$ \textbf{(1 Punkt)}
  \end{itemize}
\item Liegt ein \textit{vermutlicher} signifikanter Unterschied vor? Begründen Sie Ihre Antwort! \textbf{(2 Punkte)}
\item Schätzen Sie die Effekte der Behandlungsstufen! \textbf{(2 Punkte)}
\end{enumerate}
 
\clearpage
% -----------------------------------------------------------------------

\section{Aufgabe \hfill (9 Punkte)}

\textit{Geben Sie grundsätzlich Formeln und Rechenweg zur Lösung der Teilaufgaben mit an!} \\[1Ex]
 

 
%% --------------------------------------------------------------------
\begin{minipage}[t]{0.5\textwidth}
\includegraphics[width = 1.3cm]{/Users/kruppajo/work/GitHub/exam/avatare/Alex.png}\hspace{-4mm}\includegraphics[width = 1.3cm]{/Users/kruppajo/work/GitHub/exam/avatare/Mark.png}
\end{minipage}
\begin{minipage}[t]{0.5\textwidth}
\hfill
\href{https://youtu.be/IhecxMcCENY}{\includegraphics[width = 2cm]{img/youtube}}
\end{minipage}
%% --------------------------------------------------------------------



\paragraph{Ergebnistabelle der einfaktoriellen ANOVA}

'Uff... die einfaktorielle ANOVA. Und wie füllen wir jetzt die Tabelle der ANOVA aus und schauen, ob da was signifikant ist?', Alex hebt die Augenbraue. 'Das ist eine sehr gute Frage. Ich glaube man kann alles in der Tabelle relativ einfach mit wenigen Informationen berechnen.', meint Mark dazu. Da hilft der Hamster von Mark auch nur bedingt. Alex hatte sich in ein Freilandversuch verschiedene Maiss angeschaut. Dabei ging es herauszufinden, ob es einen Zusammenhang zwischen der Behandlung Substrattypen ($torf$, $40p60n$, $30p20n$ und $70p30n$) und dem Messwert Proteingehalt [g/kg] gibt. Nachher wollen sich beide noch mit dem Hobby Geocaching von Mark beschäftigen. Kennt Alex noch nicht, klingt aber interessant.



\vspace{1ex}

Leider kennen sich Alex und Mark mit Berechnung einer einfaktoriellen ANOVA überhaupt nicht aus. Deshalb brauchen beide bei der Erstellung Ihre Hilfe, der Hamster reicht als Hilfe nicht aus! 

\begin{enumerate}
  \item Formulieren Sie die wissenschaftliche Fragestellung! \textbf{(1 Punkt)}
  \item Formulieren Sie das statistische Hypothesenpaar! \textbf{(1 Punkt)}
\item Füllen Sie die unterstehende einfaktorielle ANOVA Ergebnistabelle aus! \textbf{(3 Punkte)}
\end{enumerate}

\vspace{1Ex}

\begin{center}
  \Large
  \begin{tabular}{lccccp{3cm}}
\toprule
     & \textbf{Df} & \textbf{Sum Sq} & \textbf{Mean Sq} & \textbf{F value} & \textbf{Pr(>F)} \strut\\
    \midrule
   \textbf{Substrattypen}  & 3 & 3375.46 &  &  &  \strut\\
   \textbf{error}  & 21 & 799.9 &  &  &  \strut\\
   \textbf{Total}  & 24 &  &  &  &  \strut\\
\bottomrule
  \end{tabular}
\end{center}

\vspace{1Ex}

\begin{enumerate}
  \setcounter{enumi}{3}
\item Schätzen Sie den p-Wert der Tabelle mit $F_{\alpha = 5\%} = 3.07$ ab. Begründen Sie Ihre Antwort! \textbf{(2 Punkte)}
\item Berechen Sie den Effektschätzer $\eta^2$. Was sagt Ihnen der Wert von $\eta^2$ aus? \textbf{(2 Punkte)}
\end{enumerate}



 
\clearpage
% -----------------------------------------------------------------------

\section{Aufgabe \hfill (12 Punkte)}

\textit{Geben Sie grundsätzlich Formeln und Rechenweg zur Lösung der Teilaufgaben mit an!} \\[1Ex]
 

 
%% --------------------------------------------------------------------
\begin{minipage}[t]{0.5\textwidth}
\includegraphics[width = 1.3cm]{/Users/kruppajo/work/GitHub/exam/avatare/Tina.png}\hspace{-4mm}\includegraphics[width = 1.3cm]{/Users/kruppajo/work/GitHub/exam/avatare/Yuki.png}
\end{minipage}
\begin{minipage}[t]{0.5\textwidth}
\hfill
\href{https://youtu.be/49hvImMwVyE}{\includegraphics[width = 2cm]{img/youtube}}
\end{minipage}
%% --------------------------------------------------------------------



\paragraph{Die einfaktoriellen ANOVA und der Student t-Test}

Tina und Yuki schauen sich etwas entnervt an. Gemeinsam schreiben die beiden ihre Abschlussarbeit und sollen nun als erstes einmal die Daten mit eine einfaktoriellen ANOVA auswerten damit abgeschätzt werden kann, ob überhaupt signifikante Ergebnisse in den multipen Gruppenvergleichen zu erwarten sind. Deshalb erstmal Reese's Peanut Butter Cups mampfen, die Yuki mitgebracht hat. Nun möchte erstmal ihre Betreuung der Arbeit eine ANOVA Tabelle sehen. Was immer da auch drin zu erkennen sein mag. Tina schaut Yuki sehen erstmla gar nichts. Die beiden waren in der Uckermark um ein Kreuzungsexperiment mit Fleischrindern durchzuführen. Dabei haben Tina und Yuki den Messwert Schlachtgewicht [kg] unter der Behandung Elterlinie ($ctrl$, $Standard$, $Yray$ und $Xray$) ermittelt. Später wollen die beiden dann noch raus um zu Boldern.



\vspace{1ex}

Leider kennen sich Tina und Yuki mit Berechnung einer einfaktoriellen ANOVA überhaupt nicht aus. Deshalb brauchen beide bei der Erstellung Ihre Hilfe! 

\begin{enumerate}
  \item Formulieren Sie die wissenschaftliche Fragestellung! \textbf{(1 Punkt)}
  \item Formulieren Sie das statistische Hypothesenpaar! \textbf{(1 Punkt)}
\item Füllen Sie die unterstehende einfaktorielle ANOVA Ergebnistabelle aus! \textbf{(3 Punkte)}
\end{enumerate}

\vspace{1Ex}

\begin{center}
  \Large
  \begin{tabular}{lccccp{3cm}}
\toprule
     & \textbf{Df} & \textbf{Sum Sq} & \textbf{Mean Sq} & \textbf{F value} & \textbf{Pr(>F)} \strut\\
    \midrule
   \textbf{Elterlinie}  & 3 & 2410.85 &  &  &  \strut\\
   \textbf{Error}  & 24 & 930 &  &  &  \strut\\
\bottomrule
  \end{tabular}
\end{center}

\vspace{1Ex}

\begin{enumerate}
  \setcounter{enumi}{3}
\item Schätzen Sie den p-Wert der Tabelle mit $F_{\alpha = 5\%} = 3.01$ ab. Begründen Sie Ihre Antwort! \textbf{(2 Punkte)}
\item Was bedeutet ein signifikantes Ergebnis in einer einfaktoriellen ANOVA? \textbf{(1 Punkt)}
\item Berechnen Sie \textit{einen} Student t-Test für den \textit{vermutlich} signifikantesten Gruppenvergleich anhand der untenstehenden Tabelle mit $T_{\alpha = 5\%} = 2.03$. Begründen Sie Ihre Auswahl! \textbf{(3 Punkte)}
\end{enumerate}


\begin{knitrout}
\definecolor{shadecolor}{rgb}{0.969, 0.969, 0.969}\color{fgcolor}\begin{table}[!h]
\centering\begingroup\fontsize{11}{13}\selectfont

\begin{tabular}{cccc}
\toprule
\textbf{Elterlinie} & \textbf{Fallzahl (n)} & \textbf{Mittelwert} & \textbf{Standardabweichung}\\
\midrule
ctrl & 8 & 6.75 & 4.50\\
Standard & 8 & 4.38 & 7.46\\
Yray & 5 & 30.60 & 9.40\\
Xray & 7 & 11.71 & 2.75\\
\bottomrule
\end{tabular}
\endgroup{}
\end{table}

\end{knitrout}


\begin{enumerate}
  \setcounter{enumi}{6}
\item Gegebenen der ANOVA Tabelle war das Ergebnis des Student t-Tests zu erwarten? Begründen Sie Ihre Antwort! \textbf{(2 Punkte)}
\end{enumerate}

 
\clearpage
% -----------------------------------------------------------------------

\section{Aufgabe \hfill (9 Punkte)}

\textit{Geben Sie grundsätzlich Formeln und Rechenweg zur Lösung der Teilaufgaben mit an!} \\[1Ex]
 

 
%% --------------------------------------------------------------------
\begin{minipage}[t]{0.5\textwidth}
\includegraphics[width = 1.3cm]{/Users/kruppajo/work/GitHub/exam/avatare/Steffen.png}
\end{minipage}
\begin{minipage}[t]{0.5\textwidth}
\hfill
\href{https://youtu.be/aXvxGC4YLqk}{\includegraphics[width = 2cm]{img/youtube}}
\end{minipage}
\vspace{-3Ex}
%% --------------------------------------------------------------------



\paragraph{Die einfaktorielle ANOVA in \Rlogo}

Steffen schaut sich fragend in der Bibliothek um. Steffen hatte gehofft, dass jemand hier sein würde, den er kennt und sich mit \Rlogo auskennt. Wird aber enttäuscht. Steffen war in der Uckermark um ein Kreuzungsexperiment mit Zandern durchzuführen. Nun möchte seine Betreuerin seiner Hausarbeit erstmal eine ANOVA sehen und die Ergebnisse präsentiert bekommen. Dabei ging es herauszufinden, ob es einen Zusammenhang zwischen der Behandlung Flüssignahrung ($ctrl$, $superIn$ und $flOw$) und dem Messwert Gewichtszuwachs in der 1LW gibt. Und eigentlich wollte Steffen doch noch zum Sport! Um zu Ringen geht Steffen dann später nochmal raus. Echte Entspannung.

\begin{knitrout}
\definecolor{shadecolor}{rgb}{0.969, 0.969, 0.969}\color{fgcolor}\begin{kframe}
\begin{verbatim}
## Analysis of Variance Table
## 
## Response: Gewichtszuwachs
##                Df Sum Sq Mean Sq F value    Pr(>F)
## Flüssignahrung  2 4279.9 2139.97  58.359 4.462e-08
## Residuals      16  586.7   36.67
\end{verbatim}
\end{kframe}
\end{knitrout}

\vspace{1ex}

Leider kennen sich Steffen mit Berechnung einer einfaktoriellen ANOVA überhaupt nicht aus. Deshalb braucht er bei der Erstellung Ihre Hilfe! 

\begin{enumerate}
  \item Formulieren Sie die wissenschaftliche Fragestellung! \textbf{(1 Punkt)}
  \item Formulieren Sie das statistische Hypothesenpaar! \textbf{(1 Punkt)}
\item Interpretieren Sie das Ergebnis der einfaktoriellen ANOVA! \textbf{(2 Punkte)} 
\item Berechnen Sie den Effektschätzer $\eta^2$. Was sagt Ihnen der Wert von $\eta^2$ aus? \textbf{(2 Punkte)}
\item Skizzieren Sie eine Abbildung, der dem obigen Ergebnis der
  einfaktoriellen ANOVA näherungsweise entspricht! \textbf{(3 Punkte)}
\end{enumerate}

 
\clearpage
% -----------------------------------------------------------------------

\section{Aufgabe \hfill (12 Punkte)}

\textit{Geben Sie grundsätzlich Formeln und Rechenweg zur Lösung der Teilaufgaben mit an!} \\[1Ex]
 

 
%% --------------------------------------------------------------------
\begin{minipage}[t]{0.5\textwidth}
\includegraphics[width = 1.3cm]{/Users/kruppajo/work/GitHub/exam/avatare/Jonas.png}
\end{minipage}
\begin{minipage}[t]{0.5\textwidth}
\hfill
\href{https://youtu.be/8Pb2sKUIMyk}{\includegraphics[width = 2cm]{img/youtube}}
\end{minipage}
\vspace{-3Ex}
%% --------------------------------------------------------------------



\paragraph{Ergebnistabelle der zweifaktoriellen ANOVA}

Jonas steht in der Uckermark. Und das ist noch langweiliger als es sich anhört. Wäre es nur so spannend wie bei seinen Kommilitonen, die in Almería waren. Ödnis wohin man nur blickt. Oder eben Maiss. Das Meerschweinchen duchbohrt ihn mit leeren Blick. 'Woher zum Teufel!', entfährt es ihm. Aber da ist es schon weg. Ja, darum geht es in seiner Abschlussarbeit. Und wäre das nicht noch alles genug, ist sein Experiment auch noch als ein Feldexperiment komplex geraten. Es wurde der Messwert Proteingehalt [g/kg] mit dem Behandlung Genotypen ($AA$, $AB$ und $BB$) sowie der Behandlung Düngestufen ($ctrl$, und $high$) untersucht. 'Hmpf', denkt Jonas und ruft 'Und jetzt!?' in die Leere. Und eigentlich wollte Jonas doch noch seinem Hobby nachgehen! Jonas liebt Stricken. Darin kann er sich wirklich verlieren und immer wieder neu begeistern.



\vspace{1ex}

Leider kennen sich Jonas mit Berechnung einer zweifaktoriellen ANOVA überhaupt nicht aus. Deshalb braucht er bei der Erstellung Ihre Hilfe! 

\begin{enumerate}
  \item Formulieren Sie die wissenschaftliche Fragestellung! \textbf{(1 Punkt)}
  \item Formulieren Sie das statistische Hypothesenpaar! \textbf{(1 Punkt)}
\item Füllen Sie die unterstehende einfaktorielle ANOVA Ergebnistabelle aus! \textbf{(3 Punkte)}
\end{enumerate}

\vspace{1Ex}

\begin{center}
  \Large
  \begin{tabular}{lccccc}
  \toprule
     & \textbf{Df} & \textbf{Sum Sq} & \textbf{Mean Sq} & \textbf{F value} & \textbf{Pr(>F)} \strut\\
    \midrule
   \textbf{Genotypen}  & 3 & 308.31 &  &  &  \strut\\
    \textbf{Düngestufen}  & 1 & 80.73 &  &  &  \strut\\
    \textbf{Genotypen:Düngestufen}  & 3 & 157.82 &  &  &  \strut\\
   \textbf{Error}  & 18 & 156.05 &  &  &  \strut\\
\bottomrule
  \end{tabular}
\end{center}

\vspace{1Ex}

\begin{enumerate}
  \setcounter{enumi}{3}
\item Schätzen Sie den p-Wert der Tabelle ab. Begründen Sie Ihre
  Antwort! \textbf{(3 Punkte)}
\end{enumerate}
  
\begin{center}
    \Large
\begin{tabular}{lc}
  \toprule
     & $\boldsymbol{F_{\alpha = 5\%}}$ \\
\midrule
  \textbf{Genotypen} & $4.26$ \\
  \textbf{Düngestufen} & $3.40$ \\
  \textbf{Genotypen:Düngestufen} & $5.23$ \\
  \bottomrule
  \end{tabular}
\end{center}

\begin{enumerate}
  \setcounter{enumi}{4}
\item Was bedeutet ein signifikantes Ergebnis in einer zweifaktoriellen ANOVA? \textbf{(2 Punkte)}
\item Was sagt der Term \textit{Genotypen:Düngestufen} aus? Interpretieren Sie das Ergebnis! \textbf{(2 Punkte)}
\end{enumerate}
 
\clearpage
% -----------------------------------------------------------------------

\section{Aufgabe \hfill (10 Punkte)}

\textit{Geben Sie grundsätzlich Formeln und Rechenweg zur Lösung der Teilaufgaben mit an!} \\[1Ex]
 

 
%% --------------------------------------------------------------------
\begin{minipage}[t]{0.5\textwidth}
\includegraphics[width = 1.3cm]{/Users/kruppajo/work/GitHub/exam/avatare/Yuki.png}
\end{minipage}
\begin{minipage}[t]{0.5\textwidth}
\hfill
\href{https://youtu.be/rWTyHXXlYjY}{\includegraphics[width = 2cm]{img/youtube}}
\end{minipage}
\vspace{-3Ex}
%% --------------------------------------------------------------------



\paragraph{Die zweifaktorielle ANOVA in \Rlogo}

Es ist schon kurz nach fünf und Yuki wird langsam nervös. Yuki wollte heute Abend noch seine E-Sport Qualifikation schauen. Stattdessen versucht sein Betreuer die Ausgabe der zweifaktoriellen ANOVA zu visualieren und zu überprüfen, ob es mit der Visualisierung der Daten als Boxplots zusammenpasst. Yuki hatte in der Uckermark ein Gewächshausexperiment mit Lauch durchgeführt. Es gab dabei zwei Behandlungen. Einmal Düngestufen ($ctrl$, $low$, $mid$ und $high$) sowie als zweite Behandlung Lichtstufen ($none$, und $600lm$). Gemessen wurde der Messwert ($Y$) Chlorophyllgehalt (SPAD-502Plus) [SPAD]. So kompliziert kann das jetzt doch nicht sein! Eigentlich wollte Yuki nachher noch zum Sport. Einfach mal raus um zu Boldern. Ohne Ziel und Uhr. Das ist was für Yuki.

\begin{knitrout}
\definecolor{shadecolor}{rgb}{0.969, 0.969, 0.969}\color{fgcolor}\begin{kframe}
\begin{verbatim}
## Analysis of Variance Table
## 
## Response: Chlorophyllgehalt
##                         Df Sum Sq Mean Sq F value    Pr(>F)
## Düngestufen              2 484.82 242.409 17.1101 6.869e-05
## Lichtstufen              1  10.81  10.811  0.7631  0.393880
## Düngestufen:Lichtstufen  2 194.14  97.071  6.8516  0.006131
## Residuals               18 255.02  14.168
\end{verbatim}
\end{kframe}
\end{knitrout}

\vspace{1ex}

Leider kennt sich Yuki mit Berechnung einer zweifaktoriellen ANOVA überhaupt nicht aus. Deshalb braucht er bei der Erstellung Ihre Hilfe! 

\begin{enumerate}
  \item Formulieren Sie die wissenschaftliche Fragestellung! \textbf{(1 Punkt)}
  \item Formulieren Sie das statistische Hypothesenpaar! \textbf{(1 Punkt)}
\item Interpretieren Sie das Ergebnis der einfaktoriellen ANOVA! \textbf{(3 Punkte)} 
\item Zeichnen Sie eine Abbildung, der dem obigen Ergebnis der
  zweifaktoriellen ANOVA näherungsweise entspricht! \textbf{(5 Punkte)}
\end{enumerate}
 
\clearpage
% -----------------------------------------------------------------------

\section{Aufgabe \hfill (12 Punkte)}

\textit{Geben Sie grundsätzlich Formeln und Rechenweg zur Lösung der Teilaufgaben mit an!} \\[1Ex]
 

 
%% --------------------------------------------------------------------
\begin{minipage}[t]{0.5\textwidth}
\includegraphics[width = 1.3cm]{/Users/kruppajo/work/GitHub/exam/avatare/Yuki.png}
\end{minipage}
\begin{minipage}[t]{0.5\textwidth}
\hfill
\href{https://youtu.be/FjjJXkFJfIY}{\includegraphics[width = 2cm]{img/youtube}}
\end{minipage}
\vspace{-3Ex}
%% --------------------------------------------------------------------



\paragraph{Zusammenhang zwischen der ANOVA und dem t-Test}

In ein Freilandversuch wurden Erbsen mit der Behandlung Lüftungssysteme ($ctrl$, $storm$, $thunder$ und $tornado$) sowie der Behandlung Düngestufen ($ctrl$, und $high$) untersucht. Yuki schaut konzentriert auf die Formeln der ANOVA und des t-Tests. In ihrem Experiment wurde als Messwert Chlorophyllgehalt (SPAD-502Plus) [SPAD] bestimmt. Yuki und die Faulheit, eine unendliche Geschichte mit kniffeligen Wendungen. Dann wäre es nicht noch komplizierter. Was war da jetzt nochmal der Zusammenhang zwischen den beiden statistischen Verfahren? Beide Verfahren haben ja irgendwie etwas miteinander zu tun und ihre Betreuerin möchte das jetzt auch noch verstehen. Muss das nicht eigentlich klar sein? Immerhin ist Yuki nicht die erste Betreuung einer Abschlussarbeit. Immerhin hat sie die beiden Formeln vorliegen. Hm, lecker Reese's Peanut Butter Cups und dazu dann im Hintergrund Matrix laufen lassen.

\begin{graybox}{Gegebene Formeln}
\begin{center}
  \begin{tabular}{cc}
    $F_{D} = \cfrac{MS_{treatment}}{MS_{error}}$ & $T_{D} = \cfrac{\bar{y}_1 - \bar{y}_2}{s_p \cdot \sqrt{2/n_g}}$\\
  \end{tabular}
\end{center}
\end{graybox}

Leider kennen sich Yuki mit dem Zusammenhang zwischen der ANOVA und dem t-Test nicht aus. Deshalb braucht sie bei der Erstellung Ihre Hilfe! 

\begin{enumerate}
\item Welche statistische Maßzahl testet der t-Test, welche die ANOVA? Begründen Sie Ihre Antwort! \textbf{(2 Punkte)}
\item Erklären Sie den Zusammenhang zwischen der $F_{D}$ Statistik und $T_{D}$ Statistik! \textbf{(2 Punkte)}
\item Visualisieren Sie in einer 2x2 Tafel den Zusammenhang von $MS_{treatment}$ und $MS_{error}$! \textbf{(2 Punkte)}
\item Beschriften Sie die erstellte 2x2 Tafel mit \underline{signifikant} und \underline{nicht signifikant}! Begründen Sie Ihre Antwort! \textbf{(2 Punkte)}
\item Nennen Sie das numerische Minimum der F-Statistik $F_D$! Begründen Sie Ihre Antwort! \textbf{(2 Punkte)}
\item Wenn die F-Statistik $F_D$ minimal ist, welche Aussage erhalten Sie über die Nullhypothese? Begründen Sie Ihre Antwort! \textbf{(2 Punkte)}
\end{enumerate}

 
\clearpage
% -----------------------------------------------------------------------

\section{Aufgabe \hfill (11 Punkte)}

\textit{Geben Sie grundsätzlich Formeln und Rechenweg zur Lösung der Teilaufgaben mit an!} \\[1Ex]
 

 
%% --------------------------------------------------------------------
\begin{minipage}[t]{0.5\textwidth}
\includegraphics[width = 1.3cm]{/Users/kruppajo/work/GitHub/exam/avatare/Alex.png}
\end{minipage}
\begin{minipage}[t]{0.5\textwidth}
\hfill
\href{https://youtu.be/2qG1Dws0MJo}{\includegraphics[width = 2cm]{img/youtube}}
\end{minipage}
\vspace{-3Ex}
%% --------------------------------------------------------------------



\paragraph{Interaktion in der zweifaktoriellen ANOVA}

In einen Leistungssteigerungsversuch wurden Milchvieh mit der Behandlung Lüftungssystem ($keins$, $storm$, $tornado$ und $thunder$) sowie der Behandlung Genotypen ($AA$ und $BB$) untersucht. Es wurde als Messwert Gewichtszuwachs in der 1LW bestimmt. Jetzt starrt Alex mit auf die \Rlogo Ausgabe einer zweifaktoriellen ANOVA. Leider starrt seine Betreuerin in der gleichen Art Alex zurück an. Es liegt anscheinend eine signifikante Interaktion vor? 'Das wird ein langer Nachmmittag.', denkt er sich und kreuselt seinen Mund. 'Und was machen wir jetzt?' entfährt es ihm überrascht entnervt. Immerhin war geht es ja um seine Abschlussarbeit. Alex hätte doch nichts mit Milchvieh machen sollen. Milchvieh -- was soll das auch bedeutendes sein? Eigentlich wollte Alex nachher noch einen Film schauen. Das Verrückte ist, dass die Katze Alien wirklich liebt. Das ist Alex sehr recht, denn er braucht Entspannung.

\vspace{1ex}

Leider kennen sich Alex und seine Betreuerin mit der zweifaktoriellen ANOVA überhaupt nicht aus. Geschweige denn mit der Interpretation einer Interaktion. Deshalb braucht er bei der Erstellung Ihre Hilfe, sonst wird es heute Abend mit seinem Hobby Starcraft nichts mehr! 

\begin{enumerate}
\item Visualisieren Sie folgende mögliche Interaktionen zwischen den Behandlungen! Beschriften Sie die Abbildung! \textbf{(4 Punkte)}
\begin{enumerate}
\item \underline{Keine} Interaktion liegt vor.
\item Eine \underline{schwache} Interaktion liegt vor. 
\item Eine \underline{starke} Interaktion liegt vor. 
\end{enumerate}
\item Erklären Sie den Unterschied zwischen den verschiedenen Interaktionen! \textbf{(2 Punkte)}
\item Welche statistische Maßzahl betrachten Sie für die Bewertung der Interaktion? \textbf{(1 Punkt)}
\item Skizzieren Sie die notwendigen Funktionen in \Rlogo für eine Post-hoc Analyse! \textbf{(2 Punkte)} 
\item Wenn eine signifikante Interaktion in den Daten vorliegt, wie ist dann das weitere Vorgehen? Berücksichtigen Sie auch die Funktion \texttt{emmeans()}! \textbf{(2 Punkte)}
\end{enumerate}

 
\clearpage
% -----------------------------------------------------------------------

\section{Aufgabe \hfill (11 Punkte)}

\textit{Geben Sie grundsätzlich Formeln und Rechenweg zur Lösung der Teilaufgaben mit an!} \\[1Ex]
 

 
%% --------------------------------------------------------------------
\begin{minipage}[t]{0.5\textwidth}
\includegraphics[width = 1.3cm]{/Users/kruppajo/work/GitHub/exam/avatare/Tina.png}
\end{minipage}
\begin{minipage}[t]{0.5\textwidth}
\hfill
\href{https://youtu.be/M9Uhm67ndxM}{\includegraphics[width = 2cm]{img/youtube}}
\end{minipage}
\vspace{-3Ex}
%% --------------------------------------------------------------------



\paragraph{Zusammenhang zwischen der ANOVA und dem Post-hoc-Test}

Es ist schon kurz nach fünf und Tina wird langsam nervös. Tina wollte heute Abend noch ihre E-Sport Qualifikation schauen. Hoffentlich kommt sie noch rechtzeitig zum Streamen. Angestrengend krault sie die Spinne. Stattdessen versucht ihr Betreuer die Ausgabe der einfaktoriellen ANOVA zu visualieren und zu überprüfen, ob es mit der Visualisierung der Daten als Boxplots zusammenpasst. Anscheinend gibt es ein Problem mit der Annahme der Normalverteilung und der Varianzhomogenität der ANOVA in den Daten. 'Wir haben jetzt bei der ANOVA einen p-Wert mit 0.058 raus sowie eine F-Statistik $F_D$ mit 1.2 berechnet. Nach den Boxplots müsste sich eigentlich ein Unterschied zwischen $getIt$ und $ctrl$ ergeben. Der Unterschied ist in \texttt{\{emmeans\}} auch signifikant mit einem p-Wert von 0.049. Wie kann das sein?', grummelt ihr Betreuer. Tina hatte in der Uckermark ein Stallexperiment mit Schweinen durchgeführt. Dabei wurden die Daten $D$ erhoben. Es gab dabei eine Behandlungen Ernährungszusatz ($ctrl$, $fedX$ und $getIt$). Gemessen wurde der Messwert ($Y$) Gewichtszuwachs in der 1LW. So kompliziert kann das jetzt doch nicht sein! Tina hat schon genug Probleme. Wenn die Wut nicht wäre, dann wäre es einfacher.

\begin{graybox}{Gegebene Formeln}
\begin{center}
  \begin{tabular}{ccc}
    $MS_{treatment} = \cfrac{SS_{treatment}}{df_{treatment}}$ &
    $MS_{error} = \cfrac{SS_{error}}{df_{error}}$ &
    $F_{D} = \cfrac{MS_{treatment}}{MS_{error}}$ \\
  \end{tabular}
\end{center}
\end{graybox}

Leider kennen sich Tina und ihr Betreuer mit der Interpretation einer ANOVA überhaupt nicht aus. Deshalb braucht sie bei der Erstellung Ihre Hilfe und die Zeit wird knapp. 

\begin{enumerate}
  \item Formulieren Sie die wissenschaftliche Fragestellung! \textbf{(1 Punkt)}
  \item Formulieren Sie das statistische Hypothesenpaar! \textbf{(1 Punkt)}
\item Was bedeutet eine signifkante ANOVA für die beobachteten Daten $D$? \textbf{(1 Punkt)}
\item Visualisieren Sie den Unterschied zwischen Varianzhomogenität und Varianzheterogenität anhand der Daten $D$! Beschriften Sie die Abbildung! \textbf{(2 Punkte)} 
\item Visualisieren Sie für die Daten $D$ die Verletzung der Annahme der Varianzhomogenität der ANOVA unter zu Hilfenahme von Boxplots! Beschriften Sie die Abbildung! \textbf{(2 Punkte)}
\item Welche Auswirkung hat die Verletzung der Annahme der Varianzhomogenität für die Teststatistik $F_D$ der ANOVA? Begründen Sie Ihre Antwort! \textbf{(2 Punkte)}
\item Erklären Sie abschließend die Diskrepanz zwischen den Ergebnis der ANOVA und dem paarweisen Gruppenvergleich in \texttt{\{emmeans\}}! \textbf{(2 Punkte)}
\end{enumerate}

 
\clearpage
% -----------------------------------------------------------------------
\part{Multiple Gruppenvergleiche}
% ----------------------------------------------------------------------- 

\section{Aufgabe \hfill (12 Punkte)}

\textit{Geben Sie grundsätzlich Formeln und Rechenweg zur Lösung der Teilaufgaben mit an!} \\[1Ex]
 

 
%% --------------------------------------------------------------------
\begin{minipage}[t]{0.5\textwidth}
\includegraphics[width = 1.3cm]{/Users/kruppajo/work/GitHub/exam/avatare/Jessica.png}\hspace{-4mm}\includegraphics[width = 1.3cm]{/Users/kruppajo/work/GitHub/exam/avatare/Mark.png}
\end{minipage}
\begin{minipage}[t]{0.5\textwidth}
\hfill
\href{https://youtu.be/kHmfEmU6lrk}{\includegraphics[width = 2cm]{img/youtube}}
\end{minipage}
%% --------------------------------------------------------------------



\paragraph{Adjustierung multipler Vergleiche}

In ein Freilandversuch mit Maiss wurde die Behandlung Lüftungssysteme ($ctrl$, $storm$, $thunder$ und $tornado$) gegen die Ergebnisse einer früheren Studie von Qui et al. (2017) verglichen. Im Rahmen des Experiments haben Mark und Jessica verschiedene Student t-Tests für den Mittelwertsvergleich für den Messwert Trockengewicht [kg/ha] gerechnet. Es ergab sich dann die folgende Tabelle der rohen p-Werte für die Vergleiche zu Qui et al. (2017). Jetzt sollen die beiden einmal schauen, was in den Daten so drin ist.

\begin{knitrout}
\definecolor{shadecolor}{rgb}{0.969, 0.969, 0.969}\color{fgcolor}\begin{table}[!h]
\centering\begingroup\fontsize{10}{12}\selectfont

\begin{tabular}{ccc}
\toprule
\textbf{Rohen p-Werte} & \textbf{Adjustierte p-Werte} & \textbf{Nullhypothese ablehnen?}\\
\midrule
0.34 &  & \\
0.03 &  & \\
0.07 &  & \\
0.76 &  & \\
\bottomrule
\end{tabular}
\endgroup{}
\end{table}

\end{knitrout}

Leider kennen sich Mark und Jessica mit der Adjustierung von $p$-Werten und dem Signifikanzniveau $\alpha$ überhaupt nicht aus. Deshalb brauchen die beiden bei der Erstellung Ihre Hilfe!

\begin{enumerate}
  \item Formulieren Sie die wissenschaftliche Fragestellung! \textbf{(1 Punkt)}
  \item Formulieren Sie die statistischen Hypothesen! \textbf{(1 Punkt)}
\item Füllen Sie die Spalte \textit{Adjustierte p-Werte} nach der Bonferoni-Methode aus! \textbf{(2 Punkte)}
\item Entscheiden Sie, ob nach der Adjustierung die Nullhypothese abgelehnt werden kann! Begründen Sie Ihre Antwort! \textbf{(2 Punkte)}
\item Wie ist Ihr Vorgehen, wenn Sie anstatt der $p$-Werte das Signifikanzniveau $\alpha$ adjustieren? \textbf{(2 Punkte)}
\item Erklären Sie warum die $p$-Werte oder das Signifikanzniveau $\alpha$ bei multiplen Vergleichen adjustiert werden müssen! \textbf{(2 Punkte)}
\item Würden Sie die Adjustierung der $p$-Werte oder die Adjustierung des Signifikanzniveaus $\alpha$ vorziehen? Begründen Sie Ihre Antwort! \textbf{(2 Punkte)}
\end{enumerate}


 
\clearpage
% ----------------------------------------------------------------------- 

\section{Aufgabe \hfill (10 Punkte)}

\textit{Geben Sie grundsätzlich Formeln und Rechenweg zur Lösung der Teilaufgaben mit an!} \\[1Ex]
 

 
%% --------------------------------------------------------------------
\begin{minipage}[t]{0.5\textwidth}
\includegraphics[width = 1.3cm]{/Users/kruppajo/work/GitHub/exam/avatare/Paula.png}
\end{minipage}
\begin{minipage}[t]{0.5\textwidth}
\hfill
\href{https://youtu.be/xq29O8qDrg0}{\includegraphics[width = 2cm]{img/youtube}}
\end{minipage}
\vspace{-3ex}
%% --------------------------------------------------------------------



\paragraph{Visualisierung des Compact Letter Displays (CLD)}

Paula sitzt schon etwas länger bei ihrer Betreuerin. So langsam macht Paula sich Gedanken, ob sie nicht doch mal anmerken sollte, dass sie von CLD noch nie was gehört hat. Aber noch kann gelauscht werden, ein Ende ist erstmal nicht in Sicht! Paula hatte in ihrer Abschlussarbeit einen Leistungssteigerungsversuch durchgeführt. Deshalb sitzt sie hier. Also eigentlich nein, deshalb nicht. Paula will fertig werden. Hat sie sich doch mit Genotypen ($00$, $AA$, $AB$ und $BB$) und Gewichtszuwachs in der 1LW schon eine Menge angeschaut. Paula beugt sich leicht nach vorne. Nein, doch keine Pause. Weiter warten auf eine Lücke im Fluss...

\begin{knitrout}
\definecolor{shadecolor}{rgb}{0.969, 0.969, 0.969}\color{fgcolor}\begin{table}[!h]
\centering\begingroup\fontsize{10}{12}\selectfont

\begin{tabular}{cc}
\toprule
\textbf{Behandlung} & \textbf{Compact letter display}\\
\midrule
00 & a\\
AA & a\\
AB & a\\
BB & a\\
\bottomrule
\end{tabular}
\endgroup{}
\end{table}

\end{knitrout}

Leider kennen sich Paula mit dem \textit{Compact letter display (CLD)} überhaupt nicht aus. Deshalb braucht sie bei der Erstellung Ihre Hilfe!

\begin{enumerate}
  \item Formulieren Sie die wissenschaftliche Fragestellung! \textbf{(1 Punkt)}
  \item Formulieren Sie die statistischen Hypothesen! \textbf{(1 Punkt)}
\item Zeichnen Sie die sich anhand des \textit{Compact letter display (CLD)} ergebenden Barplots! \textbf{(2 Punkte)}
\item Ergänzen Sie das \textit{Compact letter display (CLD)} zu den Barplots! \textbf{(1 Punkt)}
\item Erklären Sie \textit{einen} Vorteil und \textit{einen} Nachteil des \textit{Compact letter display (CLD)}! \textbf{(2 Punkte)}
\item Erstellen Sie eine Matrix mit den paarweisen $p$-Werten eines Student t-Tests, die sich näherungsweise aus dem \textit{Compact letter display (CLD)} ergeben würde! Begründen Sie Ihre Antwort! \textbf{(3 Punkte)}
\end{enumerate}

 
\clearpage
% ----------------------------------------------------------------------- 

\section{Aufgabe \hfill (12 Punkte)}

\textit{Geben Sie grundsätzlich Formeln und Rechenweg zur Lösung der Teilaufgaben mit an!} \\[1Ex]
 

 
%% --------------------------------------------------------------------
\begin{minipage}[t]{0.5\textwidth}
\includegraphics[width = 1.3cm]{/Users/kruppajo/work/GitHub/exam/avatare/Alex.png}
\end{minipage}
\begin{minipage}[t]{0.5\textwidth}
\hfill
\href{https://youtu.be/RagTFFKFbFg}{\includegraphics[width = 2cm]{img/caution}}
\end{minipage}
\vspace{-3ex}
%% --------------------------------------------------------------------



\paragraph{Berechnung des Compact Letter Displays (CLD) anhand von t-Tests}

Alex hatte in seine Abschlussarbeit ein Stallexperiment durchgeführt. Soweit so gut. Dabei hat er sich mit Zandern beschäftigt. Angeblich der neueste heiße Kram... aber das ist wiederum was anderes. So richtig mitgenommen hat Alex das Thema dann doch nicht. Hat er sich doch mit Bestandsdichte ($effizient$, $standard$, $eng$ und $kontakt$) und Fettgehalt [\%/kg] schon eine Menge an Daten angeschaut. Nach seine Betreuerin soll er nun ein CLD bestimmen. Weder weiß er was ein CLD ist, noch war sein erster Gedanke mit Köln und die LGBTQ Community richtig... Als erstes solle er die Gruppen nach absteigender Effektstärke sortieren. Was immer das jetzt bringen soll.

\begin{knitrout}
\definecolor{shadecolor}{rgb}{0.969, 0.969, 0.969}\color{fgcolor}\begin{table}[!h]
\centering\begingroup\fontsize{10}{12}\selectfont

\begin{tabular}{cccc}
\toprule
\textbf{Bestandsdichte} & \textbf{Fallzahl (n)} & \textbf{Mittelwert} & \textbf{Standardabweichung}\\
\midrule
effizient & 7 & 15.44 & 1.54\\
standard & 9 & 9.38 & 2.02\\
eng & 8 & 9.65 & 2.14\\
kontakt & 8 & 5.78 & 2.14\\
\bottomrule
\end{tabular}
\endgroup{}
\end{table}

\end{knitrout}

Leider kennen sich Alex mit dem \textit{Compact letter display (CLD)} überhaupt nicht aus. Deshalb braucht er bei der Erstellung Ihre Hilfe!

\begin{enumerate}
  \item Formulieren Sie die wissenschaftliche Fragestellung! \textbf{(1 Punkt)}
  \item Formulieren Sie die statistischen Hypothesen! \textbf{(1 Punkt)}
\item Zeichnen Sie die sich ergebenden Barplots! \textbf{(1 Punkt)}
\item Berechnen Sie die Matrix der $p$-Werte anhand von Student t-Tests! \textbf{(4 Punkte)}
\item Ergänzen Sie das \textit{Compact letter display (CLD)} zu den gezeichneten Barplots! Begründen Sie Ihre Antwort! \textbf{(4 Punkte)}
\item Interpretieren Sie das \textit{Compact letter display (CLD)} für Alex und Jessica! \textbf{(1 Punkt)} 
\end{enumerate}

 
\clearpage
% -----------------------------------------------------------------------

\section{Aufgabe \hfill (10 Punkte)}

\textit{Geben Sie grundsätzlich Formeln und Rechenweg zur Lösung der Teilaufgaben mit an!} \\[1Ex]
 

 
%% --------------------------------------------------------------------
\begin{minipage}[t]{0.5\textwidth}
\includegraphics[width = 1.3cm]{/Users/kruppajo/work/GitHub/exam/avatare/Nilufar.png}
\end{minipage}
\begin{minipage}[t]{0.5\textwidth}
\hfill
\href{https://youtu.be/RagTFFKFbFg}{\includegraphics[width = 2cm]{img/youtube}}
\end{minipage}
\vspace{-3ex}
%% --------------------------------------------------------------------



\paragraph{Berechnung des Compact Letter Displays (CLD) anhand der Matrix der p-Werte}

'Das Problem ist, dass SPSS eben keine CLD kann. Die bräuchten wir dann schon dringend für unser Poster!', merkt ihr Betreuer mit Nachdruck an. Nilufar neigt den Kopf. 'Das wussten wir nicht vorher?', entfährt es ihr leicht entnervt. Da schaut ihr Betreuer seltsam betroffen. Hilft jetzt auch so gar nicht. Nilufar hatte sich zwei Variablen mit Lüftungssystem ($keins$, $storm$, $tornado$ und $thunder$) und Protein/Fettrate [\%/kg] in ein Kreuzungsexperiment mit Milchvieh angeschaut. Jetzt möchte sie eigentlich fertig werden und nicht nochmal alles neu in \Rlogo und \texttt\{emmeans\} machen. Dabei hatte sie schon echt ne Menge in in der Uckermark gemacht. Dann eben per Hand aus der Matrix der $p$-Wert. Nilufar muss sich echt zusammenreißen.

\begin{knitrout}
\definecolor{shadecolor}{rgb}{0.969, 0.969, 0.969}\color{fgcolor}\begin{table}[!h]
\centering\begingroup\fontsize{10}{12}\selectfont

\begin{tabular}{>{}lcccc}
\toprule
\textbf{ } & \textbf{keins} & \textbf{storm} & \textbf{tornado} & \textbf{thunder}\\
\midrule
\textbf{keins} & 1.0000000 & 0.0320016 & 0.2263227 & 0.1101569\\
\textbf{storm} & 0.0320016 & 1.0000000 & 0.2783586 & 0.4287787\\
\textbf{tornado} & 0.2263227 & 0.2783586 & 1.0000000 & 0.7131588\\
\textbf{thunder} & 0.1101569 & 0.4287787 & 0.7131588 & 1.0000000\\
\bottomrule
\end{tabular}
\endgroup{}
\end{table}

\end{knitrout}

Leider kennen sich Nilufar mit dem \textit{Compact letter display (CLD)} überhaupt nicht aus. Deshalb braucht sie bei der Erstellung Ihre Hilfe!

\begin{enumerate}
  \item Formulieren Sie die wissenschaftliche Fragestellung! \textbf{(1 Punkt)}
  \item Formulieren Sie die statistischen Hypothesen! \textbf{(1 Punkt)}
\item Zeichnen Sie die sich anhand der Matrix der $p$-Werte ergebenden Barplots! \textbf{(2 Punkte)}
\item Ergänzen Sie das \textit{Compact letter display (CLD)}! Begründen Sie Ihre Antwort! \textbf{(4 Punkte)}
\item Interpretieren Sie das \textit{Compact letter display (CLD)} für Nilufar und Jessica! \textbf{(2 Punkte)} 
\end{enumerate}

 
\clearpage
% -----------------------------------------------------------------------
\part{Der Chi-Quadrat-Test \& Der diagnostische Test}
% -----------------------------------------------------------------------

\section{Aufgabe \hfill (12 Punkte)}

\textit{Geben Sie grundsätzlich Formeln und Rechenweg zur Lösung der Teilaufgaben mit an!} \\[1Ex]
 

 
%% --------------------------------------------------------------------
\begin{minipage}[t]{0.5\textwidth}
\includegraphics[width = 1.3cm]{/Users/kruppajo/work/GitHub/exam/avatare/Steffen.png}
\end{minipage}
\begin{minipage}[t]{0.5\textwidth}
\hfill
\href{https://youtu.be/-Kva5wc5Elw}{\includegraphics[width = 2cm]{img/youtube}}
\end{minipage}
\vspace{-3Ex}
%% --------------------------------------------------------------------



\paragraph{Den Chi-Quadrat-Test berechnen}

Steffen hat sich ein Herz gefasst und war für seinem Projektbericht in die Niederlande gegangen. Das war eine super Zeit in der er viel gelernt hat. Klar gab es auch die ein oder andere Besonderheit, aber das gehört hier eher nicht hin. Dann noch schnell Taylor Swift auf das Ohr und los gehts. Steffen ist schon eine ganze Zeit im Büro, da sein Betreuer möchte, dass er jetzt auf seinen Daten mit $n = 118$ Beobachtungen von Maiss einen $\mathcal{X}^2$-Test rechnet. Das ginge, da er als Behandlung \textit{Herbizideinsatz [ja/nein]} bestimmt und zum anderen die Variable \textit{Trockengewicht über Zielwert [ja/nein]} ermittelt hat. Wie genau, das ist jetzt eine andere Frage. Eigentlich wollte Steffen nachher noch einen Film schauen. Wenn Harry Potter läuft, dann ist die Schlange nicht mehr da. Aber jetzt braucht er mal Entspannung!

\vspace{5Ex}

\begin{center}
  \huge
  \begin{tabular}{c|c|c|c}
     & \phantom{\textbf{Erkrankt (ja)}} & \phantom{\textbf{Erkrankt (ja)}} & \phantom{\textbf{Erkrankt (ja)}} \strut\\
    \hline
    \phantom{\textbf{Pestizid (ja)}} & 38  & 19  &     \strut\\
    \hline
    \phantom{\textbf{Pestizid (ja)}} & 23  & 38  &      \strut\\
    \hline
     \phantom{100} & \phantom{100}  & \phantom{100}  &  \phantom{100}  \strut\\
  \end{tabular}
\end{center}

\vspace{5Ex}

Leider kennt sich Steffen mit der Berechnung eines $\mathcal{X}^2$-Test für kategoriale Daten überhaupt nicht aus. Deshalb braucht er bei der Erstellung Ihre Hilfe!

\begin{enumerate}
\item Formulieren Sie die wissenschaftliche Fragestellung! \textbf{(1 Punkt)}
\item Ergänzen Sie die Tabelle um die fehlenden Informationen! \textbf{(1 Punkt)} 
\item Visualisieren Sie den Zusammenhang zwischen den beiden kategorialen Variablen! \textbf{(2 Punkte)}
\item Berechnen Sie die Teststatistik eines Chi-Quadrat-Test! \textbf{(2 Punkte)}
\item Treffen Sie eine Entscheidung im Bezug zu der Nullhypothese gegeben
  einem $\mathcal{X}^2_{\alpha = 5\%} = 3.841$! Begründen Sie Ihre Antwort!
  \textbf{(2 Punkte)}
\item Skizzieren Sie die $\mathcal{X}^2$-Verteilung, wenn die $H_0$ wahr ist! Ergänzen Sie  $\mathcal{X}^2_{\alpha = 5\%}$ und $\mathcal{X}^2_{D}$ in der Abbildung! \textbf{(2 Punkte)}
\item Berechnen Sie den Effektschätzer $Cramers\; V$! Interpretieren Sie den
  Effektschätzer! \textbf{(2 Punkte)}
\end{enumerate} 
\clearpage
% -----------------------------------------------------------------------

\section{Aufgabe \hfill (10 Punkte)}

\textit{Geben Sie grundsätzlich Formeln und Rechenweg zur Lösung der Teilaufgaben mit an!} \\[1Ex]
 

 
%% --------------------------------------------------------------------
\begin{minipage}[t]{0.5\textwidth}
\includegraphics[width = 1.3cm]{/Users/kruppajo/work/GitHub/exam/avatare/Jessica.png}
\end{minipage}
\begin{minipage}[t]{0.5\textwidth}
\hfill
\href{https://youtu.be/jakM7fHyZfU}{\includegraphics[width = 2cm]{img/youtube}}
\end{minipage}
\vspace{-3Ex}
%% --------------------------------------------------------------------



\paragraph{Der Chi-Quadrat-Test konzeptionell verstehen}

Am Ende hätte Jessica dann doch einen normalverteilten Endpunkt in ihrer Hausarbeit nehmen sollen. Dann noch schnell Schokobons zur Stärkung und los gehts. Vor ihr liegen jetzt die Daten von zwei Variablen als Kategorien oder wie es in \Rlogo so schön heißt, als Faktoren. Aber immerhin, hofft sie das was bei den Daten rausgekommen ist. Gezählt hat Jessica einiges mit $n = 151$ Beobachtungen von Milchvieh. Zum einen hat sie als Behandlung \textit{Klimakontrolle [ja/nein]} bestimmt und zum anderen die Variable \textit{Gewichtszuwachs erreicht [ja/nein]} ermittelt. Nun möchte ihr Betreuer gerne einen $\mathcal{X}^2$-Test auf einer $2x2$-Kreuztabelle berechnet bekommen. Am Ende des Tages möchte sie dann noch ihr Hobby Warhammer genießen. Das muss auch mal sein!

\vspace{5Ex}

\begin{center}
  \huge
  \begin{tabular}{c|c|c|c}
     & \phantom{\textbf{Erkrankt (ja)}} & \phantom{\textbf{Erkrankt (ja)}} & \phantom{\textbf{Erkrankt (ja)}} \strut\\
    \hline
   \phantom{\textbf{Pestizid (ja)}} & \phantom{100}  & \phantom{100}  &   75  \strut\\
    \hline
    \phantom{\textbf{Pestizid (ja)}} & \phantom{100}  & \phantom{100}  &   76   \strut\\
    \hline
     &  79 &  72 &  151  \strut\\
  \end{tabular}
\end{center}

\vspace{5Ex}

Leider kennt sich Jessica mit der Berechnung eines $\mathcal{X}^2$-Test für kategoriale Daten überhaupt nicht aus. Deshalb braucht sie bei der Erstellung Ihre Hilfe!

\begin{enumerate}
  \item Formulieren Sie die wissenschaftliche Fragestellung! \textbf{(1 Punkt)}
\item Ergänzen Sie die Tabelle um die fehlenden Informationen! \textbf{(1 Punkt)} 
\item Ergänzen Sie die Felder innerhalb der $2x2$ Kreuztabelle, so dass \textit{ein} signifikanter Effekt zu erwarten wäre! \textbf{(2 Punkte)}
\item Begründen Sie Ihr Vorgehen an der Formel des Chi-Quadrat-Tests. Erklären Sie Ihr Vorgehen an einem Beispiel! \textbf{(2 Punkte)}
\item Visualisieren Sie den Zusammenhang zwischen den beiden kategorialen Variablen! \textbf{(2 Punkte)}
\item Was ist die Mindestanzahl an Beobachtungen je Zelle? Wenn in einer der Zellen weniger Beobachtungen auftreten, welchen Test können Sie anstatt des Standard Chi-Quadrat-Tests anwenden? \textbf{(2 Punkte)}
\end{enumerate} 
\clearpage
% -----------------------------------------------------------------------

\section{Aufgabe \hfill (10 Punkte)}

\textit{Geben Sie grundsätzlich Formeln und Rechenweg zur Lösung der Teilaufgaben mit an!} \\[1Ex]
 

 
%% --------------------------------------------------------------------
\begin{minipage}[t]{0.5\textwidth}
\includegraphics[width = 1.3cm]{/Users/kruppajo/work/GitHub/exam/avatare/Yuki.png}
\end{minipage}
\begin{minipage}[t]{0.5\textwidth}
\hfill
\href{https://youtu.be/ghArbetOr_E}{\includegraphics[width = 2cm]{img/youtube}}
\end{minipage}
\vspace{-3Ex}
%% --------------------------------------------------------------------



\paragraph{Der Chi-Quadrat-Test in \Rlogo}


'Der $\mathcal{X}^2$-Test auf einer $2x2$-Kreuztabelle berechnet.', liest Yuki in ihrer Mitschrift. So richtig helfen tut ihr das jetzt eherlichweise dann doch nicht. Dann noch schnell Matrix starten und los gehts mit der Kraft von Reese's Peanut Butter Cups. Yuki hatte sich in ein Kreuzungsexperiment $n = 151$ Beobachtungen von Schweinen angeschaut. Dabei hat er als Behandlung \textit{Außenklimakontakt [ja/nein]} bestimmt und zum anderen die Variable \textit{Fettgehalt erreicht [ja/nein]} ermittelt. Am Ende möchte dann ihr Betreuer gerne einen $\mathcal{X}^2$-Test auf einer $2x2$-Kreuztabelle berechnet bekommen. Nach ihrem Experiment erhielt sie folgende $2x2$ Kreuztabelle aus ihren erhobenen Daten.

\begin{knitrout}
\definecolor{shadecolor}{rgb}{0.969, 0.969, 0.969}\color{fgcolor}\begin{kframe}
\begin{verbatim}
##                    Außenklimakontakt
## Fettgehalt erreicht ja nein
##                ja   13    5
##                nein  7   18
\end{verbatim}
\end{kframe}
\end{knitrout}

Dann rechnete Yuki den Fisher-Exakt-Test auf der $2x2$-Kreuztabelle in \Rlogo und erhielt folgende \Rlogo Ausgabe der Funktion \texttt{fisher.test()}.

\begin{knitrout}
\definecolor{shadecolor}{rgb}{0.969, 0.969, 0.969}\color{fgcolor}\begin{kframe}
\begin{verbatim}
## 
## 	Fisher's Exact Test for Count Data
## 
## data:  Fettgehalt erreicht
## p-value = 0.005898
## alternative hypothesis: true odds ratio is not equal to 1
## 95 percent confidence interval:
##   1.462677 32.500828
## sample estimates:
## odds ratio 
##   6.352594
\end{verbatim}
\end{kframe}
\end{knitrout}

Leider kennt sich Yuki mit der Berechnung eines $\mathcal{X}^2$-Test für kategoriale Daten überhaupt nicht aus. Deshalb braucht sie bei der Erstellung Ihre Hilfe!

\begin{enumerate}
\item Formulieren Sie die wissenschaftliche Fragestellung! \textbf{(1 Punkt)}
\item Visualisieren Sie den Zusammenhang zwischen den beiden kategorialen Variablen! \textbf{(2 Punkte)}
\item Liegt ein signifikanter Unterschied zwischen den Gruppen vor? Begründen Sie Ihre Antwort! \textbf{(2 Punkte)}
\item Skizzieren Sie das sich ergebende 95\% Konfidenzintervall! \textbf{(2 Punkte)}
\item Beschriften Sie die Abbildung des 95\% Konfidenzintervalls! \textbf{(1 Punkt)} 
\item Interpretieren Sie das \textit{Odds ratio} im Kontext der wissenschaftlichen Fragestellung! \textbf{(2 Punkte)} 
\end{enumerate}
 
\clearpage
% -----------------------------------------------------------------------

\section{Aufgabe \hfill (11 Punkte)}

\textit{Geben Sie grundsätzlich Formeln und Rechenweg zur Lösung der Teilaufgaben mit an!} \\[1Ex]
 

 
%% --------------------------------------------------------------------
\begin{minipage}[t]{0.5\textwidth}
\includegraphics[width = 1.3cm]{/Users/kruppajo/work/GitHub/exam/avatare/Nilufar.png}\hspace{-4mm}\includegraphics[width = 1.3cm]{/Users/kruppajo/work/GitHub/exam/avatare/Yuki.png}
\end{minipage}
\begin{minipage}[t]{0.5\textwidth}
\hfill
\href{https://youtu.be/VQlNl8hvRII}{\includegraphics[width = 2cm]{img/youtube}}
\end{minipage}
%% --------------------------------------------------------------------



\paragraph{Den diagnostische Test am Doppelbaum berechnen}

Yuki liest laut vor. 'Die Prävalenz von Klauenseuche bei Fleischrindern wird mit 4\% angenommen. In 80\% der Fälle ist ein Test positiv, wenn das Tier erkrankt ist. In 7.5\% der Fälle ist ein Test positiv, wenn das Tier \underline{nicht} erkrankt ist und somit gesund ist. Wir führen einen Test auf West-Nil-Virus an 2000 Fleischrindern mit einem diagnostischen Test durch.' Nilufar klappt die Kinnlade runter. In der Stille duddelt London Grammar. Yuki schaut kompetent und schmeißt sich mit offenen Mund Takis Blue Heat an den Kopf vorbei.

\begin{center}
  \includegraphics[width=17cm]{/Users/kruppajo/work/GitHub/exam/question/img/diag-doppelbaum}
\end{center}

Leider kennen sich Yuki und Nilufar mit dem diagnostischen Testen überhaupt nicht aus. Deshalb brauchen beide bei der Erstellung Ihre Hilfe! 
    
\begin{enumerate}
\item Beschriften Sie die Äste des Doppelbaumes, mit denen Ihnen bekannten Informationen! \textbf{(2 Punkte)}
\item Beschriften Sie den Doppelbaum! \textbf{(2 Punkte)}
\item Füllen Sie freien Felder des Doppelbaums aus! \textbf{(4 Punkte)}
\item Berechnen Sie die Wahrscheinlichkeit $Pr(K^+|T^+)$! \textbf{(2 Punkte)}
\item Was sagt Ihnen die Wahrscheinlichkeit $Pr(K^+|T^+)$ aus? \textbf{(1 Punkt)}
\end{enumerate}






 
\clearpage
% -----------------------------------------------------------------------

\section{Aufgabe \hfill (11 Punkte)}

\textit{Geben Sie grundsätzlich Formeln und Rechenweg zur Lösung der Teilaufgaben mit an!} \\[1Ex]
 

 
%% --------------------------------------------------------------------
\begin{minipage}[t]{0.5\textwidth}
\includegraphics[width = 1.3cm]{/Users/kruppajo/work/GitHub/exam/avatare/Alex.png}\hspace{-4mm}\includegraphics[width = 1.3cm]{/Users/kruppajo/work/GitHub/exam/avatare/Jessica.png}
\end{minipage}
\begin{minipage}[t]{0.5\textwidth}
\hfill
\href{https://youtu.be/_7s44pbOc00}{\includegraphics[width = 2cm]{img/youtube}}
\end{minipage}
%% --------------------------------------------------------------------



\paragraph{Der diagnostische Test und statistische Maßzahlen}

'Was ist denn das?', entfährt es Alex. 'Hm... ich glaube es handelt sich um einen Doppelbaum, den wir beim diagnostischen Testen brauchen.', meint Jessica und dreht Alien auf dem Second Screen etwas leiser. Was jetzt beide von einem diagnostischen Test haben, ist ihnen auch nicht klar. Es ist ja schon alles komplex genug und die Gefälligkeit von Alex macht es heute auch nicht mehr einfacher. 'Es geht um Schorf an Erdbeeren.', stellt Jessica fest. Eigentlich wollte Jessica eher los um Rad zu fahren. Das wird aber wohl nichts mehr.

\begin{tikzpicture}
  \node (image) at (0,0) {
    \includegraphics[width=\textwidth]{/Users/kruppajo/work/GitHub/exam/question/img/diag-doppelbaum}
  };
  \node[] at (-4.8,0) {\huge 160};
  \node[] at (-1.7,0) {\huge 20};
  \node[] at (1.7,0) {\huge 750};
  \node[] at (4.75,0) {\huge 1450};
\end{tikzpicture}

Leider kennen sich Alex und Jessica mit dem diagnostischen Testen überhaupt nicht aus. Deshalb brauchen beide bei der Erstellung Ihre Hilfe! 
  
\begin{enumerate}
\item Beschriften Sie den Doppelbaum! \textbf{(2 Punkte)}
\item Füllen Sie freien Felder des Doppelbaums aus! \textbf{(4 Punkte)}
\item Berechnen Sie die Wahrscheinlichkeit $Pr(K^+|T^+)$! \textbf{(2 Punkte)}
\item Berechnen Sie die Prävalenz für Klauenseuche! \textbf{(1 Punkt)}
\item Berechnen Sie die Sensifität und Spezifität des diagnostischen Tests! Erstellen Sie dafür zunächst eine 2x2 Kreuztabelle! \textbf{(2 Punkte)}
\end{enumerate}

 





 
\clearpage
% -----------------------------------------------------------------------
\part{Lineare Regression \& Korrelation}
% -----------------------------------------------------------------------

\section{Aufgabe \hfill (10 Punkte)}

\textit{Geben Sie grundsätzlich Formeln und Rechenweg zur Lösung der Teilaufgaben mit an!} \\[1Ex]
 

 
%% --------------------------------------------------------------------
\begin{minipage}[t]{0.5\textwidth}
\includegraphics[width = 1.3cm]{/Users/kruppajo/work/GitHub/exam/avatare/Alex.png}\hspace{-4mm}\includegraphics[width = 1.3cm]{/Users/kruppajo/work/GitHub/exam/avatare/Mark.png}
\end{minipage}
\begin{minipage}[t]{0.5\textwidth}
\hfill
\href{https://youtu.be/kHmfEmU6lrk}{\includegraphics[width = 2cm]{img/youtube}}
\end{minipage}
%% --------------------------------------------------------------------



\paragraph{Visualisierung der linearen Regression}

'Hä? Hatten wir das als Aufgabe nicht schon mal, das wir aus kontinuierlichen Daten eine Abbildung bauen sollten?', fragt Mark. Alex schaut fragend zurück. 'Kann mich wie immer an nichts erinnern. Können wir trotzdem jetzt erstmal die Daten auswerten? Columbo?', antwortet Alex leicht angespannt. Die beiden hatten ein Gewächshausexperiment im Wendland mit Lauch durchgeführt. Dabei wurden die beiden folgenden Variablen gemessen: durchschnittlichen Anteil an Ton [\%/l] und Proteingehalt [g/kg]. Jetzt haben die beiden eigentlich alles zusammen. \textit{Eigentlich...}

\begin{table}[!h]
\centering
\begin{tabular}{cc}
\toprule
Proteingehalt [g/kg] & Durchschnittlichen Anteil an Ton [\%/l]\\
\midrule
21.8 & 41.1\\
18.2 & 32.6\\
21.2 & 36.6\\
20.2 & 34.4\\
17.9 & 36.8\\
\addlinespace
16.8 & 34.1\\
18.2 & 35.8\\
18.4 & 37.7\\
20.0 & 30.0\\
17.9 & 30.2\\
\addlinespace
14.4 & 30.1\\
11.5 & 22.9\\
\bottomrule
\end{tabular}
\end{table}



Leider kennen sich Mark und Alex mit der linearen Regression für kontinuierliche Daten überhaupt nicht aus. Deshalb brauchen beide bei der Erstellung Ihre Hilfe!

\begin{enumerate}
\item Formulieren Sie die wissenschaftliche Fragestellung! \textbf{(1 Punkt)}
\item Erstellen  Sie  eine  Visualisierung  für  die  Datentabelle.  Beschriften  Sie  die  Achsen! \textbf{(2 Punkte)}
\item Schätzen Sie die Regressionsgleichung aus der obigen Abbildung ab! \textbf{(2 Punkte)}
\item Beschriften Sie die Grade mit den statistischen Maßzahlen der linearen Regressionsgleichung! \textbf{(2 Punkte)}
\item Liegt ein Zusammenhang zwischen $x$ und $y$ vor? Begründen Sie Ihre Antwort! \textbf{(2 Punkte)}
\item Wenn kein Zusammenhang zu beobachten wäre, wie würde die Grade aussehen? \textit{Antworten Sie mit einer Skizze der Geraden!} \textbf{(1 Punkt)}
\end{enumerate} 
\clearpage
% -----------------------------------------------------------------------

\section{Aufgabe \hfill (12 Punkte)}

\textit{Geben Sie grundsätzlich Formeln und Rechenweg zur Lösung der Teilaufgaben mit an!} \\[1Ex]
 

 
%% --------------------------------------------------------------------
\begin{minipage}[t]{0.5\textwidth}
\includegraphics[width = 1.3cm]{/Users/kruppajo/work/GitHub/exam/avatare/Jessica.png}\hspace{-4mm}\includegraphics[width = 1.3cm]{/Users/kruppajo/work/GitHub/exam/avatare/Jonas.png}
\end{minipage}
\begin{minipage}[t]{0.5\textwidth}
\hfill
\href{https://youtu.be/lJp8rFmMnrs}{\includegraphics[width = 2cm]{img/youtube}}
\end{minipage}
%% --------------------------------------------------------------------



\paragraph{Interpretation der Ergebnisse einer linearen Regression}


'Ich glaube du bringst da was durcheinander. Wir nutzen zwar auch für die ANOVA die Funktion \texttt{lm()} aber hier wollen wir, glaube ich, eine Gerade durch die Punkte zeichnen.', merkt Jessica an. 'Ich sehe keine Punkte... ich sehe nur zwei Zeilen einer Tabelle und ich glaube du hast gerade was gelöscht.', antwortet Jonas sichtlich übernächtigt. 'Wir müssen die Koeffizienten der linearen Regression ja auch erst interpretieren!', spricht Jessica sehr deutlich und langsam. Die beiden hatten ein Feldexperiment im Wendland mit Kartoffeln durchgeführt. Dabei wurden die beiden folgenden Variablen gemessen: durchschnittlicher Anteil an Ton [\%/l] und Chlorophyllgehalt (SPAD-502Plus) [SPAD]. Jetzt wollen sie erstmal schauen, ob es einen Zusammenhang gibt und das soll mit der \Rlogo Ausgabe möglich sein.

\begin{table}[!h]
\centering\begingroup\fontsize{12}{14}\selectfont

\begin{tabular}{ccccc}
\toprule
term & estimate & std.error & t statistic & p-value\\
\midrule
(Intercept) & 3.10 & 2.22 &  & \\
Durchschnittlicher Anteil & 0.39 & 0.22 &  & \\
\bottomrule
\end{tabular}
\endgroup{}
\end{table}



Leider kennen sich Jessica und Jonas mit der linearen Regression für kontinuierliche Daten in \Rlogo überhaupt nicht aus. Deshalb brauchen beide bei der Erstellung Ihre Hilfe!

\begin{enumerate}
\item Formulieren Sie die wissenschaftliche Fragestellung! \textbf{(1 Punkt)}
\item Erstellen  Sie  eine  Visualisierung  der \texttt{lm()}-Ausgabe.  Beschriften  Sie  die  Achsen! \textbf{(2 Punkte)}
\item Beschriften Sie die Visualisierung mit den statistischen Maßzahlen der der \texttt{lm()}-Ausgabe! \textbf{(2 Punkte)}
\item Formulieren Sie die Regressionsgleichung! \textbf{(1 Punkt)}
\item Ergänzen Sie die t Statistik in der \texttt{lm()}-Ausgabe! \textbf{(2 Punkte)}
\item Ergänzen Sie den $p$-Wert in der \texttt{lm()}-Ausgabe mit $T_{\alpha = 5\%} = 1.96$!  \textbf{(2 Punkte)}
\item Interpretieren Sie den $p$-Wert im Kontext der wissenschaftlichen Fragestellung! \textbf{(1 Punkt)}  
\item Wie groß ist der Effekt im Kontext der wissenschaftlichen Fragestellung? \textbf{(1 Punkt)}
\end{enumerate} 
\clearpage
% -----------------------------------------------------------------------

\section{Aufgabe \hfill (11 Punkte)}

\textit{Geben Sie grundsätzlich Formeln und Rechenweg zur Lösung der Teilaufgaben mit an!} \\[1Ex]
 

 
%% --------------------------------------------------------------------
\begin{minipage}[t]{0.5\textwidth}
\includegraphics[width = 1.3cm]{/Users/kruppajo/work/GitHub/exam/avatare/Alex.png}\hspace{-4mm}\includegraphics[width = 1.3cm]{/Users/kruppajo/work/GitHub/exam/avatare/Yuki.png}
\end{minipage}
\begin{minipage}[t]{0.5\textwidth}
\hfill
\href{https://youtu.be/tNNzcndrpSk}{\includegraphics[width = 2cm]{img/youtube}}
\end{minipage}
%% --------------------------------------------------------------------



\paragraph{Interpretation der Ergebnisse einer linearen Regression in \Rlogo}


'Wichtig ist es, dass wir jetzt eine Gerade durch die Punkte zeichnen!', ruft Yuki. 'Ich sehe nur Kauderwelsch und keine Punkte. Wie soll ich da denn jetzt eine Gerade durchzeichnen? Und warum überhaupt? War das unsere Fragestellung?', fragt Alex. Yuki atmet schwer ein und starrt auf die \Rlogo Ausgabe der Funktion \texttt{lm()}. Die beiden hatten ein Stallexperiment im Wendland mit Lamas durchgeführt. Dabei wurden die beiden folgenden Variablen gemessen: mittlere Eisenkonzentration [Fe/ml] und Proteianteil [\%/kg]. Jetzt will die Betreuung von den beiden die Interpretierung der Daten in Form einer linearen Regression gerechnet bekommen. Das haben beide in \Rlogo gemacht, aber wie soll das jetzt gehen? Das mit der Interpretation?

\begin{knitrout}
\definecolor{shadecolor}{rgb}{0.969, 0.969, 0.969}\color{fgcolor}\begin{kframe}
\begin{verbatim}
## 
## Call:
## Proteianteil ~ Mittlere_Eisenkonzentration
## 
## Residuals:
##     Min      1Q  Median      3Q     Max 
## -2.2713 -0.9313 -0.2962  1.0976  2.5894 
## 
## Coefficients:
##                             Estimate Std. Error t value Pr(>|t|)
## (Intercept)                   0.7937     1.4120   0.562    0.577
## Mittlere_Eisenkonzentration   2.1043     0.1418  14.843   <2e-16
## 
## Residual standard error: 1.305 on 39 degrees of freedom
## Multiple R-squared:  0.8496,	Adjusted R-squared:  0.8457 
## F-statistic: 220.3 on 1 and 39 DF,  p-value: < 2.2e-16
\end{verbatim}
\end{kframe}
\end{knitrout}

Leider kennen sich Yuki und Alex mit der linearen Regression für kontinuierliche Daten in \Rlogo überhaupt nicht aus. Deshalb brauchen beide bei der Erstellung Ihre Hilfe!


\begin{enumerate}
\item Formulieren Sie die wissenschaftliche Fragestellung! \textbf{(1 Punkt)}
\item Wie groß ist der Effekt im Kontext der wissenschaftlichen Fragestellung? \textbf{(2 Punkte)} 
\item Interpretieren Sie die $p$-Werte im Kontext der wissenschaftlichen Fragestellung! \textbf{(2 Punkte)}
\item Visualisieren Sie die Verteilung der Residuen! \textbf{(2 Punkte)} 
\item Ist die Annahme der Normalverteilung erfüllt? Begründen Sie die Antwort! \textbf{(2 Punkte)}
\item Erklären Sie \textit{kurz} den Begriff \texttt{R-squared}! Was sagt Ihnen der Wert aus? \textbf{(2 Punkte)}
\end{enumerate}
 
\clearpage
% -----------------------------------------------------------------------

\section{Aufgabe \hfill (10 Punkte)}

\textit{Geben Sie grundsätzlich Formeln und Rechenweg zur Lösung der Teilaufgaben mit an!} \\[1Ex]
 

 
%% --------------------------------------------------------------------
\begin{minipage}[t]{0.5\textwidth}
\includegraphics[width = 1.3cm]{/Users/kruppajo/work/GitHub/exam/avatare/Yuki.png}
\end{minipage}
\begin{minipage}[t]{0.5\textwidth}
\hfill
\href{https://youtu.be/C9skfFRTHhI}{\includegraphics[width = 2cm]{img/youtube}}
\end{minipage}
\vspace{-3ex}
%% --------------------------------------------------------------------



\paragraph{Interpretation der Ergebnisse einer Korrelationsanalyse in \Rlogo}

'Hä? Was ist denn das? Das wird ja immer wilder! Hatten wir als Aufgabe eine Korrelation zu berechnen? Wir bauen aus kontinuierlichen Daten eine Abbildung und interpretieren diese dann?', fragt sich Yuki laut. 'Keine Ahnung... das ist jetzt jedenfalls keine Abbildung von irgendwas sondern eine \Rlogo Ausgabe mit ganz wilden Bezeichnungen...', denkt sie. Aber das hilft jetzt auch nicht. Wenn Matrix läuft, dann ist das Minischwein nicht mehr da. Aber jetzt braucht sie mal Entspannung! Yuki hatte ein Stallexperiment im Wendland mit Lamas durchgeführt. Dabei wurden die beiden folgenden Variablen gemessen: durchschnittliche Tagestemperatur [C/d] und Gewichtszuwachs in der 1LW. Jetzt hat sie eigentlich alles zusammen. \textit{Eigentlich...}, denn mit der \Rlogo Ausgabe hat Yuki jetzt ein Problem. Yuki und die Faulheit, eine unendliche Geschichte mit kniffeligen Wendungen. 


\begin{knitrout}
\definecolor{shadecolor}{rgb}{0.969, 0.969, 0.969}\color{fgcolor}\begin{kframe}
\begin{verbatim}
## 
## 	Pearson's correlation
## 
## data:  Durchschnittliche Tagestemperatur and Gewichtszuwachs
## t = 2.6415, df = 8, p-value = 0.02965
## alternative hypothesis: true correlation is not equal to 0
## 95 percent confidence interval:
##  0.09278834 0.91776219
## sample estimates:
##       cor 
## 0.6825393
\end{verbatim}
\end{kframe}
\end{knitrout}

Leider kennt sich Yuki mit der Korrelationsanalyse in \Rlogo überhaupt nicht aus. Deshalb braucht sie bei der Erstellung Ihre Hilfe!

\begin{enumerate}
  \item Formulieren Sie die wissenschaftliche Fragestellung! \textbf{(1 Punkt)}
  \item Formulieren Sie das statistische Hypothesenpaar! \textbf{(1 Punkt)}
\item Erstellen Sie eine Visualisierung für den Korrelationskoeffizienten! Beschriften Sie die Abbildung! \textbf{(2 Punkte)}
\item Nennen Sie die zwei Eigenschaften des Korrelationskoeffizienten! \textbf{(2 Punkte)}
\item Interpretieren Sie den Korrelationskoefizienten hinsichtlich des
  Effekts und der Signifikanz! Begründen Sie Ihre Antwort! \textbf{(2 Punkte)}
\item Visualisieren Sie das 95\% Konfidenzintervall! Beschriften Sie die Abbildung! \textbf{(2 Punkte)} 
\end{enumerate} 
\clearpage
% -----------------------------------------------------------------------

\section{Aufgabe \hfill (12 Punkte)}

\textit{Geben Sie grundsätzlich Formeln und Rechenweg zur Lösung der Teilaufgaben mit an!} \\[1Ex]
 

 
%% --------------------------------------------------------------------
\begin{minipage}[t]{0.5\textwidth}
\includegraphics[width = 1.3cm]{/Users/kruppajo/work/GitHub/exam/avatare/Steffen.png}
\end{minipage}
\begin{minipage}[t]{0.5\textwidth}
\hfill
\href{https://youtu.be/fB6nF4dxodA}{\includegraphics[width = 2cm]{img/youtube}}
\end{minipage}
\vspace{-3ex}
%% --------------------------------------------------------------------



'Hm..., drei leere Abbildungen. Was soll ich da hetzt machen?', fragt sich Steffen. Steffen kennt sich nur begrenzt bis gar nicht mit der linearen Regresion und Korrelation aus. Dafür mit etwas anderem. Klemmbausteine. Ein wunderbares Hobby um sich drin zu verlieren und Abstand zu bekommen. Steffen denkt gerne über Klemmbausteine nach. Aber das hilft hier auch nur so halb, die Aufgabe zu lösen und mehr über den Korrelationskoeffizienten zu erfahren.
\vspace{2Ex}



{\centering \includegraphics[width=\maxwidth]{img/correlation-01-1} 

}




\paragraph{Visualisierung der Korrelation und des Bestimmtheitsmaßes}

\vspace{2Ex}

Leider kennt sich Steffen mit der Korrelationsanalyse und der linearen Regression überhaupt nicht aus. Deshalb braucht er bei der Auswertung Ihre Hilfe!

\begin{enumerate}
\item Zeichnen Sie für die $\rho$-Werte eine Gerade in die entsprechende Abbildung! \textbf{(3 Punkte)}
\item Zeichnen Sie für die $R^2$-Werte die entsprechende Punktewolke um die Gerade! \textbf{(3 Punkte)}
\item Nennen Sie die zwei Eigenschaften des Korrelationskoeffizienten! \textbf{(2 Punkte)}
\item Interpretieren Sie die $R^2$-Werte für die jeweilige Gerade! \textbf{(2 Punkte)}
\item Warum müssen Sie ein $R^2$-Wert berechnen, wenn Sie die einfachere Möglichkeit der visuellen Überprüfung haben? Begründen Sie Ihre Antwort! \textbf{(2 Punkte)}
\end{enumerate}
 
\clearpage
% -----------------------------------------------------------------------

\section{Aufgabe \hfill (12 Punkte)}

\textit{Geben Sie grundsätzlich Formeln und Rechenweg zur Lösung der Teilaufgaben mit an!} \\[1Ex]
 

 
%% --------------------------------------------------------------------
\begin{minipage}[t]{0.5\textwidth}
\includegraphics[width = 1.3cm]{/Users/kruppajo/work/GitHub/exam/avatare/Mark.png}
\end{minipage}
\begin{minipage}[t]{0.5\textwidth}
\hfill
\href{https://youtu.be/2QJa19ZwLls}{\includegraphics[width = 2cm]{img/youtube}}
\end{minipage}
\vspace{-3ex}
%% --------------------------------------------------------------------



\paragraph{Schätzen der Korrelation und des Bestimmtheitsmaßes}

Der Bildschirm strahlt blau in das Gesicht von Mark. Es ist schon spät. Und das hat einen Grund. Auf seinem Second Screen läuft Columbo und Mark schaufelt Marzipankugeln. Nicht effizient, aber gut. . Mark überlegt, aber seine Gedaken sind etwas zäh. 'Was soll das hier alles bedeuten?', fragt sich Mark. Irgendwie ist ihm nicht klar wie er $\rho$-Werte oder $R^2$-Werte abschätzen soll. Alles nicht so einfach. Mark und die Unsicherheit, eine unendliche Geschichte mit kniffeligen Wendungen. 
\vspace{2Ex}



{\centering \includegraphics[width=\maxwidth]{img/correlation-02-1} 

}




Leider kennt sich Mark mit der Korrelationsanalyse und der linearen Regression überhaupt nicht aus. Deshalb braucht er bei der Auswertung Ihre Hilfe!

\begin{enumerate}
\item Schätzen Sie die $\rho$-Werte in den Abbildungen! \textbf{(2 Punkte)}
\item Schätzen Sie die $R^2$-Werte in den Abbildungen! \textbf{(2 Punkte)}
\item Interpretieren Sie die $R^2$-Werte für die jeweilige Gerade! \textbf{(2 Punkte)}
\item Was ist der optimale $R^2$-Wert im Kontext einer wissenschaftlichen Fragestellung? Begründen Sie Ihre Antwort an einem Beispiel! \textbf{(2 Punkte)}
\item Was ist der optimale $\rho$-Wert im Kontext einer wissenschaftlichen Fragestellung? Begründen Sie Ihre Antwort an einem Beispiel! \textbf{(2 Punkte)}
\item Erklären Sie die Aussage \textit{"Correlation does not imply causation!"} an einem Beispiel! \textbf{(2 Punkte)}
\end{enumerate} 
\clearpage
% -----------------------------------------------------------------------

\section{Aufgabe \hfill (11 Punkte)}

\textit{Geben Sie grundsätzlich Formeln und Rechenweg zur Lösung der Teilaufgaben mit an!} \\[1Ex]
 

 
%% --------------------------------------------------------------------
\begin{minipage}[t]{0.5\textwidth}
\includegraphics[width = 1.3cm]{/Users/kruppajo/work/GitHub/exam/avatare/Jessica.png}
\end{minipage}
\begin{minipage}[t]{0.5\textwidth}
\hfill
\href{https://youtu.be/dyQlYV9nOqY}{\includegraphics[width = 2cm]{img/youtube}}
\end{minipage}
\vspace{-3ex}
%% --------------------------------------------------------------------



\paragraph{Modellgüte der linearen Regression}

'Oh! Residuen. Die waren wichtig um zu wissen, ob eine Modellierung funktioniert hat! Da schauen wir uns dann mit der Funktion \texttt{augment()} die Werte der einzelnen Residuen an. Oder gleich den Residuenplot...da sehen wir dann... ja was eigentlich?', verkündet Jessica stolz. Leider hat Jessica vergessen wie der \Rlogo Code für den Residuenplot geht. Jessica hatte anderes im Kopf. Hm, lecker Schokobons und dazu dann im Hintergrund Herr der Ringe laufen lassen. Aber sowas hilft ihr natürlich hier nicht. Da schmeißt sich Jessica noch ein paar Schokobons in den Mund und kaut los.

\begin{knitrout}
\definecolor{shadecolor}{rgb}{0.969, 0.969, 0.969}\color{fgcolor}\begin{table}[!h]
\centering\begingroup\fontsize{12}{14}\selectfont

\begin{tabular}{cccc}
\toprule
Fettgehalt & Mittlere Anzahl & $\hat{y}$ & $\phantom{ttttt}\epsilon\phantom{ttttt}$\\
\midrule
30.3 & 13.1 & 31.0 & \\
27.9 & 10.0 & 26.4 & \\
24.1 & 10.6 & 27.2 & \\
18.6 & 4.0 & 17.4 & \\
23.6 & 8.4 & 24.0 & \\
\addlinespace
38.4 & 16.8 & 36.4 & \\
28.6 & 12.6 & 30.3 & \\
32.4 & 13.2 & 31.1 & \\
27.1 & 9.5 & 25.7 & \\
23.8 & 8.8 & 24.6 & \\
\addlinespace
32.3 & 14.5 & 33.0 & \\
19.1 & 5.2 & 19.2 & \\
\bottomrule
\end{tabular}
\endgroup{}
\end{table}

\end{knitrout}

Leider kennt sich Jessica mit der linearen Regression überhaupt nicht aus. Deshalb braucht sie bei der Auswertung Ihre Hilfe!

\begin{enumerate}
  \item Formulieren Sie die wissenschaftliche Fragestellung! \textbf{(1 Punkt)}
\item Ergänzen Sie die Werte der Residuen $\epsilon$ in der obigen Tabelle! \textbf{(2 Punkte)}
\item Zeichnen Sie den Boxplot der Residuen $\epsilon$. Beschriften Sie die Abbildung! \textbf{(2 Punkte)}
\item Zeichnen Sie den Residualplot. Beschriften Sie die Abbildung! \textbf{(2 Punkte)}
\item Gibt es auffällige Werte anhand des Residualplots? Begründen Sie Ihre Antwort! \textbf{(2 Punkte)}
\item Erklären Sie die Eigenschaft eines statistischen Modells, welche mit dem Residualplot überprüft wird! Begründen Sie Ihre Antwort anhand einer Visualisierung! \textbf{(2 Punkte)}
\end{enumerate}
 
\clearpage
% -----------------------------------------------------------------------

\section{Aufgabe \hfill (12 Punkte)}

\textit{Geben Sie grundsätzlich Formeln und Rechenweg zur Lösung der Teilaufgaben mit an!} \\[1Ex]
 

 
%% --------------------------------------------------------------------
\begin{minipage}[t]{0.5\textwidth}
\includegraphics[width = 1.3cm]{/Users/kruppajo/work/GitHub/exam/avatare/Tina.png}\hspace{-4mm}\includegraphics[width = 1.3cm]{/Users/kruppajo/work/GitHub/exam/avatare/Yuki.png}
\end{minipage}
\begin{minipage}[t]{0.5\textwidth}
\hfill
\href{https://youtu.be/kHmfEmU6lrk}{\includegraphics[width = 2cm]{img/youtube}}
\end{minipage}
%% --------------------------------------------------------------------



\paragraph{Visualisierung des Regressionskreuzes}

Yuki hat ein Gewächshausexperiment mit Spargel duchgeführt. Soweit so gut. Dann war sie bei ihrem Betreuer. Leider war der Schritt nicht so hilfreich.  Yuki und die Faulheit, eine unendliche Geschichte mit kniffeligen Wendungen. Aber es muss ja weitergehen. Yuki hatte dann in ihrer Abschlusarbeit einfach zu viele Endpunkte gemessen und ist jetzt vollkommen durcheinander, welche Analyse sie nun wie rechnen soll. Naja, dann heißt es jetzt eben London Grammar aufdrehen und darüber nachdenken, was hier eigentlich gemacht wurde. Yuki fängt einfach an und nimmt den ersten Endpunkt Anzahl Läsionen auf den Blättern. Dann kann sie sich voran arbeiten. Später dann noch raus um zu Boldern um mal zu entspannen und vielleicht ist Tina auch da. Wäre toll.

\vspace{1Ex}

Leider kennt sich Yuki mit dem Kontext der linearen Regression überhaupt nicht aus. Deshalb braucht sie bei der Auswertung Ihre Hilfe!

\begin{enumerate}
  \item Formulieren Sie die wissenschaftliche Fragestellung! \textbf{(1 Punkt)}
\item Zeichen Sie die Zeile des Regressionskreuzes für den Endpunkt mit \underline{drei} Feldern! Beschriften Sie die Abbildung! \textbf{(4 Punkte)}
\item Ergänzen Sie die entsprechenden statistische Methoden zur Analyse in jedem Feld! \textbf{(2 Punkte)}
\item Formulieren Sie die Nullhypothese für die statistische Methode in jedem Feld! \textbf{(2 Punkte)}
\item Ergänzen Sie die entsprechenden Funktionen in \Rlogo zur Analyse in jedem Feld! \textbf{(2 Punkte)}
\item Welchen Effekt erhalten Sie in jedem Feld? Geben Sie ein Beispiel! \textbf{(2 Punkte)}
\end{enumerate} 
\clearpage
% -----------------------------------------------------------------------
\part{Experimentelles Design}
% -----------------------------------------------------------------------

\section{Aufgabe \hfill (16 Punkte)}


 
%% --------------------------------------------------------------------
\begin{minipage}[t]{0.5\textwidth}
\includegraphics[width = 1.3cm]{/Users/kruppajo/work/GitHub/exam/avatare/Alex.png}\hspace{-4mm}\includegraphics[width = 1.3cm]{/Users/kruppajo/work/GitHub/exam/avatare/Jonas.png}\hspace{-4mm}\includegraphics[width = 1.3cm]{/Users/kruppajo/work/GitHub/exam/avatare/Mark.png}
\end{minipage}
\begin{minipage}[t]{0.5\textwidth}
\hfill
\href{https://youtu.be/wJqsNV1hOW8}{\includegraphics[width = 2cm]{img/caution}}
\end{minipage}
%% --------------------------------------------------------------------



\paragraph{Einfache experimentelle Designs}

Jonas und Alex sind bei Mark um sich Hilfe für eine Versuchsplanung in \Rlogo zu holen. Im Hintergrund läuft viel zu laut Andrea Berg. Dabei geht es um den Zusammenhang zwischen der Behandlung Substrattypen ($torf$, $40p60n$, $30p20n$ und $70p30n$) und dem Messwert Proteingehalt [g/kg] in Brokkoli. Der Versuch soll in einem Versuch in einer Klimakammer im Teuteburgerwald durchgeführt werden. Nach dem Dozenten ist der Messwert Proteingehalt [g/kg] normalverteilt. Die beiden entschieden sich für ein faktorielles Versuchsdesign. Im ersten Schritt überlegt Mark ein einfaches experimentelles Design zu probieren. Daher entscheiden sich alle drei für ein \textit{Complete randomized design (CRD)}. Das sollte für den Anfang erstmal reichen. 'Und jetzt, was machen wir jetzt?', Alex schaut die anderen beiden mit großen Augen an. Die zucken mit der Schulter. Alle mampfen Marzipankugeln.\\

Leider kennen sich Mark, Jonas und Alex mit dem \textit{Complete randomized design (CRD)} überhaupt nicht aus. Deshalb brauchen die Drei bei der Erstellung Ihre Hilfe!

\begin{enumerate}
  \setcounter{enumi}{0}
  \item Formulieren Sie die wissenschaftliche Fragestellung! \textbf{(1 Punkt)}
  \item Formulieren Sie das statistische Hypothesenpaar! \textbf{(1 Punkt)}
  \item Skizzieren Sie das faktorielle Versuchsdesign! \textbf{(3 Punkte)}
  \item Skizzieren Sie eine Datentabelle für das faktorielle Versuchsdesign in \Rlogo! \textbf{(2 Punkte)}
  \item Erstellen Sie das statistische Modell in der in \Rlogo üblichen Schreibweise für eine ANOVA! Skizzieren Sie die notwendige Funktionen in \Rlogo! \textbf{(3 Punkte)}
  \item Skizzieren Sie die weitere Datenanalyse hinsichtlich eines multiplen Gruppenvergleiches! \textbf{(2 Punkte)}
  \item Skizzieren Sie eine mögliche Abbildung im Kontext der wissenschaftlichen Fragestellung! Beschriften Sie die Abbildung! \textbf{(2 Punkte)}
  \item Ergänzen Sie zu der Abbildung ein mögliches Ergebnis des multiplen Gruppenvergleichs! Begründen Sie Ihre Antwort! \textbf{(2 Punkte)}
\end{enumerate}


 
\clearpage
% -----------------------------------------------------------------------

\section{Aufgabe \hfill (20 Punkte)}


 
%% --------------------------------------------------------------------
\begin{minipage}[t]{0.5\textwidth}
\includegraphics[width = 1.3cm]{/Users/kruppajo/work/GitHub/exam/avatare/Jessica.png}\hspace{-4mm}\includegraphics[width = 1.3cm]{/Users/kruppajo/work/GitHub/exam/avatare/Nilufar.png}\hspace{-4mm}\includegraphics[width = 1.3cm]{/Users/kruppajo/work/GitHub/exam/avatare/Tina.png}
\end{minipage}
\begin{minipage}[t]{0.5\textwidth}
\hfill
\href{https://youtu.be/wJqsNV1hOW8}{\includegraphics[width = 2cm]{img/caution}}
\end{minipage}
%% --------------------------------------------------------------------



\paragraph{Fortgeschrittene experimentelle Designs}

Der Hamster macht mal wieder Randale in Nilufars Zimmer und rennt davon! Jessica und Tina sind bei Nilufar in im Oldenburger Land wo der neue, bessere Versuch stattfinden soll. Dabei soll in einem Stallversuch im Oldenburger Land mit Puten durchgeführt werden. Auf dem Tisch stapeln sich Marzipankugeln aus Vollkorndinkelmehl. Eine Spezialtät der Komune hier. Jessica hasst Vollkorn wie Gummibärchen geliebt werden. In dem neuen Versuch geht es um den Zusammenhang zwischen der Behandlung Bestandsdichte ($standard$, $eng$, $weit$ und $kontakt$) sowie Genotypen ($AA$ und $BB$) sowie zwei Blöcken und dem Messwert Protein/Fettrate [\%/kg]. Immerhin ist der Messswert normalverteilt, was einges einfacher macht. Was es nicht so einfacher macht ist, dass Tina als zusätzliche Herausforderung noch die Gefälligkeit mitgebracht hat. Daher entscheiden sich alle drei für ein \textit{Split plot design oder auch Spaltanlage}. 'Naja, so viel einfacher ist es dann doch nicht...', merkt Tina an und sucht der Hamster.\\

Leider kennen sich Nilufar, Jessica und Tina mit dem \textit{Split plot design oder auch Spaltanlage} überhaupt nicht aus. Deshalb brauchen die Drei bei der Erstellung Ihre Hilfe!

\begin{enumerate}
  \setcounter{enumi}{0}
  \item Formulieren Sie die wissenschaftliche Fragestellung! \textbf{(1 Punkt)}
  \item Formulieren Sie die statistische Hypothesenpaare! \textbf{(2 Punkte)}
  \item Skizzieren Sie das faktorielle Versuchsdesign! \textbf{(3 Punkte)}
  \item Skizzieren Sie eine Datentabelle für das faktorielle Versuchsdesign in \Rlogo! \textbf{(2 Punkte)}
  \item Erstellen Sie das statistische Modell in der in \Rlogo üblichen Schreibweise für eine ANOVA! Skizzieren Sie die notwendige Funktionen in \Rlogo! \textbf{(4 Punkte)}
  \item Skizzieren Sie die weitere Datenanalyse hinsichtlich eines multiplen Gruppenvergleiches! \textbf{(2 Punkte)}
  \item Skizzieren Sie eine mögliche Abbildung im Kontext der wissenschaftlichen Fragestellung! Beschriften Sie die Abbildung! \textbf{(3 Punkte)}
  \item Ergänzen Sie zu der Abbildung ein mögliches Ergebnis des multiplen Gruppenvergleichs! Welche Annahme hinsichtlich der Modellierung haben Sie getroffen? Begründen Sie Ihre Antwort! \textbf{(3 Punkte)}
\end{enumerate} 
\clearpage
% -----------------------------------------------------------------------
\part{Programmieren in R}
% -----------------------------------------------------------------------

\section{Aufgabe \hfill (9 Punkte)}



 
%% --------------------------------------------------------------------
\begin{minipage}[t]{0.5\textwidth}
\includegraphics[width = 1.3cm]{/Users/kruppajo/work/GitHub/exam/avatare/Tina.png}
\end{minipage}
\begin{minipage}[t]{0.5\textwidth}
\hfill
\href{https://www.youtube.com/playlist?list=PLe51bCp9JvEFUnFqaJG5aRmON9i1ZbOYC}{\includegraphics[width = 2cm]{img/youtube}}
\end{minipage}
\vspace{-3ex}
%% --------------------------------------------------------------------



\paragraph{Grundlegende Kenntnisse der Programierung in \Rlogo}

Tina muss ihrer Hausarbeit mit \Rlogo arbeiten. Deshalb sitzt sie jetzt mit Ihnen zusammen und hat einige Fragen zu den Grundlagen in \Rlogo an Sie! Na dann wollen Sie mal helfen. Immerhin will ihre Betreuerin, dass \Rlogo genutzt wird.\\[1Ex]

Tina: \textit{In R hat sich doch mein Datensatz geändert. Warum ist er jetzt wieder so wie vorher, wenn ich den Datensatz wieder aufrufe? Was mache ich falsch?} \textbf{(1 Punkt)}\\[1ex]
Sie antworten:\\[2Ex]

Tina: \textit{Der Pipe-Operator wird sehr häufig genutzt. Wie sieht der aus und wie funktioniert der an einem Beispiel?} \textbf{(1 Punkt)}\\[1ex]
Sie antworten:\\[2Ex]

Tina: \textit{Ich habe den Namen der Funktion, die intern Daten speichert, vergessen. Was waren da nochmal die Vorteile?} \textbf{(1 Punkt)}\\[1ex]
Sie antworten:\\[2Ex]

Tina: \textit{Gibt es einen Vorteil von der Nutzung von \Rlogo?} \textbf{(1 Punkt)}\\[1ex]
Sie antworten:\\[2Ex]

Tina: \textit{Warum brauche ich eigentlich das RStudio und R? Wo ist denn da der Unterschied?} \textbf{(1 Punkt)}\\[1ex]
Sie antworten:\\[2Ex]

Tina: \textit{Jetzt lese ich hier von einem Faktor. Was ist ein Faktor in \Rlogo?} \textbf{(1 Punkt)}\\[1ex]
Sie antworten:\\[2Ex]

Tina: \textit{Ich sehe überall dieses \texttt{c()}. Was ist denn deren Nutzen? } \textbf{(1 Punkt)}\\[1ex]
Sie antworten:\\[2Ex]

Tina: \textit{Wie sieht der Zuweisungs-Operator aus und was ist seine Funktion? Gerne mit Beispiel!} \textbf{(1 Punkt)}\\[1ex]
Sie antworten:\\[2Ex]

Tina: \textit{Wir brauchen recht häufig die Tilde ($\sim$) in R. Wo wird die nochmal angewandt und genutzt?} \textbf{(1 Punkt)}\\[1ex]
Sie antworten:\\[2Ex] 
\clearpage
% -----------------------------------------------------------------------

\section{Aufgabe \hfill (9 Punkte)}



 
%% --------------------------------------------------------------------
\begin{minipage}[t]{0.5\textwidth}
\includegraphics[width = 1.3cm]{/Users/kruppajo/work/GitHub/exam/avatare/Yuki.png}
\end{minipage}
\begin{minipage}[t]{0.5\textwidth}
\hfill
\href{https://www.youtube.com/playlist?list=PLe51bCp9JvEFUnFqaJG5aRmON9i1ZbOYC}{\includegraphics[width = 2cm]{img/youtube}}
\end{minipage}
\vspace{1ex}
%% --------------------------------------------------------------------



\paragraph{Fortgeschrittene Kenntnisse der Programierung in \Rlogo}

Yuki muss seinem Projektbericht mit \Rlogo arbeiten. Leider ist die Analyse etwas komplexer, so dass es eben in Excel dann nicht mehr geht. Deshalb also gleich alles in \Rlogo. Das ist auch der Grund warum er jetzt mit Ihnen in der Küche sitzt und einige vertiefende Fragen zu \Rlogo an Sie hat! Na dann wollen Sie mal helfen. Immerhin will sein Betreuer, dass \Rlogo genutzt wird und die Abgabe ist dann auch schon in gut einem Monat.\\[1Ex]

Yuki fragt: \textit{Welche Funktionen brauche ich nochmal für die Erstellung eines CLD und was war noch gleich die Reihenfolge? \textbf{(2 Punkte)}}\\[1ex]
Sie antworten:\\[2Ex]

Yuki fragt: \textit{Das Dateiformat in R hat einen Namen. Wie heißt der und gerne mit Beispiel! \textbf{(1 Punkt)}}\\[1ex]
Sie antworten:\\[2Ex]

Yuki fragt: \textit{Ich will das R Paket \texttt{\{ggplot\}} nutzen, da war so eine Analogie an die ich mich nicht erinnern kann. Was war noch gleich das Prinzip von \texttt{\{ggplot\}}? Wie funktioniert \texttt{\{ggplot\}} konzeptionell? \textbf{(2 Punkte)}}\\[1ex]
Sie antworten:\\[2Ex]

Yuki fragt: \textit{Wenn ich die Spalten in Excel benenne, was muss ich da beachten? \textbf{(1 Punkt)}}\\[1ex]
Sie antworten:\\[2Ex]

Yuki fragt: \textit{Ich habe den Faktor $f_1$ und $f_2$ und möchte den Faktor $f_1$ getrennt in jedem Level des Faktors $f_2$ auszuwerten. Wie geht das in der Funktion \texttt{emmeans()}? \textbf{(1 Punkt)}}\\[1ex]
Sie antworten:\\[2Ex]

Yuki fragt: \textit{Nach der EDA zu urteilen liegt eine Interakton vor, wie spezifiziere ich diese im Modell, so dass ich die interaktion zwischen zwei Faktoren $f_1$ und $f_2$ testen kann? \textbf{(1 Punkt)}}\\[1ex]
Sie antworten:\\[2Ex]

Yuki fragt: \textit{Ich glaube ich habe Varianzheterogenität zwischen den Gruppen vorliegen. Wie funktioniert die Adjustierung dafür nochmal in  \texttt{emmeans()}? \textbf{(1 Punkt)}}\\[1ex]
Sie antworten:\\[2Ex]



 
\clearpage
% -----------------------------------------------------------------------
\part{Forschendes Lernen}

Das forschende Lernen basiert zum einen auf den folgenden wissenschaftlichen Veröffentlichungen. Für die Prüfung wird die vertiefende Kenntnis der folgenden Veröffentlichungen vorausgesetzt.\\

\textit{In der Prüfung erhalten Sie einen Auszug der wissenschaftlichen Veröffentlichung. Für die Einarbeitung in die Veröffentlichung ist in der Prüfung \underline{ausdrücklich} keine Zeit vorgesehen.}

\begin{itemize}[noitemsep]
\item Sánchez, M., Velásquez, Y., González, M., \& Cuevas, J. (2022). Hoverfly pollination enhances yield and fruit quality in mango under protected cultivation. Scientia Horticulturae, 304, 111320. [\href{https://www.sciencedirect.com/science/article/pii/S0304423822004411}{Link}]
% \item Salinas, I., Hueso, J. J., Força Baroni, D., \& Cuevas, J. (2023). Plant growth, yield, and fruit size improvements in ‘Alicia’papaya multiplied by grafting. Plants, 12(5), 1189. [\href{https://www.mdpi.com/2223-7747/12/5/1189}{Link}]
\item Petersen, F., Demann, J., Restemeyer, D., Olfs, H. W., Westendarp, H., Appenroth, K. J., \& Ulbrich, A. (2022). Influence of light intensity and spectrum on duckweed growth and proteins in a small-scale, re-circulating indoor vertical farm. Plants, 11(8), 1010. [\href{https://www.mdpi.com/2223-7747/11/8/1010}{Link}]
\item Selle, P. H., Cadogan, D. J., Li, X., \& Bryden, W. L. (2010). Implications of sorghum in broiler chicken nutrition. Animal Feed Science and Technology, 156(3-4), 57-74. [\href{https://www.sciencedirect.com/science/article/pii/S0377840110000209}{Link}]
\item Wu, G., Knabe, D. A., \& Kim, S. W. (2004). Arginine nutrition in neonatal pigs. The Journal of Nutrition, 134(10), 2783S-2790S. [\href{https://www.sciencedirect.com/science/article/pii/S0022316623031279}{Link}]
\end{itemize}

Das forschende Lernen basiert zum anderen auf den folgenden wissenschaftlichen Datensätzen und deren vertiefende Analyse werden als bekannt vorausgesetzt. Die Teilaufgaben der Aufgaben stellen nur eine zufällige Auswahl an möglichen Fragen dar. Die Datensätze werden über ILIAS bereitgestellt.\\

\textit{In der Prüfung erhalten Sie \underline{keinen Auszug} aus den wissenschaftlichen Daten. Die Datensätze werden als \underline{bekannt} in der Prüfung vorgesetzt. Sie haben sich vorab Notizen und Anmerkungen gemacht.}

\begin{itemize}[noitemsep]
\item bar
\end{itemize}

\clearpage
% -----------------------------------------------------------------------

\section{Aufgabe \hfill (20 Punkte)}

\textit{Geben Sie grundsätzlich Formeln und Rechenweg zur Lösung der Teilaufgaben mit an!} \\[1Ex]
 

 
%% --------------------------------------------------------------------
\begin{minipage}[t]{0.5\textwidth}
\includegraphics[width = 1.3cm]{/Users/kruppajo/work/GitHub/exam/avatare/Tina.png}
\end{minipage}
\begin{minipage}[t]{0.5\textwidth}
\hfill
\href{https://youtu.be/C9skfFRTHhI}{\includegraphics[width = 2cm]{img/caution}}
\end{minipage}
%% --------------------------------------------------------------------



\paragraph{Zerforschen einer wissenschaftlichen Veröffentlichung}

'Uff', denkt Tina. Das ist jetzt doch etwas umfangreicher. Tina soll die wissenschaftliche Veröffentlichung \textit{Sánchez, M., et al. (2022). Hoverfly pollination enhances yield and fruit quality in mango under protected cultivation} einmal zusammenfassen. Die Arbeit soll als eine Vorlage für ihre eigene Arbeit dienen. Daher möchte ihre Betreuerin, dass sie einmal die Veröffentlichung in einer PowerPoint Präsentation zusammenfasst. 'Das ist jetzt aber doch umfangreicher als gedacht.', mault Tina in sich hinein. Schnell nochmal ein paar Katjes zur Stärkung gegessen. Das wird dann vermutlich heute Abend nichts mehr mit ihrem Hobby Astronomie. Die Spinne schaut mitleidig.\\

Leider kennt sich Tina mit dem Lesen einer wissenschaftlichen Veröffentlichung mit Fokus auf die Statistik überhaupt nicht aus. Deshalb braucht sie bei der Erstellung Ihre Hilfe! Glücklicherweise kennen Sie die wissenschaftliche Veröffentlichung schon im Detail und können sofort helfen.

\begin{enumerate}
  \setcounter{enumi}{0}
  \item Erläutern Sie die wissenschaftliche Fragestellung der wissenschaftlichen Veröffentlichung anhand des OCAR Prinzips nach Schimel (2012)\footnote{Schimel, J. (2012). Writing science: how to write papers that get cited and proposals that get funded. OUP USA.} \textbf{(4 Punkte)}
  \item Nennen Sie die untersuchten Endpunkte in der wissenschaftlichen Veröffentlichung! Wie lautet der primäre Endpunkt? \textbf{(2 Punkte)} 
\item Erstellen Sie das statistische Modell in der in \Rlogo üblichen Schreibweise! \textbf{(2 Punkte)}
  \item Nennen Sie eine Auswahl an bedeutenden statistischen Maßzahlen in der wissenschaftlichen Veröffentlichung! \textbf{(1 Punkt)}
  \item Interpretieren Sie die Hauptaussage der wissenschaftlichen Veröffentlichung hinsichtlich der Signifkanz für den primären Endpunkt! \textbf{(2 Punkte)}
  \item Interpretieren Sie die Hauptaussage der wissenschaftlichen Veröffentlichung hinsichtlich der Effektstärke für den primären Endpunkt! \textbf{(2 Punkte)}
  \item Diskutieren Sie die ökonomische Relevanz der Hauptaussage der wissenschaftlichen Veröffentlichung im Bezug auf Signifikanz und Effektstärke für den primären Endpunkt! \textbf{(1 Punkt)}
  \item Skizzieren Sie für den primären Endpunkt den sich ergebenden Datensatz in \Rlogo für eine ausgewählte Abbildung! \textbf{(2 Punkte)}
\item Skizzieren Sie einen möglichen Versuchsplan für den primären Endpunkt! \textbf{(2 Punkte)}
  \item Schätzen Sie die benötigte Fallzahl für ein zukünftiges Experiment anhand der Ergebnisse in der wisenschaftlichen Veröffentlichung für den primären Endpunkt! \textbf{(2 Punkte)}
\end{enumerate} 
\clearpage
% -----------------------------------------------------------------------

\section{Aufgabe \hfill (20 Punkte)}

\textit{Geben Sie grundsätzlich Formeln und Rechenweg zur Lösung der Teilaufgaben mit an!} \\[1Ex]
 

 
%% --------------------------------------------------------------------
\begin{minipage}[t]{0.5\textwidth}
\includegraphics[width = 1.3cm]{/Users/kruppajo/work/GitHub/exam/avatare/Tina.png}
\end{minipage}
\begin{minipage}[t]{0.5\textwidth}
\hfill
\href{https://youtu.be/C9skfFRTHhI}{\includegraphics[width = 2cm]{img/caution}}
\end{minipage}
%% --------------------------------------------------------------------



\paragraph{Zerforschen eines wissenschaftlichen Datensatzes}

Unter einem langen Schnaufen starrt Tina auf den wissenschaftlichen Datensatz \textit{data3} in ihrem Laptop. 'Worum geht es denn eigentlich in diesem Datensatz?', fragt sie sich kopfschüttelnd und mampft noch ein paar Katjes. Tina soll die Datentabelle nutzen um das eigene Experiment zu planen und eine Blaupause zu haben. Als eine Vorlage sozusagen, die sie nur noch ausfüllen muss. Daher möchte ihre Betreuerin, dass sie einmal die Daten sinnvoll zusammenfasst. Das sollte dann doch etwas aufwendiger werden. Das wird dann vermutlich heute Abend nichts mehr mit Indiana Jones\\

Leider kennt sich Tina mit der Analyse eines wissenschaftlichen Datensatzes überhaupt nicht aus. Deshalb braucht sie bei der Auswertung Ihre Hilfe! Glücklicherweise kennen Sie den wissenschaftlichen Datensatz aus Ihren eigenen Analysen schon im Detail und können sofort helfen.

\begin{enumerate}
  \setcounter{enumi}{0}
  \item Formulieren Sie die wissenschaftliche Fragestellung des Datensatzes in Form einer PowerPoint Folie! \textbf{(2 Punkte)}
  \item Nennen Sie zwei Besonderheiten des Datensatzes! Begründen Sie Ihre Antwort! \textbf{(2 Punkte)}
  \item Nennen Sie die untersuchten Endpunkte in dem Datensatz! Wie lautet der primäre Endpunkt für die Auswertung? \textbf{(2 Punkte)}
  \item Skizzieren Sie die großen Analysebereiche der Statistik! Beschriften Sie die Abbildungen! \textbf{(2 Punkte)}
  \item In welchen der großen Analysebereiche der Statistik fällt die Auswertung des primären Endpunktes? Begründen Sie Ihre Antwort! \textbf{(2 Punkte)}
  \item Skizzieren Sie eine ikonische Abbildung für den primären Endpunkt im Kontext der wissenschaftlichen Fragestellung! \textbf{(2 Punkte)}
  \item Erstellen Sie das statistische Modell in der in \Rlogo üblichen Schreibweise! \textbf{(2 Punkte)}
  \item Skizzieren Sie die Datenanalyse hinsichtlich der Signifkanz für den primären Endpunkt! \textbf{(2 Punkte)}
  \item Skizzieren Sie die Berechnung der Effektstärke für den primären Endpunkt! \textbf{(2 Punkte)}
  \item Skizzieren Sie einen möglichen Versuchsplan für den primären Endpunkt! \textbf{(2 Punkte)}
\end{enumerate} 
\clearpage
% -----------------------------------------------------------------------
\part{Mathematik}
% -----------------------------------------------------------------------  

\section{Aufgabe \hfill (10 Punkte)}

\textit{Geben Sie grunds{\"a}tzlich Formeln und Rechenweg zur L{\"o}sung der
  Teilaufgaben mit an!} \\[1Ex]

%% --------------------------------------------------------------------
\hfill\href{https://youtu.be/Fu8kN0Uj13Y}{\includegraphics[width =
  2cm]{img/youtube}} %%youtube
\hspace{2Ex}
%% --------------------------------------------------------------------

\paragraph{Herodot – der Schimmel aus Ivenack}

W{\"a}hrend der Besetzung Mecklenburgs durch die Franzosen kamen Napoleon die
Geschichten des ber{\"u}hmten Apfelschimmels Herodot aus Ivenack zu
Geh{\"o}r. Herodot lief zwar niemals Rennen, war aber eines der ber{\"u}hmtesten
Pferde dieser Zeit. Napoleon selbst gab den Auftrag, diesen
Schimmel durch die Armee nach Frankreich zu bringen. Der Legende nach
sollen Arbeiter den Schimmel im hohlen Stamm einer 1000-j{\"a}hrigen Eiche aus Ivenack vor
den Franzosen versteckt haben. Doch Herodot verriet sein Versteck durch
lautes Wiehern, woraufhin die franz{\"o}sische Armee den Schimmel
beschlagnahmte und nach Frankreich f{\"u}hrte\footnote{Die Quelle der Inspiration
  für die Aufgabe war eine Fahrt an die Ostsee und folgender Artikel:
  \href{https://www.wald-mv.de/landingpage/ivenacker-eichen/}{Entdecke das erste Nationale Naturmonument Deutschlands - Ivenacker Eichen und Hutewald}}. \\



\textit{Forschungsfrage: "Konnten die Ivenacker den Apfelschimmel Herodot
  vor dem Zugriff von Napoleon in der 1000-j{\"a}hrigen Eiche verstecken?"} \\

Gehen Sie von einem radialen Wachstum der 1000-j{\"a}hrigen Eiche von
$0.8mm$ pro Jahr aus. Es ist bekannt, dass die Eiche im
Jahr 2022 einen Umfang von $13m$ in Brusth{\"o}he hatte.

\begin{enumerate}
\item Wie gro{\ss} war der Durchmesser in $m$ der Eiche im Jahr $1815$ als
  Herodot in der Eiche versteckt werden sollte?
  \textbf{(2 Punkte)}
\item Skizzieren Sie in einer Abbildung einen linearen Zusammenhang und einen
exponentiellen Zusammenhang f{\"u}r das Wachstum der 1000-j{\"a}hrigen Eiche. Erkl{\"a}ren Sie die
Auswirkungen der Entscheidung f{\"u}r linear oder exponentiell auf Ihre
Berechnungen! \textbf{(2 Punkte)}
\end{enumerate}
 
Herodot hatte eine Schulterh{\"o}he von $195$cm, eine Breite von
$85$cm sowie eine L{\"a}nge von  $230$cm.

\begin{enumerate}
  \setcounter{enumi}{2}
\item Berechnen Sie das effektive Volumen von Herodot in $m^3$, welches
  Herodot in der 1000-j{\"a}hrigen Eiche einnehmen w{\"u}rde! \textbf{(2 Punkte)}
\end{enumerate}

Es wurde berichtet, dass sich Herodot in der 1000-j{\"a}hrigen Eiche
$bequem$ um die eigene Achse drehen konnte.

\begin{enumerate}
  \setcounter{enumi}{3}
\item Berechnen Sie die Dicke der Eichenwand in $cm$! Verdeutlichen Sie Ihre
  Berechnungen an einer aussagekr{\"a}ftigen Skizze f{\"u}r Pferd und Eiche! \textbf{(2 Punkte)} 
\item Unter einer Dicke der Eichenwand von $25cm$ bricht
  die Eiche zusammen. Beantworten Sie die Forschungsfrage! Begr{\"u}nden Sie
  Ihre Antwort! \textbf{(2 Punkte)} 
\end{enumerate}
 
\clearpage
% ----------------------------------------------------------------------- 

\section{Aufgabe \hfill (10 Punkte)}

\textit{Geben Sie grunds{\"a}tzlich Formeln und Rechenweg zur L{\"o}sung der
  Teilaufgaben mit an!} \\[1Ex]

%% --------------------------------------------------------------------
\hfill\href{https://youtu.be/57B-yYoFSk0}{\includegraphics[width =
  2cm]{img/youtube}} %%youtube
\hspace{2Ex}
%% --------------------------------------------------------------------

\paragraph{Von T{\"o}pfen auf Tischen}



In einem Experiment wollen Sie die Wuchsh{\"o}he von 180
Stockrosen bestimmen. Bevor Sie {\"u}berhaupt mit dem Experiment beginnen
k{\"o}nnen, gibt es aber ein paar Absch{\"a}tzungen {\"u}ber die Kosten und den Aufwand
zu treffen. Zum einen m{\"u}ssen Sie die Stockrosen einpflanzen und m{\"u}ssen
daf{\"u}r Substrat bestellen. Zum anderen m{\"u}ssen Sie die Stockrosen auch
bewegen und in ein Gew{\"a}chshaus platzieren. Die T{\"o}pfe f{\"u}r die Keimung haben
einen Durchmesser von 10cm und eine H{\"o}he von 9cm. Der
Kubikmeterpreis f{\"u}r Torf liegt bei 290 EUR.

\begin{enumerate}
\item Skizzieren Sie den Versuchsplan auf \textit{vier} Tischen im
  Gew{\"a}chshaus! \textbf{(2 Punkte)}
\item Berechnen Sie die ben{\"o}tigte Anzahl an Pflanzt{\"o}pfen, wenn Sie
  Randpflanzen mit ber{\"u}cksichtigen wollen! \textbf{(1 Punkt)}
\item Welche $Tisch$fl{\"a}che in $m^2$ gegeben der Anzahl an
  Pflanzt{\"o}pfen inklusive Randpflanzen ben{\"o}tigen Sie im Gew{\"a}chshaus am
  Anfang der Keimungsphase?  \textbf{(3 Punkte)}
\item Berechnen Sie die ben{\"o}tigte Menge an Torf in Liter $l$, die Sie f{\"u}r
  das Bef{\"u}llen der Pflanzt{\"o}pfe ben{\"o}tigen! Gehen Sie von \textit{einem
    Zylinder} f{\"u}r die Pflanzt{\"o}pfe aus!  \textbf{(3 Punkte)}
\item Berechnen Sie die Kosten in EUR f{\"u}r Ihre Torfbestellung! \textbf{(1
    Punkt)}
\end{enumerate}



 
\clearpage
% ----------------------------------------------------------------------- 

\section{Aufgabe \hfill (10 Punkte)}

\textit{Geben Sie grunds{\"a}tzlich Formeln und Rechenweg zur L{\"o}sung der
  Teilaufgaben mit an!} \\[1Ex]

%% --------------------------------------------------------------------
\hfill\href{https://youtu.be/aBxLkdF-c4M}{\includegraphics[width =
  2cm]{img/youtube}} %%youtube
\hspace{2Ex}
%% --------------------------------------------------------------------

\paragraph{Solar- \& Biogasanlagen}



Um die Energiekosten Ihres Betriebes zu senken, wollen Sie eine Solaranlage
auf den Rinderstall montieren lassen. Sie messen Ihren Stall und finden
folgende Ma{\ss}e wieder. Die vordere Seite des Rinderstall hat eine H{\"o}he
$h_v$ von $5m$. Die hintere Seite des Rinderstall hat eine
H{\"o}he $h_b$ von $10m$. Der Rinderstall hat eine Tiefe $t$ von
$12m$ und eine Breite $b$ von $40m$.

\begin{enumerate}
\item Skizzieren Sie den Rinderstall auf dem die Solaranlage montiert
  werden soll! Erg{\"a}nzen Sie die Angaben f{\"u}r die H{\"o}hen $h_v$, $h_b$, die
  Tiefe $t$ und die Breite $b$ des Stalls!  \textbf{(2 Punkte)}
\item Berechnen Sie die Fl{\"a}che der schr{\"a}gen, neuen Solaranlage auf dem
  Rinderstall! \textbf{(3 Punkte)}
\end{enumerate}

Ebenfalls planen Sie eine neue Biogasanlage f{\"u}r Ihren Betrieb. Der neue
Methantank hat einen Radius $r$ von $1.2m$. Leider gibt es ein
paar bauliche Beschr{\"a}nkungen auf dem Grundst{\"u}ck. Ihr Fundament des
zylindrischen Methantanks kann nur ein Gewicht von maximal
$12t$ aushalten bevor der Tank wegbricht. Sie rechnen eine
Sicherheitstoleranz von $20\%$ ein beinhaltend das Gewicht des
Methantanks. In fl{\"u}ssiger Form hat
Methan bei $-80^\circ\text{C}$ eine Dichte von
$220kg/m^3$. Bei $-100^\circ\text{C}$ hat Methan eine Dichte
von $280kg/m^3$. Sie betrieben Ihre Anlage bei
$-95^\circ\text{C}$.

\begin{enumerate}
  \setcounter{enumi}{2}
\item Extrapolieren Sie die effektive Dichte des Methans in Ihrem
  Methantank! Welche Annahme haben Sie getroffen? \textbf{(1 Punkt)}
\item Berechnen Sie wie viel Kubikmeter $m^3$ Sie in den Methantank f{\"u}llen
  k{\"o}nnen, bevor das Fundament nachgibt! \textbf{(2 Punkte)}
\item Berechnen Sie die maximale H{\"o}he $h_{max}$ in $m$ f{\"u}r den gef{\"u}llten
  Methantank mit dem Radius $r$, bevor das Fundament wegbricht! \textbf{(2
    Punkte)}
\end{enumerate}

 
\clearpage
% -----------------------------------------------------------------------

\section{Aufgabe \hfill (10 Punkte)}

\textit{Geben Sie grunds{\"a}tzlich Formeln und Rechenweg zur L{\"o}sung der
  Teilaufgaben mit an!} \\[1Ex]

%% --------------------------------------------------------------------
\hfill\href{https://youtu.be/https://youtu.be/k2G52hMIfqk}{\includegraphics[width = 2cm]{img/youtube}} %%youtube
\hspace{2Ex}
%% --------------------------------------------------------------------

%% --------------------------------------------------------------------
{\tiny\textbf{Stichworte:} Riesenfaultier $\bullet$ Evolution der Avocado $\bullet$ Bluetooth $ \bullet$ Blauzahn $\bullet$ Colonia Dignidad $\bullet$ ODESSA $\bullet$ Rattenlinie $\bullet$ Adolf Eichmann}
%% --------------------------------------------------------------------


\paragraph{Aligatorenbirnen und Blaubeeren}



"'Sind Sie ein Riesenfautier oder warum kaufen Sie so viele
Aligatorenbirnen?"', spricht es hinter Ihnen. Irritiert drehen Sie sich um
und blicken in das puderrote Gesicht von einer Rentnerin mit Monokel. "'Wieso?"', entfährt es
Ihnen und Sie bereuen sogleich die Frage. Sofort werden Sie zu einem
Whiteboard voller roter Schnüre geschliffen und müssen folgenden
mathematischen untermauerten Argumenten im Edeka über sich ergehen
lassen. Da kommen Sie nicht mehr raus, also können Sie auch gleich
mitmachen. Das Problem liegt in Chile\footnote{Die Quelle der Inspiration
  für die Aufgabe waren folgende Reportagen:
  \href{https://www.amnesty.ch/de/ueber-amnesty/publikationen/magazin-amnesty/2021-3/bis-zum-letzten-tropfen}{"`Bis
  zum letzten Tropfen"' in AMNESTY – Magazin der Menschenrechte vom August
  2021} und
\href{https://www.welthungerhilfe.de/welternaehrung/rubriken/klima-ressourcen/wassernot-in-chile-eine-folge-der-privatisierung}{"`Wasserknappheit
  in Chile: Eine Folge der Privatisierung?"' in Die Welternährung dem
  Fachjournal der Welthungerhilfe vom April 2022.}}. Tja, die Deutschen und Südamerika.\\

Zuerst werden Ihre Fähigkeiten getestet, der Mathematik folgen zu können. Oder berechnen Sie gerade den Einkauf von einer Rentnerin mit Monokel?\\

\begin{enumerate}
\item Wenn 5 Blaubeerschalen 8.95 Euro kosten,  wie viel kosten 6 Schalen? \textbf{(2 Punkte)}
\item Wenn Sie die 6 Blaubeerschalen gekauft haben, wie viele Aligatorbirnen zu je 2.89 EUR können Sie sich dann noch für 200 EUR leisten? \textbf{(1 Punkt)}
\end{enumerate}

Das Whiteboard beinhaltet folgende Liste mit Informationen zum Wasserverbrauch bei der Produktion von Gemüse aus Chile. Seltsam, was man so alles in einem Edeka über Gemüse erfährt.
  
\begin{itemize}[noitemsep]
\item Ein Kilo Strauchtomaten benötigt 170l Wasser. Eine Strauchtomate wiegt 110 - 115g.
\item Ein Kilo Salat benötigt 130l Wasser. Ein Salatkopf wiegt 300 - 530g.
\item Ein Kilo Avocado benötigt 980l Wasser. Eine Avocado wiegt 120 - 400g.
\item Ein Kilo Blaubeeren benötigt 820l Wasser. Eine Blaubeere wiegt 3.2 - 3.6g.
\end{itemize}

\begin{enumerate}
  \setcounter{enumi}{2}
\item Berechnen Sie den Wasserverbrauch für die Produktion für jeweils eine Strauchtomate, einem Salat, einer Avocado und einer Blaubeeren. Stellen Sie das Ergebnis als Tabelle dar! \textbf{(3 Punkte)}
\end{enumerate}

Chile exportiert im großem Ausmaß Blaubeeren und Avocados. In dem Exportjahr 2022 blieben die Erträge von Blaubeeren mit \ensuremath{8\times 10^{4}}t in dem prognostizierten Rahmen. Die Menge steigerte sich um 5.1\%. Die Exporte für Avocados fielen in dem gleichen Zeitraum um 23.8\% auf \ensuremath{2.3\times 10^{5}}t.

\begin{enumerate}
  \setcounter{enumi}{3}
\item Wie viele Liter Wasser hat Chile in dem Exportjahr 2021 exportiert? \textbf{(2 Punkte)}
\end{enumerate}

Chile ist eines der wenigen Länder der Welt, die ihr Wasser komplett privatisiert haben. Derzeit sind nur drei Prozent des Wassers des Landes für den häuslichen Verbrauch vorgesehen. In den Dörfern der Anbauregionen versorgen Tankwagen die Bevölkerung jede Woche mit Wasser, es gibt etwa 61 Liter Wasser pro Kopf für den täglichen Bedarf. In \textit{Deutschland} liegt der Verbrauch bei 9 - 14 Liter pro Sp{"u}lgang und 8 - 17 Liter pro Sp{"u}lmaschinenlauf.

\begin{enumerate}
  \setcounter{enumi}{4}
\item Mit der rationierten Wassermenge aus Chiles Anbaugebieten können Sie in \textit{Deutschland} wie oft Ihren Bedarf stillen? \textbf{(1 Punkt)}
\end{enumerate}

Das alles hätten Sie nicht von einer Rentnerin mit Monokel erwartet. Ganz schön viele Informationen wurden da zusammengetragen.

\begin{enumerate}
  \setcounter{enumi}{5}  
  \item Nennen Sie eine \textit{Daten}quelle im Internet, wo Sie mehr Informationen zu landwirtschaftlichen Daten oder klimatischen, wirtschaftlichen und gesellschaftlichen Daten erhalten! \textbf{(1 Punkt)}
\end{enumerate} 
\clearpage
% ----------------------------------------------------------------------- 

\section{Aufgabe \hfill (12 Punkte)}

\textit{Geben Sie grunds{\"a}tzlich Formeln und Rechenweg zur L{\"o}sung der
  Teilaufgaben mit an!} \\[1Ex]

%% --------------------------------------------------------------------
\hfill\href{https://youtu.be/WZSxntiNF8s}{\includegraphics[width = 2cm]{img/youtube}} %%youtube
\hspace{2Ex}
%% --------------------------------------------------------------------

%% --------------------------------------------------------------------
{\tiny\textbf{Stichworte:} Kardaschow-Skala $\bullet$ Dyson-Sphäre $\bullet$ Hohlerde $\bullet$ Entropie $\bullet$ Proton $r_P = 1.7 \times 10e-15$ $\bullet$ Wasserstoff $r_H = 5.3\times 10e-11$}
%% --------------------------------------------------------------------

\paragraph{Die Dampfnudelerde}



"'Was für einen Unsinn!"', rufen Sie. Jetzt haben Sie kostbaren Schlaf prokrastiniert um einem Ernährungswissenschaftler auf YouTube über die Erde als Dampfnudel zu lauschen. Irgendwie passt es dann doch mit der Analogie. Übermüdet müssen Sie darüber nachdenken, warum vor 67 Millionen Jahren die Dinosaurier - so groß sie auch waren - nicht von der Schwerkraft zu Boden gerissen wurden. Hat der Dampfplauderer etwa recht und war die Schwerkraft vor Millionen von Jahren eine andere?  Sind deshalb alle Lebewesen auf der Erde \textit{heutzutage} so viel kleiner, weil die Schwerkraft größer ist als damals? War die Erde kleiner und hatte weniger Masse? Oder ist es nur ein Rechenfehler wie bei der Theorie der Hohlerde von Edmond Halley aus dem 17.–18. Jahrhundert? Müde reiben Sie sich die Augen. So wird es nichts mehr mit dem Schlafen, dann können Sie auch mal etwas rechnen\footnote{Die Quelle der Inspiration
  für die Aufgabe war folgender Artikel:
  \href{https://hpd.de/artikel/erde-dampfnudel-22236}{"Skeptische Anmerkungen --- Die Erde als Dampfnudel" in Der Humanistische Pressedienst}}.  \\

Betrachten wir die Schwerkraft oder Gewichtskraft, die auf Lebewesen damals und heute gewirkt haben soll. Nehmen Sie für die Fallbeschleunigung $g$ der Erde \textit{heutzutage} einen Wert von 9.65m/s$^2$ an. Im Weiteren hat die Erde einen ungefähren Durchmesser von \ensuremath{1.2742\times 10^{4}}km und eine mittlere Dichte $\rho$ von 5.21g/cm$^3$. Das Gewicht von einem heute lebenden Waldelefanten mit 2.7t liegt bei 6t und das Gewicht von einem Brachiosaurus bei bis zu 30t.

\begin{enumerate}
\item Welchen Durchmesser müsste die Erde vor 67 Millionen Jahren gehabt haben, wenn Dinosaurier und Elefanten die gleiche Gewichtskraft $\overrightarrow{F_G}$ damals und heute erfahren hätten? \textit{Beantworten Sie die Frage anhand der folgenden Teilaufgaben!}
\begin{enumerate}
\item Berechnen Sie die Fallbeschleunigung von vor 67 Millionen Jahren unter der obigen Annahme gleich wirkender Gewichtskraft $\overrightarrow{F_G}$ auf Elefant und Dinosaurier! \textbf{(1 Punkt)}
\item Berechnen Sie Masse der heutigen Erde! \textbf{(2 Punkte)}
\item Schließen Sie über die Masse auf den Durchmesser der Erde vor 67 Millionen Jahren! \textbf{(2 Punkte)}
\end{enumerate}
\item Beantworten Sie die Eingangsfrage mit 1-2 Antwortsätzen! \textbf{(1 Punkt)}
\end{enumerate}

Die Distanz zwischen Sonne und Erde entspricht 0.99 astronomische Einheiten ($AE$). Die Einheit 1 AE wird mit \ensuremath{1.48\times 10^{8}}km angegeben. Der \textit{massebehaftete} Sonnenwind besteht aus 81\% Wasserstoffkernen mit einer molaren Masse von 1.05g/mol, 12\% Heliumkernen mit 4.32g/mol sowie 7\% weiteren Atomkernen mit  145.31g/mol. Die Teilchendichte bei Eintritt in die Erdatmosphäre liegt zwischen 0.4 bis 100 Teilchen cm$^{-3}$ pro Sekunde mit einer mittleren Teilchendichte von 7cm$^{-3}$ pro Sekunde. \\

\textit{Lösen Sie den folgenden Aufgabenteil mit einer aussagekräftigen Skizze!}

\begin{enumerate}
  \setcounter{enumi}{3}
\item Berechnen Sie die Anzahl an massebehafteten Teilchen des Sonnenwindes, die die gesamte Erde pro Sekunde treffen! \textbf{(2 Punkte)}
\item Berechnen Sie die Anzahl an massebehafteten Teilchen des Sonnenwindes, die die Sonne pro Sekunde in alle Richtungen aussendet! \textbf{(2 Punkte)}
\item Berechnen Sie die Masse, die die Erde pro Jahr durch die \textit{massebehafteten} Teilchen des Sonnenwind zunimmt! \textbf{(2 Punkte)}
\end{enumerate}

%\blfootnote{\tiny\textbf{Stichworte:} Kardaschow-Skala $\bullet$ Dyson-Sphäre $\bullet$ Hohlerde $\bullet$ Entropie}

% https://de.wikipedia.org/wiki/Kardaschow-Skala 
\clearpage
% ----------------------------------------------------------------------- 

\section{Aufgabe \hfill (10 Punkte)}

\textit{Geben Sie grunds{\"a}tzlich Formeln und Rechenweg zur L{\"o}sung der
  Teilaufgaben mit an!} \\[1Ex]

%% --------------------------------------------------------------------
\hfill\href{https://youtu.be/n451XnhtSh4}{\includegraphics[width = 2cm]{img/youtube}} %%youtube
\hspace{2Ex}
%% --------------------------------------------------------------------

\paragraph{Entschuldigung, ist das Ihre Feder in meinem Auge?}



So h{\"o}rt man h{\"a}ufiger h{\"o}fliche G{"a}nse in Mastst{\"a}llen sagen. Das
ist nat{\"u}rlich etwas ung{\"u}nstig, den dann kommt es zu Picken und
Kannibalismus. Denn wenn der Nachbar nervt, dann muss zu Ma{\ss}nahmen
gegriffen werden. Kennt jeder aus einer mittelm{\"a}{\ss}igen Wohngemeinschaft. Das
wollen wir aber als vorsorgliche G{"a}nse-Halter:innen
nicht\footnote{Die Quelle der Inspiration f{\"u}r die Aufgabe war der folgende
  wissenschaftliche Artikel:
  \href{https://www.efsa.europa.eu/en/efsajournal/pub/7788}{EFSA Panel on
    Animal Health and Welfare, et al. (2023) Welfare of broilers on
    farm. EFSA Journal 21.2}.}. Betrachten wir also einmal das Platzangebot
(eng. \textit{space allowance}, abk. \textit{SA}) der G{"a}nse
f{\"u}r vier T{\"a}tigkeiten und versuchen die notwendige Fl{\"a}che zu optimieren. Wie
immer gibt es daf{\"u}r eine mathematische Formel:


\begin{center}
  \begin{tabular}{cc}
    $SA = \sum^n_{i = 1} (A_i \times PB_i)$ & $A_i = \pi \times (r_i + R_i)^2$\\
  \end{tabular}
\end{center}

\vspace{-2Ex}

mit

\begin{itemize}[noitemsep]
\item $SA$ dem ben{\"o}tigten Platzangebot aller aufsummierten Verhalten $i$.
\item $A_i$ dem ben{\"o}tigten Platz f{\"u}r ein Verhalten $i$. 
\item $PB_i$ dem Anteil des Auftretens eines Verhaltens $i$.
\item $r_i$ dem Radius Gans plus dem ben{\"o}tigten Radius f{\"u}r das Verhalten $i$.
\item $R_i$ dem notwendigen Abstand zu den Nachbarn f{\"u}r das Verhalten $i$.    
\item $i$ dem Verhalten: (1) wing/leg stretching, (2) sitting, (3)
  walking und (4) standing.
\end{itemize}

In der folgenden Tabelle 1 sind die Werte f{\"u}r $r_i$, $R_i$ und $PB_i$ f{\"u}r
ein spezifisches Verhalten $i$ aus drei wissenschaftlichen
Ver{\"o}ffentlichungen dargestellt.

\vspace{-1Ex}

{\small
\begin{knitrout}
\definecolor{shadecolor}{rgb}{0.969, 0.969, 0.969}\color{fgcolor}\begin{table}[!h]
\centering
\begin{tabular}{llll}
\toprule
  & Aldridge et al. (2021) & Baxter et al. (2022) & Jabcobs et al. (2019)\\
\midrule
wing/leg stretching & 34cm; 25cm; 12.1\% & 31cm; 22cm; 8.9\% & 38cm; 21cm; 10.1\%\\
sitting & 51cm; 19cm; 25.4\% & 32cm; 21cm; 54.1\% & 32cm; 29cm; 24.1\%\\
walking & 28cm; 25cm; 4.1\% & 34cm; 38cm; 2.1\% & 41cm; 24cm; 4.1\%\\
standing & 35cm; 22cm; 3.2\% & 36cm; 27cm; 3.2\% & 34cm; 11cm; 6.9\%\\
\bottomrule
\end{tabular}
\end{table}

\end{knitrout}
}

\begin{enumerate}
\item Erstellen Sie eine zusammenfassende Tabelle mit den mittleren Werten
  f{\"u}r $r$, $R$ und $PB$ aus der obigen Tabelle 1 f{\"u}r die jeweiligen
  Verhalten! \textbf{(3 Punkte)}
\item Erg{\"a}nzen Sie eine Spalte mit dem ben{\"o}tigten Platz $A$ f{\"u}r das jeweilige
  Verhalten, welches sich aus den mittleren Werten ergibt! \textbf{(1 Punkt)}
\item Berechnen Sie das ben{\"o}tigte Platzangebot $SA$ f{\"u}r alle betrachteten
  Verhalten! \textbf{(1 Punkt)}
\item Skizzieren Sie die Werte $r_i$, $R_i$ und $A_i$ f{\"u}r zwei
  nebeneinander agierende G{"a}nse f{\"u}r ein Verhalten $i$. Nutzen
  Sie hierf{\"u}r vereinfachte geometrische Formen! \textbf{(2 Punkte)}
\item Sie entnehmen der Literatur folgende Aussage zur Verteilung der
  G{"a}nse in der Fl{\"a}che $A$: \textit{"`Assuming, that the
    animals will optimally and equally distribute in an area $A$, we observe a
    small part, which is not covered. This area is called
    $\omega$ and is calculated with $\omega = \tfrac{A}{0.9069}$."'
  } Veranschaulichen Sie die Fl{\"a}che $\omega$ in einer aussagekr{\"a}ftigen
  Abbildung!  \textbf{(1 Punkt)}
\item Ein Tier braucht Platz f{\"u}r sich selbst. Berechnen Sie nun die
  K{\"o}rperfl{\"a}che $a$, die ein Tier einnimmt. Welche Annahmen haben Sie f{\"u}r
  die Berechnung der K{\"o}rperfl{\"a}che getroffen? \textbf{(2 Punkte)}
\end{enumerate}



 
\clearpage
% ----------------------------------------------------------------------- 

\section{Aufgabe \hfill (8 Punkte)}

\textit{Geben Sie grunds{\"a}tzlich Formeln und Rechenweg zur L{\"o}sung der
  Teilaufgaben mit an!} \\[1Ex]

%% --------------------------------------------------------------------
\hfill\href{https://youtu.be/1B53cVFIU7Q}{\includegraphics[width =
  2cm]{img/youtube}} %%youtube
\hspace{2Ex}
%% --------------------------------------------------------------------

\paragraph{Nelken von den Molukken}



In der Ausstellung "`Europa und das Meer"' im Deutschen Historischen Museum in
Berlin gab es folgendes Zitat {\"u}ber die Probleme der fr{\"u}hen Hochseeschifffahrt.

\begin{quote}
  >>Ohne ausreichende Zufuhr von Vitamin C stellen sich nach 40 Tagen die
  ersten Symptome ein; die ersten Toten sind nach 60 Tagen zu beklagen;
  nach 100 Tagen rafft die Skorbut eine ganze Schiffsbesatzung dahin<<
\end{quote}

Ferdinand Magellan stach im Jahre 1519 in See um eine Passage durch den
s{\"u}damerikanischen Kontinent zu finden. Zu seiner Flotte geh{\"o}rten
f{\"u}nf Schiffe - das Flaggschiff Trinidad, die San Antonio, die Victoria, die
Concepci{\'o}n und die Santiago - mit einer Besatzung von insgesamt
245 Mann. 

\begin{enumerate}
\item Stellen Sie den Verlauf der Anzahl an Matrosen auf einem Schiff der
  Flotte in der Form einer {\"U}berlebenszeitkurve dar! Beschriften Sie die
  Achsen entsprechend! \textbf{(2 Punkte)} 
\item Was ist die Besonderheit der Überlebenszeitkurve? Begründen Sie Ihre Antwort! \textbf{(2 Punkte)} 
\item Sch{\"a}tzen Sie die {\"U}berlebenswahrscheinlichkeit nach 90 Tagen
  aus Ihrer Abbildung ab! \textbf{(1 Punkt)} 
\end{enumerate}

Der Chronist an Bord der Trinidad, Antonio Pigafetta, schrieb in seinem
Bericht "`[...] Um nicht Hungers zu sterben, a{\ss}en wir das Leder, mit dem
die gro{\ss}e Rahe zum Schutz der Taue umwunden war."' Insbesondere die
Mannschaft der Concepci{\'o}n erlitt gro{\ss}e Verluste durch die Skrobut bei der
{\"U}berquerung des Pazifiks, da durch Erkundungsfahrten weniger Zeit blieb, um
wilden Sellerie aufzunehmen. Wilder Sellerie enth{\"a}lt
$7000\mu g/100mg$ Vitamin C. Der Bedarf liegt bei
$110mg$ pro Tag f{\"u}r M{\"a}nner.

\begin{enumerate}
  \setcounter{enumi}{2}
\item Berechnen Sie die notwendige Menge in $t$ an
  aufzunehmenden wilden Sellerie auf die Concepci{\'o}n f{\"u}r die ununterbrochene
  Fahrt von drei Monate und 22 Tage {\"u}ber den Pazifik!
  \textbf{(3 Punkte)}
\item Skizzieren Sie die {\"U}berlebenszeitkurve f{\"u}r die Concepci{\'o}n im
  Vergleich zu der {\"U}berlebenszeitkurve der Trinidad! Beschriften Sie die
  Achsen! \textbf{(2 Punkte)}
\end{enumerate}

 
\clearpage
% ----------------------------------------------------------------------- 

\section{Aufgabe \hfill (10 Punkte)}

\textit{Geben Sie grunds{\"a}tzlich Formeln und Rechenweg zur L{\"o}sung der
  Teilaufgaben mit an!} \\[1Ex]

%% --------------------------------------------------------------------
\hfill\href{https://youtu.be/q-qYK4Chslg}{\includegraphics[width =
  2cm]{img/youtube}} %%youtube
\hspace{2Ex}
%% --------------------------------------------------------------------

\paragraph{Event Horizon -- Am Rande des Universums}



Die Sonne hat eine aktuelle, angenommene Masse von $\ensuremath{2\times 10^{28}}$kg. Wenn
die Sonne nun am Ende ihrer Lebenszeit zu einem schwarzen Loch mit dem
Radius von $5000$m kollabiert, wird die Sonne $35$\%
der aktuellen Masse verloren haben. Ein Lichtteilchen mit der Masse $m_f$
und der Fluchtgeschwindigkeit $v_f$ will dem schwarzen Loch entkommen.  Sie
haben folgende Formeln f{\"u}r die kinetische Energie des Lichtteilchens
$E_{kin}$ und der Graviationsenergie des schwarzen Lochs $E_{grav}$
gegeben\footnote{Die Quelle der Inspiration
  für die Aufgabe war ein Montagnachtfilm:
  \href{https://de.wikipedia.org/wiki/Event_Horizon_–_Am_Rande_des_Universums}{Event Horizon – Am Rande des Universums}}.

\begin{center}
  \begin{tabular}{cc}
    $E_{kin} = \cfrac{1}{2}m_fv_f^2$ & $E_{grav} = \cfrac{Gm_sm_f}{r_s}$\\
  \end{tabular}
\end{center}

mit

\begin{itemize}[noitemsep]
\item $m_f$, gleich der Masse [kg] des fliehenden Objektes
\item $m_s$, gleich der Masse [kg] des station{\"a}ren Objekts
\item $r_s$, gleich dem Radius [m] des station{\"a}ren Objekts  
\item $G$, gleich der Gravitationskonstante mit $6.165 \cdot 10^{-11}
  m^3(kg \cdot s^2)^{-1}$ 
\end{itemize}

Im Folgenden wollen wir uns mit der Frage besch{\"a}ftigen, ob das
Lichtteilchen der Gravitation des schwarzen Lochs entkommen kann.

\begin{enumerate}
\item Geben Sie die Formel f{\"u}r die Fluchtgeschwindigkeit $v_f$ an! 
  \textbf{(1 Punkt)}
\item {\"U}berpr{\"u}fen Sie Ihre umgestellte Formel nach $v_f$ anhand der Einheiten!
  \textbf{(1 Punkt)} 
\item Berechnen Sie die notwendige Fluchtgeschwindigkeit $v_f$ des
  Lichtteilchens mit den angegebenen Informationen! \textbf{(2 Punkte)}
\item Gehen Sie von einer Lichtgeschwindigkeit von $\ensuremath{2.9\times 10^{8}}m/s$
  aus. Kann das Lichtteilchen der Gravitation des schwarzen Lochs
  entkommen? Begr{\"u}nden Sie Ihre Antwort! \textbf{(2 Punkte)}
\item Stellen Sie den Zusammenhang zwischen dem sich verringernden Radius
  $r$ des schwarzen Lochs bei gleichbleibender Masse $m_s$
  und der notwendigen Fluchtgeschwindigkeit $v_f$ in einer Abbildung dar!
  \textbf{(2 Punkte)}
 \item Ein Amboss und ein {"A}ffchen st{\"u}rzen aus gro{\ss}er und gleicher H{\"o}he
  in ein schwarzes Loch. Welches der beiden Objekte {\"u}berschreitet zuerst
  den Ereignishorizont des schwarzes Loches? Begr{\"u}nden
  Sie Ihre Antwort mathematisch! \textbf{(2 Punkte)}  
\end{enumerate}

 
\clearpage
% -----------------------------------------------------------------------

\section{Aufgabe \hfill (10 Punkte)}

\textit{Geben Sie grunds{\"a}tzlich Formeln und Rechenweg zur L{\"o}sung der
  Teilaufgaben mit an!} \\[1Ex]

%% --------------------------------------------------------------------
\hfill\href{https://youtu.be/iCQogS6KhPM}{\includegraphics[width =
  2cm]{img/youtube}} %%youtube
\hspace{2Ex}
%% --------------------------------------------------------------------

\paragraph{Das Fermi Paradoxon}



Der Kernphysiker Enrico Fermi diskutierte 1950 auf dem Weg zum Mittagessen
im Los Alamos National Laboratory mit seinen Kollegen angebliche
UFO-Sichtungen und fragte schlie{\ss}lich: "`Where is everybody?"'. Warum seien
weder Raumschiffe anderer Weltraumbewohner noch andere Spuren
extraterrestrischer Technik zu beobachten? Wie lange w{\"u}rde eine au{\ss}erirdische
Zivilisation ben{\"o}tigen um die gesamte Milchstra{\ss}e zu
besuchen, wenn das maximale Reisetempo die Geschwindigkeit der Voyager 1 Sonde w{\"a}re?\footnote{Die Quelle der Inspiration
  für die Aufgabe war folgender Wikipediaeintrag:
  \href{https://de.wikipedia.org/wiki/Fermi-Paradoxon}{Fermi-Paradoxon}}\\[-1ex]

Wir treffen folgende Annahmen. Eine au{\ss}erirdische Zivilisation schickt $drei$
Voyager 1 {\"a}hnliche Sonden mit der Geschwindigkeit von $\ensuremath{6.2587\times 10^{4}}km/h$
los um sich auf den erreichten Planeten selbst zu replizieren. Nach
$500$ Jahren ist die Replikation abgeschlossen und wiederum
$drei$ Sonden werden ausgesendet. Gehen Sie von
$6.23$ Lichtjahren als mittlerer Abstand der Sterne in der
Milchstra{\ss}e aus. Es gibt $\ensuremath{1.5\times 10^{11}}$ Sterne in der Milchstra{\ss}e. Nehmen
Sie eine Lichtgeschwindigkeit von $\ensuremath{2.8\times 10^{8}}m/s$ an.

\begin{enumerate}
\item Skizzieren Sie in einer Abbildung die ersten drei Schritte der
  Vervielf{\"a}ltigung der Sonden in der Galaxie! Beschriften Sie die Abbildung
  mit der Dauer und der Anzahl an Sonden f{\"u}r jeden Schritt der Vervielf{\"a}ltigung! \textbf{(4 Punkte)}
\item Berechnen Sie die theoretische Anzahl an Vervielf{\"a}ltigungsschritten die
  ben{\"o}tigt werden um mit \textit{einem einzigen Vervielf{\"a}ltigungsschritt} die
  gesamten Sterne der Milchstra{\ss}e mit Sonden zu besuchen! \textbf{(2 Punkte)}
\item Berechnen Sie die Dauer, die eine au{\ss}erirdische Zivilisation
  ann{\"a}hrungsweise ben{\"o}tigt um die gesamten Sterne der Milchstra{\ss}e mit
  Sonden zu besuchen! \textbf{(2 Punkte)}
\item Bei einem vermutetet Alter der Erde von $\ensuremath{4.5\times 10^{9}}$ Jahren,
  wie oft war dann eine Sonde einer au{\ss}erirdischen Zivilisation schon zu
  Besuch? Korrigieren Sie Ihre Antwort mit dem Wissen, dass sich die
  Kontinentalplatten einmal alle $\ensuremath{1.2\times 10^{8}}$ Jahre vollst{\"a}ndig im
  Erdinneren umgewandelt haben! \textbf{(2 Punkte)}
\end{enumerate}


 
\clearpage
% -----------------------------------------------------------------------

\section{Aufgabe \hfill (10 Punkte)}

\textit{Geben Sie grunds{\"a}tzlich Formeln und Rechenweg zur L{\"o}sung der
  Teilaufgaben mit an!} \\[1Ex]

%% --------------------------------------------------------------------
\hfill\href{https://youtu.be/tDgr6fpkkYA}{\includegraphics[width =
  2cm]{img/youtube}} %%youtube
\hspace{2Ex}
%% --------------------------------------------------------------------

\paragraph{Pyramiden bauen}



Es stehen die bayrischen Pyramidentage an und Sie sind auf
abenteuerlichen Wegen f{\"u}r den Bau der Pyramiden zust{\"a}ndig. Zu allem
{\"U}berfluss handelt es sich auch noch eine \textit{Reenactment}
Veranstaltung. Thema der diesj{\"a}hrigen Pyramidentage sind die Pyramiden von
Meroe, die den K{\"o}nigen und K{\"o}niginnen des historischen Reiches von Kusch in
Nubien, dem heutigen Sudan, als Grabst{\"a}tten dienten. Die Pyramiden in Meroe
fallen durch ihren steilen Winkel von 74 Grad im Vergleich zu
den {\"a}gyptischen Pyramiden mit 53 Grad auf. Die durchschnittliche
Seitenl{\"a}nge der Grundfl{\"a}che einer Pyramide betr{\"a}gt 33 K{\"o}nigsellen. Eine K{\"o}nigselle
misst 52.2cm.\\

\textit{L{\"o}sen Sie diese Aufgabe mit Hilfe einer Skizze der Pyramide. Bezeichnen
  Sie Seiten und die Winkel der Pyramide entsprechend!}

\begin{enumerate}
\item Bei der K{\"o}nigspyramide von Meroe soll eine Seitenl{\"a}nge der Grundfl{\"a}che
  33 K{\"o}nigsellen lang sein. Welche H{\"o}he der K{\"o}nigspyramide in $m$
  ergibt sich? \textbf{(1 Punkt)}
\item Die Au{\ss}enfl{\"a}chen der Pyramide soll begr{\"u}nt werden. F{\"u}r die
  Bepflanzung muss eine 5cm dicke Torfschicht auf die
  Pyramide aufgebracht werden. Berechnen Sie die ungef{\"a}hre Menge an
  ben{\"o}tigten Torf in $m^3$! \textbf{(2 Punkte)}
\end{enumerate}

Wie in jedem guten \textit{Reenactment} gibt es viel Oberschicht, aber nur
3 Sklaven, die Ihnen bei dem Bef{\"u}llen der Pyramide mit Schutt
zu Seite stehen. Leider haben Ihre Sklaven zu allem {\"U}berfluss auch noch
chronische R{"u}ckenschmerzen entwickelt, als sie von der
anstehenden Aufgabe erfahren haben. Gehen Sie daher von einer Effizienz der
Sklaven von 70\% aus. In eine Schubkarre passen
100 Liter.

\begin{enumerate}
  \setcounter{enumi}{2}
\item Wie oft m{\"u}ssen Ihre maladen Sklaven die Rampe mit der Schubkarre
  zur Spitze der Pyramide hochfahren um die Pyramide mit Schutt zu f{\"u}llen? \textbf{(1 Punkt)}
\item Berechnen Sie die L{\"a}nge der Rampe zur Spitze der Pyramide mit einem
  Anstellwinkel von $9^\circ$! \textbf{(2 Punkte)}
\item Wie weit reicht Ihre Rampe vom Fu{\ss} der Pyramide in die bayrische
  Landschaft?  \textbf{(2 Punkte)}
\end{enumerate}

Bei der Besichtigung der Pyramide teilt Ihnen der leicht {\"u}bergewichtige
Pharao (Nebenberuf \textit{Versicherungsverteter}) mit, das die Pyramide zu
flach sei und somit nicht in die bayrische Landschaft passen
w{\"u}rde. Sie m{\"u}ssen nochmal ran.

\begin{enumerate}
  \setcounter{enumi}{5}
\item Die Grundfl{\"a}che der Pyramide {\"a}ndert sich nicht. Berechnen Sie die
  {\"A}nderung der H{\"o}he in K{\"o}nigsellen, wenn sich der Anstellwinkel der
  Pyramide um $7^\circ$ {\"a}ndert!  \textbf{(2 Punkte)}
\end{enumerate}



% https://de.wikipedia.org/wiki/Rechtwinkliges_Dreieck
% https://www.matheretter.de/wiki/pyramide
% https://de.wikipedia.org/wiki/Knickpyramide 
\clearpage
% -----------------------------------------------------------------------

\section{Aufgabe \hfill (12 Punkte)}

\textit{Geben Sie grunds{\"a}tzlich Formeln und Rechenweg zur L{\"o}sung der
  Teilaufgaben mit an!} \\[1Ex]

%% --------------------------------------------------------------------
\hfill\href{https://youtu.be/3LAq3R0rS14}{\includegraphics[width =
  2cm]{img/youtube}} %%youtube
\hspace{2Ex}
%% --------------------------------------------------------------------

\paragraph{Geocaching -- Von Satelliten und Plastikdosen}



Es ist Wochenende und das Wetter ist \textit{sweet}. Sie schwingen sich auf
Ihr Cachermobil um mit 18km/h, geleitet von modernster
Satellietentechnologie und einem Supercompter aus dem Jahr 2000 in Ihren
H{\"a}nden, Plastikdosen in der Natur und an sehenswerten Orten zu finden. Sie
wollen diesmal endlich die abw{"a}rts Terrainchallenge
durchf{\"u}hren. Die Reihenfolge der Caches nach Terrainwertung gibt daher
die von Ihnen abzufahrenden Orte vor. Die Terrain- und
Schwierigkeitswertungen laufen von 1 (leichteste Wertung) bis 5
(schwierigste Wertung) in 0.5 Schritten. Folgende Informationen zu den
Orten und den entsprechenden Caches stehen Ihnen f{\"u}r Ihre Planung der Route
zu Verf{\"u}gung\footnote{Die Quelle der Inspiration
  für die Aufgabe war folgende Tätigkeit:
  \href{https://www.geocaching.com/play}{Geocaching -- Mach mit bei der weltweit größten Schatzsuche.}}.

\begin{center}
  \begin{tabular}{ ccc }
    \toprule
    Ort & Cache & Wertung (S|T|G) \\
    \midrule
    A & GCN2HP3 & 4.0 | 2.0 | Klein \\
    B & GCJCNBS & 4.5 | 3.0 | Klein \\ 
    C & GCXI3DG & 1.5 | 1.0 | Mikro \\ 
    D & GCUVFFW & 3.5 | 3.5 | Klein \\ 
    E & GCRO3QJ & 5.0 | 4.0 | Mikro \\     
 \bottomrule
\end{tabular}
\end{center}

Im Weiteren sind Ihnen folgende Informationen zu den Entfernungen der Orte
zug{\"a}nglich. Der Entfernungsvektor $\overrightarrow{AC}$ ist
$3$km. Im Weiteren ist Ihnen der Entfernungsvektor
$\overrightarrow{CB}$ mit $6.5$km bekannt. Der
Entfernungsvektor $\overrightarrow{BE}$ ist das $2.1$-fache
des Entfernungsvektor $\overrightarrow{CB}$. Wenn Sie von dem Ort A den Ort
C anpeilen, so liegt der Ort B ungef{\"a}hr $30^\circ$
n{"o}rdlich. Wenn Sie von dem Ort C den Ort B anpeilen, so liegt
der Ort D ungef{\"a}hr $45^\circ$ {\"o}stlich. Vom Ort B
betrachtet, bilden die Orte C und D einen rechten Winkel am Ort B. Der Ort
B liegt auf gerader Linie zwischen den Orten C und E. Somit liegt der Ort E
n{"o}rdlich von B. Die Strecke zwischen A und E ist nicht
passierbar. Sie starten an dem Ort C Ihre Cachertour. \\

\begin{enumerate}
\item L{\"o}sen Sie diese Aufgabe mit Hilfe einer aussagekr{\"a}ftigen Skizze der
  Orte und Caches. Bezeichnen Sie die Strecken und die Winkel Ihrer Skizze
  entsprechend! \textbf{(2 Punkte)}
\item Welche Strecke in $km$ legen Sie bei der Bew{\"a}ltigung der
  abw{"a}rts Terrainchallenge zur{\"u}ck? \textbf{(5
    Punkte)}
\item Gehen Sie von einer zus{\"a}tzlichen Suchzeit in Stunden f{\"u}r die
  Caches an den jeweiligen Orten zur reinen Reisezeit mit Ihrem Cachermobil
  aus. Die Suchzeit in Stunden f{\"u}r \textit{jeden einzelnen} Cache wird durch die
  Funktion
  \begin{equation*}
    Suchzeit = 0.2 + 0.25 \cdot Schwierigkeit
  \end{equation*}  
  beschreiben.  Wie lange in Stunden ben{\"o}tigen Sie um die
  abw{"a}rts Terrainchallenge zu erf{\"u}llen? \textbf{(3 Punkte)}
\item An der h{\"o}chsten Schwierigkeit m{\"u}ssen Sie angeln. Ihre Angel ist
  ausgefahren 6m lang. Erreichen Sie einen Cache in der H{\"o}he
  von 7.7m?  Berechnen Sie dazu Ihre maximale m{\"o}gliche
  Angelh{\"o}he! Welche Annahmen mussten Sie treffen um die Aufgabe zu l{\"o}sen? \textbf{(2 Punkte)} 
\end{enumerate}

 
\clearpage
% -----------------------------------------------------------------------

\section{Aufgabe \hfill (10 Punkte)}

\textit{Geben Sie grundsätzlich Formeln und Rechenweg zur L{\"o}sung der
  Teilaufgaben mit an!} \\[1Ex]

%% --------------------------------------------------------------------
\hfill\href{https://youtu.be/4-dSaPMhK9s}{\includegraphics[width =
  2cm]{img/youtube}} %%youtube
\hspace{2Ex}
%% --------------------------------------------------------------------

%% --------------------------------------------------------------------
{\tiny\textbf{Stichworte:} Brot aus Luft $\bullet$ Walöl $\bullet$ Haber-Bosch-Verfahren $\bullet$ 1. Weltkrieg $\bullet$ 40\% N im menschlichen Körper $\bullet$ Positivist}
%% --------------------------------------------------------------------

\paragraph{Die atmende Wand und Brot aus Luft}



Als Kellerkind vom Dorf wollen das Ausmaß der Radonbelastung in ihrem Kellerzimmer bestimmen und
lüften daher nicht. Spart dann auch Energie und lüften wird sowieso überschätzt. Während einer Messperiode von 7:00 Uhr bis
15:00 bestimmen Sie dreimal automatisch die Radonbelastung in
Ihrem Kellerraum in $Bq/m^3$. Es ergibt sich folgende Abbildung\footnote{Die Quelle der Inspiration
  für die Aufgabe war folgender Artikel:
  \href{https://de.wikipedia.org/wiki/Atmende_Wand}{Atmende Wand}}.

\begin{knitrout}
\definecolor{shadecolor}{rgb}{0.969, 0.969, 0.969}\color{fgcolor}

{\centering \includegraphics[width=\maxwidth]{img/math-10-1} 

}


\end{knitrout}

\vspace{-0.75cm}

\begin{enumerate}
\item Wie lange dauert es in Stunden bis Sie eine kritische Belastung von
  400$Bq/m^3$ in Ihrem ungelüfteten Kellerraum erreicht haben?
  \textbf{(2 Punkte)}
\end{enumerate}

Radon zerfällt mit einer Halbwertszeit von 2.8d zu
Polonium. Polonium wiederum zerfällt mit einer Halbwertszeit von
180d zu Blei. Nur Radon und Polonium tragen zur
radioaktiven Strahlenbelastung bei.

\begin{enumerate}
  \setcounter{enumi}{1}
\item Wie lange dauert es in Stunden bis Ihre kritische Radonbelastung von
  400$Bq/m^3$ auf unter 100$Bq/m^3$ gefallen ist?
  \textbf{(4 Punkte)}
\end{enumerate}

Folgende Tabelle enthält die Informationen zur Zusammensetzung der
normalen Umgebungsluft.

\begin{center}
  \begin{tabular}{ c|c|c|c }
     & Vol-\% & M [g/mol] & ppm \\
    \hline
    Stickstoff & 79.7 & 28.4 &
                                                    \phantom{1000000000000}\strut\\
        \hline
    Sauerstoff & 19.5 & 16.5 &
                                                    \phantom{10000000}\strut\\
        \hline
    Kohlenstoffdioxid & 0.045 & 11.8 & \phantom{10000000}\strut\\     
     \hline
\end{tabular}
\end{center}

\begin{enumerate}
   \setcounter{enumi}{2}
\item Rechnen Sie die Volumenprozente (Vol-\%) der Umgebungsluft in die
  entsprechenden ppm-Werte um und ergänzen Sie die berechneten ppm-Werte in
  die Tabelle!  \textbf{(1 Punkt)}
\end{enumerate}

Während Sie Ihr etwas pappiges Toastbrot mampfen kommt Ihnen die Dokumentation über Brot aus Luft in den Sinn. Sie denken darüber ein wenig nach. Für die Umwandlung von Stickstoff $N_2$ mit Wasserstoff $H_2$ zu Ammoniak
$NH_3$ gilt folgende Reaktionsgleichung\footnote{Die Quelle der Inspiration
  für die Aufgabe war folgender Artikel:
  \href{https://www.gdch.de/netzwerk-strukturen/fachstrukturen/ag-chemie-und-gesellschaft/projekte-und-veranstaltungen/cartoons/haber-bosch-verfahren-brot-aus-luft.html}{Haber-Bosch-Verfahren – Brot aus Luft}}:

\begin{equation*}
  N_2 + 3H_2 \rightarrow 2NH_3
\end{equation*}  

Ein Mol eines beliebigen Gases hat bei normalen Umweltbedingungen ein
Volumen von 22.4 Liter. % Ein Mol enthält $6 \cdot 10^{23}$ Teilchen.

\begin{enumerate}
  \setcounter{enumi}{3}
\item Welche Masse an Ammoniak in Kilogramm $kg$ können Sie aus einem
  Kubikmeter $m^3$ Luft unter normalen Umweltbedingungen gewinnen?
  \textbf{(2 Punkte)}
\item Wieviel Ammoniak in $mol$ erhalten Sie aus einem Kubikmeter Luft?
  \textbf{(1 Punkt)}
\end{enumerate}

 
\clearpage
% -----------------------------------------------------------------------

\section{Aufgabe \hfill (10 Punkte)}

\textit{Geben Sie grunds{\"a}tzlich Formeln und Rechenweg zur L{\"o}sung der
  Teilaufgaben mit an!} \\[1Ex]

%% --------------------------------------------------------------------
\hfill\href{https://youtu.be/Bbu6n8MXxQk}{\includegraphics[width =
  2cm]{img/youtube}} %%youtube
\hspace{2Ex}
%% --------------------------------------------------------------------

\paragraph{Armee der Finsternis}



Ihr Studentenjob war nach Ladenschluss bei Penny die Regale
einzur{\"a}umen. Dabei ist Ihnen in der Auslage der Sonderangebote das
Necronomicon\footnote{Ein wirklich gef{\"a}hrliches Buch ist:
  \textit{Du bist genug: Vom Mut, gl{\"u}cklich zu sein} von Fumitake Koga und
  Ichiro Kishimi} in die H{\"a}nde gefallen. Nun sind Sie ein Magier der
Zeichen geworden! Also eigentlich k{\"o}nnen Sie nur Mathe und das d{\"a}mliche
Necronomicon hat Sie in die Vergangenheit geschleudert... aber gut, was tut
man nicht alles im Jahr 818 n. Chr. f{\"u}r den neuen Lehnsherren
F{"u}rsten Arthur. Sie bauen nat{\"u}rlich einen Schrottkugelturm um sich den
Horden der Finsternis mit genug Schrott erwehren zu k{\"o}nnen! Ihnen stehen
zwei m{\"a}chtige magische Formeln zur Unterst{\"u}tzung zu Verf{\"u}gung.

\begin{center}
  \begin{tabular}{cc}
    $E_{kin} = \cfrac{1}{2}\cdot m \cdot v^2$ & $E_{pot} = m \cdot g \cdot h$\\
  \end{tabular}
\end{center}

mit

\begin{itemize}
\item $m$, gleich der Masse [kg] des Objekts
\item $h$, gleich der H{\"o}he [m] des ruhenden Objekts
\item $v$, gleich der Geschwindigkeit [m/s] des Objekts
\item $g$, gleich der Erdbeschleunigung mit $9.81 \tfrac{m}{s^2}$ 
\end{itemize}

Als erstes m{\"u}ssen Sie die H{\"o}he des zu bauenden Schrottkugelturmes
bestimmen. Hierf{\"u}r ist wichtig zu wissen, dass sich die
Blei\textit{tropfen} mit einem Gewicht von $10mg$ zu
gleichf{\"o}rmigen Blei\textit{kugeln} bei einer Geschwindigkeit von
$13m/s$ bilden.

\begin{enumerate}
\item Wie hoch m{\"u}ssen Sie den Schrottkugelturm bauen lassen, damit sich
  runde Bleikugeln durch die Fallgeschwindigkeit von $13m/s$ bilden?
  \textbf{(3 Punkte)}
\end{enumerate}

Ihre erstellten Schrottkugeln sind leider zu gro{\ss} und somit sind zu wenige
Schrottkugeln in einer Ladung. Damit k{\"o}nnen Sie die Armee der Finsternis
nicht aufhalten. Die Sachlage m{\"u}ssen Sie einmal mathematisch untersuchen.

\begin{enumerate}
  \setcounter{enumi}{1}
\item Nennen Sie die beiden geometrischen Formen aus denen sich
  n{\"a}herungsweise ein Tropfen zusammensetzt! Erstellen Sie eine beschriftete 
  Skizze des Tropfens!
  \textbf{(2 Punkte)}
\item Sie messen eine L{\"a}nge des Tropfens von 4.1mm. Die L{\"o}cher
  im Sieb erlauben ein Tropfendurchmesser von 1.6mm. Welchen
  Durchmesser in mm haben Ihre produzierten Blei\textit{kugeln}?  \textbf{(3 Punkte)}
\end{enumerate}

Sie haben jetzt die \ensuremath{1.2\times 10^{6}} Bleikugeln zusammen. Blei hat eine Dichte
von $11.23g/cm^3$.

\begin{enumerate}
  \setcounter{enumi}{3}
\item Wie schwer in Kilogramm $kg$ sind die \ensuremath{1.2\times 10^{6}} produzierten
  Bleikugeln, die Sie jetzt auf die Burgmauer transportieren m{\"u}ssen?
  \textbf{(1 Punkt)}
\end{enumerate}

Am Ende m{\"u}ssen Sie noch die Produktion von dem Bleischrott im Turm optimieren.

\begin{enumerate}
  \setcounter{enumi}{4}
\item Wie gro{\ss} in $cm^2$ ist Ihr quadratisches Sieb am oberen Ende des Turms,
  wenn Sie pro Fall ca. 1100 Bleikugeln produzieren wollen und
  die Bleikugel im Fall 1.2cm Abstand haben m{\"u}ssen?  \textbf{(1
    Punkt)}
\end{enumerate}
 
\clearpage
% -----------------------------------------------------------------------

\section{Aufgabe \hfill (10 Punkte)}

\textit{Geben Sie grunds{\"a}tzlich Formeln und Rechenweg zur L{\"o}sung der
  Teilaufgaben mit an!} \\[1Ex]

%% --------------------------------------------------------------------
\hfill\href{https://youtu.be/Mr6eslls4J0}{\includegraphics[width =
  2cm]{img/youtube}} %%youtube
\hspace{2Ex}
%% --------------------------------------------------------------------

\paragraph{Armee der Kaninchen}



Leider hat es mit Ihrer Faultierpension in Down Under nicht geklappt. War vielleicht
auch nicht \textit{so} die beste Idee... aber daf{\"u}r haben Sie eine Neue!
Oder wie es Mike Tyson zugeschrieben wird: "`Ich wurde nie
niedergeschlagen, ich war immer am Aufstehen!"'. Daher machen Sie jetzt
einen Gro{\ss}handel mit Kaninchenfleisch und damit dem teuersten Fleisch in
Australien auf. Moment, hopsen hier nicht, seit Thomas Austin im Jahr
1860 ungef{\"a}hr 30 Kaninchen entlassen hat,
Millionen von Kaninchen rum? Wieso ist das Kaninchenfleisch dann so
exklusiv? Dem wollen wir mal mathematisch nachgehen!\footnote{Die Quelle
  der Inspiration f{\"u}r die Aufgabe war der folgendes YouTube Video:
  \href{https://youtu.be/38fuOr3tdgc?si=Li7NL_FoByML8JtT}{ Incredible
    Stories -- Why don't they eat wild rabbits in Australia? They have
    millions of them! The reason is surprising...}}
\\

Forscherinnen fand folgende S{\"a}ttigungsfunktion f{\"u}r das j{\"a}hrliche Wachstum
der gesamten Kaninchenpopulation im westlichen Australien.
\begin{equation*}
  f(t) = \ensuremath{1.2\times 10^{10}} - \ensuremath{1.4\times 10^{9}} \cdot 1.7^{-0.3 \cdot t + 2.4}
\end{equation*}

\begin{enumerate}
\item Skizzieren Sie die S{\"a}ttigungsfunktion \textit{ann{\"a}herungsweise} in einer Abbildung! \textbf{(1
    Punkt)}
\item Wie viele Kaninchen k{\"o}nnen nach der S{\"a}ttigungsfunktion maximal im
  westlichen Australien leben? Erg{\"a}nzen Sie den Wert in Ihrer Abbildung! \textbf{(2 Punkte)}
\item Wie viele Millionen Kaninchen leben nach der S{\"a}ttigungsfunktion
  nach 18 Jahren auf dem australischen Kontinent? \textbf{(1
    Punkt)}
\end{enumerate}

Um den Kaninchen Einhalt zu gebieten wurde das Myxoma Virus und das Rabbit
Haemorrhagic Disease Virus (RHDV) in 12 Kaninchen
ausgebracht. Da die Kaninchen keine Ma{\ss}nahmen gegen die Ausbreitung
vornehmen k{\"o}nnen, verl{\"a}uft die Ausbreitung mit einem w{\"o}chentlichen
Wachstumsfakor von 1.2 nach folgender Formel.

\begin{equation*}
  N(t) = N(0) \cdot a^t
\end{equation*}

\begin{enumerate}
  \setcounter{enumi}{2}
\item Wie viele Wochen ben{\"o}tigen die Viren um theoretisch die gesamte
  Kaninchenpopulation nach 16 Jahren Wachstum zu
  durchseuchen? \textbf{(1 Punkt)}
\end{enumerate}

Das Myxoma Virus und das RHDV t{\"o}ten 99.9\% der
Kaninchenpopulation innerhalb weniger Wochen.

\begin{enumerate}
  \setcounter{enumi}{3}  
\item Wie lange in Jahren dauert es bis eine Kaninchenpopulation nach einer
  Viruspandemie wieder auf 30\% der ges{\"a}ttigten
  Kaninchenpopulation angewachsen ist?  \textbf{(2 Punkte)}
\end{enumerate}

Thomas Austin entlie{\ss} die Kaninchen im {\"a}u{\ss}ersten Osten von
Australien. Australien hat eine West-Ost-Ausdehnung von 4400km
und eine Nord-S{\"u}d-Ausdehnung von knapp 3400km. Die Kaninchen
breiten sich radial mit einer Geschwindigkeit von 7.8km pro
Jahr aus.

\begin{enumerate}
  \setcounter{enumi}{4}
\item Wie lange dauert es in Jahren bis die Kaninchen jeden Ort
  in Australien erreicht haben? \textit{L{\"o}sen Sie die Aufgabe unter der
    Verwendung einer schematischen Skizze!} \textbf{(2 Punkte)}
\end{enumerate}

Eine j{\"a}hrliche Impfung gegen das Myxoma Virus und das Rabbit Haemorrhagic
Disease Virus (RHDV) kosten 13\$ pro Tier und der
durchf{\"u}hrende Arzt verlangt ca. 40\$ pro Tier.

\begin{enumerate}
  \setcounter{enumi}{5}
\item In Ihrem Stall leben 1200 Mastkaninchen. Mit welchen
  j{\"a}hrlichen Zusatzkosten f{\"u}r die Impfungen der Kaninchen m{\"u}ssen Sie daher
  kalkulieren? \textbf{(1 Punkt)}
\end{enumerate}
 
\clearpage
% -----------------------------------------------------------------------

\section{Aufgabe \hfill (12 Punkte)}

\textit{Geben Sie grunds{\"a}tzlich Formeln und Rechenweg zur L{\"o}sung der
  Teilaufgaben mit an!} \\[1Ex]

%% --------------------------------------------------------------------
\hfill\href{https://youtu.be/fiWGgCX-cE4}{\includegraphics[width =
  2cm]{img/youtube}} %%youtube
\hspace{2Ex}
%% --------------------------------------------------------------------



\paragraph{L{"u}neburger Heide. Unendliche Weiten.}



Wir schreiben das Jahr 2024. Dies sind die Abenteuer
der Kuh Fridolin und Ihnen. Gr{\"u}nes Gras unter Ihren F{\"u}{\ss}en und
ein strammer Wind im Gesicht, egal wohin Sie schauen. Ein schmatzendes
Ger{\"a}usch ert{\"o}nt unter Ihnen. Sie sinnieren, sollten Sie Ihre weiten
Graslandschaften jetzt schon d{\"u}ngen?  Daf{\"u}r ben{\"o}tigen Sie die
\textit{Gr{\"u}nlandtemperatur}! Die Gr{\"u}nlandtemperatur (GLT) ist die Summe aller
positiven Tagesmitteltemperaturen seit Jahresbeginn. Ab einer GLT von
200$^\circ$ kann mit der Stickstoffd{\"u}ngung begonnen werden. Sie sehen nicht
ein, Geld f{\"u}r einen Agrarmetrologen zu bezahlen. Also rechnen Sie mit
folgenden Informationen zu Monatsmultiplikatoren des GLT-Wertes: Januar mit
$0.5\times$, Februar mit $0.7\times$ und M{\"a}rz mit
$1.1\times$. Sie haben noch im letzten Jahr folgende Temperaturen
gemessen.

\begin{center}
\begin{tabular}{cc}
  \toprule
  Datum & C$^\circ$ \\
  \midrule
  01. Jan 2023 & 0.4\\
  01. Feb 2023 & 1.1\\
  01. Mrz 2023 & 3.1\\
  01. Apr 2023 & 6.1\\
  \bottomrule
\end{tabular}
\end{center}

\begin{enumerate}
\item Erstellen Sie eine Skizze aus den Informationen aus der
  Temperaturtabelle!  \textbf{(1 Punkt)}
\item Stellen Sie die linearen Funktionen $f_1(t)$, $f_2(t)$ und
  $f_3(t)$ aus der obigen Temperaturtabelle auf!  \textbf{(1 Punkt)}
\item Bestimmen Sie die Stammfunktionen $F_1(t)$, $F_2(t)$ und $F_3(t)$ f{\"u}r
  Ihre linearen Funktionen aus der obigen Temperaturtabelle!  \textbf{(1
    Punkt)}
\item Osterglocken beginnen ab einer GLT von 210$^\circ C$ zu bl{\"u}hen. An
  welchem Tag im 1. Quartal des Jahres 2023 war dies der
  Fall? \textit{Ignorieren Sie ein eventuelles Schaltjahr in Ihrer Berechnung.} \textbf{(4 Punkte)}
\end{enumerate}

Auf dem Weg zu Ihrer Jonagoldplantage wurden Sie mit Ihrem Trecker von
einer Gruppe elektrifizierter Renter abgedr{\"a}ngt. Der Trecker muss wieder
aus dem Graben! Fridolin und die elektrifizierten Rentner ziehen an zwei,
separaten Seilen. Dabei zieht Fridolin mit $180N$. Die
elektrifizierter Renter  bringen eine Kraft von $210N$ auf.\\

\textit{L{\"o}sen Sie diese Aufgabe mit Hilfe einer aussagekr{\"a}ftigen Skizze der
  Kraftvektoren. Bezeichnen Sie die Kraftvektoren und die Winkel Ihrer
  Skizze entsprechend!}

\begin{enumerate}
  \setcounter{enumi}{4}  
\item Im ersten Versuch legen Sie das Seil f{\"u}r Fridolin lotrecht {\"u}ber einen
  Ast oberhalb des Treckers. Die Rentner ziehen in einer geraden Linie {\"u}ber
  die B{\"o}schung hinweg am anderen Seil. Welche Kraft wird aufgebracht?
  \textbf{(2 Punkte)}
\item Im zweiten Versuch ziehen Fridolin und die Rentner mit einem
  $40^\circ$ Winkel mit ihrem Seil an dem Trecker. Welche Kraft
  wird aufgebracht? \textbf{(2 Punkte)}
\item Mit welcher Beschleunigung ziehen Sie den $1.2t$ schweren
  Trecker \textit{jeweils} aus dem Graben, wenn $F = m \cdot a$ gilt?
  \textbf{(1 Punkt)}
\end{enumerate}

 
\clearpage
% -----------------------------------------------------------------------

\section{Aufgabe \hfill (10 Punkte)}

\textit{Geben Sie grunds{\"a}tzlich Formeln und Rechenweg zur L{\"o}sung der
  Teilaufgaben mit an!} \\[1Ex]

%% --------------------------------------------------------------------
\hfill\href{https://youtu.be/RuzMjwvwT-4}{\includegraphics[width =
  2cm]{img/youtube}} %%youtube
\hspace{2Ex}
%% --------------------------------------------------------------------

\paragraph{In der Kartonagenfabrik}



Wenn man sich zu sp{\"a}t anmeldet, dann ist die Exkursion nicht so toll. Also
geht es mit Rektor Skinner und Mrs. Krabappel in die Kartonagenfabrik. Wie
schon im vorherigen Semester... In der Kartonagenfabrik angekommen erfahren
Sie, dass die Kartons zum Versand von N{\"a}geln nicht hier zusammengebaut
werden sondern das sich die Endfertigung in Flint, Michigan befindet. Unter
anderem wird dort der ber{\"u}hmte \textit{Doppelt gewellte,
  6-mal-gefaltete, 0.5mm, 30-cm-Karton} durch
Falzung hergestellt. Beim letzten Mal war Rektor Skinner die Stimmung zu
schlecht und deshalb geht es erst nach Hause, wenn ein paar Aufgaben gel{\"o}st
sind. Martin gef{\"a}llt das. An dem Vorrat
an Zigaretten von Mrs. Krabappel meinen Sie wenig Zuversicht zu erkennen.\\

Jetzt hei{\ss}t es Kartons optimieren. Der nun zu optimierende, flache Karton
hat eine L{\"a}nge von 30cm und eine Breite von 22cm. Die
Kartonagenmaschine in Flint soll dann einen quadratischen Eckenausschnitt
der L{\"a}nge $x$ falzen.

\begin{enumerate}
\item Erstellen Sie eine Skizze des Karton\textit{blatt}rohlings!
  Beschriften Sie die Skizze mit den entsprechenden L{\"a}ngenangaben
  \textbf{(1 Punkt)}
\item Berechnen Sie die Falztiefe $x$ f{\"u}r ein maximales Volumen des flachen
  Kartons! \textbf{(3 Punkte)}
\item Welches Volumen in Liter ergibt sich mit der von Ihnen berechneten
  Falztiefe $x$?  \textbf{(1 Punkt)}
\item Sie wollen noch einen b{\"u}ndig mit dem Boden abschlie{\ss}enden Deckel f{\"u}r
  den Karton stanzen lassen. Wie gro{\ss} ist die Fl{\"a}che des
  Kartondeckel\textit{blatt}rohlings in $cm^2$? \textbf{(2 Punkte)}
\end{enumerate}

Rektor Skinner m{\"o}chte sich gerne wieder in seinem Vorgarten aufhalten und
nicht die ganze Zeit von Bart mit Erdn{\"u}ssen beworfen werden. Deshalb m{\"o}chte
er einen ger{\"a}umigen Teil seines Vorgartens einz{\"a}unen. Ein Teil der
Umz{\"a}unung bildet seine Vorderhauswand. Wegen Lieferschwierigkeiten stehen
Rektor Skinner nur 100m Zaun zu Verf{\"u}gung. Sie wollen nun die
maximale Fl{\"a}che des abgeschirmten Vorgartens in Abh{\"a}ngigkeit der
Seitenl{\"a}ngen bei der Verwendung von 100m Zaun bestimmen!

\begin{enumerate}
  \setcounter{enumi}{4}  
\item  Welche Seitenl{\"a}ngen f{\"u}r den Zaun ergeben sich f{\"u}r die
  maximale Fl{\"a}che des abgeschirmten Vorgartens? \textbf{(2 Punkte)}
\item Berechnen Sie die Fl{\"a}che des abgeschirmten Vorgartens! \textbf{(1
    Punkt)}
\end{enumerate}

 
\clearpage
% -----------------------------------------------------------------------

\section{Aufgabe \hfill (10 Punkte)}

\textit{Geben Sie grunds{\"a}tzlich Formeln und Rechenweg zur L{\"o}sung der
  Teilaufgaben mit an!} \\[1Ex]

%% --------------------------------------------------------------------
\hfill\href{https://youtu.be/OhyuH6hzEhY}{\includegraphics[width = 2cm]{img/youtube}} %%youtube
\hspace{2Ex}
%% --------------------------------------------------------------------




\paragraph{Ein Pfund Insekten, bitte!} Das wird wohl h{\"a}ufiger geh{\"o}rt
werden, wenn wir die Menschheit mit Proteinen ausreichend ern{\"a}hren
wollen\footnote{Die Quelle der Inspiration f{\"u}r die Aufgabe war der folgende
  Artikel aus dem Spiegel:
  \href{https://www.spiegel.de/ausland/ueberbevoelkerung-acht-milliarden-sind-wir-bald-zu-viele-menschen-auf-der-erde-a-3f20c7bc-3d60-4440-9f52-eb338db207f5}{Acht
    Milliarden - sind wir bald zu viele Menschen auf der Erde?}}. Schauen
wir uns dazu einmal den Vergleich Deutschland zu Nigeria an. Nigeria hat
eine der am schnellsten wachsenden Bev{\"o}lkerungen der Welt und wird
vermutlich im Jahr 2100 zu den Top 5 der bev{\"o}lkerungsreichsten L{\"a}nder
z{\"a}hlen. Im Jahr 2020 leben ca. \ensuremath{8.4\times 10^{7}} Menschen in
Deutschland und ca. \ensuremath{1.79\times 10^{8}} Menschen in Nigeria. Mit den
Informationen wollen wir anfangen und dann eine Prognose f{\"u}r den
Fleischkonsum im Jahr 2050 zu treffen. \\ 


Im folgenden ist Abbildung des Fleischkonsums im Jahr 2020 in
Deutschland und Nigeria in [kg] einmal dargestellt. 

\begin{knitrout}
\definecolor{shadecolor}{rgb}{0.969, 0.969, 0.969}\color{fgcolor}

{\centering \includegraphics[width=\maxwidth]{img/math-14-a-1} 

}


\end{knitrout}
  
\begin{enumerate}
\item Stellen Sie den Fleischkonsum in Deutschland und Nigeria im Jahr
  2020 \textit{pro Kopf} in einer aussagekr{\"a}ftigen Tabelle dar! 
  \textbf{(2 Punkte)}
\item Erg{\"a}nzen Sie in der Tabelle eine Spalte in der Sie f{\"u}r den
  Fleischkonsum in Nigeria auf Deutschland normieren, daher ins Verh{\"a}ltnis
  Nigeria/Deutschland, setzen! \textbf{(1 Punkt)}
\end{enumerate}

In der n{\"a}chsten Abbildung finden Sie die CO$_2$-Emission in [kg] nach
Lebensmittel, die durch die Produktion entsteht, abgebildet.

\begin{knitrout}
\definecolor{shadecolor}{rgb}{0.969, 0.969, 0.969}\color{fgcolor}

{\centering \includegraphics[width=\maxwidth]{img/math-14-b-1} 

}


\end{knitrout}

\begin{enumerate}
  \setcounter{enumi}{2}  
\item Stellen Sie in einer Tabelle die Treibhausgasemissionen an $CO_2$ pro
  Kopf, die durch den Fleischkonsum in Deutschland und Nigeria im Jahr
  2020 entstehen, dar! Erg{\"a}nzen Sie auch hier das Verh{\"a}ltnis
  Nigeria zu Deutschland! \textbf{(2 Punkte)}
\end{enumerate}

\newpage

In der folgenden Abbildung sehen Sie die
Bev{\"o}lkerungsentwicklung [Millionen] in Nigeria von 1950 bis ins Jahr 2030 fortgef{\"u}hrt.

\begin{knitrout}
\definecolor{shadecolor}{rgb}{0.969, 0.969, 0.969}\color{fgcolor}

{\centering \includegraphics[width=\maxwidth]{img/math-14-c-1} 

}


\end{knitrout}

\begin{enumerate}
  \setcounter{enumi}{3}  
\item Sch{\"a}tzen Sie graphisch die zu erwartende Bev{\"o}lkerung [Millionen] in Nigeria im
  Jahr 2050, die sich anhand der Informationen aus der Abbildung ergibt!
\begin{enumerate}
\item Ohne Ber{\"u}cksichtigung der Covid-19-Pandemie! \textbf{(1
    Punkt)}
\item Unter Ber{\"u}cksichtigung der Covid-19-Pandemie! \textbf{(1
    Punkt)}
\end{enumerate}
\item Berechnen Sie den gesch{\"a}tzten Fleischkonsum von Nigeria im Jahr
  2050 unter der Annahme 70\%-iger Angleichung der Lebensbedingungen zu
  Deutschland im Jahr 2020! \textbf{(1 Punkt)}
\item Berechnen Sie die prozentuale Steigerung der Treibhausgasemissionen
  an $CO_2$ in Nigeria im Jahr 2050 im Vergleich zum Jahr 2020, der
  sich durch den angeglichenen Fleischkonsum ergibt! \textbf{(1 Punkt)}
\item Berechnen Sie die prozentuale Steigerung der Treibhausgasemissionen
  an $CO_2$ in Nigeria, wenn die gesamte Proteinaufnahme 
  durch Insekten ersetzt w{\"u}rde! \textbf{(1
    Punkt)}
\end{enumerate}







 
\clearpage
% -----------------------------------------------------------------------

\section{Aufgabe \hfill (12 Punkte)}

\textit{Geben Sie grunds{\"a}tzlich Formeln und Rechenweg zur L{\"o}sung der
  Teilaufgaben mit an!} \\[1Ex]

%% --------------------------------------------------------------------
\hfill\href{https://youtu.be/flRBo1FWQy0}{\includegraphics[width = 2cm]{img/youtube}} %%youtube
\hspace{2Ex}
%% --------------------------------------------------------------------

\paragraph{T{\"o}dliche Seuche AIDS -- Die r{\"a}tselhafte Krankheit}




Irritiert legen Sie die historische Ausgabe des Spiegels aus den 80zigern
beiseite. Sie sind bei Ihrem HNO-Arzt und wollen einen AIDS-Test machen
lassen. Woanders leider keinen Termin gekriegt... Immerhin denken Sie und
Ihre Partnerin {\"u}ber Nachwuchs nach und da geht es eben nur durch
ungesch{\"u}tzten Sex. Was wissen Sie nun aber {\"u}ber AIDS und dem diagnostischen
AIDS-Test, den Sie nun machen werden?\\

Die Pr{\"a}valenz von AIDS bei einem Menschen in Europa wird mit
0.8\% angenommen. In 96\% der F{\"a}lle ist ein
HIV-Test positiv, wenn der Patient erkrankt ist. In 1\%
der F{\"a}lle ist ein HIV-Test positiv, wenn der Patient \textit{nicht}
erkrankt ist und somit gesund ist. Sie stutzen. Wie wahrscheinlich ist es
denn eigentlich an AIDS erkrankt zu sein $(K^+)$, wenn Sie einen positiven
AIDS-Test vorliegen haben $(T^+)$? Gehen Sie f{\"u}r die folgenden Berechnungen
von $n = \ensuremath{4\times 10^{4}}$ Patienten mit einem diagnostischen Test f{\"u}r AIDS
aus. Sie nehmen sich also einen Kuli und fangen an auf der historischen
Ausgabe des Spiegels zu rechnen\footnote{Die Quelle der Inspiration f{\"u}r die
  Aufgabe war der folgende wissenschaftlicher Artikel: \textit{Binder et
    al. (2022) Von Baumdiagrammen {\"u}ber Doppelb{\"a}ume zu H{\"a}ufigkeitsnetzen --
    kognitive {\"U}berlastung oder didaktische Unterst{\"u}tzung? Journal f{\"u}r
    Mathematik-Didaktik, 1-33}}.

\begin{enumerate}
\item Welche Wahrscheinlichkeit $Pr$ wollen Sie berechnen? \textbf{(1 Punkt)}
\item Zeichnen Sie einen H{\"a}ufigkeitsdoppelbaum zur Bestimmung der gesuchten
  Wahrscheinlichkeit $Pr$! \textbf{(2 Punkte)} 
\item Beschriften Sie den H{\"a}ufigkeitsdoppelbaum, mit denen Ihnen bekannten
  Informationen zu der AIDS Erkrankung und dem AIDS-Test! \textbf{(1 Punkt)}
\item F{\"u}llen Sie den H{\"a}ufigkeitsdoppelbaum mit den sich ergebenden,
  absoluten Patientenzahlen $n$ aus! \textbf{(2 Punkte)}
\item Berechnen Sie die gesuchte Wahrscheinlichkeit $Pr$! \textbf{(1 Punkt)}
\end{enumerate}

Bei dem folgenden Arztgespr{\"a}ch erfahren Sie, dass beim diagnostischen
Testen \textit{True Positives (TP)}, \textit{True Negatives (TN)},
\textit{False Positives (FP)} und \textit{False Negatives (FN)}
auftreten. Das verstehen Sie so noch nicht und deshalb stellen Sie f{\"u}r sich
den Zusammenhang in einer 2x2 Kreuztabelle dar.

\begin{enumerate}
  \setcounter{enumi}{5}
\item Tragen Sie \textit{TP}, \textit{TN}, \textit{FP} und \textit{FN} in
  eine 2x2 Kreuztablle ein. Beschriften Sie die Tabelle entsprechend!
  \textbf{(1 Punkt)}
\item Berechnen Sie die Sensitivit{\"a}t und Spezifit{\"a}t des diagnostischen Tests
  f{\"u}r AIDS! F{\"u}llen Sie daf{\"u}r die 2x2 Kreuztabelle mit den Informationen aus
  dem H{\"a}ufigkeitsdoppelbaum aus! \textbf{(2 Punkte)}
\item Was beschreibt die Sensitivit{\"a}t und die Spezifit{\"a}t im Bezug auf die
  Gesunden und Kranken? Stellen Sie beide diagnostische Ma{\ss}zahlen als
  Wahrscheinlichkeiten $Pr$ dar! \textbf{(2 Punkte)} 
\end{enumerate}



 
\clearpage
% -----------------------------------------------------------------------

\section{Aufgabe \hfill (12 Punkte)}

\textit{Geben Sie grunds{\"a}tzlich Formeln und Rechenweg zur L{\"o}sung der
  Teilaufgaben mit an!} \\[1Ex]

%% --------------------------------------------------------------------
\hfill\href{https://youtu.be/SZqp_vy3rgI}{\includegraphics[width = 2cm]{img/youtube}} %%youtube
\hspace{2Ex}
%% --------------------------------------------------------------------

\paragraph{Network-Marketing oder Schneeballschlacht!}




Eine M{\"o}glichkeit, leicht Geld zu verdienen, ist es anderen Menschen f{\"u}r Geld
zu versprechen, wie man leicht reich werden kann. Am besten nat{\"u}rlich ohne
viel Aufwand und ortsunabh{\"a}ngig. Dann wollen wir mal loslegen. Schnell
ein YouTube-Werbevideo gedreht und auf geht es mit unserem
Network-Marketing. Aber Moment, wie funktioniert Network-Marketing
eigentlich und was hat das alles mit einer Schneeballschlacht zu tun? Wir
wollen hier einmal in die Untiefen des "`passiven Einkommens"'
abtauchen\footnote{Die Quellen der Inspiration f{\"u}r die Aufgabe waren
  folgendes YouTube Video:
  \href{https://youtu.be/UOKkZF_qK9M?si=uf4foJVFKfeQMwSw}{Simplicissimus --
    Die meistgesuchte Betr{\"u}gerin der Welt} und der Artikel:
  \href{https://www.deutschlandfunkkultur.de/netzwerk-marketing-die-illusion-schnell-reich-zu-werden-100.html}{
    Deutschlandfunk Kultur -- Die Illusion, schnell reich zu werden}}.\\

Das Jahr 2022 war das erfolgreichste Jahr in der Geschichte von
KH Gesund und Sch{"o}n Components (KH-GSC). Das Unternehmen steigerte den Umsatz um rund
10 Prozent von 310 Millionen Euro im Jahr
2021. Doch wie viel kommt bei den Partnern an? Laut
KH-GSC habe das Unternehmen \ensuremath{3.6\times 10^{5}} aktive Partner.

\begin{enumerate}
\item Berechnen Sie zuerst den Umsatz der Firma KH-GSC im
  Jahr 2022! \textbf{(1 Punkt)}
\item Wie viel von dem Umsatz im Jahr 2022 wird im Durchschnitt von jedem
  aktiven Partner erwirtschaftet? \textbf{(1 Punkt)}
\item Welche \textit{monatlicher} Umsatz ergibt sich dadurch im
  Durchschnitt f{\"u}r jeden aktiven Partner bei einer direkten Provision von
  35\%? \textbf{(1 Punkt)}
\end{enumerate}

Ihr zu vermarkendes Produkt, hinter dem Sie voll stehen, kostet
200EUR pro Einheit im Direktverkauf. Die direkte Provision
f{\"u}r die erste Stufe betr{\"a}gt 40\%. F{\"u}r die zweite, dritte und
vierte Stufe betragen die indirekten Provisionen jeweils 3.25\%,
2.25\% und 1.75\%. Jeder Ihrer angeworbenen "`Partner"'
wirbt wiederum drei Partner f{\"u}r sich selbst an. Pro Monat
werden im Schnitt vier Einheiten vom Produkt verkauft. Sie wollen nun
3200EUR im Monat \textit{passiv} -- also durch indirekte
Provisionen -- erwirtschaften.

\begin{enumerate}
  \setcounter{enumi}{3}
\item Erg{\"a}nzen Sie die folgende Tabelle mit den obigen Informationen! \textbf{(2 Punkte)}
\end{enumerate}

\begin{center}
\begin{tabular}{c|c|c|c}
  \toprule
  \textbf{Stufe} & \textbf{Anzahl Partner}  & \textbf{Umsatz/Stufe} & \textbf{Provision}\\
  \midrule
  1 & Sie selber  &  & \\ \midrule
  2 &   &  &  \\ \midrule
  3 &   &  &  \\ \midrule
  4 &   &  &  \\
  \bottomrule
\end{tabular}
\end{center}

\begin{enumerate}
  \setcounter{enumi}{4}
\item Wie viele Partner m{\"u}ssen Sie auf der 2 Stufe anwerben um Ihr passives
  Einkommen durch indirekte Provision zu erreichen? Wie viele Menschen
  arbeiten am Ende indirekt f{\"u}r Sie? Stellen Sie den
  Zusammenhang graphisch dar!  \textbf{(3 Punkte)}
\end{enumerate}

Sie mussten zum Einstieg bei KH-GSC Einheiten des Produkts
f{\"u}r 7000EUR kaufen. Diese Einheiten k{\"o}nnen Sie nur direkt
verkaufen. Leider mussten Sie den Kauf {\"u}ber einen Kredit {\"u}ber
5.1\% p.a. {\"u}ber 48 Monate finanzieren.

\begin{enumerate}
  \setcounter{enumi}{5}
\item Berechnen Sie die Gesamtsumme, die Sie als Kredit abbezahlen m{\"u}ssen! \textbf{(2 Punkte)}
\item Wie viele Einheiten m{\"u}ssen Sie pro Monat verkaufen um die anfallenden Zinsen
  durch die direkte Provision zu erwirtschaften? \textbf{(1 Punkt)}
\item Wie lange in
  Monaten ben{\"o}tigen Sie um den Kredit durch die direkte
  Provision abzubezahlen? \textbf{(1 Punkt)}
\end{enumerate} 
\clearpage
% -----------------------------------------------------------------------

\section{Aufgabe \hfill (10 Punkte)}

\textit{Geben Sie grunds{\"a}tzlich Formeln und Rechenweg zur L{\"o}sung der
  Teilaufgaben mit an!} \\[1Ex]

%% --------------------------------------------------------------------
\hfill\href{https://youtu.be/8Pb2sKUIMyk}{\includegraphics[width =
  2cm]{img/youtube}} %%youtube
\hspace{2Ex}
%% --------------------------------------------------------------------

\paragraph{H{\"o}hlen \& Drachen}



Nachdem Sie sich begeistert in der Serie \textit{Stranger Thinks} verloren
haben, wollen Sie bei einem Ihrer Freunde einmal \textit{H{\"o}hlen \& Drachen}
ausprobieren. Um Geld zu sparen, das Zeug kostet echt, wurde etwas an den
Regeln gebastelt. Schnell stellen Sie fest, dass hier ganz sch{\"o}n viele
unterschiedliche W{\"u}rfel durch die Gegend fliegen. Daher m{\"u}ssen Sie sich
jetzt einiges an Fragen stellen. \\%[-1ex]

In dem Spiel haben Sie nun auf einmal 5 zw{"o}lfseitige W{"u}rfel (5d12) zum w{\"u}rfeln in der Hand. Wenn Sie eine 12 w{\"u}rfeln,
haben Sie einen Erfolg.

\begin{enumerate}
\item Berechnen Sie die Wahrscheinlichkeit \textit{genau}
  4 Erfolge zu erzielen!  \textbf{(2 Punkte)}
\item Berechnen Sie die Wahrscheinlichkeit keinen Erfolg zu erzielen!
  \textbf{(1 Punkt)}
\end{enumerate}

Sie betrachten nun aufmerksam die ausufernden Ausr{\"u}stungstabellen. Ihnen
wird aber geholfen und Sie m{\"u}ssen sich jetzt nur zwischen der Axt oder dem
Schwert entscheiden.

\begin{enumerate}
  \setcounter{enumi}{2}
\item W{\"u}rden Sie die Axt mit zwei zw{"o}lfseitigen W{"u}rfeln (2d12) als Schaden oder
  das Schwert mit einem zw{"o}lfseitigen W{"u}rfel plus 6 (1d12+6) als Schaden bevorzugen?
  Begr{\"u}nden Sie Ihre Antwort mathematisch! \textbf{(1 Punkt)}
\end{enumerate}

Jetzt wird es immer wilder, da Sie sich jetzt {\"u}berlegen m{\"u}ssen, wie
wahrscheinlich es ist, dass Ihr Rettungswurf gegen den zaubernden Hexer
funktioniert. Sie haben folgende Wahrscheinlichkeiten gegeben. Die
Wahrscheinlichkeit f{\"u}r das Ereignis $A$, der Rettungswurf ist erfolgreich,
ist $Pr(A) = 0.7$, die Wahrscheinlichkeit f{\"u}r das Ereignis $B$,
der Zauberwurf des Hexers ist erfolgreich, ist $Pr(B) = 0.9$. Sie
haben mitgez{\"a}hlt und festgestellt, dass in $45$ von 100 F{\"a}llen
Ihr Rettungswurf bei einem erfolgeichen Zauber funktioniert hat.  

\begin{enumerate}
  \setcounter{enumi}{3}
\item Erstellen Sie eine 2x2 Kreuztabelle mit den Ereignissen $A$ und $B$
  sowie den Gegenereignissen $\bar{A}$ und $\bar{B}$ mit einen
  $\Omega = 100$. Beachten Sie hierbei die entsprechenden
  Wahrscheinlichkeiten f{\"u}r die Ereignisse $A$ und $B$! \textbf{(2 Punkte)}
\item Bestimmen Sie $Pr(A \cap B)$! \textbf{(1 Punkt)}
\item Erstellen Sie ein Baumdiagramm mit den passenden Informationen aus der 2x2
  Kreuztabelle! \textbf{(2 Punkte)}
\item Bestimmen Sie Wahrscheinlichkeit $Pr(A|B)$, dass Ihr Rettungswurf gelingt, wenn
  der Hexer erfolgreich gezaubert hat! \textbf{(1 Punkt)}
\end{enumerate}

  
\clearpage
% -----------------------------------------------------------------------

\section{Aufgabe \hfill (12 Punkte)}

\textit{Geben Sie grunds{\"a}tzlich Formeln und Rechenweg zur L{\"o}sung der
  Teilaufgaben mit an!} \\[1Ex]

%% --------------------------------------------------------------------
\hfill\href{https://youtu.be/9DQKaXdxT_g}{\includegraphics[width =
  2cm]{img/youtube}} %%youtube
\hspace{2Ex}
%% --------------------------------------------------------------------

\paragraph{Retrocheck im TV}



"`Und hier ist sie wieder, die Show der fantastischen Preise. Seien Sie mit
dabei, wenn es wieder hei{\ss}t: Der Preis ist hei{\ss}!"', ert{\"o}nt es und Sie
fragen sich, ob Sie nicht doch lieber bezahlter Gast bei Barbara Salesch
h{\"a}tten sein sollten. Aber Sie brauchen das Geld und jetzt hei{\ss}t es
Spielschows farmen! Erstmal eine Kaffemaschine von Mitropa
gewinnen. Ein Kandidat gewinnt die Kaffeemaschine von Mitropa, wenn nicht alle
Kandidaten {\"u}berbieten (eng. \textit{outbid}). Mit Ihnen bilden
Annegret und Frida das Team der drei Kandidaten.

\begin{center}
\begin{tabular}{ccc}
  \toprule
  Name & $P(win)$ & $P(outbid)$\\
  \midrule
  Annegret & 0.1 & 0.12\\
  Frida & 0.2 & 0.076 \\
  \bottomrule
\end{tabular}
\end{center}

\begin{enumerate}
\item Mit welcher Wahrscheinlichkeit gewinnen Sie die Kaffeemaschine von
  Mitropa, wenn keiner der Kandidaten {\"u}berbietet? \textbf{(1 Punkt)}
\item Wenn Ihre {\"U}berbietungswahrscheinlichkeit $P(outbid)$ bei
  0.08 liegt, mit welcher Wahrscheinlichkeit gewinnt
  \textit{keiner} die Kaffeemaschine von Mitropa? \textbf{(1 Punkt)}
\end{enumerate}

Gl{\"u}cksrad f{\"u}r Arme auf der Kirmes! Leider hat es f{\"u}r Maren Gilzer nicht
gereicht. Deshalb sind Sie jetzt auf der Kirmes und spielen mit
Catwoman um das gro{\ss}e Geld. Das Gl{\"u}cksrad hat 22
Felder. Sie drehen das Gl{\"u}cksrad zweimal. Auf 6 Feldern
gewinnen Sie 5000EUR sonst 1000EUR. Ganz sch{\"o}n viel Geld
und ganz sch{\"o}n zwielichtig hier...

\begin{enumerate}
  \setcounter{enumi}{2}  
\item Skizzieren Sie das Gl{\"u}cksrad und erg{\"a}nzen Sie die
  Wahrscheinlichkeiten! \textbf{(1 Punkt)}
\item Zeichnen Sie das zugeh{\"o}rige Baumdiagramm f{\"u}r das zweimalige Drehen!
  Erg{\"a}nzen Sie die Wahrscheinlichkeiten und die entsprechenden Ereignisse
  \textbf{(2 Punkte)}
\item Mir welcher Wahrscheinlichkeit gewinnen Sie 6000EUR? \textbf{(1
    Punkt)}
\end{enumerate}

Nach Ihrem Fiebertraum reisen Sie im Zug nach K{\"o}ln um bei "`Geh aufs
Ganze!"' mitzuspielen. Sie schaffen es tats{\"a}chlich ins Finale und k{\"o}nnen
als Hauptgewinn ein Auto hinter einer der drei T{\"u}ren gewinnen. 

\begin{enumerate}
  \setcounter{enumi}{5}  
\item Bevor die Show beginnt, wird das Auto hinter eine zuf{\"a}llig bestimmte
  T{\"u}r gestellt. Mit welcher Wahrscheinlichkeit wird jeweils eine der drei
  T{\"u}ren ausgew{\"a}hlt? Zeichnen Sie ein Baumdiagramm! \textbf{(1 Punkt)}
\item Mit welcher Wahrscheinlichkeit w{\"a}hlen Sie sofort die T{\"u}r mit
  dem Auto? Erweitere Sie das Baumdiagramm entsprechend! \textbf{(1
    Punkt)}
\item Der Moderator {\"o}ffnet nun eine der nicht gew{\"a}hlten T{\"u}ren, aber
  nat{\"u}rlich nicht die mit dem Auto. Mit welcher Wahrscheinlichkeit steht
  das Auto hinter der anderen T{\"u}r? Erweitern Sie das Baumdiagramm
  entsprechend! \textbf{(2 Punkte)}
\item L{\"o}sen Sie nun das "`Ziegenproblem"'! Berechne Sie dazu die
  Wahrscheinlichkeiten der einzelnen Pfade. Lohnt sich ein Wechsel der
  anfangs gew{\"a}hlte T{\"u}r? Begr{\"u}nden Sie Ihre Antwort mathematisch!
  \textbf{(2 Punkte)}
\end{enumerate}
 

%https://unterrichten.zum.de/wiki/Laplace-Wahrscheinlichkeit_wiederholen_und_vertiefen/Ziegen

%Buchstabenh{\"a}ufigkeit im Deutschen

%%https://www.mathelike.de/abiturloesungen-mathematik-bayern/mathematik-abitur-bayern-2020/pruefungsteil-a/stochastik-2/teilaufgabe-a.html

%%https://de.wikipedia.org/wiki/Buchstabenh%C3%A4ufigkeit


 
\clearpage
% -----------------------------------------------------------------------
\part{Angewandte Nutztier- und Pflanzenwissenschaften (M.Sc.)}
% -----------------------------------------------------------------------  

\section{Aufgabe \hfill (6 Punkte)}
Vergleichen Sie die Standardabweichung mit dem Standardfehler und grenzen
Sie die beiden Kennzahlen voneinander ab.



\section{Aufgabe \hfill (8 Punkte)}
Ihnen liegt folgendes Varianzanalysemodell mit der {\"u}blichen Beschreibung
zur Auswertung des Merkmals fett- und eiwei{\ss}korrigierte Milchleistung pro
Kuh und Jahr in kg vor:

\begin{equation*}
  Y_{ijkl} = \mu + Var_i + EKA_j + VarEKA_{ij} + V_k + b(L_{ij} - L) + e_{ijkl}
\end{equation*}

mit

\begin{itemize}
\item $Y_{ijkl}$: l-te Beobachtung
\item $\mu$: Populationsmittel 
\item $Var_i$: fixer Effekt der i-ten Variante (i: Kontrolle,
  Versuchsgruppe 1, Versuchsgruppe 2)
\item $EKA_j$: fixer Effekt der j-ten Erstkalbealtergruppe (j: EKA $\leq$ 25
  Monate, EKA > 25 Monate)
\item $VarEKA_{ij}$: fixer Effekt der Interaktion Variante x
  Erstkalbealtergruppe 
\item $V_k$: zuf{\"a}lliger Effekt des Vaters 
\item $b(L_{ij} - L)$: lineare Kovariable Laktationsnummer 
\item $e_{ijkl}$: zuf{\"a}lliger
  Restfehler
\end{itemize}
Erl{\"a}utern Sie anhand dieses Beispiels die Begriffe fixer Effekt,
Interaktion, zuf{\"a}lliger Effekt und Kovariable und grenzen Sie diese
Begriffe voneinander ab.



\section{Aufgabe \hfill (6 Punkte)}
Wie bestimmen Sie die richtige Stichprobengr{\"o}{\ss}e? Welche Kennzahlen /
statistische Ma{\ss}zahlen ben{\"o}tigen Sie dabei und nennen Sie die
Voraussetzungen.

 
\end{document}
% -----------------------------------------------------------------------


  
