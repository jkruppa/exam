\documentclass[a4paper, 9pt]{scrartcl}\usepackage[]{graphicx}\usepackage[]{xcolor}
% maxwidth is the original width if it is less than linewidth
% otherwise use linewidth (to make sure the graphics do not exceed the margin)
\makeatletter
\def\maxwidth{ %
  \ifdim\Gin@nat@width>\linewidth
    \linewidth
  \else
    \Gin@nat@width
  \fi
}
\makeatother

\definecolor{fgcolor}{rgb}{0.345, 0.345, 0.345}
\newcommand{\hlnum}[1]{\textcolor[rgb]{0.686,0.059,0.569}{#1}}%
\newcommand{\hlstr}[1]{\textcolor[rgb]{0.192,0.494,0.8}{#1}}%
\newcommand{\hlcom}[1]{\textcolor[rgb]{0.678,0.584,0.686}{\textit{#1}}}%
\newcommand{\hlopt}[1]{\textcolor[rgb]{0,0,0}{#1}}%
\newcommand{\hlstd}[1]{\textcolor[rgb]{0.345,0.345,0.345}{#1}}%
\newcommand{\hlkwa}[1]{\textcolor[rgb]{0.161,0.373,0.58}{\textbf{#1}}}%
\newcommand{\hlkwb}[1]{\textcolor[rgb]{0.69,0.353,0.396}{#1}}%
\newcommand{\hlkwc}[1]{\textcolor[rgb]{0.333,0.667,0.333}{#1}}%
\newcommand{\hlkwd}[1]{\textcolor[rgb]{0.737,0.353,0.396}{\textbf{#1}}}%
\let\hlipl\hlkwb

\usepackage{framed}
\makeatletter
\newenvironment{kframe}{%
 \def\at@end@of@kframe{}%
 \ifinner\ifhmode%
  \def\at@end@of@kframe{\end{minipage}}%
  \begin{minipage}{\columnwidth}%
 \fi\fi%
 \def\FrameCommand##1{\hskip\@totalleftmargin \hskip-\fboxsep
 \colorbox{shadecolor}{##1}\hskip-\fboxsep
     % There is no \\@totalrightmargin, so:
     \hskip-\linewidth \hskip-\@totalleftmargin \hskip\columnwidth}%
 \MakeFramed {\advance\hsize-\width
   \@totalleftmargin\z@ \linewidth\hsize
   \@setminipage}}%
 {\par\unskip\endMakeFramed%
 \at@end@of@kframe}
\makeatother

\definecolor{shadecolor}{rgb}{.97, .97, .97}
\definecolor{messagecolor}{rgb}{0, 0, 0}
\definecolor{warningcolor}{rgb}{1, 0, 1}
\definecolor{errorcolor}{rgb}{1, 0, 0}
\newenvironment{knitrout}{}{} % an empty environment to be redefined in TeX

\usepackage{alltt}
\usepackage[ngerman]{babel}

% -----------------------------------------------------------------------

% -----------------------------------------------------------------------
%% ------------------------------------------------------------
%% by J.Kruppa on Friday, February 11, 2022 (11:31)
%% \def\mainDir{\Sexpr{exam_path}}
\def\source{/Users/jokruppa/source/tex}
\usepackage[margin=2cm, includefoot]{geometry}
\setlength{\parindent}{0cm}
\usepackage{booktabs}
\usepackage{amsmath}
\usepackage{scalerel,amssymb}
\usepackage{setspace}
\def\csquare{{\Large $\boxtimes$}}
\def\msquare{{\Large $\square$}}
\usepackage[normalem]{ulem}
\usepackage{array}
\usepackage{xcolor}
\usepackage{float}
\usepackage{currfile}
\usepackage{tikz}
\usepackage[nomessages]{fp}

%% beamer defs
\def\lecture{Klausurfragen der Bio Data Science}

%% exam defs
\def\examtitle{\lecture}
\def\exammodule{
\vspace{-1.75cm}  
\begin{graybox}{}
\vspace{2Ex}
\textbf{\large Name:} \rule[0ex]{16.75em}{.4pt}
\hfill \textnormal{\textit{Nicht bestanden:}} \msquare \\[2.5Ex]
\textbf{\large Vorname:} \rule[0ex]{15em}{.4pt} \\[2.5Ex]
\textbf{\large Matrikelnummer:} \rule[0ex]{10.8em}{.4pt}
\hfill Endnote: \rule[0ex]{7em}{.4pt} 
\end{graybox}
\vspace{3Ex}
\phantom{text}
}
\def\examsemester{Sommersemester \& Wintersemester}
\def\examdate{\today}
%% ------------------------------------------------------------
\definecolor{darkblue}{rgb}{0,0,.5}
\definecolor{darkpurple}{rgb}{0.4117, 0.2, 0.4117}
\definecolor{uni}{rgb}{0,0.3137,0.6078}
\definecolor{gray}{gray}{0.7}

\usepackage{tcolorbox}
\definecolor{logo1}{RGB}{0, 158, 227}
\definecolor{gray5}{RGB}{247, 247, 247}
\definecolor{gray2}{RGB}{102, 102, 102}

\newtcolorbox{graybox}[1]{
  colback=gray5,%%red!5!white,
  colframe=gray2,%%red!75!black,
  fonttitle=\bfseries\Large,
  %%valign=center,
  fontupper=\large,
  before skip=10pt plus 2pt,
  after skip=20pt plus 4pt,
  title=#1}

\newtcolorbox{takehomebox}[1]{
  colback=gray5,%%red!5!white,
  colframe=logo1,%%red!75!black,
  fonttitle=\bfseries\Large,
  %%valign=center,
  fontupper=\large,
  before skip=10pt plus 2pt,
  after skip=10pt plus 2pt,
  title=#1}

\def\Rlogo{\includegraphics[width = 0.5cm]{\string~/Documents/GitHub/exam/img/Rlogo}\;}

\usepackage[scaled=.90]{helvet} 
\usepackage{fancyhdr}
\usepackage{lastpage}
\usepackage{hyperref}
\hypersetup{
    colorlinks=true,       % false: boxed links; true: colored links
    linkcolor=black,          % color of internal links 
    urlcolor=magenta           % color of external links
}
\renewcommand{\familydefault}{\sfdefault}

\title{
\large \exammodule \\[5Ex]
\Huge \examtitle \\[2Ex] 
\Large Hochschule Osnabr{\"u}ck
}
\author{Pr{\"u}fer: Prof. Dr. Jochen Kruppa \\
Fakult{\"a}t f{\"u}r Agrarwissenschaften und Landschaftsarchitektur \\ 
j.kruppa@hs-osnabrueck.de}
\date{Version vom \examdate}

%% ------------------------------------------------------------
%% by J.Kruppa on Tuesday, September 23, 2014 (12:50)
%% Header
\renewcommand{\headrulewidth}{0pt}
\renewcommand{\footrulewidth}{0pt}
\pagestyle{fancy}

\fancyhf{}
\fancyhead[L]{}
\fancyhead[R]{}
\fancyfoot[R]{\thepage}
\fancyfoot[L]{\footnotesize \examtitle}

\fancypagestyle{empty}{
 \fancyhf{}
 \fancyhead[L]{}
 \fancyhead[R]{}
 \fancyfoot[R]{\thepage}
 \fancyfoot[L]{\footnotesize \examtitle}
}

\usepackage{arevtext,arevmath}

\newcommand\Tstrut{\rule{0pt}{2.6ex}}         % = `top' strut
\newcommand\Bstrut{\rule[-0.9ex]{0pt}{0pt}}   % = `bottom' strut
\def\strut{\Tstrut\Bstrut}

% -----------------------------------------------------------------------
\IfFileExists{upquote.sty}{\usepackage{upquote}}{}
\begin{document}
\date{Wintersemester 2024/25 
\vfill
\begin{center}
\includegraphics[width = 1.9cm]{avatare/Alex}\hspace{-8mm}
\includegraphics[width = 1.9cm]{avatare/Jessica}\hspace{-8mm}
\includegraphics[width = 1.9cm]{avatare/Jonas}\hspace{-8mm}
\includegraphics[width = 1.9cm]{avatare/Mark}\hspace{-8mm}
\includegraphics[width = 1.9cm]{avatare/Nilufar}\hspace{-8mm}
\includegraphics[width = 1.9cm]{avatare/Paula}\hspace{-8mm}
\includegraphics[width = 1.9cm]{avatare/Steffen}\hspace{-8mm}
\includegraphics[width = 1.9cm]{avatare/Tina}\hspace{-8mm}
\includegraphics[width = 1.9cm]{avatare/Yuki}\\
\small
\vspace{1.5Ex}
\textit{"`The test of a student is not how much he knows,\\ but how much he wants to know."'\\ --- Alice W. Rollins}
\end{center}}
% -----------------------------------------------------------------------
\maketitle
\fancypagestyle{empty}{
  \fancyfoot[L]{\tiny $\blacksquare\!\square\!\square\!\blacksquare\!\square\!\blacksquare\!\square\!\blacksquare\!\square\!\square\!\square\!\blacksquare\!\blacksquare\!\square\!\square\!\blacksquare\!\blacksquare\!\square\!\blacksquare\!\square$}
}
\thispagestyle{empty}
\clearpage
% -----------------------------------------------------------------------

\begin{graybox}{Erlaubte Hilfsmittel}
  \vspace{1Ex}
  \begin{itemize}
  \item Normaler Taschenrechner ohne Möglichkeit der Kommunikation mit anderen
    Geräten! Ausdrücklich kein Handy!
  \item Eine DIN A4-Seite als beidseitig, selbstgeschriebene,
    handschriftliche Formelsammlung. Keine digitalen Ausdrucke! 
  \item \textbf{\textcolor{red}{Die Verwendung eines roten Farbstiftes ist nicht gestattet! Korrekturfarbe!}}
  \item \textit{You can answer the questions in English without any consequences.}  
  \end{itemize}
\end{graybox}
\vfill

\begin{graybox}{Endnote}
  \vspace{1Ex}
  \begin{itemize}
  \item[] \rule[0ex]{3em}{.4pt}\, von 20\, Punkten sind aus den Multiple
    Choice Aufgaben erreicht.
  \item[] \rule[0ex]{3em}{.4pt}\, von 72 Punkten sind aus den Rechen- und
    Textaufgaben erreicht. 
  \item[] \rule[0ex]{3em}{.4pt}\, von 92 Punkten in Summe.
  \item[] Es wird folgender Notenschlüssel angewendet.   
  \end{itemize}
  \vspace{1ex}
\begin{center}
  \begin{tabular}[c]{cc}
    \toprule
    \textbf{Punkte}	&	\textbf{Note}	\\
    \midrule
    88.0 - 92.0	&	1,0	\\
    83.5 - 87.5	&	1,3	\\
    78.5 - 83.0	&	1,7	\\
    74.0 - 78.0	&	2,0	\\
    69.5 - 73.5	&	2,3	\\
    65.0 - 69.0	&	2,7	\\
    60.5 - 64.5	&	3,0	\\
    55.5 - 60.0	&	3,3	\\
    51.0 - 55.0	&	3,7	\\
    46.0 - 50.5	&	4,0	\\
    \bottomrule
  \end{tabular}
\end{center}
  \vspace{1ex}
\begin{itemize}
\item[] Es ergibt sich eine Endnote von \rule[0ex]{4em}{.4pt}.
\end{itemize}
  \vspace{1Ex}
\end{graybox}

% -----------------------------------------------------------------------
\newpage
% -----------------------------------------------------------------------

\begin{graybox}{Multiple Choice Aufgaben}
  \begin{itemize}
  \item Pro Multipe Choice Frage ist \emph{genau} eine Antwort richtig.
  \item \textbf{Übertragen Sie Ihre Kreuze in die Tabelle auf
      dieser Seite.}
  \end{itemize}

\begin{center}
  \large
  \begin{tabular}{|l|c|c|c|c|c?c|}
    \hline
    & \textbf{A} & \textbf{B} & \textbf{C} & \textbf{D} & \textbf{E} & $\boldsymbol{\checkmark}$\strut\\
    \hline
    \textbf{Aufgabe 1} &   &   &   &   &   & \strut\\
    \hline
    \textbf{Aufgabe 2} &   &   &   &   &   & \strut\\
    \hline
    \textbf{Aufgabe 3} &   &   &   &   &   & \strut\\
    \hline
    \textbf{Aufgabe 4} &   &   &   &   &   & \strut\\
    \hline
    \textbf{Aufgabe 5} &   &   &   &   &   & \strut\\
    \hline
    \textbf{Aufgabe 6} &   &   &   &   &   & \strut\\
    \hline
    \textbf{Aufgabe 7} &   &   &   &   &   & \strut\\
    \hline
    \textbf{Aufgabe 8} &   &   &   &   &   & \strut\\
    \hline
    \textbf{Aufgabe 9} &   &   &   &   &   & \strut\\
    \hline
    \textbf{Aufgabe 10} &   &   &   &   &   & \strut\\
    \hline
  \end{tabular}
\end{center}

\begin{itemize}
\item Es sind \rule[0ex]{2em}{.4pt}\, von 20 Punkten erreicht worden.
\end{itemize}
\end{graybox}

\vfill

\begin{graybox}{Rechen- und Textaufgaben}
  \begin{center}
    \large
    \begin{tabular}{|l|c|c|c|c|c|c|c|}
      \hline
      \textbf{Aufgabe} & \textbf{11} & \textbf{12} & \textbf{13} & \textbf{14} & \textbf{15} & \textbf{16} & \textbf{17} \strut\\
      \hline
      \textbf{Punkte} & 
      \hspace{1Ex}\Large\textcolor{gray!70}{12}\hspace{1Ex}  & 
      \hspace{1Ex}\Large\textcolor{gray!70}{10}\hspace{1Ex}  & 
      \hspace{1Ex}\Large\textcolor{gray!70}{8}\hspace{1Ex}  & 
      \hspace{1Ex}\Large\textcolor{gray!70}{9}\hspace{1Ex}  & 
      \hspace{1Ex}\Large\textcolor{gray!70}{9}\hspace{1Ex}  & 
      \hspace{1Ex}\Large\textcolor{gray!70}{12}\hspace{1Ex}  & 
      \hspace{1Ex}\Large\textcolor{gray!70}{12}\hspace{1Ex} \strut\\
      \hline
  \end{tabular}
\end{center}
\begin{itemize}
\item Es sind \rule[0ex]{2em}{.4pt}\, von 72 Punkten erreicht worden.
\end{itemize}
\end{graybox}

% -----------------------------------------------------------------------
\clearpage
% -----------------------------------------------------------------------
\begin{graybox}{Multiple Choice Aufgaben}
Die Multiple Choice Aufgaben \textcolor{red}{unterliegen dem Zufall}. Die Reihenfolge der Antworten ist zufällig. Die Fragen und Antworten sind semantisch zufällig und haben somit \textcolor{red}{verschiedene Textvarianten}. Insbesondere die reinen Textaufgaben haben verschiedene Textvarianten. Die Semeantik mag sich unterscheiden, die Inhalte sind aber gleich.
\end{graybox}
\section*{ANOVA}

\section{Aufgabe \hfill (2 Punkte)}





Nach der Berechnung einer einfaktoriellen ANOVA ergibt sich ein $\eta^2 = 0.31$. Welche Aussage ist richtig?



\begin{enumerate}
\item [\textbf{A} \msquare] Das $\eta^2$ beschreibt den Anteil der Varianz, der von den Behandlungsbedingungen nicht erkl{"a}rt wird. Somit der Rest an nicht erkl{"a}rbarer Varianz.
\item [\textbf{B} \msquare] Das $\eta^2$ ist ein Wert f{"u}r die G{"u}te der ANOVA. Je kleiner desto besser. Ein $\eta^2$ von 0 bedeutet ein perfektes Modell mit keiner Abweichung. Die Varianz ist null.
\item [\textbf{C} \msquare] Die Berechnung von $\eta^2$ ist ein Wert f{"u}r die Interaktion.
\item [\textbf{D} \msquare] Das $\eta^2$ wird genutzt um zu erfahren welchen Anteil der Varianz die Behandlungsbedingungen erklären.
\item [\textbf{E} \msquare] Das $\eta^2$ ist die Korrelation der ANOVA. Mit der Ausnahme, dass 0 der beste Wert ist.
\end{enumerate} 

\section{Aufgabe \hfill (2 Punkte)}



Sie führen ein Feldexperiment durch um das Gewicht von Erbsen zu
steigern. Die Pflanzen wachsen unter einer Kontrolle und zwei verschiedenen
Behandlungsbedingungen. Nach der Berechnung einer einfaktoriellen ANOVA
ergibt sich ein $\eta^2 = 0.3$. Welche Aussage ist richtig?



\begin{enumerate}
\item [\textbf{A} \msquare] Das $\eta^2$ beschreibt den Anteil der Varianz, der von den Umweltbedingungen erklärt wird. Daher werden 30\% der Varianz durch die Umweltbedingungen erklärt. Der Anteil der Varianz durch die Behandlungsgruppen ist dann 70\%.
\item [\textbf{B} \msquare] Es werden 70\% der Varianz durch die Behandlung erklärt. Das $\eta^2$ beschreibt den Anteil der Varianz, der von den unterschiedlichen Behandlungsbedingungen nicht erklärt wird.
\item [\textbf{C} \msquare] Es werden 30\% der Varianz durch den Versuch erklärt. Das $\eta^2$ beschreibt den Anteil der Varianz, der durch Fehler in der Versuchsdurchführung entsteht.
\item [\textbf{D} \msquare] Mit dem $\eta^2$ lässt sich auf die Qualität der Randomisierung und damit der Strukturgleichheit zwischen der Grundgesamtheit und der Stichprobe schließen. Es gilt dabei die Regel, dass ein $\eta^2$-Wert von 1 zu bevorzugen ist.
\item [\textbf{E} \msquare] Das $\eta^2$ beschreibt den Anteil der Varianz, der von den Behandlungsbedingungen erklärt wird. Daher werden 30\% der Varianz durch die Behandlungsgruppen erklärt.
\end{enumerate} 

\section{Aufgabe \hfill (2 Punkte)}



Die einfaktorielle ANOVA ist ein Standardverfahren in der agrawissenschaftlichen Forschung wenn es um den Vergleich von Behandlungsgruppen geht. Welche der folgenden Aussage zu der Berechnung der Teststatistik der einfaktoriellen ANOVA ist richtig?



\begin{enumerate}
\item [\textbf{A} \msquare] Die ANOVA berechnet die F-Statistik indem die MS des Fehlers durch die MS der Behandlung geteilt werden. Wenn die F-Statistik sich der 1 ann{"a}hert kann die Nullhypothese nicht abgelehnt werden.
\item [\textbf{B} \msquare] Die F-Statistik wird berechnet indem die MS der Behandlung durch die MS des Fehlers geteilt werden. Wenn die F-Statistik sich kaum von der Null unterscheidet kann die Nullhypothese nicht abgelehnt werden.
\item [\textbf{C} \msquare] Die ANOVA berechnt die F-Statistik aus den SS Behandlung geteilt durch die SS Fehler.
\item [\textbf{D} \msquare] Die ANOVA berechnet die T-Statistik indem den Mittelwertsunterschied der Gruppen simultan durch die Standardabweichung der Gruppen teilt. Wenn die T-Statistik h{"o}her als 1.96 ist, kann die Nullhypothese abgelehnt werden.
\item [\textbf{E} \msquare] Die ANOVA berechnet die T-Statistik aus der Multiplikation der MS Behandlung mit der MS der Fehler. Wenn die F-Statistik genau 0 ist, kann die Nullhypothese nicht abgelehnt werden.
\end{enumerate} 

\section{Aufgabe \hfill (2 Punkte)}




Die ANOVA ist ein statistisches Verfahren welches häufig in den Auswertungen von Experimenten in den Agrarwissenschaften angewendet
wird. Dabei wird die ANOVA als ein erstes statistischen Werkzeug für die
Übersicht über die Daten benutzt. Eine ANOVA testet dabei...



\begin{enumerate}
\item [\textbf{A} \msquare] ... den Unterschied zwischen zwei paarweisen Mittelwerten aus verschiedenen Behandlungsguppen. Wenn die signifikant ist, ist daher bekannt welcher Vergleich konkret unterschiedlich ist.
\item [\textbf{B} \msquare] ... den Unterschied zwischen der Varianz in den verschiedenen Behandlungsguppen und der Varianz in einer der Behandlungsgruppen. Wenn die ANOVA signifikant ist, muss über einen Posthoc-Test nachgedacht werden um den signifikanten Unterschied in einer der Gruppen exakt zu bestimmen.
\item [\textbf{C} \msquare] ... den Unterschied zwischen der Varianz über alle Behandlungsgruppen und der Varianz aus verschiedenen Behandlungsguppen. Wenn die ANOVA signifikant ist, muss ein Posthoc-Test angeschlossen werden.
\item [\textbf{D} \msquare] ... den Unterschied zwischen der Varianz über alle Behandlungsgruppen oder der Varianz aus verschiedenen Behandlungsguppen. Wenn die ANOVA signifikant ist, muss sich zwischen einem der beiden Varianzquellen entschieden werden.
\item [\textbf{E} \msquare] ... den Unterschied zwischen der F-Statistik anhand der Varianz der Gruppen. Wenn die F-Statistik exakt 0 ist, kann die Nullhypothese abgelehnt werden.
\end{enumerate} 

\section{Aufgabe \hfill (2 Punkte)}



In einer Studie zur Bewertung der Wirkung von Vitamin C auf das Zahnwachstum bei Schweine entstand folgende Abbildung. Der Versuch wurde an 46 Tieren durchgeführt, wobei jedes Tier eine von  drei Vitamin-C-Dosen (0.5, 1 und 1.5 mg/Tag) über eine von zwei Verabreichungsmethoden erhielt. Welche Aussage ist im Bezug auf eine zweifaktorielle ANOVA richtig?



{\centering \includegraphics[width=\maxwidth]{img/mc-anova-02-a-1} 

}







\begin{enumerate}
\item [\textbf{A} \msquare] Eine mittlere bis starke Interaktion liegt vor $(p \leq 0.05)$
\item [\textbf{B} \msquare] Keine Interaktion liegt vor $(p \leq 0.05)$.
\item [\textbf{C} \msquare] Eine negative Interaktion liegt vor $(\rho \geq 0.5)$.
\item [\textbf{D} \msquare] Eine Korrelation liegt vor $(p \leq 0.05)$.
\item [\textbf{E} \msquare] Die Koeffizienten sind negativ $(\beta_0 < 0; \beta_1 < 0)$.
\end{enumerate} 
\section*{Deskriptive Statistik \& Explorative Datenanalyse}

\section{Aufgabe \hfill (2 Punkte)}




Wie lautet der Mittelwert und Standardabweichung von $y$ mit 8, 7, 14, 5 und 9.



\begin{enumerate}
\item [\textbf{A} \msquare] Es berechnet sich 9.6 +/- 11.3
\item [\textbf{B} \msquare] Sie erhalten 8.6 +/- 1.68
\item [\textbf{C} \msquare] Es ergibt sich 8.6 +/- 3.36
\item [\textbf{D} \msquare] Es ergibt sich 7.6 +/- 5.65
\item [\textbf{E} \msquare] Es ergibt sich 9.6 +/- 1.68
\end{enumerate} 

\section{Aufgabe \hfill (2 Punkte)}




Gegeben ist $y$ mit 18, 19, 23, 32, 15, 26, 11, 19, 43, 29 und 63. Berechnen Sie den Median, das $1^{st}$ Quartile sowie das $3^{rd}$ Quartile.




\begin{enumerate}
\item [\textbf{A} \msquare] Es ergibt sich 27 +/- 18
\item [\textbf{B} \msquare] Es berechnet sich 24 [19; 31]
\item [\textbf{C} \msquare] Sie erhalten 23 [16; 30]
\item [\textbf{D} \msquare] Es ergibt sich 23 +/- 18
\item [\textbf{E} \msquare] Es ergibt sich 23 [18; 32]
\end{enumerate} 

\section{Aufgabe \hfill (2 Punkte)}



Die empfohlene Mindestanzahl an Beobachtungen für die Visualisierung mit einem Boxplot sind...



\begin{enumerate}
\item [\textbf{A} \msquare] Die untere Grenze liegt bei einer Beobachtung.
\item [\textbf{B} \msquare] 1 Beobachtung.
\item [\textbf{C} \msquare] 10 Beobachtungen.
\item [\textbf{D} \msquare] Boxplot
\item [\textbf{E} \msquare] Wir sollten eine Beobachtung mindestens pro Gruppe vorliegen haben.
\end{enumerate}

\section{Aufgabe \hfill (2 Punkte)}



Die Standardabweichung ist eine bedeutende deskriptive Statistik für die Analyse von Daten. Wie müssen Sie vorgehen um die Standardabweichung zu berechnen?



\begin{enumerate}
\item [\textbf{A} \msquare] Den Mittelwert berechen, dann die absoluten Abstände zum Mittelwert aufsummieren
\item [\textbf{B} \msquare] Wir berechnen erst den Mittelwert und dann die quadratischen Abstände zu dem Mittelwert. Diese quadratischen Abstände summieren wir auf und teilen am Ende durch die Fallzahl. Als letzten Schritt ziehen wir die quadratische Wurzel.
\item [\textbf{C} \msquare] Den Mittelwert berechen, dann die quadratischen Abstände zum Mittelwert aufsummieren und durch die Fallzahl teilen.
\item [\textbf{D} \msquare] Den Median berechen, dann die quadratischen Abstände zum Median aufsummieren, dann die Wurzel ziehen.
\item [\textbf{E} \msquare] Wir berechnen erst den Mittelwert und dann die absoluten Abstände zu dem Mittelwert. Diese quadratischen Abstände summieren wir auf und teilen am Ende durch die Fallzahl.
\end{enumerate} 

\section{Aufgabe \hfill (2 Punkte)}



In Ihrer Abschlußarbeit wolllen Sie Ihre Daten für den Ertrag in einem Barplot darstellen. Sie nutzen den Barplot auch, da der Barplot zu den meist genutzten Visualiserungen von Daten gehört. Welche statistischen Maßzahlen stellt der Barplot dar?

 



\begin{enumerate}
\item [\textbf{A} \msquare] Durch die Abbildung des Barplot erhalten wir die Informationen über die Mittelwerte und die Standardabweichung.
\item [\textbf{B} \msquare] Den Median und die Standardabweichung.
\item [\textbf{C} \msquare] Der Barplot stellt den Median und die Quartile dar.
\item [\textbf{D} \msquare] Durch die Abbildung des Barplot erhalten wir die Informationen über den Median und die Standardabweichung.
\item [\textbf{E} \msquare] Den Mittelwert und die Varianz.
\end{enumerate}

\section{Aufgabe \hfill (2 Punkte)}



Der Mittelwert $\bar{y}$ und der Median $\tilde{y}$ unterscheiden sich nicht in Ihren Feldexperiment zu Leistungssteigerung von Erdbeeren.  Welche Aussage ist richtig?



\begin{enumerate}
\item [\textbf{A} \msquare] Der Mittelwert und der Median sollten sich unterscheiden sein, wenn Outlier in den Daten vorliegen. 
\item [\textbf{B} \msquare] Der Mittelwert und der Median sollten gleich sein, wenn Outlier in den Daten vorliegen. 
\item [\textbf{C} \msquare] Da sich der Mittelwert und der Median nicht unterscheiden, liegen vermutlich Outlier in den Daten vor. Wir untersuchen den Datensatz nach auffälligen Beobachtungen.
\item [\textbf{D} \msquare] Da sich der Mittelwert und der Median unterscheiden, liegen vermutlich keine Outlier in den Daten vor. Wir verweden den Datensatz so wie er ist.
\item [\textbf{E} \msquare] Da sich der Mittelwert und der Median unterscheiden, liegen vermutlich Outlier in den Daten vor. Wir untersuchen den Datensatz nach auffälligen Beobachtungen.
\end{enumerate}

\section{Aufgabe \hfill (2 Punkte)}



Um zu Überprüfen, ob die Daten die Annahme einer Normalverteilung genügen, können wir folgende Visualisierung nutzen. Dabei kommt dann auch die entsprechende Regel zur Abschätzung der Annahme einer Normalverteilung zur Anwendung.



\begin{enumerate}
\item [\textbf{A} \msquare] In einer explorativen Datanalyse nutzen wir den Violinplot. Dabei sollte der Bauch am Rand liegen. Dann können wir von einer Normalverteilung ausgehen.
\item [\textbf{B} \msquare] Einen Dotplot. Die Punkte müssen sich wie an einer Perlenschnurr audreihen. Eine Abweichung führt zur Ablehnung der Annahme einer Normalverteilung.
\item [\textbf{C} \msquare] Einen Barplot. Die Mittelwerte müssen alle auf einer Höhe liegen. Die Fehlerbalken haben hier keine Informationen.
\item [\textbf{D} \msquare] Einen Boxplot. Das IQR muss über alle Behandlungen zusammen mit den Whiskers ungefähr gleich aussehen.
\item [\textbf{E} \msquare] Nach der Erstellung eines Boxplots schauen wir, ob der Median in der Mitte der Box liegt. Dabei ist der Median als dicke Linie dargestellt und die Box ist das IQR.
\end{enumerate}

\section{Aufgabe \hfill (2 Punkte)}




In der Statistik müssen wir häufig überprüfen, ob unser Outcome einer bestimmten Verteilung folgt. Meistens überprüfen wir, ob eine
Normalverteilung vorliegt. Folgende drei Abbildungen eigenen sich im Besonderen für die Überprüfung einer Verteilungsannahme an eine Variable.





\begin{enumerate}
\item [\textbf{A} \msquare] Densityplot, Boxplot, Violinplot
\item [\textbf{B} \msquare] Scatterplot, Mosaicplot, Boxplot
\item [\textbf{C} \msquare] Scatterplot, Densityplot, Barplot
\item [\textbf{D} \msquare] Barplot, Mosaicplot, Violinplot
\item [\textbf{E} \msquare] Histogramm, Scatterplot, Boxplot
\end{enumerate} 

\section{Aufgabe \hfill (2 Punkte)}



Bevor Sie in Ihrer Abschlussarbeit einen statistischen Test rechnen, wollen Sie einmal betrachten, welcher Verteilung Ihre $n = 179$ geernteten Pflanzen folgen.  Welche Verteilung ist abgebildet?



{\centering \includegraphics[width=\maxwidth]{img/mc-distribution-02-a-1} 

}







\begin{enumerate}
\item [\textbf{A} \msquare] Eine multivariate Normalverteilung.
\item [\textbf{B} \msquare] Wir haben eine Gammaverteilung vorliegen.
\item [\textbf{C} \msquare] Es handelt sich um eine Binomial-Verteilung.
\item [\textbf{D} \msquare] Wir haben eine Poisson-Verteilung vorliegen.
\item [\textbf{E} \msquare] Es handelt sich um eine Normalverteilung.
\end{enumerate} 
\section*{Lineare Regression \& Korrelation}

\section{Aufgabe \hfill (2 Punkte)}



In Ihrer Abschlussarbeit wollen Sie ein kausales Modell rechnen. Jetzt stellt sich die Frage, was diese Entscheidung für Ihre Auswertung bedeutet. Welche Aussage ist richtig?



\begin{enumerate}
\item [\textbf{A} \msquare] Wir modellieren den Zusammenhang zwischen $X$ und $Y$ wenn ein kausales Modell rerechnet wird. Dabei kann nicht der gesamte Datensatz genutzt werden. Es wird ein Trainingsdatensatz zum Trainieren des Modells benötigt.
\item [\textbf{B} \msquare] Es wird ein Trainingsdatensatz zum Modellieren des Trainingsmodells benötigt. Der Testdatensatz dient rein zur Visualisierung. Dies gilt vor allem für ein kausales Modell.
\item [\textbf{C} \msquare] Ein kausales Modell basiert auf einem Traingsdatensatz und einem Testdatensatz. Auf dem Trainingsdatensatz wird das Modell trainiert und auf dem Testdatensatz validiert.
\item [\textbf{D} \msquare] Wir modellieren den Zusammenhang zwischen $X$ und $Y$ wenn ein kausales Modell rerechnet wird. Dabei kann der gesamte Datensatz genutzt werden. Eine Aufteilung wie in einem prädiktiven Modell ist nicht notwendig.
\item [\textbf{E} \msquare] Ein kausales Modell benötigt mindestens eine Fallzahl von über 100 Beobachtungen und darf keine fehlenden Werte beinhalten. Die Varianzkomponenten müssen homogen sein.
\end{enumerate}

\section{Aufgabe \hfill (2 Punkte)}



Nach der Modellierung einer Regression stellt sich die Frage, ob die Residuen approximativ einer Normalverteilung folgen. Sie können einen QQ-Plot für die visuelle Überprüfung der Annahme an die Residuen nutzen. Welche Aussage ist richtig?



{\centering \includegraphics[width=\maxwidth]{img/mc-regression-05-a-1} 

}







\begin{enumerate}
\item [\textbf{A} \msquare] Die Annahme der normalverteilten Residuen ist nicht erfüllt. Die Punkte liegen zum überwiegenden Teil nicht auf der Geraden.
\item [\textbf{B} \msquare] Wir betrachten die Gerade. Wenn die Punkte einigermaßen gleichmäßig um die Gerade verteilt liegen, dann gehen wir von normalverteilten Residuen aus. Dies ist hier nicht der Fall. Wir haben keine normalverteilten Residuen vorliegen.
\item [\textbf{C} \msquare] Wir betrachten die Punkte. Wenn die Punkte einigermaßen gleichmäßig verteilt liegen, dann gehen wir von normalen Residuen aus.
\item [\textbf{D} \msquare] Wir betrachten insbesondere die beiden Enden der Gerade. Der Rest ist mehr oder minder egal, dann ist die Annahme an die Normalverteilung der Residuen erfüllt.
\item [\textbf{E} \msquare] Die Annahme der normalverteilten Residuen ist erfüllt. Die Punkte liegen zum überwiegenden Teil auf der Geraden.
\end{enumerate}

\section{Aufgabe \hfill (2 Punkte)}



Nach einer Regressions sollten die Residuen (\texttt{.resid}) gleichmäßig um die Gerade verortet sein. Was bei einer simplen Regression noch relativ einfach visuell in einem Scatterplot zu überprüfen ist. Für komplexere Modell liefert der Residual Plot die notwendigen Informationen. Welche Aussage ist richtig?



{\centering \includegraphics[width=\maxwidth]{img/mc-regression-06-a-1} 

}







\begin{enumerate}
\item [\textbf{A} \msquare] Die Annahme der normalverteilten Residuen ist nicht erfüllt. Es ist kein Muster zu erkennen.
\item [\textbf{B} \msquare] Wenn die Punkte gleichmäßig in dem positiven wie auch negativen Bereich ohne ein klares Muster liegen, dann hat unsere Modellierung geklappt. Wir können mit dem Modell weitermachen.
\item [\textbf{C} \msquare] Die Annahme der normalverteilten Residuen ist erfüllt. Es ist ein Muster zu erkennen und wir können damit auf die Signifkanz von $x_1, ..., x_p$ schließen.
\item [\textbf{D} \msquare] Die Punkte müssen gleichmäßig, mit ähnlichen Abständen, in dem positiven wie auch negativen Bereich liegen. Dies ist hier klar nicht der Fall. Einzelne Ausreißer können beobachtet werden. Wir können mit dem Model so nicht rechnen und müssen erst die auffälligen Werte gesondert betrachten.
\item [\textbf{E} \msquare] Wenn wir die Nulllinie betrachten so müssen die Punkte gleichmäßig unter der Nulllinie liegen. Unser Modell erfüllt somit nicht die Annahme von normalverteilten Residuen mit einem Mittelwert von $>0$ und einer Streuung von $s$.
\end{enumerate}

\section{Aufgabe \hfill (2 Punkte)}




Welche Aussage über den Korrelationskoeffizienten $\rho$ ist richtig?




\begin{enumerate}
\item [\textbf{A} \msquare] Der Korrelationskoeffizienten $\rho$ liegt zwischen -1 und 1. Darüber hinaus ist der Korrelationskoeffizienten $\rho$ als standardisierte Steigung zu verstehen, wenn eine Standardisierung durchgeführt wurde. Diese Adjustierung nach Fischer muss am Anschluß der Berechnung der Korrelation durchgeführt werden.
\item [\textbf{B} \msquare] Der Korrelationskoeffizienten $\rho$ liegt zwischen -1 und 1. Darüber hinaus ist der Korrelationskoeffizienten $\rho$ einheitslos und kann als standardisierte Steigung verstanden werden.
\item [\textbf{C} \msquare] Der Korrelationskoeffizienten $\rho$ ist eine standardisierte, statistische Maßzahl, die zwischen 0 und 1 liegt. Dabei ist Korrelationskoeffizienten $\rho$ einheitslos. Eine Signifikanz kann nicht nachgewiesen werden.
\item [\textbf{D} \msquare] Der Korrelationskoeffizienten $\rho$ ist eine veraltete Darstellungsform von Effekten in der linearen Regression und wird wie das $\eta^2$ aus der ANOVA interpretiert. Der Korrelationskoeffizienten $\rho$ beschreibt den Anteil an erklärter Varianz durch die Regression.
\item [\textbf{E} \msquare] Der Korrelationskoeffizienten $\rho$ wird wie das $\eta^2$ aus der ANOVA interpretiert. Der Korrelationskoeffizienten $\rho$ beschreibt den Anteil an erklärter Varianz durch die Regression. Dabei gibt er jedoch eine Richtung an und kann auch negativ werden.
\end{enumerate}

\section{Aufgabe \hfill (2 Punkte)}



Sie haben ein Feldexperiment mit Spitzkohl durchgeführt und wollen nun in einer simplen linearen Regression den Einfluss der $CO_2$-Konzentration in [$\mu g$] im Wasser auf das Wachstum in [$kg$] untersuchen. Sie erhalten einen $\beta_{CO_2}$ Koeffizienten von $2.3\times 10^{-9}$ und einen $p$-Wert mit $0.00032$. Welche Aussage zu der Signifikanz und dem Effekt ist richtig?




\begin{enumerate}
\item [\textbf{A} \msquare] Die Einheit der $CO_2$-Konzentration ist zu klein gewählt. Dadurch sehen wir den sehr kleinen $p$-Wert. Der $p$-Wert und die Einheit von der $CO_2$-Konzentration hängen antiproportional zusammen.
\item [\textbf{B} \msquare] Die Fallzahl ist zu hoch angesetzt. Je höher die Fallzahl ist, desto kleiner ist die Teststatistik und damit ist dann auch der $p$-Wert sehr klein. Es sollte über eine Reduzierung der Fallzahl nachgedacht werden. Dann sollte der Effekt zum p-Wert passen.
\item [\textbf{C} \msquare] Das Gewicht und die $CO_2$-Konzentration korrelieren sehr stark, deshalb wird der $\beta_{CO_2}$ Koeffizient sehr klein. Mit einer ANOVA kann für die Korrelation korrigiert werden und der Effektschätzer passt dann zum p-Wert.
\item [\textbf{D} \msquare] Manchmal ist die Einheit der Einflussvariable $X$ zu groß gewählt, so dass der Ansteig von 1 Einheit in $X$ zu einer zu großen Änderung in $y$ führt. Daher kann der Effekt $\beta_{CO_2}$ sehr klein wirken, da der p-Wert wird auf einer einheitslosen Teststatistik bestimmt wird.
\item [\textbf{E} \msquare] Wenn der Effekt $\beta_{CO_2}$ sehr klein ist, dann kann es an einer falsch gewählten Einheit liegen. Der Anstieg von einer Einheit in $X$ führt ja zu einer Änderung von $\beta_{CO_2}$ in $y$. Daher ist hier mit einer anderen Einheit in den Daten zu rechnen, so dass wir hier einen besser formatierten Effekt sehen. Der p-Wert stammt aus einer einheitslosen Teststatistik.
\end{enumerate}

\section{Aufgabe \hfill (2 Punkte)}



Neben der klassischen Regression kann die Funktion \texttt{lm()} in \Rlogo auch für welche andere Art von Anwendung genutzt werden?





\begin{enumerate}
\item [\textbf{A} \msquare] Neben der klassichen Verwendung der Funktion \texttt{lm()} in der linearen Regression kann auch ein Gruppenvergleich gerechnet werden. Dafür müssen aber alle Faktoren aus den Daten entfernt und numerishc umgewandelt werden. Dann kann das R Paket \{emmeans\} genutzt werden um die Korrelation zu berechnen. Eine Adjustierung ist dann nicht mehr notwendig.
\item [\textbf{B} \msquare] Die Funktion \texttt{lm()} in \Rlogo wird klassischerweise für die nicht-lineare Regression genutzt. Ist die Einflussvariable $X$ numerisch so werden die Gruppenmittelwerte geschätzt.
\item [\textbf{C} \msquare] Ist die Einflussvariable $X$ ein Faktor so werden die Gruppenmittelwerte geschätzt und eine anschließende ANOVA sowie multipler Gruppenvergleich mit \{emmeans\} ist möglich. Die Funktion \texttt{lm()} kann dabei eigentlich weggelassen werden, wird aber traditionell gerechnet.
\item [\textbf{D} \msquare] Die Funktion \texttt{lm()} in \Rlogo ist der letzte Schritt für einen Gruppenvergleich. Vorher kann eine ANOVA oder aber ein multipler Vergleich in \{emmeans\} gerechnet werden. In der Funktion  \texttt{lm()} werden die Gruppenvarianzen bestimmt.
\item [\textbf{E} \msquare] Ist die Einflussvariable $X$ ein Faktor so werden die Gruppenmittelwerte geschätzt und eine anschließende ANOVA sowie multipler Gruppenvergleich mit \{emmeans\} ist möglich. Dennoch muss zuerst ein lineares Modell mit der Funktion \texttt{lm()} in \Rlogo gerechnet werden.
\end{enumerate}

\section{Aufgabe \hfill (2 Punkte)}



Welche Aussage über das \textit{generalisierte lineare Modell (GLM)} ist richtig?




\begin{enumerate}
\item [\textbf{A} \msquare] Dank dem \textit{generalisierten linearen Modell (GLM)} können auch andere Verteilungsfamilien als die Normalverteilung mit einer linearen Regression modelliert werden.
\item [\textbf{B} \msquare] Dank dem \textit{generalisierten linearen Modell (GLM)} können auch andere Verteilungsfamilien -- außer die Normalverteilung -- mit einer linearen Regression modelliert werden. Dafür werden alle Verteilungen in eine Normalverteilung überführt und anschließend standardisiert.
\item [\textbf{C} \msquare] Das GLM ist eine allgemeine Erweiterung der linearen Regression auf die Normalverteilung.
\item [\textbf{D} \msquare] In \Rlogo ist mit dem \textit{generalisierten linearen Modell (GLM)} eine Modellierung implementiert, die die Poissonverteilung für Zähldaten oder die Binomialverteilung für 0/1-Daten modellieren kann. Weitere Modellierungen sind in \Rlogo auch mit zusätzlich geladenen Paketen nicht möglich.
\item [\textbf{E} \msquare] Das GLM ist ein faktisch maschineller Lernalgorithmus, der selstständig die Verteilungsfamilie für Y wählt.
\end{enumerate}
\section*{Vermischte Themen}  

\section{Aufgabe \hfill (2 Punkte)}

Die Randomisierung von Beobachtungen zu den Versuchseinheiten
ist bedeutend in der Versuchsplanung. Welche der folgenden Aussagen ist richtig?



\begin{enumerate}
\item [\textbf{A} \msquare] Randomisierung erlaubt erst die Varianzen zu schätzen. Ohne eine Randomisierung ist die Berechnung von Mittelwerten und Varianzen nicht möglich. Dadurch lässt sich erst ein Experiment auswerten.
\item [\textbf{B} \msquare] Strukturgleichheit ist durch Randomisierung gegeben. Leider hilft die Randomisierung noch nicht um von der Stichprobe auf die Grundgesamtheit zu schließen. Deshalb wurde das Falsifikationsprinzip entwickelt.
\item [\textbf{C} \msquare] Randomisierung bringt starke Unstrukturiertheit in das Experiment und erlaubt erst von der Stichprobe auf die Grundgesamtheit zurückzuschliessen.
\item [\textbf{D} \msquare] Durch eine Randomisierung können wir nicht von Strukturgleichheit zwischen der Stichprobe und der Grundgesamtheit ausgehen.
\item [\textbf{E} \msquare] Durch eine Randomisierung können wir von Strukturgleichheit zwischen der Stichprobe und der Grundgesamtheit ausgehen.
\end{enumerate}

\section{Aufgabe \hfill (2 Punkte)}



Wenn Sie einen Datensatz erstellen, dann ist es ratsam die Spalten und die Einträge in englischer Sprache zu verfassen, wenn Sie später die Daten in \Rlogo auswerten wollen. Welcher Aussage ist richtig?



\begin{enumerate}
\item [\textbf{A} \msquare] Es gibt keinen Grund nicht auch deutsche Wörter zu verwenden. Es ist ein Stilmittel.
\item [\textbf{B} \msquare] Programmiersprachen haben Probleme mit Umlauten und Sonderzeichen der deutschen Sprache. Daher ist die Nutzung in Deutsch in den AGBs von \Rlogo untersagt.
\item [\textbf{C} \msquare] Programmiersprachen haben Probleme mit Umlauten und Sonderzeichen der deutschen Sprache. Die Nutzung von englischer Sprache umgeht dieses Problem in eleganter Art.
\item [\textbf{D} \msquare] Alle Funktionen und auch Anwendungen sind in \Rlogo in englischer Sprache. Die Nutzung von deutschen Wörtern ist nicht schick und das ist zu vermeiden.
\item [\textbf{E} \msquare] Die Spracherkennung von \Rlogo ist nicht in der Lage Deutsch zu verstehen.
\end{enumerate}

\section{Aufgabe \hfill (2 Punkte)}



In Ihrer Abschlussarbeit wollen Sie zu Beginn eine explorativen Datenanalyse (EDA) in \Rlogo rechnen. Dafür gibt es eine generelle Abfolge von Prozessschritten. Welche ist hierbei die richtige Reihenfolge?



\begin{enumerate}
\item [\textbf{A} \msquare] Für eine explorativen Datenanalyse (EDA) in \Rlogo müssen wir als erstes die Daten über \texttt{read\_excel()} einlesen. Danach müssen wir schauen, dass wir die Spalten richtig über \texttt{mutate()} transformiert haben. Insbesondere müssen Variablen mit Kategorien in einen Faktor umgewandelt werden. Am Ende nutzen wir die Funktion \text{ggplot()} für die eigentlich EDA.
\item [\textbf{B} \msquare] Wir lesen die Daten über eine generische Funktion \texttt{read()} ein und müssen dann die Funktion \texttt{ggplot()} nur noch installieren. Dann haben wir die Abbildungen als \texttt{*.png} vorliegen.
\item [\textbf{C} \msquare] Wir lesen als erstes die Daten über \texttt{read\_excel()} ein, transformieren die Spalten über \texttt{mutate()} in die richtige Form und können dann  über \text{ggplot()} uns die Abbildungen erstellen lassen. Wichtig ist, dass wir keine Faktoren sondern nur numerische Variablen vorliegen haben.
\item [\textbf{D} \msquare] Für eine explorativen Datenanalyse (EDA) in \Rlogo müssen wir als erstes die Daten über \texttt{read\_excel()} einlesen. Danach müssen wir schauen, dass wir die Zeilen richtig über \texttt{mutate()} transformiert haben. Insbesondere müssen Variablen mit kontinuierlichen Werten in einen Faktor umgewandelt werden. Am Ende nutzen wir die Funktion \text{ggplot()} für die eigentlich EDA.
\item [\textbf{E} \msquare] Wir transformieren die Spalten über \texttt{mutate()} in ein \texttt{tibble} und können dann über \text{ggplot()} uns die Abbildungen erstellen lassen. Dabei beachten wir das wir keine Faktoren in den Daten haben.
\end{enumerate}

\section{Aufgabe \hfill (2 Punkte)}



Es sei $s^2_1 = s^2_2$ in dem Modell $Y \sim X$. Welche Aussage ist richtig?



\begin{enumerate}
\item [\textbf{A} \msquare] Es handelt sich um ein balanciertes Design.
\item [\textbf{B} \msquare] Es handelt sich um ein unbalanciertes Design.
\item [\textbf{C} \msquare] Es liegt Varianzhetrogenität vor.
\item [\textbf{D} \msquare] Es handelt sich um unabhängige Beobachtungen.
\item [\textbf{E} \msquare] Es liegt Varianzhomogenität vor.
\end{enumerate}

\section{Aufgabe \hfill (2 Punkte)}



In einem Zuchtexperiment messen wir die Ferkel verschiedener Sauen. Die Ferkel einer Muttersau sind daher im statistischen Sinne...



\begin{enumerate}
\item [\textbf{A} \msquare] Untereinander unabhängig. Sollten die Mütter verwandt sein, so ist die Varianzstruktur ähnlich und muss modelliert werden.
\item [\textbf{B} \msquare] Untereinander abhängig. Die Ferkel stammen von einem Muttertier und haben vermutliche eine ähnliche Varianzstruktur.
\item [\textbf{C} \msquare] Abhängig von der Stallanlage und des Experiments können die Ferkel abhängig oder unabhängig sein. Allgmein gilt, dass Ferkel von unterschiedlichen Sauen näher miteinander verwandt sind als Ferkel von gleichen Sauen. Das Fisher-Axiom.
\item [\textbf{D} \msquare] Untereinander stark korreliert. Die Ferkel sind von einer Mutter und sommit miteinander korreliert. Dies wird in der Statistik jedoch meist nicht modelliert.
\item [\textbf{E} \msquare] Die Ferkel stammen von der gleichen Sau und sind somit untereinander unabhängig.
\end{enumerate}

\section{Aufgabe \hfill (2 Punkte)}



Sie führen ein Experiment zur Behandlung von Klaueninfektionen bei Schafe durch. Bei 3 Tieren finden Sie eine Erkrankung der Klauen vor und 12 Tiere sind gesund. Welche Aussage über den Effektschätzer Risk ratio ist richtig?



\begin{enumerate}
\item [\textbf{A} \msquare] Es ergibt sich ein Risk ratio von 4, da es sich um ein Anteil handelt.
\item [\textbf{B} \msquare] Der Anteil der Kranken wird berechnet. Da es sich um ein Anteil handelt ergibt sich ein Risk ratio von 0.2.
\item [\textbf{C} \msquare] Es ergibt sich ein Risk ratio von 0.25, da es sich um ein Anteil handelt.
\item [\textbf{D} \msquare] Es ergibt sich ein Risk ratio von 0.2, da es sich um eine Chancenverhältnis handelt.
\item [\textbf{E} \msquare] Da es sich um ein Chancenverhältnis handelt ergibt sich ein Risk ratio von 0.25.
\end{enumerate}

\section{Aufgabe \hfill (2 Punkte)}



In der Bio Data Science wird häufig mit sehr großen Datensätzen gerechnet. Historisch ergibt sich nun ein Problem bei der Auswertung der Daten und deren Bewertung hinsichtlich der Signifikanz. Welche Aussage ist richtig?





\begin{enumerate}
\item [\textbf{A} \msquare] Big Data ist ein Problem der parametrischen Statistik. Parameter lassen sich nur auf kleinen Datensätzen berechnen, da es sich sonst nicht mehr um eine Stichprobe im engen Sinne der Statistik handelt.
\item [\textbf{B} \msquare] Aktuell werden immer größere Datensätze erhoben. Dadurch wird auch die Varianz immer höher was automatisch zu mehr signifikanten Ergebnissen führt.
\item [\textbf{C} \msquare] Aktuell werden zu grosse Datensätze für die gänigige Statistik gemessen. Daher wendet man maschinelle Lernverfahren für kausale Modelle an. Hier ist die Relevanz gleich Signifikanz.
\item [\textbf{D} \msquare] Mehr Fallzahl in Datensätzen bedeutet mehr signifikante Ergebnisse, da in mehr Daten auch mehr Informationen beinhaltet sind. Deshalb lohnen sich riesige Datensätze, die durch die vielen signifikanten Ergebnisse auch eine Menge an relevanten Erkenntnissen liefern.
\item [\textbf{E} \msquare] Eine erhöhte Fallzahl führt automatisch zu mehr signifikanten Ergebnissen auch wenn der Effekt klein ist und damit nicht relevant. Dadurch sind die Informationen zur Signifikanz in riesigen Datensätzen schwer zu verwerten, da fast alle Vergleiche signifikant sind.
\end{enumerate}
\section*{Multiple Gruppenvergleiche}    

\section{Aufgabe \hfill (2 Punkte)}



Sie haben folgende unadjustierten p-Werte gegeben: 0.34, 0.03, 0.001, 0.89, 0.21 und 0.02. Sie adjustieren die p-Werte nach
Bonferroni. Welche Aussage ist richtig?



\begin{enumerate}
\item [\textbf{A} \msquare] Nach der Bonferroni-Adjustierung ergeben sich die adjustierten p-Werte von 2.04, 0.18, 0.006, 5.34, 1.26 und 0.12. Die adjustierten p-Werte werden zu einem $\alpha$-Niveau von 5\% verglichen.
\item [\textbf{B} \msquare] Nach der Bonferroni-Adjustierung ergeben sich die adjustierten p-Werte von 0.0567, 0.005, 2e-04, 0.1483, 0.035 und 0.0033. Die adjustierten p-Werte werden zu einem $\alpha$-Niveau von 5\% verglichen.
\item [\textbf{C} \msquare] Nach der Bonferroni-Adjustierung ergeben sich die adjustierten p-Werte von 0.0567, 0.005, 2e-04, 0.1483, 0.035 und 0.0033. Die adjustierten p-Werte werden zu einem $\alpha$-Niveau von 0.83\% verglichen.
\item [\textbf{D} \msquare] Nach der Bonferroni-Adjustierung ergeben sich die adjustierten p-Werte von 1, 0.18, 0.006, 1, 1 und 0.12. Die adjustierten p-Werte werden zu einem $\alpha$-Niveau von 5\% verglichen.
\item [\textbf{E} \msquare] Nach der Bonferroni-Adjustierung ergeben sich die adjustierten p-Werte von 1, 0.18, 0.006, 1, 1 und 0.12. Die adjustierten p-Werte werden zu einem $\alpha$-Niveau von 0.83\% verglichen.
\end{enumerate}

\section{Aufgabe \hfill (2 Punkte)}



Auf wissenschaftlichen Postern finden Sie unter Abbildungen häufig die Abbkürzung \textit{CLD}. Für welchen statistischen Fachbegriff steht die Abbkürzung und wie interpretieren Sie ein \textit{CLD}?



\begin{enumerate}
\item [\textbf{A} \msquare] Contrast letter display. Unterschiede in den Behandlungen werden durch den gleichen Buchstaben oder Symbol dargestellt. Die Interpretation des CLD führt häufig in die Irre.
\item [\textbf{B} \msquare] Compound letter display. Gleichheit in dem Outcomes wird durch den gleichen Buchstaben oder Symbol dargestellt. Teilweise ist die Interpretation des Verbunds (eng. compound) herausfordernd, da wir ja nach dem Unterschied suchen.
\item [\textbf{C} \msquare] Compact letter display. Teilweise ist die Interpretation des CLD schwierig, da wir ja nach Unterschieden suchen aber nur Gleichheit in den Buchstaben sehen. Die Gleichheit der Behandlungen wird durch gleiche Buchstaben dargestellt.
\item [\textbf{D} \msquare] Compact letter display. Gleiche Buchstaben zeigen Gleichheit in den Behandlungen. Die Interpretation ist deshalb sehr intuitiv und einfach. Darüber hinaus ist damit das CLD auch auf einer Linie mit der Testtheorie, da wir ja auch dort die Gültigkeit der Nullhypothese nachweisen. Wir suchen ja Gleichheit.
\item [\textbf{E} \msquare] Compact line display. Gleichheit in den Behandlungen wird durch den gleichen Buchstaben oder Symbol dargestellt. Früher wurden keine Buchstaben sondern eine durchgezogene Linie verwendet. Bei mehr als drei Gruppen funktioniert die Linie aber graphisch nicht mehr.
\end{enumerate}

\section{Aufgabe \hfill (2 Punkte)}




In Ihrer Bachelorarbeit müssen Sie einen Feldversuch auswerten. Nachdem Sie die zweifaktorielle ANOVA gerechnet haben und keine signifikante Interaktion vorliegt, wollen Sie jetzt einen Posthoc-Test rechnen. Welches R Paket nutzen Sie dafür am besten?



\begin{enumerate}
\item [\textbf{A} \msquare] Das R Paket \{lm\}. Das Paket \{lm\} erstellt selbstständig Konfidenzintervalle und entsprechende p-Werte. Da wir in dem Paket nicht adjustieren müssen, ist es bei Anwendern sehr beliebt.
\item [\textbf{B} \msquare] Das R Paket \{hmisc\} erlaubt die Durchführung eines multiplen Gruppenvergleichs aus verschiedenen Modellen heraus. Aus einem hmisc Objekt lässt sich recht einfach das CLD erstellen und so über Barplots eine schnelle Interpration der statistischen Auswertung durchführen.
\item [\textbf{C} \msquare] Sie nutzen das R Paket \{emmeans\} für die Berechnung des multipnen Gruppenvergleichs. Die Ausgabe der Funktion emmeans() erlaubt zügig über \{ggplot\} einen Barplot zu erstellen und dann auch das CLD zu berechnen. Sie haben alles sofort zusammen.
\item [\textbf{D} \msquare] Das R Paket \{emmeans\} erlaubt die Durchführung eines multiplen Gruppenvergleichs. Aus einem emmeans Objekt lässt sich leider kein CLD erstellen. Dennoch ist das Paket einfach zu bedienen und wird deshalb genutzt. Die Interpretation der statistischen Auswertung wird über einen Barplot abgebildet.
\item [\textbf{E} \msquare] Das R Paket \{ggplot\}. Wir erhalten hier sofort eine Visualisierung der Daten. Anhand der Visualisierung lässt sich eine explorative Datenanalyse durchführen, die gleichwertig zu einem Posthoc-Test ist.
\end{enumerate}

\section{Aufgabe \hfill (2 Punkte)}



Bei einem multiplen Vergleich oder Posthoc Test kann es zu einer Besonderheit beim statistischen Testen kommen. Wie nennt man diese Besonderheit beim statistischen Testen und wie kann man mit ihr umgehen?



\begin{enumerate}
\item [\textbf{A} \msquare] Die Adjustierung der p-Werte nach Bonferroni erlaubt es gegen die $\alpha$-Inflation vorzugehen, die häufig beim multiplen Testen auftritt. Das globale Signifikanzniveau liegt nicht mehr bei $5\%$ sondern sehr viel höher. Das ist der Grund warum die p-Werte entsprechend adjustiert werden müssen.
\item [\textbf{B} \msquare] Beim multiplen Testen kann es zu einer $\alpha$-Inflation kommen. Das globale Signifikanzniveau liegt nicht mehr bei $5\%$ sondern weit darunter. Daher müssen die p-Werte entsprechend adjustiert werden. Hierfür gibt es verschiedene Verfahren, wobei das Verfahren zur Adjustierung der p-Werte nach Welch das bekanneste Verfahren ist.
\item [\textbf{C} \msquare] Beim multiplen Testen kann es zu Varianzheterogenität kommen. Das globale Signifikanzniveau liegt nicht mehr bei $5\%$. Daher müssen die p-Werte entsprechend adjustiert werden. Das Verfahren nach Welch, bekannt aus dem t-Test, ist hier häufig anzuwenden.
\item [\textbf{D} \msquare] Beim multiplen Testen kann es zu einer $\alpha$-Deflation kommen. Das globale Signifikanzniveau liegt nicht mehr bei $5\%$ sondern weit darunter. Daher müssen die p-Werte entsprechend adjustiert werden. Hierfür gibt es verschiedene Verfahren, wobei das Verfahren zur Adjustierung der p-Werte nach Bonferroni das bekanneste Verfahren ist. Die p-Werte werden durch die Anzahl an Vergleichen geteilt
\item [\textbf{E} \msquare] Das globale Signifikanzniveau liegt nicht mehr bei $5\%$ sondern sehr viel niedriger, bei ca. $1\%$. Es kommt zu einer $\alpha$-Hyperinflation. Dagegen kann mit der Adjustierung der p-Werte nach Bonferroni vorgegangen werden.
\end{enumerate}

\section{Aufgabe \hfill (2 Punkte)}




Sie rechnen mehrere t-Tests für einen multiplen Vergleich nachdem eine einfaktorielle ANOVA sich als signifikant herausgestellt hat. Welche Aussage im Bezug auf den Effekt ist richtig? 



\begin{enumerate}
\item [\textbf{A} \msquare] Beim multiplen Testen kann es zu einer $\Delta$-Deflation kommen. Das globale Relevanzniveau liegt nicht mehr bei $5\%$ sondern weit darunter. Daher müssen die $\Delta$-Werte entsprechend adjustiert werden. Hierfür gibt es verschiedene Verfahren, wobei das Verfahren zur Adjustierung der $\Delta$-Werte nach Bonferroni das bekanneste Verfahren ist. Die $\Delta$-Werte werden durch die Anzahl an Vergleichen geteilt.
\item [\textbf{B} \msquare] Wenn ein multipler Test gerechnet wird, dann muss der Effekt $\Delta$ nicht adjustiert werden im Gegensatz zu den p-Werten.
\item [\textbf{C} \msquare] Beim multiplen Testen kann es zu einer $\Delta$-Inflation kommen. Das globale Effektniveau liegt nicht mehr bei $20\%$. Daher müssen die Effekte entsprechend adjustiert werden. Hierfür gibt es verschiedene Verfahren, wobei das Verfahren zur Adjustierung der Effekte nach Bonferroni das bekanneste Verfahren ist.
\item [\textbf{D} \msquare] Wenn ein multipler Test gerechnet wird, dann muss der Effekt $\Delta$ adjustiert werden im Gegensatz zu den p-Werten.
\item [\textbf{E} \msquare] Wenn ein multipler Test gerechnet wird, dann muss der Effekt $\Delta$ nach Bonferroni adjustiert werden. Dafür wird der Effekt mit der Anzahl an Vergleichen $k$ multipliziert. Dies geschiet analog zu den p-Werten.
\end{enumerate}
\section*{Statistische Testtheorie}  

\section{Aufgabe \hfill (2 Punkte)}




Welche Aussage zum mathematische Ausdruck $Pr(D|H_0)$ ist richtig?



\begin{enumerate}
\item [\textbf{A} \msquare] Die Inverse der Wahrscheinlichkeit unter der die Nullhypothese nicht mehr die Alternativehypothese überdeckt.
\item [\textbf{B} \msquare] $Pr(D|H_0)$ ist die Wahrscheinlichkeit der Alternativehypothese und somit $1 - Pr(H_A)$
\item [\textbf{C} \msquare] Die Wahrscheinlichkeit der Daten unter der Nullhypothese in der Grundgesamtheit.
\item [\textbf{D} \msquare] Die Wahrscheinlichkeit für die Nullhypothese, wenn die Daten wahr sind.
\item [\textbf{E} \msquare] $Pr(D|H_0)$ ist die Wahrscheinlichkeit die Daten $D$ zu beobachten, wenn die Nullhypothese wahr ist.
\end{enumerate}

\section{Aufgabe \hfill (2 Punkte)}



Das statistische Testen basiert auf dem Falsifikationsprinzip. Es besagt,



\begin{enumerate}
\item [\textbf{A} \msquare] ... dass Fehlerterme in statistischen Modellen nicht verifiziert werden können.
\item [\textbf{B} \msquare] ... dass ein minderwertes Modell durch ein minderwertiges Modell ersetzt wird. Es gilt das Verifikationsprinzip nach Karl Popper.
\item [\textbf{C} \msquare] ... dass ein schlechtes Modell durch ein schlechteres Modell ersetzt wird. Die Wissenschaft lehnt ab und verifiziert nicht.
\item [\textbf{D} \msquare] ... dass ein schlechtes Modell durch das Falsifikationsprinzip durch ein weniger schlechtes Modell ersetzt wird.
\item [\textbf{E} \msquare] ... dass in der Wissenschaft immer etwas falsch sein muss. Sonst gebe es keinen Fortschritt.
\end{enumerate}

\section{Aufgabe \hfill (2 Punkte)}



Das Signifikanzniveau $\alpha$ wird auch Fehler 1. Art genannt und liegt bei 5\%. Warum wurde der Grenzwert von 5\% als Signifikanzschwelle gewählt?



\begin{enumerate}
\item [\textbf{A} \msquare] Als Kulturkonstante hat $\alpha = 5\%$ den Rang einer Naturkonstante und wurde nach langer Diskussion in der UN im Jahre 1983 festgesetzt. Damals auch schon mit der Zustimmung der UdSSR.
\item [\textbf{B} \msquare] Die Festlegung von $\alpha = 5\%$ ist eine Kulturkonstante. Wissenschaftler benötigt eine Schwelle für eine statistische Testentscheidung, der Wert von $\alpha$ wurde aber historisch mehr zufällig gewählt.
\item [\textbf{C} \msquare] Der Begründer der modernen Statistik, R. Fischer, hat die Grenze simuliert und berechnet. Dadurch ergibt sich dieser optimale Cut-Off.
\item [\textbf{D} \msquare] Auf einer Statistikkonferenz in Genf im Jahre 1942 wurde dieser Cut-Off nach langen Diskussionen festgelegt. Bis heute ist der Cut Off aber umstritten, da wegen dem 2. Weltkrieg viele Wissenschaftler nicht teilnehmen konnten.
\item [\textbf{E} \msquare] Da Wissenschaftler eine Schwelle für die statistische Testentscheidung benötigen wurde $\alpha$ in einer großen Konferenz 1945 gewählt. Damit ist $\alpha = 5\%$ eine Kulturkonstante mit einem Rank einer Naturkonstante.
\end{enumerate}

\section{Aufgabe \hfill (2 Punkte)}

Betrachten wir die Teststatistik aus einem abstrakteren Blickwinkel. Beim
statistischen Testen wird das \textit{"`signal"'} mit dem
\textit{"`noise"'} aus den Daten $D$ zu einer Teststatistik $T_D$ verrechnet. Welche der Formel
berechnet korrekt die Teststatistik $T_D$?



\begin{enumerate}
\item [\textbf{A} \msquare] Es gilt $T_D = (signal \cdot noise)^2$
\item [\textbf{B} \msquare] Es gilt $T_D = \cfrac{signal}{noise}$
\item [\textbf{C} \msquare] Es gilt $T_D = \cfrac{signal}{noise^2}$
\item [\textbf{D} \msquare] Es gilt $T_D = \cfrac{noise}{signal}$
\item [\textbf{E} \msquare] Es gilt $T_D = signal \cdot noise$
\end{enumerate}

%% ------------------------------------------------------------

\section{Aufgabe \hfill (2 Punkte)}



Eine Analogie kann helfen einen Sachverhalt besser zu verstehen. Wie kann folgende Aussage richtig in die Analogie der statistischen Testtheorie gesetzt werden?

\begin{center}
\textit{$H_0$ ablehnen obwohl die $H_0$ gilt}
\end{center}



\begin{enumerate}
\item [\textbf{A} \msquare] In die Analogie eines Rauchmelders: \textit{Alarm with fire}.
\item [\textbf{B} \msquare] Dem $\beta$-Fehler mit der Analogie eines Rauchmelders: \textit{Fire without alarm}.
\item [\textbf{C} \msquare] In die Analogie eines Rauchmelders: \textit{Alarm without fire}, dem $\alpha$-Fehler.
\item [\textbf{D} \msquare] \textit{Fire without alarm}, dem $\beta$-Fehler als Analogie von Rauch im Haus.
\item [\textbf{E} \msquare] Dem $\beta$-Fehler mit der Analogie eines brennenden Hauses: \textit{Fire without alarm}.
\end{enumerate}

\section{Aufgabe \hfill (2 Punkte)}



Sie sollen in Ihrer Abschlussarbeit die Relevanz und die Signifikanz in einer statistischen Maßzahl vereinen. Welche Aussage ist richtig?



\begin{enumerate}
\item [\textbf{A} \msquare] Die Teststatistik. Durch den Vergleich von $T_c$ zu $T_k$ ist es m{"o}glich die $H_0$ abzulehnen. Die Relevanz ergibt sich aus der Fläche rechts vom dem $T_c$-Wert.
\item [\textbf{B} \msquare] Über das Konfidenzintervall. Das Konfidenzinterval beitet eine Entscheidung über die Signifikanz und zusätzlich kann über die Visualizierung des Konfidenzintervals eine Relevanzschwelle definiert werden.
\item [\textbf{C} \msquare] Das OR. Als Chancenverhältnis gibt es das Verhältnis von Relevanz und Signifikanz wieder.
\item [\textbf{D} \msquare] Einem Konfidenzintervall. Das Konfidenzinterval bringt durch eine Visualisierung und drei Intervallgrenzen die Möglichkeit mit, eine Relevanzschwelle neben der Signifikanzschwelle und der $\alpha$-Schwelle zu definieren.
\item [\textbf{E} \msquare] Über das Konfidenzintervall. Das Konfidenzinterval inkludiert eine Entscheidung über die Relevanz und zusätzlich kann über die Visualizierung des Konfidenzintervals eine Signifikanzschwelle vom Forschenden definiert werden.
\end{enumerate}

\section{Aufgabe \hfill (2 Punkte)}



Welche Aussage über den $p$-Wert und dem Signifikanzniveau $\alpha$ gleich 5\% ist richtig?



\begin{enumerate}
\item [\textbf{A} \msquare] Wir vergleichen mit dem $p$-Wert und dem Signifikanzniveau $\alpha$ absolute Werte auf einem Zahlenstrahl und damit den Unterschied der Teststatistiken, wenn die $H_0$ gilt.
\item [\textbf{B} \msquare] Wir vergleichen die Effekte des $p$-Wertes mit den Effekten der Signifikanzschwelle unter der Annahme der Nullhypothese. Dabei gilt, dass wir die Nullhypothese nur ablehnen können anhand des Falsifikationsprinzips.
\item [\textbf{C} \msquare] Wir schauen, ob der $p$-Wert größer ist als das Signifikanzniveau $\alpha$ und vergleichen somit Wahrscheinlichkeiten. Die Wahrscheinlichkeiten werden als Flächen unter der Kurve der Teststaistik dargestellt, wenn die $H_A$ gilt.
\item [\textbf{D} \msquare] Wir schauen, ob der $p$-Wert kleiner ist als das Signifikanzniveau $\alpha$ und vergleichen somit Wahrscheinlichkeiten. Die Wahrscheinlichkeiten werden als Flächen unter der Kurve der Teststaistik dargestellt, wenn die $H_0$ gilt.
\item [\textbf{E} \msquare] Wir machen eine Aussage über die indivduelle Wahrscheinlichkeit des Eintretens der Nullhypothese $H_0$. Der $p$-Wert wird mit dem Signifikanzniveau verglichen und bewertet.
\end{enumerate}

\section{Aufgabe \hfill (2 Punkte)}



Die Ergebnisse der einer statistischen Analyse können in die Analogie einer Wettervorhersage gebracht werden. Welche Analogie für die Ergebnisse eines statistischen Tests trifft am besten zu?



\begin{enumerate}
\item [\textbf{A} \msquare] In der Analogie der Sonnenscheindauer: Wie lange kann mit einem entsprechenden Effekt gerechnet werden? Die Wahrscheinlichkeit für den Effekt gibt der statistische Test wieder.
\item [\textbf{B} \msquare] In der Analogie des Niederschlags oder Regenmenge: ein statistischer Test gibt die Stärke eines Effektes wieder. Zum Beispiel, wie hoch ist der Mittelwertsunterschied.
\item [\textbf{C} \msquare] Die Analogie der Regenwahrscheinlichkeit: der statistische Test erlaubt es die Wahrscheinlichkeit für Regen abzuschätzen jedoch nicht die Menge und somit den Effekt.
\item [\textbf{D} \msquare] In der Analogie der Durchschnittstemperatur: Wie oft tritt ein Effekt durchschnittlich ein? Wir erhalten eine Wahrscheinlichkeit für die Effekte. Zum Beispiel, wie hoch ist die Wahrscheinlichkeit für einen Mittelwert als Durchschnitt.
\item [\textbf{E} \msquare] In der Analogie der Regenwahrscheinlichkeit in einem bestimmten Gebiet: ein statistischer Test gibt die Wahrscheinlichkeit für ein Ereignis in einem Experiment mit den Daten $D$ wieder und lässt sich kaum verallgemeinern.
\end{enumerate}

\section{Aufgabe \hfill (2 Punkte)}



In Ihrer Forschungsarbeit wollen Sie eine Aussage über die untersuchte Population treffen. Dazu nutzen Sie eine ANOVA als statistischen Test. Erhalten Sie eine valide Aussage aus einem statistischen Test?



\begin{enumerate}
\item [\textbf{A} \msquare] Ja, es ist möglich die untersuchte Population mit einem t-Test auszuwerten. Wir erhalten dann eine Aussage zur Population.
\item [\textbf{B} \msquare] Weder eine Ausssage über die Population noch über das Individuum ist mit einem statistischen Test möglich. Wir erhalten eine Aussage über ein Experiment.
\item [\textbf{C} \msquare] Nein, wir erhalten nur eine Aussage zu zwei Individuen. Ein statistischer Test liefert Informationen zu einem Individuum im Vergleich zu einem anderen Individuum.
\item [\textbf{D} \msquare] Ja, wir erhalten eine Aussage. Müssen aber das Individuum im Kontext der Population adjustieren.
\item [\textbf{E} \msquare] Nein, die untersuchte Population können wir mit einem statistischen Test nicht auswerten. Wir erhalten keine Aussage zur Population.
\end{enumerate}

\section{Aufgabe \hfill (2 Punkte)}



In der statistischen Testtheorie gibt es den Begriff \textit{Power}. Was sagt der statistische Begriff \textit{Power} aus?



\begin{enumerate}
\item [\textbf{A} \msquare] Die Power beschreibt die Wahrscheinlichkeit die $H_A$ abzulehnen. Wir testen die Power jedoch nicht.
\item [\textbf{B} \msquare] Alle statistischen Tests sind so konstruiert, dass die $H_A$ mit 80\% \textit{bewiesen wird}. Die Power ist $1-\beta$ mit $\beta$ gleich 20\% gesetzt.
\item [\textbf{C} \msquare] Die Power $1-\beta$ wird auf 80\% gesetzt. Damit liegt die Wahrscheinlichkeit für die $H_0$ bei 20\%.
\item [\textbf{D} \msquare] Es gilt $\alpha + \beta = 1$ und somit liegt $\beta$ meist bei 95\%.
\item [\textbf{E} \msquare] Die Power wird berechnet und ist keine Eigenschaft des Tests. Die Power wird auf $80\%$ gesetzt und beschreibt mit welcher Wahrscheinlichkeit $H_0$ \textit{bewiesen wird}
\end{enumerate}

\section{Aufgabe \hfill (2 Punkte)}



In Ihrer Abschlussarbeit sollen Sie neben den p-Werten auch die Effekte mit angeben. Welche Aussage ist richtig?



\begin{enumerate}
\item [\textbf{A} \msquare] Durch den Effekt erfahren wir die biologisch interpretierbare Ausgabe eines statistischen Tests. Zum Beispiel das $\eta^2$ aus einer ANOVA. Damit können wir die Relevanz direkt mit dem Effekt verbinden. Am Ende muss der Forschende aber entscheiden, ob der Effekt entsprechend seinen Erwartungen als bedeutet zu bewerten ist.
\item [\textbf{B} \msquare] Der Effekt eines statistischen Tests beschreibt die biologisch interpretierbare Ausgabe eines Tests. Moderen Algorithmen liefern keine Effekte mehr sondern nur noch bedingte Wahrscheinlichkeiten. Der Effekt spielt in der modernen Statistik keine Rollen mehr.
\item [\textbf{C} \msquare] Der Effekt eines statistischen Tests beschreibt die biologisch interpretierbare Ausgabe eines Tests. Damit ist der Effekt direkt mit dem Begriff der Signifikanz verbunden. Die Entscheidung über die Signifikanz trifft der Forschende unabhängig von der Relevanz eines statistsichen Tests.
\item [\textbf{D} \msquare] Der Effekt eines statistischen Tests beschreibt den Output oder die Wiedergabe eines Tests in einem Computer.
\item [\textbf{E} \msquare] Der Forschende muss am Anfang wissen, ob das Eregbnis eines Experiments relevant für seine Forschung ist. Dafür kann der Effekt eines statistischen Tests genutzt werden oder auch der Prähoc-Test. Damit beschreibt der Effekt den biologischen interpretierbaren Teil eines Experimnts vor der Durchführung. Zum Beispiel der Unterschied zwischen zwei Mittelwerten.
\end{enumerate}

\section{Aufgabe \hfill (2 Punkte)}



Welche Aussage über die Entscheidung anhand des p-Wertes gegen die
Nullhypothese ist richtig?



\begin{enumerate}
\item [\textbf{A} \msquare] Ist $Pr(D|H_0)$ kleiner als das Signifikanzniveau $\alpha$ gleich $5\%$ dann wird die Nullhypothese $H_0$ abgelehnt.
\item [\textbf{B} \msquare] Anhand des p-Wertes lässt sich wie folgt eine Entscheidung treffen. Liegt der Wert über oder gleich dem Signifikanzniveau $\alpha$ dann kann die Nullhypothese abgelehnt werden.
\item [\textbf{C} \msquare] Ist in dem 95\%-Konfidenzintervall nicht die Null enthalten dann wird die Nullhypothese $H_0$ abgelehnt.
\item [\textbf{D} \msquare] Ist $T_{D}$ h{"o}her als der kritische Wert $T_{\alpha = 5\%}$ dann wird die Nullhypothese $H_0$ abgelehnt.
\item [\textbf{E} \msquare] Anhand des p-Wertes lässt sich wie folgt eine Entscheidung treffen. Liegt der Wert in dem Signifikanzniveauintervall $\alpha$ dann kann die Nullhypothese abgelehnt werden.
\end{enumerate}

\section{Aufgabe \hfill (2 Punkte)}



Ein statistischer Test benötigt für die richtige Durchführung Hypothesen $H$, sonst ist der Test nicht zu interpretieren. Welche Aussage ist richtig?



\begin{enumerate}
\item [\textbf{A} \msquare] Die Hypothesen $H_0$ und $H_A$ sind rein prosarischer Natur und bilden keinen mathematischen Hintergrund ab. In der Statistik wird die wissenschaftliche Fragestellung getestet. Daher stehen auch die verständlichen Hypothesen im Mittelpunkt der biologischen Interpretation.
\item [\textbf{B} \msquare] Mit der Nullhypothese $H_0$ und der Alternativehypothese $H_A$ oder $H_1$ gibt es zwei Hypothesen.
\item [\textbf{C} \msquare] Es gibt ein statistisches Hypothesenpaar mit der Hypothese für und gegen die wissenschaftliche Fragestellung. Die Hypothesen werden $H_{pro}$ und $H_{contra}$ bezeichnet.
\item [\textbf{D} \msquare] Es gibt - bedingt durch das das Falsifikationsprinzip - ein Set von $k$ Nullhypothesen, die iterative gegen $k-1$ Alternativhypothesen getestet werden.
\item [\textbf{E} \msquare] Mit der Nullhypothese $H_A$ und der Alternativehypothese $H_0$ gibt es zwei Hypothesen, die aber selten genutzt werden.
\end{enumerate}
\section*{Statistische Tests für Gruppenvergleiche} 

\section{Aufgabe \hfill (2 Punkte)}



Nach einem Feldexperiment wollen Sie zwei Gruppen mit einem Welch t-Test vergleichen. Welche Aussage ist auch für den Student t-Test richtig?



\begin{enumerate}
\item [\textbf{A} \msquare] Der t-Test vergleicht die Mittelwerte von zwei Gruppen unter der strikten Annahme von Varianzhomogenität. Sollte keine Varianzhomogenität vorliegen, so gibt es keine Möglichkeit den t-Test in einer Variante anzuwenden.
\item [\textbf{B} \msquare] Der t-Test berechnet die Differenz von zwei Mittelwerten als Effekt und gibt eine Entscheidung, ob sich die beiden Mittelwerte \textit{jeweils} von Null unterscheiden.
\item [\textbf{C} \msquare] Der t-Test testet generell zu einem erhöhten $\alpha$-Niveau von 20\%.
\item [\textbf{D} \msquare] Der t-Test vergleicht die Varianzen von mindestens zwei oder mehr Gruppen
\item [\textbf{E} \msquare] Der t-Test berechnet die Differenz von zwei Mittelwerten als Effekt und gibt eine Entscheidung, ob sich die beiden Mittelwerte in den Gruppen signifikant unterscheiden.
\end{enumerate}

\section{Aufgabe \hfill (2 Punkte)}



Die folgende Abbildung enthält die Daten aus einer Studie zur Bewertung der Wirkung des Mikronährstoff Nitrat auf den Ertrag in t/ha von Papaya im Vergleich zu einer Kontrolle. Der Versuch wurde in 9 Parzellen pro Gruppe durchgeführt. Welche Aussage im Bezug auf eine statistische Auswertung ist richtig?



{\centering \includegraphics[width=\maxwidth]{img/mc-testing-ttest-02-1} 

}







\begin{enumerate}
\item [\textbf{A} \msquare] Nach Betrachtung des Barplots liegt ein signifikanter Unterschied vor. Der Effekt liegt bei 6.
\item [\textbf{B} \msquare] Die Barplots deuten auf einen signifikanten Unterschied. Der Effekt liegt vermutlich bei 6 unter einer groben Abschätzung. Wir müssen aber eine ANOVA rechnen um den Effekt wirklich bestimmen zu können.
\item [\textbf{C} \msquare] Es liegt ein signifikanter Unterschied vor. Der Effekt liegt bei 0.6.
\item [\textbf{D} \msquare] Die Barplots deuten auf keinen signifikanten Unterschied. Der Effekt liegt vermutlich bei 6. Wir müssen aber einen Posthoc-Test rechnen um den Effekt wirklich bestimmen zu können.
\item [\textbf{E} \msquare] Der Test deutet auf kein signifikanten Unterschied hin. Der Effekt liegt vermutlich bei 6.
\end{enumerate}

\section{Aufgabe \hfill (2 Punkte)}




Sie rechnen einen gepaarten t-Test, da Ihre Beobachtungen verbunden sind. Welche der folgenden Aussagen ist richtig?



\begin{enumerate}
\item [\textbf{A} \msquare] Wenn die Beobachtungen unabhängig voneinander sind, rechnen wir einen gepaarten t-Test. Messen wir wiederholt an dem gleichen Tier oder Pflanze dann bilden wir das Produkt zwischen den zwei Messpunkten.
\item [\textbf{B} \msquare] Abhängige Beobachtungen müssen gesondert in einem gepaarten t-Test modelliert werden. Wenn wiederholt an dem gleichen Tier oder Pflanze gemessen wird, dann bilden wir die Differenz zwischen den beiden Zeitpunkten. Auf den Differenzen rechnen wir den gepaarten t-Test.
\item [\textbf{C} \msquare] Abhängige Beobachtungen müssen gesondert in einem gepaarten t-Test modelliert werden. Wenn wiederholt an dem gleichen Tier oder Pflanze gemessen wird, dann bilden wir den Quotienten zwischen den beiden Zeitpunkten. Auf den Quotienten rechnen wir den gepaarten t-Test.
\item [\textbf{D} \msquare] Der gepaarte t-Test wird genutzt, wenn die Differenzen der Beobachtungen verbunden sind und wir dadurch die Unabhäängigkeit nicht mehr vorliegen haben.
\item [\textbf{E} \msquare] Beim gepaarten t-Test kombinieren wir die Vorteile des Student t-Test für Varianzhomogenität mit den Vorteilen des Welch t-Test für Varianzheterogenität. Wir bilden dafür die Differenz der Einzelbeobachtungen.
\end{enumerate}

\section{Aufgabe \hfill (2 Punkte)}



Nach einem Experiment mit fünf Weizensorten ergibt eine ANOVA ($p = 0.048$) einen signifikanten Unterschied für den Ertrag. Sie führen anschließend die paarweisen t-Tests für alle Vergleiche der verschiedenen Weizensorten durch. Nach der Adjustierung für multiples Testen ist kein p-Wert unter der $\alpha$-Schwelle. Sie schauen sich auch die rohen, unadjustierten p-Werte an und finden hier als niedrigsten p-Wert $p_{3-2} = 0.053$. Welche Aussage ist richtig?




\begin{enumerate}
\item [\textbf{A} \msquare] Der Fehler liegt in den t-Tests. Wenn eine ANOVA signifikant ist, dann muss zwangsweise auch ein t-Test signifikant sein.
\item [\textbf{B} \msquare] Das ist kein Wunder. Die ANOVA testet auf der gesamten Fallzahl und die paarweisen t-Tests verlieren immer eine oder mehr Gruppen als Fallzahl. Mit steigender Fallzahl sind mehr signifikante Unterschiede zu erwarten. Die p-Werte unterscheiden sich numerisch auch kaum.
\item [\textbf{C} \msquare] Die ANOVA testet auf der gesamten Fallzahl. Es wäre besser die ANOVA auf der gleichen Fallzahl wie die einzelnen t-Tests zu rechnen.
\item [\textbf{D} \msquare] Hier kommt der Effekt der stiegenden Fallzahl auf die Anzahl an signifikante Ergebnisse zu tragen. Da die ANOVA auf weniger Fallzahl testet als die paarweisen t-Tests, kann die ANOVA schwerer einen signifikanten Unterscheid nachweisen.
\item [\textbf{E} \msquare] Das ist kein Wunder. Die ANOVA testet nicht auf der gesamten Fallzahl und die paarweisen t-Tests gewinnen immer eine oder mehr Gruppen als Fallzahl dazu. Mit steigender Fallzahl sind mehr signifikante Unterschiede zu erwarten. Die p-Werte unterscheiden sich numerisch auch kaum.
\end{enumerate}
    
% -----------------------------------------------------------------------
\clearpage
% -----------------------------------------------------------------------
\part{Deskriptive Statistik \& Explorative Datenanalyse}
% -----------------------------------------------------------------------

\section{Aufgabe \hfill (8 Punkte)}

\textit{Geben Sie grundsätzlich Formeln und Rechenweg zur Lösung der Teilaufgaben mit an!} \\[1Ex]
 

 
%% --------------------------------------------------------------------
\begin{minipage}[t]{0.5\textwidth}
\includegraphics[width = 1.3cm]{/Users/kruppajo/work/GitHub/exam/avatare/Mark.png}
\end{minipage}
\begin{minipage}[t]{0.5\textwidth}
\hfill
\href{https://youtu.be/t0WYa_LVc5o}{\includegraphics[width = 2cm]{img/youtube}}\\[1Ex]
\end{minipage}
\vspace{1ex}
%% --------------------------------------------------------------------



Mark steht vor einem ersten Problem, denn wenn es nach seiner Betreuerin geht, soll er in einem einem Feldexperiment Erbsen auswertet. Soweit eigentlich alles passend. Besser wäre was anderes gewesen. Mark liebt Geocaching. Darin kann er sich wirklich verlieren und immer wieder neu begeistern. Das heißt erstmal überlegen für Mark. Mark schmeißt noch eine Handvoll Marzipankugeln in seinen Rachen. Im Hintergrund klirrt leise der Spiegel zum Sound von Andrea Berg. Die Behandlung werden verschiedene Lüftungssysteme und Folientunnel ($ctrl$, $storm$ und $tornado$) sein. In seiner Exceldatei wird er den Endpunkt ($Y$) \textit{Frischegewicht} als \textit{freshmatter} aufnehmen. Vorab soll Mark aber eimal die folgenden Barplots seiner Betreuerin nachbauen, damit er den \Rlogo Code schonmal für später vorliegen hat. Damit geht das Problem schon los. Wenn die Unsicherheit nicht wäre, ja dann wäre wohl vieles möglich für Mark! Aber so..



{\centering \includegraphics[width=\maxwidth]{img/barplot-02-1} 

}




Leider kennt sich Mark mit der Erstellung von Barplots in \Rlogo nicht aus. Deshalb braucht er bei der Visualisierung Ihre Hilfe!

\begin{enumerate}
\item Formulieren Sie die wissenschaftliche Fragestellung! \textbf{(1 Punkt)}
\item Erstellen Sie eine Tabelle mit den statistischen Maßzahlen aus der obigen Abbildung der drei Barplots! \textit{Beachten Sie die korrekte Darstellungsform der statistischen Maßzahlen!} \textbf{(3 Punkte)}
\item Erstellen Sie einen beispielhaften Datensatz, aus dem die drei Barplots \textit{möglicherweise} erstellt wurden, im \Rlogo üblichen Format! \textbf{(2 Punkte)}
\item Kann Mark einen Unterschied zwischen den Behandlungen erwarten? Begründen Sie Ihre Antwort! \textbf{(2 Punkte)}
\end{enumerate} 
\clearpage
% -----------------------------------------------------------------------

\section{Aufgabe \hfill (8 Punkte)}

\textit{Geben Sie grundsätzlich Formeln und Rechenweg zur Lösung der Teilaufgaben mit an!} \\[1Ex]
 

 
%% --------------------------------------------------------------------
\begin{minipage}[t]{0.5\textwidth}
\includegraphics[width = 1.3cm]{/Users/kruppajo/work/GitHub/exam/avatare/Jonas.png}
\end{minipage}
\begin{minipage}[t]{0.5\textwidth}
\hfill
\href{https://youtu.be/vXnLttRL_VI}{\includegraphics[width = 2cm]{img/youtube}}\\[1Ex]
\end{minipage}
\vspace{1ex}
%% --------------------------------------------------------------------



Barplots sind bedeutend in der Darstellung von wissenschaftlichen Ergebnissen. Leider hat sich Jonas nicht gemerkt, welche statistischen Maßzahlen für einen Barplot erhoben werden müssen. Besser wäre was anderes gewesen. Stricken. Ein wunderbares Hobby um sich drin zu verlieren und Abstand zu bekommen. Jonas denkt gerne über Stricken nach. Das ist in soweit doof, da nach seiner Betreuerin nun Barplots aus seinen Daten gebaut werden sollen, bevor es mit dem statistischen Testen weitergeht. Na dann mal los. Jonas schafft sich die nötige Stimmung. Jonas streichelt liebevoll das Meerschweinchen. Der Kopf ist in seinem Schloß vergraben um den Klang von Iron Maiden zu dämpfen. Die Behandlung für Brokoli waren verschiedene Substrattypen ($torf$, $40p60n$ und $70p30n$). Erfasst wurde von Jonas als Endpunkt ($Y$) \textit{Proteingehalt}. Jonas hat dann \textit{protein} in seiner Exceldatei eintragen. Aber auch irgendwie egal. Jonas will später nochmal raus um zu Schwimmen. Druck ablassen, dass muss er auch.

\begin{table}[!h]
\centering
\begin{tabular}{cc}
\toprule
treatment & protein\\
\midrule
torf & 40.4\\
torf & 40.0\\
70p30n & 41.6\\
70p30n & 29.7\\
70p30n & 27.7\\
\addlinespace
torf & 27.7\\
70p30n & 23.6\\
70p30n & 31.8\\
40p60n & 47.3\\
40p60n & 38.6\\
\addlinespace
torf & 21.1\\
40p60n & 42.9\\
\bottomrule
\end{tabular}
\end{table}



Leider kennt sich Jonas mit der Erstellung von Barplots nicht aus. Deshalb braucht er bei der Visualisierung Ihre Hilfe!

\begin{enumerate}
\item Formulieren Sie die wissenschaftliche Fragestellung! \textbf{(1 Punkt)}
\item Zeichnen Sie in \textit{einer} Abbildung die Barplots für die Behandlung von Brokoli! Beschriften Sie die Achsen entsprechend!\textbf{(4 Punkte)}
\item Beschriften Sie \textit{einen} Barplot mit den gängigen statistischen Maßzahlen! \textbf{(2 Punkte)}
\item Wenn Jonas \textit{keinen Effekt} zwischen den Behandlungen von Brokoli erwarten würde, wie sehen dann die Barplots aus? \textit{Antworten Sie mit einer Skizze der Barplots!}
  \textbf{(1 Punkt)}
\end{enumerate} 
\clearpage
% -----------------------------------------------------------------------

\section{Aufgabe \hfill (9 Punkte)}

\textit{Geben Sie grundsätzlich Formeln und Rechenweg zur Lösung der Teilaufgaben mit an!} \\[1Ex]
 

 
%% --------------------------------------------------------------------
\begin{minipage}[t]{0.5\textwidth}
\includegraphics[width = 1.3cm]{/Users/kruppajo/work/GitHub/exam/avatare/Yuki.png}
\end{minipage}
\begin{minipage}[t]{0.5\textwidth}
\hfill
\href{https://youtu.be/Xf0yE-o7bEU}{\includegraphics[width = 2cm]{img/youtube}}\\[1Ex]
\end{minipage}
\vspace{1ex}
%% --------------------------------------------------------------------



Wenn die Faulheit nicht wäre, ja dann wäre wohl vieles möglich für Yuki! Aber so.. Deshalb gilt anschauen, was andere vor einem gemacht haben. Für Yuki ist es eine Möglichkeit schneller ans Ziel zu gelangen. Yuki soll in ihrer Hausarbeit Lauch untersuchen. Die Behandlung in ihrer Hausarbeit werden verschiedene Lüftungssystemen und Folientunneln ($ctrl$, $storm$ und $tornado$) sein. Erheben wird Yuki als Endpunkt ($Y$) \textit{Frischegewicht} benannt als \textit{freshmatter} in ihrer Exceldatei. Von ihrer Betreuerin erhält sie nun folgende Abbildung von Boxplots, die sie erstmal zur Übung nachbauen soll, bevor sie mit dem eigentlichen Versuch beginnt. Aber nur in passender Atmospäre! Schon dutzende Male gesehen: Matrix. Aber immer noch großartig zusammen mit Reese's Peanut Butter Cups.



{\centering \includegraphics[width=\maxwidth]{img/boxplot-02-zer-1} 

}




Leider kennt sich Yuki mit der Erstellung von Boxplots in \Rlogo nicht aus. Deshalb braucht sie bei der Visualisierung Ihre Hilfe!

\begin{enumerate}
\item Erstellen Sie eine Tabelle mit den statistischen Maßzahlen aus der obigen Abbildung der drei Boxplots! \textit{Beachten Sie die korrekte Darstellungsform der statistischen Maßzahlen!} \textbf{(3 Punkte)}
\item Beschriften Sie \textit{einen} der Boxplots mit den gängigen statistischen Maßzahlen! \textbf{(2 Punkte)}
\item Erstellen Sie einen beispielhaften Datensatz, aus dem die drei Boxplots \textit{möglicherweise} erstellt wurden, im \Rlogo üblichen Format! \textbf{(2 Punkte)}
\item Kann Yuki einen Unterschied zwischen den Behandlungen erwarten? Begründen Sie Ihre Antwort! \textbf{(2 Punkte)}
\end{enumerate} 
\clearpage
% -----------------------------------------------------------------------

\section{Aufgabe \hfill (9 Punkte)}

\textit{Geben Sie grundsätzlich Formeln und Rechenweg zur Lösung der Teilaufgaben mit an!} \\[1Ex]
 

 
%% --------------------------------------------------------------------
\begin{minipage}[t]{0.5\textwidth}
\includegraphics[width = 1.3cm]{/Users/kruppajo/work/GitHub/exam/avatare/Nilufar.png}
\end{minipage}
\begin{minipage}[t]{0.5\textwidth}
\hfill
\href{https://youtu.be/0xc0jIPeiyw}{\includegraphics[width = 2cm]{img/youtube}}\\[1Ex]
\end{minipage}
\vspace{1ex}
%% --------------------------------------------------------------------



Eine echte Herausforderung für sie war schon immer die Erwartung gewesen. Ein leidiges Lied. Deshalb gilt anschauen, was andere vor einem gemacht haben. Für Nilufar ist es eine Möglichkeit schneller ans Ziel zu gelangen. Deshalb hat sich Nilufar viele Poster in der Fakultät angeschaut und ist zum Schluß gekommen, dass Boxplots eine häufig genutzte Abbildung sind. Nilufar soll nun in ihrer Abschlussarbeit Erdbeeren untersuchen. Die Behandlung in ihrer Abschlussarbeit sind verschiedene Substrattypen ($torf$ und $70p30n$). Erhoben wurden von Nilufar als Endpunkt ($Y$) \textit{Trockengewicht} benannt als \textit{drymatter} in ihrer Exceldatei. Erwartungsgemäß erhält sie von ihrer Betreuerin den Auftrag die erhobenen Daten als Boxplots darzustellen. Dann kann Nilufar auch schonmal abschätzen, was bei einem statistischen Test rauskommen könnte. Darüber hinaus kann Nilufar anhand Boxplots eine Aussage über die Varianzhomogenität über die Behandlungsgruppen treffen. Na dann mal los. Nilufar schafft sich die nötige Stimmung. Nilufar nickt im Takt von Deichkind und bemerkt dabei gar nicht was das Huhn schon wieder anstellt.

\begin{table}[!h]
\centering
\begin{tabular}{cc}
\toprule
treatment & drymatter\\
\midrule
torf & 45.2\\
torf & 41.7\\
torf & 37.2\\
torf & 39.6\\
70p30n & 35.6\\
\addlinespace
70p30n & 33.4\\
torf & 33.0\\
70p30n & 52.2\\
70p30n & 25.1\\
torf & 38.2\\
\addlinespace
70p30n & 45.7\\
torf & 38.1\\
torf & 39.5\\
70p30n & 43.4\\
70p30n & 49.3\\
\addlinespace
torf & 44.7\\
70p30n & 38.5\\
torf & 38.6\\
\bottomrule
\end{tabular}
\end{table}



Leider kennt sich Nilufar mit der Erstellung von Boxplots nicht aus. Deshalb braucht sie bei der Visualisierung Ihre Hilfe!

\begin{enumerate}
\item Zeichnen Sie in \textit{einer} Abbildung die beiden Boxplots für die zwei Behandlungen von Erdbeeren! Beschriften Sie die Achsen entsprechend! \textbf{(5 Punkte)} 
\item Wie ist Ihr Vorgehen, wenn Sie eine \textit{gerade} Anzahl an
  Beobachtungen pro Gruppe haben? \textbf{(1 Punkt)}
\item Beschriften Sie \textit{einen} der beiden Boxplots mit den gängigen
  statistischen Maßzahlen! \textbf{(2 Punkte)}
\item Wenn Sie \textit{keinen Effekt} zwischen den Behandlungen von
  Erdbeeren erwarten würden, wie sehen dann die beiden Boxplots aus?
  \textit{Antworten Sie mit einer Skizze der Boxplots!}
  \textbf{(1 Punkt)}
\end{enumerate} 
\clearpage
% -----------------------------------------------------------------------

\section{Aufgabe \hfill (8 Punkte)}

\textit{Geben Sie grundsätzlich Formeln und Rechenweg zur Lösung der Teilaufgaben mit an!} \\[1Ex]
 

 
%% --------------------------------------------------------------------
\begin{minipage}[t]{0.5\textwidth}
\includegraphics[width = 1.3cm]{/Users/kruppajo/work/GitHub/exam/avatare/Jessica.png}
\end{minipage}
\begin{minipage}[t]{0.5\textwidth}
\hfill
\href{https://youtu.be/aXvxGC4YLqk}{\includegraphics[width = 2cm]{img/youtube}}\\[1Ex]
\end{minipage}
\vspace{1ex}
%% --------------------------------------------------------------------



In ihrer Abschlussarbeit möchte Jessica gerne die Daten aus einem Freilandversuch mit Kartoffeln in einem Histogramm darstellen. Das Histogramm erlaubt ihr dabei Rückschlüsse auf die Verteilung über den Messwert ($Y$) zu treffen. Aus den Boxen wummert David Bowie und ihr Mund ist verklebt von Schokobons. 'Herrlich', denkt Jessica. In seinem Experiment hat Jessica die seltsamen Verdickungen gezählt. Es wäre einfacher, wenn da nicht noch was wäre. Eine echte Herausforderung für sie war schon immer der Mangel gewesen. Ein leidiges Lied. Wenn David Bowie ertönt, dann sucht der Hund schleunigst Schutz unter dem Sofa. Jessica schüttelt den Kopf.

\begin{center}
Die seltsamen Verdickungen: 3, 4, 5, 2, 6, 6, 3, 2, 5, 2, 4, 3, 3, 2, 4, 1, 1, 4, 5, 7, 3, 3, 6, 6, 7, 0, 3, 5, 2, 4
\end{center}

Leider kennt sich Jessica mit der Erstellung von Histogrammen überhaupt nicht aus. Deshalb braucht sie bei der Erstellung Ihre Hilfe!

\begin{enumerate}
\item Zeichen Sie ein Histogramm um die Verteilung der Daten zu visualisieren! (\textbf{3 Punkte})
\item Beschriften Sie die Achsen der Abbildung! (\textbf{2 Punkte})
\item Ergänzen Sie die absoluten und relativen Häufigkeiten in der
  Abbildung! \textbf{(1 Punkt)}
\item Berechnen Sie aus den Daten die \textit{Wahrscheinlichkeit}
  mehr als die Anzahl 6 zu beobachten! \textbf{(1
    Punkt)}
\item Berechnen Sie aus den Daten die \textit{Chance} mehr
  als die Anzahl 6 zu beobachten! \textbf{(1 Punkt)}
\end{enumerate}

 
\clearpage
% -----------------------------------------------------------------------

\section{Aufgabe \hfill (8 Punkte)}

\textit{Geben Sie grundsätzlich Formeln und Rechenweg zur Lösung der Teilaufgaben mit an!} \\[1Ex]
 

 
%% --------------------------------------------------------------------
\begin{minipage}[t]{0.5\textwidth}
\includegraphics[width = 1.3cm]{/Users/kruppajo/work/GitHub/exam/avatare/Alex.png}
\end{minipage}
\begin{minipage}[t]{0.5\textwidth}
\hfill
\href{https://youtu.be/ORHSPTCdfeY}{\includegraphics[width = 2cm]{img/youtube}}\\[1Ex]
\end{minipage}
\vspace{1ex}
%% --------------------------------------------------------------------



In einem Gespräch mit seiner Betreuerin wird Alex gebeten seine Daten aus einem Feldexperiment mit Kartoffeln in einem Histogramm darzustellen. Alex schmeißt noch eine Handvoll Gummibärchen in seinen Rachen. Im Hintergrund klirrt leise der Spiegel zum Sound von Abba. In seinem Experiment hat er die mittleren Mehltauspots erst fotographiert und dann ausgezählt. Laut seiner Betreuerin soll das Histogramm helfen, die Verteilung der die mittleren Mehltauspots zu bestimmen.  Es wäre einfacher, wenn da nicht noch was wäre. Alex und die Gefälligkeit, eine unendliche Geschichte mit kniffeligen Wendungen. Wenn Abba ertönt, dann sucht die Katze schleunigst Schutz unter dem Sofa. Alex schüttelt den Kopf.

\begin{center}
Die mittleren Mehltauspots: 10.7, 5.4, 6.8, 13.6, 8.9, 12.2, 10, 9, 8.5, 9.6, 11.4, 10.4, 8.1, 8.3, 8, 9.5, 10.3, 11.5, 10, 6.5, 10.1, 11.5, 11.8, 6.8, 8.1
\end{center}

Leider kennt sich Alex mit der Erstellung von Histogrammen überhaupt nicht aus. Deshalb braucht er bei der Erstellung Ihre Hilfe!

\begin{enumerate}
\item Zeichen Sie ein Histogramm um die Verteilung der Daten zu visualisieren! (\textbf{3 Punkte})
 \item Erläutern Sie Ihr Vorgehen um ein Histogramm für kontinuierliche Daten zu zeichnen!  (\textbf{2 Punkte})
\item Beschriften Sie die Achsen der Abbildung! (\textbf{2 Punkte})
\item Ergänzen Sie die relativen Häufigkeiten in der Abbildung! \textbf{(1 Punkt)}  
\end{enumerate}

 
\clearpage
% -----------------------------------------------------------------------

\section{Aufgabe \hfill (10 Punkte)}

\textit{Geben Sie grundsätzlich Formeln und Rechenweg zur Lösung der Teilaufgaben mit an!} \\[1Ex]
 

 
%% --------------------------------------------------------------------
\begin{minipage}[t]{0.5\textwidth}
\includegraphics[width = 1.3cm]{/Users/kruppajo/work/GitHub/exam/avatare/Jessica.png}
\end{minipage}
\begin{minipage}[t]{0.5\textwidth}
\hfill
\href{https://youtu.be/VAqiUdV4WQ0}{\includegraphics[width = 2cm]{img/youtube}}\\[1Ex]
\end{minipage}
\vspace{1ex}
%% --------------------------------------------------------------------




Jessica liest laut: 'Wenn zwei kontinuierliche Variablen vorliegen, können diese in einer exploartiven Datenanalyse...'. Jessica stoppt. Jessica schmeißt noch eine Handvoll Schokobons in ihren Rachen. Im Hintergrund klirrt leise der Spiegel zum Sound von David Bowie. Was waren noch gleich kontinuierliche Variablen? In ihrer Abschlussarbeit hatte sie einen Versuch in einer Klimakammer im Teuteburgerwald durchgeführt. Dabei ging es um den Zusammenhang zwischen Proteingehalt [g/kg] und durschnittlichen Niederschlag [ml/w] im groben Kontext von Kartoffeln. Nun stellt sich die Frage für sie, ob es überhaupt einen Zusammenhang zwischen den gemessenen Variablen gibt. Dafür war eine explorative Datenanalyse gut! Wenn der Mangel nicht wäre, ja dann wäre wohl vieles möglich für Jessica! Aber so.. Dann was anderes. Wenn Herr der Ringe läuft, dann ist der Hund nicht mehr da. Aber jetzt braucht sie mal Entspannung!

\begin{table}[!h]
\centering
\begin{tabular}{cc}
\toprule
Proteingehalt [g/kg] & Durschnittlichen Niederschlag [ml/w]\\
\midrule
17.8 & 14.1\\
23.5 & 21.4\\
18.0 & 19.7\\
16.6 & 16.3\\
24.7 & 18.4\\
\addlinespace
14.8 & 15.8\\
28.7 & 25.1\\
22.8 & 17.8\\
20.4 & 18.6\\
15.2 & 17.2\\
\addlinespace
20.9 & 18.5\\
21.6 & 16.5\\
\bottomrule
\end{tabular}
\end{table}



Leider kennt sich Jessica mit der Erstellung einer explorativen Datenanalyse für kontinuierliche Daten überhaupt nicht aus. Deshalb braucht sie bei der Erstellung Ihre Hilfe!

\begin{enumerate}
\item Erstellen Sie eine Visualisierung für die Datentabelle. Beschriften Sie
  die Achsen entsprechend! \textbf{(4 Punkte)}
\item Schätzen Sie eine Gerade durch die Punkte! \textbf{(1 Punkt)}
\item Beschriften Sie die Gerade mit den gängigen statistischen Maßzahlen! Geben Sie die numerischen Zahlenwerte mit an! \textbf{(3 Punkte)}
\item Wenn \textit{ein} Effekt von $x$ auf $y$ vorhanden wäre, wie würde die Gerade verlaufen und welche Werte würden die statistischen Maßzahlen annehmen? \textbf{(2 Punkt)}
\end{enumerate} 
\clearpage
% -----------------------------------------------------------------------

\section{Aufgabe \hfill (10 Punkte)}

\textit{Geben Sie grundsätzlich Formeln und Rechenweg zur Lösung der Teilaufgaben mit an!} \\[1Ex]
 

 
%% --------------------------------------------------------------------
\begin{minipage}[t]{0.5\textwidth}
\includegraphics[width = 1.3cm]{/Users/kruppajo/work/GitHub/exam/avatare/Nilufar.png}
\end{minipage}
\begin{minipage}[t]{0.5\textwidth}
\hfill
\href{https://youtu.be/t_1KL77mfmg}{\includegraphics[width = 2cm]{img/youtube}}\\[1Ex]
\end{minipage}
\vspace{1ex}
%% --------------------------------------------------------------------



Irgendwie komisch, wenn sie Star Trek anmacht, dann ist das Huhn eigentlich sofort vor dem Bildschirm und starrt hinein. Aber Ablenkung hilft nur begrenzt. 'Uff!', denkt sich Nilufar. Jetzt hat sie doch tatsächlich zwei kategoriale Variablen in ihrer Abschlussarbeit gemessen. Zum einen die Behandlung Pestizideinsatz [ja/nein] und zum anderen die Messung Frischegewicht über Zielwert [ja/nein] im Kontext von Kartoffeln. Hierfür hat sie ein Freilandversuch im Wendland durchgeführt. Jetzt möchte Nilufar die Daten einmal in einer explorativen Datenanalyse darstellen. Danach kann sie dann über den passenden statistischen Test nachdenken. Dabei unterstützt ihr Betreuer diesen Ansatz bevor es in der Datenanalyse weiter geht. So schön wie so gut. Wenn die Erwartung nicht wäre, ja dann wäre wohl vieles möglich für Nilufar! Aber so..



\vspace{1Ex}

\begin{center}
\begin{minipage}[t]{0.45\textwidth}
%\small
\begin{center}

\begin{tabular}{p{2.5cm}p{2.5cm}p{2.5cm}p{2.5cm}}
\toprule
Frischegewicht über Zielwert & Pestizideinsatz\\
\midrule
nein & ja\\
ja & ja\\
nein & nein\\
nein & nein\\
nein & nein\\
\addlinespace
ja & ja\\
ja & nein\\
nein & ja\\
nein & nein\\
ja & nein\\
\addlinespace
ja & nein\\
ja & nein\\
ja & nein\\
nein & nein\\
ja & nein\\
\addlinespace
ja & nein\\
ja & ja\\
nein & nein\\
\bottomrule
\end{tabular}


\end{center}
\end{minipage}
\begin{minipage}[t]{0.45\textwidth}
%\small
\begin{center}

\begin{tabular}{p{2.5cm}p{2.5cm}p{2.5cm}p{2.5cm}}
\toprule
Frischegewicht über Zielwert & Pestizideinsatz\\
\midrule
nein & nein\\
nein & ja\\
nein & nein\\
ja & nein\\
ja & nein\\
\addlinespace
nein & nein\\
nein & ja\\
ja & nein\\
nein & nein\\
ja & nein\\
\addlinespace
ja & nein\\
ja & nein\\
nein & ja\\
nein & nein\\
ja & nein\\
\addlinespace
nein & nein\\
ja & ja\\
nein & nein\\
\bottomrule
\end{tabular}


\end{center}
\end{minipage}
\end{center}

\vspace{2Ex}

Leider kennt sich Nilufar mit der Erstellung einer explorativen Datenanalyse für kategoriale Daten überhaupt nicht aus. Deshalb braucht sie bei der Erstellung Ihre Hilfe!

\begin{enumerate}
\item Stellen Sie den Zusammenhang zwischen den beiden kategorialen Variablen in einer zusammenfassenden Tabelle dar! \textbf{(3 Punkte)}
\item Visualisieren Sie den Zusammenhang zwischen den beiden kategorialen Variablen! \textbf{(3 Punkte)}
\item Berechnen Sie die Verhältnisse in der Visualisierung! Welche Annahme haben Sie getroffen? \textbf{(2 Punkte)}
\item Wenn \textit{kein} Effekt von der Behandlung vorliegen würde, wie würde die Tabelle und die Visualisierung aussehen? \textbf{(2 Punkt)}
\end{enumerate} 
\clearpage
% -----------------------------------------------------------------------

\section{Aufgabe \hfill (10 Punkte)}

\textit{Geben Sie grundsätzlich Formeln und Rechenweg zur Lösung der Teilaufgaben mit an!} \\[1Ex]
 

 
%% --------------------------------------------------------------------
\begin{minipage}[t]{0.5\textwidth}
\includegraphics[width = 1.3cm]{/Users/kruppajo/work/GitHub/exam/avatare/Alex.png}\hspace{-4mm}\includegraphics[width = 1.3cm]{/Users/kruppajo/work/GitHub/exam/avatare/Nilufar.png}
\end{minipage}
\begin{minipage}[t]{0.5\textwidth}
\hfill
\href{https://youtu.be/Op-gjzASH9I}{\includegraphics[width = 2cm]{img/youtube}}
\end{minipage}
\vspace{1ex}
%% --------------------------------------------------------------------



'Was soll das denn jetzt schon wieder sein? Drei Boxplot, die auf der Seite liegen?', entfährt es Nilufar und schaut dabei Alex an. 'Keine Ahnung. Es ist bestimmt wieder so ein Lernziel mit der Verteilung und so.', meint Alex sichtlich genervt und mampft noch ein paar Gummibärchen. 'Du weißt doch wie es heißt, \textit{Frei ist, wer missfallen kann.}\footnote{Oschmann, A. (2024) Mädchen stärken: Stärken fördern, Selbstwert erhöhen und liebevoll durch Krisen begleiten. Goldegg Verlag}', merkt Nilufar nickend an. Die beiden schauen angestrengt auf die drei Boxplots. Das Ziel ist es zu verstehen, wie eine Verteilung anhand eines Boxplots bewertet werden kann. Alex und die Erwartung machen die Sache nicht einfacher.



{\centering \includegraphics[width=\maxwidth]{img/desc-stat-11-1} 

}




Jetzt brauchen Nilufar und Alex Ihre Hilfe bei der Abschätzung einer Verteilung anhand von Boxplots um ihre Arbeit dann in diesem Semester noch abschließen zu können.

\begin{enumerate}
\item Zeichnen Sie über die Boxplots die entsprechende zugehörige Verteilung! \textbf{(3 Punkte)} 
\item Zeichnen Sie unter die Boxplots die entsprechende zugehörige Beobachtungen als Stiche! \textbf{(3 Punkte)}
\item Wie viel Prozent der Beobachtungen fallen in das IQR? Ergänzen Sie die Abbildung entsprechend um den Bereich! \textbf{(2 Punkte)}
\item Wie viel Prozent der Beobachtungen fallen in $\bar{y} \pm 1s$ und $\bar{y} \pm 2s$  unter der Annahme einer Normalverteilung? \textbf{(2 Punkte)}
\end{enumerate} 
\clearpage
% -----------------------------------------------------------------------

\section{Aufgabe \hfill (10 Punkte)}

\textit{Geben Sie grundsätzlich Formeln und Rechenweg zur Lösung der Teilaufgaben mit an!} \\[1Ex]
 

 
%% --------------------------------------------------------------------
\begin{minipage}[t]{0.5\textwidth}
\includegraphics[width = 1.3cm]{/Users/kruppajo/work/GitHub/exam/avatare/Jonas.png}\hspace{-4mm}\includegraphics[width = 1.3cm]{/Users/kruppajo/work/GitHub/exam/avatare/Mark.png}
\end{minipage}
\begin{minipage}[t]{0.5\textwidth}
\hfill
\href{https://youtu.be/ZrJhn2wPbq4}{\includegraphics[width = 2cm]{img/youtube}}
\end{minipage}
\vspace{1ex}
%% --------------------------------------------------------------------



Mark und die Unsicherheit machen die Sache mit dem Studium nicht einfacher. Immerhin ist noch Jonas zur Hilfe mit dabei. Jonas hat Marzipankugeln mitgebracht und Andrea Berg aufgedreht. Das ist immerhin eine Ablenkung. Nicht so gut wie Geocaching, aber immerhin etwas. Jetzt sollen die beiden diese komische Aufgabe lösen. Es geht um verschiedene Normalverteilungen. Anscheinend hängen Normalverteilungen vom Mittelwert $\bar{y}$ und der Standardabweichung $s$ ab. 'Wozu brauchen wir nochmal Normalverteilungen?', entfährt es Mark. Durch das Mampfen von Jonas versteht er kein Wort der Antwort. Jonas lächelt.\\



Jetzt brauchen Mark und Jonas Ihre Hilfe bei der Abschätzung einer Verteilung um ihre Arbeit dann in diesem Semester noch abschließen zu können.

\begin{enumerate}
\item Skizzieren Sie drei Normalverteilungen mit $\bar{y}_1 \neq \bar{y}_2 \neq \bar{y}_3$ und $s_1 = s_2 = s_3$! \textbf{(3 Punkte)}
\item Beschriften Sie die Normalverteilungen mit den statistischen Maßzahlen! \textbf{(2 Punkte)}
\item Liegt Varianzhomogenität oder Varianzheterogenität vor? Begründen Sie Ihre Antwort! \textbf{(2 Punkte)}
\item In welchen Bereich fallen 68\% bzw. 95\% der Beobachtungen in einer Normalverteilung? Ergänzen Sie die Bereiche in \underline{einer} Normalverteilung! \textbf{(2 Punkte)}
\item Ergänzen Sie unter \underline{einer} der Normalverteilungen den entsprechenden Boxplot! \textbf{(1 Punkt)}
\end{enumerate}

 
\clearpage
% -----------------------------------------------------------------------

\section{Aufgabe \hfill (10 Punkte)}

\textit{Geben Sie grundsätzlich Formeln und Rechenweg zur Lösung der Teilaufgaben mit an!} \\[1Ex]
 

 
%% --------------------------------------------------------------------
\begin{minipage}[t]{0.5\textwidth}
\includegraphics[width = 1.3cm]{/Users/kruppajo/work/GitHub/exam/avatare/Alex.png}\hspace{-4mm}\includegraphics[width = 1.3cm]{/Users/kruppajo/work/GitHub/exam/avatare/Nilufar.png}
\end{minipage}
\begin{minipage}[t]{0.5\textwidth}
\hfill
\href{https://youtu.be/MiD42k4l5Ag}{\includegraphics[width = 2cm]{img/youtube}}
\end{minipage}
\vspace{1ex}
%% --------------------------------------------------------------------



'Was sollen wir hier dann noch zeichnen?!', entfährt es Nilufar und schaut dabei Alex an. 'Wir sollen eine Normalverteilung mit einem Mittelwert von $\bar{y}_1 = 4$ und einer Standardabweichung von $s_1 = 4$ zeichnen. Sowie eine weitere Normalverteilung mit einem Mittelwert von $\bar{y}_2 = 3$ und einer Standardabweichung von $s_2 = 4$. Keine Ahnung wie das geht. Darunter sollen dann noch eine Poissonverteilung mit einem Mittelwert von $\lambda_1 = 15$ sowie einer weiteren Poissonverteilung mit einem Mittelwert von $\lambda_2 = 1$ gezeichnet werden.', meint Alex sichtlich genervt und mampft noch ein paar Gummibärchen. Im Hintergrund spielt leise Abba. 'Wirre Geschichte...', merkt Nilufar nickend an. Die beiden schauen angestrengt auf die leeren Flächen für die Abbildungen. Alex und die Erwartung machen die Suche nach der Lösung nicht einfacher.\\




{\centering \includegraphics[width=\maxwidth]{img/histogram-01-1} 

}




Jetzt brauchen Nilufar und Alex Ihre Hilfe bei der Abschätzung einer Verteilung um ihre Arbeit dann in diesem Semester noch abschließen zu können.


\begin{enumerate}
\item Skizzieren Sie die zwei Normalverteilungen und zwei Poissonverteilungen! \textbf{(4 Punkte)}
\item Achten Sie auf die entsprechende Skalierung in den jeweiligen Abbildungen! \textbf{(2 Punkte)}
\item Ergänzen Sie unter \underline{einer} Normalverteilung den entsprechenden Boxplot! \textbf{(1 Punkt)}
\item Ergänzen Sie unter \underline{einer} Poissonverteilung den entsprechenden Boxplot! \textbf{(1 Punkt)}
\item Geben Sie ein Beispiel für ein Outcome $y$, welches einer Normalverteilung folgt! \textbf{(1 Punkt)}
\item Geben Sie ein Beispiel für ein Outcome $y$, welches einer Poissonverteilung folgt! \textbf{(1 Punkt)}
\end{enumerate} 
\clearpage
% -----------------------------------------------------------------------
\part{Statistisches Testen \& statistische Testtheorie}
% -----------------------------------------------------------------------  

\section{Aufgabe \hfill (9 Punkte)}

%% --------------------------------------------------------------------
\hfill\href{https://youtu.be/aHVYuFKTqZs}{\includegraphics[width =
  2cm]{img/youtube}}\\[1Ex]
%% --------------------------------------------------------------------

Grundlage des statistischen Testen ist das Verst{\"a}ndnis von der
Grundgesamtheit (eng. \textit{population} oder \textit{ground truth}) und
der experimentellen Stichprobe (eng. \textit{sample}). 

\begin{enumerate}
\item Nennen Sie das statistische Verfahren und zwei konkrete Beispiele zur
  Durchf{\"u}hrung um von einer Grundgesamtheit auf eine Stichprobe zu
  gelangen! \textbf{(3 Punkte)}
\item Erkl{\"a}ren Sie den Zusammenhang zwischen Stichprobe und Grundgesamtheit
  an einem Schaubild! Beschriften Sie das Schaubild entsprechend!
  \textit{Nutzen Sie hierf{\"u}r als Veranschaulichung die K{\"o}rpergr{\"o}{\ss}e von
    M{\"a}nnern oder Frauen aus den Gummib{\"a}rchendaten!}  \textbf{(3 Punkte)}
\item Erweitern Sie das Schaubild um die Entstehung von $Pr(D|H_0)$!
  \textit{Nutzen Sie hierf{\"u}r als Veranschaulichung zus{\"a}tzlich die
    Gruppierungsvariable "`Modul"' aus den Gummib{\"a}rchendaten!}  \textbf{(3
    Punkte)}
\end{enumerate} 
\clearpage
% -----------------------------------------------------------------------

\section{Aufgabe \hfill (9 Punkte)}

%% --------------------------------------------------------------------
\hfill\href{https://youtu.be/Ric8ne39DtI}{\includegraphics[width =
  2cm]{img/youtube}}\\[1Ex]
%% --------------------------------------------------------------------




F{\"u}r ein besseres Verst{\"a}ndnis der statistischen Testtheorie, auch
Null-Ritual genannt, kann eine Visualisierung als Kreuztabelle genutzt werden.  

\begin{enumerate}
\item Tragen Sie folgende statistische Fachbegriffe zur statistischen
  Testtheorie korrekt eine selbst erstellte Kreuztabelle ein! \textbf{(3
    Punkte)}
  \begin{center}
  \begin{tabular}{cccc}
  $\alpha$-Fehler & H$_0$ wahr & H$_0$ falsch & $\beta$-Fehler \\
  \end{tabular}
  \end{center}
\item Erg{\"a}nzen Sie Ihre erstellte Kreuztabelle um vier weitere, passende
  Fachbegriffe zur statistischen Testtheorie! \textbf{(2 Punkte)}
\end{enumerate}

Die Entscheidungsfindung durch einen statistischen Test kann auch durch die
Analogie zu einem Feuermelder abgebildet werden. Dabei symbolisiert der
Feuermelder den statistischen Test und es soll getestet werden, ob ein Feuer
ausgebrochen ist.

\begin{enumerate}
  \setcounter{enumi}{2}    
\item In der Analogie des Feuermelders, wie lautet der $\alpha$-Fehler? \textbf{(1 Punkt)}
\item In der Analogie des Feuermelders, wie lautet der $\beta$-Fehler? \textbf{(1 Punkt)}
\item Wenn der Feuermelder einmal pro Tag messen w{\"u}rde, wie oft w{\"u}rde der
  Feuermelder mit einem $\alpha$ von 5\% in einem Jahr Alarm schlagen?
  Begr{\"u}nden Sie Ihre Antwort! \textbf{(2 Punkte)}
\end{enumerate}



 
\clearpage
% -----------------------------------------------------------------------

\section{Aufgabe \hfill (9 Punkte)}

\textit{Geben Sie grunds{\"a}tzlich Formeln und Rechenweg zur L{\"o}sung der
  Teilaufgaben mit an!} \\[1Ex]

%% --------------------------------------------------------------------
\hfill\href{https://youtu.be/32JjH1eyuTU}{\includegraphics[width =
  2cm]{img/youtube}}\\[1Ex]
%% --------------------------------------------------------------------



Abgebildet ist die t-Verteilung unter der Anahme der G{\"u}ltigkeit der
Nullhypothese. \textit{Beachten Sie, dass im Folgenden keine
  numerisch korrekte Darstellung verlangt wird! Es gilt Erkennbarkeit vor
  Genauigkeit!}

\begin{enumerate}
\item Erg{\"a}nzen Sie eine beschriftete $x$-Achse! \textbf{(1 Punkt)}
\item Erg{\"a}nzen Sie "`$\bar{y}_1 = \bar{y}_2$"'! \textbf{(1 Punkt)} 
\item Erg{\"a}nzen Sie "`$A = 95\%$"'! \textbf{(1 Punkt)}
\item Zeichnen Sie $T_{\alpha=5\%}$ in die Abbildung! \textbf{(1 Punkt)} 
\item Zeichnen Sie das Signifikanzniveau $\alpha$ in die Abbildung! Begr{\"u}nden
  Sie Ihre Antwort! \textbf{(2 Punkte)} 
\item Zeichnen Sie $+T_{D}$ in die Abbildung! \textbf{(1
    Punkt)}
\item Zeichnen Sie einen nicht signifikant p-Wert in die Abbildung! Begr{\"u}nden
  Sie Ihre Antwort! \textbf{(2 Punkte)}   
\end{enumerate}



{\centering \includegraphics[width=\maxwidth]{img/statistisches-testen-3-1} 

}


 
\clearpage
% -----------------------------------------------------------------------

\section{Aufgabe \hfill (10 Punkte)}

%% --------------------------------------------------------------------
\hfill\href{https://youtu.be/CN_O4fYPbhs}{\includegraphics[width =
  2cm]{img/youtube}}\\[1Ex]
%% --------------------------------------------------------------------



Sie rechnen einen t-Test f{\"u}r Gruppenvergleiche der Mittelwerte. Sie
sch{\"a}tzen den Unterschied zwischen dem mittleren Befall mit Parasiten zu einer unbehandelten
Kontrolle.

\begin{enumerate}
\item Beschriften Sie die untenstehende Abbildung mit der
  Signifikanzschwelle! Begr{\"u}nden Sie Ihre Antwort! \textbf{(2 Punkte)}
\item Erg{\"a}nzen Sie eine \textit{in den Kontext passende} Relevanzschwelle!
  Begr{\"u}nden Sie Ihre Antwort! \textbf{(2 Punkte)} 
\item Skizieren Sie in die
  untenstehende Abbildung sechs einzelne Konfidenzintervalle (a-f) mit den
  jeweiligen Eigenschaften! \textbf{(6 Punkte)}
  \begin{itemize}
  \item[(a)] Ein signifikantes, relevantes 90\% Konfidenzintervall. 	
  \item[(b)] Ein 95\% Konfidenzintervall mit h{"o}herer Fallzahl $n$ in der Stichprobe als der Rest der 95\% Konfidenzintervalle 	
  \item[(c)] Ein signifikantes, relevantes 95\% Konfidenzintervall 	
  \item[(d)] Ein signifikantes, nicht relevantes 95\% Konfidenzintervall 
  \item[(e)] Ein 95\% Konfidenzintervall mit niedriger Fallzahl $n$ in der Stichprobe als der Rest 95\% der Konfidenzintervalle
  \item[(f)] Ein nicht signifikantes, nicht relevantes 95\% Konfidenzintervall
  \end{itemize}
\end{enumerate}

\begin{center}
  \includegraphics[height = 12cm]{/Users/kruppajo/work/GitHub/exam/question/img/statistisches-testen-04}
\end{center}


 
\clearpage
% -----------------------------------------------------------------------

\section{Aufgabe \hfill (10 Punkte)}

\textit{Geben Sie grunds{\"a}tzlich Formeln und Rechenweg zur L{\"o}sung der
  Teilaufgaben mit an!} \\[1Ex]

%% --------------------------------------------------------------------
\hfill\href{https://youtu.be/FgZmpnEWDag}{\includegraphics[width =
  2cm]{img/youtube}}\\[1Ex]
%% --------------------------------------------------------------------



Beim statistischen Testen gibt es einen Zusammenhang zwischen dem Effekt,
der Streuung sowie der Fallzahl. Gegeben sei die Formel f{\"u}r den Student
t-Test auf den die folgenden {\"U}berlegungen basieren sollen. Welche
Auswirkung hat die {\"A}nderungen der jeweiligen statistischen Ma{\ss}zahl des
Effekts $\Delta$, der Streuung $s$ und der Fallzahl $n$ auf die Teststistik
$T_{D}$, den p-Wert $Pr(D|H_0)$ sowie dem Konfidenzintervall
$KI_{1-\alpha}$?

\begin{enumerate}
\item Visualisieren Sie den Zusammenhang zwischen der Teststatiatik
  $T_{D}$ und dem p-Wert $Pr(D|H_0)$ f{\"u}r sich ver{\"a}ndernde $T_{D}$-Werte!
  \textit{Geben Sie daf{\"u}r ein numerisches Beispiel in dem Sie drei
    $T_{D}$-Werte und deren Einfluss auf den p-Wert vergleichen!}
  \textbf{(3 Punkte)}  
\item  F{\"u}llen Sie die untenstehende Tabelle aus in dem Sie die {\"A}nderung der
  statistischen Ma{\ss}zahlen auf die Teststatistik, den p-Wert sowie das
  Konfidenzintervall in \textit{einem} Wort oder Symbol beschreiben! \textbf{(4 Punkte)}
\begin{center}
  \large
  \begin{tabular}[c]{l|c|c|c|l|c|c|c}
    & $T_{D}$ & $Pr(D|H_0)$ & $KI_{1-\alpha}$ & & $T_{D}$ & $Pr(D|H_0)$ & $KI_{1-\alpha}$\strut\\ 
    \hline
    \textbf{$\Delta\; \uparrow$} & \hspace{1.8cm} & \hspace{1.8cm}  & \hspace{1.8cm} & \textbf{
                                                          $\Delta\; \downarrow$} &
                                                                          \hspace{1.8cm} & \hspace{1.8cm}  & \hspace{1.8cm}\strut\\
    \hline
        \textbf{$s\; \uparrow$} & \hspace{1.8cm} & \hspace{1.8cm}  & \hspace{1.8cm} & \textbf{
                                                          $s\; \downarrow$} &
                                                                          \hspace{1.8cm}
                                                & \hspace{1.8cm}  & \hspace{1.8cm}\strut\\
    \hline
        \textbf{$n\; \uparrow$} & \hspace{1.8cm} & \hspace{1.8cm}  & \hspace{1.8cm} & \textbf{
                                                          $n\; \downarrow$} &
                                                                          \hspace{1.8cm}
                                                & \hspace{1.8cm}  & \hspace{1.8cm}\strut\\
    \hline
  \end{tabular}
\end{center}
\item Visualisieren Sie ein 95\%-iges Konfidenzintervall im Vergleich
  zu einem 90\%-igen Konfidenzintervall! Begr{\"u}nden Sie Ihre Visualisierung anhand der Formel
  des Konfidenzintervalls des t-Tests mathematisch! \textbf{(3 Punkte)} 
\end{enumerate} 
\clearpage
% -----------------------------------------------------------------------
\part{Der Student t-Test, Welch t-Test \& gepaarter t-Test}
% -----------------------------------------------------------------------

\section{Aufgabe \hfill (9 Punkte)}

\textit{Geben Sie grundsätzlich Formeln und Rechenweg zur Lösung der Teilaufgaben mit an!} \\[1Ex]
 

 
%% --------------------------------------------------------------------
\begin{minipage}[t]{0.5\textwidth}
\includegraphics[width = 1.3cm]{/Users/kruppajo/work/GitHub/exam/avatare/Steffen.png}
\end{minipage}
\begin{minipage}[t]{0.5\textwidth}
\hfill
\href{https://youtu.be/eejS2uG4o-M}{\includegraphics[width = 2cm]{img/youtube}}\\[1Ex]
\end{minipage}
\vspace{1ex}
%% --------------------------------------------------------------------



'Der t-Test testet einen normalverteilten Messwert ($Y$).', liest Steffen laut. Das hilft jetzt auch nur bedingt weiter. Wenn die Romantik nicht wäre, ja dann wäre wohl vieles möglich für Steffen! Aber so.. Laut seiner Betreuerin ist zwar ihm Messwert Trockengewicht [kg/ha] normalverteilt, aber wie rechnet er jetzt einen t-Test? Für seinen Projektbericht zum Testen einer neuen technischen Anlage musste er ein Gewächshausexperiment mit Kartoffeln im Oldenburger Land durchführen. Als wäre das nicht schon anstrengend genug gewesen bei dem anspruchsvollen Pilotprojekt mit sehr geringer Fallzahl $(n_1 = n_2 = 3)$. Jetzt soll er auch noch testen, ob die Behandlung Substrattypen ($torf$ und $70p30n$) ein signifikantes Ergebnis liefert. Hm, lecker Oreos und dazu dann im Hintergrund Harry Potter laufen lassen.

\begin{table}[!h]
\centering
\begin{tabular}{cc}
\toprule
treatment & weight\\
\midrule
dose & 19.5\\
ctrl & 22.5\\
ctrl & 21.4\\
dose & 25.8\\
ctrl & 20.7\\
\addlinespace
dose & 13.6\\
\bottomrule
\end{tabular}
\end{table}



Leider kennt sich Steffen mit der Berechnung eines t-Tests überhaupt nicht aus. Deshalb braucht er bei der Berechnung Ihre Hilfe!

\begin{enumerate}
  \item Formulieren Sie das statistische Hypothesenpaar! \textbf{(1 Punkt)}
  \item Bestimmen Sie die Teststatistik $T_{D}$ eines Welch t-Tests! \textbf{(3 Punkte)}
  \item Treffen Sie mit $T_{\alpha = 5\%} = 1.96$ eine Aussage zur Nullhypothese! Begründen Sie Ihre Antwort! \textbf{(2 Punkte)}
  \item Berechnen Sie den Effekt des Welch t-Tests! \textbf{(1 Punkt)}
  \item Formulieren Sie eine Antwort an Steffen über das Ergebnis Ihrer statistischen Analyse! \textbf{(2 Punkte)}
\end{enumerate} 
\clearpage
% -----------------------------------------------------------------------

\section{Aufgabe \hfill (12 Punkte)}

\textit{Geben Sie grundsätzlich Formeln und Rechenweg zur Lösung der Teilaufgaben mit an!} \\[1Ex]
 

 
%% --------------------------------------------------------------------
\begin{minipage}[t]{0.5\textwidth}
\includegraphics[width = 1.3cm]{/Users/kruppajo/work/GitHub/exam/avatare/Jonas.png}
\end{minipage}
\begin{minipage}[t]{0.5\textwidth}
\hfill
\href{https://youtu.be/Cq_rF_z4xOk}{\includegraphics[width = 2cm]{img/youtube}}\\[1Ex]
\end{minipage}
\vspace{1ex}
%% --------------------------------------------------------------------



Das Oldenburger Land, unendliche Weiten. Wir schreiben das Jahr 2024. Dies sind die Abenteuer von Jonas, der mit seiner 1 Mann starken Besatzung 12 Wochen lang unterwegs ist, um neue Welten zu erforschen, neues Leben und neue Zivilisationen. 'Oder nennen wir es Ödnis und Verzweiflung', denkt Jonas. Für seinen Projektbericht ist Jonas ins Nichts gezogen. Eine echte Herausforderung für ihn war schon immer die Erschöpfung gewesen. Ein leidiges Lied. Was macht er nun? Jonas hat ein Feldexperiment mit Kartoffeln durchgeführt. Die Behandlung Lichtstufen ($none$ und $600lm$) wurde an Kartoffeln getestet. Gemessen hat er dann als einen normalverteilten Endpunkt ($Y$) Chlorophyllgehalt (SPAD-502Plus) [SPAD]. Jetzt soll er seiner Betreuerin nach testen, ob die Behandlung Lichtstufen ($none$ und $600lm$) ein signifikantes Ergebnis liefert. Hm..., was entspannendes wäre gut. Aus den Boxen wummert Iron Maiden und sein Mund ist verklebt von Snickers. 'Herrlich', denkt Jonas.

\begin{table}[!h]
\centering
\begin{tabular}{cc}
\toprule
Lichtstufen & Chlorophyllgehalt\\
\midrule
600lm & 43.3\\
none & 24.6\\
none & 12.9\\
none & 28.2\\
600lm & 39.9\\
\addlinespace
600lm & 19.3\\
none & 16.4\\
none & 26.5\\
600lm & 47.4\\
600lm & 23.3\\
\addlinespace
600lm & 20.9\\
none & 15.4\\
600lm & 33.8\\
none & 28.8\\
600lm & 33.5\\
\addlinespace
none & 35.1\\
600lm & -1.0\\
600lm & 17.2\\
\bottomrule
\end{tabular}
\end{table}



Leider kennt sich Jonas mit der Berechnung eines t-Tests überhaupt nicht aus. Deshalb braucht er bei der Berechnung Ihre Hilfe!

\begin{enumerate}
  \item Formulieren Sie die wissenschaftliche Fragestellung! \textbf{(1 Punkt)}
  \item Formulieren Sie das statistische Hypothesenpaar! \textbf{(1 Punkt)}
  \item Bestimmen Sie die Teststatistik $T_{D}$ eines Student t-Tests! \textbf{(3 Punkte)}
\item Treffen Sie mit $T_{\alpha = 5\%} = 1.84$ eine Aussage zur Nullhypothese! Begründen Sie Ihre Antwort! \textbf{(2 Punkte)}
\item Berechnen Sie den Effekt des Student t-Tests! \textbf{(1 Punkt)}
\item Wenn Sie \textit{einen} Unterschied zwischen den Behandlungsgruppen erwarten würden, wie groß wäre dann der Effekt? Begründen Sie Ihre Antwort! \textbf{(2 Punkte)}
\item Formulieren Sie eine Antwort an Jonas über das Ergebnis Ihrer statistischen Analyse! \textbf{(2 Punkte)}
\end{enumerate} 
\clearpage
% -----------------------------------------------------------------------

\section{Aufgabe \hfill (12 Punkte)}

\textit{Geben Sie grundsätzlich Formeln und Rechenweg zur Lösung der Teilaufgaben mit an!} \\[1Ex]
 

 
%% --------------------------------------------------------------------
\begin{minipage}[t]{0.5\textwidth}
\includegraphics[width = 1.3cm]{/Users/kruppajo/work/GitHub/exam/avatare/Paula.png}
\end{minipage}
\begin{minipage}[t]{0.5\textwidth}
\hfill
\href{https://youtu.be/TbSEOMCQYl4}{\includegraphics[width = 2cm]{img/youtube}}\\[1Ex]
\end{minipage}
\vspace{1ex}
%% --------------------------------------------------------------------



Der Teuteburgerwald, unendliche Weiten. Wir schreiben das Jahr 2024. Dies sind die Abenteuer von Paula, die mit ihrer 1 Frau starken Besatzung 12 Wochen lang unterwegs ist, um neue Welten zu erforschen, neues Leben und neue Zivilisationen. 'Oder nennen wir es Ödnis und Verzweiflung', denkt Paula. Für ihrer Hausarbeit ist Paula ins Nichts gezogen. Paula und der Perfektionismus, eine unendliche Geschichte mit kniffeligen Wendungen. Was macht sie nun? Paula hat ein Feldexperiment mit Kartoffeln durchgeführt. Die Behandlung Substrattypen ($torf$ und $70p30n$) wurde an Kartoffeln getestet. Gemessen hat sie dann als ein normalverteiltes Outcome ($Y$) Trockengewicht [kg/ha]. Jetzt soll sie ihrem Betreuer nach testen, ob die Behandlung Substrattypen ($torf$ und $70p30n$) ein signifikantes Ergebnis liefert. Hm..., was entspannendes wäre gut. 'Hm...', Smarties und White Lies. Das ist und bleibt die beste Kombination zum Nachdenken für Paula.

\begin{table}[!h]
\centering
\begin{tabular}{cc}
\toprule
Substrattypen & Trockengewicht\\
\midrule
70p30n & 47.5\\
torf & 31.5\\
70p30n & 62.8\\
70p30n & 47.0\\
torf & 49.1\\
\addlinespace
70p30n & 60.6\\
torf & 49.1\\
70p30n & 33.2\\
torf & 45.9\\
torf & 72.3\\
\addlinespace
torf & 42.3\\
70p30n & 47.1\\
70p30n & 50.4\\
70p30n & 42.1\\
torf & 51.9\\
\addlinespace
torf & 58.5\\
70p30n & 38.6\\
torf & 59.5\\
torf & 36.3\\
70p30n & 47.4\\
\addlinespace
torf & 56.9\\
70p30n & 38.7\\
\bottomrule
\end{tabular}
\end{table}



Leider kennt sich Paula mit der Berechnung eines t-Tests überhaupt nicht aus. Deshalb braucht sie bei der Berechnung Ihre Hilfe!

\begin{enumerate}
  \item Formulieren Sie die wissenschaftliche Fragestellung! \textbf{(1 Punkt)}
  \item Formulieren Sie das statistische Hypothesenpaar! \textbf{(1 Punkt)}
  \item Bestimmen Sie die Teststatistik $T_{D}$ eines  Welch t-Tests! \textbf{(3 Punkte)}
  \item Treffen Sie mit $T_{\alpha = 5\%} = 1.84$ eine Aussage zur Nullhypothese! Begründen Sie Ihre Antwort! \textbf{(2 Punkte)}
\item Berechnen Sie das 99\% Konfidenzintervall. Welche Annahmen haben Sie getroffen? \textbf{(2 Punkte)}
\item Nennen Sie den statistischen Grund, warum Sie sich zwischen einem Student t-Test und einem Welch t-Test entscheiden müssen! \textbf{(1 Punkt)}
\item Formulieren Sie eine Antwort an Paula über das Ergebnis Ihrer statistischen Analyse! \textbf{(2 Punkte)}
\end{enumerate} 
\clearpage
% -----------------------------------------------------------------------

\section{Aufgabe \hfill (11 Punkte)}

\textit{Geben Sie grundsätzlich Formeln und Rechenweg zur Lösung der Teilaufgaben mit an!} \\[1Ex]
 

 
%% --------------------------------------------------------------------
\begin{minipage}[t]{0.5\textwidth}
\includegraphics[width = 1.3cm]{/Users/kruppajo/work/GitHub/exam/avatare/Alex.png}\hspace{-4mm}\includegraphics[width = 1.3cm]{/Users/kruppajo/work/GitHub/exam/avatare/Paula.png}
\end{minipage}
\begin{minipage}[t]{0.5\textwidth}
\hfill
\href{https://youtu.be/QR90zyn0Iz8}{\includegraphics[width = 2cm]{img/youtube}}
\end{minipage}
\vspace{1ex}
%% --------------------------------------------------------------------



Alex und Paula haben sich dazu entschieden zusammenzuarbeiten. Das sollte alles etwas einfacher machen. Jeder hat zwar ein getrenntes Themenfeld aber den Hauptversuch machen beide gemeinsam. Das hat sich schonmal als gut Idee soweit herausgestellt. In einer Hausarbeit sollen beide herausfinden, ob es einen Zusammenhang zwischen Jäten ($Aussaat$ und $Ernte$) und Trockengewicht [kg/ha] gibt. Die Besonderheit ist hierbei, dass die Messungen an der gleichen Beobachtung stattfinden. Beide messen also zweimal an den gleichen Kartoffeln. Hier muss dann wohl auf einen normalverteilten Endpunkt ($Y$) ein gepaarter t-Test gerechnet werden. Alex schaut etwas flehentlich zu Paula. Wenn die Gefälligkeit nicht wäre, ja dann wäre wohl vieles möglich für Alex! Aber so... Steffen denkt derweil angestrengt an White Lies und wippt leicht mit dem Fuß.

\begin{table}[!h]
\centering
\begin{tabular}{ccc}
\toprule
ID & treatment & freshmatter\\
\midrule
2 & Ernte & 48.0\\
9 & Ernte & 32.6\\
5 & Ernte & 48.8\\
1 & Aussaat & 32.0\\
7 & Aussaat & 29.6\\
\addlinespace
2 & Aussaat & 36.2\\
1 & Ernte & 65.3\\
4 & Ernte & 67.4\\
8 & Ernte & 40.9\\
10 & Ernte & 28.8\\
\addlinespace
10 & Aussaat & 34.2\\
9 & Aussaat & 31.5\\
5 & Aussaat & 28.6\\
3 & Ernte & 67.1\\
3 & Aussaat & 36.7\\
\addlinespace
4 & Aussaat & 35.1\\
8 & Aussaat & 43.5\\
6 & Ernte & 33.8\\
7 & Ernte & 36.4\\
6 & Aussaat & 41.2\\
\addlinespace
11 & Aussaat & 35.1\\
\bottomrule
\end{tabular}
\end{table}



Leider kennen sich Alex und Paula mit der Berechnung eines gepaarten t-Tests überhaupt nicht aus. Deshalb brauchen sie beide bei der Berechnung Ihre Hilfe!

\begin{enumerate}
  \item Formulieren Sie die wissenschaftliche Fragestellung! \textbf{(1 Punkt)}
  \item Formulieren Sie das statistische Hypothesenpaar! \textbf{(1 Punkt)}
  \item Bestimmen Sie die Teststatistik $T_{D}$ eines gepaarten t-Tests! \textbf{(3 Punkte)}
  \item Treffen Sie mit $T_{\alpha = 5\%} = 2.68$ eine Aussage zur Nullhypothese! Begründen Sie Ihre Antwort! \textbf{(2 Punkte)}
\item Schätzen Sie den $p$-Wert des gepaarten t-Tests ab! Begründen Sie Ihre Antwort mit einer Skizze! \textbf{(2 Punkte)}
\item Formulieren Sie eine Antwort an Alex über das Ergebnis Ihrer statistischen Analyse! \textbf{(2 Punkte)}
\end{enumerate}


 
\clearpage
% -----------------------------------------------------------------------

\section{Aufgabe \hfill (10 Punkte)}

\textit{Geben Sie grundsätzlich Formeln und Rechenweg zur Lösung der Teilaufgaben mit an!} \\[1Ex]
 

 
%% --------------------------------------------------------------------
\begin{minipage}[t]{0.5\textwidth}
\includegraphics[width = 1.3cm]{/Users/kruppajo/work/GitHub/exam/avatare/Jonas.png}\hspace{-4mm}\includegraphics[width = 1.3cm]{/Users/kruppajo/work/GitHub/exam/avatare/Steffen.png}\hspace{-4mm}\includegraphics[width = 1.3cm]{/Users/kruppajo/work/GitHub/exam/avatare/Tina.png}
\end{minipage}
\begin{minipage}[t]{0.5\textwidth}
\hfill
\href{https://youtu.be/exDo7AyHl4Q}{\includegraphics[width = 2cm]{img/youtube}}
\end{minipage}
\vspace{1ex}
%% --------------------------------------------------------------------



'Mit dem R Paket \texttt{\{emmeans\}} können wir gleich die Gruppenvergleiche rechnen und uns das \textit{compact letter displac}' ausgeben lassen!', verkündet Jonas sichtlich stolz. Ein paar Mal hat sie schon die Erschöpfung gehindert weiterzumachen. 'Nach Meinung der Betreuerin soll es aber nur erstmal ein t-Test sein. Und die Ausgabe ist schon wirr genug.', merkt Tina an. Tina und Steffen sind bei Jonas um sich in \Rlogo helfen zu lassen. Im Hintergrund wummert Iron Maiden. Steffen streichelt zur Beruhigung das Meerschweinchen von Jonas. Die beiden waren 3 Monate im Wendland um einen Versuch mit Kartoffeln in einem Feldexperiment durchzuführen. Ziel war es das Outcome ($Y$) Proteingehalt [g/kg] zu bestimmen. Jonas überlegt, ob er die beiden nicht noch auf den Film \textit{Mission Impossible} einlädt oder dann doch lieber raus geht um zu Schwimmen? Vielleicht will ja Steffen mit. Besser als der Film.

\begin{knitrout}
\definecolor{shadecolor}{rgb}{0.969, 0.969, 0.969}\color{fgcolor}\begin{kframe}
\begin{verbatim}
## 
## 	Two Sample t-test
## 
## data:  Proteingehalt by Lüftungssystemen
## t = -1.4378, df = 19, p-value = 0.1668
## alternative hypothesis: true  is not equal to [condensed]
## 95 percent confidence interval:
##  -11.765160   2.183342
## sample estimates:
##    mean in group ctrl mean in group tornado 
##              32.30000              37.09091
\end{verbatim}
\end{kframe}
\end{knitrout}

Helfen Sie Jonas bei der Interpretation des t-Tests! Sonst geht es auch für Tina und Steffen nicht weiter.
  
\begin{enumerate}
  \item Formulieren Sie die wissenschaftliche Fragestellung! \textbf{(1 Punkt)}
  \item Formulieren Sie das statistische Hypothesenpaar! \textbf{(1 Punkt)}
\item Liegt ein signifikanter Unterschied zwischen den Gruppen vor? Begründen Sie Ihre Antwort! \textbf{(2 Punkte)}
\item Skizzieren Sie eine Abbildung in der Sie $T_{D}$, $Pr(D|H_0)$, $A=0.95$, sowie $T_{\alpha=5\%} = |2.09|$ einzeichnen! \textbf{(4 Punkte)}
\item Beschriften Sie die Abbildung! \textbf{(1 Punkt)}  
\item Berechnen Sie den Effekt des t-Tests! \textbf{(1 Punkt)}
\end{enumerate} 
\clearpage
% -----------------------------------------------------------------------

\section{Aufgabe \hfill (10 Punkte)}

\textit{Geben Sie grundsätzlich Formeln und Rechenweg zur Lösung der Teilaufgaben mit an!} \\[1Ex]
 

 
%% --------------------------------------------------------------------
\begin{minipage}[t]{0.5\textwidth}
\includegraphics[width = 1.3cm]{/Users/kruppajo/work/GitHub/exam/avatare/Mark.png}\hspace{-4mm}\includegraphics[width = 1.3cm]{/Users/kruppajo/work/GitHub/exam/avatare/Paula.png}\hspace{-4mm}\includegraphics[width = 1.3cm]{/Users/kruppajo/work/GitHub/exam/avatare/Yuki.png}
\end{minipage}
\begin{minipage}[t]{0.5\textwidth}
\hfill
\href{https://youtu.be/wJqsNV1hOW8}{\includegraphics[width = 2cm]{img/youtube}}
\end{minipage}
\vspace{1ex}
%% --------------------------------------------------------------------



'Mit dem R Paket \texttt{\{emmeans\}} können wir gleich die Gruppenvergleiche rechnen und uns das \textit{compact letter displac}' ausgeben lassen!', verkündet Paula sichtlich stolz. Ein paar Mal hat sie schon der Perfektionismus gehindert weiterzumachen. 'Nach Meinung des Betreuers soll es aber nur erstmal ein t-Test sein. Und die Ausgabe ist schon wirr genug.', merkt Mark an. Yuki und Mark sind bei Paula um sich in \Rlogo helfen zu lassen. Im Hintergrund wummert White Lies. Mark streichelt zur Beruhigung die Ratte von Paula. Die beiden waren 2 Monate im Emsland um einen Versuch mit Kartoffeln in einem Gewächshausexperiment durchzuführen. Ziel war es das Outcome ($Y$) Proteingehalt [g/kg] zu bestimmen. Paula überlegt, ob sie die beiden nicht noch auf den Film \textit{Jagd auf roter Oktober} einlädt oder dann doch lieber raus geht um zu Fechten? Vielleicht will ja Mark mit. Besser als der Film.

\begin{knitrout}
\definecolor{shadecolor}{rgb}{0.969, 0.969, 0.969}\color{fgcolor}\begin{kframe}
\begin{verbatim}
## 
## 	Two Sample t-test
## 
## data:  Proteingehalt by Substrattypen
## t = 4.6824, df = 16, p-value = 0.0002497
## alternative hypothesis: true  is not equal to [condensed]
## 95 percent confidence interval:
##   5.171643 13.728357
## sample estimates:
##   mean in group torf mean in group 70p30n 
##                47.10                37.65
\end{verbatim}
\end{kframe}
\end{knitrout}

Helfen Sie Paula bei der Interpretation des t-Tests! Sonst geht es auch für Yuki und Mark nicht weiter.

\begin{enumerate}
  \item Formulieren Sie die wissenschaftliche Fragestellung! \textbf{(1 Punkt)}
  \item Formulieren Sie das statistische Hypothesenpaar! \textbf{(1 Punkt)}
\item Liegt ein signifikanter Unterschied zwischen den Gruppen vor? Begründen Sie Ihre Antwort! \textbf{(2 Punkte)}
\item Skizieren Sie das sich ergebende 95\% Konifidenzintervall! \textbf{(2 Punkte)}
\item Beschriften Sie die Abbildung und das 95\% Konfidenzintervall entsprechend! \textbf{(2 Punkte)}  
\item Interpretieren Sie den Effekt des 95\% Konifidenzintervalls! \textbf{(2 Punkte)}
\end{enumerate} 
\clearpage
% -----------------------------------------------------------------------

\section{Aufgabe \hfill (9 Punkte)}

\textit{Geben Sie grundsätzlich Formeln und Rechenweg zur Lösung der Teilaufgaben mit an!} \\[1Ex]
 

 
%% --------------------------------------------------------------------
\begin{minipage}[t]{0.5\textwidth}
\includegraphics[width = 1.3cm]{/Users/kruppajo/work/GitHub/exam/avatare/Nilufar.png}\hspace{-4mm}\includegraphics[width = 1.3cm]{/Users/kruppajo/work/GitHub/exam/avatare/Paula.png}\hspace{-4mm}\includegraphics[width = 1.3cm]{/Users/kruppajo/work/GitHub/exam/avatare/Tina.png}
\end{minipage}
\begin{minipage}[t]{0.5\textwidth}
\hfill
\href{https://youtu.be/w62HJlbN28U}{\includegraphics[width = 2cm]{img/youtube}}
\end{minipage}
\vspace{1ex}
%% --------------------------------------------------------------------



'Wir sind uns relativ sicher, dass unser Messwert Trockengewicht [kg/ha] ist!', ruft Nilufar wild gestikulierend. Nilufar wäre mehr präsent, wenn es die Erwartung nicht gäbe. Als würde sowas die Ausgabe von \Rlogo interessieren. Nilufar und Paula sind in einem Cafè mit Tina um sich Hilfe von ihr in \Rlogo zu holen. Während Tina Kirschstreuselkuchen und Katjes mampft, versuchen die Nilufar und Paula ihren Versuch im Teuteburgerwald mit Kartoffeln in einem Versuch in einer Klimakammer zu erklären. Tina hofft insgeheim, dass die \Rlogo Ausgabe des t-Tests ihr mehr Informationen liefert. Eigentlich würde sie dann doch lieber raus um zu Boxen vielleicht mit Paula?

\begin{knitrout}
\definecolor{shadecolor}{rgb}{0.969, 0.969, 0.969}\color{fgcolor}\begin{kframe}
\begin{verbatim}
## 
## 	Two Sample t-test
## 
## data:  Trockengewicht by Lüftungssystemen
## t = -4.1293, df = 14, p-value = 0.001022
## alternative hypothesis: true  is not equal to [condensed]
## 95 percent confidence interval:
##  -17.420057  -5.510102
## sample estimates:
##    mean in group ctrl mean in group tornado 
##              27.07778              38.54286
\end{verbatim}
\end{kframe}
\end{knitrout}

Helfen Sie Tina bei der Interpretation des t-Tests! Sonst geht es auch für Nilufar und Paula nicht weiter.
  
\begin{enumerate}
  \item Formulieren Sie die wissenschaftliche Fragestellung! \textbf{(1 Punkt)}
  \item Formulieren Sie das statistische Hypothesenpaar! \textbf{(1 Punkt)}
\item Liegt ein signifikanter Unterschied zwischen den Gruppen vor?
  Begründen Sie Ihre Antwort! \textbf{(2 Punkte)}
\item Skizieren Sie die sich ergebenden Boxplot!
  Welche Annahmen an die Daten haben Sie getroffen? Begründen Sie Ihre
  Antwort! \textbf{(2 Punkte)} 
\item Skizieren Sie die sich ergebenden Barplots! \textbf{(2 Punkte)}
\item Berechnen Sie den Effekt des t-Tests! \textbf{(1 Punkt)}
\end{enumerate}
 
\clearpage
% -----------------------------------------------------------------------

\section{Aufgabe \hfill (10 Punkte)}

\textit{Geben Sie grundsätzlich Formeln und Rechenweg zur Lösung der Teilaufgaben mit an!} \\[1Ex]
 

 
%% --------------------------------------------------------------------
\begin{minipage}[t]{0.5\textwidth}
\includegraphics[width = 1.3cm]{/Users/kruppajo/work/GitHub/exam/avatare/Nilufar.png}\hspace{-4mm}\includegraphics[width = 1.3cm]{/Users/kruppajo/work/GitHub/exam/avatare/Paula.png}
\end{minipage}
\begin{minipage}[t]{0.5\textwidth}
\hfill
\href{https://youtu.be/kHmfEmU6lrk}{\includegraphics[width = 2cm]{img/youtube}}
\end{minipage}
\vspace{1ex}
%% --------------------------------------------------------------------



Es gibt ja immer die Möglichkeit sich Hilfe zu holen. Das geht natürlich auch immer in einer Hausarbeit. Deshalb arbeiten Paula und Nilufar gemeinsam an einer Hausarbeit. Das macht dann auch die Analyse ihres Hauptversuches einfacher. Zwar hat jeder von ihnen noch ein Subthema, aber auch da kann man sich ja helfen. In dem Hauptversuch wurde Folgendes von den beiden gemacht. Paula und Nilufar haben sich Kartoffeln angeschaut. Dabei geht um Zusammenhang zwischen Jäten ($Aussaat$ und $Ernte$) und Proteingehalt [g/kg]. Jetzt sollen beide einen gepaarten t-Test rechnen. Leider kennen sich beide nicht sehr gut in \Rlogo aus.

\begin{knitrout}
\definecolor{shadecolor}{rgb}{0.969, 0.969, 0.969}\color{fgcolor}\begin{kframe}
\begin{verbatim}
## 
## 	Paired t-test
## 
## data:  Proteingehalt by Jäten
## t = 4.081, df = 8, p-value = 0.003529
## alternative hypothesis: true  is not equal to [condensed]
## 95 percent confidence interval:
##   8.360466 30.083979
## sample estimates:
## mean difference 
##        19.22222
\end{verbatim}
\end{kframe}
\end{knitrout}

Jetzt brauchen Paula und Nilufar Ihre Hilfe bei der Berechnung eines gepaarten t-Tests in \Rlogo um ihre Arbeit dann in diesem Semester noch abschließen zu können.

\begin{enumerate}
  \item Formulieren Sie die wissenschaftliche Fragestellung! \textbf{(1 Punkt)}
  \item Formulieren Sie das statistische Hypothesenpaar! \textbf{(1 Punkt)}
\item Liegt ein signifikanter Unterschied zwischen den Gruppen vor?
  Begründen Sie Ihre Antwort! \textbf{(2 Punkte)}
\item Skizzieren Sie das sich ergebende 95\% Konfidenzintervall! \textbf{(2 Punkte)}
\item Interpretieren Sie den Effekt des gepaarten t-Tests! \textbf{(2 Punkte)}
\item Skizzieren Sie den sich ergebenden Boxplot der Differenzen! Welche Annahmen an die Daten haben Sie getroffen? Begründen Sie Ihre Antwort! \textbf{(2 Punkte)} 
\end{enumerate}
 
\clearpage
% -----------------------------------------------------------------------
\part{Die einfaktorielle \& zweifaktorielle ANOVA}
% -----------------------------------------------------------------------

\section{Aufgabe \hfill (11 Punkte)}

\textit{Geben Sie grundsätzlich Formeln und Rechenweg zur Lösung der Teilaufgaben mit an!} \\[1Ex]
 

 
%% --------------------------------------------------------------------
\begin{minipage}[t]{0.5\textwidth}
\includegraphics[width = 1.3cm]{/Users/kruppajo/work/GitHub/exam/avatare/Alex.png}\hspace{-4mm}\includegraphics[width = 1.3cm]{/Users/kruppajo/work/GitHub/exam/avatare/Jonas.png}
\end{minipage}
\begin{minipage}[t]{0.5\textwidth}
\hfill
\href{https://youtu.be/kHmfEmU6lrk}{\includegraphics[width = 2cm]{img/youtube}}
\end{minipage}
\vspace{1ex}
%% --------------------------------------------------------------------



'Als erstes visualiseren wir unsere Daten und dann können wir schon abschätzen, ob unser Gruppenvergleich in der ANOVA signifikant werden würde?', Alex schaut Jonas fragend an und hofft auf eine positive Regung im Gesicht. Wird aber enttäuscht. Jonas tut sich auch sehr schwer mit der einfaktoriellen ANOVA. Beide waren im Wendland um ein Freilandversuch mit Kartoffeln durchzuführen. Dabei ging es herauszufinden, ob es einen Zusammenhang zwischen der Behandlung Lüftungssystemen und Folientunneln ($ctrl$, $storm$ und $tornado$) und dem Messwert Frischegewicht [kg/ha] gibt.

\begin{knitrout}
\definecolor{shadecolor}{rgb}{0.969, 0.969, 0.969}\color{fgcolor}\begin{table}[!h]
\centering
\begin{tabular}{cc}
\toprule
Lüftungssystemen & Frischegewicht\\
\midrule
storm & 33\\
ctrl & 26\\
ctrl & 24\\
tornado & 33\\
ctrl & 24\\
\addlinespace
storm & 35\\
ctrl & 25\\
tornado & 33\\
storm & 35\\
ctrl & 25\\
\addlinespace
storm & 35\\
tornado & 34\\
ctrl & 26\\
tornado & 35\\
tornado & 35\\
\addlinespace
storm & 35\\
\bottomrule
\end{tabular}
\end{table}

\end{knitrout}

Leider kennen sich Alex und Jonas mit Darstellung einer einfaktoriellen ANOVA überhaupt nicht aus. Deshalb brauchen beide bei der Erstellung Ihre Hilfe! 

\begin{enumerate}
\item Erstellen  Sie  eine  Visualisierung  der  Datentabelle! Beschriften  Sie  die  Abbildung! \textbf{(2 Punkte)}
\item Benennen Sie die Visualisierung mit dem korrekten, statistischen Fachbegriff! \textbf{(1 Punkt)}
\item Zeichnen Sie folgende statistischen Maßzahlen passend ein! 
  \begin{itemize}
  \item Globale Mittelwert: $\beta_0$ \textbf{(1 Punkt)}
  \item Mittelwerte der einzelnen Behandlungsstufen: $\bar{y}_{0.5}$, $\bar{y}_{1.5}$ und $\bar{y}_{2.5}$ \textbf{(1 Punkt)}
  \item Mittelwertsdifferenz der einzelnen Behandlungsstufen: $\beta_{0.5}$, $\beta_{1.5}$ und $\beta_{2.5}$ \textbf{(1 Punkt)}
  \item Residuen oder Fehler: $\epsilon$ \textbf{(1 Punkt)}
  \end{itemize}
\item Liegt ein \textit{vermutlicher} signifikanter Unterschied vor? Begründen Sie Ihre Antwort! \textbf{(2 Punkte)}
\item Schätzen Sie die Effekte der Behandlungsstufen! \textbf{(2 Punkte)}
\end{enumerate}
 
\clearpage
% -----------------------------------------------------------------------

\section{Aufgabe \hfill (9 Punkte)}

\textit{Geben Sie grundsätzlich Formeln und Rechenweg zur Lösung der Teilaufgaben mit an!} \\[1Ex]
 

 
%% --------------------------------------------------------------------
\begin{minipage}[t]{0.5\textwidth}
\includegraphics[width = 1.3cm]{/Users/kruppajo/work/GitHub/exam/avatare/Alex.png}\hspace{-4mm}\includegraphics[width = 1.3cm]{/Users/kruppajo/work/GitHub/exam/avatare/Mark.png}
\end{minipage}
\begin{minipage}[t]{0.5\textwidth}
\hfill
\href{https://youtu.be/IhecxMcCENY}{\includegraphics[width = 2cm]{img/youtube}}
\end{minipage}
\vspace{1ex}
%% --------------------------------------------------------------------



'Uff... die einfaktorielle ANOVA. Und wie füllen wir jetzt die Tabelle der ANOVA aus und schauen, ob da was signifikant ist?', Mark hebt die Augenbraue. 'Das ist eine sehr gute Frage. Ich glaube man kann alles in der Tabelle relativ einfach mit wenigen Informationen berechnen.', meint Alex dazu. Mark hatte sich in ein Feldexperiment verschiedene Kartoffeln angeschaut. Dabei ging es herauszufinden, ob es einen Zusammenhang zwischen der Behandlung Lüftungssystemen und Folientunneln ($ctrl$, $storm$, $thunder$ und $tornado$) und dem Messwert Proteingehalt [g/kg] gibt.



\vspace{1ex}

Leider kennen sich Mark und Alex mit Berechnung einer einfaktoriellen ANOVA überhaupt nicht aus. Deshalb brauchen beide bei der Erstellung Ihre Hilfe! 

\begin{enumerate}
  \item Formulieren Sie die wissenschaftliche Fragestellung! \textbf{(1 Punkt)}
  \item Formulieren Sie das statistische Hypothesenpaar! \textbf{(1 Punkt)}
\item Füllen Sie die unterstehende einfaktorielle ANOVA Ergebnistabelle aus! \textbf{(3 Punkte)}
\end{enumerate}

\vspace{1Ex}

\begin{center}
  \Large
  \begin{tabular}{lccccp{3cm}}
\toprule
     & \textbf{Df} & \textbf{Sum Sq} & \textbf{Mean Sq} & \textbf{F value} & \textbf{Pr(>F)} \strut\\
    \midrule
   \textbf{Lüftungssystemen}  & 3 & 4318.69 &  &  &  \strut\\
   \textbf{error}  & 23 &  &  &  &  \strut\\
   \textbf{Total}  & 26 & 5014.67 &  &  &  \strut\\
\bottomrule
  \end{tabular}
\end{center}

\vspace{1Ex}

\begin{enumerate}
  \setcounter{enumi}{3}
\item Schätzen Sie den p-Wert der Tabelle mit $F_{\alpha = 5\%} = 3.03$ ab. Begründen Sie Ihre Antwort! \textbf{(2 Punkte)}
\item Berechen Sie den Effektschätzer $\eta^2$. Was sagt Ihnen der Wert von $\eta^2$ aus? \textbf{(2 Punkte)}
\end{enumerate}



 
\clearpage
% -----------------------------------------------------------------------

\section{Aufgabe \hfill (12 Punkte)}

\textit{Geben Sie grundsätzlich Formeln und Rechenweg zur Lösung der Teilaufgaben mit an!} \\[1Ex]
 

 
%% --------------------------------------------------------------------
\begin{minipage}[t]{0.5\textwidth}
\includegraphics[width = 1.3cm]{/Users/kruppajo/work/GitHub/exam/avatare/Jessica.png}\hspace{-4mm}\includegraphics[width = 1.3cm]{/Users/kruppajo/work/GitHub/exam/avatare/Tina.png}
\end{minipage}
\begin{minipage}[t]{0.5\textwidth}
\hfill
\href{https://youtu.be/49hvImMwVyE}{\includegraphics[width = 2cm]{img/youtube}}
\end{minipage}
\vspace{1ex}
%% --------------------------------------------------------------------



'Als erstes bauen wir uns aus unsere Daten die ANOVA Tabelle dann sehen wir schon, ob unser Gruppenvergleich in der ANOVA signifikant ist.', Jessica schaut Tina fragend an und hofft auf eine positive Regung im Gesicht. Wird aber enttäuscht. Tina tut sich auch sehr schwer mit der einfaktoriellen ANOVA. Nun möchte erstmal ihre Betreuung der Arbeit eine ANOVA Tabelle sehen. Was immer da auch drin zu erkennen sein mag. Beide waren im Wendland um ein Freilandversuch mit Kartoffeln durchzuführen. Dabei ging es herauszufinden, ob es einen Zusammenhang zwischen der Behandlung Bewässerungstypen ($ctrl$, $low$, $mid$ und $high$) und dem Messwert Frischegewicht [kg/ha] gibt.



\vspace{1ex}

Leider kennen sich Jessica und Tina mit Berechnung einer einfaktoriellen ANOVA überhaupt nicht aus. Deshalb brauchen beide bei der Erstellung Ihre Hilfe! 

\begin{enumerate}
  \item Formulieren Sie die wissenschaftliche Fragestellung! \textbf{(1 Punkt)}
  \item Formulieren Sie das statistische Hypothesenpaar! \textbf{(1 Punkt)}
\item Füllen Sie die unterstehende einfaktorielle ANOVA Ergebnistabelle aus! \textbf{(3 Punkte)}
\end{enumerate}

\vspace{1Ex}

\begin{center}
  \Large
  \begin{tabular}{lccccp{3cm}}
\toprule
     & \textbf{Df} & \textbf{Sum Sq} & \textbf{Mean Sq} & \textbf{F value} & \textbf{Pr(>F)} \strut\\
    \midrule
   \textbf{Bewässerungstypen}  & 3 & 101.06 &  &  &  \strut\\
   \textbf{Error}  & 26 & 170.8 &  &  &  \strut\\
\bottomrule
  \end{tabular}
\end{center}

\vspace{1Ex}

\begin{enumerate}
  \setcounter{enumi}{3}
\item Schätzen Sie den p-Wert der Tabelle mit $F_{\alpha = 5\%} = 2.98$ ab. Begründen Sie Ihre Antwort! \textbf{(2 Punkte)}
\item Was bedeutet ein signifikantes Ergebnis in einer einfaktoriellen ANOVA? \textbf{(1 Punkt)}
\item Berechnen Sie \textit{einen} Student t-Test für den \textit{vermutlich} signifikantesten Gruppenvergleich anhand der untenstehenden Tabelle mit $T_{\alpha = 5\%} = 2.03$. Begründen Sie Ihre Auswahl! \textbf{(3 Punkte)}
\end{enumerate}


\begin{knitrout}
\definecolor{shadecolor}{rgb}{0.969, 0.969, 0.969}\color{fgcolor}\begin{table}[!h]
\centering\begingroup\fontsize{11}{13}\selectfont

\begin{tabular}{cccc}
\toprule
\textbf{Bewässerungstypen} & \textbf{Fallzahl (n)} & \textbf{Mittelwert} & \textbf{Standardabweichung}\\
\midrule
ctrl & 7 & 5.00 & 1.73\\
low & 8 & 5.25 & 3.06\\
mid & 8 & 8.62 & 3.20\\
high & 7 & 3.71 & 1.60\\
\bottomrule
\end{tabular}
\endgroup{}
\end{table}

\end{knitrout}


\begin{enumerate}
  \setcounter{enumi}{6}
\item Gegebenen der ANOVA Tabelle war das Ergebnis des Student t-Tests zu erwarten? Begründen Sie Ihre Antwort! \textbf{(2 Punkte)}
\end{enumerate}

 
\clearpage
% -----------------------------------------------------------------------

\section{Aufgabe \hfill (9 Punkte)}

\textit{Geben Sie grundsätzlich Formeln und Rechenweg zur Lösung der Teilaufgaben mit an!} \\[1Ex]
 

 
%% --------------------------------------------------------------------
\begin{minipage}[t]{0.5\textwidth}
\includegraphics[width = 1.3cm]{/Users/kruppajo/work/GitHub/exam/avatare/Mark.png}
\end{minipage}
\begin{minipage}[t]{0.5\textwidth}
\hfill
\href{https://youtu.be/aXvxGC4YLqk}{\includegraphics[width = 2cm]{img/youtube}}\\[1Ex]
\end{minipage}
\vspace{1ex}
%% --------------------------------------------------------------------



Mark schaut sich fragend in der Bibliothek um. Mark hatte gehofft, dass jemand hier sein würde, den er kennt und sich mit \Rlogo auskennt. Wird aber enttäuscht. Mark war im Wendland um ein Freilandversuch mit Kartoffeln durchzuführen. Nun möchte sein Betreuer seinem Projektbericht erstmal eine ANOVA sehen und die Ergebnisse präsentiert bekommen. Dabei ging es herauszufinden, ob es einen Zusammenhang zwischen der Behandlung Düngestufen ($ctrl$, $low$, $mid$ und $high$) und dem Messwert Chlorophyllgehalt (SPAD-502Plus) [SPAD] gibt.

\begin{knitrout}
\definecolor{shadecolor}{rgb}{0.969, 0.969, 0.969}\color{fgcolor}\begin{kframe}
\begin{verbatim}
## Analysis of Variance Table
## 
## Response: Chlorophyllgehalt
##             Df Sum Sq Mean Sq F value    Pr(>F)
## Düngestufen  3 3647.9 1215.97  23.991 7.836e-07
## Residuals   20 1013.7   50.69
\end{verbatim}
\end{kframe}
\end{knitrout}

\vspace{1ex}

Leider kennen sich Mark mit Berechnung einer einfaktoriellen ANOVA überhaupt nicht aus. Deshalb braucht er bei der Erstellung Ihre Hilfe! 

\begin{enumerate}
  \item Formulieren Sie die wissenschaftliche Fragestellung! \textbf{(1 Punkt)}
  \item Formulieren Sie das statistische Hypothesenpaar! \textbf{(1 Punkt)}
\item Interpretieren Sie das Ergebnis der einfaktoriellen ANOVA! \textbf{(2 Punkte)} 
\item Berechnen Sie den Effektschätzer $\eta^2$. Was sagt Ihnen der Wert von $\eta^2$ aus? \textbf{(2 Punkte)}
\item Skizzieren Sie eine Abbildung, der dem obigen Ergebnis der
  einfaktoriellen ANOVA näherungsweise entspricht! \textbf{(3 Punkte)}
\end{enumerate}

 
\clearpage
% -----------------------------------------------------------------------

\section{Aufgabe \hfill (12 Punkte)}

\textit{Geben Sie grundsätzlich Formeln und Rechenweg zur Lösung der Teilaufgaben mit an!} \\[1Ex]
 

 
%% --------------------------------------------------------------------
\begin{minipage}[t]{0.5\textwidth}
\includegraphics[width = 1.3cm]{/Users/kruppajo/work/GitHub/exam/avatare/Yuki.png}
\end{minipage}
\begin{minipage}[t]{0.5\textwidth}
\hfill
\href{https://youtu.be/8Pb2sKUIMyk}{\includegraphics[width = 2cm]{img/youtube}}\\[1Ex]
\end{minipage}
\vspace{1ex}
%% --------------------------------------------------------------------



Yuki steht im Wendland. Und das ist noch langweiliger als es sich anhört. Wäre es nur so spannend wie bei seinen Kommilitonen, die in Almería waren. Ödnis wohin man nur blickt. Oder eben Kartoffeln. Ja, darum geht es in seiner Hausarbeit. Und wäre das nicht noch alles genug, ist sein Experiment auch noch als einen Versuch in einer Klimakammer komplex geraten. Es wurde der Messwert Frischegewicht [kg/ha] mit dem Behandlung Substrattypen ($torf$, $40p60n$, $30p20n$ und $70p30n$) sowie der Behandlung Lüftungssystemen und Folientunneln ($ctrl$, und $tornado$) untersucht. 'Hmpf', denkt Yuki und ruft 'Und jetzt!?' in die Leere.



\vspace{1ex}

Leider kennen sich Yuki mit Berechnung einer zweifaktoriellen ANOVA überhaupt nicht aus. Deshalb braucht er bei der Erstellung Ihre Hilfe! 

\begin{enumerate}
  \item Formulieren Sie die wissenschaftliche Fragestellung! \textbf{(1 Punkt)}
  \item Formulieren Sie das statistische Hypothesenpaar! \textbf{(1 Punkt)}
\item Füllen Sie die unterstehende einfaktorielle ANOVA Ergebnistabelle aus! \textbf{(3 Punkte)}
\end{enumerate}

\vspace{1Ex}

\begin{center}
  \Large
  \begin{tabular}{lccccc}
  \toprule
     & \textbf{Df} & \textbf{Sum Sq} & \textbf{Mean Sq} & \textbf{F value} & \textbf{Pr(>F)} \strut\\
    \midrule
   \textbf{Substrattypen}  & 3 & 112.26 &  &  &  \strut\\
    \textbf{Lüftungssystemen}  & 1 & 646.71 &  &  &  \strut\\
    \textbf{Substrattypen:Lüftungssystemen}  & 3 & 77.27 &  &  &  \strut\\
   \textbf{Error}  & 18 & 245.33 &  &  &  \strut\\
\bottomrule
  \end{tabular}
\end{center}

\vspace{1Ex}

\begin{enumerate}
  \setcounter{enumi}{3}
\item Schätzen Sie den p-Wert der Tabelle ab. Begründen Sie Ihre
  Antwort! \textbf{(3 Punkte)}
\end{enumerate}
  
\begin{center}
    \Large
\begin{tabular}{lc}
  \toprule
     & $\boldsymbol{F_{\alpha = 5\%}}$ \\
\midrule
  \textbf{Substrattypen} & $4.26$ \\
  \textbf{Lüftungssystemen} & $3.40$ \\
  \textbf{Substrattypen:Lüftungssystemen} & $5.23$ \\
  \bottomrule
  \end{tabular}
\end{center}

\begin{enumerate}
  \setcounter{enumi}{4}
\item Was bedeutet ein signifikantes Ergebnis in einer zweifaktoriellen ANOVA? \textbf{(2 Punkte)}
\item Was sagt der Term \textit{Substrattypen:Lüftungssystemen} aus? Interpretieren Sie das Ergebnis! \textbf{(2 Punkte)}
\end{enumerate}
 
\clearpage
% -----------------------------------------------------------------------

\section{Aufgabe \hfill (10 Punkte)}

\textit{Geben Sie grundsätzlich Formeln und Rechenweg zur Lösung der Teilaufgaben mit an!} \\[1Ex]
 

 
%% --------------------------------------------------------------------
\begin{minipage}[t]{0.5\textwidth}
\includegraphics[width = 1.3cm]{/Users/kruppajo/work/GitHub/exam/avatare/Nilufar.png}
\end{minipage}
\begin{minipage}[t]{0.5\textwidth}
\hfill
\href{https://youtu.be/rWTyHXXlYjY}{\includegraphics[width = 2cm]{img/youtube}}\\[1Ex]
\end{minipage}
\vspace{1ex}
%% --------------------------------------------------------------------



'Mit der zweifaktoriellen ANOVA lässt sich die Interaktion zwischen den beiden Behandlungen nachweisen!', ihre Betreuerin scheint die zweifaktoriellen ANOVA zu verstehen. Warum jetzt sie jetzt nochmal alles wiederkäuen muss, wird Nilufar echt nicht so klar. Wenn es doch so klar ist? Nilufar war im Wendland und hatte dort ein Gewächshausexperiment mit Kartoffeln durchgeführt. Die Komune wo sie untergekommen war, war cool gewesen. Nur jetzt muss eben das Experiment fertig ausgewertet werden. Nilufar hatte zwei Behandlungen auf Kartoffeln angewendet. Einmal Lüftungssysteme ($ctrl$, $storm$, $thunder$ und $tornado$) sowie als zweite Behandlung Substrattypen ($torf$, $70p30n$). Gemessen wurde der Messwert ($Y$) Proteingehalt [g/kg]. Jetzt muss das hier zu einem Ende kommen!

\begin{knitrout}
\definecolor{shadecolor}{rgb}{0.969, 0.969, 0.969}\color{fgcolor}\begin{kframe}
\begin{verbatim}
## Analysis of Variance Table
## 
## Response: Proteingehalt
##                               Df Sum Sq Mean Sq F value   Pr(>F)
## Lüftungssysteme                2 374.50 187.250  9.9828 0.001210
## Substrattypen                  1 203.81 203.813 10.8658 0.004014
## Lüftungssysteme:Substrattypen  2   7.15   3.575  0.1906 0.828105
## Residuals                     18 337.63  18.757
\end{verbatim}
\end{kframe}
\end{knitrout}

\vspace{1ex}

Leider kennt sich Nilufar mit Berechnung einer zweifaktoriellen ANOVA überhaupt nicht aus. Deshalb braucht sie bei der Erstellung Ihre Hilfe! 

\begin{enumerate}
  \item Formulieren Sie die wissenschaftliche Fragestellung! \textbf{(1 Punkt)}
  \item Formulieren Sie das statistische Hypothesenpaar! \textbf{(1 Punkt)}
\item Interpretieren Sie das Ergebnis der einfaktoriellen ANOVA! \textbf{(3 Punkte)} 
\item Zeichnen Sie eine Abbildung, der dem obigen Ergebnis der
  zweifaktoriellen ANOVA näherungsweise entspricht! \textbf{(5 Punkte)}
\end{enumerate}
 
\clearpage
% -----------------------------------------------------------------------

\section{Aufgabe \hfill (12 Punkte)}

\textit{Geben Sie grundsätzlich Formeln und Rechenweg zur Lösung der Teilaufgaben mit an!} \\[1Ex]
 

 
%% --------------------------------------------------------------------
\begin{minipage}[t]{0.5\textwidth}
\includegraphics[width = 1.3cm]{/Users/kruppajo/work/GitHub/exam/avatare/Alex.png}
\end{minipage}
\begin{minipage}[t]{0.5\textwidth}
\hfill
\href{https://youtu.be/FjjJXkFJfIY}{\includegraphics[width = 2cm]{img/youtube}}\\[1Ex]
\end{minipage}
\vspace{1ex}
%% --------------------------------------------------------------------



In ein Gewächshausexperiment wurden Kartoffeln mit der Behandlung Lichtstufen ($none$, $200lm$, $400lm$ und $600lm$) sowie der Behandlung Lüftungssystemen und Folientunneln ($ctrl$, und $tornado$) untersucht. Alex schaut konzentriert auf die Formeln der ANOVA und des t-Tests. In seinem Experiment wurde als Messwert Trockengewicht [kg/ha] bestimmt. Eine echte Herausforderung für ihn war schon immer die Gefälligkeit gewesen. Ein leidiges Lied. Dann wäre es nicht noch komplizierter. Was war da jetzt nochmal der Zusammenhang zwischen den beiden statistischen Verfahren? Beide Verfahren haben ja irgendwie etwas miteinander zu tun und seine Betreuerin möchte das jetzt auch noch verstehen. Muss das nicht eigentlich klar sein? Immerhin ist Alex nicht die erste Betreuung einer Abschlussarbeit. Immerhin hat er die beiden Formeln vorliegen. Schon dutzende Male gesehen: Alien. Aber immer noch großartig zusammen mit Gummibärchen.

\begin{graybox}{Gegebene Formeln für $\boldsymbol{F_D}$ und $\boldsymbol{T_D}$}
\begin{center}
  \begin{tabular}{cc}
    $F_{D} = \cfrac{MS_{treatment}}{MS_{error}}$ & $T_{D} = \cfrac{\bar{y}_1 - \bar{y}_2}{s_p \cdot \sqrt{2/n_g}}$\\
  \end{tabular}
\end{center}
\end{graybox}

Leider kennen sich Alex mit dem Zusammenhang zwischen der ANOVA und dem t-Test nicht aus. Deshalb braucht er bei der Erstellung Ihre Hilfe! 

\begin{enumerate}
\item Was testet der t-Test und was testet die ANOVA? Begründen Sie Ihre Antwort! \textbf{(2 Punkte)}
\item Erklären Sie den Zusammenhang zwischen der $F_{D}$ Statistik und $T_{D}$ Statistik! \textbf{(2 Punkte)}
\item Visualisieren Sie in einer 2x2 Tafel den Zusammenhang von $MS_{treatment}$ und $MS_{error}$! \textbf{(2 Punkte)}
\item Beschriften Sie die 2x2 Tafel mit \underline{signifikant} und \underline{nicht signifikant}! Begründen Sie Ihre Antwort! \textbf{(2 Punkte)}
\item Nennen Sie das Minimum der F-Statistik $F_D$! Begründen Sie Ihre Antwort! \textbf{(2 Punkte)}
\item Wenn die F-Statistik $F_D$ minimal ist, welche Aussage erhalten Sie über die Nullhypothese? Begründen Sie Ihre Antwort! \textbf{(2 Punkte)}
\end{enumerate}

 
\clearpage
% -----------------------------------------------------------------------

\section{Aufgabe \hfill (8 Punkte)}

\textit{Geben Sie grundsätzlich Formeln und Rechenweg zur Lösung der Teilaufgaben mit an!} \\[1Ex]
 

 
%% --------------------------------------------------------------------
\begin{minipage}[t]{0.5\textwidth}
\includegraphics[width = 1.3cm]{/Users/kruppajo/work/GitHub/exam/avatare/Steffen.png}
\end{minipage}
\begin{minipage}[t]{0.5\textwidth}
\hfill
\href{https://youtu.be/2qG1Dws0MJo}{\includegraphics[width = 2cm]{img/youtube}}\\[1Ex]
\end{minipage}
\vspace{1ex}
%% --------------------------------------------------------------------



In einen Versuch in einer Klimakammer wurden Kartoffeln mit der Behandlung Lichtstufen ($none$, $200lm$, $400lm$ und $600lm$) sowie der Behandlung Düngestufen ($ctrl$, und $high$) untersucht. Es wurde als Messwert Chlorophyllgehalt (SPAD-502Plus) [SPAD] bestimmt. Jetzt starrt Steffen mit auf die \Rlogo Ausgabe einer zweifaktoriellen ANOVA. Leider starrt seine Betreuerin in der gleichen Art Steffen zurück an. Es liegt anscheinend eine signifikante Interaktion vor. 'Das wird ein langer Nachmmittag.', denkt er sich und kreuselt seinen Mund. 'Und was machen wir jetzt?' entfährt es ihm überrascht entnervt. Immerhin war geht es ja um seine Hausarbeit. Steffen hätte doch nichts mit Kartoffeln machen sollen. Kartoffeln -- was soll das auch bedeutendes sein?

\vspace{1ex}

Leider kennen sich Steffen und seine Betreuerin mit Berechnung einer zweifaktoriellen ANOVA überhaupt nicht aus. Geschweige denn mit der Interpretation einer Interaktion zwischen zwei Behandlungsfaktoren. Deshalb braucht er bei der Erstellung Ihre Hilfe, sonst wird es heute Abend mit seinem Hobby Klemmbausteine nichts mehr! 

\begin{enumerate}
\item Visualisieren Sie folgende Interaktionen zwischen den Behandlungen! Beschriften Sie die Abbildung! \textbf{(4 Punkte)}
\begin{enumerate}
\item \underline{Keine} Interaktion. 
\item Eine \underline{schwache} Interaktion. 
\item Eine \underline{starke} Interaktion. 
\end{enumerate}
\item Erklären Sie den Unterschied zwischen den verschiedenen Interaktionen! \textbf{(2 Punkte)}
\item Welche statistische Maßzahl betrachten Sie für die Bewertung der Interaktion? \textbf{(1 Punkt)}
\item Skizzieren Sie die notwendigen Funktionen in \Rlogo! \textbf{(2 Punkte)} 
\item Wenn eine signifikante Interaktion in den Daten vorliegt, wie ist dann das weitere Vorgehen bei einem Post-hoc-Test? Berücksichtigen Sie auch die Funktion \texttt{emmeans()} in Ihrem Vorgehen! \textbf{(2 Punkte)}
\end{enumerate}

 
\clearpage
% -----------------------------------------------------------------------

\section{Aufgabe \hfill (9 Punkte)}

%% --------------------------------------------------------------------
\hfill\href{https://youtu.be/M9Uhm67ndxM}{\includegraphics[width =
  2cm]{img/youtube}}\\[1Ex]
%% --------------------------------------------------------------------

\textit{Geben Sie grundsätzlich Formeln und Rechenweg zur Lösung der Teilaufgaben mit an!} \\[1Ex]
 

 
%% --------------------------------------------------------------------
\begin{minipage}[t]{0.5\textwidth}
\includegraphics[width = 1.3cm]{/Users/kruppajo/work/GitHub/exam/avatare/Alex.png}
\end{minipage}
\begin{minipage}[t]{0.5\textwidth}
\hfill
\href{https://youtu.be/M9Uhm67ndxM}{\includegraphics[width = 2cm]{img/youtube}}\\[1Ex]
\end{minipage}
\vspace{1ex}
%% --------------------------------------------------------------------



Es ist schon kurz nach fünf und Alex wird langsam nervös. Alex wollte heute Abend noch seine E-Sport Qualifikation schauen. Stattdessen versucht seine Betreuerin die Ausgabe der zweifaktoriellen ANOVA zu visualieren und zu überprüfen, ob es mit der Visualisierung der Daten als Boxplots zusammenpasst. Alex hatte im Wendland einen Versuch in einer Klimakammer mit Kartoffeln durchgeführt. Es gab dabei zwei Behandlungen. Einmal Bewässerungstypen ($ctrl$, $low$, $mid$ und $high$) sowie als zweite Behandlung Genotypen ($AA$, und $BB$). Gemessen wurde der Messwert ($Y$) Trockengewicht [kg/ha]. So kompliziert kann das jetzt doch nicht sein!




Sie rechnen eine einfaktorielle ANOVA mit einem Faktor $f_1$ mit
fünf Leveln. Nachdem Sie die einfaktorielle ANOVA gerechnet
haben, erhalten Sie einen p-Wert von $0.078$ und eine F Statistik mit
$F_{D} = 1.2$. Als Sie sich die Boxplots der Behandlungen anschauen,
stellen Sie fest, dass es eigentlich einen Mittelwertsunterschied zwischen
dem zweiten und ersten Level geben müsste. Die
$IQR$-Bereiche überlappen sich nicht und die Mediane liegen auch weit vom
globalen Mittel entfernt.

\vspace{1ex}

Leider kennen sich Alex mit Berechnung einer zweifaktoriellen ANOVA überhaupt nicht aus. Deshalb braucht er bei der Erstellung Ihre Hilfe! 

\begin{enumerate}
\item Erklären Sie die Annahme der Normalverteilung und die Annahme der Varianzhomogenität für eine ANOVA an einer passenden Abbildung! \textbf{(3 Punkte)}
\item Visualisieren Sie die Berechnung von $F_{D}$ am obigen Beispiel! \textbf{(3 Punkte)}
\item Erklären Sie das Ergebnis der obigen einfaktoriellen ANOVA unter der Berücksichtigung der Annahmen an eine ANOVA! \textbf{(3 Punkte)}
\end{enumerate}

 
\clearpage
% -----------------------------------------------------------------------
\part{Multiple Gruppenvergleiche}
% ----------------------------------------------------------------------- 

\section{Aufgabe \hfill (12 Punkte)}

\textit{Geben Sie grundsätzlich Formeln und Rechenweg zur Lösung der Teilaufgaben mit an!} \\[1Ex]
 

 
%% --------------------------------------------------------------------
\begin{minipage}[t]{0.5\textwidth}
\includegraphics[width = 1.3cm]{/Users/kruppajo/work/GitHub/exam/avatare/Tina.png}\hspace{-4mm}\includegraphics[width = 1.3cm]{/Users/kruppajo/work/GitHub/exam/avatare/Yuki.png}
\end{minipage}
\begin{minipage}[t]{0.5\textwidth}
\hfill
\href{https://youtu.be/kHmfEmU6lrk}{\includegraphics[width = 2cm]{img/youtube}}
\end{minipage}
\vspace{1ex}
%% --------------------------------------------------------------------



Tina und Yuki untersuchen gemeinsam in ihrer Abschlussarbeit den Messwert Chlorophyllgehalt (SPAD-502Plus) [SPAD] in Kartoffeln. Es ist einiges schiefgelaufen, wie es immer so passiert. Hauptsächlich waren es Würmer, auch wenn man erstmal bei dem Messwert nicht unbedingt an Würmer denken würde. Aber das ist eine andere Geschichte. Jetzt wollen Tina und Yuki ihre Ergebnisse nochmal mit einer Studie von Meyer et al. (2021) vergleichen und schauen, ob was ähnliches rauskommt. Angeschaut wurde sich als Behandlung Substrattypen ($kompost$, $torf$, $40p60n$, $30p20n$ und $70p30n$). Es ergab sich dann die folgende Tabelle der rohen p-Werte für die Vergleiche zu Meyer et al. (2021).

\begin{knitrout}
\definecolor{shadecolor}{rgb}{0.969, 0.969, 0.969}\color{fgcolor}\begin{table}[!h]
\centering\begingroup\fontsize{10}{12}\selectfont

\begin{tabular}{ccc}
\toprule
\textbf{Rohen p-Werte} & \textbf{Adjustierte p-Werte} & \textbf{Nullhypothese ablehnen?}\\
\midrule
0.080 &  & \\
0.340 &  & \\
0.002 &  & \\
0.001 &  & \\
0.760 &  & \\
\bottomrule
\end{tabular}
\endgroup{}
\end{table}

\end{knitrout}

Leider kennen sich Tina und Yuki mit der Adjustierung von $p$-Werten und dem Signifikanzniveau $\alpha$ überhaupt nicht aus. Deshalb brauchen die beiden bei der Erstellung Ihre Hilfe!

\begin{enumerate}
  \item Formulieren Sie die wissenschaftliche Fragestellung! \textbf{(1 Punkt)}
  \item Formulieren Sie die statistischen Hypothesen! \textbf{(1 Punkt)}
\item Füllen Sie die Spalte \textit{Adjustierte p-Werte} nach der Bonferoni-Methode aus! \textbf{(2 Punkte)}
\item Entscheiden Sie, ob nach der Adjustierung die Nullhypothese abgelehnt werden kann! Begründen Sie Ihre Antwort! \textbf{(2 Punkte)}
\item Wie ist Ihr Vorgehen, wenn Sie anstatt der $p$-Werte das Signifikanzniveau $\alpha$ adjustieren? \textbf{(2 Punkte)}
\item Erklären Sie warum die $p$-Werte oder das Signifikanzniveau $\alpha$ bei multiplen Vergleichen adjustiert werden müssen! \textbf{(2 Punkte)}
\item Würden Sie die Adjustierung der $p$-Werte oder die Adjustierung des Signifikanzniveaus $\alpha$ vorziehen? Begründen Sie Ihre Antwort! \textbf{(2 Punkte)}
\end{enumerate}


 
\clearpage
% ----------------------------------------------------------------------- 

\section{Aufgabe \hfill (10 Punkte)}

\textit{Geben Sie grundsätzlich Formeln und Rechenweg zur Lösung der Teilaufgaben mit an!} \\[1Ex]
 

 
%% --------------------------------------------------------------------
\begin{minipage}[t]{0.5\textwidth}
\includegraphics[width = 1.3cm]{/Users/kruppajo/work/GitHub/exam/avatare/Alex.png}
\end{minipage}
\begin{minipage}[t]{0.5\textwidth}
\hfill
\href{https://youtu.be/xq29O8qDrg0}{\includegraphics[width = 2cm]{img/youtube}}
\end{minipage}
\vspace{1ex}
%% --------------------------------------------------------------------



Alex hatte in seiner Hausarbeit einen Versuch in einer Klimakammer durchgeführt. Soweit so gut. Dabei hat er sich mit Kartoffeln beschäftigt. Angeblich der neueste heiße Kram... aber das ist wiederum was anderes. So richtig mitgenommen hat Alex das Thema dann doch nicht. Hat er sich doch mit Genotypen ($00$, $AA$, $AB$ und $BB$) und Frischegewicht [kg/ha] schon eine Menge an Daten angeschaut. Nach seinem Betreuer soll er nun ein CLD bestimmen. Weder weiß er was ein CLD ist, noch war sein erster Gedanke mit Köln und die LGBTQ Community richtig...

\begin{knitrout}
\definecolor{shadecolor}{rgb}{0.969, 0.969, 0.969}\color{fgcolor}\begin{table}[!h]
\centering\begingroup\fontsize{10}{12}\selectfont

\begin{tabular}{cc}
\toprule
\textbf{Behandlung} & \textbf{Compact letter display}\\
\midrule
00 & a\\
AA & b\\
AB & b\\
BB & b\\
\bottomrule
\end{tabular}
\endgroup{}
\end{table}

\end{knitrout}

Leider kennen sich Alex mit dem \textit{Compact letter display (CLD)} überhaupt nicht aus. Deshalb braucht er bei der Erstellung Ihre Hilfe!

\begin{enumerate}
  \item Formulieren Sie die wissenschaftliche Fragestellung! \textbf{(1 Punkt)}
  \item Formulieren Sie die statistischen Hypothesen! \textbf{(1 Punkt)}
\item Zeichnen Sie die sich anhand des \textit{Compact letter display (CLD)} ergebenden Barplots! \textbf{(2 Punkte)}
\item Ergänzen Sie das \textit{Compact letter display (CLD)} zu den Barplots! \textbf{(1 Punkt)}
\item Erklären Sie \textit{einen} Vorteil und \textit{einen} Nachteil des \textit{Compact letter display (CLD)}! \textbf{(2 Punkte)}
\item Erstellen Sie eine Matrix mit den paarweisen $p$-Werten eines Student t-Tests, die sich näherungsweise aus dem \textit{Compact letter display (CLD)} ergeben würde! Begründen Sie Ihre Antwort! \textbf{(3 Punkte)}
\end{enumerate}

 
\clearpage
% ----------------------------------------------------------------------- 

\section{Aufgabe \hfill (12 Punkte)}

\textit{Geben Sie grundsätzlich Formeln und Rechenweg zur Lösung der Teilaufgaben mit an!} \\[1Ex]
 

 
%% --------------------------------------------------------------------
\begin{minipage}[t]{0.5\textwidth}
\includegraphics[width = 1.3cm]{/Users/kruppajo/work/GitHub/exam/avatare/Paula.png}
\end{minipage}
\begin{minipage}[t]{0.5\textwidth}
\hfill
\href{https://youtu.be/RagTFFKFbFg}{\includegraphics[width = 2cm]{img/caution}}
\end{minipage}
\vspace{1ex}
%% --------------------------------------------------------------------



Paula betrachtet in sich gekehrt die Poster vor dem Büro von ihr Betreuer. Viele der explorativen Abbildungen sagen ihr etwas. Die Barplots und die Boxplots könnte sie dann schon nachbauen. Das macht sie dann zuversichtlich die Abschlussarbeit auch hinzukriegen. Etwas komischer sind die seltsamen Buchstaben über den Barplots. Paula betrachtet ein Poster das sich mit Kartoffeln beschäftigt. Lüftungssysteme ($ctrl$, $storm$, $thunder$ und $tornado$) und Chlorophyllgehalt (SPAD-502Plus) [SPAD] wurden dort bestimmt. So richtig schlau, wird sie daraus nicht. Als erstes müsse müsse man die Gruppen nach absteigender Effektstärke sortieren, liest Paula im Methodenteil und ist dann noch verwirrter als vorher schon.

\begin{knitrout}
\definecolor{shadecolor}{rgb}{0.969, 0.969, 0.969}\color{fgcolor}\begin{table}[!h]
\centering\begingroup\fontsize{10}{12}\selectfont

\begin{tabular}{cccc}
\toprule
\textbf{Lüftungssysteme} & \textbf{Fallzahl (n)} & \textbf{Mittelwert} & \textbf{Standardabweichung}\\
\midrule
ctrl & 8 & 14.82 & 1.97\\
storm & 7 & 4.66 & 2.94\\
thunder & 9 & 14.73 & 2.15\\
tornado & 8 & 3.89 & 1.81\\
\bottomrule
\end{tabular}
\endgroup{}
\end{table}

\end{knitrout}

Leider kennen sich Paula mit dem \textit{Compact letter display (CLD)} überhaupt nicht aus. Deshalb braucht sie bei der Erstellung Ihre Hilfe!

\begin{enumerate}
  \item Formulieren Sie die wissenschaftliche Fragestellung! \textbf{(1 Punkt)}
  \item Formulieren Sie die statistischen Hypothesen! \textbf{(1 Punkt)}
\item Zeichnen Sie die sich ergebenden Barplots! \textbf{(1 Punkt)}
\item Berechnen Sie die Matrix der $p$-Werte anhand von Student t-Tests! \textbf{(4 Punkte)}
\item Ergänzen Sie das \textit{Compact letter display (CLD)} zu den gezeichneten Barplots! Begründen Sie Ihre Antwort! \textbf{(4 Punkte)}
\item Interpretieren Sie das \textit{Compact letter display (CLD)} für Paula und Yuki! \textbf{(1 Punkt)} 
\end{enumerate}

 
\clearpage
% -----------------------------------------------------------------------

\section{Aufgabe \hfill (10 Punkte)}

\textit{Geben Sie grundsätzlich Formeln und Rechenweg zur Lösung der Teilaufgaben mit an!} \\[1Ex]
 

 
%% --------------------------------------------------------------------
\begin{minipage}[t]{0.5\textwidth}
\includegraphics[width = 1.3cm]{/Users/kruppajo/work/GitHub/exam/avatare/Jonas.png}
\end{minipage}
\begin{minipage}[t]{0.5\textwidth}
\hfill
\href{https://youtu.be/RagTFFKFbFg}{\includegraphics[width = 2cm]{img/youtube}}
\end{minipage}
\vspace{1ex}
%% --------------------------------------------------------------------



'Das Problem ist, dass DataTab eben keine CLD kann. Die bräuchten wir dann schon dringend für unser Poster!', merkt sein Betreuer mit Nachdruck an. Jonas neigt den Kopf. 'Das wussten wir nicht vorher?', entfährt es ihm leicht entnervt. Da schaut sein Betreuer seltsam betroffen. Hilft jetzt auch so gar nicht. Jonas hatte sich zwei Variablen mit Düngestufen ($ctrl$, $low$, $mid$ und $high$) und Chlorophyllgehalt (SPAD-502Plus) [SPAD] in ein Freilandversuch mit Kartoffeln angeschaut. Jetzt möchte er eigentlich fertig werden und nicht nochmal alles neu in \Rlogo und \texttt\{emmeans\} machen. Dabei hatte er schon echt ne Menge in im Wendland gemacht. Dann eben per Hand aus der Matrix der $p$-Wert. Jonas muss sich echt zusammenreißen.

\begin{knitrout}
\definecolor{shadecolor}{rgb}{0.969, 0.969, 0.969}\color{fgcolor}\begin{table}[!h]
\centering\begingroup\fontsize{10}{12}\selectfont

\begin{tabular}{>{}lcccc}
\toprule
\textbf{ } & \textbf{ctrl} & \textbf{low} & \textbf{mid} & \textbf{high}\\
\midrule
\textbf{ctrl} & 1.0000000 & 0.5510673 & 0.2640480 & 0.0238202\\
\textbf{low} & 0.5510673 & 1.0000000 & 0.1196098 & 0.0119376\\
\textbf{mid} & 0.2640480 & 0.1196098 & 1.0000000 & 0.1632058\\
\textbf{high} & 0.0238202 & 0.0119376 & 0.1632058 & 1.0000000\\
\bottomrule
\end{tabular}
\endgroup{}
\end{table}

\end{knitrout}

Leider kennen sich Jonas mit dem \textit{Compact letter display (CLD)} überhaupt nicht aus. Deshalb braucht er bei der Erstellung Ihre Hilfe!

\begin{enumerate}
  \item Formulieren Sie die wissenschaftliche Fragestellung! \textbf{(1 Punkt)}
  \item Formulieren Sie die statistischen Hypothesen! \textbf{(1 Punkt)}
\item Zeichnen Sie die sich anhand der Matrix der $p$-Werte ergebenden Barplots! \textbf{(2 Punkte)}
\item Ergänzen Sie das \textit{Compact letter display (CLD)}! Begründen Sie Ihre Antwort! \textbf{(4 Punkte)}
\item Interpretieren Sie das \textit{Compact letter display (CLD)} für Jonas und Yuki! \textbf{(2 Punkte)} 
\end{enumerate}

 
\clearpage
% -----------------------------------------------------------------------
\part{Der Chi-Quadrat-Test \& Der diagnostische Test}
% -----------------------------------------------------------------------

\section{Aufgabe \hfill (12 Punkte)}

\textit{Geben Sie grundsätzlich Formeln und Rechenweg zur Lösung der Teilaufgaben mit an!} \\[1Ex]
 

 
%% --------------------------------------------------------------------
\begin{minipage}[t]{0.5\textwidth}
\includegraphics[width = 1.3cm]{/Users/kruppajo/work/GitHub/exam/avatare/Steffen.png}
\end{minipage}
\begin{minipage}[t]{0.5\textwidth}
\hfill
\href{https://youtu.be/-Kva5wc5Elw}{\includegraphics[width = 2cm]{img/youtube}}\\[1Ex]
\end{minipage}
\vspace{1ex}
%% --------------------------------------------------------------------



'Der $\mathcal{X}^2$-Test auf einer $2x2$-Kreuztabelle berechnet.', liest Steffen in seiner Mitschrift. So richtig helfen tut ihm das jetzt eherlichweise dann doch nicht. Steffen hatte sich in ein Feldexperiment $n = 104$ Beobachtungen von Kartoffeln angeschaut. Dabei hat er als Behandlung \textit{Pestizideinsatz [ja/nein]} bestimmt und zum anderen die Variable \textit{Trockengewicht über Zielwert [ja/nein]} ermittelt. Am Ende möchte dann seine Betreuerin gerne einen $\mathcal{X}^2$-Test auf einer $2x2$-Kreuztabelle berechnet bekommen.


\vspace{5Ex}

\begin{center}
  \huge
  \begin{tabular}{c|c|c|c}
     & \phantom{\textbf{Erkrankt (ja)}} & \phantom{\textbf{Erkrankt (ja)}} & \phantom{\textbf{Erkrankt (ja)}} \strut\\
    \hline
    \phantom{\textbf{Pestizid (ja)}} & 24  & 19  &     \strut\\
    \hline
    \phantom{\textbf{Pestizid (ja)}} & 23  & 38  &      \strut\\
    \hline
     \phantom{100} & \phantom{100}  & \phantom{100}  &  \phantom{100}  \strut\\
  \end{tabular}
\end{center}

\vspace{5Ex}

Leider kennt sich Steffen mit der Berechnung eines $\mathcal{X}^2$-Test für kategoriale Daten überhaupt nicht aus. Deshalb braucht er bei der Erstellung Ihre Hilfe!

\begin{enumerate}
\item Formulieren Sie die wissenschaftliche Fragestellung! \textbf{(1 Punkt)}
\item Ergänzen Sie die Tabelle um die fehlenden Informationen! \textbf{(1 Punkt)} 
\item Visualisieren Sie den Zusammenhang zwischen den beiden kategorialen Variablen! \textbf{(2 Punkte)}
\item Berechnen Sie die Teststatistik eines Chi-Quadrat-Test! \textbf{(2 Punkte)}
\item Treffen Sie eine Entscheidung im Bezug zu der Nullhypothese gegeben
  einem $\mathcal{X}^2_{\alpha = 5\%} = 3.841$! Begründen Sie Ihre Antwort!
  \textbf{(2 Punkte)}
\item Skizzieren Sie die $\mathcal{X}^2$-Verteilung, wenn die $H_0$ wahr ist! Ergänzen Sie  $\mathcal{X}^2_{\alpha = 5\%}$ und $\mathcal{X}^2_{D}$ in der Abbildung! \textbf{(2 Punkte)}
\item Berechnen Sie den Effektschätzer $Cramers\; V$! Interpretieren Sie den
  Effektschätzer! \textbf{(2 Punkte)}
\end{enumerate} 
\clearpage
% -----------------------------------------------------------------------

\section{Aufgabe \hfill (10 Punkte)}

\textit{Geben Sie grundsätzlich Formeln und Rechenweg zur Lösung der Teilaufgaben mit an!} \\[1Ex]
 

 
%% --------------------------------------------------------------------
\begin{minipage}[t]{0.5\textwidth}
\includegraphics[width = 1.3cm]{/Users/kruppajo/work/GitHub/exam/avatare/Jonas.png}
\end{minipage}
\begin{minipage}[t]{0.5\textwidth}
\hfill
\href{https://youtu.be/jakM7fHyZfU}{\includegraphics[width = 2cm]{img/youtube}}\\[1Ex]
\end{minipage}
\vspace{1ex}
%% --------------------------------------------------------------------



Am Ende hätte Jonas dann doch einen normalverteilten Endpunkt in seinem Projektbericht nehmen sollen. Vor ihm liegen jetzt die Daten von zwei Variablen als Kategorien oder wie es in \Rlogo so schön heißt, als Faktoren. Aber immerhin, hofft er das was bei den Daten rausgekommen ist. Gezählt hat Jonas einiges mit $n = 158$ Beobachtungen von Kartoffeln. Zum einen hat er als Behandlung \textit{KI-gesteuert [ja/nein]} bestimmt und zum anderen die Variable \textit{Trockengewicht über Zielwert [ja/nein]} ermittelt. Nun möchte seine Betreuerin gerne einen $\mathcal{X}^2$-Test auf einer $2x2$-Kreuztabelle berechnet bekommen.

\vspace{5Ex}

\begin{center}
  \huge
  \begin{tabular}{c|c|c|c}
     & \phantom{\textbf{Erkrankt (ja)}} & \phantom{\textbf{Erkrankt (ja)}} & \phantom{\textbf{Erkrankt (ja)}} \strut\\
    \hline
   \phantom{\textbf{Pestizid (ja)}} & \phantom{100}  & \phantom{100}  &   81  \strut\\
    \hline
    \phantom{\textbf{Pestizid (ja)}} & \phantom{100}  & \phantom{100}  &   77   \strut\\
    \hline
     &  97 &  61 &  158  \strut\\
  \end{tabular}
\end{center}

\vspace{5Ex}

Leider kennt sich Jonas mit der Berechnung eines $\mathcal{X}^2$-Test für kategoriale Daten überhaupt nicht aus. Deshalb braucht er bei der Erstellung Ihre Hilfe!

\begin{enumerate}
  \item Formulieren Sie die wissenschaftliche Fragestellung! \textbf{(1 Punkt)}
\item Ergänzen Sie die Tabelle um die fehlenden Informationen! \textbf{(1 Punkt)} 
\item Ergänzen Sie die Felder innerhalb der $2x2$ Kreuztabelle, so dass \textit{kein} signifikanter Effekt zu erwarten wäre! \textbf{(2 Punkte)}
\item Begründen Sie Ihr Vorgehen an der Formel des Chi-Quadrat-Tests. Erklären Sie Ihr Vorgehen an einem Beispiel! \textbf{(2 Punkte)}
\item Visualisieren Sie den Zusammenhang zwischen den beiden kategorialen Variablen! \textbf{(2 Punkte)}
\item Was ist die Mindestanzahl an Beobachtungen je Zelle? Wenn in einer der Zellen weniger Beobachtungen auftreten, welchen Test können Sie anstatt des Standard Chi-Quadrat-Tests anwenden? \textbf{(2 Punkte)}
\end{enumerate} 
\clearpage
% -----------------------------------------------------------------------

\section{Aufgabe \hfill (10 Punkte)}

\textit{Geben Sie grundsätzlich Formeln und Rechenweg zur Lösung der Teilaufgaben mit an!} \\[1Ex]
 

 
%% --------------------------------------------------------------------
\begin{minipage}[t]{0.5\textwidth}
\includegraphics[width = 1.3cm]{/Users/kruppajo/work/GitHub/exam/avatare/Mark.png}
\end{minipage}
\begin{minipage}[t]{0.5\textwidth}
\hfill
\href{https://youtu.be/ghArbetOr_E}{\includegraphics[width = 2cm]{img/youtube}}\\[1Ex]
\end{minipage}
\vspace{1ex}
%% --------------------------------------------------------------------



Mark hat sich ein Herz gefasst und war für seiner Hausarbeit in die Niederlande gegangen. Das war eine super Zeit in der er viel gelernt hat. Klar gab es auch die ein oder andere Besonderheit, aber das gehört hier eher nicht hin. Mark ist schon eine ganze Zeit im Büro, da sein Betreuer möchte, dass er jetzt auf seinen Daten mit $n = 158$ Beobachtungen von Kartoffeln einen $\mathcal{X}^2$-Test rechnet. Das ginge, da er als Behandlung \textit{Mechanische Bearbeitung [ja/nein]} bestimmt und zum anderen die Variable \textit{Frischegewicht über Zielwert [ja/nein]} ermittelt hat. Wie genau, das ist jetzt eine andere Frage. Nach seinem Experiment erhielt er folgende $2x2$ Kreuztabelle aus seinen erhobenen Daten.

\begin{knitrout}
\definecolor{shadecolor}{rgb}{0.969, 0.969, 0.969}\color{fgcolor}\begin{kframe}
\begin{verbatim}
##                             Mechanische Bearbeitung
## Frischegewicht über Zielwert ja nein
##                         ja   13    5
##                         nein  7   18
\end{verbatim}
\end{kframe}
\end{knitrout}

Dann rechnete Mark den Fisher-Exakt-Test auf der $2x2$-Kreuztabelle in \Rlogo und erhielt folgende \Rlogo Ausgabe der Funktion \texttt{fisher.test()}.

\begin{knitrout}
\definecolor{shadecolor}{rgb}{0.969, 0.969, 0.969}\color{fgcolor}\begin{kframe}
\begin{verbatim}
## 
## 	Fisher's Exact Test for Count Data
## 
## data:  Frischegewicht über Zielwert
## p-value = 0.005898
## alternative hypothesis: true odds ratio is not equal to 1
## 95 percent confidence interval:
##   1.462677 32.500828
## sample estimates:
## odds ratio 
##   6.352594
\end{verbatim}
\end{kframe}
\end{knitrout}

Leider kennt sich Mark mit der Berechnung eines $\mathcal{X}^2$-Test für kategoriale Daten überhaupt nicht aus. Deshalb braucht er bei der Erstellung Ihre Hilfe!

\begin{enumerate}
\item Formulieren Sie die wissenschaftliche Fragestellung! \textbf{(1 Punkt)}
\item Visualisieren Sie den Zusammenhang zwischen den beiden kategorialen Variablen! \textbf{(2 Punkte)}
\item Liegt ein signifikanter Unterschied zwischen den Gruppen vor? Begründen Sie Ihre Antwort! \textbf{(2 Punkte)}
\item Skizzieren Sie das sich ergebende 95\% Konfidenzintervall! \textbf{(2 Punkte)}
\item Beschriften Sie die Abbildung des 95\% Konfidenzintervalls! \textbf{(1 Punkt)} 
\item Interpretieren Sie das \textit{Odds ratio} im Kontext der wissenschaftlichen Fragestellung! \textbf{(2 Punkte)} 
\end{enumerate}
 
\clearpage
% -----------------------------------------------------------------------

\section{Aufgabe \hfill (11 Punkte)}

%% --------------------------------------------------------------------
\hfill\href{https://youtu.be/VQlNl8hvRII}{\includegraphics[width =
  2cm]{img/youtube}}\\[1Ex]
%% --------------------------------------------------------------------


Die Pr{\"a}valenz von Klauenseuche bei Wollschweinen wird mit
3\% angenommen. In 80\% der F{\"a}lle ist ein Test positiv, wenn das Wollschwein erkrankt
ist. In 8\% der F{\"a}lle ist ein Test positiv,
wenn das Wollschwein \textit{nicht} erkrankt ist und somit gesund ist. Sie
werten 4000 Wollschweine mit einem
diagnostischen Test auf Klauenseuche aus.



\begin{enumerate}
\item F{\"u}llen und beschriften Sie den untenstehenden Doppelbaum! Beschriften
  Sie auch die {\"A}ste des Doppelbaumes, mit denen Ihnen bekannten
  Informationen!  \textbf{(8 Punkte)}
\item Berechnen Sie die Wahrscheinlichkeit $Pr(K^+|T^+)$! \textbf{(2 Punkte)}
\item Was sagt Ihnen die Wahrscheinlichkeit $Pr(K^+|T^+)$ aus? \textbf{(1 Punkt)}
\end{enumerate}

\vspace{1cm}

\begin{center}
  \includegraphics[width=17cm]{/Users/kruppajo/work/GitHub/exam/question/img/diag-doppelbaum}
\end{center}



 
\clearpage
% -----------------------------------------------------------------------

\section{Aufgabe \hfill (12 Punkte)}


%% --------------------------------------------------------------------
\hfill\href{https://youtu.be/_7s44pbOc00}{\includegraphics[width =
  2cm]{img/youtube}}\\[1Ex]
%% --------------------------------------------------------------------





Folgender diagnostischer Doppelbaum nach der Testung auf Klauenseuche bei
Fleckvieh ist gegeben.

\begin{enumerate}
\item F{\"u}llen und beschriften Sie den untenstehenden Doppelbaum! \textbf{(4
    Punkte)}
\item Berechnen Sie die Wahrscheinlichkeit $Pr(K^+|T^+)$! \textbf{(2 Punkte)}
\item Berechnen Sie die Pr{\"a}valenz f{\"u}r Klauenseuche! \textbf{(2 Punkte)}
\item Berechnen Sie die Sensifit{\"a}t und Spezifit{\"a}t des diagnostischen Tests
  f{\"u}r Klauenseuche! Erstellen Sie daf{\"u}r zun{\"a}chst eine 2x2 Kreuztabelle aus
  dem ausgef{\"u}llten Doppelbaum!
  \textbf{(4 Punkte)}
\end{enumerate}

\vspace{1cm}
 
\begin{tikzpicture}
  \node (image) at (0,0) {
    \includegraphics[width=\textwidth]{/Users/kruppajo/work/GitHub/exam/question/img/diag-doppelbaum}
  };
  \node[] at (-4.8,0) {\huge 180};
  \node[] at (-1.7,0) {\huge 20};
  \node[] at (1.7,0) {\huge 750};
  \node[] at (4.75,0) {\huge 1450};
\end{tikzpicture}




 
\clearpage
% -----------------------------------------------------------------------
\part{Lineare Regression \& Korrelation}
% -----------------------------------------------------------------------

\section{Aufgabe \hfill (10 Punkte)}

\textit{Geben Sie grundsätzlich Formeln und Rechenweg zur Lösung der Teilaufgaben mit an!} \\[1Ex]
 

 
%% --------------------------------------------------------------------
\begin{minipage}[t]{0.5\textwidth}
\includegraphics[width = 1.3cm]{/Users/kruppajo/work/GitHub/exam/avatare/Steffen.png}\hspace{-4mm}\includegraphics[width = 1.3cm]{/Users/kruppajo/work/GitHub/exam/avatare/Yuki.png}
\end{minipage}
\begin{minipage}[t]{0.5\textwidth}
\hfill
\href{https://youtu.be/kHmfEmU6lrk}{\includegraphics[width = 2cm]{img/youtube}}
\end{minipage}
\vspace{1ex}
%% --------------------------------------------------------------------



'Wichtig ist es, dass wir jetzt eine Gerade durch die Punkte zeichnen!', ruft Steffen. 'Ich sehe nur eine Zahlen und keine Punkte. Wie soll ich da denn jetzt eine Gerade durchzeichnen?', fragt Yuki. Steffen atmet schwer ein. Die beiden hatten ein Feldexperiment im Oldenburger Land mit Kartoffeln durchgeführt. Dabei wurden die beiden folgenden Variablen gemessen: durchschnittlicher Regenwurmdichte [Anzahl/l] und Frischegewicht [kg/ha]. Jetzt will die Betreuung von den beiden einmal die Visualisierung der Daten und auch gleich noch die lineare Regression gerechnet bekommen.

\begin{table}[!h]
\centering
\begin{tabular}{cc}
\toprule
Durchschnittlicher Regenwurmdichte [Anzahl/l] & Frischegewicht [kg/ha]\\
\midrule
24.8 & 23.5\\
13.3 & 16.8\\
24.5 & 21.3\\
22.2 & 22.8\\
13.5 & 14.7\\
\addlinespace
19.8 & 25.1\\
25.8 & 26.1\\
24.9 & 24.3\\
30.3 & 26.9\\
13.0 & 11.3\\
\bottomrule
\end{tabular}
\end{table}



Leider kennen sich Steffen und Yuki mit der linearen Regression für kontinuierliche Daten überhaupt nicht aus. Deshalb brauchen beide bei der Erstellung Ihre Hilfe!

\begin{enumerate}
\item Formulieren Sie die wissenschaftliche Fragestellung! \textbf{(1 Punkt)}
\item Erstellen  Sie  eine  Visualisierung  für  die  Datentabelle.  Beschriften  Sie  die  Achsen! \textbf{(2 Punkte)}
\item Schätzen Sie die Regressionsgleichung aus der obigen Abbildung ab! \textbf{(2 Punkte)}
\item Beschriften Sie die Grade mit den statistischen Maßzahlen der linearen Regressionsgleichung! \textbf{(2 Punkte)}
\item Liegt ein Zusammenhang zwischen $x$ und $y$ vor? Begründen Sie Ihre Antwort! \textbf{(2 Punkte)}
\item Wenn kein Zusammenhang zu beobachten wäre, wie würde die Grade aussehen? \textit{Antworten Sie mit einer Skizze der Geraden!} \textbf{(1 Punkt)}
\end{enumerate} 
\clearpage
% -----------------------------------------------------------------------

\section{Aufgabe \hfill (12 Punkte)}

\textit{Geben Sie grundsätzlich Formeln und Rechenweg zur Lösung der Teilaufgaben mit an!} \\[1Ex]
 

 
%% --------------------------------------------------------------------
\begin{minipage}[t]{0.5\textwidth}
\includegraphics[width = 1.3cm]{/Users/kruppajo/work/GitHub/exam/avatare/Jonas.png}\hspace{-4mm}\includegraphics[width = 1.3cm]{/Users/kruppajo/work/GitHub/exam/avatare/Steffen.png}
\end{minipage}
\begin{minipage}[t]{0.5\textwidth}
\hfill
\href{https://youtu.be/lJp8rFmMnrs}{\includegraphics[width = 2cm]{img/youtube}}
\end{minipage}
\vspace{1ex}
%% --------------------------------------------------------------------



'Wichtig ist es, dass wir jetzt eine Gerade durch die Punkte zeichnen!', ruft Jonas. 'Ich sehe nur zwei Zeilen und keine Punkte. Wie soll ich da denn jetzt eine Gerade durchzeichnen?', fragt Steffen. Jonas atmet schwer ein und starrt auf die \Rlogo Ausgabe der Funktion \texttt{lm()}. Die beiden hatten ein Freilandversuch im Teuteburgerwald mit Kartoffeln durchgeführt. Dabei wurden die beiden folgenden Variablen gemessen: durchschnittliche UV-Einstrahlung [UV/d] und Proteingehalt [g/kg]. Jetzt will die Betreuung von den beiden einmal die Visualisierung der Daten und auch gleich noch die lineare Regression gerechnet bekommen. Das haben beide in \Rlogo gemacht, aber wie soll das jetzt gehen?

\begin{table}[!h]
\centering\begingroup\fontsize{12}{14}\selectfont

\begin{tabular}{ccccc}
\toprule
term & estimate & std.error & t statistic & p-value\\
\midrule
(Intercept) & 2.10 & 2.18 &  & \\
Durchschnittliche UV-Einstrahlung & 0.51 & 0.21 &  & \\
\bottomrule
\end{tabular}
\endgroup{}
\end{table}



Leider kennen sich Jonas und Steffen mit der linearen Regression für kontinuierliche Daten in \Rlogo überhaupt nicht aus. Deshalb brauchen beide bei der Erstellung Ihre Hilfe!

\begin{enumerate}
\item Formulieren Sie die wissenschaftliche Fragestellung! \textbf{(1 Punkt)}
\item Erstellen  Sie  eine  Visualisierung  der \texttt{lm()}-Ausgabe.  Beschriften  Sie  die  Achsen! \textbf{(2 Punkte)}
\item Beschriften Sie die Visualisierung mit den statistischen Maßzahlen der der \texttt{lm()}-Ausgabe! \textbf{(2 Punkte)}
\item Formulieren Sie die Regressionsgleichung! \textbf{(1 Punkt)}
\item Ergänzen Sie die t Statistik in der \texttt{lm()}-Ausgabe! \textbf{(2 Punkte)}
\item Ergänzen Sie den $p$-Wert in der \texttt{lm()}-Ausgabe mit $T_{\alpha = 5\%} = 1.96$!  \textbf{(2 Punkte)}
\item Interpretieren Sie den $p$-Wert im Kontext der wissenschaftlichen Fragestellung! \textbf{(1 Punkt)}  
\item Wie groß ist der Effekt im Kontext der wissenschaftlichen Fragestellung? \textbf{(1 Punkt)}
\end{enumerate} 
\clearpage
% -----------------------------------------------------------------------

\section{Aufgabe \hfill (11 Punkte)}

\textit{Geben Sie grundsätzlich Formeln und Rechenweg zur Lösung der Teilaufgaben mit an!} \\[1Ex]
 

 
%% --------------------------------------------------------------------
\begin{minipage}[t]{0.5\textwidth}
\includegraphics[width = 1.3cm]{/Users/kruppajo/work/GitHub/exam/avatare/Mark.png}\hspace{-4mm}\includegraphics[width = 1.3cm]{/Users/kruppajo/work/GitHub/exam/avatare/Nilufar.png}
\end{minipage}
\begin{minipage}[t]{0.5\textwidth}
\hfill
\href{https://youtu.be/tNNzcndrpSk}{\includegraphics[width = 2cm]{img/youtube}}
\end{minipage}
\vspace{1ex}
%% --------------------------------------------------------------------



'Ich glaube du bringst da was durcheinander. Wir nutzen zwar auch für die ANOVA die Funktion \texttt{lm()} aber hier wollen wir, glaube ich, eine Gerade durch die Punkte zeichnen.', merkt Nilufar an. 'Ich sehe keine Punkte... das ist doch eine Ausgabe in \Rlogo. Überhaupt, darum geht es doch gar nicht in unserem Versuch. Wir wollen doch keine Gerade zeichnen?.', antwortet Mark sichtlich übernächtigt. 'Doch wir müssen nur die Koeffizienten der linearen Regression erst richtig interpretieren und vor unserem geistigen Auge erscheint eine Gerade!', spricht Nilufar sehr deutlich und langsam. Die beiden hatten ein Freilandversuch im Emsland mit Kartoffeln durchgeführt. Dabei wurden die beiden folgenden Variablen gemessen: durchschnittlicher Anteil an Ton [\%/l] und Chlorophyllgehalt (SPAD-502Plus) [SPAD]. Jetzt wollen sie erstmal schauen, ob es einen Zusammenhang gibt und das soll mit der \Rlogo Ausgabe möglich sein.

\begin{knitrout}
\definecolor{shadecolor}{rgb}{0.969, 0.969, 0.969}\color{fgcolor}\begin{kframe}
\begin{verbatim}
## 
## Call:
## Chlorophyllgehalt ~ Durchschnittlicher_Anteil
## 
## Residuals:
##      Min       1Q   Median       3Q      Max 
## -2.77685 -0.51265 -0.09686  0.69158  3.11103 
## 
## Coefficients:
##                           Estimate Std. Error t value Pr(>|t|)
## (Intercept)                -1.6677     1.6276  -1.025    0.311
## Durchschnittlicher_Anteil   0.8755     0.1619   5.409  2.8e-06
## 
## Residual standard error: 1.325 on 42 degrees of freedom
## Multiple R-squared:  0.4106,	Adjusted R-squared:  0.3965 
## F-statistic: 29.25 on 1 and 42 DF,  p-value: 2.799e-06
\end{verbatim}
\end{kframe}
\end{knitrout}

Leider kennen sich Nilufar und Mark mit der linearen Regression für kontinuierliche Daten in \Rlogo überhaupt nicht aus. Deshalb brauchen beide bei der Erstellung Ihre Hilfe!


\begin{enumerate}
\item Formulieren Sie die wissenschaftliche Fragestellung! \textbf{(1 Punkt)}
\item Wie groß ist der Effekt im Kontext der wissenschaftlichen Fragestellung? \textbf{(2 Punkte)} 
\item Interpretieren Sie die $p$-Werte im Kontext der wissenschaftlichen Fragestellung! \textbf{(2 Punkte)}
\item Visualisieren Sie die Verteilung der Residuen! \textbf{(2 Punkte)} 
\item Ist die Annahme der Normalverteilung erfüllt? Begründen Sie die Antwort! \textbf{(2 Punkte)}
\item Erklären Sie \textit{kurz} den Begriff \texttt{R-squared}! Was sagt Ihnen der Wert $0.41$ aus? \textbf{(2 Punkte)}
\end{enumerate}
 
\clearpage
% -----------------------------------------------------------------------

\section{Aufgabe \hfill (9 Punkte)}

\textit{Geben Sie grundsätzlich Formeln und Rechenweg zur Lösung der Teilaufgaben mit an!} \\[1Ex]
 

 
%% --------------------------------------------------------------------
\begin{minipage}[t]{0.5\textwidth}
\includegraphics[width = 1.3cm]{/Users/kruppajo/work/GitHub/exam/avatare/Mark.png}
\end{minipage}
\begin{minipage}[t]{0.5\textwidth}
\hfill
\href{https://youtu.be/C9skfFRTHhI}{\includegraphics[width = 2cm]{img/youtube}}
\end{minipage}
\vspace{1ex}
%% --------------------------------------------------------------------



'Ich glaube ich bringe da was durcheinander. Ich möchte eine Gerade durch die Punkte zeichnen oder doch eine Korrelation berechnen?', merkt Mark laut an. 'Ich sehe keine Punkte... das ist doch eine Ausgabe in \Rlogo. Überhaupt, darum geht es doch gar nicht in meinem Versuch. Ich wollte doch keine Gerade zeichnen?.', antwortet Mark sich sichtlich übernächtigt selber. Die Nacht war zu lang und überhaupt. Mark hatte ein Feldexperiment im Oldenburger Land mit Kartoffeln durchgeführt. Dabei wurden die beiden folgenden Variablen gemessen: durchschnittlicher Niederschlag [ml/w] und Proteingehalt [g/kg]. Jetzt will er erstmal schauen, ob es einen Zusammenhang gibt und das soll mit der \Rlogo Ausgabe möglich sein.


\begin{knitrout}
\definecolor{shadecolor}{rgb}{0.969, 0.969, 0.969}\color{fgcolor}\begin{kframe}
\begin{verbatim}
## 
## 	Pearson's correlation
## 
## data:  Durchschnittlicher Niederschlag [ml/w] and Proteingehalt [g/kg]
## t = -1.6416, df = 8, p-value = 0.1393
## alternative hypothesis: true correlation is not equal to 0
## 95 percent confidence interval:
##  -0.8598424  0.1866402
## sample estimates:
##       cor 
## -0.501975
\end{verbatim}
\end{kframe}
\end{knitrout}

Leider kennt sich Mark mit der Korrelationsanalyse in \Rlogo überhaupt nicht aus. Deshalb braucht er bei der Erstellung Ihre Hilfe!

\begin{enumerate}
  \item Formulieren Sie die wissenschaftliche Fragestellung! \textbf{(1 Punkt)}
  \item Formulieren Sie das statistische Hypothesenpaar! \textbf{(1 Punkt)}
\item Erstellen Sie eine Visualisierung für den Korrelationskoeffizienten! Beschriften Sie die Abbildung! \textbf{(2 Punkte)}
\item Nennen Sie die zwei Eigenschaften des Korrelationskoeffizienten! \textbf{(2 Punkte)}
\item Interpretieren Sie den Korrelationskoefizienten hinsichtlich des
  Effekts und der Signifikanz! Begründen Sie Ihre Antwort! \textbf{(2 Punkte)}
\item Visualisieren Sie das 95\% Konfidenzintervall! Beschriften Sie die Abbildung! \textbf{(2 Punkte)} 
\end{enumerate} 
\clearpage
% -----------------------------------------------------------------------

\section{Aufgabe \hfill (9 Punkte)}

%% --------------------------------------------------------------------
\hfill\href{https://youtu.be/fB6nF4dxodA}{\includegraphics[width =
  2cm]{img/youtube}}\\[1Ex]
%% --------------------------------------------------------------------


In den folgenden Abbildungen sehen Sie drei leere Scatterplots. F{\"u}llen Sie diese
Scatterplots nach folgenden Anweisungen.

\begin{enumerate}
\item Zeichnen Sie f{\"u}r die angegebene $\rho$-Werte eine Gerade in die
  entsprechende Abbildung! \textbf{(3 Punkte)}
\item Zeichnen Sie f{\"u}r die angegebenen $R^2$-Werte die entsprechende
  Punktewolke um die Gerade. \textbf{(3 Punkte)}
\item Sie rechnen ein statistisches Modell. Was sagen Ihnen die $R^2$-Werte
  {\"u}ber das jeweilige Modell? \textbf{(3 Punkte)}
\end{enumerate}




{\centering \includegraphics[width=\maxwidth]{img/correlation-01-1} 

}



 
\clearpage
% -----------------------------------------------------------------------

\section{Aufgabe \hfill (9 Punkte)}

%% --------------------------------------------------------------------
\hfill\href{https://youtu.be/2QJa19ZwLls}{\includegraphics[width =
  2cm]{img/youtube}}\\[1Ex]
%% --------------------------------------------------------------------

In den folgenden Abbildungen sehen Sie vier Scatterplots. Erg{\"a}nzen Sie die {\"U}berschriften
der jeweiligen Scatterplots.


\begin{enumerate}
\item Sch{\"a}tzen Sie die $\rho$-Werte in der entsprechenden
  Abbildung! \textbf{(4 Punkte)}
\item Sch{\"a}tzen Sie die $R^2$-Werte in der entsprechenden
  Punktewolke um die Gerade! \textbf{(4 Punkte)}
\item Sie rechnen ein statistisches Modell. Was sagen Ihnen die $R^2$-Werte
  {\"u}ber das jeweilige Modell? \textbf{(1 Punkt)}
\end{enumerate}




{\centering \includegraphics[width=\maxwidth]{img/correlation-02-1} 

}



 
\clearpage
% -----------------------------------------------------------------------

\section{Aufgabe \hfill (10 Punkte)}

%% --------------------------------------------------------------------
\hfill\href{https://youtu.be/dyQlYV9nOqY}{\includegraphics[width =
  2cm]{img/youtube}}\\[1Ex]
%% --------------------------------------------------------------------

Sie rechnen eine lineare Regression um nach einem Feldexperiment den
Zusammenhang zwischen Trockengewicht kg/m$^2$ (\textit{drymatter}) und
Wassergabe l/m$^2$ (\textit{water}) bei Spargel zu bestimmen. Sie erhalten
folgende Datentabelle.

\begin{knitrout}
\definecolor{shadecolor}{rgb}{0.969, 0.969, 0.969}\color{fgcolor}\begin{table}[!h]
\centering\begingroup\fontsize{12}{14}\selectfont

\begin{tabular}{ccccc}
\toprule
.id & drymatter & water & .fitted & .resid\\
\midrule
1 & 16.3 & 6.6 & 18.8 & \\
2 & 13.9 & 4.4 & 15.3 & \\
3 & 22.9 & 7.6 & 20.3 & \\
4 & 22.4 & 7.4 & 20.1 & \\
5 & 32.3 & 14.7 & 31.6 & \\
\addlinespace
6 & 32.6 & 16.4 & 34.1 & \\
7 & 20.9 & 7.9 & 20.9 & \\
8 & 19.8 & 8.3 & 21.4 & \\
9 & 20.5 & 7.5 & 20.1 & \\
10 & 17.0 & 5.6 & 17.1 & \\
\addlinespace
11 & 25.3 & 10.1 & 24.3 & \\
\bottomrule
\end{tabular}
\endgroup{}
\end{table}

\end{knitrout}

\begin{enumerate}
\item Erg{\"a}nzen Sie die Werte in der Spalte \texttt{.resid} in der obigen
  Tabelle. Geben Sie den Rechenweg und Formel mit an! \textbf{(4 Punkte)}
\item Zeichnen Sie den sich aus der obigen Tabelle ergebenden
  Residualplot. Beschriften Sie die Abbildung! \textbf{(4 Punkte)}
\item Gibt es auff{\"a}llige Werte anhand des Residualplots? Begr{\"u}nden Sie Ihre
  Antwort! \textbf{(2 Punkte)}
\end{enumerate}
 
\clearpage
% -----------------------------------------------------------------------

\section{Aufgabe \hfill (12 Punkte)}

%% --------------------------------------------------------------------
\hfill\href{https://youtu.be/lHzRgm7hPw0}{\includegraphics[width =
  2cm]{img/youtube}}\\[1Ex]
%% --------------------------------------------------------------------



\begin{enumerate}
\item Zeichen Sie in die drei untenstehenden, leeren Abbilungen die Zeile des
  Regressionskreuzes der Binomialverteilung. W{\"a}hlen Sie die Beschriftung der
  y-Achse sowie der x-Achse entsprechend aus! \textbf{(6 Punkte)}
\item Erg{\"a}nzen Sie die jeweiligen statistischen Methoden zu der Abbildung! \textbf{(2 Punkte)}
\item Welchen Effektsch{\"a}tzer erhalten Sie aus der entsprechend linearen
  Regression bzw. den Gruppenvergleich? Geben Sie ein Beispiel! \textbf{(2 Punkte)}
\item Wenn Sie keinen Effekt erwarten, welchen \textit{Zahlenraum} nimmt dann
  der Effektsch{\"a}tzer ein? Geben Sie ein Beispiel! \textbf{(2 Punkte)}
\end{enumerate}



{\centering \includegraphics[width=\maxwidth]{img/regression-01-1} 

}



 
\clearpage
% -----------------------------------------------------------------------

\section{Aufgabe \hfill (9 Punkte)}

%% --------------------------------------------------------------------
\hfill\href{https://youtu.be/AwQEcQWLFCw}{\includegraphics[width =
  2cm]{img/youtube}}\\[1Ex]
%% --------------------------------------------------------------------



Ein Feldexperiment wurde mit $n = 200$ Pflanzen durchgef{\"u}hrt. Folgende
Einflussvariablen ($x$) wurden erhoben: height, block und variety. Als m{\"o}gliche Outcomevariablen stehen Ihnen nun
folgende gemessene Endpunkte zu Verf{\"u}gung: drymatter, yield, count, quality\_score und dead.

\begin{enumerate}
\item W{\"a}hlen Sie ein Outcome was zu der Verteilungsfamilie
  \textit{Poisson} geh{\"o}rt! \textbf{(1 Punkt)}
\item Schreiben Sie das Modell in der Form $y \sim x$ wie es in \Rlogo in
  der Funktion \texttt{glm()}
  {\"u}blich ist \textit{ohne Interaktionsterm}! \textbf{(3 Punkte)}
\item Schreiben Sie das Modell in der Form $y \sim x$ wie es in \Rlogo
  {\"u}blich ist und erg{\"a}nzen Sie \textit{einen} Interaktionsterm nach Wahl! \textbf{(1 Punkt)} 
\item Zeichen Sie eine \textit{starke}
  Interaktion in die Abbildung unten f{\"u}r den Endpunkt
  \textit{yield}. Erg{\"a}nzen Sie eine aussagekr{\"a}ftige Legende. Wie erkennen
  Sie eine Interaktion? Begr{\"u}nden Sie Ihre Antwort! \textbf{(4 Punkte)}
\end{enumerate}



{\centering \includegraphics[width=\maxwidth]{img/modeling-R-01-1} 

}


 
\clearpage
% -----------------------------------------------------------------------
\part{Experimentelles Design}
% -----------------------------------------------------------------------

\section{Aufgabe \hfill (16 Punkte)}


 
%% --------------------------------------------------------------------
\begin{minipage}[t]{0.5\textwidth}
\includegraphics[width = 1.3cm]{/Users/kruppajo/work/GitHub/exam/avatare/Jessica.png}\hspace{-4mm}\includegraphics[width = 1.3cm]{/Users/kruppajo/work/GitHub/exam/avatare/Paula.png}\hspace{-4mm}\includegraphics[width = 1.3cm]{/Users/kruppajo/work/GitHub/exam/avatare/Steffen.png}
\end{minipage}
\begin{minipage}[t]{0.5\textwidth}
\hfill
\href{https://youtu.be/wJqsNV1hOW8}{\includegraphics[width = 2cm]{img/caution}}
\end{minipage}
\vspace{1ex}
%% --------------------------------------------------------------------



Steffen und Jessica sind bei Paula um sich Hilfe für eine Versuchsplanung in \Rlogo zu holen. Dabei geht es um den Zusammenhang zwischen der Behandlung Lichtstufen ($none$, $200lm$, $400lm$ und $600lm$) sowie Substrattypen ($torf$, $70p30n$) und dem Messwert Proteingehalt [g/kg] in Kartoffeln. Der Versuch soll in einem Feldexperiment im Teuteburgerwald durchgeführt werden. Nach der Dozentin ist der Messwert Proteingehalt [g/kg] normalverteilt. Die beiden entschieden sich für ein faktorielles Versuchsdesign. Im ersten Schritt überlegt Paula ein einfaches experimentelles Design zu probieren. Daher entscheiden sich alle drei für ein \textit{Latin square design} mit nur einem der beiden Faktoren. Das sollte für den anfang erstmal reichen. 'Und jetzt, was machen wir jetzt?', Jessica schaut die anderen beiden mit großen Augen an. Die zucken mit der Schulter.\\

Leider kennen sich Paula, Steffen und Jessica mit dem \textit{Latin square design} überhaupt nicht aus. Deshalb brauchen die Drei bei der Erstellung Ihre Hilfe!

\begin{enumerate}
  \setcounter{enumi}{0}
  \item Formulieren Sie die wissenschaftliche Fragestellung! \textbf{(1 Punkt)}
  \item Formulieren Sie das statistische Hypothesenpaar! \textbf{(1 Punkt)}
  \item Skizzieren Sie das faktorielle Versuchsdesign! \textbf{(3 Punkte)}
  \item Skizzieren Sie eine Datentabelle für das faktorielle Versuchsdesign in \Rlogo! \textbf{(2 Punkte)}
  \item Erstellen Sie das statistische Modell in der in \Rlogo üblichen Schreibweise für eine ANOVA! Skizzieren Sie die notwendige Funktionen in \Rlogo! \textbf{(3 Punkte)}
  \item Skizzieren Sie die weitere Datenanalyse hinsichtlich eines multiplen Gruppenvergleiches! \textbf{(2 Punkte)}
  \item Skizzieren Sie eine mögliche Abbildung im Kontext der wissenschaftlichen Fragestellung! Beschriften Sie die Abbildung! \textbf{(2 Punkte)}
  \item Ergänzen Sie zu der Abbildung ein mögliches Ergebnis des multiplen Gruppenvergleichs! Begründen Sie Ihre Antwort! \textbf{(2 Punkte)}
\end{enumerate}


 
\clearpage
% -----------------------------------------------------------------------

\section{Aufgabe \hfill (20 Punkte)}


 
%% --------------------------------------------------------------------
\begin{minipage}[t]{0.5\textwidth}
\includegraphics[width = 1.3cm]{/Users/kruppajo/work/GitHub/exam/avatare/Alex.png}\hspace{-4mm}\includegraphics[width = 1.3cm]{/Users/kruppajo/work/GitHub/exam/avatare/Tina.png}\hspace{-4mm}\includegraphics[width = 1.3cm]{/Users/kruppajo/work/GitHub/exam/avatare/Yuki.png}
\end{minipage}
\begin{minipage}[t]{0.5\textwidth}
\hfill
\href{https://youtu.be/wJqsNV1hOW8}{\includegraphics[width = 2cm]{img/caution}}
\end{minipage}
\vspace{1ex}
%% --------------------------------------------------------------------



Tina und Alex sind bei Yuki um sich Hilfe für eine Versuchsplanung in \Rlogo zu holen. Dabei geht es um den Zusammenhang zwischen der Behandlung Substrattypen ($torf$, $40p60n$, $30p20n$ und $70p30n$) sowie Lüftungssystemen und Folientunneln ($ctrl$, und $tornado$) sowie drei Blöcken und dem Messwert Trockengewicht [kg/ha] in Kartoffeln. Der Versuch soll in einem Freilandversuch im Emsland durchgeführt werden. Nach dem Dozenten ist der Messwert Trockengewicht [kg/ha] normalverteilt. Die beiden entschieden sich für ein faktorielles Versuchsdesign. Im ersten Schritt überlegt Yuki ein komplexeres experimentelles Design zu probieren. Daher entscheiden sich alle drei für ein \textit{Randomized complete block design} mit Berücksichtigung einer Interaktion. Das sollte für den anfang erstmal reichen. 'Und jetzt, was machen wir jetzt?', Alex schaut die anderen beiden mit großen Augen an. Die zucken mit der Schulter.\\

Leider kennen sich Yuki, Tina und Alex mit dem \textit{Randomized complete block design} überhaupt nicht aus. Deshalb brauchen die Drei bei der Erstellung Ihre Hilfe!

\begin{enumerate}
  \setcounter{enumi}{0}
  \item Formulieren Sie die wissenschaftliche Fragestellung! \textbf{(1 Punkt)}
  \item Formulieren Sie die statistische Hypothesenpaare! \textbf{(2 Punkte)}
  \item Skizzieren Sie das faktorielle Versuchsdesign! \textbf{(3 Punkte)}
  \item Skizzieren Sie eine Datentabelle für das faktorielle Versuchsdesign in \Rlogo! \textbf{(2 Punkte)}
  \item Erstellen Sie das statistische Modell in der in \Rlogo üblichen Schreibweise für eine ANOVA! Skizzieren Sie die notwendige Funktionen in \Rlogo! \textbf{(4 Punkte)}
  \item Skizzieren Sie die weitere Datenanalyse hinsichtlich eines multiplen Gruppenvergleiches! \textbf{(2 Punkte)}
  \item Skizzieren Sie eine mögliche Abbildung im Kontext der wissenschaftlichen Fragestellung! Beschriften Sie die Abbildung! \textbf{(3 Punkte)}
  \item Ergänzen Sie zu der Abbildung ein mögliches Ergebnis des multiplen Gruppenvergleichs! Welche Annahme hinsichtlich der Modellierung haben Sie getroffen? Begründen Sie Ihre Antwort! \textbf{(3 Punkte)}
\end{enumerate} 
\clearpage
% -----------------------------------------------------------------------
\part{Programmieren in R}
% -----------------------------------------------------------------------

\section{Aufgabe \hfill (9 Punkte)}



 
%% --------------------------------------------------------------------
\begin{minipage}[t]{0.5\textwidth}
\includegraphics[width = 1.3cm]{/Users/kruppajo/work/GitHub/exam/avatare/Nilufar.png}
\end{minipage}
\begin{minipage}[t]{0.5\textwidth}
\hfill
\href{https://youtu.be/C9skfFRTHhI}{\includegraphics[width = 2cm]{img/caution}}
\end{minipage}
\vspace{1ex}
%% --------------------------------------------------------------------



Nilufar muss ihrer Abschlussarbeit mit \Rlogo arbeiten. Deshalb sitzt sie jetzt mit Ihnen zusammen und hat einige Fragen zu den Grundlagen in \Rlogo an Sie! Na dann wollen Sie mal helfen. Immerhin will ihre Betreuerin, dass \Rlogo genutzt wird.\\[1Ex]

Nilufar: \textit{Warum brauche ich eigentlich das RStudio und R? Wo ist denn da der Unterschied?} \textbf{(1 Punkt)}\\[1ex]
Sie antworten:\\[2Ex]

Nilufar: \textit{Ich habe doch die Spalte mutiert und geändert. Warum sehe ich das in R aber mein Datensatz ändert sich nicht?} \textbf{(1 Punkt)}\\[1ex]
Sie antworten:\\[2Ex]

Nilufar: \textit{Jetzt lese ich hier von einem Faktor. Was ist ein Faktor in \Rlogo?} \textbf{(1 Punkt)}\\[1ex]
Sie antworten:\\[2Ex]

Nilufar: \textit{Ich verstehe den Pipe-Operator nicht. Wie sieht der aus und was macht der? Gebe mal ein Beispiel!} \textbf{(1 Punkt)}\\[1ex]
Sie antworten:\\[2Ex]

Nilufar: \textit{Ich verstehe den Unterschied zwischen \texttt{library()} und \texttt{Packages} nicht. Warum gibt es die?} \textbf{(1 Punkt)}\\[1ex]
Sie antworten:\\[2Ex]

Nilufar: \textit{Wo nutzen wir nochmal die Tilde ($\sim$) in R. Das war irgendwie voll wichtig. Wo wird diese genutzt?} \textbf{(1 Punkt)}\\[1ex]
Sie antworten:\\[2Ex]

Nilufar: \textit{Der Zuweisungs-Operator wird sehr häufig genutzt. Wie sieht der aus und wie funktioniert der an einem Beispiel?} \textbf{(1 Punkt)}\\[1ex]
Sie antworten:\\[2Ex]

Nilufar: \textit{Ich habe den Namen der Funktion, die intern Daten speichert, vergessen. Was waren da nochmal die Vorteile?} \textbf{(1 Punkt)}\\[1ex]
Sie antworten:\\[2Ex]

Nilufar: \textit{Ich habe gehört, dass es Vorteile gibt \Rlogo zu nutzen. Nenne mir mal einen Vorteil!} \textbf{(1 Punkt)}\\[1ex]
Sie antworten:\\[2Ex] 
\clearpage
% -----------------------------------------------------------------------

\section{Aufgabe \hfill (9 Punkte)}



 
%% --------------------------------------------------------------------
\begin{minipage}[t]{0.5\textwidth}
\includegraphics[width = 1.3cm]{/Users/kruppajo/work/GitHub/exam/avatare/Jonas.png}
\end{minipage}
\begin{minipage}[t]{0.5\textwidth}
\hfill
\href{https://youtu.be/C9skfFRTHhI}{\includegraphics[width = 2cm]{img/caution}}
\end{minipage}
\vspace{1ex}
%% --------------------------------------------------------------------



'Hm...am Ende ist dann \Rlogo eigentlich gar nicht so schwer, wenn ich Hilfe habe.', meint  Jonas stolz und lacht Sie an. Nur leider kennt er sich überhaupt nicht mit \Rlogo aus! Das heißt, Sie müssen hier einmal Rede und Antwort stehen und helfen. Sonst wird es für Jonas dann in seinem Projektbericht nichts mit der Auswertung und Abgabe. Das kann auch keine Lösung für Jonas und Sie sein. Immerhin haben Sie schon so viel gelernt.\\[1Ex]

Jonas fragt: \textit{Ich hatte mir eine Analogie für das R Paket \texttt{\{ggplot\}} gemerkt. Wie war noch gleich die Analogie und das damit verbundene Prinzip von \texttt{\{ggplot\}}? \textbf{(2 Punkte)}}\\[1ex]
Sie antworten:\\[2Ex]

Jonas fragt: \textit{Warum wurde jetzt nach dem Laden der Daten die Funktion \texttt{mutate()} genutzt? \textbf{(1 Punkt)}}\\[1ex]
Sie antworten:\\[2Ex]

Jonas fragt: \textit{Oh, wie baue ich mir nochmal die Daten in R? Wie heißt das Dateiformat? Gerne einmal mit Beispiel! \textbf{(1 Punkt)}}\\[1ex]
Sie antworten:\\[2Ex]

Jonas fragt: \textit{Ich glaube ich habe Varianzheterogenität zwischen den Gruppen vorliegen. Wie funktioniert die Adjustierung dafür nochmal in  \texttt{emmeans()}? \textbf{(1 Punkt)}}\\[1ex]
Sie antworten:\\[2Ex]

Jonas fragt: \textit{Wie spezifizieren wir nochmal eine Interaktion in einem Modell mit zwei Faktoren $f_1$ und $f_2$? \textbf{(1 Punkt)}}\\[1ex]
Sie antworten:\\[2Ex]

Jonas fragt: \textit{Was muss ich bei der Eingabe eines Datums in Excel beachten, wenn ich später die Exceldatei in R einlesen will? Wie lautet das Format? \textbf{(1 Punkt)}}\\[1ex]
Sie antworten:\\[2Ex]

Jonas fragt: \textit{Ich möchte ein CLD erstellen. Welche Funktionen muss ich in welcher Reihenfolge nutzen? \textbf{(2 Punkte)}}\\[1ex]
Sie antworten:\\[2Ex]



 
\clearpage
% -----------------------------------------------------------------------
\part{Forschendes Lernen}

Das forschende Lernen basiert zum einen auf den folgenden wissenschaftlichen Veröffentlichungen. Für die Prüfung wird die vertiefende Kenntnis der folgenden Veröffentlichungen vorausgesetzt.\\

\textit{In der Prüfung erhalten Sie einen Auszug der wissenschaftlichen Veröffentlichung. Für die Einarbeitung in die Veröffentlichung ist in der Prüfung \underline{ausdrücklich} keine Zeit vorgesehen.}

\begin{itemize}[noitemsep]
\item Sánchez, M., Velásquez, Y., González, M., \& Cuevas, J. (2022). Hoverfly pollination enhances yield and fruit quality in mango under protected cultivation. Scientia Horticulturae, 304, 111320. [\href{https://www.sciencedirect.com/science/article/pii/S0304423822004411}{Link}]
% \item Salinas, I., Hueso, J. J., Força Baroni, D., \& Cuevas, J. (2023). Plant growth, yield, and fruit size improvements in ‘Alicia’papaya multiplied by grafting. Plants, 12(5), 1189. [\href{https://www.mdpi.com/2223-7747/12/5/1189}{Link}]
\item Petersen, F., Demann, J., Restemeyer, D., Olfs, H. W., Westendarp, H., Appenroth, K. J., \& Ulbrich, A. (2022). Influence of light intensity and spectrum on duckweed growth and proteins in a small-scale, re-circulating indoor vertical farm. Plants, 11(8), 1010. [\href{https://www.mdpi.com/2223-7747/11/8/1010}{Link}]
\item Selle, P. H., Cadogan, D. J., Li, X., \& Bryden, W. L. (2010). Implications of sorghum in broiler chicken nutrition. Animal Feed Science and Technology, 156(3-4), 57-74. [\href{https://www.sciencedirect.com/science/article/pii/S0377840110000209}{Link}]
\item Wu, G., Knabe, D. A., \& Kim, S. W. (2004). Arginine nutrition in neonatal pigs. The Journal of Nutrition, 134(10), 2783S-2790S. [\href{https://www.sciencedirect.com/science/article/pii/S0022316623031279}{Link}]
\end{itemize}

Das forschende Lernen basiert zum anderen auf den folgenden wissenschaftlichen Datensätzen und deren vertiefende Analyse werden als bekannt vorausgesetzt. Die Teilaufgaben der Aufgaben stellen nur eine zufällige Auswahl an möglichen Fragen dar. Die Datensätze werden über ILIAS bereitgestellt.\\

\textit{In der Prüfung erhalten Sie \underline{keinen Auszug} aus den wissenschaftlichen Daten. Die Datensätze werden als \underline{bekannt} in der Prüfung vorgesetzt. Sie haben sich vorab Notizen und Anmerkungen gemacht.}

\begin{itemize}[noitemsep]
\item bar
\end{itemize}

\clearpage
% -----------------------------------------------------------------------

\section{Aufgabe \hfill (20 Punkte)}

\textit{Geben Sie grundsätzlich Formeln und Rechenweg zur Lösung der Teilaufgaben mit an!} \\[1Ex]
 

 
%% --------------------------------------------------------------------
\begin{minipage}[t]{0.5\textwidth}
\includegraphics[width = 1.3cm]{/Users/kruppajo/work/GitHub/exam/avatare/Alex.png}
\end{minipage}
\begin{minipage}[t]{0.5\textwidth}
\hfill
\href{https://youtu.be/C9skfFRTHhI}{\includegraphics[width = 2cm]{img/caution}}
\end{minipage}
\vspace{1ex}
%% --------------------------------------------------------------------



Alex hält die wissenschaftliche Veröffentlichung \textit{Wu, G., et al. (2004). Arginine nutrition in neonatal pigs} unter einem Schnaufen in die Luft. 'Worum geht es denn eigentlich in dieser Arbeit?', fragt er stirnrunzelnd. Alex soll die Veröffentlichung nutzen um das eigene Experiment zu planen. Als eine Vorlage sozusagen. Daher möchte sein Betreuer, dass er einmal die Veröffentlichung sinnvoll zusammenfasst. Das sollte dann doch etwas aufwendiger werden.\\

Leider kennt sich Alex mit dem Lesen einer wissenschaftlichen Veröffentlichung mit Fokus auf die Statistik überhaupt nicht aus. Deshalb braucht er bei der Erstellung Ihre Hilfe! Glücklicherweise kennen Sie die wissenschaftliche Veröffentlichung schon im Detail und können sofort helfen.

\begin{enumerate}
  \setcounter{enumi}{0}
  \item Erläutern Sie die wissenschaftliche Fragestellung der wissenschaftlichen Veröffentlichung anhand des OCAR Prinzips nach Schimel (2012)\footnote{Schimel, J. (2012). Writing science: how to write papers that get cited and proposals that get funded. OUP USA.} \textbf{(4 Punkte)}
  \item Nennen Sie die untersuchten Endpunkte in der wissenschaftlichen Veröffentlichung! Wie lautet der primäre Endpunkt? \textbf{(2 Punkte)} 
\item Erstellen Sie das statistische Modell in der in \Rlogo üblichen Schreibweise! \textbf{(2 Punkte)}
  \item Nennen Sie eine Auswahl an bedeutenden statistischen Maßzahlen in der wissenschaftlichen Veröffentlichung! \textbf{(1 Punkt)}
  \item Interpretieren Sie die Hauptaussage der wissenschaftlichen Veröffentlichung hinsichtlich der Signifkanz für den primären Endpunkt! \textbf{(2 Punkte)}
  \item Interpretieren Sie die Hauptaussage der wissenschaftlichen Veröffentlichung hinsichtlich der Effektstärke für den primären Endpunkt! \textbf{(2 Punkte)}
  \item Diskutieren Sie die ökonomische Relevanz der Hauptaussage der wissenschaftlichen Veröffentlichung im Bezug auf Signifikanz und Effektstärke für den primären Endpunkt! \textbf{(1 Punkt)}
  \item Skizzieren Sie für den primären Endpunkt den sich ergebenden Datensatz in \Rlogo für eine ausgewählte Abbildung! \textbf{(2 Punkte)}
\item Skizzieren Sie einen möglichen Versuchsplan für den primären Endpunkt! \textbf{(2 Punkte)}
  \item Schätzen Sie die benötigte Fallzahl für ein zukünftiges Experiment anhand der Ergebnisse in der wisenschaftlichen Veröffentlichung für den primären Endpunkt! \textbf{(2 Punkte)}
\end{enumerate} 
\clearpage
% -----------------------------------------------------------------------

\section{Aufgabe \hfill (20 Punkte)}

\textit{Geben Sie grundsätzlich Formeln und Rechenweg zur Lösung der Teilaufgaben mit an!} \\[1Ex]
 

 
%% --------------------------------------------------------------------
\begin{minipage}[t]{0.5\textwidth}
\includegraphics[width = 1.3cm]{/Users/kruppajo/work/GitHub/exam/avatare/Alex.png}
\end{minipage}
\begin{minipage}[t]{0.5\textwidth}
\hfill
\href{https://youtu.be/C9skfFRTHhI}{\includegraphics[width = 2cm]{img/caution}}
\end{minipage}
\vspace{1ex}
%% --------------------------------------------------------------------



Unter einem langen Schnaufen starrt Alex auf den wissenschaftlichen Datensatz \textit{data2} in seinem Laptop. 'Worum geht es denn eigentlich in diesem Datensatz?', fragt er kopfschüttelnd. Alex soll die Datentabelle nutzen um das eigene Experiment zu planen und eine Blaupause zu haben. Als eine Vorlage sozusagen, die er nur noch ausfüllen muss. Daher möchte seine Betreuerin, dass er einmal die Daten sinnvoll zusammenfasst. Das sollte dann doch etwas aufwendiger werden.\\

Leider kennt sich Alex mit der Analyse eines wissenschaftlichen Datensatzes überhaupt nicht aus. Deshalb braucht er bei der Auswertung Ihre Hilfe! Glücklicherweise kennen Sie den wissenschaftlichen Datensatz aus Ihren eigenen Analysen schon im Detail und können sofort helfen.

\begin{enumerate}
  \setcounter{enumi}{0}
  \item Formulieren Sie die wissenschaftliche Fragestellung des Datensatzes in Form einer PowerPoint Folie! \textbf{(2 Punkte)}
  \item Nennen Sie zwei Besonderheiten des Datensatzes! Begründen Sie Ihre Antwort! \textbf{(2 Punkte)}
  \item Nennen Sie die untersuchten Endpunkte in dem Datensatz! Wie lautet der primäre Endpunkt für die Auswertung? \textbf{(2 Punkte)}
  \item Skizzieren Sie die großen Analysebereiche der Statistik! Beschriften Sie die Abbildungen! \textbf{(2 Punkte)}
  \item In welchen der großen Analysebereiche der Statistik fällt die Auswertung des primären Endpunktes? Begründen Sie Ihre Antwort! \textbf{(2 Punkte)}
  \item Skizzieren Sie eine ikonische Abbildung für den primären Endpunkt im Kontext der wissenschaftlichen Fragestellung! \textbf{(2 Punkte)}
  \item Erstellen Sie das statistische Modell in der in \Rlogo üblichen Schreibweise! \textbf{(2 Punkte)}
  \item Skizzieren Sie die Datenanalyse hinsichtlich der Signifkanz für den primären Endpunkt! \textbf{(2 Punkte)}
  \item Skizzieren Sie die Berechnung der Effektstärke für den primären Endpunkt! \textbf{(2 Punkte)}
  \item Skizzieren Sie einen möglichen Versuchsplan für den primären Endpunkt! \textbf{(2 Punkte)}
\end{enumerate} 
\clearpage
% -----------------------------------------------------------------------
\part{Mathematik}
% -----------------------------------------------------------------------  

\section{Aufgabe \hfill (10 Punkte)}

\textit{Geben Sie grunds{\"a}tzlich Formeln und Rechenweg zur L{\"o}sung der
  Teilaufgaben mit an!} \\[1Ex]

%% --------------------------------------------------------------------
\hfill\href{https://youtu.be/Fu8kN0Uj13Y}{\includegraphics[width =
  2cm]{img/youtube}} %%youtube
\hspace{2Ex}
%% --------------------------------------------------------------------

\paragraph{Herodot – der Schimmel aus Ivenack}

W{\"a}hrend der Besetzung Mecklenburgs durch die Franzosen kamen Napoleon die
Geschichten des ber{\"u}hmten Apfelschimmels Herodot aus Ivenack zu
Geh{\"o}r. Herodot lief zwar niemals Rennen, war aber eines der ber{\"u}hmtesten
Pferde dieser Zeit. Napoleon selbst gab den Auftrag, diesen
Schimmel durch die Armee nach Frankreich zu bringen. Der Legende nach
sollen Arbeiter den Schimmel im hohlen Stamm einer 1000-j{\"a}hrigen Eiche aus Ivenack vor
den Franzosen versteckt haben. Doch Herodot verriet sein Versteck durch
lautes Wiehern, woraufhin die franz{\"o}sische Armee den Schimmel
beschlagnahmte und nach Frankreich f{\"u}hrte\footnote{Die Quelle der Inspiration
  für die Aufgabe war eine Fahrt an die Ostsee und folgender Artikel:
  \href{https://www.wald-mv.de/landingpage/ivenacker-eichen/}{Entdecke das erste Nationale Naturmonument Deutschlands - Ivenacker Eichen und Hutewald}}. \\



\textit{Forschungsfrage: "Konnten die Ivenacker den Apfelschimmel Herodot
  vor dem Zugriff von Napoleon in der 1000-j{\"a}hrigen Eiche verstecken?"} \\

Gehen Sie von einem radialen Wachstum der 1000-j{\"a}hrigen Eiche von
$0.9mm$ pro Jahr aus. Es ist bekannt, dass die Eiche im
Jahr 2022 einen Umfang von $12.5m$ in Brusth{\"o}he hatte.

\begin{enumerate}
\item Wie gro{\ss} war der Durchmesser in $m$ der Eiche im Jahr $1820$ als
  Herodot in der Eiche versteckt werden sollte?
  \textbf{(2 Punkte)}
\item Skizzieren Sie in einer Abbildung einen linearen Zusammenhang und einen
exponentiellen Zusammenhang f{\"u}r das Wachstum der 1000-j{\"a}hrigen Eiche. Erkl{\"a}ren Sie die
Auswirkungen der Entscheidung f{\"u}r linear oder exponentiell auf Ihre
Berechnungen! \textbf{(2 Punkte)}
\end{enumerate}
 
Herodot hatte eine Schulterh{\"o}he von $195$cm, eine Breite von
$80$cm sowie eine L{\"a}nge von  $240$cm.

\begin{enumerate}
  \setcounter{enumi}{2}
\item Berechnen Sie das effektive Volumen von Herodot in $m^3$, welches
  Herodot in der 1000-j{\"a}hrigen Eiche einnehmen w{\"u}rde! \textbf{(2 Punkte)}
\end{enumerate}

Es wurde berichtet, dass sich Herodot in der 1000-j{\"a}hrigen Eiche
$m{"u}hsam$ um die eigene Achse drehen konnte.

\begin{enumerate}
  \setcounter{enumi}{3}
\item Berechnen Sie die Dicke der Eichenwand in $cm$! Verdeutlichen Sie Ihre
  Berechnungen an einer aussagekr{\"a}ftigen Skizze f{\"u}r Pferd und Eiche! \textbf{(2 Punkte)} 
\item Unter einer Dicke der Eichenwand von $20cm$ bricht
  die Eiche zusammen. Beantworten Sie die Forschungsfrage! Begr{\"u}nden Sie
  Ihre Antwort! \textbf{(2 Punkte)} 
\end{enumerate}
 
\clearpage
% ----------------------------------------------------------------------- 

\section{Aufgabe \hfill (10 Punkte)}

\textit{Geben Sie grunds{\"a}tzlich Formeln und Rechenweg zur L{\"o}sung der
  Teilaufgaben mit an!} \\[1Ex]

%% --------------------------------------------------------------------
\hfill\href{https://youtu.be/57B-yYoFSk0}{\includegraphics[width =
  2cm]{img/youtube}} %%youtube
\hspace{2Ex}
%% --------------------------------------------------------------------

\paragraph{Von T{\"o}pfen auf Tischen}



In einem Experiment wollen Sie die Wuchsh{\"o}he von 180
Maispflanzen bestimmen. Bevor Sie {\"u}berhaupt mit dem Experiment beginnen
k{\"o}nnen, gibt es aber ein paar Absch{\"a}tzungen {\"u}ber die Kosten und den Aufwand
zu treffen. Zum einen m{\"u}ssen Sie die Maispflanzen einpflanzen und m{\"u}ssen
daf{\"u}r Substrat bestellen. Zum anderen m{\"u}ssen Sie die Maispflanzen auch
bewegen und in ein Gew{\"a}chshaus platzieren. Die T{\"o}pfe f{\"u}r die Keimung haben
einen Durchmesser von 9cm und eine H{\"o}he von 8cm. Der
Kubikmeterpreis f{\"u}r Torf liegt bei 290 EUR.

\begin{enumerate}
\item Skizzieren Sie den Versuchsplan auf \textit{drei} Tischen im
  Gew{\"a}chshaus! \textbf{(2 Punkte)}
\item Berechnen Sie die ben{\"o}tigte Anzahl an Pflanzt{\"o}pfen, wenn Sie
  Randpflanzen mit ber{\"u}cksichtigen wollen! \textbf{(1 Punkt)}
\item Welche $Tisch$fl{\"a}che in $m^2$ gegeben der Anzahl an
  Pflanzt{\"o}pfen inklusive Randpflanzen ben{\"o}tigen Sie im Gew{\"a}chshaus am
  Anfang der Keimungsphase?  \textbf{(3 Punkte)}
\item Berechnen Sie die ben{\"o}tigte Menge an Torf in Liter $l$, die Sie f{\"u}r
  das Bef{\"u}llen der Pflanzt{\"o}pfe ben{\"o}tigen! Gehen Sie von \textit{einem
    Zylinder} f{\"u}r die Pflanzt{\"o}pfe aus!  \textbf{(3 Punkte)}
\item Berechnen Sie die Kosten in EUR f{\"u}r Ihre Torfbestellung! \textbf{(1
    Punkt)}
\end{enumerate}



 
\clearpage
% ----------------------------------------------------------------------- 

\section{Aufgabe \hfill (10 Punkte)}

\textit{Geben Sie grunds{\"a}tzlich Formeln und Rechenweg zur L{\"o}sung der
  Teilaufgaben mit an!} \\[1Ex]

%% --------------------------------------------------------------------
\hfill\href{https://youtu.be/aBxLkdF-c4M}{\includegraphics[width =
  2cm]{img/youtube}} %%youtube
\hspace{2Ex}
%% --------------------------------------------------------------------

\paragraph{Solar- \& Biogasanlagen}



Um die Energiekosten Ihres Betriebes zu senken, wollen Sie eine Solaranlage
auf den Rinderstall montieren lassen. Sie messen Ihren Stall und finden
folgende Ma{\ss}e wieder. Die vordere Seite des Rinderstall hat eine H{\"o}he
$h_v$ von $5.5m$. Die hintere Seite des Rinderstall hat eine
H{\"o}he $h_b$ von $9.5m$. Der Rinderstall hat eine Tiefe $t$ von
$13m$ und eine Breite $b$ von $70m$.

\begin{enumerate}
\item Skizzieren Sie den Rinderstall auf dem die Solaranlage montiert
  werden soll! Erg{\"a}nzen Sie die Angaben f{\"u}r die H{\"o}hen $h_v$, $h_b$, die
  Tiefe $t$ und die Breite $b$ des Stalls!  \textbf{(2 Punkte)}
\item Berechnen Sie die Fl{\"a}che der schr{\"a}gen, neuen Solaranlage auf dem
  Rinderstall! \textbf{(3 Punkte)}
\end{enumerate}

Ebenfalls planen Sie eine neue Biogasanlage f{\"u}r Ihren Betrieb. Der neue
Methantank hat einen Radius $r$ von $1.5m$. Leider gibt es ein
paar bauliche Beschr{\"a}nkungen auf dem Grundst{\"u}ck. Ihr Fundament des
zylindrischen Methantanks kann nur ein Gewicht von maximal
$12t$ aushalten bevor der Tank wegbricht. Sie rechnen eine
Sicherheitstoleranz von $25\%$ ein beinhaltend das Gewicht des
Methantanks. In fl{\"u}ssiger Form hat
Methan bei $-80^\circ\text{C}$ eine Dichte von
$220kg/m^3$. Bei $-100^\circ\text{C}$ hat Methan eine Dichte
von $300kg/m^3$. Sie betrieben Ihre Anlage bei
$-92^\circ\text{C}$.

\begin{enumerate}
  \setcounter{enumi}{2}
\item Extrapolieren Sie die effektive Dichte des Methans in Ihrem
  Methantank! Welche Annahme haben Sie getroffen? \textbf{(1 Punkt)}
\item Berechnen Sie wie viel Kubikmeter $m^3$ Sie in den Methantank f{\"u}llen
  k{\"o}nnen, bevor das Fundament nachgibt! \textbf{(2 Punkte)}
\item Berechnen Sie die maximale H{\"o}he $h_{max}$ in $m$ f{\"u}r den gef{\"u}llten
  Methantank mit dem Radius $r$, bevor das Fundament wegbricht! \textbf{(2
    Punkte)}
\end{enumerate}

 
\clearpage
% -----------------------------------------------------------------------

\section{Aufgabe \hfill (10 Punkte)}

\textit{Geben Sie grunds{\"a}tzlich Formeln und Rechenweg zur L{\"o}sung der
  Teilaufgaben mit an!} \\[1Ex]

%% --------------------------------------------------------------------
\hfill\href{https://youtu.be/https://youtu.be/k2G52hMIfqk}{\includegraphics[width = 2cm]{img/youtube}} %%youtube
\hspace{2Ex}
%% --------------------------------------------------------------------

%% --------------------------------------------------------------------
{\tiny\textbf{Stichworte:} Riesenfaultier $\bullet$ Evolution der Avocado $\bullet$ Bluetooth $ \bullet$ Blauzahn $\bullet$ Colonia Dignidad $\bullet$ ODESSA $\bullet$ Rattenlinie $\bullet$ Adolf Eichmann}
%% --------------------------------------------------------------------


\paragraph{Aligatorenbirnen und Blaubeeren}



"'Sind Sie ein Riesenfautier oder warum kaufen Sie so viele
Aligatorenbirnen?"', spricht es hinter Ihnen. Irritiert drehen Sie sich um
und blicken in das puderrote Gesicht von einem Studenten im Karohemd. "'Wieso?"', entfährt es
Ihnen und Sie bereuen sogleich die Frage. Sofort werden Sie zu einem
Whiteboard voller roter Schnüre geschliffen und müssen folgenden
mathematischen untermauerten Argumenten im Rewe über sich ergehen
lassen. Da kommen Sie nicht mehr raus, also können Sie auch gleich
mitmachen. Das Problem liegt in Chile\footnote{Die Quelle der Inspiration
  für die Aufgabe waren folgende Reportagen:
  \href{https://www.amnesty.ch/de/ueber-amnesty/publikationen/magazin-amnesty/2021-3/bis-zum-letzten-tropfen}{"`Bis
  zum letzten Tropfen"' in AMNESTY – Magazin der Menschenrechte vom August
  2021} und
\href{https://www.welthungerhilfe.de/welternaehrung/rubriken/klima-ressourcen/wassernot-in-chile-eine-folge-der-privatisierung}{"`Wasserknappheit
  in Chile: Eine Folge der Privatisierung?"' in Die Welternährung dem
  Fachjournal der Welthungerhilfe vom April 2022.}}. Tja, die Deutschen und Südamerika.\\

Zuerst werden Ihre Fähigkeiten getestet, der Mathematik folgen zu können. Oder berechnen Sie gerade den Einkauf von einem Studenten im Karohemd?\\

\begin{enumerate}
\item Wenn 5 Blaubeerschalen 7.95 Euro kosten,  wie viel kosten 9 Schalen? \textbf{(2 Punkte)}
\item Wenn Sie die 9 Blaubeerschalen gekauft haben, wie viele Aligatorbirnen zu je 2.89 EUR können Sie sich dann noch für 50 EUR leisten? \textbf{(1 Punkt)}
\end{enumerate}

Das Whiteboard beinhaltet folgende Liste mit Informationen zum Wasserverbrauch bei der Produktion von Gemüse aus Chile. Seltsam, was man so alles in einem Rewe über Gemüse erfährt.
  
\begin{itemize}[noitemsep]
\item Ein Kilo Strauchtomaten benötigt 190l Wasser. Eine Strauchtomate wiegt 90 - 120g.
\item Ein Kilo Salat benötigt 120l Wasser. Ein Salatkopf wiegt 310 - 510g.
\item Ein Kilo Avocado benötigt 1050l Wasser. Eine Avocado wiegt 150 - 420g.
\item Ein Kilo Blaubeeren benötigt 830l Wasser. Eine Blaubeere wiegt 3.1 - 3.5g.
\end{itemize}

\begin{enumerate}
  \setcounter{enumi}{2}
\item Berechnen Sie den Wasserverbrauch für die Produktion für jeweils eine Strauchtomate, einem Salat, einer Avocado und einer Blaubeeren. Stellen Sie das Ergebnis als Tabelle dar! \textbf{(3 Punkte)}
\end{enumerate}

Chile exportiert im großem Ausmaß Blaubeeren und Avocados. In dem Exportjahr 2021 blieben die Erträge von Blaubeeren mit \ensuremath{9\times 10^{4}}t in dem prognostizierten Rahmen. Die Menge steigerte sich um 8.6\%. Die Exporte für Avocados stiegen in dem gleichen Zeitraum um 22.1\% auf \ensuremath{1.9\times 10^{5}}t.

\begin{enumerate}
  \setcounter{enumi}{3}
\item Wie viele Tonnen Wasser hat Chile in dem Exportjahr 2020 exportiert? \textbf{(2 Punkte)}
\end{enumerate}

Chile ist eines der wenigen Länder der Welt, die ihr Wasser komplett privatisiert haben. Derzeit sind nur drei Prozent des Wassers des Landes für den häuslichen Verbrauch vorgesehen. In den Dörfern der Anbauregionen versorgen Tankwagen die Bevölkerung jede Woche mit Wasser, es gibt etwa 61 Liter Wasser pro Kopf für den täglichen Bedarf. In \textit{Deutschland} liegt der Verbrauch bei 35 - 115 Liter pro Waschgang einer Waschmaschine und 9 - 14 Liter pro Sp{"u}lgang.

\begin{enumerate}
  \setcounter{enumi}{4}
\item Mit der rationierten Wassermenge aus Chiles Anbaugebieten können Sie in \textit{Deutschland} wie oft Ihren Bedarf stillen? \textbf{(1 Punkt)}
\end{enumerate}

Das alles hätten Sie nicht von einem Studenten im Karohemd erwartet. Ganz schön viele Informationen wurden da zusammengetragen.

\begin{enumerate}
  \setcounter{enumi}{5}  
  \item Nennen Sie eine \textit{Daten}quelle im Internet, wo Sie mehr Informationen zu landwirtschaftlichen Daten oder klimatischen, wirtschaftlichen und gesellschaftlichen Daten erhalten! \textbf{(1 Punkt)}
\end{enumerate} 
\clearpage
% ----------------------------------------------------------------------- 

\section{Aufgabe \hfill (12 Punkte)}

\textit{Geben Sie grunds{\"a}tzlich Formeln und Rechenweg zur L{\"o}sung der
  Teilaufgaben mit an!} \\[1Ex]

%% --------------------------------------------------------------------
\hfill\href{https://youtu.be/WZSxntiNF8s}{\includegraphics[width = 2cm]{img/youtube}} %%youtube
\hspace{2Ex}
%% --------------------------------------------------------------------

%% --------------------------------------------------------------------
{\tiny\textbf{Stichworte:} Kardaschow-Skala $\bullet$ Dyson-Sphäre $\bullet$ Hohlerde $\bullet$ Entropie $\bullet$ Proton $r_P = 1.7 \times 10e-15$ $\bullet$ Wasserstoff $r_H = 5.3\times 10e-11$}
%% --------------------------------------------------------------------

\paragraph{Die Dampfnudelerde}



"'Was für einen Unsinn!"', rufen Sie. Jetzt haben Sie kostbaren Schlaf prokrastiniert um einem Ernährungswissenschaftler auf YouTube über die Erde als Dampfnudel zu lauschen. Irgendwie passt es dann doch mit der Analogie. Übermüdet müssen Sie darüber nachdenken, warum vor 67 Millionen Jahren die Dinosaurier - so groß sie auch waren - nicht von der Schwerkraft zu Boden gerissen wurden. Hat der Dampfplauderer etwa recht und war die Schwerkraft vor Millionen von Jahren eine andere?  Sind deshalb alle Lebewesen auf der Erde \textit{heutzutage} so viel kleiner, weil die Schwerkraft größer ist als damals? War die Erde kleiner und hatte weniger Masse? Oder ist es nur ein Rechenfehler wie bei der Theorie der Hohlerde von Edmond Halley aus dem 17.–18. Jahrhundert? Müde reiben Sie sich die Augen. So wird es nichts mehr mit dem Schlafen, dann können Sie auch mal etwas rechnen\footnote{Die Quelle der Inspiration
  für die Aufgabe war folgender Artikel:
  \href{https://hpd.de/artikel/erde-dampfnudel-22236}{"Skeptische Anmerkungen --- Die Erde als Dampfnudel" in Der Humanistische Pressedienst}}.  \\

Betrachten wir die Schwerkraft oder Gewichtskraft, die auf Lebewesen damals und heute gewirkt haben soll. Nehmen Sie für die Fallbeschleunigung $g$ der Erde \textit{heutzutage} einen Wert von 9.81m/s$^2$ an. Im Weiteren hat die Erde einen ungefähren Durchmesser von \ensuremath{1.2742\times 10^{4}}km und eine mittlere Dichte $\rho$ von 5.51g/cm$^3$. Das Gewicht von einem heute lebenden Waldelefanten mit 2.7t liegt bei 6t und das Gewicht von einem Brachiosaurus bei bis zu 30t.

\begin{enumerate}
\item Welchen Durchmesser müsste die Erde vor 67 Millionen Jahren gehabt haben, wenn Dinosaurier und Elefanten die gleiche Gewichtskraft $\overrightarrow{F_G}$ damals und heute erfahren hätten? \textit{Beantworten Sie die Frage anhand der folgenden Teilaufgaben!}
\begin{enumerate}
\item Berechnen Sie die Fallbeschleunigung von vor 67 Millionen Jahren unter der obigen Annahme gleich wirkender Gewichtskraft $\overrightarrow{F_G}$ auf Elefant und Dinosaurier! \textbf{(1 Punkt)}
\item Berechnen Sie Masse der heutigen Erde! \textbf{(2 Punkte)}
\item Schließen Sie über die Masse auf den Durchmesser der Erde vor 67 Millionen Jahren! \textbf{(2 Punkte)}
\end{enumerate}
\item Beantworten Sie die Eingangsfrage mit 1-2 Antwortsätzen! \textbf{(1 Punkt)}
\end{enumerate}

Die Distanz zwischen Sonne und Erde entspricht 0.99 astronomische Einheiten ($AE$). Die Einheit 1 AE wird mit \ensuremath{1.48\times 10^{8}}km angegeben. Der \textit{massebehaftete} Sonnenwind besteht aus 81\% Wasserstoffkernen mit einer molaren Masse von 1.08g/mol, 11\% Heliumkernen mit 4.01g/mol sowie 8\% weiteren Atomkernen mit  89.32g/mol. Die Teilchendichte bei Eintritt in die Erdatmosphäre liegt zwischen 0.4 bis 100 Teilchen cm$^{-3}$ pro Sekunde mit einer mittleren Teilchendichte von 5cm$^{-3}$ pro Sekunde. \\

\textit{Lösen Sie den folgenden Aufgabenteil mit einer aussagekräftigen Skizze!}

\begin{enumerate}
  \setcounter{enumi}{3}
\item Berechnen Sie die Anzahl an massebehafteten Teilchen des Sonnenwindes, die die gesamte Erde pro Sekunde treffen! \textbf{(2 Punkte)}
\item Berechnen Sie die Anzahl an massebehafteten Teilchen des Sonnenwindes, die die Sonne pro Sekunde in alle Richtungen aussendet! \textbf{(2 Punkte)}
\item Berechnen Sie die Masse, die die Erde pro Jahr durch die \textit{massebehafteten} Teilchen des Sonnenwind zunimmt! \textbf{(2 Punkte)}
\end{enumerate}

%\blfootnote{\tiny\textbf{Stichworte:} Kardaschow-Skala $\bullet$ Dyson-Sphäre $\bullet$ Hohlerde $\bullet$ Entropie}

% https://de.wikipedia.org/wiki/Kardaschow-Skala 
\clearpage
% ----------------------------------------------------------------------- 

\section{Aufgabe \hfill (10 Punkte)}

\textit{Geben Sie grunds{\"a}tzlich Formeln und Rechenweg zur L{\"o}sung der
  Teilaufgaben mit an!} \\[1Ex]

%% --------------------------------------------------------------------
\hfill\href{https://youtu.be/n451XnhtSh4}{\includegraphics[width = 2cm]{img/youtube}} %%youtube
\hspace{2Ex}
%% --------------------------------------------------------------------

\paragraph{Entschuldigung, ist das Ihre Feder in meinem Auge?}



So h{\"o}rt man h{\"a}ufiger h{\"o}fliche G{"a}nse in Mastst{\"a}llen sagen. Das
ist nat{\"u}rlich etwas ung{\"u}nstig, den dann kommt es zu Picken und
Kannibalismus. Denn wenn der Nachbar nervt, dann muss zu Ma{\ss}nahmen
gegriffen werden. Kennt jeder aus einer mittelm{\"a}{\ss}igen Wohngemeinschaft. Das
wollen wir aber als vorsorgliche G{"a}nse-Halter:innen
nicht\footnote{Die Quelle der Inspiration f{\"u}r die Aufgabe war der folgende
  wissenschaftliche Artikel:
  \href{https://www.efsa.europa.eu/en/efsajournal/pub/7788}{EFSA Panel on
    Animal Health and Welfare, et al. (2023) Welfare of broilers on
    farm. EFSA Journal 21.2}.}. Betrachten wir also einmal das Platzangebot
(eng. \textit{space allowance}, abk. \textit{SA}) der G{"a}nse
f{\"u}r vier T{\"a}tigkeiten und versuchen die notwendige Fl{\"a}che zu optimieren. Wie
immer gibt es daf{\"u}r eine mathematische Formel:


\begin{center}
  \begin{tabular}{cc}
    $SA = \sum^n_{i = 1} (A_i \times PB_i)$ & $A_i = \pi \times (r_i + R_i)^2$\\
  \end{tabular}
\end{center}

\vspace{-2Ex}

mit

\begin{itemize}[noitemsep]
\item $SA$ dem ben{\"o}tigten Platzangebot aller aufsummierten Verhalten $i$.
\item $A_i$ dem ben{\"o}tigten Platz f{\"u}r ein Verhalten $i$. 
\item $PB_i$ dem Anteil des Auftretens eines Verhaltens $i$.
\item $r_i$ dem Radius Gans plus dem ben{\"o}tigten Radius f{\"u}r das Verhalten $i$.
\item $R_i$ dem notwendigen Abstand zu den Nachbarn f{\"u}r das Verhalten $i$.    
\item $i$ dem Verhalten: (1) standing, (2) sitting, (3)
  foraging incl. scratching und (4) walking.
\end{itemize}

In der folgenden Tabelle 1 sind die Werte f{\"u}r $r_i$, $R_i$ und $PB_i$ f{\"u}r
ein spezifisches Verhalten $i$ aus drei wissenschaftlichen
Ver{\"o}ffentlichungen dargestellt.

\vspace{-1Ex}

{\small
\begin{knitrout}
\definecolor{shadecolor}{rgb}{0.969, 0.969, 0.969}\color{fgcolor}\begin{table}[!h]
\centering
\begin{tabular}{llll}
\toprule
  & Aldridge et al. (2021) & Baxter et al. (2022) & Jabcobs et al. (2019)\\
\midrule
standing & 36cm; 22cm; 8.9\% & 33cm; 27cm; 3.5\% & 40cm; 22cm; 10.1\%\\
sitting & 36cm; 22cm; 50.1\% & 30cm; 17cm; 64.1\% & 39cm; 26cm; 25.4\%\\
foraging incl. scratching & 35cm; 24cm; 2.1\% & 40cm; 24cm; 4.1\% & 38cm; 27cm; 1.8\%\\
walking & 39cm; 23cm; 0.2\% & 38cm; 22cm; 0.2\% & 41cm; 29cm; 0.4\%\\
\bottomrule
\end{tabular}
\end{table}

\end{knitrout}
}

\begin{enumerate}
\item Erstellen Sie eine zusammenfassende Tabelle mit den mittleren Werten
  f{\"u}r $r$, $R$ und $PB$ aus der obigen Tabelle 1 f{\"u}r die jeweiligen
  Verhalten! \textbf{(3 Punkte)}
\item Erg{\"a}nzen Sie eine Spalte mit dem ben{\"o}tigten Platz $A$ f{\"u}r das jeweilige
  Verhalten, welches sich aus den mittleren Werten ergibt! \textbf{(1 Punkt)}
\item Berechnen Sie das ben{\"o}tigte Platzangebot $SA$ f{\"u}r alle betrachteten
  Verhalten! \textbf{(1 Punkt)}
\item Skizzieren Sie die Werte $r_i$, $R_i$ und $A_i$ f{\"u}r zwei
  nebeneinander agierende G{"a}nse f{\"u}r ein Verhalten $i$. Nutzen
  Sie hierf{\"u}r vereinfachte geometrische Formen! \textbf{(2 Punkte)}
\item Sie entnehmen der Literatur folgende Aussage zur Verteilung der
  G{"a}nse in der Fl{\"a}che $A$: \textit{"`Assuming, that the
    animals will optimally and equally distribute in an area $A$, we observe a
    small part, which is not covered. This area is called
    $\omega$ and is calculated with $\omega = \tfrac{A}{0.9069}$."'
  } Veranschaulichen Sie die Fl{\"a}che $\omega$ in einer aussagekr{\"a}ftigen
  Abbildung!  \textbf{(1 Punkt)}
\item Ein Tier braucht Platz f{\"u}r sich selbst. Berechnen Sie nun die
  K{\"o}rperfl{\"a}che $a$, die ein Tier einnimmt. Welche Annahmen haben Sie f{\"u}r
  die Berechnung der K{\"o}rperfl{\"a}che getroffen? \textbf{(2 Punkte)}
\end{enumerate}



 
\clearpage
% ----------------------------------------------------------------------- 

\section{Aufgabe \hfill (8 Punkte)}

\textit{Geben Sie grunds{\"a}tzlich Formeln und Rechenweg zur L{\"o}sung der
  Teilaufgaben mit an!} \\[1Ex]

%% --------------------------------------------------------------------
\hfill\href{https://youtu.be/1B53cVFIU7Q}{\includegraphics[width =
  2cm]{img/youtube}} %%youtube
\hspace{2Ex}
%% --------------------------------------------------------------------

\paragraph{Nelken von den Molukken}



In der Ausstellung "`Europa und das Meer"' im Deutschen Historischen Museum in
Berlin gab es folgendes Zitat {\"u}ber die Probleme der fr{\"u}hen Hochseeschifffahrt.

\begin{quote}
  >>Ohne ausreichende Zufuhr von Vitamin C stellen sich nach 50 Tagen die
  ersten Symptome ein; die ersten Toten sind nach 60 Tagen zu beklagen;
  nach 115 Tagen rafft die Skorbut eine ganze Schiffsbesatzung dahin<<
\end{quote}

Ferdinand Magellan stach im Jahre 1519 in See um eine Passage durch den
s{\"u}damerikanischen Kontinent zu finden. Zu seiner Flotte geh{\"o}rten
f{\"u}nf Schiffe - das Flaggschiff Trinidad, die San Antonio, die Victoria, die
Concepci{\'o}n und die Santiago - mit einer Besatzung von insgesamt
222 Mann. 

\begin{enumerate}
\item Stellen Sie den Verlauf der Anzahl an Matrosen auf einem Schiff der
  Flotte in der Form einer {\"U}berlebenszeitkurve dar! Beschriften Sie die
  Achsen entsprechend! \textbf{(2 Punkte)} 
\item Was ist die Besonderheit der Überlebenszeitkurve? Begründen Sie Ihre Antwort! \textbf{(2 Punkte)} 
\item Sch{\"a}tzen Sie die {\"U}berlebenswahrscheinlichkeit nach 95 Tagen
  aus Ihrer Abbildung ab! \textbf{(1 Punkt)} 
\end{enumerate}

Der Chronist an Bord der Trinidad, Antonio Pigafetta, schrieb in seinem
Bericht "`[...] Um nicht Hungers zu sterben, a{\ss}en wir das Leder, mit dem
die gro{\ss}e Rahe zum Schutz der Taue umwunden war."' Insbesondere die
Mannschaft der Concepci{\'o}n erlitt gro{\ss}e Verluste durch die Skrobut bei der
{\"U}berquerung des Pazifiks, da durch Erkundungsfahrten weniger Zeit blieb, um
wilden Sellerie aufzunehmen. Wilder Sellerie enth{\"a}lt
$8000\mu g/100mg$ Vitamin C. Der Bedarf liegt bei
$120mg$ pro Tag f{\"u}r M{\"a}nner.

\begin{enumerate}
  \setcounter{enumi}{2}
\item Berechnen Sie die notwendige Menge in $t$ an
  aufzunehmenden wilden Sellerie auf die Concepci{\'o}n f{\"u}r die ununterbrochene
  Fahrt von drei Monate und 20 Tage {\"u}ber den Pazifik!
  \textbf{(3 Punkte)}
\item Skizzieren Sie die {\"U}berlebenszeitkurve f{\"u}r die Concepci{\'o}n im
  Vergleich zu der {\"U}berlebenszeitkurve der Trinidad! Beschriften Sie die
  Achsen! \textbf{(2 Punkte)}
\end{enumerate}

 
\clearpage
% ----------------------------------------------------------------------- 

\section{Aufgabe \hfill (10 Punkte)}

\textit{Geben Sie grunds{\"a}tzlich Formeln und Rechenweg zur L{\"o}sung der
  Teilaufgaben mit an!} \\[1Ex]

%% --------------------------------------------------------------------
\hfill\href{https://youtu.be/q-qYK4Chslg}{\includegraphics[width =
  2cm]{img/youtube}} %%youtube
\hspace{2Ex}
%% --------------------------------------------------------------------

\paragraph{Event Horizon -- Am Rande des Universums}



Die Sonne hat eine aktuelle, angenommene Masse von $\ensuremath{2\times 10^{27}}$kg. Wenn
die Sonne nun am Ende ihrer Lebenszeit zu einem schwarzen Loch mit dem
Radius von $3000$m kollabiert, wird die Sonne $45$\%
der aktuellen Masse verloren haben. Ein Lichtteilchen mit der Masse $m_f$
und der Fluchtgeschwindigkeit $v_f$ will dem schwarzen Loch entkommen.  Sie
haben folgende Formeln f{\"u}r die kinetische Energie des Lichtteilchens
$E_{kin}$ und der Graviationsenergie des schwarzen Lochs $E_{grav}$
gegeben\footnote{Die Quelle der Inspiration
  für die Aufgabe war ein Montagnachtfilm:
  \href{https://de.wikipedia.org/wiki/Event_Horizon_–_Am_Rande_des_Universums}{Event Horizon – Am Rande des Universums}}.

\begin{center}
  \begin{tabular}{cc}
    $E_{kin} = \cfrac{1}{2}m_fv_f^2$ & $E_{grav} = \cfrac{Gm_sm_f}{r_s}$\\
  \end{tabular}
\end{center}

mit

\begin{itemize}[noitemsep]
\item $m_f$, gleich der Masse [kg] des fliehenden Objektes
\item $m_s$, gleich der Masse [kg] des station{\"a}ren Objekts
\item $r_s$, gleich dem Radius [m] des station{\"a}ren Objekts  
\item $G$, gleich der Gravitationskonstante mit $5.974 \cdot 10^{-11}
  m^3(kg \cdot s^2)^{-1}$ 
\end{itemize}

Im Folgenden wollen wir uns mit der Frage besch{\"a}ftigen, ob das
Lichtteilchen der Gravitation des schwarzen Lochs entkommen kann.

\begin{enumerate}
\item Geben Sie die Formel f{\"u}r die Fluchtgeschwindigkeit $v_f$ an! 
  \textbf{(1 Punkt)}
\item {\"U}berpr{\"u}fen Sie Ihre umgestellte Formel nach $v_f$ anhand der Einheiten!
  \textbf{(1 Punkt)} 
\item Berechnen Sie die notwendige Fluchtgeschwindigkeit $v_f$ des
  Lichtteilchens mit den angegebenen Informationen! \textbf{(2 Punkte)}
\item Gehen Sie von einer Lichtgeschwindigkeit von $\ensuremath{2.9\times 10^{8}}m/s$
  aus. Kann das Lichtteilchen der Gravitation des schwarzen Lochs
  entkommen? Begr{\"u}nden Sie Ihre Antwort! \textbf{(2 Punkte)}
\item Stellen Sie den Zusammenhang zwischen dem sich verringernden Radius
  $r$ des schwarzen Lochs bei gleichbleibender Masse $m_s$
  und der notwendigen Fluchtgeschwindigkeit $v_f$ in einer Abbildung dar!
  \textbf{(2 Punkte)}
 \item Ein Auto und ein {"A}ffchen st{\"u}rzen aus gro{\ss}er und gleicher H{\"o}he
  in ein schwarzes Loch. Welches der beiden Objekte {\"u}berschreitet zuerst
  den Ereignishorizont des schwarzes Loches? Begr{\"u}nden
  Sie Ihre Antwort mathematisch! \textbf{(2 Punkte)}  
\end{enumerate}

 
\clearpage
% -----------------------------------------------------------------------

\section{Aufgabe \hfill (10 Punkte)}

\textit{Geben Sie grunds{\"a}tzlich Formeln und Rechenweg zur L{\"o}sung der
  Teilaufgaben mit an!} \\[1Ex]

%% --------------------------------------------------------------------
\hfill\href{https://youtu.be/iCQogS6KhPM}{\includegraphics[width =
  2cm]{img/youtube}} %%youtube
\hspace{2Ex}
%% --------------------------------------------------------------------

\paragraph{Das Fermi Paradoxon}



Der Kernphysiker Enrico Fermi diskutierte 1950 auf dem Weg zum Mittagessen
im Los Alamos National Laboratory mit seinen Kollegen angebliche
UFO-Sichtungen und fragte schlie{\ss}lich: "`Where is everybody?"'. Warum seien
weder Raumschiffe anderer Weltraumbewohner noch andere Spuren
extraterrestrischer Technik zu beobachten? Wie lange w{\"u}rde eine au{\ss}erirdische
Zivilisation ben{\"o}tigen um die gesamte Milchstra{\ss}e zu
besuchen, wenn das maximale Reisetempo die Geschwindigkeit der Voyager 1 Sonde w{\"a}re?\footnote{Die Quelle der Inspiration
  für die Aufgabe war folgender Wikipediaeintrag:
  \href{https://de.wikipedia.org/wiki/Fermi-Paradoxon}{Fermi-Paradoxon}}\\[-1ex]

Wir treffen folgende Annahmen. Eine au{\ss}erirdische Zivilisation schickt $drei$
Voyager 1 {\"a}hnliche Sonden mit der Geschwindigkeit von $\ensuremath{6.0523\times 10^{4}}km/h$
los um sich auf den erreichten Planeten selbst zu replizieren. Nach
$750$ Jahren ist die Replikation abgeschlossen und wiederum
$drei$ Sonden werden ausgesendet. Gehen Sie von
$5.16$ Lichtjahren als mittlerer Abstand der Sterne in der
Milchstra{\ss}e aus. Es gibt $\ensuremath{10^{11}}$ Sterne in der Milchstra{\ss}e. Nehmen
Sie eine Lichtgeschwindigkeit von $\ensuremath{2.8\times 10^{8}}m/s$ an.

\begin{enumerate}
\item Skizzieren Sie in einer Abbildung die ersten vier Schritte der
  Vervielf{\"a}ltigung der Sonden in der Galaxie! Beschriften Sie die Abbildung
  mit der Dauer und der Anzahl an Sonden f{\"u}r jeden Schritt der Vervielf{\"a}ltigung! \textbf{(4 Punkte)}
\item Berechnen Sie die theoretische Anzahl an Vervielf{\"a}ltigungsschritten die
  ben{\"o}tigt werden um mit \textit{einem einzigen Vervielf{\"a}ltigungsschritt} die
  gesamten Sterne der Milchstra{\ss}e mit Sonden zu besuchen! \textbf{(2 Punkte)}
\item Berechnen Sie die Dauer, die eine au{\ss}erirdische Zivilisation
  ann{\"a}hrungsweise ben{\"o}tigt um die gesamten Sterne der Milchstra{\ss}e mit
  Sonden zu besuchen! \textbf{(2 Punkte)}
\item Bei einem vermutetet Alter der Erde von $\ensuremath{4.5\times 10^{9}}$ Jahren,
  wie oft war dann eine Sonde einer au{\ss}erirdischen Zivilisation schon zu
  Besuch? Korrigieren Sie Ihre Antwort mit dem Wissen, dass sich die
  Kontinentalplatten einmal alle $\ensuremath{10^{8}}$ Jahre vollst{\"a}ndig im
  Erdinneren umgewandelt haben! \textbf{(2 Punkte)}
\end{enumerate}


 
\clearpage
% -----------------------------------------------------------------------

\section{Aufgabe \hfill (10 Punkte)}

\textit{Geben Sie grunds{\"a}tzlich Formeln und Rechenweg zur L{\"o}sung der
  Teilaufgaben mit an!} \\[1Ex]

%% --------------------------------------------------------------------
\hfill\href{https://youtu.be/tDgr6fpkkYA}{\includegraphics[width =
  2cm]{img/youtube}} %%youtube
\hspace{2Ex}
%% --------------------------------------------------------------------

\paragraph{Pyramiden bauen}



Es stehen die mecklemburgischen Pyramidentage an und Sie sind auf
abenteuerlichen Wegen f{\"u}r den Bau der Pyramiden zust{\"a}ndig. Zu allem
{\"U}berfluss handelt es sich auch noch eine \textit{Reenactment}
Veranstaltung. Thema der diesj{\"a}hrigen Pyramidentage sind die Pyramiden von
Meroe, die den K{\"o}nigen und K{\"o}niginnen des historischen Reiches von Kusch in
Nubien, dem heutigen Sudan, als Grabst{\"a}tten dienten. Die Pyramiden in Meroe
fallen durch ihren steilen Winkel von 74 Grad im Vergleich zu
den {\"a}gyptischen Pyramiden mit 54 Grad auf. Die durchschnittliche
Seitenl{\"a}nge der Grundfl{\"a}che einer Pyramide betr{\"a}gt 44 K{\"o}nigsellen. Eine K{\"o}nigselle
misst 52.4cm.\\

\textit{L{\"o}sen Sie diese Aufgabe mit Hilfe einer Skizze der Pyramide. Bezeichnen
  Sie Seiten und die Winkel der Pyramide entsprechend!}

\begin{enumerate}
\item Bei der K{\"o}nigspyramide von Meroe soll eine Seitenl{\"a}nge der Grundfl{\"a}che
  44 K{\"o}nigsellen lang sein. Welche H{\"o}he der K{\"o}nigspyramide in $m$
  ergibt sich? \textbf{(1 Punkt)}
\item Die Au{\ss}enfl{\"a}chen der Pyramide soll begr{\"u}nt werden. F{\"u}r die
  Bepflanzung muss eine 4cm dicke Torfschicht auf die
  Pyramide aufgebracht werden. Berechnen Sie die ungef{\"a}hre Menge an
  ben{\"o}tigten Torf in $m^3$! \textbf{(2 Punkte)}
\end{enumerate}

Wie in jedem guten \textit{Reenactment} gibt es viel Oberschicht, aber nur
2 Sklaven, die Ihnen bei dem Bef{\"u}llen der Pyramide mit Schutt
zu Seite stehen. Leider haben Ihre Sklaven zu allem {\"U}berfluss auch noch
chronische Knieschmerzen entwickelt, als sie von der
anstehenden Aufgabe erfahren haben. Gehen Sie daher von einer Effizienz der
Sklaven von 85\% aus. In eine Schubkarre passen
100 Liter.

\begin{enumerate}
  \setcounter{enumi}{2}
\item Wie oft m{\"u}ssen Ihre maladen Sklaven die Rampe mit der Schubkarre
  zur Spitze der Pyramide hochfahren um die Pyramide mit Schutt zu f{\"u}llen? \textbf{(1 Punkt)}
\item Berechnen Sie die L{\"a}nge der Rampe zur Spitze der Pyramide mit einem
  Anstellwinkel von $14^\circ$! \textbf{(2 Punkte)}
\item Wie weit reicht Ihre Rampe vom Fu{\ss} der Pyramide in die mecklemburgische
  Landschaft?  \textbf{(2 Punkte)}
\end{enumerate}

Bei der Besichtigung der Pyramide teilt Ihnen der leicht {\"u}bergewichtige
Pharao (Nebenberuf \textit{Chefredaktuer}) mit, das die Pyramide zu
flach sei und somit nicht in die mecklemburgische Landschaft passen
w{\"u}rde. Sie m{\"u}ssen nochmal ran.

\begin{enumerate}
  \setcounter{enumi}{5}
\item Die Grundfl{\"a}che der Pyramide {\"a}ndert sich nicht. Berechnen Sie die
  {\"A}nderung der H{\"o}he in K{\"o}nigsellen, wenn sich der Anstellwinkel der
  Pyramide um $7^\circ$ {\"a}ndert!  \textbf{(2 Punkte)}
\end{enumerate}



% https://de.wikipedia.org/wiki/Rechtwinkliges_Dreieck
% https://www.matheretter.de/wiki/pyramide
% https://de.wikipedia.org/wiki/Knickpyramide 
\clearpage
% -----------------------------------------------------------------------

\section{Aufgabe \hfill (12 Punkte)}

\textit{Geben Sie grunds{\"a}tzlich Formeln und Rechenweg zur L{\"o}sung der
  Teilaufgaben mit an!} \\[1Ex]

%% --------------------------------------------------------------------
\hfill\href{https://youtu.be/3LAq3R0rS14}{\includegraphics[width =
  2cm]{img/youtube}} %%youtube
\hspace{2Ex}
%% --------------------------------------------------------------------

\paragraph{Geocaching -- Von Satelliten und Plastikdosen}



Es ist Wochenende und das Wetter ist \textit{sweet}. Sie schwingen sich auf
Ihr Cachermobil um mit 17km/h, geleitet von modernster
Satellietentechnologie und einem Supercompter aus dem Jahr 2000 in Ihren
H{\"a}nden, Plastikdosen in der Natur und an sehenswerten Orten zu finden. Sie
wollen diesmal endlich die aufw{"a}rts Terrainchallenge
durchf{\"u}hren. Die Reihenfolge der Caches nach Terrainwertung gibt daher
die von Ihnen abzufahrenden Orte vor. Die Terrain- und
Schwierigkeitswertungen laufen von 1 (leichteste Wertung) bis 5
(schwierigste Wertung) in 0.5 Schritten. Folgende Informationen zu den
Orten und den entsprechenden Caches stehen Ihnen f{\"u}r Ihre Planung der Route
zu Verf{\"u}gung\footnote{Die Quelle der Inspiration
  für die Aufgabe war folgende Tätigkeit:
  \href{https://www.geocaching.com/play}{Geocaching -- Mach mit bei der weltweit größten Schatzsuche.}}.

\begin{center}
  \begin{tabular}{ ccc }
    \toprule
    Ort & Cache & Wertung (S|T|G) \\
    \midrule
    A & GC3QNER & 3.5 | 1.5 | Normal \\
    B & GCX0Y3R & 5.0 | 2.5 | Mikro \\ 
    C & GCAHL5C & 2.5 | 1.0 | Mikro \\ 
    D & GCNP8YT & 2.0 | 2.0 | Mikro \\ 
    E & GCGETFY & 3.0 | 3.0 | Mikro \\     
 \bottomrule
\end{tabular}
\end{center}

Im Weiteren sind Ihnen folgende Informationen zu den Entfernungen der Orte
zug{\"a}nglich. Der Entfernungsvektor $\overrightarrow{AC}$ ist
$4$km. Im Weiteren ist Ihnen der Entfernungsvektor
$\overrightarrow{CB}$ mit $5.5$km bekannt. Der
Entfernungsvektor $\overrightarrow{BE}$ ist das $1.5$-fache
des Entfernungsvektor $\overrightarrow{CB}$. Wenn Sie von dem Ort A den Ort
C anpeilen, so liegt der Ort B ungef{\"a}hr $30^\circ$
s{"u}dlich. Wenn Sie von dem Ort C den Ort B anpeilen, so liegt
der Ort D ungef{\"a}hr $50^\circ$ {\"o}stlich. Vom Ort B
betrachtet, bilden die Orte C und D einen rechten Winkel am Ort B. Der Ort
B liegt auf gerader Linie zwischen den Orten C und E. Somit liegt der Ort E
s{"u}dlich von B. Die Strecke zwischen A und E ist nicht
passierbar. Sie starten an dem Ort C Ihre Cachertour. \\

\begin{enumerate}
\item L{\"o}sen Sie diese Aufgabe mit Hilfe einer aussagekr{\"a}ftigen Skizze der
  Orte und Caches. Bezeichnen Sie die Strecken und die Winkel Ihrer Skizze
  entsprechend! \textbf{(2 Punkte)}
\item Welche Strecke in $km$ legen Sie bei der Bew{\"a}ltigung der
  aufw{"a}rts Terrainchallenge zur{\"u}ck? \textbf{(5
    Punkte)}
\item Gehen Sie von einer zus{\"a}tzlichen Suchzeit in Stunden f{\"u}r die
  Caches an den jeweiligen Orten zur reinen Reisezeit mit Ihrem Cachermobil
  aus. Die Suchzeit in Stunden f{\"u}r \textit{jeden einzelnen} Cache wird durch die
  Funktion
  \begin{equation*}
    Suchzeit = 0.1 + 0.13 \cdot Schwierigkeit
  \end{equation*}  
  beschreiben.  Wie lange in Stunden ben{\"o}tigen Sie um die
  aufw{"a}rts Terrainchallenge zu erf{\"u}llen? \textbf{(3 Punkte)}
\item An der h{\"o}chsten Schwierigkeit m{\"u}ssen Sie angeln. Ihre Angel ist
  ausgefahren 8m lang. Erreichen Sie einen Cache in der H{\"o}he
  von 9.9m?  Berechnen Sie dazu Ihre maximale m{\"o}gliche
  Angelh{\"o}he! Welche Annahmen mussten Sie treffen um die Aufgabe zu l{\"o}sen? \textbf{(2 Punkte)} 
\end{enumerate}

 
\clearpage
% -----------------------------------------------------------------------

\section{Aufgabe \hfill (10 Punkte)}

\textit{Geben Sie grundsätzlich Formeln und Rechenweg zur L{\"o}sung der
  Teilaufgaben mit an!} \\[1Ex]

%% --------------------------------------------------------------------
\hfill\href{https://youtu.be/4-dSaPMhK9s}{\includegraphics[width =
  2cm]{img/youtube}} %%youtube
\hspace{2Ex}
%% --------------------------------------------------------------------

%% --------------------------------------------------------------------
{\tiny\textbf{Stichworte:} Brot aus Luft $\bullet$ Walöl $\bullet$ Haber-Bosch-Verfahren $\bullet$ 1. Weltkrieg $\bullet$ 40\% N im menschlichen Körper $\bullet$ Positivist}
%% --------------------------------------------------------------------

\paragraph{Die atmende Wand und Brot aus Luft}



Als Kellerkind vom Dorf wollen das Ausmaß der Radonbelastung in ihrem Kellerzimmer bestimmen und
lüften daher nicht. Spart dann auch Energie und lüften wird sowieso überschätzt. Während einer Messperiode von 7:00 Uhr bis
21:00 bestimmen Sie dreimal automatisch die Radonbelastung in
Ihrem Kellerraum in $Bq/m^3$. Es ergibt sich folgende Abbildung\footnote{Die Quelle der Inspiration
  für die Aufgabe war folgender Artikel:
  \href{https://de.wikipedia.org/wiki/Atmende_Wand}{Atmende Wand}}.

\begin{knitrout}
\definecolor{shadecolor}{rgb}{0.969, 0.969, 0.969}\color{fgcolor}

{\centering \includegraphics[width=\maxwidth]{img/math-10-1} 

}


\end{knitrout}

\vspace{-0.75cm}

\begin{enumerate}
\item Wie lange dauert es in Stunden bis Sie eine kritische Belastung von
  400$Bq/m^3$ in Ihrem ungelüfteten Kellerraum erreicht haben?
  \textbf{(2 Punkte)}
\end{enumerate}

Radon zerfällt mit einer Halbwertszeit von 1.8d zu
Polonium. Polonium wiederum zerfällt mit einer Halbwertszeit von
135d zu Blei. Nur Radon und Polonium tragen zur
radioaktiven Strahlenbelastung bei.

\begin{enumerate}
  \setcounter{enumi}{1}
\item Wie lange dauert es in Stunden bis Ihre kritische Radonbelastung von
  400$Bq/m^3$ auf unter 80$Bq/m^3$ gefallen ist?
  \textbf{(4 Punkte)}
\end{enumerate}

Folgende Tabelle enthält die Informationen zur Zusammensetzung der
normalen Umgebungsluft.

\begin{center}
  \begin{tabular}{ c|c|c|c }
     & Vol-\% & M [g/mol] & ppm \\
    \hline
    Stickstoff & 79.7 & 28.4 &
                                                    \phantom{1000000000000}\strut\\
        \hline
    Sauerstoff & 21.3 & 16.2 &
                                                    \phantom{10000000}\strut\\
        \hline
    Kohlenstoffdioxid & 0.045 & 11.8 & \phantom{10000000}\strut\\     
     \hline
\end{tabular}
\end{center}

\begin{enumerate}
   \setcounter{enumi}{2}
\item Rechnen Sie die Volumenprozente (Vol-\%) der Umgebungsluft in die
  entsprechenden ppm-Werte um und ergänzen Sie die berechneten ppm-Werte in
  die Tabelle!  \textbf{(1 Punkt)}
\end{enumerate}

Während Sie Ihr etwas pappiges Toastbrot mampfen kommt Ihnen die Dokumentation über Brot aus Luft in den Sinn. Sie denken darüber ein wenig nach. Für die Umwandlung von Stickstoff $N_2$ mit Wasserstoff $H_2$ zu Ammoniak
$NH_3$ gilt folgende Reaktionsgleichung\footnote{Die Quelle der Inspiration
  für die Aufgabe war folgender Artikel:
  \href{https://www.gdch.de/netzwerk-strukturen/fachstrukturen/ag-chemie-und-gesellschaft/projekte-und-veranstaltungen/cartoons/haber-bosch-verfahren-brot-aus-luft.html}{Haber-Bosch-Verfahren – Brot aus Luft}}:

\begin{equation*}
  N_2 + 3H_2 \rightarrow 2NH_3
\end{equation*}  

Ein Mol eines beliebigen Gases hat bei normalen Umweltbedingungen ein
Volumen von 22.4 Liter. % Ein Mol enthält $6 \cdot 10^{23}$ Teilchen.

\begin{enumerate}
  \setcounter{enumi}{3}
\item Welche Masse an Ammoniak in Kilogramm $kg$ können Sie aus einem
  Kubikmeter $m^3$ Luft unter normalen Umweltbedingungen gewinnen?
  \textbf{(2 Punkte)}
\item Wieviel Ammoniak in $mol$ erhalten Sie aus einem Kubikmeter Luft?
  \textbf{(1 Punkt)}
\end{enumerate}

 
\clearpage
% -----------------------------------------------------------------------

\section{Aufgabe \hfill (10 Punkte)}

\textit{Geben Sie grunds{\"a}tzlich Formeln und Rechenweg zur L{\"o}sung der
  Teilaufgaben mit an!} \\[1Ex]

%% --------------------------------------------------------------------
\hfill\href{https://youtu.be/Bbu6n8MXxQk}{\includegraphics[width =
  2cm]{img/youtube}} %%youtube
\hspace{2Ex}
%% --------------------------------------------------------------------

\paragraph{Armee der Finsternis}



Ihr Studentenjob war nach Ladenschluss bei Aldi die Regale
einzur{\"a}umen. Dabei ist Ihnen in der Auslage der Sonderangebote das
Necronomicon\footnote{Ein wirklich gef{\"a}hrliches Buch ist:
  \textit{Du bist genug: Vom Mut, gl{\"u}cklich zu sein} von Fumitake Koga und
  Ichiro Kishimi} in die H{\"a}nde gefallen. Nun sind Sie ein Magier der
Zeichen geworden! Also eigentlich k{\"o}nnen Sie nur Mathe und das d{\"a}mliche
Necronomicon hat Sie in die Vergangenheit geschleudert... aber gut, was tut
man nicht alles im Jahr 939 n. Chr. f{\"u}r den neuen Lehnsherren
F{"u}rsten Arthur. Sie bauen nat{\"u}rlich einen Schrottkugelturm um sich den
Horden der Finsternis mit genug Schrott erwehren zu k{\"o}nnen! Ihnen stehen
zwei m{\"a}chtige magische Formeln zur Unterst{\"u}tzung zu Verf{\"u}gung.

\begin{center}
  \begin{tabular}{cc}
    $E_{kin} = \cfrac{1}{2}\cdot m \cdot v^2$ & $E_{pot} = m \cdot g \cdot h$\\
  \end{tabular}
\end{center}

mit

\begin{itemize}
\item $m$, gleich der Masse [kg] des Objekts
\item $h$, gleich der H{\"o}he [m] des ruhenden Objekts
\item $v$, gleich der Geschwindigkeit [m/s] des Objekts
\item $g$, gleich der Erdbeschleunigung mit $9.81 \tfrac{m}{s^2}$ 
\end{itemize}

Als erstes m{\"u}ssen Sie die H{\"o}he des zu bauenden Schrottkugelturmes
bestimmen. Hierf{\"u}r ist wichtig zu wissen, dass sich die
Blei\textit{tropfen} mit einem Gewicht von $30mg$ zu
gleichf{\"o}rmigen Blei\textit{kugeln} bei einer Geschwindigkeit von
$13m/s$ bilden.

\begin{enumerate}
\item Wie hoch m{\"u}ssen Sie den Schrottkugelturm bauen lassen, damit sich
  runde Bleikugeln durch die Fallgeschwindigkeit von $13m/s$ bilden?
  \textbf{(3 Punkte)}
\end{enumerate}

Ihre erstellten Schrottkugeln sind leider zu gro{\ss} und somit sind zu wenige
Schrottkugeln in einer Ladung. Damit k{\"o}nnen Sie die Armee der Finsternis
nicht aufhalten. Die Sachlage m{\"u}ssen Sie einmal mathematisch untersuchen.

\begin{enumerate}
  \setcounter{enumi}{1}
\item Nennen Sie die beiden geometrischen Formen aus denen sich
  n{\"a}herungsweise ein Tropfen zusammensetzt! Erstellen Sie eine beschriftete 
  Skizze des Tropfens!
  \textbf{(2 Punkte)}
\item Sie messen eine L{\"a}nge des Tropfens von 4.1mm. Die L{\"o}cher
  im Sieb erlauben ein Tropfendurchmesser von 1.6mm. Welchen
  Durchmesser in mm haben Ihre produzierten Blei\textit{kugeln}?  \textbf{(3 Punkte)}
\end{enumerate}

Sie haben jetzt die \ensuremath{1.2\times 10^{6}} Bleikugeln zusammen. Blei hat eine Dichte
von $15.1g/cm^3$.

\begin{enumerate}
  \setcounter{enumi}{3}
\item Wie schwer in Kilogramm $kg$ sind die \ensuremath{1.2\times 10^{6}} produzierten
  Bleikugeln, die Sie jetzt auf die Burgmauer transportieren m{\"u}ssen?
  \textbf{(1 Punkt)}
\end{enumerate}

Am Ende m{\"u}ssen Sie noch die Produktion von dem Bleischrott im Turm optimieren.

\begin{enumerate}
  \setcounter{enumi}{4}
\item Wie gro{\ss} in $cm^2$ ist Ihr quadratisches Sieb am oberen Ende des Turms,
  wenn Sie pro Fall ca. 900 Bleikugeln produzieren wollen und
  die Bleikugel im Fall 1.4cm Abstand haben m{\"u}ssen?  \textbf{(1
    Punkt)}
\end{enumerate}
 
\clearpage
% -----------------------------------------------------------------------

\section{Aufgabe \hfill (10 Punkte)}

\textit{Geben Sie grunds{\"a}tzlich Formeln und Rechenweg zur L{\"o}sung der
  Teilaufgaben mit an!} \\[1Ex]

%% --------------------------------------------------------------------
\hfill\href{https://youtu.be/Mr6eslls4J0}{\includegraphics[width =
  2cm]{img/youtube}} %%youtube
\hspace{2Ex}
%% --------------------------------------------------------------------

\paragraph{Armee der Kaninchen}



Leider hat es mit Ihrer Surfschule in Down Under nicht geklappt. War vielleicht
auch nicht \textit{so} die beste Idee... aber daf{\"u}r haben Sie eine Neue!
Oder wie es Mike Tyson zugeschrieben wird: "`Ich wurde nie
niedergeschlagen, ich war immer am Aufstehen!"'. Daher machen Sie jetzt
einen Gro{\ss}handel mit Kaninchenfleisch und damit dem teuersten Fleisch in
Australien auf. Moment, hopsen hier nicht, seit Thomas Austin im Jahr
1860 ungef{\"a}hr 28 Kaninchen entlassen hat,
Millionen von Kaninchen rum? Wieso ist das Kaninchenfleisch dann so
exklusiv? Dem wollen wir mal mathematisch nachgehen!\footnote{Die Quelle
  der Inspiration f{\"u}r die Aufgabe war der folgendes YouTube Video:
  \href{https://youtu.be/38fuOr3tdgc?si=Li7NL_FoByML8JtT}{ Incredible
    Stories -- Why don't they eat wild rabbits in Australia? They have
    millions of them! The reason is surprising...}}
\\

Forscherinnen fand folgende S{\"a}ttigungsfunktion f{\"u}r das j{\"a}hrliche Wachstum
der gesamten Kaninchenpopulation im westlichen Australien.
\begin{equation*}
  f(t) = \ensuremath{10^{10}} - \ensuremath{1.1\times 10^{9}} \cdot 2.2^{-0.15 \cdot t + 2.7}
\end{equation*}

\begin{enumerate}
\item Skizzieren Sie die S{\"a}ttigungsfunktion \textit{ann{\"a}herungsweise} in einer Abbildung! \textbf{(1
    Punkt)}
\item Wie viele Kaninchen k{\"o}nnen nach der S{\"a}ttigungsfunktion maximal im
  westlichen Australien leben? Erg{\"a}nzen Sie den Wert in Ihrer Abbildung! \textbf{(2 Punkte)}
\item Wie viele Millionen Kaninchen leben nach der S{\"a}ttigungsfunktion
  nach 6 Jahren auf dem australischen Kontinent? \textbf{(1
    Punkt)}
\end{enumerate}

Um den Kaninchen Einhalt zu gebieten wurde das Myxoma Virus und das Rabbit
Haemorrhagic Disease Virus (RHDV) in 12 Kaninchen
ausgebracht. Da die Kaninchen keine Ma{\ss}nahmen gegen die Ausbreitung
vornehmen k{\"o}nnen, verl{\"a}uft die Ausbreitung mit einem w{\"o}chentlichen
Wachstumsfakor von 1.8 nach folgender Formel.

\begin{equation*}
  N(t) = N(0) \cdot a^t
\end{equation*}

\begin{enumerate}
  \setcounter{enumi}{2}
\item Wie viele Wochen ben{\"o}tigen die Viren um theoretisch die gesamte
  Kaninchenpopulation nach 4 Jahren Wachstum zu
  durchseuchen? \textbf{(1 Punkt)}
\end{enumerate}

Das Myxoma Virus und das RHDV t{\"o}ten 98.5\% der
Kaninchenpopulation innerhalb weniger Wochen.

\begin{enumerate}
  \setcounter{enumi}{3}  
\item Wie lange in Jahren dauert es bis eine Kaninchenpopulation nach einer
  Viruspandemie wieder auf 40\% der ges{\"a}ttigten
  Kaninchenpopulation angewachsen ist?  \textbf{(2 Punkte)}
\end{enumerate}

Thomas Austin entlie{\ss} die Kaninchen im {\"a}u{\ss}ersten S{"u}den von
Australien. Australien hat eine West-Ost-Ausdehnung von 4400km
und eine Nord-S{\"u}d-Ausdehnung von knapp 3500km. Die Kaninchen
breiten sich radial mit einer Geschwindigkeit von 7.8km pro
Jahr aus.

\begin{enumerate}
  \setcounter{enumi}{4}
\item Wie lange dauert es in Jahren bis die Kaninchen jeden Ort
  in Australien erreicht haben? \textit{L{\"o}sen Sie die Aufgabe unter der
    Verwendung einer schematischen Skizze!} \textbf{(2 Punkte)}
\end{enumerate}

Eine j{\"a}hrliche Impfung gegen das Myxoma Virus und das Rabbit Haemorrhagic
Disease Virus (RHDV) kosten 9\$ pro Tier und der
durchf{\"u}hrende Arzt verlangt ca. 45\$ pro Tier.

\begin{enumerate}
  \setcounter{enumi}{5}
\item In Ihrem Stall leben 1000 Mastkaninchen. Mit welchen
  j{\"a}hrlichen Zusatzkosten f{\"u}r die Impfungen der Kaninchen m{\"u}ssen Sie daher
  kalkulieren? \textbf{(1 Punkt)}
\end{enumerate}
 
\clearpage
% -----------------------------------------------------------------------

\section{Aufgabe \hfill (12 Punkte)}

\textit{Geben Sie grunds{\"a}tzlich Formeln und Rechenweg zur L{\"o}sung der
  Teilaufgaben mit an!} \\[1Ex]

%% --------------------------------------------------------------------
\hfill\href{https://youtu.be/fiWGgCX-cE4}{\includegraphics[width =
  2cm]{img/youtube}} %%youtube
\hspace{2Ex}
%% --------------------------------------------------------------------



\paragraph{Uckermark. Unendliche Weiten.}



Wir schreiben das Jahr 2024. Dies sind die Abenteuer
 des Schafs Frida und Ihnen. Gr{\"u}nes Gras unter Ihren F{\"u}{\ss}en und
ein strammer Wind im Gesicht, egal wohin Sie schauen. Ein schmatzendes
Ger{\"a}usch ert{\"o}nt unter Ihnen. Sie sinnieren, sollten Sie Ihre weiten
Graslandschaften jetzt schon d{\"u}ngen?  Daf{\"u}r ben{\"o}tigen Sie die
\textit{Gr{\"u}nlandtemperatur}! Die Gr{\"u}nlandtemperatur (GLT) ist die Summe aller
positiven Tagesmitteltemperaturen seit Jahresbeginn. Ab einer GLT von
200$^\circ$ kann mit der Stickstoffd{\"u}ngung begonnen werden. Sie sehen nicht
ein, Geld f{\"u}r einen Agrarmetrologen zu bezahlen. Also rechnen Sie mit
folgenden Informationen zu Monatsmultiplikatoren des GLT-Wertes: Januar mit
$0.5\times$, Februar mit $0.7\times$ und M{\"a}rz mit
$1.2\times$. Sie haben noch im letzten Jahr folgende Temperaturen
gemessen.

\begin{center}
\begin{tabular}{cc}
  \toprule
  Datum & C$^\circ$ \\
  \midrule
  01. Jan 2023 & 0.3\\
  01. Feb 2023 & 1.5\\
  01. Mrz 2023 & 3.5\\
  01. Apr 2023 & 4.3\\
  \bottomrule
\end{tabular}
\end{center}

\begin{enumerate}
\item Erstellen Sie eine Skizze aus den Informationen aus der
  Temperaturtabelle!  \textbf{(1 Punkt)}
\item Stellen Sie die linearen Funktionen $f_1(t)$, $f_2(t)$ und
  $f_3(t)$ aus der obigen Temperaturtabelle auf!  \textbf{(1 Punkt)}
\item Bestimmen Sie die Stammfunktionen $F_1(t)$, $F_2(t)$ und $F_3(t)$ f{\"u}r
  Ihre linearen Funktionen aus der obigen Temperaturtabelle!  \textbf{(1
    Punkt)}
\item Osterglocken beginnen ab einer GLT von 190$^\circ C$ zu bl{\"u}hen. An
  welchem Tag im 1. Quartal des Jahres 2023 war dies der
  Fall? \textit{Ignorieren Sie ein eventuelles Schaltjahr in Ihrer Berechnung.} \textbf{(4 Punkte)}
\end{enumerate}

Auf dem Weg zu Ihrer Pink Lady Plantage wurden Sie mit Ihrem Trecker von
einer Gruppe elektrifizierter Renter abgedr{\"a}ngt. Der Trecker muss wieder
aus dem Graben! Frida und die elektrifizierten Rentner ziehen an zwei,
separaten Seilen. Dabei zieht Frida mit $180N$. Die
elektrifizierter Renter  bringen eine Kraft von $190N$ auf.\\

\textit{L{\"o}sen Sie diese Aufgabe mit Hilfe einer aussagekr{\"a}ftigen Skizze der
  Kraftvektoren. Bezeichnen Sie die Kraftvektoren und die Winkel Ihrer
  Skizze entsprechend!}

\begin{enumerate}
  \setcounter{enumi}{4}  
\item Im ersten Versuch legen Sie das Seil f{\"u}r Frida lotrecht {\"u}ber einen
  Ast oberhalb des Treckers. Die Rentner ziehen in einer geraden Linie {\"u}ber
  die B{\"o}schung hinweg am anderen Seil. Welche Kraft wird aufgebracht?
  \textbf{(2 Punkte)}
\item Im zweiten Versuch ziehen Frida und die Rentner mit einem
  $30^\circ$ Winkel mit ihrem Seil an dem Trecker. Welche Kraft
  wird aufgebracht? \textbf{(2 Punkte)}
\item Mit welcher Beschleunigung ziehen Sie den $1.3t$ schweren
  Trecker \textit{jeweils} aus dem Graben, wenn $F = m \cdot a$ gilt?
  \textbf{(1 Punkt)}
\end{enumerate}

 
\clearpage
% -----------------------------------------------------------------------

\section{Aufgabe \hfill (10 Punkte)}

\textit{Geben Sie grunds{\"a}tzlich Formeln und Rechenweg zur L{\"o}sung der
  Teilaufgaben mit an!} \\[1Ex]

%% --------------------------------------------------------------------
\hfill\href{https://youtu.be/RuzMjwvwT-4}{\includegraphics[width =
  2cm]{img/youtube}} %%youtube
\hspace{2Ex}
%% --------------------------------------------------------------------

\paragraph{In der Kartonagenfabrik}



Wenn man sich zu sp{\"a}t anmeldet, dann ist die Exkursion nicht so toll. Also
geht es mit Rektor Skinner und Mrs. Krabappel in die Kartonagenfabrik. Wie
schon im vorherigen Semester... In der Kartonagenfabrik angekommen erfahren
Sie, dass die Kartons zum Versand von N{\"a}geln nicht hier zusammengebaut
werden sondern das sich die Endfertigung in Flint, Michigan befindet. Unter
anderem wird dort der ber{\"u}hmte \textit{Doppelt gewellte,
  6-mal-gefaltete, 0.7mm, 40-cm-Karton} durch
Falzung hergestellt. Beim letzten Mal war Rektor Skinner die Stimmung zu
schlecht und deshalb geht es erst nach Hause, wenn ein paar Aufgaben gel{\"o}st
sind. Martin gef{\"a}llt das. An dem Vorrat
an Zigaretten von Mrs. Krabappel meinen Sie wenig Zuversicht zu erkennen.\\

Jetzt hei{\ss}t es Kartons optimieren. Der nun zu optimierende, flache Karton
hat eine L{\"a}nge von 40cm und eine Breite von 21cm. Die
Kartonagenmaschine in Flint soll dann einen quadratischen Eckenausschnitt
der L{\"a}nge $x$ falzen.

\begin{enumerate}
\item Erstellen Sie eine Skizze des Karton\textit{blatt}rohlings!
  Beschriften Sie die Skizze mit den entsprechenden L{\"a}ngenangaben
  \textbf{(1 Punkt)}
\item Berechnen Sie die Falztiefe $x$ f{\"u}r ein maximales Volumen des flachen
  Kartons! \textbf{(3 Punkte)}
\item Welches Volumen in Liter ergibt sich mit der von Ihnen berechneten
  Falztiefe $x$?  \textbf{(1 Punkt)}
\item Sie wollen noch einen b{\"u}ndig mit dem Boden abschlie{\ss}enden Deckel f{\"u}r
  den Karton stanzen lassen. Wie gro{\ss} ist die Fl{\"a}che des
  Kartondeckel\textit{blatt}rohlings in $cm^2$? \textbf{(2 Punkte)}
\end{enumerate}

Rektor Skinner m{\"o}chte sich gerne wieder in seinem Vorgarten aufhalten und
nicht die ganze Zeit von Bart mit Erdn{\"u}ssen beworfen werden. Deshalb m{\"o}chte
er einen ger{\"a}umigen Teil seines Vorgartens einz{\"a}unen. Ein Teil der
Umz{\"a}unung bildet seine Vorderhauswand. Wegen Lieferschwierigkeiten stehen
Rektor Skinner nur 110m Zaun zu Verf{\"u}gung. Sie wollen nun die
maximale Fl{\"a}che des abgeschirmten Vorgartens in Abh{\"a}ngigkeit der
Seitenl{\"a}ngen bei der Verwendung von 110m Zaun bestimmen!

\begin{enumerate}
  \setcounter{enumi}{4}  
\item  Welche Seitenl{\"a}ngen f{\"u}r den Zaun ergeben sich f{\"u}r die
  maximale Fl{\"a}che des abgeschirmten Vorgartens? \textbf{(2 Punkte)}
\item Berechnen Sie die Fl{\"a}che des abgeschirmten Vorgartens! \textbf{(1
    Punkt)}
\end{enumerate}

 
\clearpage
% -----------------------------------------------------------------------

\section{Aufgabe \hfill (10 Punkte)}

\textit{Geben Sie grunds{\"a}tzlich Formeln und Rechenweg zur L{\"o}sung der
  Teilaufgaben mit an!} \\[1Ex]

%% --------------------------------------------------------------------
\hfill\href{https://youtu.be/OhyuH6hzEhY}{\includegraphics[width = 2cm]{img/youtube}} %%youtube
\hspace{2Ex}
%% --------------------------------------------------------------------




\paragraph{Ein Pfund Insekten, bitte!} Das wird wohl h{\"a}ufiger geh{\"o}rt
werden, wenn wir die Menschheit mit Proteinen ausreichend ern{\"a}hren
wollen\footnote{Die Quelle der Inspiration f{\"u}r die Aufgabe war der folgende
  Artikel aus dem Spiegel:
  \href{https://www.spiegel.de/ausland/ueberbevoelkerung-acht-milliarden-sind-wir-bald-zu-viele-menschen-auf-der-erde-a-3f20c7bc-3d60-4440-9f52-eb338db207f5}{Acht
    Milliarden - sind wir bald zu viele Menschen auf der Erde?}}. Schauen
wir uns dazu einmal den Vergleich Deutschland zu Nigeria an. Nigeria hat
eine der am schnellsten wachsenden Bev{\"o}lkerungen der Welt und wird
vermutlich im Jahr 2100 zu den Top 5 der bev{\"o}lkerungsreichsten L{\"a}nder
z{\"a}hlen. Im Jahr 2023 leben ca. \ensuremath{8\times 10^{7}} Menschen in
Deutschland und ca. \ensuremath{1.84\times 10^{8}} Menschen in Nigeria. Mit den
Informationen wollen wir anfangen und dann eine Prognose f{\"u}r den
Fleischkonsum im Jahr 2050 zu treffen. \\ 


Im folgenden ist Abbildung des Fleischkonsums im Jahr 2023 in
Deutschland und Nigeria in [kg] einmal dargestellt. 

\begin{knitrout}
\definecolor{shadecolor}{rgb}{0.969, 0.969, 0.969}\color{fgcolor}

{\centering \includegraphics[width=\maxwidth]{img/math-14-a-1} 

}


\end{knitrout}
  
\begin{enumerate}
\item Stellen Sie den Fleischkonsum in Deutschland und Nigeria im Jahr
  2023 \textit{pro Kopf} in einer aussagekr{\"a}ftigen Tabelle dar! 
  \textbf{(2 Punkte)}
\item Erg{\"a}nzen Sie in der Tabelle eine Spalte in der Sie f{\"u}r den
  Fleischkonsum in Nigeria auf Deutschland normieren, daher ins Verh{\"a}ltnis
  Nigeria/Deutschland, setzen! \textbf{(1 Punkt)}
\end{enumerate}

In der n{\"a}chsten Abbildung finden Sie die CO$_2$-Emission in [kg] nach
Lebensmittel, die durch die Produktion entsteht, abgebildet.

\begin{knitrout}
\definecolor{shadecolor}{rgb}{0.969, 0.969, 0.969}\color{fgcolor}

{\centering \includegraphics[width=\maxwidth]{img/math-14-b-1} 

}


\end{knitrout}

\begin{enumerate}
  \setcounter{enumi}{2}  
\item Stellen Sie in einer Tabelle die Treibhausgasemissionen an $CO_2$ pro
  Kopf, die durch den Fleischkonsum in Deutschland und Nigeria im Jahr
  2023 entstehen, dar! Erg{\"a}nzen Sie auch hier das Verh{\"a}ltnis
  Nigeria zu Deutschland! \textbf{(2 Punkte)}
\end{enumerate}

\newpage

In der folgenden Abbildung sehen Sie die
Bev{\"o}lkerungsentwicklung [Millionen] in Nigeria von 1950 bis ins Jahr 2030 fortgef{\"u}hrt.

\begin{knitrout}
\definecolor{shadecolor}{rgb}{0.969, 0.969, 0.969}\color{fgcolor}

{\centering \includegraphics[width=\maxwidth]{img/math-14-c-1} 

}


\end{knitrout}

\begin{enumerate}
  \setcounter{enumi}{3}  
\item Sch{\"a}tzen Sie graphisch die zu erwartende Bev{\"o}lkerung [Millionen] in Nigeria im
  Jahr 2050, die sich anhand der Informationen aus der Abbildung ergibt!
\begin{enumerate}
\item Ohne Ber{\"u}cksichtigung der Covid-19-Pandemie! \textbf{(1
    Punkt)}
\item Unter Ber{\"u}cksichtigung der Covid-19-Pandemie! \textbf{(1
    Punkt)}
\end{enumerate}
\item Berechnen Sie den gesch{\"a}tzten Fleischkonsum von Nigeria im Jahr
  2050 unter der Annahme 60\%-iger Angleichung der Lebensbedingungen zu
  Deutschland im Jahr 2023! \textbf{(1 Punkt)}
\item Berechnen Sie die prozentuale Steigerung der Treibhausgasemissionen
  an $CO_2$ in Nigeria im Jahr 2050 im Vergleich zum Jahr 2023, der
  sich durch den angeglichenen Fleischkonsum ergibt! \textbf{(1 Punkt)}
\item Berechnen Sie die prozentuale Steigerung der Treibhausgasemissionen
  an $CO_2$ in Nigeria, wenn die gesamte Proteinaufnahme 
  durch Insekten ersetzt w{\"u}rde! \textbf{(1
    Punkt)}
\end{enumerate}







 
\clearpage
% -----------------------------------------------------------------------

\section{Aufgabe \hfill (12 Punkte)}

\textit{Geben Sie grunds{\"a}tzlich Formeln und Rechenweg zur L{\"o}sung der
  Teilaufgaben mit an!} \\[1Ex]

%% --------------------------------------------------------------------
\hfill\href{https://youtu.be/flRBo1FWQy0}{\includegraphics[width = 2cm]{img/youtube}} %%youtube
\hspace{2Ex}
%% --------------------------------------------------------------------

\paragraph{T{\"o}dliche Seuche AIDS -- Die r{\"a}tselhafte Krankheit}




Irritiert legen Sie die historische Ausgabe des Spiegels aus den 80zigern
beiseite. Sie sind bei Ihrem Hautarzt und wollen einen AIDS-Test machen
lassen. Woanders leider keinen Termin gekriegt... Immerhin denken Sie und
Ihr Partner {\"u}ber Nachwuchs nach und da geht es eben nur durch
ungesch{\"u}tzten Sex. Was wissen Sie nun aber {\"u}ber AIDS und dem diagnostischen
AIDS-Test, den Sie nun machen werden?\\

Die Pr{\"a}valenz von AIDS bei einem Menschen in Europa wird mit
0.8\% angenommen. In 92\% der F{\"a}lle ist ein
HIV-Test positiv, wenn der Patient erkrankt ist. In 4\%
der F{\"a}lle ist ein HIV-Test positiv, wenn der Patient \textit{nicht}
erkrankt ist und somit gesund ist. Sie stutzen. Wie wahrscheinlich ist es
denn eigentlich an AIDS erkrankt zu sein $(K^+)$, wenn Sie einen positiven
AIDS-Test vorliegen haben $(T^+)$? Gehen Sie f{\"u}r die folgenden Berechnungen
von $n = \ensuremath{2\times 10^{4}}$ Patienten mit einem diagnostischen Test f{\"u}r AIDS
aus. Sie nehmen sich also einen Kuli und fangen an auf der historischen
Ausgabe des Spiegels zu rechnen\footnote{Die Quelle der Inspiration f{\"u}r die
  Aufgabe war der folgende wissenschaftlicher Artikel: \textit{Binder et
    al. (2022) Von Baumdiagrammen {\"u}ber Doppelb{\"a}ume zu H{\"a}ufigkeitsnetzen --
    kognitive {\"U}berlastung oder didaktische Unterst{\"u}tzung? Journal f{\"u}r
    Mathematik-Didaktik, 1-33}}.

\begin{enumerate}
\item Welche Wahrscheinlichkeit $Pr$ wollen Sie berechnen? \textbf{(1 Punkt)}
\item Zeichnen Sie einen H{\"a}ufigkeitsdoppelbaum zur Bestimmung der gesuchten
  Wahrscheinlichkeit $Pr$! \textbf{(2 Punkte)} 
\item Beschriften Sie den H{\"a}ufigkeitsdoppelbaum, mit denen Ihnen bekannten
  Informationen zu der AIDS Erkrankung und dem AIDS-Test! \textbf{(1 Punkt)}
\item F{\"u}llen Sie den H{\"a}ufigkeitsdoppelbaum mit den sich ergebenden,
  absoluten Patientenzahlen $n$ aus! \textbf{(2 Punkte)}
\item Berechnen Sie die gesuchte Wahrscheinlichkeit $Pr$! \textbf{(1 Punkt)}
\end{enumerate}

Bei dem folgenden Arztgespr{\"a}ch erfahren Sie, dass beim diagnostischen
Testen \textit{True Positives (TP)}, \textit{True Negatives (TN)},
\textit{False Positives (FP)} und \textit{False Negatives (FN)}
auftreten. Das verstehen Sie so noch nicht und deshalb stellen Sie f{\"u}r sich
den Zusammenhang in einer 2x2 Kreuztabelle dar.

\begin{enumerate}
  \setcounter{enumi}{5}
\item Tragen Sie \textit{TP}, \textit{TN}, \textit{FP} und \textit{FN} in
  eine 2x2 Kreuztablle ein. Beschriften Sie die Tabelle entsprechend!
  \textbf{(1 Punkt)}
\item Berechnen Sie die Sensitivit{\"a}t und Spezifit{\"a}t des diagnostischen Tests
  f{\"u}r AIDS! F{\"u}llen Sie daf{\"u}r die 2x2 Kreuztabelle mit den Informationen aus
  dem H{\"a}ufigkeitsdoppelbaum aus! \textbf{(2 Punkte)}
\item Was beschreibt die Sensitivit{\"a}t und die Spezifit{\"a}t im Bezug auf die
  Gesunden und Kranken? Stellen Sie beide diagnostische Ma{\ss}zahlen als
  Wahrscheinlichkeiten $Pr$ dar! \textbf{(2 Punkte)} 
\end{enumerate}



 
\clearpage
% -----------------------------------------------------------------------

\section{Aufgabe \hfill (12 Punkte)}

\textit{Geben Sie grunds{\"a}tzlich Formeln und Rechenweg zur L{\"o}sung der
  Teilaufgaben mit an!} \\[1Ex]

%% --------------------------------------------------------------------
\hfill\href{https://youtu.be/SZqp_vy3rgI}{\includegraphics[width = 2cm]{img/youtube}} %%youtube
\hspace{2Ex}
%% --------------------------------------------------------------------

\paragraph{Network-Marketing oder Schneeballschlacht!}




Eine M{\"o}glichkeit, leicht Geld zu verdienen, ist es anderen Menschen f{\"u}r Geld
zu versprechen, wie man leicht reich werden kann. Am besten nat{\"u}rlich ohne
viel Aufwand und ortsunabh{\"a}ngig. Dann wollen wir mal loslegen. Schnell
ein YouTube-Werbevideo gedreht und auf geht es mit unserem
Network-Marketing. Aber Moment, wie funktioniert Network-Marketing
eigentlich und was hat das alles mit einer Schneeballschlacht zu tun? Wir
wollen hier einmal in die Untiefen des "`passiven Einkommens"'
abtauchen\footnote{Die Quellen der Inspiration f{\"u}r die Aufgabe waren
  folgendes YouTube Video:
  \href{https://youtu.be/UOKkZF_qK9M?si=uf4foJVFKfeQMwSw}{Simplicissimus --
    Die meistgesuchte Betr{\"u}gerin der Welt} und der Artikel:
  \href{https://www.deutschlandfunkkultur.de/netzwerk-marketing-die-illusion-schnell-reich-zu-werden-100.html}{
    Deutschlandfunk Kultur -- Die Illusion, schnell reich zu werden}}.\\

Das Jahr 2022 war das erfolgreichste Jahr in der Geschichte von
KH Gesund und Sch{"o}n Components (KH-GSC). Das Unternehmen steigerte den Umsatz um rund
15 Prozent von 260 Millionen Euro im Jahr
2021. Doch wie viel kommt bei den Partnern an? Laut
KH-GSC habe das Unternehmen \ensuremath{3.8\times 10^{5}} aktive Partner.

\begin{enumerate}
\item Berechnen Sie zuerst den Umsatz der Firma KH-GSC im
  Jahr 2022! \textbf{(1 Punkt)}
\item Wie viel von dem Umsatz im Jahr 2022 wird im Durchschnitt von jedem
  aktiven Partner erwirtschaftet? \textbf{(1 Punkt)}
\item Welche \textit{monatlicher} Umsatz ergibt sich dadurch im
  Durchschnitt f{\"u}r jeden aktiven Partner bei einer direkten Provision von
  30\%? \textbf{(1 Punkt)}
\end{enumerate}

Ihr zu vermarkendes Produkt, hinter dem Sie voll stehen, kostet
200EUR pro Einheit im Direktverkauf. Die direkte Provision
f{\"u}r die erste Stufe betr{\"a}gt 40\%. F{\"u}r die zweite, dritte und
vierte Stufe betragen die indirekten Provisionen jeweils 3\%,
2\% und 1.5\%. Jeder Ihrer angeworbenen "`Partner"'
wirbt wiederum f{"u}nf Partner f{\"u}r sich selbst an. Pro Monat
werden im Schnitt f{"u}nf Einheiten vom Produkt verkauft. Sie wollen nun
3200EUR im Monat \textit{passiv} -- also durch indirekte
Provisionen -- erwirtschaften.

\begin{enumerate}
  \setcounter{enumi}{3}
\item Erg{\"a}nzen Sie die folgende Tabelle mit den obigen Informationen! \textbf{(2 Punkte)}
\end{enumerate}

\begin{center}
\begin{tabular}{c|c|c|c}
  \toprule
  \textbf{Stufe} & \textbf{Anzahl Partner}  & \textbf{Umsatz/Stufe} & \textbf{Provision}\\
  \midrule
  1 & Sie selber  &  & \\ \midrule
  2 &   &  &  \\ \midrule
  3 &   &  &  \\ \midrule
  4 &   &  &  \\
  \bottomrule
\end{tabular}
\end{center}

\begin{enumerate}
  \setcounter{enumi}{4}
\item Wie viele Partner m{\"u}ssen Sie auf der 2 Stufe anwerben um Ihr passives
  Einkommen durch indirekte Provision zu erreichen? Wie viele Menschen
  arbeiten am Ende indirekt f{\"u}r Sie? Stellen Sie den
  Zusammenhang graphisch dar!  \textbf{(3 Punkte)}
\end{enumerate}

Sie mussten zum Einstieg bei KH-GSC Einheiten des Produkts
f{\"u}r 6000EUR kaufen. Diese Einheiten k{\"o}nnen Sie nur direkt
verkaufen. Leider mussten Sie den Kauf {\"u}ber einen Kredit {\"u}ber
5.1\% p.a. {\"u}ber 48 Monate finanzieren.

\begin{enumerate}
  \setcounter{enumi}{5}
\item Berechnen Sie die Gesamtsumme, die Sie als Kredit abbezahlen m{\"u}ssen! \textbf{(2 Punkte)}
\item Wie viele Einheiten m{\"u}ssen Sie pro Monat verkaufen um die anfallenden Zinsen
  durch die direkte Provision zu erwirtschaften? \textbf{(1 Punkt)}
\item Wie lange in
  Monaten ben{\"o}tigen Sie um den Kredit durch die direkte
  Provision abzubezahlen? \textbf{(1 Punkt)}
\end{enumerate} 
\clearpage
% -----------------------------------------------------------------------

\section{Aufgabe \hfill (10 Punkte)}

\textit{Geben Sie grunds{\"a}tzlich Formeln und Rechenweg zur L{\"o}sung der
  Teilaufgaben mit an!} \\[1Ex]

%% --------------------------------------------------------------------
\hfill\href{https://youtu.be/8Pb2sKUIMyk}{\includegraphics[width =
  2cm]{img/youtube}} %%youtube
\hspace{2Ex}
%% --------------------------------------------------------------------

\paragraph{H{\"o}hlen \& Drachen}



Nachdem Sie sich begeistert in der Serie \textit{Stranger Thinks} verloren
haben, wollen Sie bei einer Ihrer Freundinnen einmal \textit{H{\"o}hlen \& Drachen}
ausprobieren. Um Geld zu sparen, das Zeug kostet echt, wurde etwas an den
Regeln gebastelt. Schnell stellen Sie fest, dass hier ganz sch{\"o}n viele
unterschiedliche W{\"u}rfel durch die Gegend fliegen. Daher m{\"u}ssen Sie sich
jetzt einiges an Fragen stellen. \\%[-1ex]

In dem Spiel haben Sie nun auf einmal 5 achtseitige W{"u}rfel (5d8) zum w{\"u}rfeln in der Hand. Wenn Sie eine 8 w{\"u}rfeln,
haben Sie einen Erfolg.

\begin{enumerate}
\item Berechnen Sie die Wahrscheinlichkeit \textit{genau}
  3 Erfolge zu erzielen!  \textbf{(2 Punkte)}
\item Berechnen Sie die Wahrscheinlichkeit keinen Erfolg zu erzielen!
  \textbf{(1 Punkt)}
\end{enumerate}

Sie betrachten nun aufmerksam die ausufernden Ausr{\"u}stungstabellen. Ihnen
wird aber geholfen und Sie m{\"u}ssen sich jetzt nur zwischen der Axt oder dem
Schwert entscheiden.

\begin{enumerate}
  \setcounter{enumi}{2}
\item W{\"u}rden Sie die Axt mit zwei zw{"o}lfseitigen W{"u}rfeln (2d12) als Schaden oder
  das Schwert mit einem vierseitigen W{"u}rfel plus 6 (1d4+6) als Schaden bevorzugen?
  Begr{\"u}nden Sie Ihre Antwort mathematisch! \textbf{(1 Punkt)}
\end{enumerate}

Jetzt wird es immer wilder, da Sie sich jetzt {\"u}berlegen m{\"u}ssen, wie
wahrscheinlich es ist, dass Ihr Rettungswurf gegen den zaubernden Hexer
funktioniert. Sie haben folgende Wahrscheinlichkeiten gegeben. Die
Wahrscheinlichkeit f{\"u}r das Ereignis $A$, der Rettungswurf ist erfolgreich,
ist $Pr(A) = 0.7$, die Wahrscheinlichkeit f{\"u}r das Ereignis $B$,
der Zauberwurf des Hexers ist erfolgreich, ist $Pr(B) = 0.9$. Sie
haben mitgez{\"a}hlt und festgestellt, dass in $45$ von 100 F{\"a}llen
Ihr Rettungswurf bei einem erfolgeichen Zauber funktioniert hat.  

\begin{enumerate}
  \setcounter{enumi}{3}
\item Erstellen Sie eine 2x2 Kreuztabelle mit den Ereignissen $A$ und $B$
  sowie den Gegenereignissen $\bar{A}$ und $\bar{B}$ mit einen
  $\Omega = 100$. Beachten Sie hierbei die entsprechenden
  Wahrscheinlichkeiten f{\"u}r die Ereignisse $A$ und $B$! \textbf{(2 Punkte)}
\item Bestimmen Sie $Pr(A \cap B)$! \textbf{(1 Punkt)}
\item Erstellen Sie ein Baumdiagramm mit den passenden Informationen aus der 2x2
  Kreuztabelle! \textbf{(2 Punkte)}
\item Bestimmen Sie Wahrscheinlichkeit $Pr(A|B)$, dass Ihr Rettungswurf gelingt, wenn
  der Hexer erfolgreich gezaubert hat! \textbf{(1 Punkt)}
\end{enumerate}

  
\clearpage
% -----------------------------------------------------------------------

\section{Aufgabe \hfill (12 Punkte)}

\textit{Geben Sie grunds{\"a}tzlich Formeln und Rechenweg zur L{\"o}sung der
  Teilaufgaben mit an!} \\[1Ex]

%% --------------------------------------------------------------------
\hfill\href{https://youtu.be/9DQKaXdxT_g}{\includegraphics[width =
  2cm]{img/youtube}} %%youtube
\hspace{2Ex}
%% --------------------------------------------------------------------

\paragraph{Retrocheck im TV}



"`Und hier ist sie wieder, die Show der fantastischen Preise. Seien Sie mit
dabei, wenn es wieder hei{\ss}t: Der Preis ist hei{\ss}!"', ert{\"o}nt es und Sie
fragen sich, ob Sie nicht doch lieber bezahlter Gast bei Barbara Salesch
h{\"a}tten sein sollten. Aber Sie brauchen das Geld und jetzt hei{\ss}t es
Spielschows farmen! Erstmal eine Kaffemaschine von Mitropa
gewinnen. Ein Kandidat gewinnt die Kaffeemaschine von Mitropa, wenn nicht alle
Kandidaten {\"u}berbieten (eng. \textit{outbid}). Mit Ihnen bilden
Thorsten und Annegret das Team der drei Kandidaten.

\begin{center}
\begin{tabular}{ccc}
  \toprule
  Name & $P(win)$ & $P(outbid)$\\
  \midrule
  Thorsten & 0.2 & 0.043\\
  Annegret & 0.2 & 0.08 \\
  \bottomrule
\end{tabular}
\end{center}

\begin{enumerate}
\item Mit welcher Wahrscheinlichkeit gewinnen Sie die Kaffeemaschine von
  Mitropa, wenn keiner der Kandidaten {\"u}berbietet? \textbf{(1 Punkt)}
\item Wenn Ihre {\"U}berbietungswahrscheinlichkeit $P(outbid)$ bei
  0.12 liegt, mit welcher Wahrscheinlichkeit gewinnt
  \textit{keiner} die Kaffeemaschine von Mitropa? \textbf{(1 Punkt)}
\end{enumerate}

Gl{\"u}cksrad f{\"u}r Arme auf der Kirmes! Leider hat es f{\"u}r Maren Gilzer nicht
gereicht. Deshalb sind Sie jetzt auf der Kirmes und spielen mit
Catwoman um das gro{\ss}e Geld. Das Gl{\"u}cksrad hat 22
Felder. Sie drehen das Gl{\"u}cksrad zweimal. Auf 8 Feldern
gewinnen Sie 5000EUR sonst 1000EUR. Ganz sch{\"o}n viel Geld
und ganz sch{\"o}n zwielichtig hier...

\begin{enumerate}
  \setcounter{enumi}{2}  
\item Skizzieren Sie das Gl{\"u}cksrad und erg{\"a}nzen Sie die
  Wahrscheinlichkeiten! \textbf{(1 Punkt)}
\item Zeichnen Sie das zugeh{\"o}rige Baumdiagramm f{\"u}r das zweimalige Drehen!
  Erg{\"a}nzen Sie die Wahrscheinlichkeiten und die entsprechenden Ereignisse
  \textbf{(2 Punkte)}
\item Mir welcher Wahrscheinlichkeit gewinnen Sie 6000EUR? \textbf{(1
    Punkt)}
\end{enumerate}

Nach Ihrem Fiebertraum reisen Sie im Zug nach K{\"o}ln um bei "`Geh aufs
Ganze!"' mitzuspielen. Sie schaffen es tats{\"a}chlich ins Finale und k{\"o}nnen
als Hauptgewinn ein Auto hinter einer der drei T{\"u}ren gewinnen. 

\begin{enumerate}
  \setcounter{enumi}{5}  
\item Bevor die Show beginnt, wird das Auto hinter eine zuf{\"a}llig bestimmte
  T{\"u}r gestellt. Mit welcher Wahrscheinlichkeit wird jeweils eine der drei
  T{\"u}ren ausgew{\"a}hlt? Zeichnen Sie ein Baumdiagramm! \textbf{(1 Punkt)}
\item Mit welcher Wahrscheinlichkeit w{\"a}hlen Sie sofort die T{\"u}r mit
  dem Auto? Erweitere Sie das Baumdiagramm entsprechend! \textbf{(1
    Punkt)}
\item Der Moderator {\"o}ffnet nun eine der nicht gew{\"a}hlten T{\"u}ren, aber
  nat{\"u}rlich nicht die mit dem Auto. Mit welcher Wahrscheinlichkeit steht
  das Auto hinter der anderen T{\"u}r? Erweitern Sie das Baumdiagramm
  entsprechend! \textbf{(2 Punkte)}
\item L{\"o}sen Sie nun das "`Ziegenproblem"'! Berechne Sie dazu die
  Wahrscheinlichkeiten der einzelnen Pfade. Lohnt sich ein Wechsel der
  anfangs gew{\"a}hlte T{\"u}r? Begr{\"u}nden Sie Ihre Antwort mathematisch!
  \textbf{(2 Punkte)}
\end{enumerate}
 

%https://unterrichten.zum.de/wiki/Laplace-Wahrscheinlichkeit_wiederholen_und_vertiefen/Ziegen

%Buchstabenh{\"a}ufigkeit im Deutschen

%%https://www.mathelike.de/abiturloesungen-mathematik-bayern/mathematik-abitur-bayern-2020/pruefungsteil-a/stochastik-2/teilaufgabe-a.html

%%https://de.wikipedia.org/wiki/Buchstabenh%C3%A4ufigkeit


 
\clearpage
% -----------------------------------------------------------------------
\part{Angewandte Nutztier- und Pflanzenwissenschaften (M.Sc.)}
% -----------------------------------------------------------------------  

\section{Aufgabe \hfill (6 Punkte)}
Vergleichen Sie die Standardabweichung mit dem Standardfehler und grenzen
Sie die beiden Kennzahlen voneinander ab.



\section{Aufgabe \hfill (8 Punkte)}
Ihnen liegt folgendes Varianzanalysemodell mit der {\"u}blichen Beschreibung
zur Auswertung des Merkmals fett- und eiwei{\ss}korrigierte Milchleistung pro
Kuh und Jahr in kg vor:

\begin{equation*}
  Y_{ijkl} = \mu + Var_i + EKA_j + VarEKA_{ij} + V_k + b(L_{ij} - L) + e_{ijkl}
\end{equation*}

mit

\begin{itemize}
\item $Y_{ijkl}$: l-te Beobachtung
\item $\mu$: Populationsmittel 
\item $Var_i$: fixer Effekt der i-ten Variante (i: Kontrolle,
  Versuchsgruppe 1, Versuchsgruppe 2)
\item $EKA_j$: fixer Effekt der j-ten Erstkalbealtergruppe (j: EKA $\leq$ 25
  Monate, EKA > 25 Monate)
\item $VarEKA_{ij}$: fixer Effekt der Interaktion Variante x
  Erstkalbealtergruppe 
\item $V_k$: zuf{\"a}lliger Effekt des Vaters 
\item $b(L_{ij} - L)$: lineare Kovariable Laktationsnummer 
\item $e_{ijkl}$: zuf{\"a}lliger
  Restfehler
\end{itemize}
Erl{\"a}utern Sie anhand dieses Beispiels die Begriffe fixer Effekt,
Interaktion, zuf{\"a}lliger Effekt und Kovariable und grenzen Sie diese
Begriffe voneinander ab.



\section{Aufgabe \hfill (6 Punkte)}
Wie bestimmen Sie die richtige Stichprobengr{\"o}{\ss}e? Welche Kennzahlen /
statistische Ma{\ss}zahlen ben{\"o}tigen Sie dabei und nennen Sie die
Voraussetzungen.

 
\end{document}
% -----------------------------------------------------------------------


  
