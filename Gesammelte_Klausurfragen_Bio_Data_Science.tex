\documentclass[a4paper, 9pt]{scrartcl}\usepackage[]{graphicx}\usepackage[]{xcolor}
% maxwidth is the original width if it is less than linewidth
% otherwise use linewidth (to make sure the graphics do not exceed the margin)
\makeatletter
\def\maxwidth{ %
  \ifdim\Gin@nat@width>\linewidth
    \linewidth
  \else
    \Gin@nat@width
  \fi
}
\makeatother

\definecolor{fgcolor}{rgb}{0.345, 0.345, 0.345}
\newcommand{\hlnum}[1]{\textcolor[rgb]{0.686,0.059,0.569}{#1}}%
\newcommand{\hlstr}[1]{\textcolor[rgb]{0.192,0.494,0.8}{#1}}%
\newcommand{\hlcom}[1]{\textcolor[rgb]{0.678,0.584,0.686}{\textit{#1}}}%
\newcommand{\hlopt}[1]{\textcolor[rgb]{0,0,0}{#1}}%
\newcommand{\hlstd}[1]{\textcolor[rgb]{0.345,0.345,0.345}{#1}}%
\newcommand{\hlkwa}[1]{\textcolor[rgb]{0.161,0.373,0.58}{\textbf{#1}}}%
\newcommand{\hlkwb}[1]{\textcolor[rgb]{0.69,0.353,0.396}{#1}}%
\newcommand{\hlkwc}[1]{\textcolor[rgb]{0.333,0.667,0.333}{#1}}%
\newcommand{\hlkwd}[1]{\textcolor[rgb]{0.737,0.353,0.396}{\textbf{#1}}}%
\let\hlipl\hlkwb

\usepackage{framed}
\makeatletter
\newenvironment{kframe}{%
 \def\at@end@of@kframe{}%
 \ifinner\ifhmode%
  \def\at@end@of@kframe{\end{minipage}}%
  \begin{minipage}{\columnwidth}%
 \fi\fi%
 \def\FrameCommand##1{\hskip\@totalleftmargin \hskip-\fboxsep
 \colorbox{shadecolor}{##1}\hskip-\fboxsep
     % There is no \\@totalrightmargin, so:
     \hskip-\linewidth \hskip-\@totalleftmargin \hskip\columnwidth}%
 \MakeFramed {\advance\hsize-\width
   \@totalleftmargin\z@ \linewidth\hsize
   \@setminipage}}%
 {\par\unskip\endMakeFramed%
 \at@end@of@kframe}
\makeatother

\definecolor{shadecolor}{rgb}{.97, .97, .97}
\definecolor{messagecolor}{rgb}{0, 0, 0}
\definecolor{warningcolor}{rgb}{1, 0, 1}
\definecolor{errorcolor}{rgb}{1, 0, 0}
\newenvironment{knitrout}{}{} % an empty environment to be redefined in TeX

\usepackage{alltt}
\usepackage[ngerman]{babel}

% -----------------------------------------------------------------------

% -----------------------------------------------------------------------
%% ------------------------------------------------------------
%% by J.Kruppa on Friday, February 11, 2022 (11:31)
%% \def\mainDir{\Sexpr{exam_path}}
\def\source{/Users/jokruppa/source/tex}
\usepackage[margin=2cm, includefoot]{geometry}
\setlength{\parindent}{0cm}
\usepackage{booktabs}
\usepackage{amsmath}
\usepackage{scalerel,amssymb}
\usepackage{setspace}
\def\csquare{{\Large $\boxtimes$}}
\def\msquare{{\Large $\square$}}
\usepackage[normalem]{ulem}
\usepackage{array}
\usepackage{xcolor}
\usepackage{float}
\usepackage{currfile}
\usepackage{tikz}
\usepackage[nomessages]{fp}

%% beamer defs
\def\lecture{Klausurfragen der Bio Data Science}

%% exam defs
\def\examtitle{\lecture}
\def\exammodule{
\vspace{-1.75cm}  
\begin{graybox}{}
\vspace{2Ex}
\textbf{\large Name:} \rule[0ex]{16.75em}{.4pt}
\hfill \textnormal{\textit{Nicht bestanden:}} \msquare \\[2.5Ex]
\textbf{\large Vorname:} \rule[0ex]{15em}{.4pt} \\[2.5Ex]
\textbf{\large Matrikelnummer:} \rule[0ex]{10.8em}{.4pt}
\hfill Endnote: \rule[0ex]{7em}{.4pt} 
\end{graybox}
\vspace{3Ex}
\phantom{text}
}
\def\examsemester{Sommersemester \& Wintersemester}
\def\examdate{\today}
%% ------------------------------------------------------------
\definecolor{darkblue}{rgb}{0,0,.5}
\definecolor{darkpurple}{rgb}{0.4117, 0.2, 0.4117}
\definecolor{uni}{rgb}{0,0.3137,0.6078}
\definecolor{gray}{gray}{0.7}

\usepackage{tcolorbox}
\definecolor{logo1}{RGB}{0, 158, 227}
\definecolor{gray5}{RGB}{247, 247, 247}
\definecolor{gray2}{RGB}{102, 102, 102}

\newtcolorbox{graybox}[1]{
  colback=gray5,%%red!5!white,
  colframe=gray2,%%red!75!black,
  fonttitle=\bfseries\Large,
  %%valign=center,
  fontupper=\large,
  before skip=10pt plus 2pt,
  after skip=20pt plus 4pt,
  title=#1}

\newtcolorbox{takehomebox}[1]{
  colback=gray5,%%red!5!white,
  colframe=logo1,%%red!75!black,
  fonttitle=\bfseries\Large,
  %%valign=center,
  fontupper=\large,
  before skip=10pt plus 2pt,
  after skip=10pt plus 2pt,
  title=#1}

\def\Rlogo{\includegraphics[width = 0.5cm]{\string~/Documents/GitHub/exam/img/Rlogo}\;}

\usepackage[scaled=.90]{helvet} 
\usepackage{fancyhdr}
\usepackage{lastpage}
\usepackage{hyperref}
\hypersetup{
    colorlinks=true,       % false: boxed links; true: colored links
    linkcolor=black,          % color of internal links 
    urlcolor=magenta           % color of external links
}
\renewcommand{\familydefault}{\sfdefault}

\title{
\large \exammodule \\[5Ex]
\Huge \examtitle \\[2Ex] 
\Large Hochschule Osnabr{\"u}ck
}
\author{Pr{\"u}fer: Prof. Dr. Jochen Kruppa \\
Fakult{\"a}t f{\"u}r Agrarwissenschaften und Landschaftsarchitektur \\ 
j.kruppa@hs-osnabrueck.de}
\date{Version vom \examdate}

%% ------------------------------------------------------------
%% by J.Kruppa on Tuesday, September 23, 2014 (12:50)
%% Header
\renewcommand{\headrulewidth}{0pt}
\renewcommand{\footrulewidth}{0pt}
\pagestyle{fancy}

\fancyhf{}
\fancyhead[L]{}
\fancyhead[R]{}
\fancyfoot[R]{\thepage}
\fancyfoot[L]{\footnotesize \examtitle}

\fancypagestyle{empty}{
 \fancyhf{}
 \fancyhead[L]{}
 \fancyhead[R]{}
 \fancyfoot[R]{\thepage}
 \fancyfoot[L]{\footnotesize \examtitle}
}

\usepackage{arevtext,arevmath}

\newcommand\Tstrut{\rule{0pt}{2.6ex}}         % = `top' strut
\newcommand\Bstrut{\rule[-0.9ex]{0pt}{0pt}}   % = `bottom' strut
\def\strut{\Tstrut\Bstrut}

% -----------------------------------------------------------------------
\IfFileExists{upquote.sty}{\usepackage{upquote}}{}
\begin{document}
\date{Wintersemester 2024/25 
\vfill
\begin{center}
\includegraphics[width = 1.9cm]{avatare/Alex}\hspace{-8mm}
\includegraphics[width = 1.9cm]{avatare/Jessica}\hspace{-8mm}
\includegraphics[width = 1.9cm]{avatare/Jonas}\hspace{-8mm}
\includegraphics[width = 1.9cm]{avatare/Mark}\hspace{-8mm}
\includegraphics[width = 1.9cm]{avatare/Nilufar}\hspace{-8mm}
\includegraphics[width = 1.9cm]{avatare/Paula}\hspace{-8mm}
\includegraphics[width = 1.9cm]{avatare/Steffen}\hspace{-8mm}
\includegraphics[width = 1.9cm]{avatare/Tina}\hspace{-8mm}
\includegraphics[width = 1.9cm]{avatare/Yuki}\\
\small
\vspace{1.5Ex}
\textit{"`The test of a student is not how much he knows,\\ but how much he wants to know."'\\ --- Alice W. Rollins}
\end{center}}
% -----------------------------------------------------------------------
\maketitle
\fancypagestyle{empty}{
  \fancyfoot[L]{\tiny $\blacksquare\!\blacksquare\!\square\!\blacksquare\!\square\!\square\!\blacksquare\!\blacksquare\!\square\!\blacksquare\!\square\!\blacksquare\!\blacksquare\!\square\!\square\!\blacksquare\!\square\!\blacksquare\!\blacksquare\!\blacksquare$}
}
\thispagestyle{empty}
\clearpage
% -----------------------------------------------------------------------

\begin{graybox}{Erlaubte Hilfsmittel}
  \vspace{1Ex}
  \begin{itemize}
  \item Normaler Taschenrechner ohne Möglichkeit der Kommunikation mit anderen
    Geräten! Ausdrücklich kein Handy!
  \item Eine DIN A4-Seite als beidseitig, selbstgeschriebene,
    handschriftliche Formelsammlung. Keine digitalen Ausdrucke! 
  \item \textbf{\textcolor{red}{Die Verwendung eines roten Farbstiftes ist nicht gestattet! Korrekturfarbe!}}
  \item \textit{You can answer the questions in English without any consequences.}  
  \end{itemize}
\end{graybox}
\vfill

\begin{graybox}{Endnote}
  \vspace{1Ex}
  \begin{itemize}
  \item[] \rule[0ex]{3em}{.4pt}\, von 20\, Punkten sind aus den Multiple
    Choice Aufgaben erreicht.
  \item[] \rule[0ex]{3em}{.4pt}\, von 73 Punkten sind aus den Rechen- und
    Textaufgaben erreicht. 
  \item[] \rule[0ex]{3em}{.4pt}\, von 93 Punkten in Summe.
  \item[] Es wird folgender Notenschlüssel angewendet.   
  \end{itemize}
  \vspace{1ex}
\begin{center}
  \begin{tabular}[c]{cc}
    \toprule
    \textbf{Punkte}	&	\textbf{Note}	\\
    \midrule
    89.0 - 93.0	&	1,0	\\
    84.5 - 88.5	&	1,3	\\
    79.5 - 84.0	&	1,7	\\
    75.0 - 79.0	&	2,0	\\
    70.5 - 74.5	&	2,3	\\
    66.0 - 70.0	&	2,7	\\
    61.5 - 65.5	&	3,0	\\
    56.5 - 61.0	&	3,3	\\
    52.0 - 56.0	&	3,7	\\
    46.5 - 51.5	&	4,0	\\
    \bottomrule
  \end{tabular}
\end{center}
  \vspace{1ex}
\begin{itemize}
\item[] Es ergibt sich eine Endnote von \rule[0ex]{4em}{.4pt}.
\end{itemize}
  \vspace{1Ex}
\end{graybox}

% -----------------------------------------------------------------------
\newpage
% -----------------------------------------------------------------------

\begin{graybox}{Multiple Choice Aufgaben}
  \begin{itemize}
  \item Pro Multipe Choice Frage ist \emph{genau} eine Antwort richtig.
  \item \textbf{Übertragen Sie Ihre Kreuze in die Tabelle auf
      dieser Seite.}
  \end{itemize}

\begin{center}
  \large
  \begin{tabular}{|l|c|c|c|c|c?c|}
    \hline
    & \textbf{A} & \textbf{B} & \textbf{C} & \textbf{D} & \textbf{E} & $\boldsymbol{\checkmark}$\strut\\
    \hline
    \textbf{Aufgabe 1} &   &   &   &   &   & \strut\\
    \hline
    \textbf{Aufgabe 2} &   &   &   &   &   & \strut\\
    \hline
    \textbf{Aufgabe 3} &   &   &   &   &   & \strut\\
    \hline
    \textbf{Aufgabe 4} &   &   &   &   &   & \strut\\
    \hline
    \textbf{Aufgabe 5} &   &   &   &   &   & \strut\\
    \hline
    \textbf{Aufgabe 6} &   &   &   &   &   & \strut\\
    \hline
    \textbf{Aufgabe 7} &   &   &   &   &   & \strut\\
    \hline
    \textbf{Aufgabe 8} &   &   &   &   &   & \strut\\
    \hline
    \textbf{Aufgabe 9} &   &   &   &   &   & \strut\\
    \hline
    \textbf{Aufgabe 10} &   &   &   &   &   & \strut\\
    \hline
  \end{tabular}
\end{center}

\begin{itemize}
\item Es sind \rule[0ex]{2em}{.4pt}\, von 20 Punkten erreicht worden.
\end{itemize}
\end{graybox}

\vfill

\begin{graybox}{Rechen- und Textaufgaben}
  \begin{center}
    \large
    \begin{tabular}{|l|c|c|c|c|c|c|c|}
      \hline
      \textbf{Aufgabe} & \textbf{11} & \textbf{12} & \textbf{13} & \textbf{14} & \textbf{15} & \textbf{16} & \textbf{17} \strut\\
      \hline
      \textbf{Punkte} & 
      \hspace{1Ex}\Large\textcolor{gray!70}{10}\hspace{1Ex}  & 
      \hspace{1Ex}\Large\textcolor{gray!70}{11}\hspace{1Ex}  & 
      \hspace{1Ex}\Large\textcolor{gray!70}{11}\hspace{1Ex}  & 
      \hspace{1Ex}\Large\textcolor{gray!70}{10}\hspace{1Ex}  & 
      \hspace{1Ex}\Large\textcolor{gray!70}{12}\hspace{1Ex}  & 
      \hspace{1Ex}\Large\textcolor{gray!70}{9}\hspace{1Ex}  & 
      \hspace{1Ex}\Large\textcolor{gray!70}{10}\hspace{1Ex} \strut\\
      \hline
  \end{tabular}
\end{center}
\begin{itemize}
\item Es sind \rule[0ex]{2em}{.4pt}\, von 73 Punkten erreicht worden.
\end{itemize}
\end{graybox}

% -----------------------------------------------------------------------
\clearpage
% -----------------------------------------------------------------------
\begin{graybox}{Multiple Choice Aufgaben}
Die Multiple Choice Aufgaben \textcolor{red}{unterliegen dem Zufall}. Die Reihenfolge der Antworten ist zufällig. Die Fragen und Antworten sind semantisch zufällig und haben somit \textcolor{red}{verschiedene Textvarianten}. Insbesondere die reinen Textaufgaben haben verschiedene Textvarianten. Die Semeantik mag sich unterscheiden, die Inhalte sind aber gleich.
\end{graybox}
\section*{ANOVA}

\section{Aufgabe \hfill (2 Punkte)}





Sie führen einen Versuch mit einer Behandlung und drei Faktorleveln durch. Danach rechnen Sie eine einfaktorielle ANOVA und es ergibt sich ein $\eta^2 = 0.12$. Welche Aussage ist richtig?



\begin{enumerate}
\item [\textbf{A} \msquare] Das $\eta^2$ ist ein Wert f{"u}r die G{"u}te der ANOVA. Je kleiner desto besser. Ein $\eta^2$ von 0 bedeutet ein perfektes Modell mit keiner Abweichung. Die Varianz ist null.
\item [\textbf{B} \msquare] Das $\eta^2$ ist die Korrelation der ANOVA. Mit der Ausnahme, dass 0 der beste Wert ist.
\item [\textbf{C} \msquare] Das $\eta^2$ ist damit mit dem $R^2$ aus der linearen Regression zu vergleichen und beschreibt den Anteil der Varianz, der von den Behandlungsbedingungen erkl{"a}rt wird.
\item [\textbf{D} \msquare] Die Berechnung von $\eta^2$ ist ein Wert f{"u}r die Interaktion.
\item [\textbf{E} \msquare] Das $\eta^2$ beschreibt den Anteil der Varianz, der von den Behandlungsbedingungen nicht erkl{"a}rt wird. Somit der Rest an nicht erkl{"a}rbarer Varianz.
\end{enumerate} 

\section{Aufgabe \hfill (2 Punkte)}



Sie führen ein Feldexperiment durch um das Gewicht von Lauch zu
steigern. Die Pflanzen wachsen unter einer Kontrolle und zwei verschiedenen
Behandlungsbedingungen. Nach der Berechnung einer einfaktoriellen ANOVA
ergibt sich ein $\eta^2 = 0.19$. Welche Aussage ist richtig?



\begin{enumerate}
\item [\textbf{A} \msquare] Das $\eta^2$ beschreibt den Anteil der Varianz, der durch den Forschenden entsteht. Es gilt die Regel, dass ca. 70\% der Varianz eines Versuches durch die Versuchsdurchführung entstehen sollen.
\item [\textbf{B} \msquare] Mit dem $\eta^2$ lässt sich auf die Qualität der Randomisierung und damit der Strukturgleichheit zwischen der Grundgesamtheit und der Stichprobe schließen. Es gilt dabei die Regel, dass ein $\eta^2$-Wert von 1 zu bevorzugen ist.
\item [\textbf{C} \msquare] Es werden 81\% der Varianz durch die Behandlung erklärt. Das $\eta^2$ beschreibt den Anteil der Varianz, der von den unterschiedlichen Behandlungsbedingungen nicht erklärt wird.
\item [\textbf{D} \msquare] Das $\eta^2$ beschreibt den Anteil der Varianz, der von den Umweltbedingungen erklärt wird. Daher werden 19\% der Varianz durch die Umweltbedingungen erklärt. Der Anteil der Varianz durch die Behandlungsgruppen ist dann 81\%.
\item [\textbf{E} \msquare] Das $\eta^2$ beschreibt den Anteil der Varianz, der von den Behandlungsbedingungen erklärt wird. Daher werden 19\% der Varianz durch die Behandlungsgruppen erklärt.
\end{enumerate} 

\section{Aufgabe \hfill (2 Punkte)}



Sie rechnen eine einfaktorielle ANOVA und erhalten eine Teststatistik. Nun müssen Sie diese Teststatistik interpretieren. Welche Aussage ist richtig?



\begin{enumerate}
\item [\textbf{A} \msquare] Die ANOVA berechnet die T-Statistik aus der Multiplikation der MS Behandlung mit der MS der Fehler. Wenn die F-Statistik genau 0 ist, kann die Nullhypothese nicht abgelehnt werden.
\item [\textbf{B} \msquare] Die ANOVA berechnet die F-Statistik indem die MS des Fehlers durch die MS der Behandlung geteilt werden. Wenn die F-Statistik sich der 0 ann{"a}hert kann die Nullhypothese abgelehnt werden.
\item [\textbf{C} \msquare] Die F-Statistik wird berechnet indem die MS der Behandlung durch die MS des Fehlers geteilt werden. Wenn die F-Statistik sich kaum von der Null unterscheidet kann die Nullhypothese nicht abgelehnt werden.
\item [\textbf{D} \msquare] Die ANOVA berechnet die T-Statistik indem den Mittelwertsunterschied der Gruppen simultan durch die Standardabweichung der Gruppen teilt. Wenn die T-Statistik h{"o}her als 1.96 ist, kann die Nullhypothese abgelehnt werden.
\item [\textbf{E} \msquare] Die ANOVA berechnt die F-Statistik aus den SS Behandlung geteilt durch die SS Fehler.
\end{enumerate} 

\section{Aufgabe \hfill (2 Punkte)}




Viele statistische Verfahren nutzen eine Teststatistik um eine Aussage über den Zusammenhang zwischen der Grundgesamthat und der Stichprobe abzubilden. Ein statistisches Testwerkzeug ist hierbei die ANOVA. Die ANOVA rechnet dabei...



\begin{enumerate}
\item [\textbf{A} \msquare] ... den Unterschied zwischen der Varianz in den verschiedenen Behandlungsguppen und der Varianz in einer der Behandlungsgruppen. Wenn die ANOVA signifikant ist, muss über einen Posthoc-Test nachgedacht werden um den signifikanten Unterschied in einer der Gruppen exakt zu bestimmen.
\item [\textbf{B} \msquare] ... den Unterschied zwischen zwei paarweisen Mittelwerten aus verschiedenen Behandlungsguppen. Wenn die signifikant ist, ist daher bekannt welcher Vergleich konkret unterschiedlich ist.
\item [\textbf{C} \msquare] ... den Unterschied zwischen der globalen Varianz und der Varianz aus verschiedenen Behandlungsguppen. Wenn die ANOVA signifikant ist, ist nicht bekannt welcher Vergleich konkret unterschiedlich ist.
\item [\textbf{D} \msquare] ... den Unterschied zwischen der Mittelwerte und der Varianz aus verschiedenen Behandlungsguppen. Wenn die ANOVA signifikant ist, ist bekannt welcher Vergleich konkret unterschiedlich ist.
\item [\textbf{E} \msquare] ... den Unterschied zwischen mehreren Varianzen aus verschiedenen Behandlungsguppen. Wenn die ANOVA signifikant ist, ist nicht bekannt welcher Vergleich konkret unterschiedlich ist.
\end{enumerate} 

\section{Aufgabe \hfill (2 Punkte)}



In einer Studie zur Bewertung der Wirkung von Vitamin E auf das Zahnwachstum bei Igeln entstand folgende Abbildung. Der Versuch wurde an 54 Tieren durchgeführt, wobei jedes Tier eine von  drei Vitamin-C-Dosen (0.5, 1 und 1.5 mg/Tag) über eine von zwei Verabreichungsmethoden erhielt. Welche Aussage ist im Bezug auf eine zweifaktorielle ANOVA richtig?



{\centering \includegraphics[width=\maxwidth]{img/mc-anova-02-a-1} 

}







\begin{enumerate}
\item [\textbf{A} \msquare] Keine Interaktion liegt vor $(p \leq 0.05)$.
\item [\textbf{B} \msquare] Eine mittlere bis starke Interaktion liegt vor $(p \leq 0.05)$
\item [\textbf{C} \msquare] Keine Korrelation liegt vor $(p \geq 0.05)$.
\item [\textbf{D} \msquare] Die Koeffizienten sind positiv $(\beta_0 > 0; \beta_1 > 0)$.
\item [\textbf{E} \msquare] Eine negative Interaktion liegt vor $(\rho \geq 0.5)$.
\end{enumerate} 
\section*{Deskriptive Statistik \& Explorative Datenanalyse}

\section{Aufgabe \hfill (2 Punkte)}




Gegeben ist $y$ mit 14, 8, 10, 10 und 16. Berechnen Sie den Mittelwert und Standardabweichung.



\begin{enumerate}
\item [\textbf{A} \msquare] Es ergibt sich 12.6 +/- 1.645
\item [\textbf{B} \msquare] Sie erhalten 11.6 +/- 1.81
\item [\textbf{C} \msquare] Sie erhalten 11.6 +/- 1.645
\item [\textbf{D} \msquare] Es berechnet sich 12.6 +/- 10.8
\item [\textbf{E} \msquare] Es berechnet sich 11.6 +/- 3.29
\end{enumerate} 

\section{Aufgabe \hfill (2 Punkte)}




Gegeben ist $y$ mit 9, 29, 19, 22, 21, 15, 29, 24, 5, 17 und 51. Berechnen Sie den Median, das $1^{st}$ Quartile sowie das $3^{rd}$ Quartile.




\begin{enumerate}
\item [\textbf{A} \msquare] Sie erhalten 21 [15; 29]
\item [\textbf{B} \msquare] Es berechnet sich 22 [16; 30]
\item [\textbf{C} \msquare] Es berechnet sich 22 [16; 28]
\item [\textbf{D} \msquare] Sie erhalten 21 [13; 27]
\item [\textbf{E} \msquare] Es ergibt sich 22 +/- 15
\end{enumerate} 

\section{Aufgabe \hfill (2 Punkte)}



Mit einem Boxplot  können Sie sehr gut die Verteilung von Daten visualisieren. Die empfohlene Mindestanzahl an Beobachtungen ist dabei?



\begin{enumerate}
\item [\textbf{A} \msquare] Die untere Grenze liegt bei zwei bis fünf Beobachtungen.
\item [\textbf{B} \msquare] 10 Beobachtungen.
\item [\textbf{C} \msquare] 1 Beobachtung.
\item [\textbf{D} \msquare] Die Mindestanzahl liegt bei fünf Beobachtungen.
\item [\textbf{E} \msquare] Wir sollten eine Beobachtung mindestens pro Gruppe vorliegen haben.
\end{enumerate}

\section{Aufgabe \hfill (2 Punkte)}



Um die Varianz zu berechnen müssen wir folgende Rechenoperationen durchführen.



\begin{enumerate}
\item [\textbf{A} \msquare] Den Mittelwert berechen, dann die absoluten Abstände zum Mittelwert aufsummieren
\item [\textbf{B} \msquare] Den Median berechen, dann die quadratischen Abstände zum Median aufsummieren, dann die Wurzel ziehen.
\item [\textbf{C} \msquare] Den Mittelwert berechnen und die Abstände quadrieren. Die Summe mit der Fallzahl multiplizieren.
\item [\textbf{D} \msquare] Wir berechnen erst den Mittelwert und dann die quadratischen Abstände zu dem Mittelwert. Diese quadratischen Abstände summieren wir auf und teilen am Ende durch die Fallzahl. Als letzten Schritt ziehen wir die quadratische Wurzel.
\item [\textbf{E} \msquare] Den Mittelwert berechen, dann die quadratischen Abstände zum Mittelwert aufsummieren und durch die Fallzahl teilen.
\end{enumerate} 

\section{Aufgabe \hfill (2 Punkte)}



Der Barplot stellt folgende statistische Maßzahlen in einer Abbildung dar. Damit gehört der Barplot zu einem der am meisten genutzten statistischen Verfahren zur Visualisierung von Daten.

 



\begin{enumerate}
\item [\textbf{A} \msquare] Den Mittelwert sowie den Median und die Streuung.
\item [\textbf{B} \msquare] Den Mittelwert und die Standardabweichung.
\item [\textbf{C} \msquare] Den Mittelwert und die Varianz.
\item [\textbf{D} \msquare] Der Barplot stellt den Median und die Quartile dar.
\item [\textbf{E} \msquare] Durch die Abbildung des Barplot erhalten wir die Informationen über die Mittelwerte und die Varianz.
\end{enumerate}

\section{Aufgabe \hfill (2 Punkte)}



Nachdem Sie in einem Feldexperiment zu Leistungssteigerung von Erdbeeren durchgeführt haben, berechnen Sie den Mittelwert und den Median. Der Mittelwert $\bar{y}$ und der Median $\tilde{y}$ unterscheiden sich nicht. Welche Aussage ist richtig?



\begin{enumerate}
\item [\textbf{A} \msquare] Da sich der Mittelwert und der Median unterscheiden, ist der Datensatz nicht zu verwenden. Mittelwert und Median müssen gleich sein.
\item [\textbf{B} \msquare] Da sich der Mittelwert und der Median nicht unterscheiden, liegen vermutlich Outlier in den Daten vor. Wir untersuchen den Datensatz nach auffälligen Beobachtungen.
\item [\textbf{C} \msquare] Der Mittelwert und der Median sollten gleich sein, wenn Outlier in den Daten vorliegen. 
\item [\textbf{D} \msquare] Wenn sich der Mittelwert und der Median unterscheiden, liegen vermutlich Outlier in den Daten vor.
\item [\textbf{E} \msquare] Der Mittelwert und der Median sollten sich unterscheiden sein, wenn Outlier in den Daten vorliegen. 
\end{enumerate}

\section{Aufgabe \hfill (2 Punkte)}



Ihre Betreuung der Abschlussarbeit fragt überraschend in der letzten Besprechung, ob Ihre Messwerte einer Normalverteilung genügen. Sonst könnten Sie ja gar nicht einen t-Test rechnen. Da Ihnen die Zeit wegrennt, entscheiden Sie sich für eine schnelle Visualisierung im Anhang. Welche Visualisierung nutzen Sie und welche Regel kommt zur Abschätzung einer Normalverteilung zur Anwendung?



\begin{enumerate}
\item [\textbf{A} \msquare] Nach der Erstellung eines Boxplots schauen wir, ob der Median in der Mitte der Box liegt. Dabei ist der Median als dicke Linie dargestellt und die Box ist das IQR.
\item [\textbf{B} \msquare] Einen Dotplot. Die Punkte müssen sich wie an einer Perlenschnurr audreihen. Eine Abweichung führt zur Ablehnung der Annahme einer Normalverteilung.
\item [\textbf{C} \msquare] Wir erstellen uns für jede Behandlung einen Dotplot und schauen, ob die Dots und damit die Varianz für jede Behandlung gleich groß sind.
\item [\textbf{D} \msquare] In einer explorativen Datanalyse nutzen wir den Violinplot. Dabei sollte der Bauch am Rand liegen. Dann können wir von einer Normalverteilung ausgehen.
\item [\textbf{E} \msquare] Nach dem Einlesen der Daten nutzen wir einen Boxplot um zu schauen, ob alle Boxen über alle Behandlungen in etwa gleich groß sind. Damit ist dann auch das IQR in allen Behandlungen in etwa gleich.
\end{enumerate}

\section{Aufgabe \hfill (2 Punkte)}




Sie wollen in Ihrer Abschlussarbeit über eine explorative Datenanalyse überprüfen, ob Ihr gemessener Endpunkt einer Normalverteilung folgt. Welche drei Abbildungen eignen sich insbesondere für die Überprüfung?





\begin{enumerate}
\item [\textbf{A} \msquare] Scatterplot, Mosaicplot, Boxplot
\item [\textbf{B} \msquare] Boxplot, Violinplot, Mosaicplot
\item [\textbf{C} \msquare] Histogramm, Scatterplot, Boxplot
\item [\textbf{D} \msquare] Densityplot, Boxplot, Violinplot
\item [\textbf{E} \msquare] Violinplot, Scatterplot, Barplot
\end{enumerate} 

\section{Aufgabe \hfill (2 Punkte)}



Sie haben $n = 191$ Pflanzen geerntet und wollen sich nun die Verteilung der Pflanzen einmal in einem Histogramm anschauen. Welche Verteilung ist dargestellt?



{\centering \includegraphics[width=\maxwidth]{img/mc-distribution-02-a-1} 

}







\begin{enumerate}
\item [\textbf{A} \msquare] Es handelt sich um eine Poisson-Verteilung.
\item [\textbf{B} \msquare] In dem Histogramm ist eine Ordinalverteilung dargestellt.
\item [\textbf{C} \msquare] Wir haben eine Normalverteilung vorliegen.
\item [\textbf{D} \msquare] Es handelt sich um eine Binomial-Verteilung.
\item [\textbf{E} \msquare] Eine multivariate Normalverteilung.
\end{enumerate} 
\section*{Lineare Regression \& Korrelation}

\section{Aufgabe \hfill (2 Punkte)}



In Ihrer Abschlussarbeit wollen Sie ein prädiktives Modell rechnen. Jetzt stellt sich die Frage, was diese Entscheidung für Ihre Auswertung bedeutet. Welche Aussage ist richtig?



\begin{enumerate}
\item [\textbf{A} \msquare] Wenn ein prädiktives Modell gerechnet werden soll dann kann dies auf dem gesamten Datensatz geschehen. Das Ziel ist es einen Zusammenhang von $X$ auf $Y$ zu modellieren. Wie wirken sich die Einflussvariablen $Y$ auf die gemessenen Endpunkte $X = x_1, ..., x_p$ aus?
\item [\textbf{B} \msquare] Ein prädiktives Modell wird auf einem Trainingsdatensatz trainiert und anschliessend über eine explorative Datenanalyse validiert. Signifikanzen über $\beta_i$ können hier nicht festgestellt werden.
\item [\textbf{C} \msquare] Ein prädiktives Modell basiert auf einem Traingsdatensatz und einem Testdatensatz. Auf dem Trainingsdatensatz wird das Modell trainiert und auf dem Testdatensatz validiert.
\item [\textbf{D} \msquare] Wir modellieren den Zusammenhang zwischen $X$ und $Y$ wenn ein prädiktives Modell rerechnet wird. Dabei kann nicht der gesamte Datensatz genutzt werden. Es wird ein Trainingsdatensatz zum Trainieren des Modells benötigt.
\item [\textbf{E} \msquare] Wir modellieren den Zusammenhang zwischen $X$ und $Y$ wenn ein prädiktives Modell rerechnet wird. Dabei kann der gesamte Datensatz genutzt werden. Eine Aufteilung wie in einem prädiktiven Modell ist nicht notwendig.
\end{enumerate}

\section{Aufgabe \hfill (2 Punkte)}



Nach der Modellierung einer Regression stellt sich die Frage, ob die Residuen approximativ einer Normalverteilung folgen. Sie können einen QQ-Plot für die visuelle Überprüfung der Annahme an die Residuen nutzen. Welche Aussage ist richtig?



{\centering \includegraphics[width=\maxwidth]{img/mc-regression-05-a-1} 

}







\begin{enumerate}
\item [\textbf{A} \msquare] Wir betrachten die Gerade, die durch die einzelnen Punkte laufen sollte. Wenn die 95\% der Punkte von der Geraden getroffen werden, dann gehen wir von normalverteilten Residuen aus.
\item [\textbf{B} \msquare] Wir betrachten insbesondere die beiden Enden der Gerade. Der Rest ist mehr oder minder egal, dann ist die Annahme an die Normalverteilung der Residuen erfüllt.
\item [\textbf{C} \msquare] Die Annahme der normalverteilten Residuen ist nicht erfüllt. Die Punkte liegen zum überwiegenden Teil nicht auf der Geraden.
\item [\textbf{D} \msquare] Wir betrachten die Punkte auf der Geraden. Wenn die Punkte einigermaßen auf der Geraden liegen, dann gehen wir von normalverteilten Residuen aus. Wir können hier von normalverteilten Residuen ausgehen.
\item [\textbf{E} \msquare] Wir betrachten die Gerade. Wenn die Punkte einigermaßen gleichmäßig um die Gerade verteilt liegen, dann gehen wir von normalverteilten Residuen aus. Dies ist hier nicht der Fall. Wir haben keine normalverteilten Residuen vorliegen.
\end{enumerate}

\section{Aufgabe \hfill (2 Punkte)}



Sie rechnen eine linearen Regression und erhalten folgende Abbildung der Residuen (\texttt{.resid}). Welche Aussage ist richtig?



{\centering \includegraphics[width=\maxwidth]{img/mc-regression-06-a-1} 

}







\begin{enumerate}
\item [\textbf{A} \msquare] Die Annahme der normalverteilten Residuen ist nicht erfüllt. Vereinzelte Punkte liegen oberhalb bzw. unterhalb der Geraden um die 0 Linie weiter entfernt. Ein klares Muster ist zu erkennen.
\item [\textbf{B} \msquare] Die Annahme der normalverteilten Residuen ist nicht erfüllt. Ein klares Muster ist zu erkennen und/oder einige Outlier sind zu beobachten.
\item [\textbf{C} \msquare] Die Punkte müssen gleichmäßig in dem positiven Bereich liegen. Dies ist hier klar nicht der Fall. Einzelne Ausreißer können beobachtet werden. Die Analyse ist gescheitert.
\item [\textbf{D} \msquare] Die Annahme der normalverteilten Residuen ist erfüllt. Es ist ein Muster zu erkennen und wir können damit auf die Signifkanz von $x_1, ..., x_p$ schließen.
\item [\textbf{E} \msquare] Wir betrachten die Nulllinie und alle Punkte sollten ohne Muster gleichmäßig um die Nulllinie liegen. Da dies der Fal ist, gehen wir von keinen Ausreißern aus.
\end{enumerate}

\section{Aufgabe \hfill (2 Punkte)}




Sie berechnen in Ihgrer Abschlussarbeit den Korrelationskoeffizienten $\rho$. Welche Aussage über den Korrelationskoeffizienten $\rho$ ist richtig?




\begin{enumerate}
\item [\textbf{A} \msquare] Der Korrelationskoeffizienten $\rho$ ist eine standardisierte, statistische Maßzahl, die zwischen -1 und 1 liegt. Dabei ist Korrelationskoeffizienten $\rho$ einheitslos.
\item [\textbf{B} \msquare] Der Korrelationskoeffizienten $\rho$ ist eine standardisierte, statistische Maßzahl, die zwischen 0 und 1 liegt. Dabei ist Korrelationskoeffizienten $\rho$ einheitslos. Eine Signifikanz kann nicht nachgewiesen werden.
\item [\textbf{C} \msquare] Der Korrelationskoeffizienten $\rho$ ist eine veraltete Darstellungsform von Effekten in der linearen Regression und wird wie das $\eta^2$ aus der ANOVA interpretiert. Der Korrelationskoeffizienten $\rho$ beschreibt den Anteil an erklärter Varianz durch die Regression.
\item [\textbf{D} \msquare] Der Korrelationskoeffizienten $\rho$ liegt zwischen -1 und 1. Darüber hinaus ist der Korrelationskoeffizienten $\rho$ als standardisierte Steigung zu verstehen, wenn eine Standardisierung durchgeführt wurde. Diese Adjustierung nach Fischer muss am Anschluß der Berechnung der Korrelation durchgeführt werden.
\item [\textbf{E} \msquare] Der Korrelationskoeffizienten $\rho$ zeigt keinen Zusammenhang zwischen zwei Variablen $x$ und $y$ bei einem Wert von 0. Einen negativen Zusammenhang Richtung -1 und somit auch einen positiven Zusammenhang Richtung 1. Je größer die Zahl allgemein, desto stärker der Effekt.
\end{enumerate}

\section{Aufgabe \hfill (2 Punkte)}



Nach einer simplen linearen Regression zur Untersuchung vom Einfluss der $CO_2$-Konzentration in [$\mu g$] im Wasser auf das Wachstum von Erbsen in [$kg$] erhalten Sie einen $\beta_{CO_2}$ Koeffizienten von $1.1\times 10^{-5}$ und einen hoch signifikanten $p$-Wert mit $2e-04$. Warum sehen Sie so einen kleinen Effekt bei einer so deutlichen Signifikanz?




\begin{enumerate}
\item [\textbf{A} \msquare] Die Fallzahl ist zu klein angesetzt. Je kleiner die Fallzahl ist, desto höher ist die Teststatsitik und damit auch der $p$-Wert kleiner. Wir brauchen also mehr Fallzahl um den geringen Effekt noch signifikant zu krigen.
\item [\textbf{B} \msquare] Die Einheit der $CO_2$-Konzentration ist zu klein gewählt. Die Erhöhung der $CO_2$-Konzentration um 1 Einheit führt nur zu einem sehr winzigen Anstieg von $\beta_{CO_2}$ im Gewicht der Wasserlinsen. Die Einheit [$\mu g$] muss besser gewählt werden.
\item [\textbf{C} \msquare] Die Einheit der $CO_2$-Konzentration ist zu klein gewählt. Dadurch sehen wir den sehr kleinen $p$-Wert. Der $p$-Wert und die Einheit von der $CO_2$-Konzentration hängen antiproportional zusammen.
\item [\textbf{D} \msquare] Manchmal ist die Einheit der Einflussvariable $X$ zu groß gewählt, so dass der Ansteig von 1 Einheit in $X$ zu einer zu großen Änderung in $y$ führt. Daher kann der Effekt $\beta_{CO_2}$ sehr klein wirken, da der p-Wert wird auf einer einheitslosen Teststatistik bestimmt wird.
\item [\textbf{E} \msquare] Wenn der Effekt $\beta_{CO_2}$ winzig ist, dann kann es an einer falsch gewählten Einheit liegen. Der Anstieg von einer Einheit in $X$ führt ja zu einer Änderung von $\beta_{CO_2}$ in $x$. Wir müssen daher die Einheit von $y$ entsprechend anpassen.
\end{enumerate}

\section{Aufgabe \hfill (2 Punkte)}



Nachdem Sie Ihr Experiment abgeschlossen haben, stehen Sie vor der Frage wie Sie Ihre Daten modellieren sollen. In der Beispielauswertung von Ihrem Betreuenden finden Sie die Funktion \texttt{lm()} in \Rlogo. Welche Aussage ist richtig?





\begin{enumerate}
\item [\textbf{A} \msquare] Neben der klassichen Verwendung der Funktion \texttt{lm()} in der linearen Regression kann auch ein Gruppenvergleich gerechnet werden. Dafür müssen aber alle Faktoren aus den Daten entfernt und numerishc umgewandelt werden. Dann kann das R Paket \{emmeans\} genutzt werden um die Korrelation zu berechnen. Eine Adjustierung ist dann nicht mehr notwendig.
\item [\textbf{B} \msquare] Ist die Einflussvariable $X$ ein Faktor so werden die Gruppenmittelwerte geschätzt und eine anschließende ANOVA sowie multipler Gruppenvergleich mit \{emmeans\} ist möglich. Die Funktion \texttt{lm()} kann dabei eigentlich weggelassen werden, wird aber traditionell gerechnet.
\item [\textbf{C} \msquare] Die Funktion \texttt{lm()} in \Rlogo ist der erste Schritt für einen Gruppenvergleich. Danach kann eine ANOVA oder aber ein multipler Vergleich in \{emmeans\} gerechnet werden. In der Funktion  \texttt{lm()} werden die Gruppenmittelwerte bestimmt.
\item [\textbf{D} \msquare] Die Funktion \texttt{lm()} in \Rlogo ist der letzte Schritt für einen Gruppenvergleich. Vorher kann eine ANOVA oder aber ein multipler Vergleich in \{emmeans\} gerechnet werden. In der Funktion  \texttt{lm()} werden die Gruppenvarianzen bestimmt.
\item [\textbf{E} \msquare] Die Funktion \texttt{lm()} berechnet die Varianzstruktur für eine ANOVA. Dannach kann dann über eine explorative Datenalayse nochmal eine Signifikanz berechnet werden. Sollte vor der Verwendung der Funktion \texttt{lm()} schon eine EDA gerechnet worden sein, so ist die Analyse wertlos.
\end{enumerate}

\section{Aufgabe \hfill (2 Punkte)}



Wenn Ihr gemessener Endpunkt nicht einer Normalverteilung folgt, so können Sie dennoch Ihre Daten modellieren. Hierzu nutzen Sie dann das \textit{generalisierte lineare Modell (GLM)}. Welche Aussage ist richtig?




\begin{enumerate}
\item [\textbf{A} \msquare] In \Rlogo ist mit dem \textit{generalisierten linearen Modell (GLM)} eine Modellierung implementiert, die die Poissonverteilung für Zähldaten oder die Binomialverteilung für 0/1-Daten modellieren kann. Weitere Modellierungen sind in \Rlogo auch mit zusätzlich geladenen Paketen nicht möglich.
\item [\textbf{B} \msquare] Das \textit{generalisierte lineare Modell (GLM)} erlaubt auch weitere Verteilungsfamilien für das $Y$ bzw. das Outcome in einer linearen Regression zu wählen.
\item [\textbf{C} \msquare] Das GLM ist eine allgemeine Erweiterung der linearen Regression auf die Normalverteilung.
\item [\textbf{D} \msquare] Das GLM ist ein faktisch maschineller Lernalgorithmus, der selstständig die Verteilungsfamilie für Y wählt.
\item [\textbf{E} \msquare] Dank dem \textit{generalisierten linearen Modell (GLM)} können auch andere Verteilungsfamilien -- außer die Normalverteilung -- mit einer linearen Regression modelliert werden. Dafür werden alle Verteilungen in eine Normalverteilung überführt und anschließend standardisiert.
\end{enumerate}
\section*{Vermischte Themen}  

\section{Aufgabe \hfill (2 Punkte)}

Die Randomisierung von Beobachtungen zu den Versuchseinheiten
ist bedeutend in der Versuchsplanung. Welche der folgenden Aussagen ist richtig?



\begin{enumerate}
\item [\textbf{A} \msquare] Strukturgleichheit ist durch Randomisierung gegeben. Leider hilft die Randomisierung noch nicht um von der Stichprobe auf die Grundgesamtheit zu schließen. Deshalb wurde das Falsifikationsprinzip entwickelt.
\item [\textbf{B} \msquare] Randomisierung erlaubt erst die Varianzen zu schätzen. Ohne eine Randomisierung ist die Berechnung von Mittelwerten und Varianzen nicht möglich. Dadurch lässt sich erst ein Experiment auswerten.
\item [\textbf{C} \msquare] Randomisierung bringt starke Unstrukturiertheit in das Experiment und erlaubt erst von der Stichprobe auf die Grundgesamtheit zurückzuschliessen.
\item [\textbf{D} \msquare] Durch eine Randomisierung können wir von Strukturgleichheit zwischen der Stichprobe und der Grundgesamtheit ausgehen.
\item [\textbf{E} \msquare] Durch eine Randomisierung können wir nicht von Strukturgleichheit zwischen der Stichprobe und der Grundgesamtheit ausgehen.
\end{enumerate}

\section{Aufgabe \hfill (2 Punkte)}



Sie wollen Ihren Datensatz in \Rlogo einlesen und stehen nun vor einem Problem. Sie stellen fest, dass die Hilfeseiten alle in englischer Sprache verfasst sind. Warum mag die Nutzung von Deutsch problematisch sein?



\begin{enumerate}
\item [\textbf{A} \msquare] \Rlogo Pakete sind nur in englischer Sprache verfasst. Es macht keinen Sinn \Rlogo daher in Deutsch zu bedienen.
\item [\textbf{B} \msquare] Programmiersprachen können nur englische Begriffe verarbeiten. Zusätzliche Pakete können zwar geladen werden, aber meist funktionieren diese Pakete nicht richtig. Deutsch ist International nicht bedeutend genug.
\item [\textbf{C} \msquare] Die Spracherkennung von \Rlogo ist nicht in der Lage Deutsch zu verstehen.
\item [\textbf{D} \msquare] Programmiersprachen haben Probleme mit Umlauten und Sonderzeichen der deutschen Sprache. Die Nutzung von englischer Sprache umgeht dieses Problem in eleganter Art.
\item [\textbf{E} \msquare] Programmiersprachen haben Probleme mit Umlauten und Sonderzeichen der deutschen Sprache. Daher ist die Nutzung in Deutsch in den AGBs von \Rlogo untersagt.
\end{enumerate}

\section{Aufgabe \hfill (2 Punkte)}



In Ihrer Abschlussarbeit wollen Sie zu Beginn eine explorativen Datenanalyse (EDA) in \Rlogo rechnen. Dafür gibt es eine generelle Abfolge von Prozessschritten. Welche ist hierbei die richtige Reihenfolge?



\begin{enumerate}
\item [\textbf{A} \msquare] Wir transformieren die Spalten über \texttt{mutate()} in ein \texttt{tibble} und können dann über \text{ggplot()} uns die Abbildungen erstellen lassen. Dabei beachten wir das wir keine Faktoren in den Daten haben.
\item [\textbf{B} \msquare] Wir lesen die Daten ein und mutieren die Daten. Dabei ist wichtig, dass wir nicht das Paket \texttt{tidyverse} nutzen, da dieses Paket veraltet ist. über die Funktion \texttt{library(tidyverse)} entfernen wir das Paket von der Analyse.
\item [\textbf{C} \msquare] Die Funktionsreihenfolge ist wie folgt: \texttt{read\_excel()} ->  \texttt{mutate()} -> \text{ggplot()}. Dabei ist bei der Transformation der Daten darauf zu achten, dass keine Faktoren erstellt werden.
\item [\textbf{D} \msquare] Wir lesen als erstes die Daten über \texttt{read\_excel()} ein, transformieren die Spalten über \texttt{mutate()} in die richtige Form und können dann  über \text{ggplot()} uns die Abbildungen erstellen lassen. Wichtig ist, dass wir keine Faktoren sondern nur numerische Variablen vorliegen haben.
\item [\textbf{E} \msquare] Die Funktionsreihenfolge ist wie folgt: \texttt{read\_excel()} ->  \texttt{mutate()} -> \text{ggplot()}. Dabei ist bei der Transformation der Daten darauf zu achten, dass die Faktoren richtig erstellt werden.
\end{enumerate}

\section{Aufgabe \hfill (2 Punkte)}



Gegeben ist das Modell $Y \sim X$. Welche Aussage über $s^2_1 \neq s^2_2$ ist richtig?



\begin{enumerate}
\item [\textbf{A} \msquare] Es handelt sich um ein balanciertes Design.
\item [\textbf{B} \msquare] Es handelt sich um unabhängige Beobachtungen.
\item [\textbf{C} \msquare] Es liegt Varianzhetrogenität vor.
\item [\textbf{D} \msquare] Es handelt sich um ein unbalanciertes Design.
\item [\textbf{E} \msquare] Es liegt Varianzhomogenität vor.
\end{enumerate}

\section{Aufgabe \hfill (2 Punkte)}



Im Rahmen Ihrer Abschlussarbeit werten Sie ein Experiment mit Ferkel aus. Es geht um die Leistungssteigerung der Ferkelproduktion. Sie messen jeweils die Gewichtszunahme der Ferkel. Die Ferkel einer Muttersau sind dabei...



\begin{enumerate}
\item [\textbf{A} \msquare] Untereinander unabhängig. Sollten die Mütter verwandt sein, so ist die Varianzstruktur ähnlich und muss modelliert werden.
\item [\textbf{B} \msquare] Untereinander abhängig, wenn die Mütter ebenfalls miteinander verwandt sind. Erst die Abhängigkeit 2. Grades wird in der Statistik modelliert.
\item [\textbf{C} \msquare] Die Ferkel stammen von der gleichen Sau und sind somit untereinander unabhängig.
\item [\textbf{D} \msquare] Untereinander abhängig. Die Ferkel stammen von einem Muttertier und haben vermutliche eine ähnliche Varianzstruktur.
\item [\textbf{E} \msquare] Untereinander stark korreliert. Die Ferkel sind von einer Mutter und sommit miteinander korreliert. Dies wird in der Statistik jedoch meist nicht modelliert.
\end{enumerate}

\section{Aufgabe \hfill (2 Punkte)}



In einer Studie wollen Sie den Effektschätzer Odds ratio berechnen. Sie finden in Ihrem Experiment zur Behandlung von Klaueninfektionen bei Ziegen in 5 Tieren Erkrankung der Klauen vor. 12 Tiere sind gesund. Welche Aussage ist richtig?



\begin{enumerate}
\item [\textbf{A} \msquare] Es ergibt sich ein Odds ratio von 0.29, da es sich um ein Anteil handelt. Wir berechnen den Anteil der Kranken.
\item [\textbf{B} \msquare] Das Verhältnis von Chancen Odds ratio ergibt ein Chancenverhältnis von 0.42.
\item [\textbf{C} \msquare] Das Verhältnis der Chancen Odds ratio ergibt ein Chancenverhältnis von 0.29. Wir sind an der Chance krank zu sein interessiert.
\item [\textbf{D} \msquare] Es ergibt sich ein Odds ratio von 0.29, da es sich um eine Chancenverhältnis handelt.
\item [\textbf{E} \msquare] Es ergibt sich ein Odds ratio von 0.42, da es sich um ein Anteil handelt.
\end{enumerate}

\section{Aufgabe \hfill (2 Punkte)}



Sie werten in Ihrer Abschlussarbeit einen sehr großen Datensatz aus einer öffentlichen Datenbank aus. Nun stellen Sie fest, dass Sie ein Problem mit der Bewertung Ihrer Ergbnisse anhand der Signifikanz bekommen. Wie Sie herausfinden, scheint dies ein häufiges Problem in der Bio Data Science zu sein. Welche Aussage ist richtig?




\begin{enumerate}
\item [\textbf{A} \msquare] Mehr Fallzahl in Datensätzen bedeutet mehr signifikante Ergebnisse, da in mehr Daten auch mehr Informationen beinhaltet sind. Deshalb lohnen sich riesige Datensätze, die durch die vielen signifikanten Ergebnisse auch eine Menge an relevanten Erkenntnissen liefern.
\item [\textbf{B} \msquare] Riesige Datensätz haben mehr Fallzahl was zur $\alpha$-Inflation führt. Durch eine Adjustoerung kann dem Problem entgegengewirkt werden.
\item [\textbf{C} \msquare] Aktuell werden immer größere Datensätze erhoben. Dadurch wird auch die Varianz immer höher was automatisch zu mehr signifikanten Ergebnissen führt.
\item [\textbf{D} \msquare] Aktuell werden immer größere Datensätze erhoben. Eine erhöhte Fallzahl führt automatisch auch zu mehr signifikanten Ergebnissen, selbst wenn die eigentlichen Effekte nicht relevant sind.
\item [\textbf{E} \msquare] Relevanz und Signifikanz haben nichts miteinander zu tun. Daher gibt es auch keinen Zusammenhang zwischen hoher Fahlzahl (n > 10000) und einem signifikanten Test. Ein Effekt ist immer relevant und somit signifikant.
\end{enumerate}
\section*{Multiple Gruppenvergleiche}    

\section{Aufgabe \hfill (2 Punkte)}



Sie haben folgende unadjustierten p-Werte gegeben: 0.21, 0.89, 0.03, 0.001 und 0.34. Sie adjustieren die p-Werte nach
Bonferroni. Welche Aussage ist richtig?



\begin{enumerate}
\item [\textbf{A} \msquare] Nach der Bonferroni-Adjustierung ergeben sich die adjustierten p-Werte von 0.042, 0.178, 0.006, 2e-04 und 0.068. Die adjustierten p-Werte werden zu einem $\alpha$-Niveau von 5\% verglichen.
\item [\textbf{B} \msquare] Nach der Bonferroni-Adjustierung ergeben sich die adjustierten p-Werte von 1, 1, 0.15, 0.005 und 1. Die adjustierten p-Werte werden zu einem $\alpha$-Niveau von 1\% verglichen.
\item [\textbf{C} \msquare] Nach der Bonferroni-Adjustierung ergeben sich die adjustierten p-Werte von 1.05, 4.45, 0.15, 0.005 und 1.7. Die adjustierten p-Werte werden zu einem $\alpha$-Niveau von 5\% verglichen.
\item [\textbf{D} \msquare] Nach der Bonferroni-Adjustierung ergeben sich die adjustierten p-Werte von 1, 1, 0.15, 0.005 und 1. Die adjustierten p-Werte werden zu einem $\alpha$-Niveau von 5\% verglichen.
\item [\textbf{E} \msquare] Nach der Bonferroni-Adjustierung ergeben sich die adjustierten p-Werte von 0.042, 0.178, 0.006, 2e-04 und 0.068. Die adjustierten p-Werte werden zu einem $\alpha$-Niveau von 1\% verglichen.
\end{enumerate}

\section{Aufgabe \hfill (2 Punkte)}



Sie rechnen einen PostHoc-Test. Nun sollen Sie ein \textit{CLD} erstellen. Was bedeutet dieser Fachbegriff und welche folgende Beschreibung der Interpretation ist korrekt?



\begin{enumerate}
\item [\textbf{A} \msquare] Contrast letter display. Unterschiede in den Behandlungen werden durch den gleichen Buchstaben oder Symbol dargestellt. Die Interpretation des CLD führt häufig in die Irre.
\item [\textbf{B} \msquare] Compact letter display. Gleiche Buchstaben zeigen Gleichheit in den Behandlungen. Die Interpretation ist deshalb sehr intuitiv und einfach. Darüber hinaus ist damit das CLD auch auf einer Linie mit der Testtheorie, da wir ja auch dort die Gültigkeit der Nullhypothese nachweisen. Wir suchen ja Gleichheit.
\item [\textbf{C} \msquare] Compact letter display. Gleiche Buchstaben bedeuten, dass sich die Behandlungen unterscheiden. Daher ist das CLD sehr unintuitiv. Es wäre besser, wenn gleiche Buchstaben Gleichheit anzeigen würden. Dies ist aber leider in der statistischen Testtheorie nicht möglich.
\item [\textbf{D} \msquare] Compact line display. Gleichheit in den Behandlungen wird durch den gleichen Buchstaben oder Symbol dargestellt. Früher wurden keine Buchstaben sondern eine durchgezogene Linie verwendet. Bei mehr als drei Gruppen funktioniert die Linie aber graphisch nicht mehr.
\item [\textbf{E} \msquare] Compact letter display. Teilweise ist die Interpretation des CLD schwierig, da wir ja nach Unterschieden suchen aber nur Gleichheit in den Buchstaben sehen. Die Gleichheit der Behandlungen wird durch gleiche Buchstaben dargestellt.
\end{enumerate}

\section{Aufgabe \hfill (2 Punkte)}




In Ihrer Bachelorarbeit müssen Sie einen Feldversuch auswerten. Nachdem Sie die zweifaktorielle ANOVA gerechnet haben und keine signifikante Interaktion vorliegt, wollen Sie jetzt einen Posthoc-Test rechnen. Welches R Paket nutzen Sie dafür am besten?



\begin{enumerate}
\item [\textbf{A} \msquare] Das R Paket \{lm\}. Das Paket \{lm\} erstellt selbstständig Konfidenzintervalle und entsprechende p-Werte. Da wir in dem Paket nicht adjustieren müssen, ist es bei Anwendern sehr beliebt.
\item [\textbf{B} \msquare] Das R Paket \{emmeans\} erlaubt die Durchführung eines multiplen Gruppenvergleichs. Aus einem \{emmeans\} Objekt lässt sich recht einfach das CLD erstellen und so über Barplots eine schnelle Interpration der statistischen Auswertung durchführen.
\item [\textbf{C} \msquare] Das R Paket \{hmisc\} erlaubt die Durchführung eines multiplen Gruppenvergleichs aus verschiedenen Modellen heraus. Aus einem hmisc Objekt lässt sich recht einfach das CLD erstellen und so über Barplots eine schnelle Interpration der statistischen Auswertung durchführen.
\item [\textbf{D} \msquare] Das R Paket \{emmeans\} erlaubt die Durchführung eines multiplen Gruppenvergleichs. Aus einem emmeans Objekt lässt sich leider kein CLD erstellen. Dennoch ist das Paket einfach zu bedienen und wird deshalb genutzt. Die Interpretation der statistischen Auswertung wird über einen Barplot abgebildet.
\item [\textbf{E} \msquare] Das R Paket \{ggplot\}. Wir erhalten hier sofort eine Visualisierung der Daten. Anhand der Visualisierung lässt sich eine explorative Datenanalyse durchführen, die gleichwertig zu einem Posthoc-Test ist.
\end{enumerate}

\section{Aufgabe \hfill (2 Punkte)}



Bei einem Posthoc-Test kann es zu einer überraschenden Besonderheit beim statistischen Testen kommen. Wie lautet der Fachbegriff und wie kann mit der überraschenden Besonderheit umgegangen werden?



\begin{enumerate}
\item [\textbf{A} \msquare] Die Adjustierung der p-Werte nach Bonferroni erlaubt es gegen die $\beta$-Inflation vorzugehen, die häufig beim multiplen Testen auftritt. Das globale Powerniveau liegt nicht mehr bei $80\%$ sondern sehr viel niedriger.
\item [\textbf{B} \msquare] Beim multiplen Testen kann es zu einer $\alpha$-Inflation kommen. Das globale Signifikanzniveau liegt nicht mehr bei $5\%$ sondern weit darunter. Daher müssen die p-Werte entsprechend adjustiert werden. Hierfür gibt es verschiedene Verfahren, wobei das Verfahren zur Adjustierung der p-Werte nach Welch das bekanneste Verfahren ist.
\item [\textbf{C} \msquare] Beim multiplen Testen kann es zu einer $\beta$-Inflation kommen. Das globale Signifikanzniveau liegt nicht mehr bei $20\%$. Daher müssen die p-Werte entsprechend adjustiert werden. Hierfür gibt es verschiedene Verfahren, wobei das Verfahren zur Adjustierung der p-Werte nach Bonferroni das bekanneste Verfahren ist.
\item [\textbf{D} \msquare] Das globale Signifikanzniveau liegt nicht mehr bei $5\%$ sondern sehr viel niedriger, bei ca. $1\%$. Es kommt zu einer $\alpha$-Hyperinflation. Dagegen kann mit der Adjustierung der p-Werte nach Bonferroni vorgegangen werden.
\item [\textbf{E} \msquare] Beim multiplen Testen kann es zu einer $\alpha$-Inflation kommen. Das globale Signifikanzniveau liegt nicht mehr bei $5\%$ sondern sehr viel höher. Daher müssen die p-Werte entsprechend adjustiert werden. Hierfür gibt es verschiedene Verfahren, wobei das Verfahren zur Adjustierung der p-Werte nach Bonferroni das bekanneste Verfahren ist.
\end{enumerate}

\section{Aufgabe \hfill (2 Punkte)}




In einem Feldversuch haben Sie einen Behandlungsfaktor mit mehreren Leveln vorliegen. Sie rechnen einen multiplen Vergleich. Vorher hatten Sie eine einfaktorielle ANOVA mit einem signifikanten Ergebnis vorliegen. Welche Aussage ist richtig?



\begin{enumerate}
\item [\textbf{A} \msquare] Beim multiplen Testen kann es zu einer Effektüberschätzung ($\Delta$-Inflation) kommen. Daher müssen die Effekte angepasst werden. Dies geschieht nicht händisch sondern intern in den angewendeten Algorithmen.
\item [\textbf{B} \msquare] Beim multiplen Testen kann es zu einer $\Delta$-Deflation kommen. Das globale Relevanzniveau liegt nicht mehr bei $5\%$ sondern weit darunter. Daher müssen die $\Delta$-Werte entsprechend adjustiert werden. Hierfür gibt es verschiedene Verfahren, wobei das Verfahren zur Adjustierung der $\Delta$-Werte nach Bonferroni das bekanneste Verfahren ist. Die $\Delta$-Werte werden durch die Anzahl an Vergleichen geteilt.
\item [\textbf{C} \msquare] Beim multiplen Testen werden die Effekte der paarweisen Vergleiche ignoriert. Der Nachteil des multiplen Testens ist ja auch, dass wir am Ende keine Effekte mehr vorliegen haben. Eine ANOVA liefert hier bessere Informationen.
\item [\textbf{D} \msquare] Beim multiplen Testen muss der Effekt, hier der Mittelwertsunterschied $\Delta$ aus den paarweisen t-Tests, nicht adjusiert werden.
\item [\textbf{E} \msquare] Wenn ein multipler Test gerechnet wird, dann muss der Effekt $\Delta$ adjustiert werden im Gegensatz zu den p-Werten.
\end{enumerate}
\section*{Statistische Testtheorie}  

\section{Aufgabe \hfill (2 Punkte)}




Geben ist $Pr(D|H_0)$ als mathematischer Ausdruck, welche Aussage ist richtig?



\begin{enumerate}
\item [\textbf{A} \msquare] $Pr(D|H_0)$ ist die Wahrscheinlichkeit nicht die Daten $D$ zu beobachten sondern die Nullhypothese, wenn diese wahr ist.
\item [\textbf{B} \msquare] Die Wahrscheinlichkeit für die Nullhypothese, wenn die Daten wahr sind.
\item [\textbf{C} \msquare] Die Wahrscheinlichkeit der Daten unter der Nullhypothese in der Grundgesamtheit.
\item [\textbf{D} \msquare] $Pr(D|H_0)$ beschreibt die Wahrscheinlichkeit die Teststatistik $T_D$ aus den Daten $D$ zu beobachten, wenn die Nullhypothese wahr ist.
\item [\textbf{E} \msquare] $Pr(D|H_0)$ stellt die Wahrscheinlichkeit die Teststatistik $T$ zu beobachten dar, wenn die Nullhypothese falsch ist.
\end{enumerate}

\section{Aufgabe \hfill (2 Punkte)}



Das statistische Testen basiert auf dem Falsifikationsprinzip. Es besagt,



\begin{enumerate}
\item [\textbf{A} \msquare] ... dass in der Wissenschaft immer etwas falsch sein muss. Sonst gebe es keinen Fortschritt.
\item [\textbf{B} \msquare] ... dass Annahmen an statistische Modelle meist falsch sind.
\item [\textbf{C} \msquare] ... dass ein minderwertes Modell durch ein minderwertiges Modell ersetzt wird. Es gilt das Verifikationsprinzip nach Karl Popper.
\item [\textbf{D} \msquare] ... dass ein schlechtes Modell durch ein schlechteres Modell ersetzt wird. Die Wissenschaft lehnt ab und verifiziert nicht.
\item [\textbf{E} \msquare] ... dass ein minderwertes Modell durch ein weniger minderwertiges Modell ersetzt wird. Es gilt das Falsifikationsprinzip nach Karl Popper.
\end{enumerate}

\section{Aufgabe \hfill (2 Punkte)}



Das Signifikanzniveau $\alpha$ wird auch Fehler 1. Art genannt und liegt bei 5\%. Warum wurde der Grenzwert von 5\% als Signifikanzschwelle gewählt?



\begin{enumerate}
\item [\textbf{A} \msquare] Der Wert ergab sich aus einer Auswertung von 1042 wissenschaftlichen Veröffentlichungen zwischen 1914 und 1948. Der Wert $5\%$ wurde in $28\%$ der Veröffentlichungen genutzt. Daher legte man sich auf diese Zahl fest.
\item [\textbf{B} \msquare] Im Rahmen eines langen Disputs zwischen Neyman und Fischer wurde $\alpha = 5\%$ festgelegt. Leider werden die Randbedingungen und Voraussetzungen an statistsiche Modelle heute immer wieder ignoriert.
\item [\textbf{C} \msquare] In der Wissenschaft gibt es neben der Naturkonstante, die sich aus der Beobachtung der Welt ergibt, noch die Kulturkonstante, die von einer Gruppe Menschen selbstgewählt wird. Dabei ist $\alpha = 5\%$ eine Kulturkonstante und wurde somit eher zufällig gewählt.
\item [\textbf{D} \msquare] Da Wissenschaftler eine Schwelle für die statistische Testentscheidung benötigen wurde $\alpha$ in einer großen Konferenz 1945 gewählt. Damit ist $\alpha = 5\%$ eine Kulturkonstante mit einem Rank einer Naturkonstante.
\item [\textbf{E} \msquare] Als Kulturkonstante hat $\alpha = 5\%$ den Rang einer Naturkonstante und wurde nach langer Diskussion in der UN im Jahre 1983 festgesetzt. Damals auch schon mit der Zustimmung der UdSSR.
\end{enumerate}

\section{Aufgabe \hfill (2 Punkte)}

Betrachten wir die Teststatistik aus einem abstrakteren Blickwinkel. Beim
statistischen Testen wird das \textit{"`signal"'} mit dem
\textit{"`noise"'} aus den Daten $D$ zu einer Teststatistik $T_D$ verrechnet. Welche der Formel
berechnet korrekt die Teststatistik $T_D$?



\begin{enumerate}
\item [\textbf{A} \msquare] Es gilt $T_D = \cfrac{signal}{noise^2}$
\item [\textbf{B} \msquare] Es gilt $T_D = (signal \cdot noise)^2$
\item [\textbf{C} \msquare] Es gilt $T_D = \cfrac{signal}{noise}$
\item [\textbf{D} \msquare] Es gilt $T_D = signal \cdot noise$
\item [\textbf{E} \msquare] Es gilt $T_D = \cfrac{noise}{signal}$
\end{enumerate}

%% ------------------------------------------------------------

\section{Aufgabe \hfill (2 Punkte)}



Eine Analogie kann helfen einen Sachverhalt besser zu verstehen. Wie kann folgende Aussage richtig in die Analogie der statistischen Testtheorie gesetzt werden?

\begin{center}
\textit{$H_0$ ablehnen obwohl die $H_0$ gilt}
\end{center}



\begin{enumerate}
\item [\textbf{A} \msquare] In die Analogie eines Rauchmelders: \textit{Alarm without fire}, dem $\alpha$-Fehler.
\item [\textbf{B} \msquare] \textit{Fire without alarm}, dem $\beta$-Fehler als Analogie von Rauch im Haus.
\item [\textbf{C} \msquare] In die Analogie eines Rauchmelders: \textit{Fire without alarm}, dem $\beta$-Fehler.
\item [\textbf{D} \msquare] \textit{Alarm with fire}, dem $\alpha$-Fehler in der Analogie von Feuer.
\item [\textbf{E} \msquare] In die Analogie eines Rauchmelders: \textit{Alarm with fire}.
\end{enumerate}

\section{Aufgabe \hfill (2 Punkte)}



Sie sollen in Ihrer Abschlussarbeit die Relevanz und die Signifikanz in einer statistischen Maßzahl vereinen. Welche Aussage ist richtig?



\begin{enumerate}
\item [\textbf{A} \msquare] Das OR. Als Chancenverhältnis gibt es das Verhältnis von Relevanz und Signifikanz wieder.
\item [\textbf{B} \msquare] Über das Konfidenzintervall. Das Konfidenzinterval beitet eine Entscheidung über die Signifikanz und zusätzlich kann über die Visualizierung des Konfidenzintervals eine Relevanzschwelle definiert werden.
\item [\textbf{C} \msquare] Über das Konfidenzintervall. Das Konfidenzinterval inkludiert eine Entscheidung über die Relevanz und zusätzlich kann über die Visualizierung des Konfidenzintervals eine Signifikanzschwelle vom Forschenden definiert werden.
\item [\textbf{D} \msquare] Der p-Wert. Durch den Vergleich mit $\alpha$ lässt sich über die Signifikanz entscheiden und der $\beta$-Fehler erlaubt über die Power eine Einschätzung der Relevanz.
\item [\textbf{E} \msquare] Einem Konfidenzintervall. Das Konfidenzinterval bringt durch eine Visualisierung und drei Intervallgrenzen die Möglichkeit mit, eine Relevanzschwelle neben der Signifikanzschwelle und der $\alpha$-Schwelle zu definieren.
\end{enumerate}

\section{Aufgabe \hfill (2 Punkte)}



Sie haben ein Signifikanzniveau $\alpha$ gleich 5\% vorliegen. Welche Aussage zusammen mit dem $p$-Wert ist richtig?



\begin{enumerate}
\item [\textbf{A} \msquare] Wir schauen, ob der $p$-Wert größer ist als das Signifikanzniveau $\alpha$ und vergleichen somit Wahrscheinlichkeiten. Die Wahrscheinlichkeiten werden als Flächen unter der Kurve der Teststaistik dargestellt, wenn die $H_A$ gilt.
\item [\textbf{B} \msquare] Wir vergleichen mit dem $p$-Wert und dem Signifikanzniveau $\alpha$ Wahrscheinlichkeiten und damit die absoluten Werte auf einem Zahlenstrahl, wenn die $H_0$ gilt.
\item [\textbf{C} \msquare] Wir vergleichen die Effekte des $p$-Wertes mit den Effekten der Signifikanzschwelle unter der Annahme der Nullhypothese. Dabei gilt, dass wir die Nullhypothese nur ablehnen können anhand des Falsifikationsprinzips.
\item [\textbf{D} \msquare] Wir machen ein Aussage über die Flächen und der Kurve der Teststatistik, wenn die $H_0$ gilt. Dabei werden Wahrscheinlichkeiten vergleichen, die durch die Flächen unter der Kurve repräsentiert werden.
\item [\textbf{E} \msquare] Wir vergleichen mit dem $p$-Wert und dem Signifikanzniveau $\alpha$ absolute Werte auf einem Zahlenstrahl und damit den Unterschied der Teststatistiken, wenn die $H_0$ gilt.
\end{enumerate}

\section{Aufgabe \hfill (2 Punkte)}



Um die Testtheorie besser zu verstehen, mag es manchmal sinnvoll sein ein Beispiel aus dem Alltag zu wählen. Die Ergebnisse der Analyse durch einen statistischen Test können auch in grobe Analogie zur Wettervorhersage gebracht werden. Welche Aussage trifft am ehesten zu?



\begin{enumerate}
\item [\textbf{A} \msquare] In der Analogie der Regenwahrscheinlichkeit in einem bestimmten Gebiet: ein statistischer Test gibt die Wahrscheinlichkeit für ein Ereignis in einem Experiment mit den Daten $D$ wieder und lässt sich kaum verallgemeinern.
\item [\textbf{B} \msquare] In der Analogie der Sonnenscheindauer: Wie lange kann mit einem entsprechenden Effekt gerechnet werden? Die Wahrscheinlichkeit für den Effekt gibt der statistische Test wieder.
\item [\textbf{C} \msquare] In der Analogie der Wahrscheinlichkeit für Regen: ein statistischer Test erlaubt die Wahrscheinlichkeit für ein Ereignis abzuschätzen. Die Stärke des Effektes können wir nicht bestimmen.
\item [\textbf{D} \msquare] In der Analogie des Niederschlags oder Regenmenge: ein statistischer Test gibt die Stärke eines Effektes wieder. Zum Beispiel, wie hoch ist der Mittelwertsunterschied.
\item [\textbf{E} \msquare] In der Analogie der Maximaltemperatur: Was ist der maximale Unterschied zwischen zwei Gruppen. Wir erhalten hier eine Aussage über die Spannweite und den maximalen Effekt.
\end{enumerate}

\section{Aufgabe \hfill (2 Punkte)}



In Ihrer Forschungsarbeit wollen Sie eine Aussage über ein untersuchtes Individuum treffen. Dazu nutzen Sie eine ANOVA als statistischen Test. Erhalten Sie eine valide Aussage aus einem statistischen Test?



\begin{enumerate}
\item [\textbf{A} \msquare] Nein, wir können ein untersuchtes Individuum nicht mit einem t-Test auswerten. Wir erhalten keine Aussage zum Individuum. Wir können aber den Effekt als Quelle der Relevanz nutzen.
\item [\textbf{B} \msquare] Nein, wir können ein untersuchtes Individuum nicht mit einer ANOVA auswerten. Wir erhalten keine Aussage zum Individuum. Wir können aber den Test adjustieren und so die Auswertung ermöglichen.
\item [\textbf{C} \msquare] Nein, ein untersuchtes Individuum können wir mit einem statistischen Test nicht auswerten. Wir erhalten keine Aussage zum Individuum.
\item [\textbf{D} \msquare] Ja, wir können ein untersuchtes Individuum mit einer ANOVA auswerten. Wir erhalten eine Aussage zum Individuum.
\item [\textbf{E} \msquare] Weder eine Ausssage über die Population noch über das Individuum ist mit einem statistischen Test möglich. Wir erhalten eine Aussage über ein Experiment.
\end{enumerate}

\section{Aufgabe \hfill (2 Punkte)}



In der statistischen Testtheorie gibt es den Begriff \textit{Power}. Was sagt der statistische Begriff \textit{Power} aus?



\begin{enumerate}
\item [\textbf{A} \msquare] Die Power wird berechnet und ist keine Eigenschaft des Tests. Die Power wird auf $80\%$ gesetzt und beschreibt mit welcher Wahrscheinlichkeit $H_0$ \textit{bewiesen wird}
\item [\textbf{B} \msquare] Die Power $1-\beta$ wird auf 80\% gesetzt. Alle statistischen Tests sind so konstruiert, dass die $H_A$ mit 80\% \textit{bewiesen wird}.
\item [\textbf{C} \msquare] Die Power $1-\beta$ wird auf 80\% gesetzt. Damit liegt die Wahrscheinlichkeit für die $H_0$ bei 20\%.
\item [\textbf{D} \msquare] Alle statistischen Tests sind so konstruiert, dass die $H_A$ mit 20\% \textit{bewiesen wird}. Die Power ist $1-\beta$ mit $\beta$ gleich 80\% gesetzt.
\item [\textbf{E} \msquare] Die Power beschreibt die Wahrscheinlichkeit die $H_A$ abzulehnen. Wir testen die Power jedoch nicht.
\end{enumerate}

\section{Aufgabe \hfill (2 Punkte)}



Welche Aussage über den Effekt eines statistischen Tests ist richtig?



\begin{enumerate}
\item [\textbf{A} \msquare] Der Effekt eines statistischen Tests beschreibt die biologisch interpretierbare Ausgabe eines Tests. Damit ist der Effekt direkt mit dem Begriff der Signifikanz verbunden. Die Entscheidung über die Signifikanz trifft der Forschende unabhängig von der Relevanz eines statistsichen Tests.
\item [\textbf{B} \msquare] Der Effekt eines statistischen Tests beschreibt die biologisch interpretierbare Ausgabe eines Tests. Moderen Algorithmen liefern keine Effekte mehr sondern nur noch bedingte Wahrscheinlichkeiten. Der Effekt spielt in der modernen Statistik keine Rollen mehr.
\item [\textbf{C} \msquare] Durch den Effekt erfahren wir die biologisch interpretierbare Ausgabe eines statistischen Tests. Zum Beispiel das $\eta^2$ aus einer ANOVA. Damit können wir die Relevanz direkt mit dem Effekt verbinden. Am Ende muss der Forschende aber entscheiden, ob der Effekt entsprechend seinen Erwartungen als bedeutet zu bewerten ist.
\item [\textbf{D} \msquare] Der Effekt eines statistischen Tests beschreibt die mathematisch interpretierbare Ausgabe eines Tests. Damit ist der Effekt direkt mit dem Begriff der Signifikanz verbunden. Die Entscheidung über die Signifikanz trifft der Forschende unabhängig von der Relevanz eines statistsichen Tests.
\item [\textbf{E} \msquare] Der Forschende muss am Anfang wissen, ob das Eregbnis eines Experiments relevant für seine Forschung ist. Dafür kann der Effekt eines statistischen Tests genutzt werden oder auch der Prähoc-Test. Damit beschreibt der Effekt den biologischen interpretierbaren Teil eines Experimnts vor der Durchführung. Zum Beispiel der Unterschied zwischen zwei Mittelwerten.
\end{enumerate}

\section{Aufgabe \hfill (2 Punkte)}



Welche Aussage über die Entscheidung anhand des 95\%-Konfidenzintervalls gegen die
Nullhypothese ist richtig?



\begin{enumerate}
\item [\textbf{A} \msquare] Ist $T_{D}$ h{"o}her als der kritische Wert $T_{\alpha = 5\%}$ dann wird die Nullhypothese $H_0$ abgelehnt.
\item [\textbf{B} \msquare] Ist in dem 95\%-Konfidenzintervall nicht die Null enthalten dann wird die Nullhypothese $H_0$ abgelehnt.
\item [\textbf{C} \msquare] Anhand des 95\%-Konfidenzintervalls lässt sich wie folgt eine Entscheidung treffen. Liegt der Wert in dem Signifikanzniveauintervall $\alpha$ dann kann die Nullhypothese abgelehnt werden.
\item [\textbf{D} \msquare] Ist $Pr(D|H_0)$ kleiner als das Signifikanzniveau $\alpha$ gleich $5\%$ dann wird die Nullhypothese $H_0$ abgelehnt.
\item [\textbf{E} \msquare] Anhand des 95\%-Konfidenzintervalls lässt sich wie folgt eine Entscheidung treffen. Liegt der Wert über oder gleich dem Signifikanzniveau $\alpha$ dann kann die Nullhypothese abgelehnt werden.
\end{enumerate}

\section{Aufgabe \hfill (2 Punkte)}



Ein statistischer Test benötigt für die richtige Durchführung Hypothesen $H$, sonst ist der Test nicht zu interpretieren. Welche Aussage ist richtig?



\begin{enumerate}
\item [\textbf{A} \msquare] Die Hypothesen $H_0$ und $H_A$ sind rein prosarischer Natur und bilden keinen mathematischen Hintergrund ab. In der Statistik wird die wissenschaftliche Fragestellung getestet. Daher stehen auch die verständlichen Hypothesen im Mittelpunkt der biologischen Interpretation.
\item [\textbf{B} \msquare] Es gibt ein statistisches Hypothesenpaar mit der Hypothese für und gegen die wissenschaftliche Fragestellung. Die Hypothesen werden $H_{pro}$ und $H_{contra}$ bezeichnet.
\item [\textbf{C} \msquare] Mit der Nullhypothese $H_A$ und der Alternativehypothese $H_0$ gibt es zwei Hypothesen, die aber selten genutzt werden.
\item [\textbf{D} \msquare] Es gibt ein statistisches Hypothesenpaar mit der Nullhypothese $H_0$ und der Alternativehypothese $H_A$ oder $H_1$.
\item [\textbf{E} \msquare] Es gibt ein Hypothesenset bestehend aus $k$ Hypothesen. Meistens wird die Nullhypothese $H_0$ und die Alternativhypothese $H_A$ verwendet. Wegen des Falsifikationsprinzips ist es wichtig, die bekannte falsche und unbekannte richtige Hypothese mit in das Set zu nehmen.
\end{enumerate}
\section*{Statistische Tests für Gruppenvergleiche} 

\section{Aufgabe \hfill (2 Punkte)}



Nach einem Feldexperiment wollen Sie zwei Gruppen mit einem Welch t-Test vergleichen. Welche Aussage ist auch für den Student t-Test richtig?



\begin{enumerate}
\item [\textbf{A} \msquare] Der t-Test vergleicht zwei oder mehr Gruppen indem die Mittelwerte miteinander verglichen werden.
\item [\textbf{B} \msquare] Der t-Test testet generell zu einem erhöhten $\alpha$-Niveau von 20\%.
\item [\textbf{C} \msquare] Der t-Test ist ein Vortest der ANOVA und basiert daher auf dem Vergleich von Streuungsparametern
\item [\textbf{D} \msquare] Der t-Test berechnet die Differenz von zwei Mittelwerten als Effekt und gibt eine Entscheidung, ob sich die beiden Mittelwerte in den Gruppen signifikant unterscheiden.
\item [\textbf{E} \msquare] Der t-Test vergleicht die Mittelwerte von zwei Gruppen unter der strikten Annahme von Varianzhomogenität. Sollte keine Varianzhomogenität vorliegen, so gibt es keine Möglichkeit den t-Test in einer Variante anzuwenden.
\end{enumerate}

\section{Aufgabe \hfill (2 Punkte)}



In einer Studie zur Bewertung der Wirkung des Mikronährstoff Sulfit auf den Ertrag in t/ha  von Mais im Vergleich zu einer Kontrolle entstand folgende Abbildung. Der Versuch wurde in 16 Parzellen pro Gruppe durchgeführt. Welche Aussage ist im Bezug auf einen t-Test ist richtig?



{\centering \includegraphics[width=\maxwidth]{img/mc-testing-ttest-02-1} 

}







\begin{enumerate}
\item [\textbf{A} \msquare] Die Barplots deuten auf einen signifikanten Unterschied. Der Effekt liegt vermutlich bei -5. Wir müssen aber einen Posthoc-Test rechnen um den Effekt wirklich bestimmen zu können.
\item [\textbf{B} \msquare] Der Test deutet auf ein signifikanten Unterschied hin. Der Effekt liegt vermutlich bei -5.
\item [\textbf{C} \msquare] Der Effekt und die Signifikanz lassen sich nicht aus Barplots abschätzen. Höchtens der Effekt als relativer Unterschied zwischen der Höhe der Barplots. Standard ist der mediane Unterschied aus Boxplots.
\item [\textbf{D} \msquare] Nach Betrachtung des Barplots liegt kein signifikanter Unterschied vor. Der Effekt liegt bei -5.
\item [\textbf{E} \msquare] Nach Betrachtung des Barplots liegt kein signifikanter Unterschied vor. Der Effekt kann nicht bei einem t-Test aus Barplots bestimmt werden.
\end{enumerate}

\section{Aufgabe \hfill (2 Punkte)}




Sie rechnen einen gepaarten t-Test, da Ihre Beobachtungen verbunden sind. Welche der folgenden Aussagen ist richtig?



\begin{enumerate}
\item [\textbf{A} \msquare] Der gepaarte t-Test wird gerechnet, wenn die Beobachtungen nicht unabhängig voneinander sind. Wir messen wiederholt an dem gleichen Probanden oder Tier oder Pflanze. Wir bilden die Differenzen um den gepaarten t-Test rechnen zu können.
\item [\textbf{B} \msquare] Wenn die Beobachtungen unabhängig voneinander sind, rechnen wir einen gepaarten t-Test. Messen wir wiederholt an dem gleichen Tier oder Pflanze dann bilden wir das Produkt zwischen den zwei Messpunkten.
\item [\textbf{C} \msquare] Abhängige Beobachtungen müssen gesondert in einem gepaarten t-Test modelliert werden. Wenn wiederholt an dem gleichen Tier oder Pflanze gemessen wird, dann bilden wir den Quotienten zwischen den beiden Zeitpunkten. Auf den Quotienten rechnen wir den gepaarten t-Test.
\item [\textbf{D} \msquare] Der gepaarte t-Test nutzt die Varianz der Beobachtungen jeweils paarweise und bildet dafür eine verbundene Stichprobe. Dieser Datensatz $d$ dient dann zur Differenzbildung.
\item [\textbf{E} \msquare] Der gepaarte t-Test wird genutzt, wenn die Differenzen der Beobachtungen verbunden sind und wir dadurch die Unabhäängigkeit nicht mehr vorliegen haben.
\end{enumerate}

\section{Aufgabe \hfill (2 Punkte)}



Nach einem Experiment mit drei Weizensorten ergibt eine ANOVA ($p = 0.048$) einen signifikanten Unterschied für den Ertrag. Sie führen anschließend die paarweisen t-Tests für alle Vergleiche der verschiedenen Weizensorten durch. Nach der Adjustierung für multiples Testen ist kein p-Wert unter der $\alpha$-Schwelle. Sie schauen sich auch die rohen, unadjustierten p-Werte an und finden hier als niedrigsten p-Wert $p_{3-2} = 0.051$. Welche Aussage ist richtig?




\begin{enumerate}
\item [\textbf{A} \msquare] Die ANOVA testet auf der gesamten Fallzahl. Es wäre besser die ANOVA auf der gleichen Fallzahl wie die einzelnen t-Tests zu rechnen.
\item [\textbf{B} \msquare] Hier kommt der Effekt der stiegenden Fallzahl auf die Anzahl an signifikante Ergebnisse zu tragen. Da die ANOVA auf weniger Fallzahl testet als die paarweisen t-Tests, kann die ANOVA schwerer einen signifikanten Unterscheid nachweisen.
\item [\textbf{C} \msquare] Es gibt einen Fehler in der Varianzstruktur. Daher kann die ANOVA nicht richtig sein und paarweise t-Tests liefern das richtige Ergebnis.
\item [\textbf{D} \msquare] Das ist kein Wunder. Die ANOVA testet nicht auf der gesamten Fallzahl und die paarweisen t-Tests gewinnen immer eine oder mehr Gruppen als Fallzahl dazu. Mit steigender Fallzahl sind mehr signifikante Unterschiede zu erwarten. Die p-Werte unterscheiden sich numerisch auch kaum.
\item [\textbf{E} \msquare] Das ist kein Wunder. Die ANOVA testet auf der gesamten Fallzahl und die paarweisen t-Tests verlieren immer eine oder mehr Gruppen als Fallzahl. Mit steigender Fallzahl sind mehr signifikante Unterschiede zu erwarten. Die p-Werte unterscheiden sich numerisch auch kaum.
\end{enumerate}
    
% -----------------------------------------------------------------------
\clearpage
% -----------------------------------------------------------------------
\part{Deskriptive Statistik \& Explorative Datenanalyse}
% -----------------------------------------------------------------------

\section{Aufgabe \hfill (8 Punkte)}

\textit{Geben Sie grundsätzlich Formeln und Rechenweg zur Lösung der Teilaufgaben mit an!} \\[1Ex]
 

 
%% --------------------------------------------------------------------
\begin{minipage}[t]{0.5\textwidth}
\includegraphics[width = 1.3cm]{/Users/kruppajo/work/GitHub/exam/avatare/Paula.png}
\end{minipage}
\begin{minipage}[t]{0.5\textwidth}
\hfill
\href{https://youtu.be/t0WYa_LVc5o}{\includegraphics[width = 2cm]{img/youtube}}\\[1Ex]
\end{minipage}
\vspace{1ex}
%% --------------------------------------------------------------------



Eine echte Herausforderung für sie war schon immer der Perfektionismus gewesen. Ein leidiges Lied. Deshalb gilt anschauen, was andere vor einem gemacht haben. Für Paula ist es eine Möglichkeit schneller ans Ziel zu gelangen. Paula soll in ihrer Abschlussarbeit Brokoli untersuchen. Die Behandlung in ihrer Abschlussarbeit werden verschiedene Düngestufen ($ctrl$, $low$ und $high$) sein. Erheben wird Paula als Endpunkt ($Y$) \textit{Trockengewicht} benannt als \textit{drymatter} in ihrer Exceldatei. Von ihrer Betreuer erhält sie nun folgende Abbildung von Barplots, die sie erstmal zur Übung nachbauen soll, bevor sie mit dem eigentlichen Versuch beginnt.



{\centering \includegraphics[width=\maxwidth]{img/barplot-02-1} 

}




Leider kennt sich Paula mit der Erstellung von Barplots in \Rlogo nicht aus. Deshalb braucht sie bei der Visualisierung Ihre Hilfe!

\begin{enumerate}
\item Formulieren Sie die wissenschaftliche Fragestellung! \textbf{(1 Punkt)}
\item Erstellen Sie eine Tabelle mit den statistischen Maßzahlen der drei Barplots! \textit{Beachten Sie die korrekte Darstellungsform der statistischen Maßzahlen!} \textbf{(3 Punkte)}
\item Erstellen Sie einen beispielhaften Datensatz im \Rlogo üblichen Format, aus dem die drei Barplots \textit{möglicherweise} erstellt wurden! \textbf{(2 Punkte)}
\item Kann Paula einen Unterschied zwischen den Behandlungen erwarten? Begründen Sie Ihre Antwort! \textbf{(2 Punkte)}
\end{enumerate} 
\clearpage
% -----------------------------------------------------------------------

\section{Aufgabe \hfill (8 Punkte)}

\textit{Geben Sie grundsätzlich Formeln und Rechenweg zur Lösung der Teilaufgaben mit an!} \\[1Ex]
 

 
%% --------------------------------------------------------------------
\begin{minipage}[t]{0.5\textwidth}
\includegraphics[width = 1.3cm]{/Users/kruppajo/work/GitHub/exam/avatare/Nilufar.png}
\end{minipage}
\begin{minipage}[t]{0.5\textwidth}
\hfill
\href{https://youtu.be/vXnLttRL_VI}{\includegraphics[width = 2cm]{img/youtube}}\\[1Ex]
\end{minipage}
\vspace{1ex}
%% --------------------------------------------------------------------



Wenn die Erwartung nicht wäre, ja dann wäre wohl vieles möglich für Nilufar! Aber so.. Deshalb gilt anschauen, was andere vor einem gemacht haben. Für Nilufar ist es eine Möglichkeit schneller ans Ziel zu gelangen. Deshalb hat sich Nilufar viele Poster in der Fakultät angeschaut und ist zum Schluß gekommen, dass Barplots eine häufig genutzte Abbildung sind. Nilufar soll nun in ihrem Projektbericht Lauch untersuchen. Die Behandlung in ihrem Projektbericht sind verschiedene Substrattypen ($torf$, $40p60n$ und $70p30n$). Erhoben wurden von Nilufar als Messwert ($Y$) \textit{Frischegewicht} benannt als \textit{freshmatter} in ihrer Exceldatei. Erwartungsgemäß erhält sie von ihrem Betreuer den Auftrag die erhobenen Daten als Barplots darzustellen. Dann kann Nilufar auch schonmal abschätzen, was bei einem statistischen Test rauskommen könnte. Na dann mal los. Nilufar schafft sich die nötige Stimmung. Nilufar nickt im Takt von Deichkind und bemerkt dabei gar nicht was das Huhn schon wieder anstellt.

\begin{table}[!h]
\centering
\begin{tabular}{cc}
\toprule
treatment & freshmatter\\
\midrule
70p30n & 27.5\\
torf & 35.0\\
70p30n & 35.3\\
40p60n & 35.9\\
70p30n & 32.1\\
\addlinespace
70p30n & 33.6\\
40p60n & 27.3\\
torf & 35.7\\
torf & 44.0\\
70p30n & 35.0\\
\addlinespace
40p60n & 36.3\\
\bottomrule
\end{tabular}
\end{table}



Leider kennt sich Nilufar mit der Erstellung von Barplots nicht aus. Deshalb braucht sie bei der Visualisierung Ihre Hilfe!

\begin{enumerate}
\item Formulieren Sie die wissenschaftliche Fragestellung! \textbf{(1 Punkt)}
\item Zeichnen Sie in \textit{einer} Abbildung die Barplots für die Behandlung von Lauch! Beschriften Sie die Achsen entsprechend!\textbf{(4 Punkte)}
\item Beschriften Sie \textit{einen} Barplot mit den gängigen statistischen Maßzahlen! \textbf{(2 Punkte)}
\item Wenn Nilufar \textit{keinen Effekt} zwischen den Behandlungen von Lauch erwarten würde, wie sehen dann die Barplots aus? \textit{Antworten Sie mit einer Skizze der Barplots!}
  \textbf{(1 Punkt)}
\end{enumerate} 
\clearpage
% -----------------------------------------------------------------------

\section{Aufgabe \hfill (9 Punkte)}

\textit{Geben Sie grundsätzlich Formeln und Rechenweg zur Lösung der Teilaufgaben mit an!} \\[1Ex]
 

 
%% --------------------------------------------------------------------
\begin{minipage}[t]{0.5\textwidth}
\includegraphics[width = 1.3cm]{/Users/kruppajo/work/GitHub/exam/avatare/Jonas.png}
\end{minipage}
\begin{minipage}[t]{0.5\textwidth}
\hfill
\href{https://youtu.be/Xf0yE-o7bEU}{\includegraphics[width = 2cm]{img/youtube}}\\[1Ex]
\end{minipage}
\vspace{1ex}
%% --------------------------------------------------------------------



Boxplots sind bedeutend in der Darstellung von wissenschaftlichen Ergebnissen. Leider hat sich Jonas nicht gemerkt, welche statistischen Maßzahlen für einen Barplot erhoben werden müssen. Besser wäre was anderes gewesen. Jonas liebt Stricken. Darin kann er sich wirklich verlieren und immer wieder neu begeistern. Das ist in soweit doof, da nach seiner Betreuerin erstmal ein Barplot nachgebaut werden soll, bevor es mit seinem Projektbericht losgeht. Dann hat er schonmal den \Rlogo Code vorliegen und nachher geht dann alles schneller. Na dann mal los. Jonas schafft sich die nötige Stimmung. Wenn Iron Maiden ertönt, dann sucht das Meerschweinchen schleunigst Schutz unter dem Sofa. Jonas schüttelt den Kopf. In der Behandlung für Lauch werden verschiedene Bewässerungstypen ($low$, $mid$ und $high$) sein. Erfasst wird als Messwert ($Y$) \textit{Frischegewicht}. Jonas soll dann \textit{freshmatter} in seiner Exceldatei eintragen. Aber nur in passender Atmospäre! Auf seinem Second Screen läuft Mission Impossible und Jonas schaufelt Snickers. Nicht effizient, aber gut.



{\centering \includegraphics[width=\maxwidth]{img/boxplot-02-zer-1} 

}




Leider kennt sich Jonas mit der Erstellung von Boxplots in \Rlogo nicht aus. Deshalb braucht er bei der Visualisierung Ihre Hilfe!

\begin{enumerate}
\item Erstellen Sie eine Tabelle mit den statistischen Maßzahlen aus der obigen Abbildung der drei Boxplots! \textit{Beachten Sie die korrekte Darstellungsform der statistischen Maßzahlen!} \textbf{(3 Punkte)}
\item Beschriften Sie \textit{einen} der Boxplots mit den gängigen statistischen Maßzahlen! \textbf{(2 Punkte)}
\item Erstellen Sie einen beispielhaften Datensatz, aus dem die drei Boxplots \textit{möglicherweise} erstellt wurden, im \Rlogo üblichen Format! \textbf{(2 Punkte)}
\item Kann Jonas einen Unterschied zwischen den Behandlungen erwarten? Begründen Sie Ihre Antwort! \textbf{(2 Punkte)}
\end{enumerate} 
\clearpage
% -----------------------------------------------------------------------

\section{Aufgabe \hfill (9 Punkte)}

\textit{Geben Sie grundsätzlich Formeln und Rechenweg zur Lösung der Teilaufgaben mit an!} \\[1Ex]
 

 
%% --------------------------------------------------------------------
\begin{minipage}[t]{0.5\textwidth}
\includegraphics[width = 1.3cm]{/Users/kruppajo/work/GitHub/exam/avatare/Jessica.png}
\end{minipage}
\begin{minipage}[t]{0.5\textwidth}
\hfill
\href{https://youtu.be/0xc0jIPeiyw}{\includegraphics[width = 2cm]{img/youtube}}\\[1Ex]
\end{minipage}
\vspace{1ex}
%% --------------------------------------------------------------------



Boxplots sind bedeutend in der Darstellung von wissenschaftlichen Ergebnissen. Leider hat sich Jessica nicht gemerkt, welche statistischen Maßzahlen für einen Boxplot erhoben werden müssen. Besser wäre was anderes gewesen. Warhammer. Ein wunderbares Hobby um sich drin zu verlieren und Abstand zu bekommen. Jessica denkt gerne über Warhammer nach. Das ist in soweit doof, da nach ihrem Betreuer nun Boxplots aus ihren Daten gebaut werden sollen, bevor es mit dem statistischen Testen weitergeht. Anhand von Boxplots lässt sich eine Aussage über die Varianzhomogenität über die Behandlungsgruppen treffen. Die Behandlung für Lauch waren verschiedene Lüftungssystemen und Folientunneln ($ctrl$ und $tornado$). Erfasst wurde von Jessica als Messwert ($Y$) \textit{Trockengewicht}. Jessica hat dann \textit{drymatter} in ihrer Exceldatei eintragen. Aber nur in passender Atmospäre! Schon dutzende Male gesehen: Herr der Ringe. Aber immer noch großartig zusammen mit Schokobons.

\begin{table}[!h]
\centering
\begin{tabular}{cc}
\toprule
treatment & drymatter\\
\midrule
ctrl & 20.0\\
tornado & 35.4\\
tornado & 39.1\\
tornado & 37.9\\
tornado & 37.3\\
\addlinespace
tornado & 42.4\\
tornado & 37.7\\
ctrl & 24.6\\
ctrl & 34.6\\
tornado & 33.9\\
\addlinespace
ctrl & 18.6\\
ctrl & 18.5\\
ctrl & 38.0\\
ctrl & 28.1\\
ctrl & 40.3\\
\addlinespace
tornado & 36.6\\
tornado & 39.4\\
ctrl & 42.4\\
tornado & 47.0\\
ctrl & 26.1\\
\bottomrule
\end{tabular}
\end{table}



Leider kennt sich Jessica mit der Erstellung von Boxplots nicht aus. Deshalb braucht sie bei der Visualisierung Ihre Hilfe!

\begin{enumerate}
\item Zeichnen Sie in \textit{einer} Abbildung die beiden Boxplots für die zwei Behandlungen von Lauch! Beschriften Sie die Achsen entsprechend! \textbf{(5 Punkte)} 
\item Wie ist Ihr Vorgehen, wenn Sie eine \textit{gerade} Anzahl an
  Beobachtungen pro Gruppe haben? \textbf{(1 Punkt)}
\item Beschriften Sie \textit{einen} der beiden Boxplots mit den gängigen
  statistischen Maßzahlen! \textbf{(2 Punkte)}
\item Wenn Sie \textit{keinen Effekt} zwischen den Behandlungen von
  Lauch erwarten würden, wie sehen dann die beiden Boxplots aus?
  \textit{Antworten Sie mit einer Skizze der Boxplots!}
  \textbf{(1 Punkt)}
\end{enumerate} 
\clearpage
% -----------------------------------------------------------------------

\section{Aufgabe \hfill (8 Punkte)}

\textit{Geben Sie grundsätzlich Formeln und Rechenweg zur Lösung der Teilaufgaben mit an!} \\[1Ex]
 

 
%% --------------------------------------------------------------------
\begin{minipage}[t]{0.5\textwidth}
\includegraphics[width = 1.3cm]{/Users/kruppajo/work/GitHub/exam/avatare/Tina.png}
\end{minipage}
\begin{minipage}[t]{0.5\textwidth}
\hfill
\href{https://youtu.be/aXvxGC4YLqk}{\includegraphics[width = 2cm]{img/youtube}}\\[1Ex]
\end{minipage}
\vspace{1ex}
%% --------------------------------------------------------------------



In einem Gespräch mit ihrem Betreuer wird Tina gebeten seine Daten aus einem Stallexperiment mit Hühnern in einem Histogramm darzustellen. Aus den Boxen wummert Tocotronic und ihr Mund ist verklebt von Katjes. 'Herrlich', denkt Tina. In ihrem Experiment hat er die Anzahl an gedrehten Haaren/Federn pro $cm^2$ erst fotographiert und dann ausgezählt. Laut ihrem Betreuer soll das Histogramm helfen, die Verteilung der die Anzahl an gedrehten Haaren/Federn pro $cm^2$ zu bestimmen. Es wäre einfacher, wenn da nicht noch was wäre. Wenn die Wut nicht wäre, ja dann wäre wohl vieles möglich für Tina! Aber so.. Tina streichelt liebevoll die Spinne. Der Kopf ist in ihrem Schloß vergraben um den Klang von Tocotronic zu dämpfen.

\begin{center}
Die Anzahl an gedrehten Haaren/Federn pro $cm^2$: 3, 2, 3, 1, 4, 2, 6, 5, 4, 5, 7, 3, 2, 6, 1, 5, 10, 4, 4, 9, 4, 2, 4, 3, 6, 8, 3, 2, 8, 8, 6, 4, 3
\end{center}

Leider kennt sich Tina mit der Erstellung von Histogrammen überhaupt nicht aus. Deshalb braucht sie bei der Erstellung Ihre Hilfe!

\begin{enumerate}
\item Zeichen Sie ein Histogramm um die Verteilung der Daten zu visualisieren! (\textbf{3 Punkte})
\item Beschriften Sie die Achsen der Abbildung! (\textbf{2 Punkte})
\item Ergänzen Sie die absoluten und relativen Häufigkeiten in der
  Abbildung! \textbf{(1 Punkt)}
\item Berechnen Sie aus den Daten die \textit{Wahrscheinlichkeit}
  mehr als die Anzahl 6 zu beobachten! \textbf{(1
    Punkt)}
\item Berechnen Sie aus den Daten die \textit{Chance} mehr
  als die Anzahl 6 zu beobachten! \textbf{(1 Punkt)}
\end{enumerate}

 
\clearpage
% -----------------------------------------------------------------------

\section{Aufgabe \hfill (8 Punkte)}

\textit{Geben Sie grundsätzlich Formeln und Rechenweg zur Lösung der Teilaufgaben mit an!} \\[1Ex]
 

 
%% --------------------------------------------------------------------
\begin{minipage}[t]{0.5\textwidth}
\includegraphics[width = 1.3cm]{/Users/kruppajo/work/GitHub/exam/avatare/Jessica.png}
\end{minipage}
\begin{minipage}[t]{0.5\textwidth}
\hfill
\href{https://youtu.be/ORHSPTCdfeY}{\includegraphics[width = 2cm]{img/youtube}}\\[1Ex]
\end{minipage}
\vspace{1ex}
%% --------------------------------------------------------------------



Aus den Boxen wummert David Bowie und ihr Mund ist verklebt von Schokobons. 'Herrlich', denkt Jessica. Jessica betrachtet die folgenden Daten nach einem Kreuzungsexperiment mit Hühnern. In dem Experiment wurden die mittlere Anzahl an gedrehten Haaren pro $cm^2$ gezählt. Nach der Meinung ihrer Betreuerin muss als erstes geschaut werden, wie diese verteilt sind. Also welcher statistischen Verteilung die mittlere Anzahl an gedrehten Haaren pro $cm^2$ folgen. Dazu soll Jessica ein Histogramm verwenden. Dann hätte man auch einen guten Überblick über den Endpunkt ($Y$). Es wäre einfacher, wenn da nicht noch was wäre. Eine echte Herausforderung für sie war schon immer der Mangel gewesen. Ein leidiges Lied. Jessica streichelt liebevoll die Hündin. Der Kopf ist in ihrem Schloß vergraben um den Klang von David Bowie zu dämpfen.

\begin{center}
Die mittlere Anzahl an gedrehten Haaren pro $cm^2$: 8.5, 8.2, 9.8, 11, 13.6, 10, 9.1, 11.8, 9.8, 9.8, 9.2, 8.6, 8.5, 11.2, 9.4, 10.6, 10, 8.8, 8.8, 11.1, 13.4, 9.8, 10.4, 6.1, 10.9, 6.6, 7.4, 12
\end{center}

Leider kennt sich Jessica mit der Erstellung von Histogrammen überhaupt nicht aus. Deshalb braucht sie bei der Erstellung Ihre Hilfe!

\begin{enumerate}
\item Zeichen Sie ein Histogramm um die Verteilung der Daten zu visualisieren! (\textbf{3 Punkte})
 \item Erläutern Sie Ihr Vorgehen um ein Histogramm für kontinuierliche Daten zu zeichnen!  (\textbf{2 Punkte})
\item Beschriften Sie die Achsen der Abbildung! (\textbf{2 Punkte})
\item Ergänzen Sie die relativen Häufigkeiten in der Abbildung! \textbf{(1 Punkt)}  
\end{enumerate}

 
\clearpage
% -----------------------------------------------------------------------

\section{Aufgabe \hfill (10 Punkte)}

\textit{Geben Sie grundsätzlich Formeln und Rechenweg zur Lösung der Teilaufgaben mit an!} \\[1Ex]
 

 
%% --------------------------------------------------------------------
\begin{minipage}[t]{0.5\textwidth}
\includegraphics[width = 1.3cm]{/Users/kruppajo/work/GitHub/exam/avatare/Tina.png}
\end{minipage}
\begin{minipage}[t]{0.5\textwidth}
\hfill
\href{https://youtu.be/VAqiUdV4WQ0}{\includegraphics[width = 2cm]{img/youtube}}\\[1Ex]
\end{minipage}
\vspace{1ex}
%% --------------------------------------------------------------------




Tina schmeißt noch eine Handvoll Katjes in ihren Rachen. Im Hintergrund klirrt leise der Spiegel zum Sound von Tocotronic. Jetzt heißt es aber erstmal auf was anderes konzentrieren. Tina möchte gerne den Zusammenhang zwischen durschnittlichen Niederschlag [ml/w] und Trockengewicht [kg/ha] im Kontext von Brokkoli herausfinden. Hierfür hat Tina ein Feldexperiment im Teuteburgerwald durchgeführt. Nach einigen unvorgesehenen Ereignissen hat sie es geschafft folgende Datentabelle zu erstellen. Tina und die Wut, eine unendliche Geschichte mit kniffeligen Wendungen. Aber das steht auch nicht im Zentrum. Nun stellt sich die Frage für sie, ob es überhaupt einen Zusammenhang zwischen den gemessenen Variablen gibt. Deshalb möchte Tina als erstes eine explorative Datenanalyse durchführen. Dann was anderes. Irgendwie komisch, wenn sie Indiana Jones anmacht, dann ist die Spinne eigentlich sofort vor dem Bildschirm und starrt hinein.

\begin{table}[!h]
\centering
\begin{tabular}{cc}
\toprule
Trockengewicht [kg/ha] & Durschnittlichen Niederschlag [ml/w]\\
\midrule
23.0 & 31.1\\
22.9 & 29.0\\
26.0 & 31.8\\
22.4 & 28.5\\
24.8 & 36.2\\
\addlinespace
26.7 & 34.2\\
24.0 & 33.0\\
27.9 & 33.7\\
24.1 & 30.3\\
27.4 & 32.9\\
\bottomrule
\end{tabular}
\end{table}



Leider kennt sich Tina mit der Erstellung einer explorativen Datenanalyse für kontinuierliche Daten überhaupt nicht aus. Deshalb braucht sie bei der Erstellung Ihre Hilfe!

\begin{enumerate}
\item Erstellen Sie eine Visualisierung für die Datentabelle. Beschriften Sie
  die Achsen entsprechend! \textbf{(4 Punkte)}
\item Schätzen Sie eine Gerade durch die Punkte! \textbf{(1 Punkt)}
\item Beschriften Sie die Gerade mit den gängigen statistischen Maßzahlen! Geben Sie die numerischen Zahlenwerte mit an! \textbf{(3 Punkte)}
\item Wenn \textit{ein} Effekt von $x$ auf $y$ vorhanden wäre, wie würde die Gerade verlaufen und welche Werte würden die statistischen Maßzahlen annehmen? \textbf{(2 Punkt)}
\end{enumerate} 
\clearpage
% -----------------------------------------------------------------------

\section{Aufgabe \hfill (10 Punkte)}

\textit{Geben Sie grundsätzlich Formeln und Rechenweg zur Lösung der Teilaufgaben mit an!} \\[1Ex]
 

 
%% --------------------------------------------------------------------
\begin{minipage}[t]{0.5\textwidth}
\includegraphics[width = 1.3cm]{/Users/kruppajo/work/GitHub/exam/avatare/Alex.png}
\end{minipage}
\begin{minipage}[t]{0.5\textwidth}
\hfill
\href{https://youtu.be/t_1KL77mfmg}{\includegraphics[width = 2cm]{img/youtube}}\\[1Ex]
\end{minipage}
\vspace{1ex}
%% --------------------------------------------------------------------



In ein Freilandversuch im Emsland hatte Alex sich zum einen die Behandlung KI-gesteuert [ja/nein] und zum anderen die Messung Trockengewicht über Zielwert [ja/nein] im Kontext von Erbsen angeschaut. Nun steht Alex vor dem Problem, dass er zwei kategoriale Variablen in seinem Projektbericht gemessen hat. Dazu kommt dann noch was anderes. Alex und die Gefälligkeit, eine unendliche Geschichte mit kniffeligen Wendungen. Da sein Betreuer erstmal die langen Tabellen mit ja/nein in einer explorativen Datenanalyse zusammengefasst und präsentiert bekommen möchte bevor es überhaupt weitergeht, muss er jetzt eine Lösung finden. Was alles auch nicht einfacher macht. Am liebsten würde er ja was anderes machen. Am Ende dann doch besser Starcraft. Wunderbar. Eine echte Ablenkung für Alex.



\vspace{1Ex}

\begin{center}
\begin{minipage}[t]{0.45\textwidth}
%\small
\begin{center}

\begin{tabular}{p{2.5cm}p{2.5cm}p{2.5cm}p{2.5cm}}
\toprule
KI-gesteuert & Trockengewicht über Zielwert\\
\midrule
nein & nein\\
nein & ja\\
nein & nein\\
nein & nein\\
nein & nein\\
\addlinespace
ja & nein\\
nein & nein\\
ja & nein\\
ja & nein\\
nein & ja\\
\addlinespace
ja & nein\\
ja & nein\\
ja & nein\\
ja & nein\\
ja & ja\\
\addlinespace
ja & nein\\
\bottomrule
\end{tabular}


\end{center}
\end{minipage}
\begin{minipage}[t]{0.45\textwidth}
%\small
\begin{center}

\begin{tabular}{p{2.5cm}p{2.5cm}p{2.5cm}p{2.5cm}}
\toprule
KI-gesteuert & Trockengewicht über Zielwert\\
\midrule
ja & nein\\
ja & nein\\
ja & nein\\
nein & ja\\
ja & ja\\
\addlinespace
nein & ja\\
ja & ja\\
nein & nein\\
ja & ja\\
nein & nein\\
\addlinespace
nein & ja\\
ja & nein\\
nein & nein\\
ja & nein\\
nein & nein\\
\addlinespace
ja & nein\\
\bottomrule
\end{tabular}


\end{center}
\end{minipage}
\end{center}

\vspace{2Ex}

Leider kennt sich Alex mit der Erstellung einer explorativen Datenanalyse für kategoriale Daten überhaupt nicht aus. Deshalb braucht er bei der Erstellung Ihre Hilfe!

\begin{enumerate}
\item Stellen Sie den Zusammenhang zwischen den beiden kategorialen Variablen in einer zusammenfassenden Tabelle dar! \textbf{(3 Punkte)}
\item Visualisieren Sie den Zusammenhang zwischen den beiden kategorialen Variablen! \textbf{(3 Punkte)}
\item Berechnen Sie die Verhältnisse in der Visualisierung! Welche Annahme haben Sie getroffen? \textbf{(2 Punkte)}
\item Wenn \textit{ein} Effekt von der Behandlung vorliegen würde, wie würde die Tabelle und die Visualisierung aussehen? \textbf{(2 Punkt)}
\end{enumerate} 
\clearpage
% -----------------------------------------------------------------------

\section{Aufgabe \hfill (10 Punkte)}

\textit{Geben Sie grundsätzlich Formeln und Rechenweg zur Lösung der Teilaufgaben mit an!} \\[1Ex]
 

 
%% --------------------------------------------------------------------
\begin{minipage}[t]{0.5\textwidth}
\includegraphics[width = 1.3cm]{/Users/kruppajo/work/GitHub/exam/avatare/Alex.png}\hspace{-4mm}\includegraphics[width = 1.3cm]{/Users/kruppajo/work/GitHub/exam/avatare/Tina.png}
\end{minipage}
\begin{minipage}[t]{0.5\textwidth}
\hfill
\href{https://youtu.be/Op-gjzASH9I}{\includegraphics[width = 2cm]{img/youtube}}
\end{minipage}
\vspace{1ex}
%% --------------------------------------------------------------------



'Was soll das denn jetzt schon wieder sein? Drei Boxplot, die auf der Seite liegen?', entfährt es Alex und schaut dabei Tina an. 'Keine Ahnung. Es ist bestimmt wieder so ein Lernziel mit der Verteilung und so.', meint Tina sichtlich genervt und mampft noch ein paar Katjes. 'Du weißt doch wie es heißt, \textit{Frei ist, wer missfallen kann.}\footnote{Oschmann, A. (2024) Mädchen stärken: Stärken fördern, Selbstwert erhöhen und liebevoll durch Krisen begleiten. Goldegg Verlag}', merkt Alex nickend an. Die beiden schauen angestrengt auf die drei Boxplots. Das Ziel ist es zu verstehen, wie eine Verteilung anhand eines Boxplots bewertet werden kann. Tina und die Gefälligkeit machen die Sache nicht einfacher.



{\centering \includegraphics[width=\maxwidth]{img/desc-stat-11-1} 

}




Jetzt brauchen Alex und Tina Ihre Hilfe bei der Abschätzung einer Verteilung anhand von Boxplots um ihre Arbeit dann in diesem Semester noch abschließen zu können.

\begin{enumerate}
\item Zeichnen Sie über die Boxplots die entsprechende zugehörige Verteilung! \textbf{(3 Punkte)} 
\item Zeichnen Sie unter die Boxplots die entsprechende zugehörige Beobachtungen als Stiche! \textbf{(3 Punkte)}
\item Wie viel Prozent der Beobachtungen fallen in das IQR? Ergänzen Sie die Abbildung entsprechend um den Bereich! \textbf{(2 Punkte)}
\item Wie viel Prozent der Beobachtungen fallen in $\bar{y} \pm 1s$ und $\bar{y} \pm 2s$  unter der Annahme einer Normalverteilung? \textbf{(2 Punkte)}
\end{enumerate} 
\clearpage
% -----------------------------------------------------------------------

\section{Aufgabe \hfill (10 Punkte)}

\textit{Geben Sie grundsätzlich Formeln und Rechenweg zur Lösung der Teilaufgaben mit an!} \\[1Ex]
 

 
%% --------------------------------------------------------------------
\begin{minipage}[t]{0.5\textwidth}
\includegraphics[width = 1.3cm]{/Users/kruppajo/work/GitHub/exam/avatare/Mark.png}\hspace{-4mm}\includegraphics[width = 1.3cm]{/Users/kruppajo/work/GitHub/exam/avatare/Steffen.png}
\end{minipage}
\begin{minipage}[t]{0.5\textwidth}
\hfill
\href{https://youtu.be/ZrJhn2wPbq4}{\includegraphics[width = 2cm]{img/youtube}}
\end{minipage}
\vspace{1ex}
%% --------------------------------------------------------------------



Mark und die Unsicherheit machen die Sache mit dem Studium nicht einfacher. Immerhin ist noch Steffen zur Hilfe mit dabei. Steffen hat Marzipankugeln mitgebracht und Andrea Berg aufgedreht. Das ist immerhin eine Ablenkung. Nicht so gut wie Geocaching, aber immerhin etwas. Jetzt sollen die beiden diese komische Aufgabe lösen. Es geht um verschiedene Normalverteilungen. Anscheinend hängen Normalverteilungen vom Mittelwert $\bar{y}$ und der Standardabweichung $s$ ab. 'Wozu brauchen wir nochmal Normalverteilungen?', entfährt es Mark. Durch das Mampfen von Steffen versteht er kein Wort der Antwort. Steffen lächelt.\\



Jetzt brauchen Mark und Steffen Ihre Hilfe bei der Abschätzung einer Verteilung um ihre Arbeit dann in diesem Semester noch abschließen zu können.

\begin{enumerate}
\item Skizzieren Sie zwei Normalverteilungen mit $\bar{y}_1 \neq \bar{y}_2$ und $s_1 = s_2$! \textbf{(3 Punkte)}
\item Beschriften Sie die Normalverteilungen mit den statistischen Maßzahlen! \textbf{(2 Punkte)}
\item Liegt Varianzhomogenität oder Varianzheterogenität vor? Begründen Sie Ihre Antwort! \textbf{(2 Punkte)}
\item In welchen Bereich fallen 68\% bzw. 95\% der Beobachtungen in einer Normalverteilung? Ergänzen Sie die Bereiche in \underline{einer} Normalverteilung! \textbf{(2 Punkte)}
\item Ergänzen Sie unter \underline{einer} der Normalverteilungen den entsprechenden Boxplot! \textbf{(1 Punkt)}
\end{enumerate}

 
\clearpage
% -----------------------------------------------------------------------

\section{Aufgabe \hfill (10 Punkte)}

\textit{Geben Sie grundsätzlich Formeln und Rechenweg zur Lösung der Teilaufgaben mit an!} \\[1Ex]
 

 
%% --------------------------------------------------------------------
\begin{minipage}[t]{0.5\textwidth}
\includegraphics[width = 1.3cm]{/Users/kruppajo/work/GitHub/exam/avatare/Jonas.png}\hspace{-4mm}\includegraphics[width = 1.3cm]{/Users/kruppajo/work/GitHub/exam/avatare/Tina.png}
\end{minipage}
\begin{minipage}[t]{0.5\textwidth}
\hfill
\href{https://youtu.be/MiD42k4l5Ag}{\includegraphics[width = 2cm]{img/youtube}}
\end{minipage}
\vspace{1ex}
%% --------------------------------------------------------------------



'Was sollen wir hier dann noch zeichnen?!', entfährt es Tina und schaut dabei Jonas an. 'Wir sollen eine Normalverteilung mit einem Mittelwert von $\bar{y}_1 = 2$ und einer Standardabweichung von $s_1 = 9$ zeichnen. Sowie eine weitere Normalverteilung mit einem Mittelwert von $\bar{y}_2 = 1$ und einer Standardabweichung von $s_2 = 9$. Keine Ahnung wie das geht. Darunter sollen dann noch eine Poissonverteilung mit einem Mittelwert von $\lambda_1 = 2$ sowie einer weiteren Poissonverteilung mit einem Mittelwert von $\lambda_2 = 25$ gezeichnet werden.', meint Jonas sichtlich genervt und mampft noch ein paar Snickers. Im Hintergrund spielt leise Iron Maiden. 'Wirre Geschichte...', merkt Tina nickend an. Die beiden schauen angestrengt auf die leeren Flächen für die Abbildungen. Jonas und die Wut machen die Suche nach der Lösung nicht einfacher.\\




{\centering \includegraphics[width=\maxwidth]{img/histogram-01-1} 

}




Jetzt brauchen Tina und Jonas Ihre Hilfe bei der Abschätzung einer Verteilung um ihre Arbeit dann in diesem Semester noch abschließen zu können.


\begin{enumerate}
\item Skizzieren Sie die zwei Normalverteilungen und zwei Poissonverteilungen! \textbf{(4 Punkte)}
\item Achten Sie auf die entsprechende Skalierung in den jeweiligen Abbildungen! \textbf{(2 Punkte)}
\item Ergänzen Sie unter \underline{einer} Normalverteilung den entsprechenden Boxplot! \textbf{(1 Punkt)}
\item Ergänzen Sie unter \underline{einer} Poissonverteilung den entsprechenden Boxplot! \textbf{(1 Punkt)}
\item Geben Sie ein Beispiel für ein Outcome $y$, welches einer Normalverteilung folgt! \textbf{(1 Punkt)}
\item Geben Sie ein Beispiel für ein Outcome $y$, welches einer Poissonverteilung folgt! \textbf{(1 Punkt)}
\end{enumerate} 
\clearpage
% -----------------------------------------------------------------------
\part{Statistisches Testen \& statistische Testtheorie}
% -----------------------------------------------------------------------  

\section{Aufgabe \hfill (9 Punkte)}

%% --------------------------------------------------------------------
\hfill\href{https://youtu.be/aHVYuFKTqZs}{\includegraphics[width =
  2cm]{img/youtube}}\\[1Ex]
%% --------------------------------------------------------------------

Grundlage des statistischen Testen ist das Verst{\"a}ndnis von der
Grundgesamtheit (eng. \textit{population} oder \textit{ground truth}) und
der experimentellen Stichprobe (eng. \textit{sample}). 

\begin{enumerate}
\item Nennen Sie das statistische Verfahren und zwei konkrete Beispiele zur
  Durchf{\"u}hrung um von einer Grundgesamtheit auf eine Stichprobe zu
  gelangen! \textbf{(3 Punkte)}
\item Erkl{\"a}ren Sie den Zusammenhang zwischen Stichprobe und Grundgesamtheit
  an einem Schaubild! Beschriften Sie das Schaubild entsprechend!
  \textit{Nutzen Sie hierf{\"u}r als Veranschaulichung die K{\"o}rpergr{\"o}{\ss}e von
    M{\"a}nnern oder Frauen aus den Gummib{\"a}rchendaten!}  \textbf{(3 Punkte)}
\item Erweitern Sie das Schaubild um die Entstehung von $Pr(D|H_0)$!
  \textit{Nutzen Sie hierf{\"u}r als Veranschaulichung zus{\"a}tzlich die
    Gruppierungsvariable "`Modul"' aus den Gummib{\"a}rchendaten!}  \textbf{(3
    Punkte)}
\end{enumerate} 
\clearpage
% -----------------------------------------------------------------------

\section{Aufgabe \hfill (9 Punkte)}

%% --------------------------------------------------------------------
\hfill\href{https://youtu.be/Ric8ne39DtI}{\includegraphics[width =
  2cm]{img/youtube}}\\[1Ex]
%% --------------------------------------------------------------------




F{\"u}r ein besseres Verst{\"a}ndnis der statistischen Testtheorie, auch
Null-Ritual genannt, kann eine Visualisierung als Kreuztabelle genutzt werden.  

\begin{enumerate}
\item Tragen Sie folgende statistische Fachbegriffe zur statistischen
  Testtheorie korrekt eine selbst erstellte Kreuztabelle ein! \textbf{(3
    Punkte)}
  \begin{center}
  \begin{tabular}{cccc}
  H$_0$ abgelehnt & Richtige Entscheidung & Testentscheidung & H$_0$ falsch \\
  \end{tabular}
  \end{center}
\item Erg{\"a}nzen Sie Ihre erstellte Kreuztabelle um vier weitere, passende
  Fachbegriffe zur statistischen Testtheorie! \textbf{(2 Punkte)}
\end{enumerate}

Die Entscheidungsfindung durch einen statistischen Test kann auch durch die
Analogie zu einem Feuermelder abgebildet werden. Dabei symbolisiert der
Feuermelder den statistischen Test und es soll getestet werden, ob ein Feuer
ausgebrochen ist.

\begin{enumerate}
  \setcounter{enumi}{2}    
\item In der Analogie des Feuermelders, wie lautet der $\alpha$-Fehler? \textbf{(1 Punkt)}
\item In der Analogie des Feuermelders, wie lautet der $\beta$-Fehler? \textbf{(1 Punkt)}
\item Wenn der Feuermelder einmal pro Tag messen w{\"u}rde, wie oft w{\"u}rde der
  Feuermelder mit einem $\alpha$ von 5\% in einem halben Jahr Alarm schlagen?
  Begr{\"u}nden Sie Ihre Antwort! \textbf{(2 Punkte)}
\end{enumerate}



 
\clearpage
% -----------------------------------------------------------------------

\section{Aufgabe \hfill (9 Punkte)}

\textit{Geben Sie grunds{\"a}tzlich Formeln und Rechenweg zur L{\"o}sung der
  Teilaufgaben mit an!} \\[1Ex]

%% --------------------------------------------------------------------
\hfill\href{https://youtu.be/32JjH1eyuTU}{\includegraphics[width =
  2cm]{img/youtube}}\\[1Ex]
%% --------------------------------------------------------------------



Abgebildet ist die t-Verteilung unter der Anahme der G{\"u}ltigkeit der
Nullhypothese. \textit{Beachten Sie, dass im Folgenden keine
  numerisch korrekte Darstellung verlangt wird! Es gilt Erkennbarkeit vor
  Genauigkeit!}

\begin{enumerate}
\item Erg{\"a}nzen Sie eine beschriftete $x$-Achse! \textbf{(1 Punkt)}
\item Erg{\"a}nzen Sie "`$\bar{y}_1 = \bar{y}_2$"'! \textbf{(1 Punkt)} 
\item Erg{\"a}nzen Sie "`$A = 0.95$"'! \textbf{(1 Punkt)}
\item Zeichnen Sie $T_{\alpha=5\%}$ in die Abbildung! \textbf{(1 Punkt)} 
\item Zeichnen Sie das Signifikanzniveau $\alpha$ in die Abbildung! Begr{\"u}nden
  Sie Ihre Antwort! \textbf{(2 Punkte)} 
\item Zeichnen Sie $-T_{D}$ in die Abbildung! \textbf{(1
    Punkt)}
\item Zeichnen Sie einen nicht signifikant p-Wert in die Abbildung! Begr{\"u}nden
  Sie Ihre Antwort! \textbf{(2 Punkte)}   
\end{enumerate}



{\centering \includegraphics[width=\maxwidth]{img/statistisches-testen-3-1} 

}


 
\clearpage
% -----------------------------------------------------------------------

\section{Aufgabe \hfill (10 Punkte)}

%% --------------------------------------------------------------------
\hfill\href{https://youtu.be/CN_O4fYPbhs}{\includegraphics[width =
  2cm]{img/youtube}}\\[1Ex]
%% --------------------------------------------------------------------



Sie rechnen einen t-Test f{\"u}r Gruppenvergleiche der Mittelwerte. Sie
sch{\"a}tzen den Unterschied zwischen dem mittleren Trockengewicht nach D{"u}ngergabe zu einer unbehandelten
Kontrolle.

\begin{enumerate}
\item Beschriften Sie die untenstehende Abbildung mit der
  Signifikanzschwelle! Begr{\"u}nden Sie Ihre Antwort! \textbf{(2 Punkte)}
\item Erg{\"a}nzen Sie eine \textit{in den Kontext passende} Relevanzschwelle!
  Begr{\"u}nden Sie Ihre Antwort! \textbf{(2 Punkte)} 
\item Skizieren Sie in die
  untenstehende Abbildung sechs einzelne Konfidenzintervalle (a-f) mit den
  jeweiligen Eigenschaften! \textbf{(6 Punkte)}
  \begin{itemize}
  \item[(a)] Ein signifikantes, relevantes 90\% Konfidenzintervall. 	
  \item[(b)] Ein signifikantes, nicht relevantes 95\% Konfidenzintervall 	
  \item[(c)] Ein nicht signifikantes, nicht relevantes 95\% Konfidenzintervall 	
  \item[(d)] Ein 95\% Konfidenzintervall mit niedriger Fallzahl $n$ in der Stichprobe als der Rest 95\% der Konfidenzintervalle 
  \item[(e)] Ein 95\% Konfidenzintervall mit h{"o}herer Fallzahl $n$ in der Stichprobe als der Rest der 95\% Konfidenzintervalle
  \item[(f)] Ein signifikantes, relevantes 95\% Konfidenzintervall
  \end{itemize}
\end{enumerate}

\begin{center}
  \includegraphics[height = 12cm]{/Users/kruppajo/work/GitHub/exam/question/img/statistisches-testen-04}
\end{center}


 
\clearpage
% -----------------------------------------------------------------------

\section{Aufgabe \hfill (10 Punkte)}

\textit{Geben Sie grunds{\"a}tzlich Formeln und Rechenweg zur L{\"o}sung der
  Teilaufgaben mit an!} \\[1Ex]

%% --------------------------------------------------------------------
\hfill\href{https://youtu.be/FgZmpnEWDag}{\includegraphics[width =
  2cm]{img/youtube}}\\[1Ex]
%% --------------------------------------------------------------------



Beim statistischen Testen gibt es einen Zusammenhang zwischen dem Effekt,
der Streuung sowie der Fallzahl. Gegeben sei die Formel f{\"u}r den Student
t-Test auf den die folgenden {\"U}berlegungen basieren sollen. Welche
Auswirkung hat die {\"A}nderungen der jeweiligen statistischen Ma{\ss}zahl des
Effekts $\Delta$, der Streuung $s$ und der Fallzahl $n$ auf die Teststistik
$T_{D}$, den p-Wert $Pr(D|H_0)$ sowie dem Konfidenzintervall
$KI_{1-\alpha}$?

\begin{enumerate}
\item Visualisieren Sie den Zusammenhang zwischen der Teststatiatik
  $T_{D}$ und dem p-Wert $Pr(D|H_0)$ f{\"u}r sich ver{\"a}ndernde $T_{D}$-Werte!
  \textit{Geben Sie daf{\"u}r ein numerisches Beispiel in dem Sie drei
    $T_{D}$-Werte und deren Einfluss auf den p-Wert vergleichen!}
  \textbf{(3 Punkte)}  
\item  F{\"u}llen Sie die untenstehende Tabelle aus in dem Sie die {\"A}nderung der
  statistischen Ma{\ss}zahlen auf die Teststatistik, den p-Wert sowie das
  Konfidenzintervall in \textit{einem} Wort oder Symbol beschreiben! \textbf{(4 Punkte)}
\begin{center}
  \large
  \begin{tabular}[c]{l|c|c|c|l|c|c|c}
    & $T_{D}$ & $Pr(D|H_0)$ & $KI_{1-\alpha}$ & & $T_{D}$ & $Pr(D|H_0)$ & $KI_{1-\alpha}$\strut\\ 
    \hline
    \textbf{$\Delta\; \uparrow$} & \hspace{1.8cm} & \hspace{1.8cm}  & \hspace{1.8cm} & \textbf{
                                                          $\Delta\; \downarrow$} &
                                                                          \hspace{1.8cm} & \hspace{1.8cm}  & \hspace{1.8cm}\strut\\
    \hline
        \textbf{$s\; \uparrow$} & \hspace{1.8cm} & \hspace{1.8cm}  & \hspace{1.8cm} & \textbf{
                                                          $s\; \downarrow$} &
                                                                          \hspace{1.8cm}
                                                & \hspace{1.8cm}  & \hspace{1.8cm}\strut\\
    \hline
        \textbf{$n\; \uparrow$} & \hspace{1.8cm} & \hspace{1.8cm}  & \hspace{1.8cm} & \textbf{
                                                          $n\; \downarrow$} &
                                                                          \hspace{1.8cm}
                                                & \hspace{1.8cm}  & \hspace{1.8cm}\strut\\
    \hline
  \end{tabular}
\end{center}
\item Visualisieren Sie ein 95\%-iges Konfidenzintervall im Vergleich
  zu einem 90\%-igen Konfidenzintervall! Begr{\"u}nden Sie Ihre Visualisierung anhand der Formel
  des Konfidenzintervalls des t-Tests mathematisch! \textbf{(3 Punkte)} 
\end{enumerate} 
\clearpage
% -----------------------------------------------------------------------
\part{Der Student t-Test, Welch t-Test \& gepaarter t-Test}
% -----------------------------------------------------------------------

\section{Aufgabe \hfill (9 Punkte)}

\textit{Geben Sie grundsätzlich Formeln und Rechenweg zur Lösung der Teilaufgaben mit an!} \\[1Ex]
 

 
%% --------------------------------------------------------------------
\begin{minipage}[t]{0.5\textwidth}
\includegraphics[width = 1.3cm]{/Users/kruppajo/work/GitHub/exam/avatare/Tina.png}
\end{minipage}
\begin{minipage}[t]{0.5\textwidth}
\hfill
\href{https://youtu.be/eejS2uG4o-M}{\includegraphics[width = 2cm]{img/youtube}}\\[1Ex]
\end{minipage}
\vspace{1ex}
%% --------------------------------------------------------------------



Der t-Test. Tina erschaudert. Eine echte Herausforderung für sie war schon immer die Wut gewesen. Ein leidiges Lied. Ein mächtiges Werkzeug ist der t-Test in den Händen desjenigen, der einen normalverteilten Endpunkt ($Y$) hat. Aber erstmal überhaupt den t-Test rechnen können. Wie sah das Experiment von Tina überhaupt aus? Aus den Boxen wummert Tocotronic und ihr Mund ist verklebt von Katjes. 'Herrlich', denkt Tina. Tina hat ein Gewächshausexperiment mit Spargel durchgeführt um eine neue technische Versuchsanlage zu testen. Bei dem Pilotexperiment mit sehr geringer Fallzahl $(n_1 = n_2 = 3)$ wurde die Behandlung Bewässerungstypen ($low$ und $high$) an den Spargel getestet und dabei wurde geschaut, ob der Versuch überhaupt technisch klappen könnte. Gemessen hat Tina dann als Messwert Trockengewicht [kg/ha]. Warum der Versuch im Teuteburgerwald für ihrer Hausarbeit stattfinden musste, ist ihr bis heute ein Rätsel. Egal. Gibt es jetzt einen Zusammenhang zwischen der Behandlung und Trockengewicht [kg/ha]?

\begin{table}[!h]
\centering
\begin{tabular}{cc}
\toprule
treatment & weight\\
\midrule
ctrl & 15.2\\
ctrl & 16.5\\
ctrl & 12.1\\
dose & 22.1\\
dose & 14.3\\
\addlinespace
dose & 11.8\\
\bottomrule
\end{tabular}
\end{table}



Leider kennt sich Tina mit der Berechnung eines t-Tests überhaupt nicht aus. Deshalb braucht sie bei der Berechnung Ihre Hilfe!

\begin{enumerate}
  \item Formulieren Sie das statistische Hypothesenpaar! \textbf{(1 Punkt)}
  \item Bestimmen Sie die Teststatistik $T_{D}$ eines Student t-Tests! \textbf{(3 Punkte)}
  \item Treffen Sie mit $T_{\alpha = 5\%} = 1.96$ eine Aussage zur Nullhypothese! Begründen Sie Ihre Antwort! \textbf{(2 Punkte)}
  \item Berechnen Sie den Effekt des Student t-Tests! \textbf{(1 Punkt)}
  \item Formulieren Sie eine Antwort an Tina über das Ergebnis Ihrer statistischen Analyse! \textbf{(2 Punkte)}
\end{enumerate} 
\clearpage
% -----------------------------------------------------------------------

\section{Aufgabe \hfill (12 Punkte)}

\textit{Geben Sie grundsätzlich Formeln und Rechenweg zur Lösung der Teilaufgaben mit an!} \\[1Ex]
 

 
%% --------------------------------------------------------------------
\begin{minipage}[t]{0.5\textwidth}
\includegraphics[width = 1.3cm]{/Users/kruppajo/work/GitHub/exam/avatare/Nilufar.png}
\end{minipage}
\begin{minipage}[t]{0.5\textwidth}
\hfill
\href{https://youtu.be/Cq_rF_z4xOk}{\includegraphics[width = 2cm]{img/youtube}}\\[1Ex]
\end{minipage}
\vspace{1ex}
%% --------------------------------------------------------------------



Der t-Test. Nilufar erschaudert. Nilufar und die Erwartung, eine unendliche Geschichte mit kniffeligen Wendungen. Ein mächtiges Werkzeug ist der t-Test in den Händen desjenigen, der einen normalverteilten Endpunkt ($Y$) hat. Aber erstmal überhaupt den t-Test rechnen können. Wie sah das Experiment von Nilufar überhaupt aus? 'Hm...', Takis Blue Heat und Deichkind. Das ist und bleibt die beste Kombination zum Nachdenken für Nilufar. Nilufar hat ein Gewächshausexperiment mit Spargel durchgeführt. Dabei wurde die Behandlung Substrattypen ($torf$ und $70p30n$) an den Spargel getestet. Gemessen hat Nilufar dann als Messwert Proteingehalt [g/kg]. Warum der Versuch im Emsland für ihren Projektbericht stattfinden musste, ist ihr bis heute ein Rätsel. Egal. Gibt es jetzt einen Zusammenhang zwischen der Behandlung und Proteingehalt [g/kg]?

\begin{table}[!h]
\centering
\begin{tabular}{cc}
\toprule
Substrattypen & Proteingehalt\\
\midrule
torf & 27.4\\
torf & 36.1\\
torf & 23.3\\
70p30n & 32.7\\
70p30n & 40.0\\
\addlinespace
torf & 33.8\\
torf & 36.2\\
torf & 31.5\\
torf & 26.4\\
70p30n & 26.1\\
\addlinespace
70p30n & 41.3\\
70p30n & 36.9\\
torf & 24.5\\
torf & 36.7\\
70p30n & 38.9\\
\addlinespace
70p30n & 36.8\\
70p30n & 30.2\\
torf & 31.9\\
70p30n & 34.1\\
torf & 20.4\\
\addlinespace
70p30n & 50.2\\
\bottomrule
\end{tabular}
\end{table}



Leider kennt sich Nilufar mit der Berechnung eines t-Tests überhaupt nicht aus. Deshalb braucht sie bei der Berechnung Ihre Hilfe!

\begin{enumerate}
  \item Formulieren Sie die wissenschaftliche Fragestellung! \textbf{(1 Punkt)}
  \item Formulieren Sie das statistische Hypothesenpaar! \textbf{(1 Punkt)}
  \item Bestimmen Sie die Teststatistik $T_{D}$ eines Student t-Tests! \textbf{(3 Punkte)}
\item Treffen Sie mit $T_{\alpha = 5\%} = 1.96$ eine Aussage zur Nullhypothese! Begründen Sie Ihre Antwort! \textbf{(2 Punkte)}
\item Berechnen Sie den Effekt des Student t-Tests! \textbf{(1 Punkt)}
\item Wenn Sie \textit{keinen} Unterschied zwischen den Behandlungsgruppen erwarten würden, wie groß wäre dann die Teststatistik $T_{D}$? Begründen Sie Ihre Antwort! \textbf{(2 Punkte)}
\item Formulieren Sie eine Antwort an Nilufar über das Ergebnis Ihrer statistischen Analyse! \textbf{(2 Punkte)}
\end{enumerate} 
\clearpage
% -----------------------------------------------------------------------

\section{Aufgabe \hfill (12 Punkte)}

\textit{Geben Sie grundsätzlich Formeln und Rechenweg zur Lösung der Teilaufgaben mit an!} \\[1Ex]
 

 
%% --------------------------------------------------------------------
\begin{minipage}[t]{0.5\textwidth}
\includegraphics[width = 1.3cm]{/Users/kruppajo/work/GitHub/exam/avatare/Mark.png}
\end{minipage}
\begin{minipage}[t]{0.5\textwidth}
\hfill
\href{https://youtu.be/TbSEOMCQYl4}{\includegraphics[width = 2cm]{img/youtube}}\\[1Ex]
\end{minipage}
\vspace{1ex}
%% --------------------------------------------------------------------



Der Teuteburgerwald, unendliche Weiten. Wir schreiben das Jahr 2024. Dies sind die Abenteuer von Mark, der mit seiner 1 Mann starken Besatzung 12 Wochen lang unterwegs ist, um neue Welten zu erforschen, neues Leben und neue Zivilisationen. 'Oder nennen wir es Ödnis und Verzweiflung', denkt Mark. Für seiner Hausarbeit ist Mark ins Nichts gezogen. Eine echte Herausforderung für ihn war schon immer die Unsicherheit gewesen. Ein leidiges Lied. Was macht er nun? Mark hat ein Feldexperiment mit Erbsen durchgeführt. Die Behandlung Lüftungssystemen und Folientunneln ($ctrl$ und $tornado$) wurde an Erbsen getestet. Gemessen hat er dann als ein normalverteiltes Outcome ($Y$) Proteingehalt [g/kg]. Jetzt soll er seinem Betreuer nach testen, ob die Behandlung Lüftungssystemen und Folientunneln ($ctrl$ und $tornado$) ein signifikantes Ergebnis liefert. Hm..., was entspannendes wäre gut. Mark schmeißt noch eine Handvoll Marzipankugeln in seinen Rachen. Im Hintergrund klirrt leise der Spiegel zum Sound von Andrea Berg.

\begin{table}[!h]
\centering
\begin{tabular}{cc}
\toprule
Lüftungssystemen & Proteingehalt\\
\midrule
ctrl & 34.8\\
tornado & 48.3\\
tornado & 29.5\\
tornado & 27.0\\
ctrl & 49.5\\
\addlinespace
tornado & 35.1\\
ctrl & 26.7\\
tornado & 43.0\\
tornado & 33.7\\
ctrl & 44.7\\
\addlinespace
ctrl & 34.2\\
tornado & 27.7\\
ctrl & 50.2\\
tornado & 21.2\\
ctrl & 40.0\\
\addlinespace
tornado & 53.1\\
\bottomrule
\end{tabular}
\end{table}



Leider kennt sich Mark mit der Berechnung eines t-Tests überhaupt nicht aus. Deshalb braucht er bei der Berechnung Ihre Hilfe!

\begin{enumerate}
  \item Formulieren Sie die wissenschaftliche Fragestellung! \textbf{(1 Punkt)}
  \item Formulieren Sie das statistische Hypothesenpaar! \textbf{(1 Punkt)}
  \item Bestimmen Sie die Teststatistik $T_{D}$ eines  Welch t-Tests! \textbf{(3 Punkte)}
  \item Treffen Sie mit $T_{\alpha = 5\%} = 1.96$ eine Aussage zur Nullhypothese! Begründen Sie Ihre Antwort! \textbf{(2 Punkte)}
\item Berechnen Sie das 99\% Konfidenzintervall. Welche Annahmen haben Sie getroffen? \textbf{(2 Punkte)}
\item Nennen Sie den statistischen Grund, warum Sie sich zwischen einem Student t-Test und einem Welch t-Test entscheiden müssen! \textbf{(1 Punkt)}
\item Formulieren Sie eine Antwort an Mark über das Ergebnis Ihrer statistischen Analyse! \textbf{(2 Punkte)}
\end{enumerate} 
\clearpage
% -----------------------------------------------------------------------

\section{Aufgabe \hfill (11 Punkte)}

\textit{Geben Sie grundsätzlich Formeln und Rechenweg zur Lösung der Teilaufgaben mit an!} \\[1Ex]
 

 
%% --------------------------------------------------------------------
\begin{minipage}[t]{0.5\textwidth}
\includegraphics[width = 1.3cm]{/Users/kruppajo/work/GitHub/exam/avatare/Nilufar.png}\hspace{-4mm}\includegraphics[width = 1.3cm]{/Users/kruppajo/work/GitHub/exam/avatare/Paula.png}
\end{minipage}
\begin{minipage}[t]{0.5\textwidth}
\hfill
\href{https://youtu.be/QR90zyn0Iz8}{\includegraphics[width = 2cm]{img/youtube}}
\end{minipage}
\vspace{1ex}
%% --------------------------------------------------------------------



Alles voll mit Puten. Das haben Paula und Nilufar gemeinsam in einer Abschlussarbeit gemacht! Worum ging es aber konkret? Beide haben als ein normalverteiltes Outcome ($Y$) Gewichtszuwachs in der 1LW von Puten bestimmt. Die Daten haben beide zusammen in einem Kreuzungsexperiment erhoben. In dem Experiment ging es um eine vorher/nachher Untersuchung an den gleichen Puten. Als Behandlung wurde Bestandsdichte ($hoch$ und $niedrig$) eingesetzt. Nach der Meinung des Betreuers muss hier ein gepaarter t-Test gerechnet werden. Paula schaut nachdenklich zu Nilufar. Eine echte Herausforderung für sie war schon immer der Perfektionismus gewesen. Ein leidiges Lied.. Steffen denkt derweil angestrengt an Takis Blue Heat. Im Hintergrund wummert White Lies.

\begin{table}[!h]
\centering
\begin{tabular}{ccc}
\toprule
ID & treatment & freshmatter\\
\midrule
4 & hoch & 32.4\\
8 & hoch & 29.1\\
6 & niedrig & 38.1\\
3 & hoch & 37.1\\
1 & hoch & 31.2\\
\addlinespace
7 & hoch & 25.2\\
5 & hoch & 28.4\\
2 & niedrig & 28.9\\
3 & niedrig & 20.6\\
4 & niedrig & 35.3\\
\addlinespace
5 & niedrig & 57.3\\
2 & hoch & 30.5\\
1 & niedrig & 15.3\\
7 & niedrig & 35.4\\
6 & hoch & 28.5\\
\bottomrule
\end{tabular}
\end{table}



Leider kennen sich Paula und Nilufar mit der Berechnung eines gepaarten t-Tests überhaupt nicht aus. Deshalb brauchen sie beide bei der Berechnung Ihre Hilfe!

\begin{enumerate}
  \item Formulieren Sie die wissenschaftliche Fragestellung! \textbf{(1 Punkt)}
  \item Formulieren Sie das statistische Hypothesenpaar! \textbf{(1 Punkt)}
  \item Bestimmen Sie die Teststatistik $T_{D}$ eines gepaarten t-Tests! \textbf{(3 Punkte)}
  \item Treffen Sie mit $T_{\alpha = 5\%} = 2.68$ eine Aussage zur Nullhypothese! Begründen Sie Ihre Antwort! \textbf{(2 Punkte)}
\item Schätzen Sie den $p$-Wert des gepaarten t-Tests ab! Begründen Sie Ihre Antwort mit einer Skizze! \textbf{(2 Punkte)}
\item Formulieren Sie eine Antwort an Paula über das Ergebnis Ihrer statistischen Analyse! \textbf{(2 Punkte)}
\end{enumerate}


 
\clearpage
% -----------------------------------------------------------------------

\section{Aufgabe \hfill (10 Punkte)}

\textit{Geben Sie grundsätzlich Formeln und Rechenweg zur Lösung der Teilaufgaben mit an!} \\[1Ex]
 

 
%% --------------------------------------------------------------------
\begin{minipage}[t]{0.5\textwidth}
\includegraphics[width = 1.3cm]{/Users/kruppajo/work/GitHub/exam/avatare/Mark.png}\hspace{-4mm}\includegraphics[width = 1.3cm]{/Users/kruppajo/work/GitHub/exam/avatare/Steffen.png}\hspace{-4mm}\includegraphics[width = 1.3cm]{/Users/kruppajo/work/GitHub/exam/avatare/Yuki.png}
\end{minipage}
\begin{minipage}[t]{0.5\textwidth}
\hfill
\href{https://youtu.be/exDo7AyHl4Q}{\includegraphics[width = 2cm]{img/youtube}}
\end{minipage}
\vspace{1ex}
%% --------------------------------------------------------------------



'Mit dem R Paket \texttt{\{emmeans\}} können wir gleich die Gruppenvergleiche rechnen und uns das \textit{compact letter displac}' ausgeben lassen!', verkündet Yuki sichtlich stolz. Ein paar Mal hat sie schon die Faulheit gehindert weiterzumachen. 'Nach Meinung der Betreuerin soll es aber nur erstmal ein t-Test sein. Und die Ausgabe ist schon wirr genug.', merkt Steffen an. Steffen und Mark sind bei Yuki um sich in \Rlogo helfen zu lassen. Im Hintergrund wummert London Grammar. Mark streichelt zur Beruhigung das Minischwein von Yuki. Die beiden waren 1 Monate im Emsland um einen Versuch mit Erbsen in einem Feldexperiment durchzuführen. Ziel war es das Outcome ($Y$) Trockengewicht [kg/ha] zu bestimmen. Yuki überlegt, ob er die beiden nicht noch auf den Film \textit{Matrix} einlädt oder dann doch lieber raus geht um zu Boldern? Vielleicht will ja Mark mit. Besser als der Film.

\begin{knitrout}
\definecolor{shadecolor}{rgb}{0.969, 0.969, 0.969}\color{fgcolor}\begin{kframe}
\begin{verbatim}
## 
## 	Two Sample t-test
## 
## data:  Trockengewicht by Lichtstufen
## t = 0.52969, df = 13, p-value = 0.6053
## alternative hypothesis: true  is not equal to [condensed]
## 95 percent confidence interval:
##  -3.974615  6.556758
## sample estimates:
##  mean in group none mean in group 600lm 
##            29.86250            28.57143
\end{verbatim}
\end{kframe}
\end{knitrout}

Helfen Sie Yuki bei der Interpretation des t-Tests! Sonst geht es auch für Steffen und Mark nicht weiter.
  
\begin{enumerate}
  \item Formulieren Sie die wissenschaftliche Fragestellung! \textbf{(1 Punkt)}
  \item Formulieren Sie das statistische Hypothesenpaar! \textbf{(1 Punkt)}
\item Liegt ein signifikanter Unterschied zwischen den Gruppen vor? Begründen Sie Ihre Antwort! \textbf{(2 Punkte)}
\item Skizzieren Sie eine Abbildung in der Sie $T_{D}$, $Pr(D|H_0)$, $A=0.95$, sowie $T_{\alpha=5\%} = |2.16|$ einzeichnen! \textbf{(4 Punkte)}
\item Beschriften Sie die Abbildung! \textbf{(1 Punkt)}  
\item Berechnen Sie den Effekt des t-Tests! \textbf{(1 Punkt)}
\end{enumerate} 
\clearpage
% -----------------------------------------------------------------------

\section{Aufgabe \hfill (10 Punkte)}

\textit{Geben Sie grundsätzlich Formeln und Rechenweg zur Lösung der Teilaufgaben mit an!} \\[1Ex]
 

 
%% --------------------------------------------------------------------
\begin{minipage}[t]{0.5\textwidth}
\includegraphics[width = 1.3cm]{/Users/kruppajo/work/GitHub/exam/avatare/Jessica.png}\hspace{-4mm}\includegraphics[width = 1.3cm]{/Users/kruppajo/work/GitHub/exam/avatare/Mark.png}\hspace{-4mm}\includegraphics[width = 1.3cm]{/Users/kruppajo/work/GitHub/exam/avatare/Nilufar.png}
\end{minipage}
\begin{minipage}[t]{0.5\textwidth}
\hfill
\href{https://youtu.be/wJqsNV1hOW8}{\includegraphics[width = 2cm]{img/youtube}}
\end{minipage}
\vspace{1ex}
%% --------------------------------------------------------------------



\Rlogo ist schon ein tolles Programm, wenn man mit dem Ding umgehen kann. Super umgehen kann damit Nilufar. Eine echte Herausforderung für Nilufar ist die Erwartung, aber das ist noch eine andere Sache. Deshalb sind aber Jessica und Mark nicht bei ihr. Sondern um sich bei einem gemeinsamen Projekt helfen zu lassen. Im Hintergrund wummert Deichkind. Beide arbeiten gemeinsam an einer Hausarbeit. In dem zu beschreibenden Versuch geht es im Teuteburgerwald um einem Kreuzungsexperiment mit Hühnern. Dabei ging darum herauszufinden, ob es einen Zusammenhang zwischen der Behandlung Lüftungssystem ($keins$ und $vorhanden$) und dem Messwert Schlachtgewicht [kg] gibt. Da der Messwert Schlachtgewicht [kg] normalverteilt ist kann ein t-Test gerechnet werden. Jessica möchte dann später noch mehr über Nilufars Hobby Hip Hop erfahren.

\begin{knitrout}
\definecolor{shadecolor}{rgb}{0.969, 0.969, 0.969}\color{fgcolor}\begin{kframe}
\begin{verbatim}
## 
## 	Two Sample t-test
## 
## data:  Schlachtgewicht by Lüftungssystem
## t = 5.3249, df = 17, p-value = 5.585e-05
## alternative hypothesis: true  is not equal to [condensed]
## 95 percent confidence interval:
##   5.703754 13.189580
## sample estimates:
##     mean in group keins mean in group vorhanden 
##                38.56667                29.12000
\end{verbatim}
\end{kframe}
\end{knitrout}

Helfen Sie Nilufar bei der Interpretation des t-Tests! Sonst geht es auch für Jessica und Mark nicht weiter.

\begin{enumerate}
  \item Formulieren Sie die wissenschaftliche Fragestellung! \textbf{(1 Punkt)}
  \item Formulieren Sie das statistische Hypothesenpaar! \textbf{(1 Punkt)}
\item Liegt ein signifikanter Unterschied zwischen den Gruppen vor? Begründen Sie Ihre Antwort! \textbf{(2 Punkte)}
\item Skizieren Sie das sich ergebende 95\% Konifidenzintervall! \textbf{(2 Punkte)}
\item Beschriften Sie die Abbildung und das 95\% Konfidenzintervall entsprechend! \textbf{(2 Punkte)}  
\item Interpretieren Sie den Effekt des 95\% Konifidenzintervalls! \textbf{(2 Punkte)}
\end{enumerate} 
\clearpage
% -----------------------------------------------------------------------

\section{Aufgabe \hfill (9 Punkte)}

\textit{Geben Sie grundsätzlich Formeln und Rechenweg zur Lösung der Teilaufgaben mit an!} \\[1Ex]
 

 
%% --------------------------------------------------------------------
\begin{minipage}[t]{0.5\textwidth}
\includegraphics[width = 1.3cm]{/Users/kruppajo/work/GitHub/exam/avatare/Jonas.png}\hspace{-4mm}\includegraphics[width = 1.3cm]{/Users/kruppajo/work/GitHub/exam/avatare/Mark.png}\hspace{-4mm}\includegraphics[width = 1.3cm]{/Users/kruppajo/work/GitHub/exam/avatare/Paula.png}
\end{minipage}
\begin{minipage}[t]{0.5\textwidth}
\hfill
\href{https://youtu.be/w62HJlbN28U}{\includegraphics[width = 2cm]{img/youtube}}
\end{minipage}
\vspace{1ex}
%% --------------------------------------------------------------------



'Mit dem R Paket \texttt{\{emmeans\}} können wir gleich die Gruppenvergleiche rechnen und uns das \textit{compact letter displac}' ausgeben lassen!', verkündet Paula sichtlich stolz. Ein paar Mal hat sie schon der Perfektionismus gehindert weiterzumachen. 'Nach Meinung des Betreuers soll es aber nur erstmal ein t-Test sein. Und die Ausgabe ist schon wirr genug.', merkt Jonas an. Jonas und Mark sind bei Paula um sich in \Rlogo helfen zu lassen. Im Hintergrund wummert White Lies. Mark streichelt zur Beruhigung die Ratte von Paula. Die beiden waren 2 Monate im Emsland um einen Versuch mit Puten in einem Kreuzungsexperiment durchzuführen. Ziel war es das Outcome ($Y$) Protein/Fettrate [\%/kg] zu bestimmen. Paula überlegt, ob sie die beiden nicht noch auf den Film \textit{Jagd auf roter Oktober} einlädt oder dann doch lieber raus geht um zu Fechten? Vielleicht will ja Mark mit. Besser als der Film.

\begin{knitrout}
\definecolor{shadecolor}{rgb}{0.969, 0.969, 0.969}\color{fgcolor}\begin{kframe}
\begin{verbatim}
## 
## 	Two Sample t-test
## 
## data:  Protein/Fettrate by Genotypen
## t = -3.3644, df = 19, p-value = 0.003256
## alternative hypothesis: true  is not equal to [condensed]
## 95 percent confidence interval:
##  -21.588796  -5.029386
## sample estimates:
## mean in group AA mean in group BB 
##         29.90000         43.20909
\end{verbatim}
\end{kframe}
\end{knitrout}

Helfen Sie Paula bei der Interpretation des t-Tests! Sonst geht es auch für Jonas und Mark nicht weiter.
  
\begin{enumerate}
  \item Formulieren Sie die wissenschaftliche Fragestellung! \textbf{(1 Punkt)}
  \item Formulieren Sie das statistische Hypothesenpaar! \textbf{(1 Punkt)}
\item Liegt ein signifikanter Unterschied zwischen den Gruppen vor?
  Begründen Sie Ihre Antwort! \textbf{(2 Punkte)}
\item Skizieren Sie die sich ergebenden Boxplot!
  Welche Annahmen an die Daten haben Sie getroffen? Begründen Sie Ihre
  Antwort! \textbf{(2 Punkte)} 
\item Skizieren Sie die sich ergebenden Barplots! \textbf{(2 Punkte)}
\item Berechnen Sie den Effekt des t-Tests! \textbf{(1 Punkt)}
\end{enumerate}
 
\clearpage
% -----------------------------------------------------------------------

\section{Aufgabe \hfill (10 Punkte)}

\textit{Geben Sie grundsätzlich Formeln und Rechenweg zur Lösung der Teilaufgaben mit an!} \\[1Ex]
 

 
%% --------------------------------------------------------------------
\begin{minipage}[t]{0.5\textwidth}
\includegraphics[width = 1.3cm]{/Users/kruppajo/work/GitHub/exam/avatare/Mark.png}\hspace{-4mm}\includegraphics[width = 1.3cm]{/Users/kruppajo/work/GitHub/exam/avatare/Yuki.png}
\end{minipage}
\begin{minipage}[t]{0.5\textwidth}
\hfill
\href{https://youtu.be/kHmfEmU6lrk}{\includegraphics[width = 2cm]{img/youtube}}
\end{minipage}
\vspace{1ex}
%% --------------------------------------------------------------------



Es gibt ja immer die Möglichkeit sich Hilfe zu holen. Das geht natürlich auch immer in einer Abschlussarbeit. Deshalb arbeiten Mark und Yuki gemeinsam an einer Abschlussarbeit. Das macht dann auch die Analyse ihres Hauptversuches einfacher. Zwar hat jeder von ihnen noch ein Subthema, aber auch da kann man sich ja helfen. Das hilft dann teilweise nur bedingt. Mark und die Unsicherheit, eine unendliche Geschichte mit kniffeligen Wendungen. In dem Hauptversuch wurde Folgendes von den beiden gemacht. Mark und Yuki haben sich Schweinen angeschaut. Dabei geht um Zusammenhang zwischen Genotypisierung ($0d$ und $14d$) und Schlachtgewicht [kg]. Jetzt sollen beide einen gepaarten t-Test rechnen. Leider kennen sich beide nicht sehr gut in \Rlogo aus. Aber wenigtens haben beide eine Menge an Marzipankugeln und in der Wohnung wummert Andrea Berg.

\begin{knitrout}
\definecolor{shadecolor}{rgb}{0.969, 0.969, 0.969}\color{fgcolor}\begin{kframe}
\begin{verbatim}
## 
## 	Paired t-test
## 
## data:  Schlachtgewicht by Genotypisierung
## t = -1.6689, df = 8, p-value = 0.1337
## alternative hypothesis: true  is not equal to [condensed]
## 95 percent confidence interval:
##  -10.373926   1.662815
## sample estimates:
## mean difference 
##       -4.355556
\end{verbatim}
\end{kframe}
\end{knitrout}

Jetzt brauchen Mark und Yuki Ihre Hilfe bei der Berechnung eines gepaarten t-Tests in \Rlogo um ihre Arbeit dann in diesem Semester noch abschließen zu können.

\begin{enumerate}
  \item Formulieren Sie die wissenschaftliche Fragestellung! \textbf{(1 Punkt)}
  \item Formulieren Sie das statistische Hypothesenpaar! \textbf{(1 Punkt)}
\item Liegt ein signifikanter Unterschied zwischen den Gruppen vor?
  Begründen Sie Ihre Antwort! \textbf{(2 Punkte)}
\item Skizzieren Sie das sich ergebende 95\% Konfidenzintervall! \textbf{(2 Punkte)}
\item Interpretieren Sie den Effekt des gepaarten t-Tests! \textbf{(2 Punkte)}
\item Skizzieren Sie den sich ergebenden Boxplot der Differenzen! Welche Annahmen an die Daten haben Sie getroffen? Begründen Sie Ihre Antwort! \textbf{(2 Punkte)} 
\end{enumerate}
 
\clearpage
% -----------------------------------------------------------------------
\part{Die einfaktorielle \& zweifaktorielle ANOVA}
% -----------------------------------------------------------------------

\section{Aufgabe \hfill (11 Punkte)}

\textit{Geben Sie grundsätzlich Formeln und Rechenweg zur Lösung der Teilaufgaben mit an!} \\[1Ex]
 

 
%% --------------------------------------------------------------------
\begin{minipage}[t]{0.5\textwidth}
\includegraphics[width = 1.3cm]{/Users/kruppajo/work/GitHub/exam/avatare/Paula.png}\hspace{-4mm}\includegraphics[width = 1.3cm]{/Users/kruppajo/work/GitHub/exam/avatare/Tina.png}
\end{minipage}
\begin{minipage}[t]{0.5\textwidth}
\hfill
\href{https://youtu.be/kHmfEmU6lrk}{\includegraphics[width = 2cm]{img/youtube}}
\end{minipage}
\vspace{1ex}
%% --------------------------------------------------------------------



'Uff... die einfaktorielle ANOVA. Und wir können jetzt anhand der Visualisuierung sehen, ob da schon was signifikant ist?', Tina hebt die Augenbraue. 'Ja, können wir. Dafür müssen wir aber erstmal in \texttt{\{ggplot\}} uns die Daten anschauen. Oder wir zeichnen es flott mit der Hand. Geht auch.', meint Paula dazu. Tina hatte sich in einen Versuch in einer Klimakammer verschiedene Erbsen angeschaut. Dabei ging es herauszufinden, ob es einen Zusammenhang zwischen der Behandlung Lichtstufen ($none$, $200lm$ und $600lm$) und dem Messwert Chlorophyllgehalt (SPAD-502Plus) [SPAD] gibt.

\begin{knitrout}
\definecolor{shadecolor}{rgb}{0.969, 0.969, 0.969}\color{fgcolor}\begin{table}[!h]
\centering
\begin{tabular}{cc}
\toprule
Lichtstufen & Chlorophyllgehalt\\
\midrule
200lm & 23\\
200lm & 24\\
none & 44\\
none & 43\\
none & 42\\
\addlinespace
none & 44\\
200lm & 23\\
600lm & 41\\
200lm & 26\\
600lm & 40\\
\addlinespace
600lm & 41\\
600lm & 40\\
none & 45\\
600lm & 37\\
600lm & 39\\
\addlinespace
200lm & 25\\
600lm & 40\\
200lm & 27\\
none & 46\\
none & 46\\
\bottomrule
\end{tabular}
\end{table}

\end{knitrout}

Leider kennen sich Tina und Paula mit Darstellung einer einfaktoriellen ANOVA überhaupt nicht aus. Deshalb brauchen beide bei der Erstellung Ihre Hilfe! 

\begin{enumerate}
\item Erstellen  Sie  eine  Visualisierung  der  Datentabelle! Beschriften  Sie  die  Abbildung! \textbf{(2 Punkte)}
\item Benennen Sie die Visualisierung mit dem korrekten, statistischen Fachbegriff! \textbf{(1 Punkt)}
\item Zeichnen Sie folgende statistischen Maßzahlen passend ein! 
  \begin{itemize}
  \item Globale Mittelwert: $\beta_0$ \textbf{(1 Punkt)}
  \item Mittelwerte der einzelnen Behandlungsstufen: $\bar{y}_{0.5}$, $\bar{y}_{1.5}$ und $\bar{y}_{2.5}$ \textbf{(1 Punkt)}
  \item Mittelwertsdifferenz der einzelnen Behandlungsstufen: $\beta_{0.5}$, $\beta_{1.5}$ und $\beta_{2.5}$ \textbf{(1 Punkt)}
  \item Residuen oder Fehler: $\epsilon$ \textbf{(1 Punkt)}
  \end{itemize}
\item Liegt ein \textit{vermutlicher} signifikanter Unterschied vor? Begründen Sie Ihre Antwort! \textbf{(2 Punkte)}
\item Schätzen Sie die Effekte der Behandlungsstufen! \textbf{(2 Punkte)}
\end{enumerate}
 
\clearpage
% -----------------------------------------------------------------------

\section{Aufgabe \hfill (9 Punkte)}

\textit{Geben Sie grundsätzlich Formeln und Rechenweg zur Lösung der Teilaufgaben mit an!} \\[1Ex]
 

 
%% --------------------------------------------------------------------
\begin{minipage}[t]{0.5\textwidth}
\includegraphics[width = 1.3cm]{/Users/kruppajo/work/GitHub/exam/avatare/Alex.png}\hspace{-4mm}\includegraphics[width = 1.3cm]{/Users/kruppajo/work/GitHub/exam/avatare/Jonas.png}
\end{minipage}
\begin{minipage}[t]{0.5\textwidth}
\hfill
\href{https://youtu.be/IhecxMcCENY}{\includegraphics[width = 2cm]{img/youtube}}
\end{minipage}
\vspace{1ex}
%% --------------------------------------------------------------------



'Als erstes bauen wir uns aus unsere Daten die ANOVA Tabelle dann sehen wir schon, ob unser Gruppenvergleich in der ANOVA signifikant ist.', Jonas schaut Alex fragend an und hofft auf eine positive Regung im Gesicht. Wird aber enttäuscht. Alex tut sich auch sehr schwer mit der einfaktoriellen ANOVA. Beide waren im Wendland um ein Kreuzungsexperiment mit Schweinen durchzuführen. Dabei ging es herauszufinden, ob es einen Zusammenhang zwischen der Behandlung Flüssignahrung ($ctrl$, $superIn$ und $flOw$) und dem Messwert Schlachtgewicht [kg] gibt.



\vspace{1ex}

Leider kennen sich Jonas und Alex mit Berechnung einer einfaktoriellen ANOVA überhaupt nicht aus. Deshalb brauchen beide bei der Erstellung Ihre Hilfe! 

\begin{enumerate}
  \item Formulieren Sie die wissenschaftliche Fragestellung! \textbf{(1 Punkt)}
  \item Formulieren Sie das statistische Hypothesenpaar! \textbf{(1 Punkt)}
\item Füllen Sie die unterstehende einfaktorielle ANOVA Ergebnistabelle aus! \textbf{(3 Punkte)}
\end{enumerate}

\vspace{1Ex}

\begin{center}
  \Large
  \begin{tabular}{lccccp{3cm}}
\toprule
     & \textbf{Df} & \textbf{Sum Sq} & \textbf{Mean Sq} & \textbf{F value} & \textbf{Pr(>F)} \strut\\
    \midrule
   \textbf{Flüssignahrung}  & 2 &  &  &  &  \strut\\
   \textbf{error}  & 19 & 346.3 &  &  &  \strut\\
   \textbf{Total}  & 21 & 359.82 &  &  &  \strut\\
\bottomrule
  \end{tabular}
\end{center}

\vspace{1Ex}

\begin{enumerate}
  \setcounter{enumi}{3}
\item Schätzen Sie den p-Wert der Tabelle mit $F_{\alpha = 5\%} = 3.52$ ab. Begründen Sie Ihre Antwort! \textbf{(2 Punkte)}
\item Berechen Sie den Effektschätzer $\eta^2$. Was sagt Ihnen der Wert von $\eta^2$ aus? \textbf{(2 Punkte)}
\end{enumerate}



 
\clearpage
% -----------------------------------------------------------------------

\section{Aufgabe \hfill (12 Punkte)}

\textit{Geben Sie grundsätzlich Formeln und Rechenweg zur Lösung der Teilaufgaben mit an!} \\[1Ex]
 

 
%% --------------------------------------------------------------------
\begin{minipage}[t]{0.5\textwidth}
\includegraphics[width = 1.3cm]{/Users/kruppajo/work/GitHub/exam/avatare/Alex.png}\hspace{-4mm}\includegraphics[width = 1.3cm]{/Users/kruppajo/work/GitHub/exam/avatare/Yuki.png}
\end{minipage}
\begin{minipage}[t]{0.5\textwidth}
\hfill
\href{https://youtu.be/49hvImMwVyE}{\includegraphics[width = 2cm]{img/youtube}}
\end{minipage}
\vspace{1ex}
%% --------------------------------------------------------------------



Yuki und Alex schauen sich etwas entnervt an. Gemeinsam schreiben die beiden ihre Abschlussarbeit und sollen nun als erstes einmal die Daten mit eine einfaktoriellen ANOVA auswerten damit abgeschätzt werden kann, ob überhaupt signifikante Ergebnisse in den multipen Gruppenvergleichen zu erwarten sind. Nun möchte erstmal ihre Betreuung der Arbeit eine ANOVA Tabelle sehen. Was immer da auch drin zu erkennen sein mag. Yuki schaut Alex sehen erstmla gar nichts. Die beiden waren im Wendland um einen Leistungssteigerungsversuch mit Hühnern durchzuführen. Dabei haben Yuki und Alex den Messwert Fettgehalt [\%/kg] unter der Behandung Elterlinie ($ctrl$, $Standard$, $Yray$ und $Xray$) ermittelt.



\vspace{1ex}

Leider kennen sich Yuki und Alex mit Berechnung einer einfaktoriellen ANOVA überhaupt nicht aus. Deshalb brauchen beide bei der Erstellung Ihre Hilfe! 

\begin{enumerate}
  \item Formulieren Sie die wissenschaftliche Fragestellung! \textbf{(1 Punkt)}
  \item Formulieren Sie das statistische Hypothesenpaar! \textbf{(1 Punkt)}
\item Füllen Sie die unterstehende einfaktorielle ANOVA Ergebnistabelle aus! \textbf{(3 Punkte)}
\end{enumerate}

\vspace{1Ex}

\begin{center}
  \Large
  \begin{tabular}{lccccp{3cm}}
\toprule
     & \textbf{Df} & \textbf{Sum Sq} & \textbf{Mean Sq} & \textbf{F value} & \textbf{Pr(>F)} \strut\\
    \midrule
   \textbf{Elterlinie}  & 3 & 489.92 &  &  &  \strut\\
   \textbf{Error}  & 19 & 314.08 &  &  &  \strut\\
\bottomrule
  \end{tabular}
\end{center}

\vspace{1Ex}

\begin{enumerate}
  \setcounter{enumi}{3}
\item Schätzen Sie den p-Wert der Tabelle mit $F_{\alpha = 5\%} = 3.13$ ab. Begründen Sie Ihre Antwort! \textbf{(2 Punkte)}
\item Was bedeutet ein signifikantes Ergebnis in einer einfaktoriellen ANOVA? \textbf{(1 Punkt)}
\item Berechnen Sie \textit{einen} Student t-Test für den \textit{vermutlich} signifikantesten Gruppenvergleich anhand der untenstehenden Tabelle mit $T_{\alpha = 5\%} = 2.03$. Begründen Sie Ihre Auswahl! \textbf{(3 Punkte)}
\end{enumerate}


\begin{knitrout}
\definecolor{shadecolor}{rgb}{0.969, 0.969, 0.969}\color{fgcolor}\begin{table}[!h]
\centering\begingroup\fontsize{11}{13}\selectfont

\begin{tabular}{cccc}
\toprule
\textbf{Elterlinie} & \textbf{Fallzahl (n)} & \textbf{Mittelwert} & \textbf{Standardabweichung}\\
\midrule
ctrl & 5 & 4.60 & 2.97\\
Standard & 8 & 12.12 & 5.22\\
Yray & 5 & 11.60 & 3.13\\
Xray & 5 & 1.20 & 3.49\\
\bottomrule
\end{tabular}
\endgroup{}
\end{table}

\end{knitrout}


\begin{enumerate}
  \setcounter{enumi}{6}
\item Gegebenen der ANOVA Tabelle war das Ergebnis des Student t-Tests zu erwarten? Begründen Sie Ihre Antwort! \textbf{(2 Punkte)}
\end{enumerate}

 
\clearpage
% -----------------------------------------------------------------------

\section{Aufgabe \hfill (9 Punkte)}

\textit{Geben Sie grundsätzlich Formeln und Rechenweg zur Lösung der Teilaufgaben mit an!} \\[1Ex]
 

 
%% --------------------------------------------------------------------
\begin{minipage}[t]{0.5\textwidth}
\includegraphics[width = 1.3cm]{/Users/kruppajo/work/GitHub/exam/avatare/Mark.png}
\end{minipage}
\begin{minipage}[t]{0.5\textwidth}
\hfill
\href{https://youtu.be/aXvxGC4YLqk}{\includegraphics[width = 2cm]{img/youtube}}\\[1Ex]
\end{minipage}
\vspace{1ex}
%% --------------------------------------------------------------------



'Uff... die einfaktorielle ANOVA und \Rlogo. Nicht so einfach... Was sagt mir jetzt die Ausgabe der ANOVA und wo sehe ich, ob da was signifikant ist?', denkt Mark und hebt die Augenbraue. Mark hatte sich einen Leistungssteigerungsversuch mit Milchvieh angeschaut. Als wäre das nicht alles schon schwer genug. Mark und die Unsicherheit, eine unendliche Geschichte mit kniffeligen Wendungen. Dabei ging es beim Experiment herauszufinden, ob es einen Zusammenhang zwischen der Behandlung Flüssignahrung ($ctrl$, $superIn$ und $flOw$) und dem Messwert Protein/Fettrate [\%/kg] gibt. Nun möchte sein Betreuer seinem Projektbericht erstmal eine ANOVA sehen und die Ergebnisse präsentiert bekommen. Und eigentlich will er ja was anderes... Auf seinem Second Screen läuft Columbo und Mark schaufelt Marzipankugeln. Nicht effizient, aber gut.

\begin{knitrout}
\definecolor{shadecolor}{rgb}{0.969, 0.969, 0.969}\color{fgcolor}\begin{kframe}
\begin{verbatim}
## Analysis of Variance Table
## 
## Response: Protein/Fettrate
##                Df Sum Sq Mean Sq F value Pr(>F)
## Flüssignahrung  2  38.14  19.071  0.9368 0.4102
## Residuals      18 366.43  20.357
\end{verbatim}
\end{kframe}
\end{knitrout}

\vspace{1ex}

Leider kennen sich Mark mit Berechnung einer einfaktoriellen ANOVA überhaupt nicht aus. Deshalb braucht er bei der Erstellung Ihre Hilfe! 

\begin{enumerate}
  \item Formulieren Sie die wissenschaftliche Fragestellung! \textbf{(1 Punkt)}
  \item Formulieren Sie das statistische Hypothesenpaar! \textbf{(1 Punkt)}
\item Interpretieren Sie das Ergebnis der einfaktoriellen ANOVA! \textbf{(2 Punkte)} 
\item Berechnen Sie den Effektschätzer $\eta^2$. Was sagt Ihnen der Wert von $\eta^2$ aus? \textbf{(2 Punkte)}
\item Skizzieren Sie eine Abbildung, der dem obigen Ergebnis der
  einfaktoriellen ANOVA näherungsweise entspricht! \textbf{(3 Punkte)}
\end{enumerate}

 
\clearpage
% -----------------------------------------------------------------------

\section{Aufgabe \hfill (12 Punkte)}

\textit{Geben Sie grundsätzlich Formeln und Rechenweg zur Lösung der Teilaufgaben mit an!} \\[1Ex]
 

 
%% --------------------------------------------------------------------
\begin{minipage}[t]{0.5\textwidth}
\includegraphics[width = 1.3cm]{/Users/kruppajo/work/GitHub/exam/avatare/Jonas.png}
\end{minipage}
\begin{minipage}[t]{0.5\textwidth}
\hfill
\href{https://youtu.be/8Pb2sKUIMyk}{\includegraphics[width = 2cm]{img/youtube}}\\[1Ex]
\end{minipage}
\vspace{1ex}
%% --------------------------------------------------------------------



Wie absolut ärgerlich. Jetzt stellt sich tatsächlich heraus, dass seiner Betreuerin keine Ahnung von der zweifaktoriellen ANOVA hat. Woher soll Jonas jetzt das Wissen nehmen? Jonas mampft aus Frust noch eine Handvoll Snickers. Immerhin muss er ja noch mit seiner Abschlussarbeit dieses Jahr fertig werden. In ein Kreuzungsexperiment hatte er Lamas mit der Behandlung Flüssignahrung ($ctrl$, $superIn$ und $flOw$) sowie der Behandlung Genotypen ($AA$ und $BB$) im Wendland untersucht. Es wurde als Messwert Gewichtszuwachs in der 1LW bestimmt. Jetzt muss er erstmal die zweifaktorielle ANOVA verstehen. Und eigentlich wollte Jonas doch noch zum Sport! Um zu Schwimmen geht Jonas dann später nochmal raus. Echte Entspannung.



\vspace{1ex}

Leider kennen sich Jonas mit Berechnung einer zweifaktoriellen ANOVA überhaupt nicht aus. Deshalb braucht er bei der Erstellung Ihre Hilfe! 

\begin{enumerate}
  \item Formulieren Sie die wissenschaftliche Fragestellung! \textbf{(1 Punkt)}
  \item Formulieren Sie das statistische Hypothesenpaar! \textbf{(1 Punkt)}
\item Füllen Sie die unterstehende einfaktorielle ANOVA Ergebnistabelle aus! \textbf{(3 Punkte)}
\end{enumerate}

\vspace{1Ex}

\begin{center}
  \Large
  \begin{tabular}{lccccc}
  \toprule
     & \textbf{Df} & \textbf{Sum Sq} & \textbf{Mean Sq} & \textbf{F value} & \textbf{Pr(>F)} \strut\\
    \midrule
   \textbf{Flüssignahrung}  & 3 & 190.47 &  &  &  \strut\\
    \textbf{Genotypen}  & 1 & 52.36 &  &  &  \strut\\
    \textbf{Flüssignahrung:Genotypen}  & 3 & 239.55 &  &  &  \strut\\
   \textbf{Error}  & 18 & 225.98 &  &  &  \strut\\
\bottomrule
  \end{tabular}
\end{center}

\vspace{1Ex}

\begin{enumerate}
  \setcounter{enumi}{3}
\item Schätzen Sie den p-Wert der Tabelle ab. Begründen Sie Ihre
  Antwort! \textbf{(3 Punkte)}
\end{enumerate}
  
\begin{center}
    \Large
\begin{tabular}{lc}
  \toprule
     & $\boldsymbol{F_{\alpha = 5\%}}$ \\
\midrule
  \textbf{Flüssignahrung} & $4.26$ \\
  \textbf{Genotypen} & $3.40$ \\
  \textbf{Flüssignahrung:Genotypen} & $5.23$ \\
  \bottomrule
  \end{tabular}
\end{center}

\begin{enumerate}
  \setcounter{enumi}{4}
\item Was bedeutet ein signifikantes Ergebnis in einer zweifaktoriellen ANOVA? \textbf{(2 Punkte)}
\item Was sagt der Term \textit{Flüssignahrung:Genotypen} aus? Interpretieren Sie das Ergebnis! \textbf{(2 Punkte)}
\end{enumerate}
 
\clearpage
% -----------------------------------------------------------------------

\section{Aufgabe \hfill (10 Punkte)}

\textit{Geben Sie grundsätzlich Formeln und Rechenweg zur Lösung der Teilaufgaben mit an!} \\[1Ex]
 

 
%% --------------------------------------------------------------------
\begin{minipage}[t]{0.5\textwidth}
\includegraphics[width = 1.3cm]{/Users/kruppajo/work/GitHub/exam/avatare/Tina.png}
\end{minipage}
\begin{minipage}[t]{0.5\textwidth}
\hfill
\href{https://youtu.be/rWTyHXXlYjY}{\includegraphics[width = 2cm]{img/youtube}}\\[1Ex]
\end{minipage}
\vspace{1ex}
%% --------------------------------------------------------------------



'Mit der zweifaktoriellen ANOVA lässt sich die Interaktion zwischen den beiden Behandlungen nachweisen!', ihre Betreuerin scheint die zweifaktoriellen ANOVA zu verstehen. Warum jetzt sie jetzt nochmal alles wiederkäuen muss, wird Tina echt nicht so klar. Wenn es doch so klar ist? Tina war im Wendland und hatte dort ein Feldexperiment mit Kartoffeln durchgeführt. Die Komune wo sie untergekommen war, war cool gewesen. Nur jetzt muss eben das Experiment fertig ausgewertet werden. Tina hatte zwei Behandlungen auf Kartoffeln angewendet. Einmal Lichtstufen ($none$, $200lm$, $400lm$ und $600lm$) sowie als zweite Behandlung Düngestufen ($ctrl$, und $high$). Gemessen wurde der Messwert ($Y$) Frischegewicht [kg/ha]. Jetzt muss das hier zu einem Ende kommen! Eigentlich wollte Tina nachher noch einen Film schauen. Irgendwie komisch, wenn sie Indiana Jones anmacht, dann ist die Spinne eigentlich sofort vor dem Bildschirm und starrt hinein.

\begin{knitrout}
\definecolor{shadecolor}{rgb}{0.969, 0.969, 0.969}\color{fgcolor}\begin{kframe}
\begin{verbatim}
## Analysis of Variance Table
## 
## Response: Frischegewicht
##                         Df Sum Sq Mean Sq F value    Pr(>F)
## Lichtstufen              2 270.05 135.024  9.9497 0.0012296
## Düngestufen              1  10.88  10.879  0.8016 0.3824249
## Lichtstufen:Düngestufen  2 428.54 214.268 15.7892 0.0001096
## Residuals               18 244.27  13.571
\end{verbatim}
\end{kframe}
\end{knitrout}

\vspace{1ex}

Leider kennt sich Tina mit Berechnung einer zweifaktoriellen ANOVA überhaupt nicht aus. Deshalb braucht sie bei der Erstellung Ihre Hilfe! 

\begin{enumerate}
  \item Formulieren Sie die wissenschaftliche Fragestellung! \textbf{(1 Punkt)}
  \item Formulieren Sie das statistische Hypothesenpaar! \textbf{(1 Punkt)}
\item Interpretieren Sie das Ergebnis der einfaktoriellen ANOVA! \textbf{(3 Punkte)} 
\item Zeichnen Sie eine Abbildung, der dem obigen Ergebnis der
  zweifaktoriellen ANOVA näherungsweise entspricht! \textbf{(5 Punkte)}
\end{enumerate}
 
\clearpage
% -----------------------------------------------------------------------

\section{Aufgabe \hfill (12 Punkte)}

\textit{Geben Sie grundsätzlich Formeln und Rechenweg zur Lösung der Teilaufgaben mit an!} \\[1Ex]
 

 
%% --------------------------------------------------------------------
\begin{minipage}[t]{0.5\textwidth}
\includegraphics[width = 1.3cm]{/Users/kruppajo/work/GitHub/exam/avatare/Alex.png}
\end{minipage}
\begin{minipage}[t]{0.5\textwidth}
\hfill
\href{https://youtu.be/FjjJXkFJfIY}{\includegraphics[width = 2cm]{img/youtube}}\\[1Ex]
\end{minipage}
\vspace{1ex}
%% --------------------------------------------------------------------



Die Katze dreht durch und verwüstet Alexs Palme zu kleinen Schnetzeln. Aber dafür hat er jetzt keine Zeit. Alex muss verstehen wie die Formeln der ANOVA und des t-Tests miteinander zusammen hängen und was das verbindene Konzept ist. Alex dreht Abba lauter, damit die Katze sie nicht mehr stört. Die Palme leidet still. Was hat Alex eigentlich gemacht? In ein Stallexperiment wurden Fleischrindern mit der Behandlung Lüftungssystem ($keins$, $storm$, $tornado$ und $thunder$) sowie der Behandlung Flüssignahrung ($ctrl$ und $flOw$) untersucht. Das hilft der Palme auch nicht mehr. Aber das ist nicht das einzige Problem von Alex. Eine echte Herausforderung für ihn war schon immer die Gefälligkeit gewesen. Ein leidiges Lied.

\begin{graybox}{Gegebene Formeln}
\begin{center}
  \begin{tabular}{cc}
    $F_{D} = \cfrac{MS_{treatment}}{MS_{error}}$ & $T_{D} = \cfrac{\bar{y}_1 - \bar{y}_2}{s_p \cdot \sqrt{2/n_g}}$\\
  \end{tabular}
\end{center}
\end{graybox}

Leider kennen sich Alex mit dem Zusammenhang zwischen der ANOVA und dem t-Test nicht aus. Deshalb braucht er bei der Erstellung Ihre Hilfe! 

\begin{enumerate}
\item Welche statistische Maßzahl testet der t-Test, welche die ANOVA? Begründen Sie Ihre Antwort! \textbf{(2 Punkte)}
\item Erklären Sie den Zusammenhang zwischen der $F_{D}$ Statistik und $T_{D}$ Statistik! \textbf{(2 Punkte)}
\item Visualisieren Sie in einer 2x2 Tafel den Zusammenhang von $MS_{treatment}$ und $MS_{error}$! \textbf{(2 Punkte)}
\item Beschriften Sie die erstellte 2x2 Tafel mit \underline{signifikant} und \underline{nicht signifikant}! Begründen Sie Ihre Antwort! \textbf{(2 Punkte)}
\item Nennen Sie das numerische Minimum der F-Statistik $F_D$! Begründen Sie Ihre Antwort! \textbf{(2 Punkte)}
\item Wenn die F-Statistik $F_D$ minimal ist, welche Aussage erhalten Sie über die Nullhypothese? Begründen Sie Ihre Antwort! \textbf{(2 Punkte)}
\end{enumerate}

 
\clearpage
% -----------------------------------------------------------------------

\section{Aufgabe \hfill (11 Punkte)}

\textit{Geben Sie grundsätzlich Formeln und Rechenweg zur Lösung der Teilaufgaben mit an!} \\[1Ex]
 

 
%% --------------------------------------------------------------------
\begin{minipage}[t]{0.5\textwidth}
\includegraphics[width = 1.3cm]{/Users/kruppajo/work/GitHub/exam/avatare/Mark.png}
\end{minipage}
\begin{minipage}[t]{0.5\textwidth}
\hfill
\href{https://youtu.be/2qG1Dws0MJo}{\includegraphics[width = 2cm]{img/youtube}}\\[1Ex]
\end{minipage}
\vspace{1ex}
%% --------------------------------------------------------------------



'Mit der zweifaktoriellen ANOVA lässt sich die Interaktion zwischen den beiden Behandlungen nachweisen!', seine Betreuerin scheint die zweifaktoriellen ANOVA zu verstehen. Warum jetzt er jetzt nochmal alles wiederkäuen muss, wird Mark echt nicht so klar. Wenn es doch so klar ist? Mark war im Wendland und hatte dort ein Freilandversuch mit Erdbeeren durchgeführt. Die Komune wo er untergekommen war, war cool gewesen. Nur jetzt muss eben das Experiment fertig ausgewertet werden. Es liegt anscheinend eine signifikante Interaktion vor? Mark hatte zwei Behandlungen auf Erdbeeren angewendet. Einmal Lichtstufen ($none$, $200lm$, $400lm$ und $600lm$) sowie als zweite Behandlung Bewässerungstypen ($ctrl$, und $high$). Gemessen wurde der Messwert ($Y$) Proteingehalt [g/kg]. Jetzt muss das hier zu einem Ende kommen! Eigentlich wollte Mark nachher noch einen Film schauen. Irgendwie komisch, wenn er Columbo anmacht, dann ist der Hamster eigentlich sofort vor dem Bildschirm und starrt hinein.

\vspace{1ex}

Leider kennen sich Mark und seine Betreuerin mit der zweifaktoriellen ANOVA überhaupt nicht aus. Geschweige denn mit der Interpretation einer Interaktion. Deshalb braucht er bei der Erstellung Ihre Hilfe, sonst wird es heute Abend mit seinem Hobby Geocaching nichts mehr! 

\begin{enumerate}
\item Visualisieren Sie folgende mögliche Interaktionen zwischen den Behandlungen! Beschriften Sie die Abbildung! \textbf{(4 Punkte)}
\begin{enumerate}
\item \underline{Keine} Interaktion liegt vor.
\item Eine \underline{schwache} Interaktion liegt vor. 
\item Eine \underline{starke} Interaktion liegt vor. 
\end{enumerate}
\item Erklären Sie den Unterschied zwischen den verschiedenen Interaktionen! \textbf{(2 Punkte)}
\item Welche statistische Maßzahl betrachten Sie für die Bewertung der Interaktion? \textbf{(1 Punkt)}
\item Skizzieren Sie die notwendigen Funktionen in \Rlogo für eine Post-hoc Analyse! \textbf{(2 Punkte)} 
\item Wenn eine signifikante Interaktion in den Daten vorliegt, wie ist dann das weitere Vorgehen? Berücksichtigen Sie auch die Funktion \texttt{emmeans()}! \textbf{(2 Punkte)}
\end{enumerate}

 
\clearpage
% -----------------------------------------------------------------------

\section{Aufgabe \hfill (11 Punkte)}

\textit{Geben Sie grundsätzlich Formeln und Rechenweg zur Lösung der Teilaufgaben mit an!} \\[1Ex]
 

 
%% --------------------------------------------------------------------
\begin{minipage}[t]{0.5\textwidth}
\includegraphics[width = 1.3cm]{/Users/kruppajo/work/GitHub/exam/avatare/Steffen.png}
\end{minipage}
\begin{minipage}[t]{0.5\textwidth}
\hfill
\href{https://youtu.be/M9Uhm67ndxM}{\includegraphics[width = 2cm]{img/youtube}}\\[1Ex]
\end{minipage}
\vspace{1ex}
%% --------------------------------------------------------------------



Es ist schon kurz nach fünf und Steffen wird langsam nervös. Steffen wollte heute Abend noch seine E-Sport Qualifikation schauen. Hoffentlich kommt er noch rechtzeitig zum Streamen. Angestrengend krault er die Schlange. Stattdessen versucht sein Betreuer die Ausgabe der einfaktoriellen ANOVA zu visualieren und zu überprüfen, ob es mit der Visualisierung der Daten als Boxplots zusammenpasst. Anscheinend gibt es ein Problem mit der Annahme der Normalverteilung und der Varianzhomogenität der ANOVA in den Daten. 'Wir haben jetzt bei der ANOVA einen p-Wert mit 0.058 raus sowie eine F-Statistik $F_D$ mit 1.78 berechnet. Nach den Boxplots müsste sich eigentlich ein Unterschied zwischen $thunder$ und $storm$ ergeben. Der Unterschied ist in \texttt{\{emmeans\}} auch signifikant mit einem p-Wert von 0.021. Wie kann das sein?', grummelt sein Betreuer. Steffen hatte im Wendland ein Feldexperiment mit Lauch durchgeführt. Dabei wurden die Daten $D$ erhoben. Es gab dabei eine Behandlungen Lüftungssysteme ($ctrl$, $storm$, $thunder$ und $tornado$). Gemessen wurde der Messwert ($Y$) Frischegewicht [kg/ha]. So kompliziert kann das jetzt doch nicht sein! Steffen hat schon genug Probleme. Wenn die Romantik nicht wäre, dann wäre es einfacher.

\begin{graybox}{Gegebene Formeln}
\begin{center}
  \begin{tabular}{ccc}
    $MS_{treatment} = \cfrac{SS_{treatment}}{df_{treatment}}$ &
    $MS_{error} = \cfrac{SS_{error}}{df_{error}}$ &
    $F_{D} = \cfrac{MS_{treatment}}{MS_{error}}$ \\
  \end{tabular}
\end{center}
\end{graybox}

Leider kennen sich Steffen und sein Betreuer mit der Interpretation einer ANOVA überhaupt nicht aus. Deshalb braucht er bei der Erstellung Ihre Hilfe und die Zeit wird knapp. 

\begin{enumerate}
  \item Formulieren Sie die wissenschaftliche Fragestellung! \textbf{(1 Punkt)}
  \item Formulieren Sie das statistische Hypothesenpaar! \textbf{(1 Punkt)}
\item Was bedeutet eine signifkante ANOVA für die beobachteten Daten $D$? \textbf{(1 Punkt)}
\item Visualisieren Sie den Unterschied zwischen Varianzhomogenität und Varianzheterogenität anhand der Daten $D$! Beschriften Sie die Abbildung! \textbf{(2 Punkte)} 
\item Visualisieren Sie für die Daten $D$ die Verletzung der Annahme der Varianzhomogenität der ANOVA unter zu Hilfenahme von Boxplots! Beschriften Sie die Abbildung! \textbf{(2 Punkte)}
\item Welche Auswirkung hat die Verletzung der Annahme der Varianzhomogenität für die Teststatistik $F_D$ der ANOVA? Begründen Sie Ihre Antwort! \textbf{(2 Punkte)}
\item Erklären Sie abschließend die Diskrepanz zwischen den Ergebnis der ANOVA und dem paarweisen Gruppenvergleich in \texttt{\{emmeans\}}! \textbf{(2 Punkte)}
\end{enumerate}

 
\clearpage
% -----------------------------------------------------------------------
\part{Multiple Gruppenvergleiche}
% ----------------------------------------------------------------------- 

\section{Aufgabe \hfill (12 Punkte)}

\textit{Geben Sie grundsätzlich Formeln und Rechenweg zur Lösung der Teilaufgaben mit an!} \\[1Ex]
 

 
%% --------------------------------------------------------------------
\begin{minipage}[t]{0.5\textwidth}
\includegraphics[width = 1.3cm]{/Users/kruppajo/work/GitHub/exam/avatare/Jessica.png}\hspace{-4mm}\includegraphics[width = 1.3cm]{/Users/kruppajo/work/GitHub/exam/avatare/Jonas.png}
\end{minipage}
\begin{minipage}[t]{0.5\textwidth}
\hfill
\href{https://youtu.be/kHmfEmU6lrk}{\includegraphics[width = 2cm]{img/youtube}}
\end{minipage}
\vspace{1ex}
%% --------------------------------------------------------------------



In einen Leistungssteigerungsversuch mit Hühnern wurde die Behandlung Ernährungszusatz ($ctrl$, $fedX$, $proteinX$ und $getIt$) gegen die Ergebnisse einer früheren Studie von Meyer et al. (2021) verglichen. Im Rahmen des Experiments haben Jonas und Jessica verschiedene Student t-Tests für den Mittelwertsvergleich für den Messwert Protein/Fettrate [\%/kg] gerechnet. Es ergab sich dann die folgende Tabelle der rohen p-Werte für die Vergleiche zu Meyer et al. (2021). Jetzt sollen die beiden einmal schauen, was in den Daten so drin ist.

\begin{knitrout}
\definecolor{shadecolor}{rgb}{0.969, 0.969, 0.969}\color{fgcolor}\begin{table}[!h]
\centering\begingroup\fontsize{10}{12}\selectfont

\begin{tabular}{ccc}
\toprule
\textbf{Rohen p-Werte} & \textbf{Adjustierte p-Werte} & \textbf{Nullhypothese ablehnen?}\\
\midrule
0.08 &  & \\
0.23 &  & \\
0.06 &  & \\
0.02 &  & \\
\bottomrule
\end{tabular}
\endgroup{}
\end{table}

\end{knitrout}

Leider kennen sich Jonas und Jessica mit der Adjustierung von $p$-Werten und dem Signifikanzniveau $\alpha$ überhaupt nicht aus. Deshalb brauchen die beiden bei der Erstellung Ihre Hilfe!

\begin{enumerate}
  \item Formulieren Sie die wissenschaftliche Fragestellung! \textbf{(1 Punkt)}
  \item Formulieren Sie die statistischen Hypothesen! \textbf{(1 Punkt)}
\item Füllen Sie die Spalte \textit{Adjustierte p-Werte} nach der Bonferoni-Methode aus! \textbf{(2 Punkte)}
\item Entscheiden Sie, ob nach der Adjustierung die Nullhypothese abgelehnt werden kann! Begründen Sie Ihre Antwort! \textbf{(2 Punkte)}
\item Wie ist Ihr Vorgehen, wenn Sie anstatt der $p$-Werte das Signifikanzniveau $\alpha$ adjustieren? \textbf{(2 Punkte)}
\item Erklären Sie warum die $p$-Werte oder das Signifikanzniveau $\alpha$ bei multiplen Vergleichen adjustiert werden müssen! \textbf{(2 Punkte)}
\item Würden Sie die Adjustierung der $p$-Werte oder die Adjustierung des Signifikanzniveaus $\alpha$ vorziehen? Begründen Sie Ihre Antwort! \textbf{(2 Punkte)}
\end{enumerate}


 
\clearpage
% ----------------------------------------------------------------------- 

\section{Aufgabe \hfill (10 Punkte)}

\textit{Geben Sie grundsätzlich Formeln und Rechenweg zur Lösung der Teilaufgaben mit an!} \\[1Ex]
 

 
%% --------------------------------------------------------------------
\begin{minipage}[t]{0.5\textwidth}
\includegraphics[width = 1.3cm]{/Users/kruppajo/work/GitHub/exam/avatare/Yuki.png}
\end{minipage}
\begin{minipage}[t]{0.5\textwidth}
\hfill
\href{https://youtu.be/xq29O8qDrg0}{\includegraphics[width = 2cm]{img/youtube}}
\end{minipage}
\vspace{1ex}
%% --------------------------------------------------------------------



Yuki sitzt schon etwas länger bei seinem Betreuer. So langsam macht Yuki sich Gedanken, ob er nicht doch mal anmerken sollte, dass er von CLD noch nie was gehört hat. Aber noch kann gelauscht werden, ein Ende ist erstmal nicht in Sicht! Yuki hatte in seiner Abschlussarbeit ein Stallexperiment durchgeführt. Deshalb sitzt er hier. Also eigentlich nein, deshalb nicht. Yuki will fertig werden. Hat er sich doch mit Genotypen ($00$, $AA$, $AB$ und $BB$) und Schlachtgewicht [kg] schon eine Menge angeschaut. Yuki beugt sich leicht nach vorne. Nein, doch keine Pause. Weiter warten auf eine Lücke im Fluss...

\begin{knitrout}
\definecolor{shadecolor}{rgb}{0.969, 0.969, 0.969}\color{fgcolor}\begin{table}[!h]
\centering\begingroup\fontsize{10}{12}\selectfont

\begin{tabular}{cc}
\toprule
\textbf{Behandlung} & \textbf{Compact letter display}\\
\midrule
00 & a\\
AA & a\\
AB & a\\
BB & a\\
\bottomrule
\end{tabular}
\endgroup{}
\end{table}

\end{knitrout}

Leider kennen sich Yuki mit dem \textit{Compact letter display (CLD)} überhaupt nicht aus. Deshalb braucht er bei der Erstellung Ihre Hilfe!

\begin{enumerate}
  \item Formulieren Sie die wissenschaftliche Fragestellung! \textbf{(1 Punkt)}
  \item Formulieren Sie die statistischen Hypothesen! \textbf{(1 Punkt)}
\item Zeichnen Sie die sich anhand des \textit{Compact letter display (CLD)} ergebenden Barplots! \textbf{(2 Punkte)}
\item Ergänzen Sie das \textit{Compact letter display (CLD)} zu den Barplots! \textbf{(1 Punkt)}
\item Erklären Sie \textit{einen} Vorteil und \textit{einen} Nachteil des \textit{Compact letter display (CLD)}! \textbf{(2 Punkte)}
\item Erstellen Sie eine Matrix mit den paarweisen $p$-Werten eines Student t-Tests, die sich näherungsweise aus dem \textit{Compact letter display (CLD)} ergeben würde! Begründen Sie Ihre Antwort! \textbf{(3 Punkte)}
\end{enumerate}

 
\clearpage
% ----------------------------------------------------------------------- 

\section{Aufgabe \hfill (12 Punkte)}

\textit{Geben Sie grundsätzlich Formeln und Rechenweg zur Lösung der Teilaufgaben mit an!} \\[1Ex]
 

 
%% --------------------------------------------------------------------
\begin{minipage}[t]{0.5\textwidth}
\includegraphics[width = 1.3cm]{/Users/kruppajo/work/GitHub/exam/avatare/Yuki.png}
\end{minipage}
\begin{minipage}[t]{0.5\textwidth}
\hfill
\href{https://youtu.be/RagTFFKFbFg}{\includegraphics[width = 2cm]{img/caution}}
\end{minipage}
\vspace{1ex}
%% --------------------------------------------------------------------



Yuki betrachtet in sich gekehrt die Poster vor dem Büro von ihr Betreuer. Viele der explorativen Abbildungen sagen ihr etwas. Die Barplots und die Boxplots könnte sie dann schon nachbauen. Das macht sie dann zuversichtlich die Abschlussarbeit auch hinzukriegen. Etwas komischer sind die seltsamen Buchstaben über den Barplots. Yuki betrachtet ein Poster das sich mit Maiss beschäftigt. Substrattypen ($kompost$, $torf$, $40p60n$ und $70p30n$) und Proteingehalt [g/kg] wurden dort bestimmt. So richtig schlau, wird sie daraus nicht. Als erstes müsse müsse man die Gruppen nach absteigender Effektstärke sortieren, liest Yuki im Methodenteil und ist dann noch verwirrter als vorher schon.

\begin{knitrout}
\definecolor{shadecolor}{rgb}{0.969, 0.969, 0.969}\color{fgcolor}\begin{table}[!h]
\centering\begingroup\fontsize{10}{12}\selectfont

\begin{tabular}{cccc}
\toprule
\textbf{Substrattypen} & \textbf{Fallzahl (n)} & \textbf{Mittelwert} & \textbf{Standardabweichung}\\
\midrule
kompost & 7 & 14.63 & 3.06\\
torf & 9 & 5.97 & 2.53\\
40p60n & 9 & 5.62 & 2.94\\
70p30n & 8 & 10.76 & 1.79\\
\bottomrule
\end{tabular}
\endgroup{}
\end{table}

\end{knitrout}

Leider kennen sich Yuki mit dem \textit{Compact letter display (CLD)} überhaupt nicht aus. Deshalb braucht sie bei der Erstellung Ihre Hilfe!

\begin{enumerate}
  \item Formulieren Sie die wissenschaftliche Fragestellung! \textbf{(1 Punkt)}
  \item Formulieren Sie die statistischen Hypothesen! \textbf{(1 Punkt)}
\item Zeichnen Sie die sich ergebenden Barplots! \textbf{(1 Punkt)}
\item Berechnen Sie die Matrix der $p$-Werte anhand von Student t-Tests! \textbf{(4 Punkte)}
\item Ergänzen Sie das \textit{Compact letter display (CLD)} zu den gezeichneten Barplots! Begründen Sie Ihre Antwort! \textbf{(4 Punkte)}
\item Interpretieren Sie das \textit{Compact letter display (CLD)} für Yuki und Jessica! \textbf{(1 Punkt)} 
\end{enumerate}

 
\clearpage
% -----------------------------------------------------------------------

\section{Aufgabe \hfill (10 Punkte)}

\textit{Geben Sie grundsätzlich Formeln und Rechenweg zur Lösung der Teilaufgaben mit an!} \\[1Ex]
 

 
%% --------------------------------------------------------------------
\begin{minipage}[t]{0.5\textwidth}
\includegraphics[width = 1.3cm]{/Users/kruppajo/work/GitHub/exam/avatare/Steffen.png}
\end{minipage}
\begin{minipage}[t]{0.5\textwidth}
\hfill
\href{https://youtu.be/RagTFFKFbFg}{\includegraphics[width = 2cm]{img/youtube}}
\end{minipage}
\vspace{1ex}
%% --------------------------------------------------------------------



'Okay, dann nochmal für mich. Ich habe jetzt alles in SPSS gemacht, aber das Wichtigste, was gemacht werden soll, nämlich das CLD, das kann ich nicht in SPSS machen?', Steffen muss sich echt beherrschen. Immerhin betreut seine Betreuerin ja erst nicht seit gestern Abschlussarbeiten und wusste ja was gemacht werden soll! Steffen hatte sich zwei Variablen mit Lüftungssystem ($keins$, $storm$, $tornado$ und $thunder$) und Schlachtgewicht [kg] in einen Leistungssteigerungsversuch mit Puten angeschaut. Jetzt möchte er eigentlich fertig werden und nicht nochmal alles neu in \Rlogo und \texttt\{emmeans\} machen. Deshalb soll jetzt das CLD per Hand aus der Matrix der $p$-Wert abgeleitet werden. 'Ich glaube ich wechsel nochmal das Thema...', denkt Steffen, verwirft dann aber den Gedanken.

\begin{knitrout}
\definecolor{shadecolor}{rgb}{0.969, 0.969, 0.969}\color{fgcolor}\begin{table}[!h]
\centering\begingroup\fontsize{10}{12}\selectfont

\begin{tabular}{>{}lcccc}
\toprule
\textbf{ } & \textbf{keins} & \textbf{storm} & \textbf{tornado} & \textbf{thunder}\\
\midrule
\textbf{keins} & 1.0000000 & 0.0249588 & 0.6510403 & 0.2412882\\
\textbf{storm} & 0.0249588 & 1.0000000 & 0.0166510 & 0.3342426\\
\textbf{tornado} & 0.6510403 & 0.0166510 & 1.0000000 & 0.1451388\\
\textbf{thunder} & 0.2412882 & 0.3342426 & 0.1451388 & 1.0000000\\
\bottomrule
\end{tabular}
\endgroup{}
\end{table}

\end{knitrout}

Leider kennen sich Steffen mit dem \textit{Compact letter display (CLD)} überhaupt nicht aus. Deshalb braucht er bei der Erstellung Ihre Hilfe!

\begin{enumerate}
  \item Formulieren Sie die wissenschaftliche Fragestellung! \textbf{(1 Punkt)}
  \item Formulieren Sie die statistischen Hypothesen! \textbf{(1 Punkt)}
\item Zeichnen Sie die sich anhand der Matrix der $p$-Werte ergebenden Barplots! \textbf{(2 Punkte)}
\item Ergänzen Sie das \textit{Compact letter display (CLD)}! Begründen Sie Ihre Antwort! \textbf{(4 Punkte)}
\item Interpretieren Sie das \textit{Compact letter display (CLD)} für Steffen und Jessica! \textbf{(2 Punkte)} 
\end{enumerate}

 
\clearpage
% -----------------------------------------------------------------------
\part{Der Chi-Quadrat-Test \& Der diagnostische Test}
% -----------------------------------------------------------------------

\section{Aufgabe \hfill (12 Punkte)}

\textit{Geben Sie grundsätzlich Formeln und Rechenweg zur Lösung der Teilaufgaben mit an!} \\[1Ex]
 

 
%% --------------------------------------------------------------------
\begin{minipage}[t]{0.5\textwidth}
\includegraphics[width = 1.3cm]{/Users/kruppajo/work/GitHub/exam/avatare/Steffen.png}
\end{minipage}
\begin{minipage}[t]{0.5\textwidth}
\hfill
\href{https://youtu.be/-Kva5wc5Elw}{\includegraphics[width = 2cm]{img/youtube}}\\[1Ex]
\end{minipage}
\vspace{1ex}
%% --------------------------------------------------------------------



'Der $\mathcal{X}^2$-Test auf einer $2x2$-Kreuztabelle berechnet.', liest Steffen in seiner Mitschrift. So richtig helfen tut ihm das jetzt eherlichweise dann doch nicht. Steffen hatte sich in ein Kreuzungsexperiment $n = 146$ Beobachtungen von Hühnern angeschaut. Dabei hat er als Behandlung \textit{Außenklimakontakt [ja/nein]} bestimmt und zum anderen die Variable \textit{Schlachtgewicht im Zielbereich [ja/nein]} ermittelt. Am Ende möchte dann seine Betreuerin gerne einen $\mathcal{X}^2$-Test auf einer $2x2$-Kreuztabelle berechnet bekommen.


\vspace{5Ex}

\begin{center}
  \huge
  \begin{tabular}{c|c|c|c}
     & \phantom{\textbf{Erkrankt (ja)}} & \phantom{\textbf{Erkrankt (ja)}} & \phantom{\textbf{Erkrankt (ja)}} \strut\\
    \hline
    \phantom{\textbf{Pestizid (ja)}} & 56  & 11  &     \strut\\
    \hline
    \phantom{\textbf{Pestizid (ja)}} & 27  & 52  &      \strut\\
    \hline
     \phantom{100} & \phantom{100}  & \phantom{100}  &  \phantom{100}  \strut\\
  \end{tabular}
\end{center}

\vspace{5Ex}

Leider kennt sich Steffen mit der Berechnung eines $\mathcal{X}^2$-Test für kategoriale Daten überhaupt nicht aus. Deshalb braucht er bei der Erstellung Ihre Hilfe!

\begin{enumerate}
\item Formulieren Sie die wissenschaftliche Fragestellung! \textbf{(1 Punkt)}
\item Ergänzen Sie die Tabelle um die fehlenden Informationen! \textbf{(1 Punkt)} 
\item Visualisieren Sie den Zusammenhang zwischen den beiden kategorialen Variablen! \textbf{(2 Punkte)}
\item Berechnen Sie die Teststatistik eines Chi-Quadrat-Test! \textbf{(2 Punkte)}
\item Treffen Sie eine Entscheidung im Bezug zu der Nullhypothese gegeben
  einem $\mathcal{X}^2_{\alpha = 5\%} = 3.841$! Begründen Sie Ihre Antwort!
  \textbf{(2 Punkte)}
\item Skizzieren Sie die $\mathcal{X}^2$-Verteilung, wenn die $H_0$ wahr ist! Ergänzen Sie  $\mathcal{X}^2_{\alpha = 5\%}$ und $\mathcal{X}^2_{D}$ in der Abbildung! \textbf{(2 Punkte)}
\item Berechnen Sie den Effektschätzer $Cramers\; V$! Interpretieren Sie den
  Effektschätzer! \textbf{(2 Punkte)}
\end{enumerate} 
\clearpage
% -----------------------------------------------------------------------

\section{Aufgabe \hfill (10 Punkte)}

\textit{Geben Sie grundsätzlich Formeln und Rechenweg zur Lösung der Teilaufgaben mit an!} \\[1Ex]
 

 
%% --------------------------------------------------------------------
\begin{minipage}[t]{0.5\textwidth}
\includegraphics[width = 1.3cm]{/Users/kruppajo/work/GitHub/exam/avatare/Steffen.png}
\end{minipage}
\begin{minipage}[t]{0.5\textwidth}
\hfill
\href{https://youtu.be/jakM7fHyZfU}{\includegraphics[width = 2cm]{img/youtube}}\\[1Ex]
\end{minipage}
\vspace{1ex}
%% --------------------------------------------------------------------



Am Ende hätte Steffen dann doch einen normalverteilten Endpunkt in seinem Projektbericht nehmen sollen. Vor ihm liegen jetzt die Daten von zwei Variablen als Kategorien oder wie es in \Rlogo so schön heißt, als Faktoren. Aber immerhin, hofft er das was bei den Daten rausgekommen ist. Gezählt hat Steffen einiges mit $n = 132$ Beobachtungen von Kartoffeln. Zum einen hat er als Behandlung \textit{Mechanische Bearbeitung [ja/nein]} bestimmt und zum anderen die Variable \textit{Trockengewicht über Zielwert [ja/nein]} ermittelt. Nun möchte sein Betreuer gerne einen $\mathcal{X}^2$-Test auf einer $2x2$-Kreuztabelle berechnet bekommen.

\vspace{5Ex}

\begin{center}
  \huge
  \begin{tabular}{c|c|c|c}
     & \phantom{\textbf{Erkrankt (ja)}} & \phantom{\textbf{Erkrankt (ja)}} & \phantom{\textbf{Erkrankt (ja)}} \strut\\
    \hline
   \phantom{\textbf{Pestizid (ja)}} & \phantom{100}  & \phantom{100}  &   84  \strut\\
    \hline
    \phantom{\textbf{Pestizid (ja)}} & \phantom{100}  & \phantom{100}  &   48   \strut\\
    \hline
     &  64 &  68 &  132  \strut\\
  \end{tabular}
\end{center}

\vspace{5Ex}

Leider kennt sich Steffen mit der Berechnung eines $\mathcal{X}^2$-Test für kategoriale Daten überhaupt nicht aus. Deshalb braucht er bei der Erstellung Ihre Hilfe!

\begin{enumerate}
  \item Formulieren Sie die wissenschaftliche Fragestellung! \textbf{(1 Punkt)}
\item Ergänzen Sie die Tabelle um die fehlenden Informationen! \textbf{(1 Punkt)} 
\item Ergänzen Sie die Felder innerhalb der $2x2$ Kreuztabelle, so dass \textit{ein} signifikanter Effekt zu erwarten wäre! \textbf{(2 Punkte)}
\item Begründen Sie Ihr Vorgehen an der Formel des Chi-Quadrat-Tests. Erklären Sie Ihr Vorgehen an einem Beispiel! \textbf{(2 Punkte)}
\item Visualisieren Sie den Zusammenhang zwischen den beiden kategorialen Variablen! \textbf{(2 Punkte)}
\item Was ist die Mindestanzahl an Beobachtungen je Zelle? Wenn in einer der Zellen weniger Beobachtungen auftreten, welchen Test können Sie anstatt des Standard Chi-Quadrat-Tests anwenden? \textbf{(2 Punkte)}
\end{enumerate} 
\clearpage
% -----------------------------------------------------------------------

\section{Aufgabe \hfill (10 Punkte)}

\textit{Geben Sie grundsätzlich Formeln und Rechenweg zur Lösung der Teilaufgaben mit an!} \\[1Ex]
 

 
%% --------------------------------------------------------------------
\begin{minipage}[t]{0.5\textwidth}
\includegraphics[width = 1.3cm]{/Users/kruppajo/work/GitHub/exam/avatare/Yuki.png}
\end{minipage}
\begin{minipage}[t]{0.5\textwidth}
\hfill
\href{https://youtu.be/ghArbetOr_E}{\includegraphics[width = 2cm]{img/youtube}}\\[1Ex]
\end{minipage}
\vspace{1ex}
%% --------------------------------------------------------------------



Yuki hat sich ein Herz gefasst und war für ihrer Hausarbeit in die Niederlande gegangen. Das war eine super Zeit in der sie viel gelernt hat. Klar gab es auch die ein oder andere Besonderheit, aber das gehört hier eher nicht hin. Yuki ist schon eine ganze Zeit im Büro, da ihre Betreuerin möchte, dass sie jetzt auf ihren Daten mit $n = 132$ Beobachtungen von Maiss einen $\mathcal{X}^2$-Test rechnet. Das ginge, da sie als Behandlung \textit{Pestizideinsatz [ja/nein]} bestimmt und zum anderen die Variable \textit{Frischegewicht über Zielwert [ja/nein]} ermittelt hat. Wie genau, das ist jetzt eine andere Frage. Nach ihrem Experiment erhielt sie folgende $2x2$ Kreuztabelle aus ihren erhobenen Daten.

\begin{knitrout}
\definecolor{shadecolor}{rgb}{0.969, 0.969, 0.969}\color{fgcolor}\begin{kframe}
\begin{verbatim}
##                             Pestizideinsatz
## Frischegewicht über Zielwert ja nein
##                         ja   13    5
##                         nein  7   18
\end{verbatim}
\end{kframe}
\end{knitrout}

Dann rechnete Yuki den Fisher-Exakt-Test auf der $2x2$-Kreuztabelle in \Rlogo und erhielt folgende \Rlogo Ausgabe der Funktion \texttt{fisher.test()}.

\begin{knitrout}
\definecolor{shadecolor}{rgb}{0.969, 0.969, 0.969}\color{fgcolor}\begin{kframe}
\begin{verbatim}
## 
## 	Fisher's Exact Test for Count Data
## 
## data:  Frischegewicht über Zielwert
## p-value = 0.005898
## alternative hypothesis: true odds ratio is not equal to 1
## 95 percent confidence interval:
##   1.462677 32.500828
## sample estimates:
## odds ratio 
##   6.352594
\end{verbatim}
\end{kframe}
\end{knitrout}

Leider kennt sich Yuki mit der Berechnung eines $\mathcal{X}^2$-Test für kategoriale Daten überhaupt nicht aus. Deshalb braucht sie bei der Erstellung Ihre Hilfe!

\begin{enumerate}
\item Formulieren Sie die wissenschaftliche Fragestellung! \textbf{(1 Punkt)}
\item Visualisieren Sie den Zusammenhang zwischen den beiden kategorialen Variablen! \textbf{(2 Punkte)}
\item Liegt ein signifikanter Unterschied zwischen den Gruppen vor? Begründen Sie Ihre Antwort! \textbf{(2 Punkte)}
\item Skizzieren Sie das sich ergebende 95\% Konfidenzintervall! \textbf{(2 Punkte)}
\item Beschriften Sie die Abbildung des 95\% Konfidenzintervalls! \textbf{(1 Punkt)} 
\item Interpretieren Sie das \textit{Odds ratio} im Kontext der wissenschaftlichen Fragestellung! \textbf{(2 Punkte)} 
\end{enumerate}
 
\clearpage
% -----------------------------------------------------------------------

\section{Aufgabe \hfill (11 Punkte)}

\textit{Geben Sie grundsätzlich Formeln und Rechenweg zur Lösung der Teilaufgaben mit an!} \\[1Ex]
 

 
%% --------------------------------------------------------------------
\begin{minipage}[t]{0.5\textwidth}
\includegraphics[width = 1.3cm]{/Users/kruppajo/work/GitHub/exam/avatare/Alex.png}\hspace{-4mm}\includegraphics[width = 1.3cm]{/Users/kruppajo/work/GitHub/exam/avatare/Steffen.png}
\end{minipage}
\begin{minipage}[t]{0.5\textwidth}
\hfill
\href{https://youtu.be/VQlNl8hvRII}{\includegraphics[width = 2cm]{img/youtube}}
\end{minipage}
\vspace{1ex}
%% --------------------------------------------------------------------



Steffen liest laut vor. 'Die Prävalenz von Klauenseuche bei Lauch wird mit 2\% angenommen. In 75\% der Fälle ist ein Test positiv, wenn das Pflanze erkrankt ist. In 8.5\% der Fälle ist ein Test positiv, wenn das Pflanze \underline{nicht} erkrankt ist und somit gesund ist. Wir führen einen Test auf Kräuselkrankheit an 1000 Lauch mit einem diagnostischen Test durch.' Alex klappt die Kinnlade runter. In der Stille duddelt Taylor Swift. Steffen schaut kompetent und schmeißt sich mit offenen Mund Gummibärchen an den Kopf vorbei.

\begin{center}
  \includegraphics[width=17cm]{/Users/kruppajo/work/GitHub/exam/question/img/diag-doppelbaum}
\end{center}

Leider kennen sich Steffen und Alex mit dem diagnostischen Testen überhaupt nicht aus. Deshalb brauchen beide bei der Erstellung Ihre Hilfe! 
    
\begin{enumerate}
\item Beschriften Sie die Äste des Doppelbaumes, mit denen Ihnen bekannten Informationen! \textbf{(2 Punkte)}
\item Beschriften Sie den Doppelbaum! \textbf{(2 Punkte)}
\item Füllen Sie freien Felder des Doppelbaums aus! \textbf{(4 Punkte)}
\item Berechnen Sie die Wahrscheinlichkeit $Pr(K^+|T^+)$! \textbf{(2 Punkte)}
\item Was sagt Ihnen die Wahrscheinlichkeit $Pr(K^+|T^+)$ aus? \textbf{(1 Punkt)}
\end{enumerate}






 
\clearpage
% -----------------------------------------------------------------------

\section{Aufgabe \hfill (11 Punkte)}

\textit{Geben Sie grundsätzlich Formeln und Rechenweg zur Lösung der Teilaufgaben mit an!} \\[1Ex]
 

 
%% --------------------------------------------------------------------
\begin{minipage}[t]{0.5\textwidth}
\includegraphics[width = 1.3cm]{/Users/kruppajo/work/GitHub/exam/avatare/Alex.png}\hspace{-4mm}\includegraphics[width = 1.3cm]{/Users/kruppajo/work/GitHub/exam/avatare/Steffen.png}
\end{minipage}
\begin{minipage}[t]{0.5\textwidth}
\hfill
\href{https://youtu.be/_7s44pbOc00}{\includegraphics[width = 2cm]{img/youtube}}
\end{minipage}
\vspace{1ex}
%% --------------------------------------------------------------------



'Was ist denn das?', entfährt es Alex. 'Hm... ich glaube es handelt sich um einen Doppelbaum, den wir beim diagnostischen Testen brauchen.', meint Steffen und dreht Alien auf dem Second Screen etwas leiser. Was jetzt beide von einem diagnostischen Test haben, ist ihnen auch nicht klar. Es ist ja schon alles komplex genug und die Gefälligkeit von Alex macht es heute auch nicht mehr einfacher. 'Es geht um Escherichia coli (E. coli) an Zandern.', stellt Steffen fest. Eigentlich wollte Steffen eher los um zu Ringen. Das wird aber wohl nichts mehr.

\begin{tikzpicture}
  \node (image) at (0,0) {
    \includegraphics[width=\textwidth]{/Users/kruppajo/work/GitHub/exam/question/img/diag-doppelbaum}
  };
  \node[] at (-4.8,0) {\huge 180};
  \node[] at (-1.7,0) {\huge 60};
  \node[] at (1.7,0) {\huge 850};
  \node[] at (4.75,0) {\huge 1450};
\end{tikzpicture}

Leider kennen sich Alex und Steffen mit dem diagnostischen Testen überhaupt nicht aus. Deshalb brauchen beide bei der Erstellung Ihre Hilfe! 
  
\begin{enumerate}
\item Beschriften Sie den Doppelbaum! \textbf{(2 Punkte)}
\item Füllen Sie freien Felder des Doppelbaums aus! \textbf{(4 Punkte)}
\item Berechnen Sie die Wahrscheinlichkeit $Pr(K^+|T^+)$! \textbf{(2 Punkte)}
\item Berechnen Sie die Prävalenz für Klauenseuche! \textbf{(1 Punkt)}
\item Berechnen Sie die Sensifität und Spezifität des diagnostischen Tests! Erstellen Sie dafür zunächst eine 2x2 Kreuztabelle! \textbf{(2 Punkte)}
\end{enumerate}

 





 
\clearpage
% -----------------------------------------------------------------------
\part{Lineare Regression \& Korrelation}
% -----------------------------------------------------------------------

\section{Aufgabe \hfill (10 Punkte)}

\textit{Geben Sie grundsätzlich Formeln und Rechenweg zur Lösung der Teilaufgaben mit an!} \\[1Ex]
 

 
%% --------------------------------------------------------------------
\begin{minipage}[t]{0.5\textwidth}
\includegraphics[width = 1.3cm]{/Users/kruppajo/work/GitHub/exam/avatare/Jonas.png}\hspace{-4mm}\includegraphics[width = 1.3cm]{/Users/kruppajo/work/GitHub/exam/avatare/Tina.png}
\end{minipage}
\begin{minipage}[t]{0.5\textwidth}
\hfill
\href{https://youtu.be/kHmfEmU6lrk}{\includegraphics[width = 2cm]{img/youtube}}
\end{minipage}
\vspace{1ex}
%% --------------------------------------------------------------------



'Hä? Hatten wir das als Aufgabe nicht schon mal, das wir aus kontinuierlichen Daten eine Abbildung bauen sollten?', fragt Tina. Jonas schaut fragend zurück. 'Kann mich wie immer an nichts erinnern. Können wir trotzdem jetzt erstmal die Daten auswerten? Columbo?', antwortet Jonas leicht angespannt. Die beiden hatten ein Kreuzungsexperiment im Emsland mit Fleischrindern durchgeführt. Dabei wurden die beiden folgenden Variablen gemessen: durchschnittlichen Bewegungsscore [Movement/h] und Schlachtgewicht [kg]. Jetzt haben die beiden eigentlich alles zusammen. \textit{Eigentlich...}

\begin{table}[!h]
\centering
\begin{tabular}{cc}
\toprule
Durchschnittlichen Bewegungsscore [Movement/h] & Schlachtgewicht [kg]\\
\midrule
22.0 & 36.9\\
21.5 & 35.6\\
25.0 & 40.9\\
27.1 & 41.6\\
22.9 & 36.8\\
\addlinespace
16.5 & 29.3\\
35.5 & 52.2\\
28.5 & 45.5\\
29.2 & 45.4\\
19.7 & 33.9\\
\addlinespace
30.4 & 43.2\\
\bottomrule
\end{tabular}
\end{table}



Leider kennen sich Tina und Jonas mit der linearen Regression für kontinuierliche Daten überhaupt nicht aus. Deshalb brauchen beide bei der Erstellung Ihre Hilfe!

\begin{enumerate}
\item Formulieren Sie die wissenschaftliche Fragestellung! \textbf{(1 Punkt)}
\item Erstellen  Sie  eine  Visualisierung  für  die  Datentabelle.  Beschriften  Sie  die  Achsen! \textbf{(2 Punkte)}
\item Schätzen Sie die Regressionsgleichung aus der obigen Abbildung ab! \textbf{(2 Punkte)}
\item Beschriften Sie die Grade mit den statistischen Maßzahlen der linearen Regressionsgleichung! \textbf{(2 Punkte)}
\item Liegt ein Zusammenhang zwischen $x$ und $y$ vor? Begründen Sie Ihre Antwort! \textbf{(2 Punkte)}
\item Wenn kein Zusammenhang zu beobachten wäre, wie würde die Grade aussehen? \textit{Antworten Sie mit einer Skizze der Geraden!} \textbf{(1 Punkt)}
\end{enumerate} 
\clearpage
% -----------------------------------------------------------------------

\section{Aufgabe \hfill (12 Punkte)}

\textit{Geben Sie grundsätzlich Formeln und Rechenweg zur Lösung der Teilaufgaben mit an!} \\[1Ex]
 

 
%% --------------------------------------------------------------------
\begin{minipage}[t]{0.5\textwidth}
\includegraphics[width = 1.3cm]{/Users/kruppajo/work/GitHub/exam/avatare/Mark.png}\hspace{-4mm}\includegraphics[width = 1.3cm]{/Users/kruppajo/work/GitHub/exam/avatare/Nilufar.png}
\end{minipage}
\begin{minipage}[t]{0.5\textwidth}
\hfill
\href{https://youtu.be/lJp8rFmMnrs}{\includegraphics[width = 2cm]{img/youtube}}
\end{minipage}
\vspace{1ex}
%% --------------------------------------------------------------------



'Hä? Was ist denn das? Hatten wir das als Aufgabe eine lineare Regression zu rechnen? Wir bauen aus kontinuierlichen Daten eine Abbildung und interpretieren diese dann?', fragt Mark. Nilufar schaut fragend zurück. 'Keine Ahnung... das ist jetzt jedenfalls keine Abbildung von irgendwas sondern Zahlen in einer Tabelle...', antwortet Nilufar leicht angespannt. Die beiden hatten ein Kreuzungsexperiment im Emsland mit Fleischrindern durchgeführt. Dabei wurden die beiden folgenden Variablen gemessen: mittlere Anzahl an weißen Blutkörperchen [LEU/ml] und Fettgehalt [\%/kg]. Jetzt haben die beiden eigentlich alles zusammen. \textit{Eigentlich...}, denn mit der \Rlogo Ausgabe haben beide jetzt ein Problem.

\begin{table}[!h]
\centering\begingroup\fontsize{12}{14}\selectfont

\begin{tabular}{ccccc}
\toprule
term & estimate & std.error & t statistic & p-value\\
\midrule
(Intercept) & 3.23 & 1.74 &  & \\
Mittlere Anzahl & 1.91 & 0.17 &  & \\
\bottomrule
\end{tabular}
\endgroup{}
\end{table}



Leider kennen sich Mark und Nilufar mit der linearen Regression für kontinuierliche Daten in \Rlogo überhaupt nicht aus. Deshalb brauchen beide bei der Erstellung Ihre Hilfe!

\begin{enumerate}
\item Formulieren Sie die wissenschaftliche Fragestellung! \textbf{(1 Punkt)}
\item Erstellen  Sie  eine  Visualisierung  der \texttt{lm()}-Ausgabe.  Beschriften  Sie  die  Achsen! \textbf{(2 Punkte)}
\item Beschriften Sie die Visualisierung mit den statistischen Maßzahlen der der \texttt{lm()}-Ausgabe! \textbf{(2 Punkte)}
\item Formulieren Sie die Regressionsgleichung! \textbf{(1 Punkt)}
\item Ergänzen Sie die t Statistik in der \texttt{lm()}-Ausgabe! \textbf{(2 Punkte)}
\item Ergänzen Sie den $p$-Wert in der \texttt{lm()}-Ausgabe mit $T_{\alpha = 5\%} = 1.96$!  \textbf{(2 Punkte)}
\item Interpretieren Sie den $p$-Wert im Kontext der wissenschaftlichen Fragestellung! \textbf{(1 Punkt)}  
\item Wie groß ist der Effekt im Kontext der wissenschaftlichen Fragestellung? \textbf{(1 Punkt)}
\end{enumerate} 
\clearpage
% -----------------------------------------------------------------------

\section{Aufgabe \hfill (11 Punkte)}

\textit{Geben Sie grundsätzlich Formeln und Rechenweg zur Lösung der Teilaufgaben mit an!} \\[1Ex]
 

 
%% --------------------------------------------------------------------
\begin{minipage}[t]{0.5\textwidth}
\includegraphics[width = 1.3cm]{/Users/kruppajo/work/GitHub/exam/avatare/Jonas.png}\hspace{-4mm}\includegraphics[width = 1.3cm]{/Users/kruppajo/work/GitHub/exam/avatare/Paula.png}
\end{minipage}
\begin{minipage}[t]{0.5\textwidth}
\hfill
\href{https://youtu.be/tNNzcndrpSk}{\includegraphics[width = 2cm]{img/youtube}}
\end{minipage}
\vspace{1ex}
%% --------------------------------------------------------------------



'Hä? Was ist denn das? Das wird ja immer wilder! Hatten wir das als Aufgabe eine lineare Regression zu rechnen? Wir bauen aus kontinuierlichen Daten eine Abbildung und interpretieren diese dann?', fragt Jonas. Paula schaut fragend zurück. 'Keine Ahnung... das ist jetzt jedenfalls keine Abbildung von irgendwas sondern eine \Rlogo Ausgabe mit ganz wilden Bezeichnungen...', antwortet Paula leicht angespannt. Die beiden hatten ein Feldexperiment im Oldenburger Land mit Lauch durchgeführt. Dabei wurden die beiden folgenden Variablen gemessen: durchschnittlicher Anteil an Ton [\%/l] und Proteingehalt [g/kg]. Jetzt haben die beiden eigentlich alles zusammen. \textit{Eigentlich...}, denn mit der \Rlogo Ausgabe haben beide jetzt ein Problem.

\begin{knitrout}
\definecolor{shadecolor}{rgb}{0.969, 0.969, 0.969}\color{fgcolor}\begin{kframe}
\begin{verbatim}
## 
## Call:
## Proteingehalt ~ Durchschnittlicher_Anteil
## 
## Residuals:
##     Min      1Q  Median      3Q     Max 
## -3.3677 -0.7910 -0.0465  1.1031  2.7717 
## 
## Coefficients:
##                           Estimate Std. Error t value Pr(>|t|)
## (Intercept)                 1.5531     1.5471   1.004    0.322
## Durchschnittlicher_Anteil   0.1479     0.1577   0.938    0.354
## 
## Residual standard error: 1.506 on 37 degrees of freedom
## Multiple R-squared:  0.02322,	Adjusted R-squared:  -0.003182 
## F-statistic: 0.8795 on 1 and 37 DF,  p-value: 0.3544
\end{verbatim}
\end{kframe}
\end{knitrout}

Leider kennen sich Jonas und Paula mit der linearen Regression für kontinuierliche Daten in \Rlogo überhaupt nicht aus. Deshalb brauchen beide bei der Erstellung Ihre Hilfe!


\begin{enumerate}
\item Formulieren Sie die wissenschaftliche Fragestellung! \textbf{(1 Punkt)}
\item Wie groß ist der Effekt im Kontext der wissenschaftlichen Fragestellung? \textbf{(2 Punkte)} 
\item Interpretieren Sie die $p$-Werte im Kontext der wissenschaftlichen Fragestellung! \textbf{(2 Punkte)}
\item Visualisieren Sie die Verteilung der Residuen! \textbf{(2 Punkte)} 
\item Ist die Annahme der Normalverteilung erfüllt? Begründen Sie die Antwort! \textbf{(2 Punkte)}
\item Erklären Sie \textit{kurz} den Begriff \texttt{R-squared}! Was sagt Ihnen der Wert aus? \textbf{(2 Punkte)}
\end{enumerate}
 
\clearpage
% -----------------------------------------------------------------------

\section{Aufgabe \hfill (10 Punkte)}

\textit{Geben Sie grundsätzlich Formeln und Rechenweg zur Lösung der Teilaufgaben mit an!} \\[1Ex]
 

 
%% --------------------------------------------------------------------
\begin{minipage}[t]{0.5\textwidth}
\includegraphics[width = 1.3cm]{/Users/kruppajo/work/GitHub/exam/avatare/Jessica.png}
\end{minipage}
\begin{minipage}[t]{0.5\textwidth}
\hfill
\href{https://youtu.be/C9skfFRTHhI}{\includegraphics[width = 2cm]{img/youtube}}
\end{minipage}
\vspace{1ex}
%% --------------------------------------------------------------------



'Hä? Was ist denn das? Das wird ja immer wilder! Hatten wir als Aufgabe eine Korrelation zu berechnen? Wir bauen aus kontinuierlichen Daten eine Abbildung und interpretieren diese dann?', fragt sich Jessica laut. 'Keine Ahnung... das ist jetzt jedenfalls keine Abbildung von irgendwas sondern eine \Rlogo Ausgabe mit ganz wilden Bezeichnungen...', denkt sie. Aber das hilft jetzt auch nicht. Irgendwie komisch, wenn sie Herr der Ringe anmacht, dann ist die Hündin eigentlich sofort vor dem Bildschirm und starrt hinein. Jessica hatte ein Kreuzungsexperiment in der Uckermark mit Hühnern durchgeführt. Dabei wurden die beiden folgenden Variablen gemessen: durchschnittliche Tagestemperatur [C/d] und Gewichtszuwachs in der 1LW. Jetzt hat sie eigentlich alles zusammen. \textit{Eigentlich...}, denn mit der \Rlogo Ausgabe hat Jessica jetzt ein Problem. Eine echte Herausforderung für sie war schon immer der Mangel gewesen. Ein leidiges Lied. 


\begin{knitrout}
\definecolor{shadecolor}{rgb}{0.969, 0.969, 0.969}\color{fgcolor}\begin{kframe}
\begin{verbatim}
## 
## 	Pearson's correlation
## 
## data:  Durchschnittliche Tagestemperatur and Gewichtszuwachs
## t = 3.8997, df = 8, p-value = 0.004547
## alternative hypothesis: true correlation is not equal to 0
## 95 percent confidence interval:
##  0.3668440 0.9532634
## sample estimates:
##      cor 
## 0.809497
\end{verbatim}
\end{kframe}
\end{knitrout}

Leider kennt sich Jessica mit der Korrelationsanalyse in \Rlogo überhaupt nicht aus. Deshalb braucht sie bei der Erstellung Ihre Hilfe!

\begin{enumerate}
  \item Formulieren Sie die wissenschaftliche Fragestellung! \textbf{(1 Punkt)}
  \item Formulieren Sie das statistische Hypothesenpaar! \textbf{(1 Punkt)}
\item Erstellen Sie eine Visualisierung für den Korrelationskoeffizienten! Beschriften Sie die Abbildung! \textbf{(2 Punkte)}
\item Nennen Sie die zwei Eigenschaften des Korrelationskoeffizienten! \textbf{(2 Punkte)}
\item Interpretieren Sie den Korrelationskoefizienten hinsichtlich des
  Effekts und der Signifikanz! Begründen Sie Ihre Antwort! \textbf{(2 Punkte)}
\item Visualisieren Sie das 95\% Konfidenzintervall! Beschriften Sie die Abbildung! \textbf{(2 Punkte)} 
\end{enumerate} 
\clearpage
% -----------------------------------------------------------------------

\section{Aufgabe \hfill (12 Punkte)}

\textit{Geben Sie grundsätzlich Formeln und Rechenweg zur Lösung der Teilaufgaben mit an!} \\[1Ex]
 

 
%% --------------------------------------------------------------------
\begin{minipage}[t]{0.5\textwidth}
\includegraphics[width = 1.3cm]{/Users/kruppajo/work/GitHub/exam/avatare/Jonas.png}
\end{minipage}
\begin{minipage}[t]{0.5\textwidth}
\hfill
\href{https://youtu.be/fB6nF4dxodA}{\includegraphics[width = 2cm]{img/youtube}}
\end{minipage}
\vspace{1ex}
%% --------------------------------------------------------------------



'Hm..., drei leere Abbildungen. Was soll ich da hetzt machen?', fragt sich Jonas. Jonas kennt sich nur begrenzt bis gar nicht mit der linearen Regresion und Korrelation aus. Dafür mit etwas anderem. Jonas liebt Stricken. Darin kann er sich wirklich verlieren und immer wieder neu begeistern. Aber das hilft hier auch nur so halb, die Aufgabe zu lösen und mehr über den Korrelationskoeffizienten zu erfahren.
\vspace{2Ex}



{\centering \includegraphics[width=\maxwidth]{img/correlation-01-1} 

}




\vspace{2Ex}

Leider kennt sich Jonas mit der Korrelationsanalyse und der linearen Regression überhaupt nicht aus. Deshalb braucht er bei der Auswertung Ihre Hilfe!

\begin{enumerate}
\item Zeichnen Sie für die $\rho$-Werte eine Gerade in die entsprechende Abbildung! \textbf{(3 Punkte)}
\item Zeichnen Sie für die $R^2$-Werte die entsprechende Punktewolke um die Gerade! \textbf{(3 Punkte)}
\item Nennen Sie die zwei Eigenschaften des Korrelationskoeffizienten! \textbf{(2 Punkte)}
\item Interpretieren Sie die $R^2$-Werte für die jeweilige Gerade! \textbf{(2 Punkte)}
\item Warum müssen Sie ein $R^2$-Wert berechnen, wenn Sie die einfachere Möglichkeit der visuellen Überprüfung haben? Begründen Sie Ihre Antwort! \textbf{(2 Punkte)}
\end{enumerate}
 
\clearpage
% -----------------------------------------------------------------------

\section{Aufgabe \hfill (12 Punkte)}

\textit{Geben Sie grundsätzlich Formeln und Rechenweg zur Lösung der Teilaufgaben mit an!} \\[1Ex]
 

 
%% --------------------------------------------------------------------
\begin{minipage}[t]{0.5\textwidth}
\includegraphics[width = 1.3cm]{/Users/kruppajo/work/GitHub/exam/avatare/Yuki.png}
\end{minipage}
\begin{minipage}[t]{0.5\textwidth}
\hfill
\href{https://youtu.be/2QJa19ZwLls}{\includegraphics[width = 2cm]{img/youtube}}
\end{minipage}
\vspace{1ex}
%% --------------------------------------------------------------------



Der Bildschirm strahlt blau in das Gesicht von Yuki. Es ist schon spät. Und das hat einen Grund. Auf seinem Second Screen läuft Matrix und Yuki schaufelt Reese's Peanut Butter Cups. Nicht effizient, aber gut. . Yuki überlegt, aber seine Gedaken sind etwas zäh. 'Was soll das hier alles bedeuten?', fragt sich Yuki. Irgendwie ist ihm nicht klar wie er $\rho$-Werte oder $R^2$-Werte abschätzen soll. Alles nicht so einfach. Eine echte Herausforderung für ihn war schon immer die Faulheit gewesen. Ein leidiges Lied. 
\vspace{2Ex}



{\centering \includegraphics[width=\maxwidth]{img/correlation-02-1} 

}




Leider kennt sich Yuki mit der Korrelationsanalyse und der linearen Regression überhaupt nicht aus. Deshalb braucht er bei der Auswertung Ihre Hilfe!

\begin{enumerate}
\item Schätzen Sie die $\rho$-Werte in den Abbildungen! \textbf{(2 Punkte)}
\item Schätzen Sie die $R^2$-Werte in den Abbildungen! \textbf{(2 Punkte)}
\item Interpretieren Sie die $R^2$-Werte für die jeweilige Gerade! \textbf{(2 Punkte)}
\item Was ist der optimale $R^2$-Wert im Kontext einer wissenschaftlichen Fragestellung? Begründen Sie Ihre Antwort an einem Beispiel! \textbf{(2 Punkte)}
\item Was ist der optimale $\rho$-Wert im Kontext einer wissenschaftlichen Fragestellung? Begründen Sie Ihre Antwort an einem Beispiel! \textbf{(2 Punkte)}
\item Erklären Sie die Aussage \textit{"Correlation does not imply causation!"} an einem Beispiel! \textbf{(2 Punkte)}
\end{enumerate} 
\clearpage
% -----------------------------------------------------------------------

\section{Aufgabe \hfill (11 Punkte)}

\textit{Geben Sie grundsätzlich Formeln und Rechenweg zur Lösung der Teilaufgaben mit an!} \\[1Ex]
 

 
%% --------------------------------------------------------------------
\begin{minipage}[t]{0.5\textwidth}
\includegraphics[width = 1.3cm]{/Users/kruppajo/work/GitHub/exam/avatare/Steffen.png}
\end{minipage}
\begin{minipage}[t]{0.5\textwidth}
\hfill
\href{https://youtu.be/dyQlYV9nOqY}{\includegraphics[width = 2cm]{img/youtube}}
\end{minipage}
\vspace{1ex}
%% --------------------------------------------------------------------



'Oh! Residuen. Die waren wichtig um zu wissen, ob eine Modellierung funktioniert hat! Da schauen wir uns dann mit der Funktion \texttt{augment()} die Werte der einzelnen Residuen an. Oder gleich den Residuenplot...da sehen wir dann... ja was eigentlich?', verkündet Steffen stolz. Leider hat Steffen vergessen wie der \Rlogo Code für den Residuenplot geht. Steffen hatte anderes im Kopf. Schon dutzende Male gesehen: Harry Potter. Aber immer noch großartig zusammen mit Oreos. Aber sowas hilft ihm natürlich hier nicht. Da schmeißt sich Steffen noch ein paar Oreos in den Mund und kaut los.

\begin{knitrout}
\definecolor{shadecolor}{rgb}{0.969, 0.969, 0.969}\color{fgcolor}\begin{table}[!h]
\centering\begingroup\fontsize{12}{14}\selectfont

\begin{tabular}{cccc}
\toprule
Chlorophyllgehalt & Durchschnittliche Regenwurmdichte & $\hat{y}$ & $\phantom{ttttt}\epsilon\phantom{ttttt}$\\
\midrule
23.6 & 9.9 & 24.7 & \\
31.6 & 13.9 & 32.1 & \\
27.3 & 9.9 & 24.8 & \\
20.7 & 6.4 & 18.3 & \\
19.5 & 8.8 & 22.7 & \\
\addlinespace
21.2 & 9.3 & 23.6 & \\
18.6 & 6.5 & 18.5 & \\
18.6 & 6.4 & 18.3 & \\
32.7 & 13.2 & 30.7 & \\
\bottomrule
\end{tabular}
\endgroup{}
\end{table}

\end{knitrout}

Leider kennt sich Steffen mit der linearen Regression überhaupt nicht aus. Deshalb braucht er bei der Auswertung Ihre Hilfe!

\begin{enumerate}
  \item Formulieren Sie die wissenschaftliche Fragestellung! \textbf{(1 Punkt)}
\item Ergänzen Sie die Werte der Residuen $\epsilon$ in der obigen Tabelle! \textbf{(2 Punkte)}
\item Zeichnen Sie den Boxplot der Residuen $\epsilon$. Beschriften Sie die Abbildung! \textbf{(2 Punkte)}
\item Zeichnen Sie den Residualplot. Beschriften Sie die Abbildung! \textbf{(2 Punkte)}
\item Gibt es auffällige Werte anhand des Residualplots? Begründen Sie Ihre Antwort! \textbf{(2 Punkte)}
\item Erklären Sie die Eigenschaft eines statistischen Modells, welche mit dem Residualplot überprüft wird! Begründen Sie Ihre Antwort anhand einer Visualisierung! \textbf{(2 Punkte)}
\end{enumerate}
 
\clearpage
% -----------------------------------------------------------------------

\section{Aufgabe \hfill (12 Punkte)}

\textit{Geben Sie grundsätzlich Formeln und Rechenweg zur Lösung der Teilaufgaben mit an!} \\[1Ex]
 

 
%% --------------------------------------------------------------------
\begin{minipage}[t]{0.5\textwidth}
\includegraphics[width = 1.3cm]{/Users/kruppajo/work/GitHub/exam/avatare/Jessica.png}\hspace{-4mm}\includegraphics[width = 1.3cm]{/Users/kruppajo/work/GitHub/exam/avatare/Yuki.png}
\end{minipage}
\begin{minipage}[t]{0.5\textwidth}
\hfill
\href{https://youtu.be/kHmfEmU6lrk}{\includegraphics[width = 2cm]{img/youtube}}
\end{minipage}
\vspace{1ex}
%% --------------------------------------------------------------------



Yuki hat einen Versuch in einer Klimakammer mit Brokkoli duchgeführt. Soweit so gut. Dann war sie bei ihrer Betreuerin. Leider war der Schritt nicht so hilfreich.  Yuki und die Faulheit, eine unendliche Geschichte mit kniffeligen Wendungen. Aber es muss ja weitergehen. Yuki hatte dann in ihrer Abschlusarbeit einfach zu viele Endpunkte gemessen und ist jetzt vollkommen durcheinander, welche Analyse sie nun wie rechnen soll. Naja, dann heißt es jetzt eben London Grammar aufdrehen und darüber nachdenken, was hier eigentlich gemacht wurde. Yuki fängt einfach an und nimmt den ersten Endpunkt Anzahl Läsionen auf den Blättern. Dann kann sie sich voran arbeiten. Später dann noch raus um zu Boldern um mal zu entspannen und vielleicht ist Jessica auch da. Wäre toll.

\vspace{1Ex}

Leider kennt sich Yuki mit dem Kontext der linearen Regression überhaupt nicht aus. Deshalb braucht sie bei der Auswertung Ihre Hilfe!

\begin{enumerate}
  \item Formulieren Sie die wissenschaftliche Fragestellung! \textbf{(1 Punkt)}
\item Zeichen Sie die Zeile des Regressionskreuzes für den Endpunkt mit \underline{drei} Feldern! Beschriften Sie die Abbildung! \textbf{(4 Punkte)}
\item Ergänzen Sie die entsprechenden statistische Methoden zur Analyse in jedem Feld! \textbf{(2 Punkte)}
\item Formulieren Sie die Nullhypothese für die statistische Methode in jedem Feld! \textbf{(2 Punkte)}
\item Ergänzen Sie die entsprechenden Funktionen in \Rlogo zur Analyse in jedem Feld! \textbf{(2 Punkte)}
\item Welchen Effekt erhalten Sie in jedem Feld? Geben Sie ein Beispiel! \textbf{(2 Punkte)}
\end{enumerate} 
\clearpage
% -----------------------------------------------------------------------
\part{Experimentelles Design}
% -----------------------------------------------------------------------

\section{Aufgabe \hfill (16 Punkte)}


 
%% --------------------------------------------------------------------
\begin{minipage}[t]{0.5\textwidth}
\includegraphics[width = 1.3cm]{/Users/kruppajo/work/GitHub/exam/avatare/Alex.png}\hspace{-4mm}\includegraphics[width = 1.3cm]{/Users/kruppajo/work/GitHub/exam/avatare/Steffen.png}\hspace{-4mm}\includegraphics[width = 1.3cm]{/Users/kruppajo/work/GitHub/exam/avatare/Yuki.png}
\end{minipage}
\begin{minipage}[t]{0.5\textwidth}
\hfill
\href{https://youtu.be/wJqsNV1hOW8}{\includegraphics[width = 2cm]{img/caution}}
\end{minipage}
\vspace{1ex}
%% --------------------------------------------------------------------



Das Minischwein macht mal wieder Randale in Yukis Zimmer und rennt davon! Alex und Steffen sind bei Yuki in im Oldenburger Land wo der neue, bessere Versuch stattfinden soll. Dabei soll in einem Stallversuch im Oldenburger Land mit Schweinen durchgeführt werden. Auf dem Tisch stapeln sich Reese's Peanut Butter Cups aus Vollkorndinkelmehl. Eine Spezialtät der Komune hier. Alex hasst Vollkorn wie Oreos geliebt werden. In dem neuen Versuch geht es um den Zusammenhang zwischen der Behandlung Lüftungssystem ($keins$, $storm$, $tornado$ und $thunder$) und dem Messwert Protein/Fettrate [\%/kg]. Immerhin ist der Messwert normalverteilt, was einges einfacher macht. Was es nicht so einfacher macht ist, dass Steffen als zusätzliche Herausforderung noch die Romantik mitgebracht hat. Daher entscheiden sich alle drei für ein \textit{Randomized complete block design (RCBD) mit drei Blöcken}. 'Naja, so viel einfacher ist es dann doch nicht...', merkt Steffen an und sucht das Minischwein.\\

Leider kennen sich Yuki, Alex und Steffen mit dem \textit{Randomized complete block design (RCBD) mit drei Blöcken} überhaupt nicht aus. Deshalb brauchen die Drei bei der Erstellung Ihre Hilfe!

\begin{enumerate}
  \setcounter{enumi}{0}
  \item Formulieren Sie die wissenschaftliche Fragestellung! \textbf{(1 Punkt)}
  \item Formulieren Sie das statistische Hypothesenpaar! \textbf{(1 Punkt)}
  \item Skizzieren Sie das faktorielle Versuchsdesign! \textbf{(3 Punkte)}
  \item Skizzieren Sie eine Datentabelle für das faktorielle Versuchsdesign in \Rlogo! \textbf{(2 Punkte)}
  \item Erstellen Sie das statistische Modell in der in \Rlogo üblichen Schreibweise für eine ANOVA! Skizzieren Sie die notwendige Funktionen in \Rlogo! \textbf{(3 Punkte)}
  \item Skizzieren Sie die weitere Datenanalyse hinsichtlich eines multiplen Gruppenvergleiches! \textbf{(2 Punkte)}
  \item Skizzieren Sie eine mögliche Abbildung im Kontext der wissenschaftlichen Fragestellung! Beschriften Sie die Abbildung! \textbf{(2 Punkte)}
  \item Ergänzen Sie zu der Abbildung ein mögliches Ergebnis des multiplen Gruppenvergleichs! Begründen Sie Ihre Antwort! \textbf{(2 Punkte)}
\end{enumerate}


 
\clearpage
% -----------------------------------------------------------------------

\section{Aufgabe \hfill (20 Punkte)}


 
%% --------------------------------------------------------------------
\begin{minipage}[t]{0.5\textwidth}
\includegraphics[width = 1.3cm]{/Users/kruppajo/work/GitHub/exam/avatare/Alex.png}\hspace{-4mm}\includegraphics[width = 1.3cm]{/Users/kruppajo/work/GitHub/exam/avatare/Mark.png}\hspace{-4mm}\includegraphics[width = 1.3cm]{/Users/kruppajo/work/GitHub/exam/avatare/Yuki.png}
\end{minipage}
\begin{minipage}[t]{0.5\textwidth}
\hfill
\href{https://youtu.be/wJqsNV1hOW8}{\includegraphics[width = 2cm]{img/caution}}
\end{minipage}
\vspace{1ex}
%% --------------------------------------------------------------------



Neuer Versuch neues Glück! Yuki und Mark sind bei Alex um sich Hilfe für eine Versuchsplanung in \Rlogo zu holen. Im Hintergrund läuft viel zu laut Matrix. Daher hat das Minischwein schon lange reißaus genommen. In dem neuen Versuch geht es um den Zusammenhang zwischen der Behandlung Genotypen ($AA$, $AB$ und $BB$) sowie Substrattypen ($torf$, $70p30n$) sowie drei Blöcken und dem Messwert Frischegewicht [kg/ha] in Spargel. Der Versuch soll in einem Freilandversuch in der Uckermark durchgeführt werden. Immerhin ist der Messswert normalverteilt, was einges einfacher macht. Was es nicht so einfacher macht ist, dass Yuki noch als zusätzliche Herausforderung etwas anderes umtreibt: die Gefälligkeit. Im ersten Schritt überlegt Alex ein komplexeres experimentelles Design zu probieren. Daher entscheiden sich alle drei für ein \textit{Split plot design oder auch Spaltanlage}. Ob es das jetzt einfacher macht?\\

Leider kennen sich Alex, Yuki und Mark mit dem \textit{Split plot design oder auch Spaltanlage} überhaupt nicht aus. Deshalb brauchen die Drei bei der Erstellung Ihre Hilfe!

\begin{enumerate}
  \setcounter{enumi}{0}
  \item Formulieren Sie die wissenschaftliche Fragestellung! \textbf{(1 Punkt)}
  \item Formulieren Sie die statistische Hypothesenpaare! \textbf{(2 Punkte)}
  \item Skizzieren Sie das faktorielle Versuchsdesign! \textbf{(3 Punkte)}
  \item Skizzieren Sie eine Datentabelle für das faktorielle Versuchsdesign in \Rlogo! \textbf{(2 Punkte)}
  \item Erstellen Sie das statistische Modell in der in \Rlogo üblichen Schreibweise für eine ANOVA! Skizzieren Sie die notwendige Funktionen in \Rlogo! \textbf{(4 Punkte)}
  \item Skizzieren Sie die weitere Datenanalyse hinsichtlich eines multiplen Gruppenvergleiches! \textbf{(2 Punkte)}
  \item Skizzieren Sie eine mögliche Abbildung im Kontext der wissenschaftlichen Fragestellung! Beschriften Sie die Abbildung! \textbf{(3 Punkte)}
  \item Ergänzen Sie zu der Abbildung ein mögliches Ergebnis des multiplen Gruppenvergleichs! Welche Annahme hinsichtlich der Modellierung haben Sie getroffen? Begründen Sie Ihre Antwort! \textbf{(3 Punkte)}
\end{enumerate} 
\clearpage
% -----------------------------------------------------------------------
\part{Programmieren in R}
% -----------------------------------------------------------------------

\section{Aufgabe \hfill (9 Punkte)}



 
%% --------------------------------------------------------------------
\begin{minipage}[t]{0.5\textwidth}
\includegraphics[width = 1.3cm]{/Users/kruppajo/work/GitHub/exam/avatare/Alex.png}
\end{minipage}
\begin{minipage}[t]{0.5\textwidth}
\hfill
\href{https://www.youtube.com/playlist?list=PLe51bCp9JvEFUnFqaJG5aRmON9i1ZbOYC}{\includegraphics[width = 2cm]{img/youtube}}
\end{minipage}
\vspace{1ex}
%% --------------------------------------------------------------------



Alex muss seinem Projektbericht mit \Rlogo arbeiten. Deshalb sitzt er jetzt mit Ihnen zusammen und hat einige Fragen zu den Grundlagen in \Rlogo an Sie! Na dann wollen Sie mal helfen. Immerhin will sein Betreuer, dass \Rlogo genutzt wird.\\[1Ex]

Alex: \textit{Der Pipe-Operator wird sehr häufig genutzt. Wie sieht der aus und wie funktioniert der an einem Beispiel?} \textbf{(1 Punkt)}\\[1ex]
Sie antworten:\\[2Ex]

Alex: \textit{Wir brauchen recht häufig die Tilde ($\sim$) in R. Wo wird die nochmal angewandt und genutzt?} \textbf{(1 Punkt)}\\[1ex]
Sie antworten:\\[2Ex]

Alex: \textit{Hä? Warum ändert sich nichts an meinen Daten? In R sehe ich doch die Änderungen aber irgendwie speicher R meine Änderungen meines Datensatzes ab. Was ist da los?} \textbf{(1 Punkt)}\\[1ex]
Sie antworten:\\[2Ex]

Alex: \textit{Was ist der Unterschied zwischen \texttt{library()} und \texttt{Packages} und warum brauche ich sowas in \Rlogo?} \textbf{(1 Punkt)}\\[1ex]
Sie antworten:\\[2Ex]

Alex: \textit{Ich habe die Namen der beiden \Rlogo Pakete vergessen, die wir eigentlich immer laden. Wie heißen die noch gleich?} \textbf{(1 Punkt)}\\[1ex]
Sie antworten:\\[2Ex]

Alex: \textit{Was war eigentlich nochmal ein Vorteil von der Nutzng von \Rlogo?} \textbf{(1 Punkt)}\\[1ex]
Sie antworten:\\[2Ex]

Alex: \textit{Was ist eigentlich ein Faktor in \Rlogo?} \textbf{(1 Punkt)}\\[1ex]
Sie antworten:\\[2Ex]

Alex: \textit{Was ist der Unterschied zwischen dem RStudio und R?} \textbf{(1 Punkt)}\\[1ex]
Sie antworten:\\[2Ex]

Alex: \textit{Ich verstehe den Zuweisungs-Operator nicht. Wie sieht der aus und was macht der? Gebe mal ein Beispiel!} \textbf{(1 Punkt)}\\[1ex]
Sie antworten:\\[2Ex] 
\clearpage
% -----------------------------------------------------------------------

\section{Aufgabe \hfill (9 Punkte)}



 
%% --------------------------------------------------------------------
\begin{minipage}[t]{0.5\textwidth}
\includegraphics[width = 1.3cm]{/Users/kruppajo/work/GitHub/exam/avatare/Paula.png}
\end{minipage}
\begin{minipage}[t]{0.5\textwidth}
\hfill
\href{https://www.youtube.com/playlist?list=PLe51bCp9JvEFUnFqaJG5aRmON9i1ZbOYC}{\includegraphics[width = 2cm]{img/youtube}}
\end{minipage}
\vspace{1ex}
%% --------------------------------------------------------------------



'Unter den Blinden ist der Einäuge König!', ruft Ihnen Paula entgegen. Das können Sie schon nicht mehr hören. Nur weil Sie einmal gesagt haben, dass Sie sich schonmal mit \Rlogo beschäftigt haben, stehen hier alle Schlange. Aber gut, das hat Sie dann doch vorangebracht. Leider kennt sich Paula auch wieder überhaupt nicht mit \Rlogo aus aber ihr Betreuer möchte gerne, dass die Auswertung in \Rlogo gemacht wird. Da müssen Sie dann wohl mal nochmal ran und helfen.\\[1Ex]

Paula fragt: \textit{Was muss ich bei der Eingabe eines Datums in Excel beachten, wenn ich später die Exceldatei in R einlesen will? Wie lautet das Format? \textbf{(1 Punkt)}}\\[1ex]
Sie antworten:\\[2Ex]

Paula fragt: \textit{Wozu war nochmal die Funktion \texttt{mutate()} gut?  \textbf{(1 Punkt)}}\\[1ex]
Sie antworten:\\[2Ex]

Paula fragt: \textit{Ich will eine ANOVA in R rechnen. Dazu brauche ich zwei Funktionen. Welche waren das noch gleich und wie war die Reihenfolge? \textbf{(1 Punkt)}}\\[1ex]
Sie antworten:\\[2Ex]

Paula fragt: \textit{Ich hatte mir eine Analogie für das R Paket \texttt{\{ggplot\}} gemerkt. Wie war noch gleich die Analogie und das damit verbundene Prinzip von \texttt{\{ggplot\}}? \textbf{(2 Punkte)}}\\[1ex]
Sie antworten:\\[2Ex]

Paula fragt: \textit{Was muss ich bei der Benennung von Spalten in Excel beachten? \textbf{(1 Punkt)}}\\[1ex]
Sie antworten:\\[2Ex]

Paula fragt: \textit{Ich möchte ein CLD erstellen. Welche Funktionen muss ich in welcher Reihenfolge nutzen? \textbf{(2 Punkte)}}\\[1ex]
Sie antworten:\\[2Ex]

Paula fragt: \textit{Die Funktion \texttt{emmeans()} erlaubt es den Faktor $f_1$ getrennt in jedem Level des Faktors $f_2$ auszuwerten. Wie mache ich das? \textbf{(1 Punkt)}}\\[1ex]
Sie antworten:\\[2Ex]



 
\clearpage
% -----------------------------------------------------------------------
\part{Forschendes Lernen}

Das forschende Lernen basiert zum einen auf den folgenden wissenschaftlichen Veröffentlichungen. Für die Prüfung wird die vertiefende Kenntnis der folgenden Veröffentlichungen vorausgesetzt.\\

\textit{In der Prüfung erhalten Sie einen Auszug der wissenschaftlichen Veröffentlichung. Für die Einarbeitung in die Veröffentlichung ist in der Prüfung \underline{ausdrücklich} keine Zeit vorgesehen.}

\begin{itemize}[noitemsep]
\item Sánchez, M., Velásquez, Y., González, M., \& Cuevas, J. (2022). Hoverfly pollination enhances yield and fruit quality in mango under protected cultivation. Scientia Horticulturae, 304, 111320. [\href{https://www.sciencedirect.com/science/article/pii/S0304423822004411}{Link}]
% \item Salinas, I., Hueso, J. J., Força Baroni, D., \& Cuevas, J. (2023). Plant growth, yield, and fruit size improvements in ‘Alicia’papaya multiplied by grafting. Plants, 12(5), 1189. [\href{https://www.mdpi.com/2223-7747/12/5/1189}{Link}]
\item Petersen, F., Demann, J., Restemeyer, D., Olfs, H. W., Westendarp, H., Appenroth, K. J., \& Ulbrich, A. (2022). Influence of light intensity and spectrum on duckweed growth and proteins in a small-scale, re-circulating indoor vertical farm. Plants, 11(8), 1010. [\href{https://www.mdpi.com/2223-7747/11/8/1010}{Link}]
\item Selle, P. H., Cadogan, D. J., Li, X., \& Bryden, W. L. (2010). Implications of sorghum in broiler chicken nutrition. Animal Feed Science and Technology, 156(3-4), 57-74. [\href{https://www.sciencedirect.com/science/article/pii/S0377840110000209}{Link}]
\item Wu, G., Knabe, D. A., \& Kim, S. W. (2004). Arginine nutrition in neonatal pigs. The Journal of Nutrition, 134(10), 2783S-2790S. [\href{https://www.sciencedirect.com/science/article/pii/S0022316623031279}{Link}]
\end{itemize}

Das forschende Lernen basiert zum anderen auf den folgenden wissenschaftlichen Datensätzen und deren vertiefende Analyse werden als bekannt vorausgesetzt. Die Teilaufgaben der Aufgaben stellen nur eine zufällige Auswahl an möglichen Fragen dar. Die Datensätze werden über ILIAS bereitgestellt.\\

\textit{In der Prüfung erhalten Sie \underline{keinen Auszug} aus den wissenschaftlichen Daten. Die Datensätze werden als \underline{bekannt} in der Prüfung vorgesetzt. Sie haben sich vorab Notizen und Anmerkungen gemacht.}

\begin{itemize}[noitemsep]
\item bar
\end{itemize}

\clearpage
% -----------------------------------------------------------------------

\section{Aufgabe \hfill (20 Punkte)}

\textit{Geben Sie grundsätzlich Formeln und Rechenweg zur Lösung der Teilaufgaben mit an!} \\[1Ex]
 

 
%% --------------------------------------------------------------------
\begin{minipage}[t]{0.5\textwidth}
\includegraphics[width = 1.3cm]{/Users/kruppajo/work/GitHub/exam/avatare/Alex.png}
\end{minipage}
\begin{minipage}[t]{0.5\textwidth}
\hfill
\href{https://youtu.be/C9skfFRTHhI}{\includegraphics[width = 2cm]{img/caution}}
\end{minipage}
\vspace{1ex}
%% --------------------------------------------------------------------



Vor dem Start der eigenen Arbeit möchte sein Betreuer, dass Alex einmal die wissenschaftliche Veröffentlichung \textit{Petersen, F., et al. (2022). Influence of light intensity and spectrum on duckweed growth and proteins in a small-scale, re-circulating indoor vertical farm} sinnvoll zusammenfasst. Dann würde die eigene Arbeit auch leichter von der Hand gehen und Alex hätte dann schon eine Vorlage. 'Das ist jetzt aber umfangreicher als gedacht!', schnauft er und runzelt die Stirn. Im Hintergrund spielt viel zu leise Abba. Das wird dann vermutlich heute Abend nichts mehr mit seinem Hobby Starcraft Die Katze schaut mitleidig.\\

Leider kennt sich Alex mit dem Lesen einer wissenschaftlichen Veröffentlichung mit Fokus auf die Statistik überhaupt nicht aus. Deshalb braucht er bei der Erstellung Ihre Hilfe! Glücklicherweise kennen Sie die wissenschaftliche Veröffentlichung schon im Detail und können sofort helfen.

\begin{enumerate}
  \setcounter{enumi}{0}
  \item Erläutern Sie die wissenschaftliche Fragestellung der wissenschaftlichen Veröffentlichung anhand des OCAR Prinzips nach Schimel (2012)\footnote{Schimel, J. (2012). Writing science: how to write papers that get cited and proposals that get funded. OUP USA.} \textbf{(4 Punkte)}
  \item Nennen Sie die untersuchten Endpunkte in der wissenschaftlichen Veröffentlichung! Wie lautet der primäre Endpunkt? \textbf{(2 Punkte)} 
\item Erstellen Sie das statistische Modell in der in \Rlogo üblichen Schreibweise! \textbf{(2 Punkte)}
  \item Nennen Sie eine Auswahl an bedeutenden statistischen Maßzahlen in der wissenschaftlichen Veröffentlichung! \textbf{(1 Punkt)}
  \item Interpretieren Sie die Hauptaussage der wissenschaftlichen Veröffentlichung hinsichtlich der Signifkanz für den primären Endpunkt! \textbf{(2 Punkte)}
  \item Interpretieren Sie die Hauptaussage der wissenschaftlichen Veröffentlichung hinsichtlich der Effektstärke für den primären Endpunkt! \textbf{(2 Punkte)}
  \item Diskutieren Sie die ökonomische Relevanz der Hauptaussage der wissenschaftlichen Veröffentlichung im Bezug auf Signifikanz und Effektstärke für den primären Endpunkt! \textbf{(1 Punkt)}
  \item Skizzieren Sie für den primären Endpunkt den sich ergebenden Datensatz in \Rlogo für eine ausgewählte Abbildung! \textbf{(2 Punkte)}
\item Skizzieren Sie einen möglichen Versuchsplan für den primären Endpunkt! \textbf{(2 Punkte)}
  \item Schätzen Sie die benötigte Fallzahl für ein zukünftiges Experiment anhand der Ergebnisse in der wisenschaftlichen Veröffentlichung für den primären Endpunkt! \textbf{(2 Punkte)}
\end{enumerate} 
\clearpage
% -----------------------------------------------------------------------

\section{Aufgabe \hfill (20 Punkte)}

\textit{Geben Sie grundsätzlich Formeln und Rechenweg zur Lösung der Teilaufgaben mit an!} \\[1Ex]
 

 
%% --------------------------------------------------------------------
\begin{minipage}[t]{0.5\textwidth}
\includegraphics[width = 1.3cm]{/Users/kruppajo/work/GitHub/exam/avatare/Jessica.png}
\end{minipage}
\begin{minipage}[t]{0.5\textwidth}
\hfill
\href{https://youtu.be/C9skfFRTHhI}{\includegraphics[width = 2cm]{img/caution}}
\end{minipage}
\vspace{1ex}
%% --------------------------------------------------------------------



Unter einem langen Schnaufen starrt Jessica auf den wissenschaftlichen Datensatz \textit{data3} in ihrem Laptop. 'Worum geht es denn eigentlich in diesem Datensatz?', fragt sie sich kopfschüttelnd und mampft noch ein paar Schokobons. Jessica soll die Datentabelle nutzen um das eigene Experiment zu planen und eine Blaupause zu haben. Als eine Vorlage sozusagen, die sie nur noch ausfüllen muss. Daher möchte ihr Betreuer, dass sie einmal die Daten sinnvoll zusammenfasst. Das sollte dann doch etwas aufwendiger werden. Das wird dann vermutlich heute Abend nichts mehr mit Herr der Ringe\\

Leider kennt sich Jessica mit der Analyse eines wissenschaftlichen Datensatzes überhaupt nicht aus. Deshalb braucht sie bei der Auswertung Ihre Hilfe! Glücklicherweise kennen Sie den wissenschaftlichen Datensatz aus Ihren eigenen Analysen schon im Detail und können sofort helfen.

\begin{enumerate}
  \setcounter{enumi}{0}
  \item Formulieren Sie die wissenschaftliche Fragestellung des Datensatzes in Form einer PowerPoint Folie! \textbf{(2 Punkte)}
  \item Nennen Sie zwei Besonderheiten des Datensatzes! Begründen Sie Ihre Antwort! \textbf{(2 Punkte)}
  \item Nennen Sie die untersuchten Endpunkte in dem Datensatz! Wie lautet der primäre Endpunkt für die Auswertung? \textbf{(2 Punkte)}
  \item Skizzieren Sie die großen Analysebereiche der Statistik! Beschriften Sie die Abbildungen! \textbf{(2 Punkte)}
  \item In welchen der großen Analysebereiche der Statistik fällt die Auswertung des primären Endpunktes? Begründen Sie Ihre Antwort! \textbf{(2 Punkte)}
  \item Skizzieren Sie eine ikonische Abbildung für den primären Endpunkt im Kontext der wissenschaftlichen Fragestellung! \textbf{(2 Punkte)}
  \item Erstellen Sie das statistische Modell in der in \Rlogo üblichen Schreibweise! \textbf{(2 Punkte)}
  \item Skizzieren Sie die Datenanalyse hinsichtlich der Signifkanz für den primären Endpunkt! \textbf{(2 Punkte)}
  \item Skizzieren Sie die Berechnung der Effektstärke für den primären Endpunkt! \textbf{(2 Punkte)}
  \item Skizzieren Sie einen möglichen Versuchsplan für den primären Endpunkt! \textbf{(2 Punkte)}
\end{enumerate} 
\clearpage
% -----------------------------------------------------------------------
\part{Mathematik}
% -----------------------------------------------------------------------  

\section{Aufgabe \hfill (10 Punkte)}

\textit{Geben Sie grunds{\"a}tzlich Formeln und Rechenweg zur L{\"o}sung der
  Teilaufgaben mit an!} \\[1Ex]

%% --------------------------------------------------------------------
\hfill\href{https://youtu.be/Fu8kN0Uj13Y}{\includegraphics[width =
  2cm]{img/youtube}} %%youtube
\hspace{2Ex}
%% --------------------------------------------------------------------

\paragraph{Herodot – der Schimmel aus Ivenack}

W{\"a}hrend der Besetzung Mecklenburgs durch die Franzosen kamen Napoleon die
Geschichten des ber{\"u}hmten Apfelschimmels Herodot aus Ivenack zu
Geh{\"o}r. Herodot lief zwar niemals Rennen, war aber eines der ber{\"u}hmtesten
Pferde dieser Zeit. Napoleon selbst gab den Auftrag, diesen
Schimmel durch die Armee nach Frankreich zu bringen. Der Legende nach
sollen Arbeiter den Schimmel im hohlen Stamm einer 1000-j{\"a}hrigen Eiche aus Ivenack vor
den Franzosen versteckt haben. Doch Herodot verriet sein Versteck durch
lautes Wiehern, woraufhin die franz{\"o}sische Armee den Schimmel
beschlagnahmte und nach Frankreich f{\"u}hrte\footnote{Die Quelle der Inspiration
  für die Aufgabe war eine Fahrt an die Ostsee und folgender Artikel:
  \href{https://www.wald-mv.de/landingpage/ivenacker-eichen/}{Entdecke das erste Nationale Naturmonument Deutschlands - Ivenacker Eichen und Hutewald}}. \\



\textit{Forschungsfrage: "Konnten die Ivenacker den Apfelschimmel Herodot
  vor dem Zugriff von Napoleon in der 1000-j{\"a}hrigen Eiche verstecken?"} \\

Gehen Sie von einem radialen Wachstum der 1000-j{\"a}hrigen Eiche von
$0.8mm$ pro Jahr aus. Es ist bekannt, dass die Eiche im
Jahr 2022 einen Umfang von $14m$ in Brusth{\"o}he hatte.

\begin{enumerate}
\item Wie gro{\ss} war der Durchmesser in $m$ der Eiche im Jahr $1815$ als
  Herodot in der Eiche versteckt werden sollte?
  \textbf{(2 Punkte)}
\item Skizzieren Sie in einer Abbildung einen linearen Zusammenhang und einen
exponentiellen Zusammenhang f{\"u}r das Wachstum der 1000-j{\"a}hrigen Eiche. Erkl{\"a}ren Sie die
Auswirkungen der Entscheidung f{\"u}r linear oder exponentiell auf Ihre
Berechnungen! \textbf{(2 Punkte)}
\end{enumerate}
 
Herodot hatte eine Schulterh{\"o}he von $180$cm, eine Breite von
$95$cm sowie eine L{\"a}nge von  $240$cm.

\begin{enumerate}
  \setcounter{enumi}{2}
\item Berechnen Sie das effektive Volumen von Herodot in $m^3$, welches
  Herodot in der 1000-j{\"a}hrigen Eiche einnehmen w{\"u}rde! \textbf{(2 Punkte)}
\end{enumerate}

Es wurde berichtet, dass sich Herodot in der 1000-j{\"a}hrigen Eiche
$m{"u}hsam$ um die eigene Achse drehen konnte.

\begin{enumerate}
  \setcounter{enumi}{3}
\item Berechnen Sie die Dicke der Eichenwand in $cm$! Verdeutlichen Sie Ihre
  Berechnungen an einer aussagekr{\"a}ftigen Skizze f{\"u}r Pferd und Eiche! \textbf{(2 Punkte)} 
\item Unter einer Dicke der Eichenwand von $20cm$ bricht
  die Eiche zusammen. Beantworten Sie die Forschungsfrage! Begr{\"u}nden Sie
  Ihre Antwort! \textbf{(2 Punkte)} 
\end{enumerate}
 
\clearpage
% ----------------------------------------------------------------------- 

\section{Aufgabe \hfill (10 Punkte)}

\textit{Geben Sie grunds{\"a}tzlich Formeln und Rechenweg zur L{\"o}sung der
  Teilaufgaben mit an!} \\[1Ex]

%% --------------------------------------------------------------------
\hfill\href{https://youtu.be/57B-yYoFSk0}{\includegraphics[width =
  2cm]{img/youtube}} %%youtube
\hspace{2Ex}
%% --------------------------------------------------------------------

\paragraph{Von T{\"o}pfen auf Tischen}



In einem Experiment wollen Sie die Wuchsh{\"o}he von 200
Sonnenblumen bestimmen. Bevor Sie {\"u}berhaupt mit dem Experiment beginnen
k{\"o}nnen, gibt es aber ein paar Absch{\"a}tzungen {\"u}ber die Kosten und den Aufwand
zu treffen. Zum einen m{\"u}ssen Sie die Sonnenblumen einpflanzen und m{\"u}ssen
daf{\"u}r Substrat bestellen. Zum anderen m{\"u}ssen Sie die Sonnenblumen auch
bewegen und in ein Gew{\"a}chshaus platzieren. Die T{\"o}pfe f{\"u}r die Keimung haben
einen Durchmesser von 10cm und eine H{\"o}he von 9cm. Der
Kubikmeterpreis f{\"u}r Torf liegt bei 290 EUR.

\begin{enumerate}
\item Skizzieren Sie den Versuchsplan auf \textit{vier} Tischen im
  Gew{\"a}chshaus! \textbf{(2 Punkte)}
\item Berechnen Sie die ben{\"o}tigte Anzahl an Pflanzt{\"o}pfen, wenn Sie
  Randpflanzen mit ber{\"u}cksichtigen wollen! \textbf{(1 Punkt)}
\item Welche $Tisch$fl{\"a}che in $m^2$ gegeben der Anzahl an
  Pflanzt{\"o}pfen inklusive Randpflanzen ben{\"o}tigen Sie im Gew{\"a}chshaus am
  Anfang der Keimungsphase?  \textbf{(3 Punkte)}
\item Berechnen Sie die ben{\"o}tigte Menge an Torf in Liter $l$, die Sie f{\"u}r
  das Bef{\"u}llen der Pflanzt{\"o}pfe ben{\"o}tigen! Gehen Sie von \textit{einem
    Zylinder} f{\"u}r die Pflanzt{\"o}pfe aus!  \textbf{(3 Punkte)}
\item Berechnen Sie die Kosten in EUR f{\"u}r Ihre Torfbestellung! \textbf{(1
    Punkt)}
\end{enumerate}



 
\clearpage
% ----------------------------------------------------------------------- 

\section{Aufgabe \hfill (10 Punkte)}

\textit{Geben Sie grunds{\"a}tzlich Formeln und Rechenweg zur L{\"o}sung der
  Teilaufgaben mit an!} \\[1Ex]

%% --------------------------------------------------------------------
\hfill\href{https://youtu.be/aBxLkdF-c4M}{\includegraphics[width =
  2cm]{img/youtube}} %%youtube
\hspace{2Ex}
%% --------------------------------------------------------------------

\paragraph{Solar- \& Biogasanlagen}



Um die Energiekosten Ihres Betriebes zu senken, wollen Sie eine Solaranlage
auf den Rinderstall montieren lassen. Sie messen Ihren Stall und finden
folgende Ma{\ss}e wieder. Die vordere Seite des Rinderstall hat eine H{\"o}he
$h_v$ von $6.5m$. Die hintere Seite des Rinderstall hat eine
H{\"o}he $h_b$ von $8m$. Der Rinderstall hat eine Tiefe $t$ von
$15m$ und eine Breite $b$ von $50m$.

\begin{enumerate}
\item Skizzieren Sie den Rinderstall auf dem die Solaranlage montiert
  werden soll! Erg{\"a}nzen Sie die Angaben f{\"u}r die H{\"o}hen $h_v$, $h_b$, die
  Tiefe $t$ und die Breite $b$ des Stalls!  \textbf{(2 Punkte)}
\item Berechnen Sie die Fl{\"a}che der schr{\"a}gen, neuen Solaranlage auf dem
  Rinderstall! \textbf{(3 Punkte)}
\end{enumerate}

Ebenfalls planen Sie eine neue Biogasanlage f{\"u}r Ihren Betrieb. Der neue
Methantank hat einen Radius $r$ von $1.2m$. Leider gibt es ein
paar bauliche Beschr{\"a}nkungen auf dem Grundst{\"u}ck. Ihr Fundament des
zylindrischen Methantanks kann nur ein Gewicht von maximal
$15t$ aushalten bevor der Tank wegbricht. Sie rechnen eine
Sicherheitstoleranz von $10\%$ ein beinhaltend das Gewicht des
Methantanks. In fl{\"u}ssiger Form hat
Methan bei $-80^\circ\text{C}$ eine Dichte von
$240kg/m^3$. Bei $-100^\circ\text{C}$ hat Methan eine Dichte
von $280kg/m^3$. Sie betrieben Ihre Anlage bei
$-90^\circ\text{C}$.

\begin{enumerate}
  \setcounter{enumi}{2}
\item Extrapolieren Sie die effektive Dichte des Methans in Ihrem
  Methantank! Welche Annahme haben Sie getroffen? \textbf{(1 Punkt)}
\item Berechnen Sie wie viel Kubikmeter $m^3$ Sie in den Methantank f{\"u}llen
  k{\"o}nnen, bevor das Fundament nachgibt! \textbf{(2 Punkte)}
\item Berechnen Sie die maximale H{\"o}he $h_{max}$ in $m$ f{\"u}r den gef{\"u}llten
  Methantank mit dem Radius $r$, bevor das Fundament wegbricht! \textbf{(2
    Punkte)}
\end{enumerate}

 
\clearpage
% -----------------------------------------------------------------------

\section{Aufgabe \hfill (10 Punkte)}

\textit{Geben Sie grunds{\"a}tzlich Formeln und Rechenweg zur L{\"o}sung der
  Teilaufgaben mit an!} \\[1Ex]

%% --------------------------------------------------------------------
\hfill\href{https://youtu.be/https://youtu.be/k2G52hMIfqk}{\includegraphics[width = 2cm]{img/youtube}} %%youtube
\hspace{2Ex}
%% --------------------------------------------------------------------

%% --------------------------------------------------------------------
{\tiny\textbf{Stichworte:} Riesenfaultier $\bullet$ Evolution der Avocado $\bullet$ Bluetooth $ \bullet$ Blauzahn $\bullet$ Colonia Dignidad $\bullet$ ODESSA $\bullet$ Rattenlinie $\bullet$ Adolf Eichmann}
%% --------------------------------------------------------------------


\paragraph{Aligatorenbirnen und Blaubeeren}



"'Sind Sie ein Riesenfautier oder warum kaufen Sie so viele
Aligatorenbirnen?"', spricht es hinter Ihnen. Irritiert drehen Sie sich um
und blicken in das puderrote Gesicht von einem Studienrat mit Stock. "'Wieso?"', entfährt es
Ihnen und Sie bereuen sogleich die Frage. Sofort werden Sie zu einem
Whiteboard voller roter Schnüre geschliffen und müssen folgenden
mathematischen untermauerten Argumenten im Lidl über sich ergehen
lassen. Da kommen Sie nicht mehr raus, also können Sie auch gleich
mitmachen. Das Problem liegt in Chile\footnote{Die Quelle der Inspiration
  für die Aufgabe waren folgende Reportagen:
  \href{https://www.amnesty.ch/de/ueber-amnesty/publikationen/magazin-amnesty/2021-3/bis-zum-letzten-tropfen}{"`Bis
  zum letzten Tropfen"' in AMNESTY – Magazin der Menschenrechte vom August
  2021} und
\href{https://www.welthungerhilfe.de/welternaehrung/rubriken/klima-ressourcen/wassernot-in-chile-eine-folge-der-privatisierung}{"`Wasserknappheit
  in Chile: Eine Folge der Privatisierung?"' in Die Welternährung dem
  Fachjournal der Welthungerhilfe vom April 2022.}}. Tja, die Deutschen und Südamerika.\\

Zuerst werden Ihre Fähigkeiten getestet, der Mathematik folgen zu können. Oder berechnen Sie gerade den Einkauf von einem Studienrat mit Stock?\\

\begin{enumerate}
\item Wenn 4 Blaubeerschalen 7.56 Euro kosten,  wie viel kosten 8 Schalen? \textbf{(2 Punkte)}
\item Wenn Sie die 8 Blaubeerschalen gekauft haben, wie viele Aligatorbirnen zu je 0.99 EUR können Sie sich dann noch für 50 EUR leisten? \textbf{(1 Punkt)}
\end{enumerate}

Das Whiteboard beinhaltet folgende Liste mit Informationen zum Wasserverbrauch bei der Produktion von Gemüse aus Chile. Seltsam, was man so alles in einem Lidl über Gemüse erfährt.
  
\begin{itemize}[noitemsep]
\item Ein Kilo Strauchtomaten benötigt 170l Wasser. Eine Strauchtomate wiegt 90 - 125g.
\item Ein Kilo Salat benötigt 120l Wasser. Ein Salatkopf wiegt 310 - 520g.
\item Ein Kilo Avocado benötigt 1100l Wasser. Eine Avocado wiegt 130 - 400g.
\item Ein Kilo Blaubeeren benötigt 880l Wasser. Eine Blaubeere wiegt 3.2 - 3.8g.
\end{itemize}

\begin{enumerate}
  \setcounter{enumi}{2}
\item Berechnen Sie den Wasserverbrauch für die Produktion für jeweils eine Strauchtomate, einem Salat, einer Avocado und einer Blaubeeren. Stellen Sie das Ergebnis als Tabelle dar! \textbf{(3 Punkte)}
\end{enumerate}

Chile exportiert im großem Ausmaß Blaubeeren und Avocados. In dem Exportjahr 2021 blieben die Erträge von Blaubeeren mit \ensuremath{8\times 10^{4}}t in dem prognostizierten Rahmen. Die Menge steigerte sich um 6.1\%. Die Exporte für Avocados stiegen in dem gleichen Zeitraum um 22.1\% auf \ensuremath{2\times 10^{5}}t.

\begin{enumerate}
  \setcounter{enumi}{3}
\item Wie viele Liter Wasser hat Chile in dem Exportjahr 2020 exportiert? \textbf{(2 Punkte)}
\end{enumerate}

Chile ist eines der wenigen Länder der Welt, die ihr Wasser komplett privatisiert haben. Derzeit sind nur zwei Prozent des Wassers des Landes für den häuslichen Verbrauch vorgesehen. In den Dörfern der Anbauregionen versorgen Tankwagen die Bevölkerung jede Woche mit Wasser, es gibt etwa 61 Liter Wasser pro Kopf für den täglichen Bedarf. In \textit{Deutschland} liegt der Verbrauch bei 10 - 15 Liter pro Minute Duschen und 8 - 17 Liter pro Sp{"u}lmaschinenlauf.

\begin{enumerate}
  \setcounter{enumi}{4}
\item Mit der rationierten Wassermenge aus Chiles Anbaugebieten können Sie in \textit{Deutschland} wie oft Ihren Bedarf stillen? \textbf{(1 Punkt)}
\end{enumerate}

Das alles hätten Sie nicht von einem Studienrat mit Stock erwartet. Ganz schön viele Informationen wurden da zusammengetragen.

\begin{enumerate}
  \setcounter{enumi}{5}  
  \item Nennen Sie eine \textit{Daten}quelle im Internet, wo Sie mehr Informationen zu landwirtschaftlichen Daten oder klimatischen, wirtschaftlichen und gesellschaftlichen Daten erhalten! \textbf{(1 Punkt)}
\end{enumerate} 
\clearpage
% ----------------------------------------------------------------------- 

\section{Aufgabe \hfill (12 Punkte)}

\textit{Geben Sie grunds{\"a}tzlich Formeln und Rechenweg zur L{\"o}sung der
  Teilaufgaben mit an!} \\[1Ex]

%% --------------------------------------------------------------------
\hfill\href{https://youtu.be/WZSxntiNF8s}{\includegraphics[width = 2cm]{img/youtube}} %%youtube
\hspace{2Ex}
%% --------------------------------------------------------------------

%% --------------------------------------------------------------------
{\tiny\textbf{Stichworte:} Kardaschow-Skala $\bullet$ Dyson-Sphäre $\bullet$ Hohlerde $\bullet$ Entropie $\bullet$ Proton $r_P = 1.7 \times 10e-15$ $\bullet$ Wasserstoff $r_H = 5.3\times 10e-11$}
%% --------------------------------------------------------------------

\paragraph{Die Dampfnudelerde}



"'Was für einen Unsinn!"', rufen Sie. Jetzt haben Sie kostbaren Schlaf prokrastiniert um einem Ernährungswissenschaftler auf YouTube über die Erde als Dampfnudel zu lauschen. Irgendwie passt es dann doch mit der Analogie. Übermüdet müssen Sie darüber nachdenken, warum vor 66 Millionen Jahren die Dinosaurier - so groß sie auch waren - nicht von der Schwerkraft zu Boden gerissen wurden. Hat der Dampfplauderer etwa recht und war die Schwerkraft vor Millionen von Jahren eine andere?  Sind deshalb alle Lebewesen auf der Erde \textit{heutzutage} so viel kleiner, weil die Schwerkraft größer ist als damals? War die Erde kleiner und hatte weniger Masse? Oder ist es nur ein Rechenfehler wie bei der Theorie der Hohlerde von Edmond Halley aus dem 17.–18. Jahrhundert? Müde reiben Sie sich die Augen. So wird es nichts mehr mit dem Schlafen, dann können Sie auch mal etwas rechnen\footnote{Die Quelle der Inspiration
  für die Aufgabe war folgender Artikel:
  \href{https://hpd.de/artikel/erde-dampfnudel-22236}{"Skeptische Anmerkungen --- Die Erde als Dampfnudel" in Der Humanistische Pressedienst}}.  \\

Betrachten wir die Schwerkraft oder Gewichtskraft, die auf Lebewesen damals und heute gewirkt haben soll. Nehmen Sie für die Fallbeschleunigung $g$ der Erde \textit{heutzutage} einen Wert von 9.78m/s$^2$ an. Im Weiteren hat die Erde einen ungefähren Durchmesser von \ensuremath{1.235\times 10^{4}}km und eine mittlere Dichte $\rho$ von 5.86g/cm$^3$. Das Gewicht von einem heute lebenden asiatischen Elefanten liegt bei 3t bis 5t und das Gewicht von einem Tyrannosaurus rex (T. rex) bei 4.5t bis 8t.

\begin{enumerate}
\item Welchen Durchmesser müsste die Erde vor 66 Millionen Jahren gehabt haben, wenn Dinosaurier und Elefanten die gleiche Gewichtskraft $\overrightarrow{F_G}$ damals und heute erfahren hätten? \textit{Beantworten Sie die Frage anhand der folgenden Teilaufgaben!}
\begin{enumerate}
\item Berechnen Sie die Fallbeschleunigung von vor 66 Millionen Jahren unter der obigen Annahme gleich wirkender Gewichtskraft $\overrightarrow{F_G}$ auf Elefant und Dinosaurier! \textbf{(1 Punkt)}
\item Berechnen Sie Masse der heutigen Erde! \textbf{(2 Punkte)}
\item Schließen Sie über die Masse auf den Durchmesser der Erde vor 66 Millionen Jahren! \textbf{(2 Punkte)}
\end{enumerate}
\item Beantworten Sie die Eingangsfrage mit 1-2 Antwortsätzen! \textbf{(1 Punkt)}
\end{enumerate}

Die Distanz zwischen Sonne und Erde entspricht 1.01 astronomische Einheiten ($AE$). Die Einheit 1 AE wird mit \ensuremath{1.52\times 10^{8}}km angegeben. Der \textit{massebehaftete} Sonnenwind besteht aus 85\% Wasserstoffkernen mit einer molaren Masse von 1.08g/mol, 10\% Heliumkernen mit 3.92g/mol sowie 5\% weiteren Atomkernen mit  152.01g/mol. Die Teilchendichte bei Eintritt in die Erdatmosphäre liegt zwischen 0.4 bis 100 Teilchen cm$^{-3}$ pro Sekunde mit einer mittleren Teilchendichte von 7cm$^{-3}$ pro Sekunde. \\

\textit{Lösen Sie den folgenden Aufgabenteil mit einer aussagekräftigen Skizze!}

\begin{enumerate}
  \setcounter{enumi}{3}
\item Berechnen Sie die Anzahl an massebehafteten Teilchen des Sonnenwindes, die die gesamte Erde pro Sekunde treffen! \textbf{(2 Punkte)}
\item Berechnen Sie die Anzahl an massebehafteten Teilchen des Sonnenwindes, die die Sonne pro Sekunde in alle Richtungen aussendet! \textbf{(2 Punkte)}
\item Berechnen Sie die Masse, die die Erde pro Jahr durch die \textit{massebehafteten} Teilchen des Sonnenwind zunimmt! \textbf{(2 Punkte)}
\end{enumerate}

%\blfootnote{\tiny\textbf{Stichworte:} Kardaschow-Skala $\bullet$ Dyson-Sphäre $\bullet$ Hohlerde $\bullet$ Entropie}

% https://de.wikipedia.org/wiki/Kardaschow-Skala 
\clearpage
% ----------------------------------------------------------------------- 

\section{Aufgabe \hfill (10 Punkte)}

\textit{Geben Sie grunds{\"a}tzlich Formeln und Rechenweg zur L{\"o}sung der
  Teilaufgaben mit an!} \\[1Ex]

%% --------------------------------------------------------------------
\hfill\href{https://youtu.be/n451XnhtSh4}{\includegraphics[width = 2cm]{img/youtube}} %%youtube
\hspace{2Ex}
%% --------------------------------------------------------------------

\paragraph{Entschuldigung, ist das Ihre Feder in meinem Auge?}



So h{\"o}rt man h{\"a}ufiger h{\"o}fliche Enten in Mastst{\"a}llen sagen. Das
ist nat{\"u}rlich etwas ung{\"u}nstig, den dann kommt es zu Picken und
Kannibalismus. Denn wenn der Nachbar nervt, dann muss zu Ma{\ss}nahmen
gegriffen werden. Kennt jeder aus einer mittelm{\"a}{\ss}igen Wohngemeinschaft. Das
wollen wir aber als vorsorgliche Enten-Halter:innen
nicht\footnote{Die Quelle der Inspiration f{\"u}r die Aufgabe war der folgende
  wissenschaftliche Artikel:
  \href{https://www.efsa.europa.eu/en/efsajournal/pub/7788}{EFSA Panel on
    Animal Health and Welfare, et al. (2023) Welfare of broilers on
    farm. EFSA Journal 21.2}.}. Betrachten wir also einmal das Platzangebot
(eng. \textit{space allowance}, abk. \textit{SA}) der Enten
f{\"u}r vier T{\"a}tigkeiten und versuchen die notwendige Fl{\"a}che zu optimieren. Wie
immer gibt es daf{\"u}r eine mathematische Formel:


\begin{center}
  \begin{tabular}{cc}
    $SA = \sum^n_{i = 1} (A_i \times PB_i)$ & $A_i = \pi \times (r_i + R_i)^2$\\
  \end{tabular}
\end{center}

\vspace{-2Ex}

mit

\begin{itemize}[noitemsep]
\item $SA$ dem ben{\"o}tigten Platzangebot aller aufsummierten Verhalten $i$.
\item $A_i$ dem ben{\"o}tigten Platz f{\"u}r ein Verhalten $i$. 
\item $PB_i$ dem Anteil des Auftretens eines Verhaltens $i$.
\item $r_i$ dem Radius Ente plus dem ben{\"o}tigten Radius f{\"u}r das Verhalten $i$.
\item $R_i$ dem notwendigen Abstand zu den Nachbarn f{\"u}r das Verhalten $i$.    
\item $i$ dem Verhalten: (1) preening, (2) wing/leg stretching, (3)
  wingflapping und (4) dustbathing.
\end{itemize}

In der folgenden Tabelle 1 sind die Werte f{\"u}r $r_i$, $R_i$ und $PB_i$ f{\"u}r
ein spezifisches Verhalten $i$ aus drei wissenschaftlichen
Ver{\"o}ffentlichungen dargestellt.

\vspace{-1Ex}

{\small
\begin{knitrout}
\definecolor{shadecolor}{rgb}{0.969, 0.969, 0.969}\color{fgcolor}\begin{table}[!h]
\centering
\begin{tabular}{llll}
\toprule
  & Aldridge et al. (2021) & Baxter et al. (2022) & Jabcobs et al. (2019)\\
\midrule
preening & 31cm; 20cm; 8.1\% & 33cm; 17cm; 8.1\% & 34cm; 24cm; 1.2\%\\
wing/leg stretching & 34cm; 16cm; 6.3\% & 39cm; 25cm; 3.6\% & 34cm; 35cm; 6.3\%\\
wingflapping & 36cm; 26cm; 6.2\% & 34cm; 32cm; 1.2\% & 37cm; 25cm; 1.2\%\\
dustbathing & 37cm; 28cm; 1.2\% & 27cm; 30cm; 1.2\% & 28cm; 25cm; 1\%\\
\bottomrule
\end{tabular}
\end{table}

\end{knitrout}
}

\begin{enumerate}
\item Erstellen Sie eine zusammenfassende Tabelle mit den mittleren Werten
  f{\"u}r $r$, $R$ und $PB$ aus der obigen Tabelle 1 f{\"u}r die jeweiligen
  Verhalten! \textbf{(3 Punkte)}
\item Erg{\"a}nzen Sie eine Spalte mit dem ben{\"o}tigten Platz $A$ f{\"u}r das jeweilige
  Verhalten, welches sich aus den mittleren Werten ergibt! \textbf{(1 Punkt)}
\item Berechnen Sie das ben{\"o}tigte Platzangebot $SA$ f{\"u}r alle betrachteten
  Verhalten! \textbf{(1 Punkt)}
\item Skizzieren Sie die Werte $r_i$, $R_i$ und $A_i$ f{\"u}r zwei
  nebeneinander agierende Enten f{\"u}r ein Verhalten $i$. Nutzen
  Sie hierf{\"u}r vereinfachte geometrische Formen! \textbf{(2 Punkte)}
\item Sie entnehmen der Literatur folgende Aussage zur Verteilung der
  Enten in der Fl{\"a}che $A$: \textit{"`Assuming, that the
    animals will optimally and equally distribute in an area $A$, we observe a
    small part, which is not covered. This area is called
    $\omega$ and is calculated with $\omega = \tfrac{A}{0.9069}$."'
  } Veranschaulichen Sie die Fl{\"a}che $\omega$ in einer aussagekr{\"a}ftigen
  Abbildung!  \textbf{(1 Punkt)}
\item Ein Tier braucht Platz f{\"u}r sich selbst. Berechnen Sie nun die
  K{\"o}rperfl{\"a}che $a$, die ein Tier einnimmt. Welche Annahmen haben Sie f{\"u}r
  die Berechnung der K{\"o}rperfl{\"a}che getroffen? \textbf{(2 Punkte)}
\end{enumerate}



 
\clearpage
% ----------------------------------------------------------------------- 

\section{Aufgabe \hfill (8 Punkte)}

\textit{Geben Sie grunds{\"a}tzlich Formeln und Rechenweg zur L{\"o}sung der
  Teilaufgaben mit an!} \\[1Ex]

%% --------------------------------------------------------------------
\hfill\href{https://youtu.be/1B53cVFIU7Q}{\includegraphics[width =
  2cm]{img/youtube}} %%youtube
\hspace{2Ex}
%% --------------------------------------------------------------------

\paragraph{Nelken von den Molukken}



In der Ausstellung "`Europa und das Meer"' im Deutschen Historischen Museum in
Berlin gab es folgendes Zitat {\"u}ber die Probleme der fr{\"u}hen Hochseeschifffahrt.

\begin{quote}
  >>Ohne ausreichende Zufuhr von Vitamin C stellen sich nach 45 Tagen die
  ersten Symptome ein; die ersten Toten sind nach 72 Tagen zu beklagen;
  nach 105 Tagen rafft die Skorbut eine ganze Schiffsbesatzung dahin<<
\end{quote}

Ferdinand Magellan stach im Jahre 1519 in See um eine Passage durch den
s{\"u}damerikanischen Kontinent zu finden. Zu seiner Flotte geh{\"o}rten
f{\"u}nf Schiffe - das Flaggschiff Trinidad, die San Antonio, die Victoria, die
Concepci{\'o}n und die Santiago - mit einer Besatzung von insgesamt
245 Mann. 

\begin{enumerate}
\item Stellen Sie den Verlauf der Anzahl an Matrosen auf einem Schiff der
  Flotte in der Form einer {\"U}berlebenszeitkurve dar! Beschriften Sie die
  Achsen entsprechend! \textbf{(2 Punkte)} 
\item Was ist die Besonderheit der Überlebenszeitkurve? Begründen Sie Ihre Antwort! \textbf{(2 Punkte)} 
\item Sch{\"a}tzen Sie die {\"U}berlebenswahrscheinlichkeit nach 100 Tagen
  aus Ihrer Abbildung ab! \textbf{(1 Punkt)} 
\end{enumerate}

Der Chronist an Bord der Trinidad, Antonio Pigafetta, schrieb in seinem
Bericht "`[...] Um nicht Hungers zu sterben, a{\ss}en wir das Leder, mit dem
die gro{\ss}e Rahe zum Schutz der Taue umwunden war."' Insbesondere die
Mannschaft der Concepci{\'o}n erlitt gro{\ss}e Verluste durch die Skrobut bei der
{\"U}berquerung des Pazifiks, da durch Erkundungsfahrten weniger Zeit blieb, um
wilden Sellerie aufzunehmen. Wilder Sellerie enth{\"a}lt
$8000\mu g/150mg$ Vitamin C. Der Bedarf liegt bei
$110mg$ pro Tag f{\"u}r M{\"a}nner.

\begin{enumerate}
  \setcounter{enumi}{2}
\item Berechnen Sie die notwendige Menge in $t$ an
  aufzunehmenden wilden Sellerie auf die Concepci{\'o}n f{\"u}r die ununterbrochene
  Fahrt von drei Monate und 20 Tage {\"u}ber den Pazifik!
  \textbf{(3 Punkte)}
\item Skizzieren Sie die {\"U}berlebenszeitkurve f{\"u}r die Concepci{\'o}n im
  Vergleich zu der {\"U}berlebenszeitkurve der Trinidad! Beschriften Sie die
  Achsen! \textbf{(2 Punkte)}
\end{enumerate}

 
\clearpage
% ----------------------------------------------------------------------- 

\section{Aufgabe \hfill (10 Punkte)}

\textit{Geben Sie grunds{\"a}tzlich Formeln und Rechenweg zur L{\"o}sung der
  Teilaufgaben mit an!} \\[1Ex]

%% --------------------------------------------------------------------
\hfill\href{https://youtu.be/q-qYK4Chslg}{\includegraphics[width =
  2cm]{img/youtube}} %%youtube
\hspace{2Ex}
%% --------------------------------------------------------------------

\paragraph{Event Horizon -- Am Rande des Universums}



Die Sonne hat eine aktuelle, angenommene Masse von $\ensuremath{2\times 10^{29}}$kg. Wenn
die Sonne nun am Ende ihrer Lebenszeit zu einem schwarzen Loch mit dem
Radius von $3000$m kollabiert, wird die Sonne $30$\%
der aktuellen Masse verloren haben. Ein Lichtteilchen mit der Masse $m_f$
und der Fluchtgeschwindigkeit $v_f$ will dem schwarzen Loch entkommen.  Sie
haben folgende Formeln f{\"u}r die kinetische Energie des Lichtteilchens
$E_{kin}$ und der Graviationsenergie des schwarzen Lochs $E_{grav}$
gegeben\footnote{Die Quelle der Inspiration
  für die Aufgabe war ein Montagnachtfilm:
  \href{https://de.wikipedia.org/wiki/Event_Horizon_–_Am_Rande_des_Universums}{Event Horizon – Am Rande des Universums}}.

\begin{center}
  \begin{tabular}{cc}
    $E_{kin} = \cfrac{1}{2}m_fv_f^2$ & $E_{grav} = \cfrac{Gm_sm_f}{r_s}$\\
  \end{tabular}
\end{center}

mit

\begin{itemize}[noitemsep]
\item $m_f$, gleich der Masse [kg] des fliehenden Objektes
\item $m_s$, gleich der Masse [kg] des station{\"a}ren Objekts
\item $r_s$, gleich dem Radius [m] des station{\"a}ren Objekts  
\item $G$, gleich der Gravitationskonstante mit $5.974 \cdot 10^{-11}
  m^3(kg \cdot s^2)^{-1}$ 
\end{itemize}

Im Folgenden wollen wir uns mit der Frage besch{\"a}ftigen, ob das
Lichtteilchen der Gravitation des schwarzen Lochs entkommen kann.

\begin{enumerate}
\item Geben Sie die Formel f{\"u}r die Fluchtgeschwindigkeit $v_f$ an! 
  \textbf{(1 Punkt)}
\item {\"U}berpr{\"u}fen Sie Ihre umgestellte Formel nach $v_f$ anhand der Einheiten!
  \textbf{(1 Punkt)} 
\item Berechnen Sie die notwendige Fluchtgeschwindigkeit $v_f$ des
  Lichtteilchens mit den angegebenen Informationen! \textbf{(2 Punkte)}
\item Gehen Sie von einer Lichtgeschwindigkeit von $\ensuremath{2.8\times 10^{8}}m/s$
  aus. Kann das Lichtteilchen der Gravitation des schwarzen Lochs
  entkommen? Begr{\"u}nden Sie Ihre Antwort! \textbf{(2 Punkte)}
\item Stellen Sie den Zusammenhang zwischen dem sich verringernden Radius
  $r$ des schwarzen Lochs bei gleichbleibender Masse $m_s$
  und der notwendigen Fluchtgeschwindigkeit $v_f$ in einer Abbildung dar!
  \textbf{(2 Punkte)}
 \item Ein Flugzeug und ein Handtuch st{\"u}rzen aus gro{\ss}er und gleicher H{\"o}he
  in ein schwarzes Loch. Welches der beiden Objekte {\"u}berschreitet zuerst
  den Ereignishorizont des schwarzes Loches? Begr{\"u}nden
  Sie Ihre Antwort mathematisch! \textbf{(2 Punkte)}  
\end{enumerate}

 
\clearpage
% -----------------------------------------------------------------------

\section{Aufgabe \hfill (10 Punkte)}

\textit{Geben Sie grunds{\"a}tzlich Formeln und Rechenweg zur L{\"o}sung der
  Teilaufgaben mit an!} \\[1Ex]

%% --------------------------------------------------------------------
\hfill\href{https://youtu.be/iCQogS6KhPM}{\includegraphics[width =
  2cm]{img/youtube}} %%youtube
\hspace{2Ex}
%% --------------------------------------------------------------------

\paragraph{Das Fermi Paradoxon}



Der Kernphysiker Enrico Fermi diskutierte 1950 auf dem Weg zum Mittagessen
im Los Alamos National Laboratory mit seinen Kollegen angebliche
UFO-Sichtungen und fragte schlie{\ss}lich: "`Where is everybody?"'. Warum seien
weder Raumschiffe anderer Weltraumbewohner noch andere Spuren
extraterrestrischer Technik zu beobachten? Wie lange w{\"u}rde eine au{\ss}erirdische
Zivilisation ben{\"o}tigen um die gesamte Milchstra{\ss}e zu
besuchen, wenn das maximale Reisetempo die Geschwindigkeit der Voyager 1 Sonde w{\"a}re?\footnote{Die Quelle der Inspiration
  für die Aufgabe war folgender Wikipediaeintrag:
  \href{https://de.wikipedia.org/wiki/Fermi-Paradoxon}{Fermi-Paradoxon}}\\[-1ex]

Wir treffen folgende Annahmen. Eine au{\ss}erirdische Zivilisation schickt $drei$
Voyager 1 {\"a}hnliche Sonden mit der Geschwindigkeit von $\ensuremath{5.9256\times 10^{4}}km/h$
los um sich auf den erreichten Planeten selbst zu replizieren. Nach
$1000$ Jahren ist die Replikation abgeschlossen und wiederum
$drei$ Sonden werden ausgesendet. Gehen Sie von
$6.23$ Lichtjahren als mittlerer Abstand der Sterne in der
Milchstra{\ss}e aus. Es gibt $\ensuremath{10^{11}}$ Sterne in der Milchstra{\ss}e. Nehmen
Sie eine Lichtgeschwindigkeit von $\ensuremath{2.8\times 10^{8}}m/s$ an.

\begin{enumerate}
\item Skizzieren Sie in einer Abbildung die ersten drei Schritte der
  Vervielf{\"a}ltigung der Sonden in der Galaxie! Beschriften Sie die Abbildung
  mit der Dauer und der Anzahl an Sonden f{\"u}r jeden Schritt der Vervielf{\"a}ltigung! \textbf{(4 Punkte)}
\item Berechnen Sie die theoretische Anzahl an Vervielf{\"a}ltigungsschritten die
  ben{\"o}tigt werden um mit \textit{einem einzigen Vervielf{\"a}ltigungsschritt} die
  gesamten Sterne der Milchstra{\ss}e mit Sonden zu besuchen! \textbf{(2 Punkte)}
\item Berechnen Sie die Dauer, die eine au{\ss}erirdische Zivilisation
  ann{\"a}hrungsweise ben{\"o}tigt um die gesamten Sterne der Milchstra{\ss}e mit
  Sonden zu besuchen! \textbf{(2 Punkte)}
\item Bei einem vermutetet Alter der Erde von $\ensuremath{4.1\times 10^{9}}$ Jahren,
  wie oft war dann eine Sonde einer au{\ss}erirdischen Zivilisation schon zu
  Besuch? Korrigieren Sie Ihre Antwort mit dem Wissen, dass sich die
  Kontinentalplatten einmal alle $\ensuremath{10^{8}}$ Jahre vollst{\"a}ndig im
  Erdinneren umgewandelt haben! \textbf{(2 Punkte)}
\end{enumerate}


 
\clearpage
% -----------------------------------------------------------------------

\section{Aufgabe \hfill (10 Punkte)}

\textit{Geben Sie grunds{\"a}tzlich Formeln und Rechenweg zur L{\"o}sung der
  Teilaufgaben mit an!} \\[1Ex]

%% --------------------------------------------------------------------
\hfill\href{https://youtu.be/tDgr6fpkkYA}{\includegraphics[width =
  2cm]{img/youtube}} %%youtube
\hspace{2Ex}
%% --------------------------------------------------------------------

\paragraph{Pyramiden bauen}



Es stehen die bayrischen Pyramidentage an und Sie sind auf
abenteuerlichen Wegen f{\"u}r den Bau der Pyramiden zust{\"a}ndig. Zu allem
{\"U}berfluss handelt es sich auch noch eine \textit{Reenactment}
Veranstaltung. Thema der diesj{\"a}hrigen Pyramidentage sind die Pyramiden von
Meroe, die den K{\"o}nigen und K{\"o}niginnen des historischen Reiches von Kusch in
Nubien, dem heutigen Sudan, als Grabst{\"a}tten dienten. Die Pyramiden in Meroe
fallen durch ihren steilen Winkel von 73 Grad im Vergleich zu
den {\"a}gyptischen Pyramiden mit 54 Grad auf. Die durchschnittliche
Seitenl{\"a}nge der Grundfl{\"a}che einer Pyramide betr{\"a}gt 32 K{\"o}nigsellen. Eine K{\"o}nigselle
misst 52.4cm.\\

\textit{L{\"o}sen Sie diese Aufgabe mit Hilfe einer Skizze der Pyramide. Bezeichnen
  Sie Seiten und die Winkel der Pyramide entsprechend!}

\begin{enumerate}
\item Bei der K{\"o}nigspyramide von Meroe soll eine Seitenl{\"a}nge der Grundfl{\"a}che
  32 K{\"o}nigsellen lang sein. Welche H{\"o}he der K{\"o}nigspyramide in $m$
  ergibt sich? \textbf{(1 Punkt)}
\item Die Au{\ss}enfl{\"a}chen der Pyramide soll begr{\"u}nt werden. F{\"u}r die
  Bepflanzung muss eine 4cm dicke Torfschicht auf die
  Pyramide aufgebracht werden. Berechnen Sie die ungef{\"a}hre Menge an
  ben{\"o}tigten Torf in $m^3$! \textbf{(2 Punkte)}
\end{enumerate}

Wie in jedem guten \textit{Reenactment} gibt es viel Oberschicht, aber nur
5 Sklaven, die Ihnen bei dem Bef{\"u}llen der Pyramide mit Schutt
zu Seite stehen. Leider haben Ihre Sklaven zu allem {\"U}berfluss auch noch
chronische Schulterschmerzen entwickelt, als sie von der
anstehenden Aufgabe erfahren haben. Gehen Sie daher von einer Effizienz der
Sklaven von 80\% aus. In eine Schubkarre passen
90 Liter.

\begin{enumerate}
  \setcounter{enumi}{2}
\item Wie oft m{\"u}ssen Ihre maladen Sklaven die Rampe mit der Schubkarre
  zur Spitze der Pyramide hochfahren um die Pyramide mit Schutt zu f{\"u}llen? \textbf{(1 Punkt)}
\item Berechnen Sie die L{\"a}nge der Rampe zur Spitze der Pyramide mit einem
  Anstellwinkel von $9^\circ$! \textbf{(2 Punkte)}
\item Wie weit reicht Ihre Rampe vom Fu{\ss} der Pyramide in die bayrische
  Landschaft?  \textbf{(2 Punkte)}
\end{enumerate}

Bei der Besichtigung der Pyramide teilt Ihnen der leicht {\"u}bergewichtige
Pharao (Nebenberuf \textit{Versicherungsverteter}) mit, das die Pyramide zu
flach sei und somit nicht in die bayrische Landschaft passen
w{\"u}rde. Sie m{\"u}ssen nochmal ran.

\begin{enumerate}
  \setcounter{enumi}{5}
\item Die Grundfl{\"a}che der Pyramide {\"a}ndert sich nicht. Berechnen Sie die
  {\"A}nderung der H{\"o}he in K{\"o}nigsellen, wenn sich der Anstellwinkel der
  Pyramide um $6^\circ$ {\"a}ndert!  \textbf{(2 Punkte)}
\end{enumerate}



% https://de.wikipedia.org/wiki/Rechtwinkliges_Dreieck
% https://www.matheretter.de/wiki/pyramide
% https://de.wikipedia.org/wiki/Knickpyramide 
\clearpage
% -----------------------------------------------------------------------

\section{Aufgabe \hfill (12 Punkte)}

\textit{Geben Sie grunds{\"a}tzlich Formeln und Rechenweg zur L{\"o}sung der
  Teilaufgaben mit an!} \\[1Ex]

%% --------------------------------------------------------------------
\hfill\href{https://youtu.be/3LAq3R0rS14}{\includegraphics[width =
  2cm]{img/youtube}} %%youtube
\hspace{2Ex}
%% --------------------------------------------------------------------

\paragraph{Geocaching -- Von Satelliten und Plastikdosen}



Es ist Wochenende und das Wetter ist \textit{sweet}. Sie schwingen sich auf
Ihr Cachermobil um mit 19km/h, geleitet von modernster
Satellietentechnologie und einem Supercompter aus dem Jahr 2000 in Ihren
H{\"a}nden, Plastikdosen in der Natur und an sehenswerten Orten zu finden. Sie
wollen diesmal endlich die abw{"a}rts Schwierigkeitschallenge
durchf{\"u}hren. Die Reihenfolge der Caches nach Schwierigkeitswertung gibt daher
die von Ihnen abzufahrenden Orte vor. Die Terrain- und
Schwierigkeitswertungen laufen von 1 (leichteste Wertung) bis 5
(schwierigste Wertung) in 0.5 Schritten. Folgende Informationen zu den
Orten und den entsprechenden Caches stehen Ihnen f{\"u}r Ihre Planung der Route
zu Verf{\"u}gung\footnote{Die Quelle der Inspiration
  für die Aufgabe war folgende Tätigkeit:
  \href{https://www.geocaching.com/play}{Geocaching -- Mach mit bei der weltweit größten Schatzsuche.}}.

\begin{center}
  \begin{tabular}{ ccc }
    \toprule
    Ort & Cache & Wertung (S|T|G) \\
    \midrule
    A & GCIRRXX & 4.0 | 1.5 | Normal \\
    B & GCL1SGO & 2.0 | 4.5 | Klein \\ 
    C & GC9PGOU & 4.5 | 2.5 | Klein \\ 
    D & GCXUW9P & 1.5 | 3.5 | Klein \\ 
    E & GCV61AO & 3.5 | 3.0 | Mikro \\     
 \bottomrule
\end{tabular}
\end{center}

Im Weiteren sind Ihnen folgende Informationen zu den Entfernungen der Orte
zug{\"a}nglich. Der Entfernungsvektor $\overrightarrow{AB}$ ist
$4$km. Im Weiteren ist Ihnen der Entfernungsvektor
$\overrightarrow{CB}$ mit $5.5$km bekannt. Der
Entfernungsvektor $\overrightarrow{BE}$ ist das $1.2$-fache
des Entfernungsvektor $\overrightarrow{CB}$. Wenn Sie von dem Ort A den Ort
C anpeilen, so liegt der Ort B ungef{\"a}hr $25^\circ$
n{"o}rdlich. Wenn Sie von dem Ort C den Ort B anpeilen, so liegt
der Ort D ungef{\"a}hr $50^\circ$ {\"o}stlich. Vom Ort B
betrachtet, bilden die Orte C und D einen rechten Winkel am Ort B. Der Ort
B liegt auf gerader Linie zwischen den Orten C und E. Somit liegt der Ort E
n{"o}rdlich von B. Die Strecke zwischen A und E ist nicht
passierbar. Sie starten an dem Ort D Ihre Cachertour. \\

\begin{enumerate}
\item L{\"o}sen Sie diese Aufgabe mit Hilfe einer aussagekr{\"a}ftigen Skizze der
  Orte und Caches. Bezeichnen Sie die Strecken und die Winkel Ihrer Skizze
  entsprechend! \textbf{(2 Punkte)}
\item Welche Strecke in $km$ legen Sie bei der Bew{\"a}ltigung der
  abw{"a}rts Schwierigkeitschallenge zur{\"u}ck? \textbf{(5
    Punkte)}
\item Gehen Sie von einer zus{\"a}tzlichen Suchzeit in Stunden f{\"u}r die
  Caches an den jeweiligen Orten zur reinen Reisezeit mit Ihrem Cachermobil
  aus. Die Suchzeit in Stunden f{\"u}r \textit{jeden einzelnen} Cache wird durch die
  Funktion
  \begin{equation*}
    Suchzeit = 0.2 + 0.15 \cdot Schwierigkeit
  \end{equation*}  
  beschreiben.  Wie lange in Stunden ben{\"o}tigen Sie um die
  abw{"a}rts Schwierigkeitschallenge zu erf{\"u}llen? \textbf{(3 Punkte)}
\item An der h{\"o}chsten Schwierigkeit m{\"u}ssen Sie angeln. Ihre Angel ist
  ausgefahren 5m lang. Erreichen Sie einen Cache in der H{\"o}he
  von 6.8m?  Berechnen Sie dazu Ihre maximale m{\"o}gliche
  Angelh{\"o}he! Welche Annahmen mussten Sie treffen um die Aufgabe zu l{\"o}sen? \textbf{(2 Punkte)} 
\end{enumerate}

 
\clearpage
% -----------------------------------------------------------------------

\section{Aufgabe \hfill (10 Punkte)}

\textit{Geben Sie grundsätzlich Formeln und Rechenweg zur L{\"o}sung der
  Teilaufgaben mit an!} \\[1Ex]

%% --------------------------------------------------------------------
\hfill\href{https://youtu.be/4-dSaPMhK9s}{\includegraphics[width =
  2cm]{img/youtube}} %%youtube
\hspace{2Ex}
%% --------------------------------------------------------------------

%% --------------------------------------------------------------------
{\tiny\textbf{Stichworte:} Brot aus Luft $\bullet$ Walöl $\bullet$ Haber-Bosch-Verfahren $\bullet$ 1. Weltkrieg $\bullet$ 40\% N im menschlichen Körper $\bullet$ Positivist}
%% --------------------------------------------------------------------

\paragraph{Die atmende Wand und Brot aus Luft}



Als Kellerkind vom Dorf wollen das Ausmaß der Radonbelastung in ihrem Kellerzimmer bestimmen und
lüften daher nicht. Spart dann auch Energie und lüften wird sowieso überschätzt. Während einer Messperiode von 7:00 Uhr bis
19:00 bestimmen Sie dreimal automatisch die Radonbelastung in
Ihrem Kellerraum in $Bq/m^3$. Es ergibt sich folgende Abbildung\footnote{Die Quelle der Inspiration
  für die Aufgabe war folgender Artikel:
  \href{https://de.wikipedia.org/wiki/Atmende_Wand}{Atmende Wand}}.

\begin{knitrout}
\definecolor{shadecolor}{rgb}{0.969, 0.969, 0.969}\color{fgcolor}

{\centering \includegraphics[width=\maxwidth]{img/math-10-1} 

}


\end{knitrout}

\vspace{-0.75cm}

\begin{enumerate}
\item Wie lange dauert es in Stunden bis Sie eine kritische Belastung von
  400$Bq/m^3$ in Ihrem ungelüfteten Kellerraum erreicht haben?
  \textbf{(2 Punkte)}
\end{enumerate}

Radon zerfällt mit einer Halbwertszeit von 2.8d zu
Polonium. Polonium wiederum zerfällt mit einer Halbwertszeit von
160d zu Blei. Nur Radon und Polonium tragen zur
radioaktiven Strahlenbelastung bei.

\begin{enumerate}
  \setcounter{enumi}{1}
\item Wie lange dauert es in Stunden bis Ihre kritische Radonbelastung von
  400$Bq/m^3$ auf unter 80$Bq/m^3$ gefallen ist?
  \textbf{(4 Punkte)}
\end{enumerate}

Folgende Tabelle enthält die Informationen zur Zusammensetzung der
normalen Umgebungsluft.

\begin{center}
  \begin{tabular}{ c|c|c|c }
     & Vol-\% & M [g/mol] & ppm \\
    \hline
    Stickstoff & 79.7 & 28.1 &
                                                    \phantom{1000000000000}\strut\\
        \hline
    Sauerstoff & 20.45 & 16.5 &
                                                    \phantom{10000000}\strut\\
        \hline
    Kohlenstoffdioxid & 0.029 & 12.5 & \phantom{10000000}\strut\\     
     \hline
\end{tabular}
\end{center}

\begin{enumerate}
   \setcounter{enumi}{2}
\item Rechnen Sie die Volumenprozente (Vol-\%) der Umgebungsluft in die
  entsprechenden ppm-Werte um und ergänzen Sie die berechneten ppm-Werte in
  die Tabelle!  \textbf{(1 Punkt)}
\end{enumerate}

Während Sie Ihr etwas pappiges Toastbrot mampfen kommt Ihnen die Dokumentation über Brot aus Luft in den Sinn. Sie denken darüber ein wenig nach. Für die Umwandlung von Stickstoff $N_2$ mit Wasserstoff $H_2$ zu Ammoniak
$NH_3$ gilt folgende Reaktionsgleichung\footnote{Die Quelle der Inspiration
  für die Aufgabe war folgender Artikel:
  \href{https://www.gdch.de/netzwerk-strukturen/fachstrukturen/ag-chemie-und-gesellschaft/projekte-und-veranstaltungen/cartoons/haber-bosch-verfahren-brot-aus-luft.html}{Haber-Bosch-Verfahren – Brot aus Luft}}:

\begin{equation*}
  N_2 + 3H_2 \rightarrow 2NH_3
\end{equation*}  

Ein Mol eines beliebigen Gases hat bei normalen Umweltbedingungen ein
Volumen von 22.4 Liter. % Ein Mol enthält $6 \cdot 10^{23}$ Teilchen.

\begin{enumerate}
  \setcounter{enumi}{3}
\item Welche Masse an Ammoniak in Kilogramm $kg$ können Sie aus einem
  Kubikmeter $m^3$ Luft unter normalen Umweltbedingungen gewinnen?
  \textbf{(2 Punkte)}
\item Wieviel Ammoniak in $mol$ erhalten Sie aus einem Kubikmeter Luft?
  \textbf{(1 Punkt)}
\end{enumerate}

 
\clearpage
% -----------------------------------------------------------------------

\section{Aufgabe \hfill (10 Punkte)}

\textit{Geben Sie grunds{\"a}tzlich Formeln und Rechenweg zur L{\"o}sung der
  Teilaufgaben mit an!} \\[1Ex]

%% --------------------------------------------------------------------
\hfill\href{https://youtu.be/Bbu6n8MXxQk}{\includegraphics[width =
  2cm]{img/youtube}} %%youtube
\hspace{2Ex}
%% --------------------------------------------------------------------

\paragraph{Armee der Finsternis}



Ihr Studentenjob war nach Ladenschluss bei Penny die Regale
einzur{\"a}umen. Dabei ist Ihnen in der Auslage der Sonderangebote das
Necronomicon\footnote{Ein wirklich gef{\"a}hrliches Buch ist:
  \textit{Du bist genug: Vom Mut, gl{\"u}cklich zu sein} von Fumitake Koga und
  Ichiro Kishimi} in die H{\"a}nde gefallen. Nun sind Sie ein Magier der
Zeichen geworden! Also eigentlich k{\"o}nnen Sie nur Mathe und das d{\"a}mliche
Necronomicon hat Sie in die Vergangenheit geschleudert... aber gut, was tut
man nicht alles im Jahr 884 n. Chr. f{\"u}r den neuen Lehnsherren
F{"u}rsten Arthur. Sie bauen nat{\"u}rlich einen Schrottkugelturm um sich den
Horden der Finsternis mit genug Schrott erwehren zu k{\"o}nnen! Ihnen stehen
zwei m{\"a}chtige magische Formeln zur Unterst{\"u}tzung zu Verf{\"u}gung.

\begin{center}
  \begin{tabular}{cc}
    $E_{kin} = \cfrac{1}{2}\cdot m \cdot v^2$ & $E_{pot} = m \cdot g \cdot h$\\
  \end{tabular}
\end{center}

mit

\begin{itemize}
\item $m$, gleich der Masse [kg] des Objekts
\item $h$, gleich der H{\"o}he [m] des ruhenden Objekts
\item $v$, gleich der Geschwindigkeit [m/s] des Objekts
\item $g$, gleich der Erdbeschleunigung mit $9.81 \tfrac{m}{s^2}$ 
\end{itemize}

Als erstes m{\"u}ssen Sie die H{\"o}he des zu bauenden Schrottkugelturmes
bestimmen. Hierf{\"u}r ist wichtig zu wissen, dass sich die
Blei\textit{tropfen} mit einem Gewicht von $20mg$ zu
gleichf{\"o}rmigen Blei\textit{kugeln} bei einer Geschwindigkeit von
$10m/s$ bilden.

\begin{enumerate}
\item Wie hoch m{\"u}ssen Sie den Schrottkugelturm bauen lassen, damit sich
  runde Bleikugeln durch die Fallgeschwindigkeit von $10m/s$ bilden?
  \textbf{(3 Punkte)}
\end{enumerate}

Ihre erstellten Schrottkugeln sind leider zu gro{\ss} und somit sind zu wenige
Schrottkugeln in einer Ladung. Damit k{\"o}nnen Sie die Armee der Finsternis
nicht aufhalten. Die Sachlage m{\"u}ssen Sie einmal mathematisch untersuchen.

\begin{enumerate}
  \setcounter{enumi}{1}
\item Nennen Sie die beiden geometrischen Formen aus denen sich
  n{\"a}herungsweise ein Tropfen zusammensetzt! Erstellen Sie eine beschriftete 
  Skizze des Tropfens!
  \textbf{(2 Punkte)}
\item Sie messen eine L{\"a}nge des Tropfens von 3.2mm. Die L{\"o}cher
  im Sieb erlauben ein Tropfendurchmesser von 1.5mm. Welchen
  Durchmesser in mm haben Ihre produzierten Blei\textit{kugeln}?  \textbf{(3 Punkte)}
\end{enumerate}

Sie haben jetzt die \ensuremath{2.3\times 10^{5}} Bleikugeln zusammen. Blei hat eine Dichte
von $11.34g/cm^3$.

\begin{enumerate}
  \setcounter{enumi}{3}
\item Wie schwer in Kilogramm $kg$ sind die \ensuremath{2.3\times 10^{5}} produzierten
  Bleikugeln, die Sie jetzt auf die Burgmauer transportieren m{\"u}ssen?
  \textbf{(1 Punkt)}
\end{enumerate}

Am Ende m{\"u}ssen Sie noch die Produktion von dem Bleischrott im Turm optimieren.

\begin{enumerate}
  \setcounter{enumi}{4}
\item Wie gro{\ss} in $cm^2$ ist Ihr quadratisches Sieb am oberen Ende des Turms,
  wenn Sie pro Fall ca. 700 Bleikugeln produzieren wollen und
  die Bleikugel im Fall 1.2cm Abstand haben m{\"u}ssen?  \textbf{(1
    Punkt)}
\end{enumerate}
 
\clearpage
% -----------------------------------------------------------------------

\section{Aufgabe \hfill (10 Punkte)}

\textit{Geben Sie grunds{\"a}tzlich Formeln und Rechenweg zur L{\"o}sung der
  Teilaufgaben mit an!} \\[1Ex]

%% --------------------------------------------------------------------
\hfill\href{https://youtu.be/Mr6eslls4J0}{\includegraphics[width =
  2cm]{img/youtube}} %%youtube
\hspace{2Ex}
%% --------------------------------------------------------------------

\paragraph{Armee der Kaninchen}



Leider hat es mit Ihrer Koalakuschelschule in Down Under nicht geklappt. War vielleicht
auch nicht \textit{so} die beste Idee... aber daf{\"u}r haben Sie eine Neue!
Oder wie es Mike Tyson zugeschrieben wird: "`Ich wurde nie
niedergeschlagen, ich war immer am Aufstehen!"'. Daher machen Sie jetzt
einen Gro{\ss}handel mit Kaninchenfleisch und damit dem teuersten Fleisch in
Australien auf. Moment, hopsen hier nicht, seit Thomas Austin im Jahr
1860 ungef{\"a}hr 28 Kaninchen entlassen hat,
Millionen von Kaninchen rum? Wieso ist das Kaninchenfleisch dann so
exklusiv? Dem wollen wir mal mathematisch nachgehen!\footnote{Die Quelle
  der Inspiration f{\"u}r die Aufgabe war der folgendes YouTube Video:
  \href{https://youtu.be/38fuOr3tdgc?si=Li7NL_FoByML8JtT}{ Incredible
    Stories -- Why don't they eat wild rabbits in Australia? They have
    millions of them! The reason is surprising...}}
\\

Forscherinnen fand folgende S{\"a}ttigungsfunktion f{\"u}r das j{\"a}hrliche Wachstum
der gesamten Kaninchenpopulation im westlichen Australien.
\begin{equation*}
  f(t) = \ensuremath{9\times 10^{9}} - \ensuremath{1.2\times 10^{9}} \cdot 2.2^{-0.15 \cdot t + 4.1}
\end{equation*}

\begin{enumerate}
\item Skizzieren Sie die S{\"a}ttigungsfunktion \textit{ann{\"a}herungsweise} in einer Abbildung! \textbf{(1
    Punkt)}
\item Wie viele Kaninchen k{\"o}nnen nach der S{\"a}ttigungsfunktion maximal im
  westlichen Australien leben? Erg{\"a}nzen Sie den Wert in Ihrer Abbildung! \textbf{(2 Punkte)}
\item Wie viele Millionen Kaninchen leben nach der S{\"a}ttigungsfunktion
  nach 18 Jahren auf dem australischen Kontinent? \textbf{(1
    Punkt)}
\end{enumerate}

Um den Kaninchen Einhalt zu gebieten wurde das Myxoma Virus und das Rabbit
Haemorrhagic Disease Virus (RHDV) in 20 Kaninchen
ausgebracht. Da die Kaninchen keine Ma{\ss}nahmen gegen die Ausbreitung
vornehmen k{\"o}nnen, verl{\"a}uft die Ausbreitung mit einem w{\"o}chentlichen
Wachstumsfakor von 1.4 nach folgender Formel.

\begin{equation*}
  N(t) = N(0) \cdot a^t
\end{equation*}

\begin{enumerate}
  \setcounter{enumi}{2}
\item Wie viele Wochen ben{\"o}tigen die Viren um theoretisch die gesamte
  Kaninchenpopulation nach 16 Jahren Wachstum zu
  durchseuchen? \textbf{(1 Punkt)}
\end{enumerate}

Das Myxoma Virus und das RHDV t{\"o}ten 98.5\% der
Kaninchenpopulation innerhalb weniger Wochen.

\begin{enumerate}
  \setcounter{enumi}{3}  
\item Wie lange in Jahren dauert es bis eine Kaninchenpopulation nach einer
  Viruspandemie wieder auf 50\% der ges{\"a}ttigten
  Kaninchenpopulation angewachsen ist?  \textbf{(2 Punkte)}
\end{enumerate}

Thomas Austin entlie{\ss} die Kaninchen im {\"a}u{\ss}ersten Westen von
Australien. Australien hat eine West-Ost-Ausdehnung von 4400km
und eine Nord-S{\"u}d-Ausdehnung von knapp 3400km. Die Kaninchen
breiten sich radial mit einer Geschwindigkeit von 9.8km pro
Jahr aus.

\begin{enumerate}
  \setcounter{enumi}{4}
\item Wie lange dauert es in Jahren bis die Kaninchen jeden Ort
  in Australien erreicht haben? \textit{L{\"o}sen Sie die Aufgabe unter der
    Verwendung einer schematischen Skizze!} \textbf{(2 Punkte)}
\end{enumerate}

Eine j{\"a}hrliche Impfung gegen das Myxoma Virus und das Rabbit Haemorrhagic
Disease Virus (RHDV) kosten 12\$ pro Tier und der
durchf{\"u}hrende Arzt verlangt ca. 35\$ pro Tier.

\begin{enumerate}
  \setcounter{enumi}{5}
\item In Ihrem Stall leben 1000 Mastkaninchen. Mit welchen
  j{\"a}hrlichen Zusatzkosten f{\"u}r die Impfungen der Kaninchen m{\"u}ssen Sie daher
  kalkulieren? \textbf{(1 Punkt)}
\end{enumerate}
 
\clearpage
% -----------------------------------------------------------------------

\section{Aufgabe \hfill (12 Punkte)}

\textit{Geben Sie grunds{\"a}tzlich Formeln und Rechenweg zur L{\"o}sung der
  Teilaufgaben mit an!} \\[1Ex]

%% --------------------------------------------------------------------
\hfill\href{https://youtu.be/fiWGgCX-cE4}{\includegraphics[width =
  2cm]{img/youtube}} %%youtube
\hspace{2Ex}
%% --------------------------------------------------------------------



\paragraph{Uckermark. Unendliche Weiten.}



Wir schreiben das Jahr 2024. Dies sind die Abenteuer
der Kuh Fridolin und Ihnen. Gr{\"u}nes Gras unter Ihren F{\"u}{\ss}en und
ein strammer Wind im Gesicht, egal wohin Sie schauen. Ein schmatzendes
Ger{\"a}usch ert{\"o}nt unter Ihnen. Sie sinnieren, sollten Sie Ihre weiten
Graslandschaften jetzt schon d{\"u}ngen?  Daf{\"u}r ben{\"o}tigen Sie die
\textit{Gr{\"u}nlandtemperatur}! Die Gr{\"u}nlandtemperatur (GLT) ist die Summe aller
positiven Tagesmitteltemperaturen seit Jahresbeginn. Ab einer GLT von
200$^\circ$ kann mit der Stickstoffd{\"u}ngung begonnen werden. Sie sehen nicht
ein, Geld f{\"u}r einen Agrarmetrologen zu bezahlen. Also rechnen Sie mit
folgenden Informationen zu Monatsmultiplikatoren des GLT-Wertes: Januar mit
$0.5\times$, Februar mit $0.75\times$ und M{\"a}rz mit
$1.1\times$. Sie haben noch im letzten Jahr folgende Temperaturen
gemessen.

\begin{center}
\begin{tabular}{cc}
  \toprule
  Datum & C$^\circ$ \\
  \midrule
  01. Jan 2023 & 0.1\\
  01. Feb 2023 & 1.2\\
  01. Mrz 2023 & 3.5\\
  01. Apr 2023 & 5.8\\
  \bottomrule
\end{tabular}
\end{center}

\begin{enumerate}
\item Erstellen Sie eine Skizze aus den Informationen aus der
  Temperaturtabelle!  \textbf{(1 Punkt)}
\item Stellen Sie die linearen Funktionen $f_1(t)$, $f_2(t)$ und
  $f_3(t)$ aus der obigen Temperaturtabelle auf!  \textbf{(1 Punkt)}
\item Bestimmen Sie die Stammfunktionen $F_1(t)$, $F_2(t)$ und $F_3(t)$ f{\"u}r
  Ihre linearen Funktionen aus der obigen Temperaturtabelle!  \textbf{(1
    Punkt)}
\item Osterglocken beginnen ab einer GLT von 210$^\circ C$ zu bl{\"u}hen. An
  welchem Tag im 1. Quartal des Jahres 2023 war dies der
  Fall? \textit{Ignorieren Sie ein eventuelles Schaltjahr in Ihrer Berechnung.} \textbf{(4 Punkte)}
\end{enumerate}

Auf dem Weg zu Ihrer Jonagoldplantage wurden Sie mit Ihrem Trecker von
einer Gruppe elektrifizierter Renter abgedr{\"a}ngt. Der Trecker muss wieder
aus dem Graben! Fridolin und die elektrifizierten Rentner ziehen an zwei,
separaten Seilen. Dabei zieht Fridolin mit $180N$. Die
elektrifizierter Renter  bringen eine Kraft von $190N$ auf.\\

\textit{L{\"o}sen Sie diese Aufgabe mit Hilfe einer aussagekr{\"a}ftigen Skizze der
  Kraftvektoren. Bezeichnen Sie die Kraftvektoren und die Winkel Ihrer
  Skizze entsprechend!}

\begin{enumerate}
  \setcounter{enumi}{4}  
\item Im ersten Versuch legen Sie das Seil f{\"u}r Fridolin lotrecht {\"u}ber einen
  Ast oberhalb des Treckers. Die Rentner ziehen in einer geraden Linie {\"u}ber
  die B{\"o}schung hinweg am anderen Seil. Welche Kraft wird aufgebracht?
  \textbf{(2 Punkte)}
\item Im zweiten Versuch ziehen Fridolin und die Rentner mit einem
  $40^\circ$ Winkel mit ihrem Seil an dem Trecker. Welche Kraft
  wird aufgebracht? \textbf{(2 Punkte)}
\item Mit welcher Beschleunigung ziehen Sie den $1.3t$ schweren
  Trecker \textit{jeweils} aus dem Graben, wenn $F = m \cdot a$ gilt?
  \textbf{(1 Punkt)}
\end{enumerate}

 
\clearpage
% -----------------------------------------------------------------------

\section{Aufgabe \hfill (10 Punkte)}

\textit{Geben Sie grunds{\"a}tzlich Formeln und Rechenweg zur L{\"o}sung der
  Teilaufgaben mit an!} \\[1Ex]

%% --------------------------------------------------------------------
\hfill\href{https://youtu.be/RuzMjwvwT-4}{\includegraphics[width =
  2cm]{img/youtube}} %%youtube
\hspace{2Ex}
%% --------------------------------------------------------------------

\paragraph{In der Kartonagenfabrik}



Wenn man sich zu sp{\"a}t anmeldet, dann ist die Exkursion nicht so toll. Also
geht es mit Rektor Skinner und Mrs. Krabappel in die Kartonagenfabrik. Wie
schon im vorherigen Semester... In der Kartonagenfabrik angekommen erfahren
Sie, dass die Kartons zum Versand von N{\"a}geln nicht hier zusammengebaut
werden sondern das sich die Endfertigung in Flint, Michigan befindet. Unter
anderem wird dort der ber{\"u}hmte \textit{Doppelt gewellte,
  5-mal-gefaltete, 0.7mm, 50-cm-Karton} durch
Falzung hergestellt. Beim letzten Mal war Rektor Skinner die Stimmung zu
schlecht und deshalb geht es erst nach Hause, wenn ein paar Aufgaben gel{\"o}st
sind. Martin gef{\"a}llt das. An dem Vorrat
an Zigaretten von Mrs. Krabappel meinen Sie wenig Zuversicht zu erkennen.\\

Jetzt hei{\ss}t es Kartons optimieren. Der nun zu optimierende, flache Karton
hat eine L{\"a}nge von 50cm und eine Breite von 21cm. Die
Kartonagenmaschine in Flint soll dann einen quadratischen Eckenausschnitt
der L{\"a}nge $x$ falzen.

\begin{enumerate}
\item Erstellen Sie eine Skizze des Karton\textit{blatt}rohlings!
  Beschriften Sie die Skizze mit den entsprechenden L{\"a}ngenangaben
  \textbf{(1 Punkt)}
\item Berechnen Sie die Falztiefe $x$ f{\"u}r ein maximales Volumen des flachen
  Kartons! \textbf{(3 Punkte)}
\item Welches Volumen in Liter ergibt sich mit der von Ihnen berechneten
  Falztiefe $x$?  \textbf{(1 Punkt)}
\item Sie wollen noch einen b{\"u}ndig mit dem Boden abschlie{\ss}enden Deckel f{\"u}r
  den Karton stanzen lassen. Wie gro{\ss} ist die Fl{\"a}che des
  Kartondeckel\textit{blatt}rohlings in $cm^2$? \textbf{(2 Punkte)}
\end{enumerate}

Rektor Skinner m{\"o}chte sich gerne wieder in seinem Vorgarten aufhalten und
nicht die ganze Zeit von Bart mit Erdn{\"u}ssen beworfen werden. Deshalb m{\"o}chte
er einen ger{\"a}umigen Teil seines Vorgartens einz{\"a}unen. Ein Teil der
Umz{\"a}unung bildet seine Vorderhauswand. Wegen Lieferschwierigkeiten stehen
Rektor Skinner nur 100m Zaun zu Verf{\"u}gung. Sie wollen nun die
maximale Fl{\"a}che des abgeschirmten Vorgartens in Abh{\"a}ngigkeit der
Seitenl{\"a}ngen bei der Verwendung von 100m Zaun bestimmen!

\begin{enumerate}
  \setcounter{enumi}{4}  
\item  Welche Seitenl{\"a}ngen f{\"u}r den Zaun ergeben sich f{\"u}r die
  maximale Fl{\"a}che des abgeschirmten Vorgartens? \textbf{(2 Punkte)}
\item Berechnen Sie die Fl{\"a}che des abgeschirmten Vorgartens! \textbf{(1
    Punkt)}
\end{enumerate}

 
\clearpage
% -----------------------------------------------------------------------

\section{Aufgabe \hfill (10 Punkte)}

\textit{Geben Sie grunds{\"a}tzlich Formeln und Rechenweg zur L{\"o}sung der
  Teilaufgaben mit an!} \\[1Ex]

%% --------------------------------------------------------------------
\hfill\href{https://youtu.be/OhyuH6hzEhY}{\includegraphics[width = 2cm]{img/youtube}} %%youtube
\hspace{2Ex}
%% --------------------------------------------------------------------




\paragraph{Ein Pfund Insekten, bitte!} Das wird wohl h{\"a}ufiger geh{\"o}rt
werden, wenn wir die Menschheit mit Proteinen ausreichend ern{\"a}hren
wollen\footnote{Die Quelle der Inspiration f{\"u}r die Aufgabe war der folgende
  Artikel aus dem Spiegel:
  \href{https://www.spiegel.de/ausland/ueberbevoelkerung-acht-milliarden-sind-wir-bald-zu-viele-menschen-auf-der-erde-a-3f20c7bc-3d60-4440-9f52-eb338db207f5}{Acht
    Milliarden - sind wir bald zu viele Menschen auf der Erde?}}. Schauen
wir uns dazu einmal den Vergleich Deutschland zu Nigeria an. Nigeria hat
eine der am schnellsten wachsenden Bev{\"o}lkerungen der Welt und wird
vermutlich im Jahr 2100 zu den Top 5 der bev{\"o}lkerungsreichsten L{\"a}nder
z{\"a}hlen. Im Jahr 2020 leben ca. \ensuremath{8\times 10^{7}} Menschen in
Deutschland und ca. \ensuremath{1.79\times 10^{8}} Menschen in Nigeria. Mit den
Informationen wollen wir anfangen und dann eine Prognose f{\"u}r den
Fleischkonsum im Jahr 2050 zu treffen. \\ 


Im folgenden ist Abbildung des Fleischkonsums im Jahr 2020 in
Deutschland und Nigeria in [kg] einmal dargestellt. 

\begin{knitrout}
\definecolor{shadecolor}{rgb}{0.969, 0.969, 0.969}\color{fgcolor}

{\centering \includegraphics[width=\maxwidth]{img/math-14-a-1} 

}


\end{knitrout}
  
\begin{enumerate}
\item Stellen Sie den Fleischkonsum in Deutschland und Nigeria im Jahr
  2020 \textit{pro Kopf} in einer aussagekr{\"a}ftigen Tabelle dar! 
  \textbf{(2 Punkte)}
\item Erg{\"a}nzen Sie in der Tabelle eine Spalte in der Sie f{\"u}r den
  Fleischkonsum in Nigeria auf Deutschland normieren, daher ins Verh{\"a}ltnis
  Nigeria/Deutschland, setzen! \textbf{(1 Punkt)}
\end{enumerate}

In der n{\"a}chsten Abbildung finden Sie die CO$_2$-Emission in [kg] nach
Lebensmittel, die durch die Produktion entsteht, abgebildet.

\begin{knitrout}
\definecolor{shadecolor}{rgb}{0.969, 0.969, 0.969}\color{fgcolor}

{\centering \includegraphics[width=\maxwidth]{img/math-14-b-1} 

}


\end{knitrout}

\begin{enumerate}
  \setcounter{enumi}{2}  
\item Stellen Sie in einer Tabelle die Treibhausgasemissionen an $CO_2$ pro
  Kopf, die durch den Fleischkonsum in Deutschland und Nigeria im Jahr
  2020 entstehen, dar! Erg{\"a}nzen Sie auch hier das Verh{\"a}ltnis
  Nigeria zu Deutschland! \textbf{(2 Punkte)}
\end{enumerate}

\newpage

In der folgenden Abbildung sehen Sie die
Bev{\"o}lkerungsentwicklung [Millionen] in Nigeria von 1950 bis ins Jahr 2030 fortgef{\"u}hrt.

\begin{knitrout}
\definecolor{shadecolor}{rgb}{0.969, 0.969, 0.969}\color{fgcolor}

{\centering \includegraphics[width=\maxwidth]{img/math-14-c-1} 

}


\end{knitrout}

\begin{enumerate}
  \setcounter{enumi}{3}  
\item Sch{\"a}tzen Sie graphisch die zu erwartende Bev{\"o}lkerung [Millionen] in Nigeria im
  Jahr 2050, die sich anhand der Informationen aus der Abbildung ergibt!
\begin{enumerate}
\item Ohne Ber{\"u}cksichtigung der Covid-19-Pandemie! \textbf{(1
    Punkt)}
\item Unter Ber{\"u}cksichtigung der Covid-19-Pandemie! \textbf{(1
    Punkt)}
\end{enumerate}
\item Berechnen Sie den gesch{\"a}tzten Fleischkonsum von Nigeria im Jahr
  2050 unter der Annahme 80\%-iger Angleichung der Lebensbedingungen zu
  Deutschland im Jahr 2020! \textbf{(1 Punkt)}
\item Berechnen Sie die prozentuale Steigerung der Treibhausgasemissionen
  an $CO_2$ in Nigeria im Jahr 2050 im Vergleich zum Jahr 2020, der
  sich durch den angeglichenen Fleischkonsum ergibt! \textbf{(1 Punkt)}
\item Berechnen Sie die prozentuale Steigerung der Treibhausgasemissionen
  an $CO_2$ in Nigeria, wenn die gesamte Proteinaufnahme 
  durch Insekten ersetzt w{\"u}rde! \textbf{(1
    Punkt)}
\end{enumerate}







 
\clearpage
% -----------------------------------------------------------------------

\section{Aufgabe \hfill (12 Punkte)}

\textit{Geben Sie grunds{\"a}tzlich Formeln und Rechenweg zur L{\"o}sung der
  Teilaufgaben mit an!} \\[1Ex]

%% --------------------------------------------------------------------
\hfill\href{https://youtu.be/flRBo1FWQy0}{\includegraphics[width = 2cm]{img/youtube}} %%youtube
\hspace{2Ex}
%% --------------------------------------------------------------------

\paragraph{T{\"o}dliche Seuche AIDS -- Die r{\"a}tselhafte Krankheit}




Irritiert legen Sie die historische Ausgabe des Spiegels aus den 80zigern
beiseite. Sie sind bei Ihrem Orthop{"a}den und wollen einen AIDS-Test machen
lassen. Woanders leider keinen Termin gekriegt... Immerhin denken Sie und
Ihr Partner {\"u}ber Nachwuchs nach und da geht es eben nur durch
ungesch{\"u}tzten Sex. Was wissen Sie nun aber {\"u}ber AIDS und dem diagnostischen
AIDS-Test, den Sie nun machen werden?\\

Die Pr{\"a}valenz von AIDS bei einem Menschen in Europa wird mit
0.5\% angenommen. In 90\% der F{\"a}lle ist ein
HIV-Test positiv, wenn der Patient erkrankt ist. In 2.5\%
der F{\"a}lle ist ein HIV-Test positiv, wenn der Patient \textit{nicht}
erkrankt ist und somit gesund ist. Sie stutzen. Wie wahrscheinlich ist es
denn eigentlich an AIDS erkrankt zu sein $(K^+)$, wenn Sie einen positiven
AIDS-Test vorliegen haben $(T^+)$? Gehen Sie f{\"u}r die folgenden Berechnungen
von $n = \ensuremath{3\times 10^{4}}$ Patienten mit einem diagnostischen Test f{\"u}r AIDS
aus. Sie nehmen sich also einen Kuli und fangen an auf der historischen
Ausgabe des Spiegels zu rechnen\footnote{Die Quelle der Inspiration f{\"u}r die
  Aufgabe war der folgende wissenschaftlicher Artikel: \textit{Binder et
    al. (2022) Von Baumdiagrammen {\"u}ber Doppelb{\"a}ume zu H{\"a}ufigkeitsnetzen --
    kognitive {\"U}berlastung oder didaktische Unterst{\"u}tzung? Journal f{\"u}r
    Mathematik-Didaktik, 1-33}}.

\begin{enumerate}
\item Welche Wahrscheinlichkeit $Pr$ wollen Sie berechnen? \textbf{(1 Punkt)}
\item Zeichnen Sie einen H{\"a}ufigkeitsdoppelbaum zur Bestimmung der gesuchten
  Wahrscheinlichkeit $Pr$! \textbf{(2 Punkte)} 
\item Beschriften Sie den H{\"a}ufigkeitsdoppelbaum, mit denen Ihnen bekannten
  Informationen zu der AIDS Erkrankung und dem AIDS-Test! \textbf{(1 Punkt)}
\item F{\"u}llen Sie den H{\"a}ufigkeitsdoppelbaum mit den sich ergebenden,
  absoluten Patientenzahlen $n$ aus! \textbf{(2 Punkte)}
\item Berechnen Sie die gesuchte Wahrscheinlichkeit $Pr$! \textbf{(1 Punkt)}
\end{enumerate}

Bei dem folgenden Arztgespr{\"a}ch erfahren Sie, dass beim diagnostischen
Testen \textit{True Positives (TP)}, \textit{True Negatives (TN)},
\textit{False Positives (FP)} und \textit{False Negatives (FN)}
auftreten. Das verstehen Sie so noch nicht und deshalb stellen Sie f{\"u}r sich
den Zusammenhang in einer 2x2 Kreuztabelle dar.

\begin{enumerate}
  \setcounter{enumi}{5}
\item Tragen Sie \textit{TP}, \textit{TN}, \textit{FP} und \textit{FN} in
  eine 2x2 Kreuztablle ein. Beschriften Sie die Tabelle entsprechend!
  \textbf{(1 Punkt)}
\item Berechnen Sie die Sensitivit{\"a}t und Spezifit{\"a}t des diagnostischen Tests
  f{\"u}r AIDS! F{\"u}llen Sie daf{\"u}r die 2x2 Kreuztabelle mit den Informationen aus
  dem H{\"a}ufigkeitsdoppelbaum aus! \textbf{(2 Punkte)}
\item Was beschreibt die Sensitivit{\"a}t und die Spezifit{\"a}t im Bezug auf die
  Gesunden und Kranken? Stellen Sie beide diagnostische Ma{\ss}zahlen als
  Wahrscheinlichkeiten $Pr$ dar! \textbf{(2 Punkte)} 
\end{enumerate}



 
\clearpage
% -----------------------------------------------------------------------

\section{Aufgabe \hfill (12 Punkte)}

\textit{Geben Sie grunds{\"a}tzlich Formeln und Rechenweg zur L{\"o}sung der
  Teilaufgaben mit an!} \\[1Ex]

%% --------------------------------------------------------------------
\hfill\href{https://youtu.be/SZqp_vy3rgI}{\includegraphics[width = 2cm]{img/youtube}} %%youtube
\hspace{2Ex}
%% --------------------------------------------------------------------

\paragraph{Network-Marketing oder Schneeballschlacht!}




Eine M{\"o}glichkeit, leicht Geld zu verdienen, ist es anderen Menschen f{\"u}r Geld
zu versprechen, wie man leicht reich werden kann. Am besten nat{\"u}rlich ohne
viel Aufwand und ortsunabh{\"a}ngig. Dann wollen wir mal loslegen. Schnell
ein YouTube-Werbevideo gedreht und auf geht es mit unserem
Network-Marketing. Aber Moment, wie funktioniert Network-Marketing
eigentlich und was hat das alles mit einer Schneeballschlacht zu tun? Wir
wollen hier einmal in die Untiefen des "`passiven Einkommens"'
abtauchen\footnote{Die Quellen der Inspiration f{\"u}r die Aufgabe waren
  folgendes YouTube Video:
  \href{https://youtu.be/UOKkZF_qK9M?si=uf4foJVFKfeQMwSw}{Simplicissimus --
    Die meistgesuchte Betr{\"u}gerin der Welt} und der Artikel:
  \href{https://www.deutschlandfunkkultur.de/netzwerk-marketing-die-illusion-schnell-reich-zu-werden-100.html}{
    Deutschlandfunk Kultur -- Die Illusion, schnell reich zu werden}}.\\

Das Jahr 2022 war das erfolgreichste Jahr in der Geschichte von
Direct Finanzanlagen Left/Right (D-FL/R). Das Unternehmen steigerte den Umsatz um rund
18 Prozent von 280 Millionen Euro im Jahr
2021. Doch wie viel kommt bei den Partnern an? Laut
D-FL/R habe das Unternehmen \ensuremath{2.8\times 10^{5}} aktive Partner.

\begin{enumerate}
\item Berechnen Sie zuerst den Umsatz der Firma D-FL/R im
  Jahr 2022! \textbf{(1 Punkt)}
\item Wie viel von dem Umsatz im Jahr 2022 wird im Durchschnitt von jedem
  aktiven Partner erwirtschaftet? \textbf{(1 Punkt)}
\item Welche \textit{monatlicher} Umsatz ergibt sich dadurch im
  Durchschnitt f{\"u}r jeden aktiven Partner bei einer direkten Provision von
  30\%? \textbf{(1 Punkt)}
\end{enumerate}

Ihr zu vermarkendes Produkt, hinter dem Sie voll stehen, kostet
200EUR pro Einheit im Direktverkauf. Die direkte Provision
f{\"u}r die erste Stufe betr{\"a}gt 25\%. F{\"u}r die zweite, dritte und
vierte Stufe betragen die indirekten Provisionen jeweils 3\%,
2\% und 1.5\%. Jeder Ihrer angeworbenen "`Partner"'
wirbt wiederum f{"u}nf Partner f{\"u}r sich selbst an. Pro Monat
werden im Schnitt vier Einheiten vom Produkt verkauft. Sie wollen nun
2500EUR im Monat \textit{passiv} -- also durch indirekte
Provisionen -- erwirtschaften.

\begin{enumerate}
  \setcounter{enumi}{3}
\item Erg{\"a}nzen Sie die folgende Tabelle mit den obigen Informationen! \textbf{(2 Punkte)}
\end{enumerate}

\begin{center}
\begin{tabular}{c|c|c|c}
  \toprule
  \textbf{Stufe} & \textbf{Anzahl Partner}  & \textbf{Umsatz/Stufe} & \textbf{Provision}\\
  \midrule
  1 & Sie selber  &  & \\ \midrule
  2 &   &  &  \\ \midrule
  3 &   &  &  \\ \midrule
  4 &   &  &  \\
  \bottomrule
\end{tabular}
\end{center}

\begin{enumerate}
  \setcounter{enumi}{4}
\item Wie viele Partner m{\"u}ssen Sie auf der 2 Stufe anwerben um Ihr passives
  Einkommen durch indirekte Provision zu erreichen? Wie viele Menschen
  arbeiten am Ende indirekt f{\"u}r Sie? Stellen Sie den
  Zusammenhang graphisch dar!  \textbf{(3 Punkte)}
\end{enumerate}

Sie mussten zum Einstieg bei D-FL/R Einheiten des Produkts
f{\"u}r 7000EUR kaufen. Diese Einheiten k{\"o}nnen Sie nur direkt
verkaufen. Leider mussten Sie den Kauf {\"u}ber einen Kredit {\"u}ber
4.7\% p.a. {\"u}ber 72 Monate finanzieren.

\begin{enumerate}
  \setcounter{enumi}{5}
\item Berechnen Sie die Gesamtsumme, die Sie als Kredit abbezahlen m{\"u}ssen! \textbf{(2 Punkte)}
\item Wie viele Einheiten m{\"u}ssen Sie pro Monat verkaufen um die anfallenden Zinsen
  durch die direkte Provision zu erwirtschaften? \textbf{(1 Punkt)}
\item Wie lange in
  Monaten ben{\"o}tigen Sie um den Kredit durch die direkte
  Provision abzubezahlen? \textbf{(1 Punkt)}
\end{enumerate} 
\clearpage
% -----------------------------------------------------------------------

\section{Aufgabe \hfill (10 Punkte)}

\textit{Geben Sie grunds{\"a}tzlich Formeln und Rechenweg zur L{\"o}sung der
  Teilaufgaben mit an!} \\[1Ex]

%% --------------------------------------------------------------------
\hfill\href{https://youtu.be/8Pb2sKUIMyk}{\includegraphics[width =
  2cm]{img/youtube}} %%youtube
\hspace{2Ex}
%% --------------------------------------------------------------------

\paragraph{H{\"o}hlen \& Drachen}



Nachdem Sie sich begeistert in der Serie \textit{Stranger Thinks} verloren
haben, wollen Sie bei einem Ihrer Freunde einmal \textit{H{\"o}hlen \& Drachen}
ausprobieren. Um Geld zu sparen, das Zeug kostet echt, wurde etwas an den
Regeln gebastelt. Schnell stellen Sie fest, dass hier ganz sch{\"o}n viele
unterschiedliche W{\"u}rfel durch die Gegend fliegen. Daher m{\"u}ssen Sie sich
jetzt einiges an Fragen stellen. \\%[-1ex]

In dem Spiel haben Sie nun auf einmal 6 zw{"o}lfseitige W{"u}rfel (6d12) zum w{\"u}rfeln in der Hand. Wenn Sie eine 12 w{\"u}rfeln,
haben Sie einen Erfolg.

\begin{enumerate}
\item Berechnen Sie die Wahrscheinlichkeit \textit{genau}
  4 Erfolge zu erzielen!  \textbf{(2 Punkte)}
\item Berechnen Sie die Wahrscheinlichkeit keinen Erfolg zu erzielen!
  \textbf{(1 Punkt)}
\end{enumerate}

Sie betrachten nun aufmerksam die ausufernden Ausr{\"u}stungstabellen. Ihnen
wird aber geholfen und Sie m{\"u}ssen sich jetzt nur zwischen der Axt oder dem
Schwert entscheiden.

\begin{enumerate}
  \setcounter{enumi}{2}
\item W{\"u}rden Sie die Axt mit zwei achtseitigen W{"u}rfeln (2d8) als Schaden oder
  das Schwert mit einem vierseitigen W{"u}rfel plus 3 (1d4+3) als Schaden bevorzugen?
  Begr{\"u}nden Sie Ihre Antwort mathematisch! \textbf{(1 Punkt)}
\end{enumerate}

Jetzt wird es immer wilder, da Sie sich jetzt {\"u}berlegen m{\"u}ssen, wie
wahrscheinlich es ist, dass Ihr Rettungswurf gegen den zaubernden Hexer
funktioniert. Sie haben folgende Wahrscheinlichkeiten gegeben. Die
Wahrscheinlichkeit f{\"u}r das Ereignis $A$, der Rettungswurf ist erfolgreich,
ist $Pr(A) = 0.7$, die Wahrscheinlichkeit f{\"u}r das Ereignis $B$,
der Zauberwurf des Hexers ist erfolgreich, ist $Pr(B) = 0.75$. Sie
haben mitgez{\"a}hlt und festgestellt, dass in $50$ von 100 F{\"a}llen
Ihr Rettungswurf bei einem erfolgeichen Zauber funktioniert hat.  

\begin{enumerate}
  \setcounter{enumi}{3}
\item Erstellen Sie eine 2x2 Kreuztabelle mit den Ereignissen $A$ und $B$
  sowie den Gegenereignissen $\bar{A}$ und $\bar{B}$ mit einen
  $\Omega = 100$. Beachten Sie hierbei die entsprechenden
  Wahrscheinlichkeiten f{\"u}r die Ereignisse $A$ und $B$! \textbf{(2 Punkte)}
\item Bestimmen Sie $Pr(A \cap B)$! \textbf{(1 Punkt)}
\item Erstellen Sie ein Baumdiagramm mit den passenden Informationen aus der 2x2
  Kreuztabelle! \textbf{(2 Punkte)}
\item Bestimmen Sie Wahrscheinlichkeit $Pr(A|B)$, dass Ihr Rettungswurf gelingt, wenn
  der Hexer erfolgreich gezaubert hat! \textbf{(1 Punkt)}
\end{enumerate}

  
\clearpage
% -----------------------------------------------------------------------

\section{Aufgabe \hfill (12 Punkte)}

\textit{Geben Sie grunds{\"a}tzlich Formeln und Rechenweg zur L{\"o}sung der
  Teilaufgaben mit an!} \\[1Ex]

%% --------------------------------------------------------------------
\hfill\href{https://youtu.be/9DQKaXdxT_g}{\includegraphics[width =
  2cm]{img/youtube}} %%youtube
\hspace{2Ex}
%% --------------------------------------------------------------------

\paragraph{Retrocheck im TV}



"`Und hier ist sie wieder, die Show der fantastischen Preise. Seien Sie mit
dabei, wenn es wieder hei{\ss}t: Der Preis ist hei{\ss}!"', ert{\"o}nt es und Sie
fragen sich, ob Sie nicht doch lieber bezahlter Gast bei Barbara Salesch
h{\"a}tten sein sollten. Aber Sie brauchen das Geld und jetzt hei{\ss}t es
Spielschows farmen! Erstmal eine Kaffemaschine von Mitropa
gewinnen. Ein Kandidat gewinnt die Kaffeemaschine von Mitropa, wenn nicht alle
Kandidaten {\"u}berbieten (eng. \textit{outbid}). Mit Ihnen bilden
G{"u}nther und Elke das Team der drei Kandidaten.

\begin{center}
\begin{tabular}{ccc}
  \toprule
  Name & $P(win)$ & $P(outbid)$\\
  \midrule
  G{"u}nther & 0.2 & 0.05\\
  Elke & 0.3 & 0.11 \\
  \bottomrule
\end{tabular}
\end{center}

\begin{enumerate}
\item Mit welcher Wahrscheinlichkeit gewinnen Sie die Kaffeemaschine von
  Mitropa, wenn keiner der Kandidaten {\"u}berbietet? \textbf{(1 Punkt)}
\item Wenn Ihre {\"U}berbietungswahrscheinlichkeit $P(outbid)$ bei
  0.076 liegt, mit welcher Wahrscheinlichkeit gewinnt
  \textit{keiner} die Kaffeemaschine von Mitropa? \textbf{(1 Punkt)}
\end{enumerate}

Gl{\"u}cksrad f{\"u}r Arme auf der Kirmes! Leider hat es f{\"u}r Maren Gilzer nicht
gereicht. Deshalb sind Sie jetzt auf der Kirmes und spielen mit
Catwoman um das gro{\ss}e Geld. Das Gl{\"u}cksrad hat 22
Felder. Sie drehen das Gl{\"u}cksrad zweimal. Auf 6 Feldern
gewinnen Sie 3000EUR sonst 1500EUR. Ganz sch{\"o}n viel Geld
und ganz sch{\"o}n zwielichtig hier...

\begin{enumerate}
  \setcounter{enumi}{2}  
\item Skizzieren Sie das Gl{\"u}cksrad und erg{\"a}nzen Sie die
  Wahrscheinlichkeiten! \textbf{(1 Punkt)}
\item Zeichnen Sie das zugeh{\"o}rige Baumdiagramm f{\"u}r das zweimalige Drehen!
  Erg{\"a}nzen Sie die Wahrscheinlichkeiten und die entsprechenden Ereignisse
  \textbf{(2 Punkte)}
\item Mir welcher Wahrscheinlichkeit gewinnen Sie 4500EUR? \textbf{(1
    Punkt)}
\end{enumerate}

Nach Ihrem Fiebertraum reisen Sie im Zug nach K{\"o}ln um bei "`Geh aufs
Ganze!"' mitzuspielen. Sie schaffen es tats{\"a}chlich ins Finale und k{\"o}nnen
als Hauptgewinn ein Auto hinter einer der drei T{\"u}ren gewinnen. 

\begin{enumerate}
  \setcounter{enumi}{5}  
\item Bevor die Show beginnt, wird das Auto hinter eine zuf{\"a}llig bestimmte
  T{\"u}r gestellt. Mit welcher Wahrscheinlichkeit wird jeweils eine der drei
  T{\"u}ren ausgew{\"a}hlt? Zeichnen Sie ein Baumdiagramm! \textbf{(1 Punkt)}
\item Mit welcher Wahrscheinlichkeit w{\"a}hlen Sie sofort die T{\"u}r mit
  dem Auto? Erweitere Sie das Baumdiagramm entsprechend! \textbf{(1
    Punkt)}
\item Der Moderator {\"o}ffnet nun eine der nicht gew{\"a}hlten T{\"u}ren, aber
  nat{\"u}rlich nicht die mit dem Auto. Mit welcher Wahrscheinlichkeit steht
  das Auto hinter der anderen T{\"u}r? Erweitern Sie das Baumdiagramm
  entsprechend! \textbf{(2 Punkte)}
\item L{\"o}sen Sie nun das "`Ziegenproblem"'! Berechne Sie dazu die
  Wahrscheinlichkeiten der einzelnen Pfade. Lohnt sich ein Wechsel der
  anfangs gew{\"a}hlte T{\"u}r? Begr{\"u}nden Sie Ihre Antwort mathematisch!
  \textbf{(2 Punkte)}
\end{enumerate}
 

%https://unterrichten.zum.de/wiki/Laplace-Wahrscheinlichkeit_wiederholen_und_vertiefen/Ziegen

%Buchstabenh{\"a}ufigkeit im Deutschen

%%https://www.mathelike.de/abiturloesungen-mathematik-bayern/mathematik-abitur-bayern-2020/pruefungsteil-a/stochastik-2/teilaufgabe-a.html

%%https://de.wikipedia.org/wiki/Buchstabenh%C3%A4ufigkeit


 
\clearpage
% -----------------------------------------------------------------------
\part{Angewandte Nutztier- und Pflanzenwissenschaften (M.Sc.)}
% -----------------------------------------------------------------------  

\section{Aufgabe \hfill (6 Punkte)}
Vergleichen Sie die Standardabweichung mit dem Standardfehler und grenzen
Sie die beiden Kennzahlen voneinander ab.



\section{Aufgabe \hfill (8 Punkte)}
Ihnen liegt folgendes Varianzanalysemodell mit der {\"u}blichen Beschreibung
zur Auswertung des Merkmals fett- und eiwei{\ss}korrigierte Milchleistung pro
Kuh und Jahr in kg vor:

\begin{equation*}
  Y_{ijkl} = \mu + Var_i + EKA_j + VarEKA_{ij} + V_k + b(L_{ij} - L) + e_{ijkl}
\end{equation*}

mit

\begin{itemize}
\item $Y_{ijkl}$: l-te Beobachtung
\item $\mu$: Populationsmittel 
\item $Var_i$: fixer Effekt der i-ten Variante (i: Kontrolle,
  Versuchsgruppe 1, Versuchsgruppe 2)
\item $EKA_j$: fixer Effekt der j-ten Erstkalbealtergruppe (j: EKA $\leq$ 25
  Monate, EKA > 25 Monate)
\item $VarEKA_{ij}$: fixer Effekt der Interaktion Variante x
  Erstkalbealtergruppe 
\item $V_k$: zuf{\"a}lliger Effekt des Vaters 
\item $b(L_{ij} - L)$: lineare Kovariable Laktationsnummer 
\item $e_{ijkl}$: zuf{\"a}lliger
  Restfehler
\end{itemize}
Erl{\"a}utern Sie anhand dieses Beispiels die Begriffe fixer Effekt,
Interaktion, zuf{\"a}lliger Effekt und Kovariable und grenzen Sie diese
Begriffe voneinander ab.



\section{Aufgabe \hfill (6 Punkte)}
Wie bestimmen Sie die richtige Stichprobengr{\"o}{\ss}e? Welche Kennzahlen /
statistische Ma{\ss}zahlen ben{\"o}tigen Sie dabei und nennen Sie die
Voraussetzungen.

 
\end{document}
% -----------------------------------------------------------------------


  
