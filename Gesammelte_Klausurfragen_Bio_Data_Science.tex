\documentclass[a4paper, 9pt]{scrartcl}\usepackage[]{graphicx}\usepackage[]{xcolor}
% maxwidth is the original width if it is less than linewidth
% otherwise use linewidth (to make sure the graphics do not exceed the margin)
\makeatletter
\def\maxwidth{ %
  \ifdim\Gin@nat@width>\linewidth
    \linewidth
  \else
    \Gin@nat@width
  \fi
}
\makeatother

\definecolor{fgcolor}{rgb}{0.345, 0.345, 0.345}
\newcommand{\hlnum}[1]{\textcolor[rgb]{0.686,0.059,0.569}{#1}}%
\newcommand{\hlsng}[1]{\textcolor[rgb]{0.192,0.494,0.8}{#1}}%
\newcommand{\hlcom}[1]{\textcolor[rgb]{0.678,0.584,0.686}{\textit{#1}}}%
\newcommand{\hlopt}[1]{\textcolor[rgb]{0,0,0}{#1}}%
\newcommand{\hldef}[1]{\textcolor[rgb]{0.345,0.345,0.345}{#1}}%
\newcommand{\hlkwa}[1]{\textcolor[rgb]{0.161,0.373,0.58}{\textbf{#1}}}%
\newcommand{\hlkwb}[1]{\textcolor[rgb]{0.69,0.353,0.396}{#1}}%
\newcommand{\hlkwc}[1]{\textcolor[rgb]{0.333,0.667,0.333}{#1}}%
\newcommand{\hlkwd}[1]{\textcolor[rgb]{0.737,0.353,0.396}{\textbf{#1}}}%
\let\hlipl\hlkwb

\usepackage{framed}
\makeatletter
\newenvironment{kframe}{%
 \def\at@end@of@kframe{}%
 \ifinner\ifhmode%
  \def\at@end@of@kframe{\end{minipage}}%
  \begin{minipage}{\columnwidth}%
 \fi\fi%
 \def\FrameCommand##1{\hskip\@totalleftmargin \hskip-\fboxsep
 \colorbox{shadecolor}{##1}\hskip-\fboxsep
     % There is no \\@totalrightmargin, so:
     \hskip-\linewidth \hskip-\@totalleftmargin \hskip\columnwidth}%
 \MakeFramed {\advance\hsize-\width
   \@totalleftmargin\z@ \linewidth\hsize
   \@setminipage}}%
 {\par\unskip\endMakeFramed%
 \at@end@of@kframe}
\makeatother

\definecolor{shadecolor}{rgb}{.97, .97, .97}
\definecolor{messagecolor}{rgb}{0, 0, 0}
\definecolor{warningcolor}{rgb}{1, 0, 1}
\definecolor{errorcolor}{rgb}{1, 0, 0}
\newenvironment{knitrout}{}{} % an empty environment to be redefined in TeX

\usepackage{alltt}
\usepackage[ngerman]{babel}

% -----------------------------------------------------------------------


% -----------------------------------------------------------------------
%% ------------------------------------------------------------
%% by J.Kruppa on Friday, February 11, 2022 (11:31)
%% \def\mainDir{\Sexpr{exam_path}}
\def\source{/Users/jokruppa/source/tex}
\usepackage[margin=2cm, includefoot]{geometry}
\setlength{\parindent}{0cm}
\usepackage{booktabs}
\usepackage{amsmath}
\usepackage{scalerel,amssymb}
\usepackage{setspace}
\def\csquare{{\Large $\boxtimes$}}
\def\msquare{{\Large $\square$}}
\usepackage[normalem]{ulem}
\usepackage{array}
\usepackage{xcolor}
\usepackage{float}
\usepackage{currfile}
\usepackage{tikz}
\usepackage[nomessages]{fp}

%% beamer defs
\def\lecture{Klausurfragen der Bio Data Science}

%% exam defs
\def\examtitle{\lecture}
\def\exammodule{
\vspace{-1.75cm}  
\begin{graybox}{}
\vspace{2Ex}
\textbf{\large Name:} \rule[0ex]{16.75em}{.4pt}
\hfill \textnormal{\textit{Nicht bestanden:}} \msquare \\[2.5Ex]
\textbf{\large Vorname:} \rule[0ex]{15em}{.4pt} \\[2.5Ex]
\textbf{\large Matrikelnummer:} \rule[0ex]{10.8em}{.4pt}
\hfill Endnote: \rule[0ex]{7em}{.4pt} 
\end{graybox}
\vspace{3Ex}
\phantom{text}
}
\def\examsemester{Sommersemester \& Wintersemester}
\def\examdate{\today}
%% ------------------------------------------------------------
\definecolor{darkblue}{rgb}{0,0,.5}
\definecolor{darkpurple}{rgb}{0.4117, 0.2, 0.4117}
\definecolor{uni}{rgb}{0,0.3137,0.6078}
\definecolor{gray}{gray}{0.7}

\usepackage{tcolorbox}
\definecolor{logo1}{RGB}{0, 158, 227}
\definecolor{gray5}{RGB}{247, 247, 247}
\definecolor{gray2}{RGB}{102, 102, 102}

\newtcolorbox{graybox}[1]{
  colback=gray5,%%red!5!white,
  colframe=gray2,%%red!75!black,
  fonttitle=\bfseries\Large,
  %%valign=center,
  fontupper=\large,
  before skip=10pt plus 2pt,
  after skip=20pt plus 4pt,
  title=#1}

\newtcolorbox{takehomebox}[1]{
  colback=gray5,%%red!5!white,
  colframe=logo1,%%red!75!black,
  fonttitle=\bfseries\Large,
  %%valign=center,
  fontupper=\large,
  before skip=10pt plus 2pt,
  after skip=10pt plus 2pt,
  title=#1}

\def\Rlogo{\includegraphics[width = 0.5cm]{\string~/Documents/GitHub/exam/img/Rlogo}\;}

\usepackage[scaled=.90]{helvet} 
\usepackage{fancyhdr}
\usepackage{lastpage}
\usepackage{hyperref}
\hypersetup{
    colorlinks=true,       % false: boxed links; true: colored links
    linkcolor=black,          % color of internal links 
    urlcolor=magenta           % color of external links
}
\renewcommand{\familydefault}{\sfdefault}

\title{
\large \exammodule \\[5Ex]
\Huge \examtitle \\[2Ex] 
\Large Hochschule Osnabr{\"u}ck
}
\author{Pr{\"u}fer: Prof. Dr. Jochen Kruppa \\
Fakult{\"a}t f{\"u}r Agrarwissenschaften und Landschaftsarchitektur \\ 
j.kruppa@hs-osnabrueck.de}
\date{Version vom \examdate}

%% ------------------------------------------------------------
%% by J.Kruppa on Tuesday, September 23, 2014 (12:50)
%% Header
\renewcommand{\headrulewidth}{0pt}
\renewcommand{\footrulewidth}{0pt}
\pagestyle{fancy}

\fancyhf{}
\fancyhead[L]{}
\fancyhead[R]{}
\fancyfoot[R]{\thepage}
\fancyfoot[L]{\footnotesize \examtitle}

\fancypagestyle{empty}{
 \fancyhf{}
 \fancyhead[L]{}
 \fancyhead[R]{}
 \fancyfoot[R]{\thepage}
 \fancyfoot[L]{\footnotesize \examtitle}
}

\usepackage{arevtext,arevmath}

\newcommand\Tstrut{\rule{0pt}{2.6ex}}         % = `top' strut
\newcommand\Bstrut{\rule[-0.9ex]{0pt}{0pt}}   % = `bottom' strut
\def\strut{\Tstrut\Bstrut}

% -----------------------------------------------------------------------
\IfFileExists{upquote.sty}{\usepackage{upquote}}{}
\begin{document}
\date{
\vspace{1.5Ex}
\textbf{\Large\textcolor{red}{>>\underline{\underline{\examsemester}}<<}}
\vfill
\begin{center}
\includegraphics[width = 1.9cm]{avatare/Alex}\hspace{-8mm}
\includegraphics[width = 1.9cm]{avatare/Jessica}\hspace{-8mm}
\includegraphics[width = 1.9cm]{avatare/Jonas}\hspace{-8mm}
\includegraphics[width = 1.9cm]{avatare/Mark}\hspace{-8mm}
\includegraphics[width = 1.9cm]{avatare/Nilufar}\hspace{-8mm}
\includegraphics[width = 1.9cm]{avatare/Paula}\hspace{-8mm}
\includegraphics[width = 1.9cm]{avatare/Steffen}\hspace{-8mm}
\includegraphics[width = 1.9cm]{avatare/Tina}\hspace{-8mm}
\includegraphics[width = 1.9cm]{avatare/Yuki}\\
\small
\vspace{1.5Ex}
\textit{"`The test of a student is not how much she or he knows,\\ but how much the student wants to know."'\\ --- Alice W. Rollins}
\end{center}}
% -----------------------------------------------------------------------
\maketitle
\fancypagestyle{empty}{
  \fancyfoot[L]{\tiny $\blacksquare\!\square\!\blacksquare\!\blacksquare\!\blacksquare\!\square\!\square\!\square\!\blacksquare\!\blacksquare\!\square\!\square\!\blacksquare\!\blacksquare\!\square\!\square\!\square\!\blacksquare\!\blacksquare\!\blacksquare$}
}
\thispagestyle{empty}
\clearpage
% -----------------------------------------------------------------------
\begin{minipage}[c]{0.125\textwidth}
\includegraphics[width = 1.9cm]{avatare/Alex}
\end{minipage}
\begin{minipage}[c]{0.875\textwidth}
\textit{Alex studiert im 3. Semester und wiederholt das Modul, da er im ersten Jahr andere Prioritäten für sich gesetzt hat. Das musste sein, da er sich ziemlich im Abitur verausgabt hat. Darüber hinaus war die WG auch eher auf Party angelegt. Alex hofft jetzt über Pünktlichkeit wieder in die Bahn zu kommen. Dafür steht er jetzt immer um 5 Uhr auf! Freunde von ihm beschreiben ihn eher als extrovertiert. Er kennt Paula noch aus der Schulzeit und er überlegt, ob nicht beide Mal nach Mallorca sollten.} 
\end{minipage}\\[2.75Ex]
% -----------------------------------------------------------------------
\begin{minipage}[c]{0.875\textwidth}
\textit{Nach zwei Semestern Studium an der Universität Osnabrück war es dann Jessica doch viel zu theoretisch. Etwas angewandtes sollte es sein, wo sie auch eine Fähigkeit lernt, die frau nutzen kann. Deshalb hat sich Jessica an der Hochschule eingeschrieben. Hoffentlich lernt sie etwas nützliches, wo andere für Geld geben würden. Wer nützlich ist, ist wertvoll. Ihr Traum ist ja eine Hundeschule aufzumachen. Die großen Parties hat sie immer gemieden. Sie ist lieber mit ihrer Hündin im Teuteburgerwald.}
\end{minipage}
\begin{minipage}[c]{0.125\textwidth}
\includegraphics[width = 1.9cm]{avatare/Jessica}
\end{minipage}\\[2.75Ex]
% -----------------------------------------------------------------------
\begin{minipage}[c]{0.125\textwidth}
\includegraphics[width = 1.9cm]{avatare/Jonas}
\end{minipage}
\begin{minipage}[c]{0.875\textwidth}
\textit{Das ist jetzt der letzte Versuch mit einem Studium. Wenn es nicht klappt dann überlegt Jonas das \href{https://www.ihk.de/osnabrueck/aus-und-weiterbildung/ausbildung/ausbildungsbetriebe/projekt-neustart-1087206}{Programm der IHK zu Ausbildungsvermittlung} zu nutzen. Aber eine Runde gibt er sich noch. Struktur ist eigentlich das Wichtigste und diesmal hat er sich alle Altklausuren der Fachschaft besorgt. Dann ist er auch noch gleich der Fachschaft beigetreten um mehr Informationen abzugreifen. Und er versucht nicht seine Zeit mit Alex zu verdaddeln oder in der Fachschaft bei einem Bier oder so...}
\end{minipage}\\[2.75Ex]
% -----------------------------------------------------------------------
\begin{minipage}[c]{0.875\textwidth}
\textit{Nächstes Semester geht es nach Kanada davon hat er schon auf der Berufsschule geträumt. Deshalb konzentriert er sich sehr auf die Prüfungen. Ein Schiff ist im Hafen sicher, aber dafür ist es nicht gebaut worden. Das \href{https://www.hs-osnabrueck.de/wir/fakultaeten/aul/international/}{International Faculty Office} der Fakultät Agrarwissenschaften und Landschaftsarchitektur hat super geholfen, aber es waren einiges an Unterlagen. Jetzt hofft er, dass Tina dann doch noch mitkommt. Aber sonst macht er das eben alleine. Obwohl das eher nicht so seine Art ist. Vielleicht sollte er sich mal einen Tipp bei Tina holen, sie wirkt sehr entschlossen.} 
\end{minipage}
\begin{minipage}[c]{0.125\textwidth}
\includegraphics[width = 1.9cm]{avatare/Mark}
\end{minipage}\\[2.75Ex]
% -----------------------------------------------------------------------
\begin{minipage}[c]{0.125\textwidth}
\includegraphics[width = 1.9cm]{avatare/Nilufar}
\end{minipage}
\begin{minipage}[c]{0.875\textwidth}
\textit{Nach der Ausbildung wollte Nilufar eigentlich gleich anfangen zu arbeiten, aber nach einem Praktikum und der Probezeit stellte sie fest, dass es ohne einen Hochschulabschluss schwer wird Führungsverantwortung zu übernehmen. Mit Menschen kann sie schon immer und dann auch eigene Projekte mit anderen verwirklichen, dass ist doch was. Mit dem notwendigen Abschluss sollte der Start um so einfacher sein. Dann ist sie keine Befehlsempfängerin mehr sondern gibt die Marschrichtung vor. Schon jetzt koordiniert Nilufar das Studium von anderen.}
\end{minipage}\\[2.75Ex]
% -----------------------------------------------------------------------
\begin{minipage}[c]{0.875\textwidth}
\textit{Paula möchte die Welt zu einem besseren Ort machen. Wenn da nicht die anderen Mitmenschen wären. Paula muss das Modul nochmal wiederholen, da es dann am Ende des Semesters zu viel für sie wurde. Eine Lerngruppe hätte geholfen, aber das ist dann gar nicht so einfach eine zu finden. Zwar kennt sie schon Nilufar, aber Nilufar ist ihr manchmal zu forsch. Ihr schwant aber, dass alleine das Studium sehr schwer werden wird. Das Abitur war schon so ein Lernhorror, das möchte sie nicht nochmal. Alex sieht sie da als Vorbild.}
\end{minipage}
\begin{minipage}[c]{0.125\textwidth}
\includegraphics[width = 1.9cm]{avatare/Paula}
\end{minipage}\\[2.75Ex]
% -----------------------------------------------------------------------
\begin{minipage}[c]{0.125\textwidth}
\includegraphics[width = 1.9cm]{avatare/Steffen}
\end{minipage}
\begin{minipage}[c]{0.875\textwidth}
\textit{Sommer, Sonne, Natur. Das ist es was Steffen mag. Raus in die Komune und die Natur genießen. Leider hat Steffen noch andere Bedürfnisse, die ein Einkommen benötigen. Da Studierte mehr verdienen, würde dann in Teilzeit auch mehr rausspringen. Wenn er dann privat was anbauen kann, dann spart er gleich doppelt. Leider sind viele seiner Kommilitonen total verkrampfte Karrieristen. Es geht nur ums Äußere. Dabei verliert sich Steffen gerne im Prozess. Das hat auch seinen Schulabschluss etwas verzögert. Steffen lässt sich eben Zeit.}
\end{minipage}\\[2.75Ex]
% -----------------------------------------------------------------------
\begin{minipage}[c]{0.875\textwidth}
\textit{Wille  war es, die es Tina hierher gebracht hat und Wille wird es sein, die Tina dann auch zum Abschluß treibt. Nach einem Rückschlag muss Tina jetzt einige Module wiederholen, damit sie dann auch fertig wird. Ab und zu ist sie schwach gewesen und das hat dann Zeit gekostet. Das Tina es dann manchmal übertreibt, weiß sie nur zu gut, aber irgendwie muss die Kontrolle ja erhalten bleiben? Insbesondere, wenn sie mal wieder die Nacht durchgefeiert hat, verachtet Tina sich. Dann baut Nilufar sie dann bei einem Tee wieder auf.}
\end{minipage}
\begin{minipage}[c]{0.125\textwidth}
\includegraphics[width = 1.9cm]{avatare/Tina}
\end{minipage}\\[2.75Ex]
% -----------------------------------------------------------------------
\begin{minipage}[c]{0.125\textwidth}
\includegraphics[width = 1.9cm]{avatare/Yuki}
\end{minipage}
\begin{minipage}[c]{0.875\textwidth}
\textit{Für Yuki war es nicht einfach. Teilweise war die Krankheit sehr hinderlich, dann war es Yuki selber. Dann muss man auch wieder auf die Beine kommen und es dauert eben seine Zeit. Aber immerhin hat Yuki es jetzt den Abschluss gekriegt und hat einen Studienplatz. Jetzt heißt es in den Rhythmus kommen und schauen, was noch so passiert. Immerhin hat Yuki schon eine kleine Gruppe gefunden, in der Yuki dann Hilfe findet. Ist aber auch sehr unübersichtlich so ein Studium. Steffen ist immer super entspannt.}
\end{minipage}
\clearpage
% -----------------------------------------------------------------------


\begin{graybox}{Erlaubte Hilfsmittel}
  \vspace{1Ex}
  \begin{itemize}
  \item Normaler Taschenrechner ohne Möglichkeit der Kommunikation mit anderen
    Geräten! Ausdrücklich kein Handy!
  \item Eine DIN A4-Seite als beidseitig, selbstgeschriebene,
    handschriftliche Formelsammlung. Keine digitalen Ausdrucke! 
  \item \textit{You can answer the questions in English without any consequences.}  
  \end{itemize}
\end{graybox}
\vfill

\begin{graybox}{Formalia}
  \vspace{1Ex}
  \begin{itemize}
  \item Um eine faire Korrektur zu gewährleisten,
  \begin{itemize}
  \item \textcolor{red}{ist die Verwendung eines roten Farbstiftes ist nicht gestattet. Es ist die Korrekturfarbe.}
  \item schreiben Sie bitte groß genug und leserlich mit einem dokumentenechten Stift (kein Bleistift).
  \item erstellen Sie Abbildungen in angemessener Größe und achten Sie auf eine klare Erkennbarkeit. Nutzen Sie bei Platzmangel bitte ein separates Blatt.
  \end{itemize}
  \item Verstöße können mit einem Punktabzug geahndet werden:  \hfill \msquare\; \textbf{(-2 Punkte)}
  \end{itemize}
\end{graybox}
\vfill

\begin{graybox}{Endnote}
  \vspace{1Ex}
  \begin{itemize}
  \item[] \rule[0ex]{3em}{.4pt}\, von 20\, Punkten sind aus den Multiple
    Choice Aufgaben erreicht.
  \item[] \rule[0ex]{3em}{.4pt}\, von 73 Punkten sind aus den Rechen- und
    Textaufgaben erreicht. 
  \item[] \rule[0ex]{3em}{.4pt}\, von 93 Punkten in Summe.
  \item[] Es wird folgender Notenschlüssel angewendet.   
  \end{itemize}
  \vspace{1ex}
\begin{center}
  \begin{tabular}[c]{cc}
    \toprule
    \textbf{Punkte}	&	\textbf{Note}	\\
    \midrule
    89.0 - 93.0	&	1,0	\\
    84.5 - 88.5	&	1,3	\\
    79.5 - 84.0	&	1,7	\\
    75.0 - 79.0	&	2,0	\\
    70.5 - 74.5	&	2,3	\\
    66.0 - 70.0	&	2,7	\\
    61.5 - 65.5	&	3,0	\\
    56.5 - 61.0	&	3,3	\\
    52.0 - 56.0	&	3,7	\\
    46.5 - 51.5	&	4,0	\\
    \bottomrule
  \end{tabular}
\end{center}
  \vspace{1ex}
\begin{itemize}
\item[] Es ergibt sich eine Endnote von \rule[0ex]{4em}{.4pt}.
\end{itemize}
  \vspace{1Ex}
\end{graybox}

% -----------------------------------------------------------------------
\newpage
% -----------------------------------------------------------------------

\begin{graybox}{Multiple Choice Aufgaben}
  \begin{itemize}
  \item Pro Multipe Choice Frage ist \emph{genau} eine Antwort richtig.
  \item \textbf{Übertragen Sie Ihre Kreuze in die Tabelle auf
      dieser Seite.}
  \end{itemize}

\begin{center}
  \large
  \begin{tabular}{|l|c|c|c|c|c?c|}
    \hline
    & \textbf{A} & \textbf{B} & \textbf{C} & \textbf{D} & \textbf{E} & $\boldsymbol{\checkmark}$\strut\\
    \hline
    \textbf{Aufgabe 1} &   &   &   &   &   & \strut\\
    \hline
    \textbf{Aufgabe 2} &   &   &   &   &   & \strut\\
    \hline
    \textbf{Aufgabe 3} &   &   &   &   &   & \strut\\
    \hline
    \textbf{Aufgabe 4} &   &   &   &   &   & \strut\\
    \hline
    \textbf{Aufgabe 5} &   &   &   &   &   & \strut\\
    \hline
    \textbf{Aufgabe 6} &   &   &   &   &   & \strut\\
    \hline
    \textbf{Aufgabe 7} &   &   &   &   &   & \strut\\
    \hline
    \textbf{Aufgabe 8} &   &   &   &   &   & \strut\\
    \hline
    \textbf{Aufgabe 9} &   &   &   &   &   & \strut\\
    \hline
    \textbf{Aufgabe 10} &   &   &   &   &   & \strut\\
    \hline
  \end{tabular}
\end{center}

\begin{itemize}
\item Es sind \rule[0ex]{2em}{.4pt}\, von 20 Punkten erreicht worden.
\end{itemize}
\end{graybox}

\vfill

\begin{graybox}{Rechen- und Textaufgaben}
  \begin{center}
    \large
    \begin{tabular}{|l|c|c|c|c|c|c|c|}
      \hline
      \textbf{Aufgabe} & \textbf{11} & \textbf{12} & \textbf{13} & \textbf{14} & \textbf{15} & \textbf{16} & \textbf{17} \strut\\
      \hline
      \textbf{Punkte} & 
      \hspace{1Ex}\Large\textcolor{gray!70}{9}\hspace{1Ex}  & 
      \hspace{1Ex}\Large\textcolor{gray!70}{8}\hspace{1Ex}  & 
      \hspace{1Ex}\Large\textcolor{gray!70}{12}\hspace{1Ex}  & 
      \hspace{1Ex}\Large\textcolor{gray!70}{12}\hspace{1Ex}  & 
      \hspace{1Ex}\Large\textcolor{gray!70}{10}\hspace{1Ex}  & 
      \hspace{1Ex}\Large\textcolor{gray!70}{10}\hspace{1Ex}  & 
      \hspace{1Ex}\Large\textcolor{gray!70}{12}\hspace{1Ex} \strut\\
      \hline
  \end{tabular}
\end{center}
\begin{itemize}
\item Es sind \rule[0ex]{2em}{.4pt}\, von 73 Punkten erreicht worden.
\end{itemize}
\end{graybox}

% -----------------------------------------------------------------------
\clearpage
% -----------------------------------------------------------------------
\begin{graybox}{Multiple Choice Aufgaben}
Die Multiple Choice Aufgaben \textcolor{red}{unterliegen dem Zufall}. Die Reihenfolge der Antworten ist zufällig. Die Fragen und Antworten sind semantisch zufällig und haben somit \textcolor{red}{verschiedene Textvarianten}. Insbesondere die reinen Textaufgaben haben verschiedene Textvarianten. Die Semeantik mag sich unterscheiden, die Inhalte sind aber gleich.
\end{graybox}
\section*{Programmieren in R} 

\section{Aufgabe \hfill (2 Punkte)}

%% --------------------------------------------------------------------
\ifcollection
\begin{flushright}
\tiny\vspace{-2Ex}
\textbf{\examinhaltstart}
\exammodulemathstat $\;\bullet$
\exammodulestat $\;\bullet$
\exammodulestatversuch $\;\bullet$
\exammodulebiostat
\vspace{-1Ex}
\end{flushright}
\fi
%% --------------------------------------------------------------------




Sie wollen Ihren Datensatz in \Rlogo einlesen und stehen nun vor einem Problem. Sie stellen fest, dass die Hilfeseiten alle in englischer Sprache verfasst sind. Warum mag die Nutzung von Deutsch problematisch sein?



\begin{enumerate}
\item [\textbf{A} \msquare] Es gibt keinen Grund nicht auch deutsche Wörter zu verwenden. Es ist ein Stilmittel.
\item [\textbf{B} \msquare] Programmiersprachen haben Probleme mit Umlauten und Sonderzeichen der deutschen Sprache. Daher ist die Nutzung in Deutsch in den AGBs von \Rlogo untersagt.
\item [\textbf{C} \msquare] \Rlogo Pakete sind nur in englischer Sprache verfasst. Es macht keinen Sinn \Rlogo daher in Deutsch zu bedienen.
\item [\textbf{D} \msquare] Programmiersprachen haben Probleme mit Umlauten und Sonderzeichen der deutschen Sprache. Die Nutzung von englischer Sprache umgeht dieses Problem in eleganter Art.
\item [\textbf{E} \msquare] Die Spracherkennung von \Rlogo ist nicht in der Lage Deutsch zu verstehen.
\end{enumerate} 

\section{Aufgabe \hfill (2 Punkte)}

%% --------------------------------------------------------------------
\ifcollection
\begin{flushright}
\tiny\vspace{-2Ex}
\textbf{\examinhaltstart}
\exammodulemathstat $\;\bullet$
\exammodulestat $\;\bullet$
\exammodulestatversuch $\;\bullet$
\exammodulebiostat
\vspace{-1Ex}
\end{flushright}
\fi
%% --------------------------------------------------------------------




In \Rlogo müssen die Daten eines Experiments in einem bestimmten Format vorliegen. Wir sprechen auch von dem log-Format der Daten. Welche Aussage zu dem Long-Format ist richtig?



\begin{enumerate}
\item [\textbf{A} \msquare] Das Long-Format beschreibt in den Spalten die Beobachtungen sowie in den Zeilen die \textit{unabhängigen} Beobachtungen.
\item [\textbf{B} \msquare] In den Spalten finden sich die einzelnen Beobachtungen und in den Zeilen die gemessenen Variablen ($Y$) sowie die experimentellen Faktoren ($X$).
\item [\textbf{C} \msquare] In den Spalten sind die Beobachtungen in den Zeilen die Variablen, wie die Messwerte und experiementellen Faktoren.
\item [\textbf{D} \msquare] Das Long-Format hat die gleiche Anzahl an Spalten wie auch an Zeilen.
\item [\textbf{E} \msquare] In den Spalten finden sich die experimentellen Faktoren ($X$) sowie die Messwerte ($Y$). In den Zeilen finden sich dann die einzelnen Beobachtungen.
\end{enumerate} 

\section{Aufgabe \hfill (2 Punkte)}

%% --------------------------------------------------------------------
\ifcollection
\begin{flushright}
\tiny\vspace{-2Ex}
\textbf{\examinhaltstart}
\exammodulemathstat $\;\bullet$
\exammodulestat $\;\bullet$
\exammodulestatversuch $\;\bullet$
\exammodulebiostat
\vspace{-1Ex}
\end{flushright}
\fi
%% --------------------------------------------------------------------



In Ihrer Abschlussarbeit wollen Sie zu Beginn eine explorativen Datenanalyse (EDA) in \Rlogo rechnen. Dafür gibt es eine generelle Abfolge von Prozessschritten. Welche ist hierbei die richtige Reihenfolge?



\begin{enumerate}
\item [\textbf{A} \msquare] Wir lesen die Daten über eine generische Funktion \texttt{read()} ein und müssen dann die Funktion \texttt{ggplot()} nur noch installieren. Dann haben wir die Abbildungen als \texttt{*.png} vorliegen.
\item [\textbf{B} \msquare] Wir transformieren die Spalten über \texttt{mutate()} in ein \texttt{tibble} und können dann über \text{ggplot()} uns die Abbildungen erstellen lassen. Dabei beachten wir das wir keine Faktoren in den Daten haben.
\item [\textbf{C} \msquare] Für eine explorativen Datenanalyse (EDA) in \Rlogo müssen wir als erstes die Daten über \texttt{read\_excel()} einlesen. Danach müssen wir schauen, dass wir die Zeilen richtig über \texttt{mutate()} transformiert haben. Insbesondere müssen Variablen mit kontinuierlichen Werten in einen Faktor umgewandelt werden. Am Ende nutzen wir die Funktion \text{ggplot()} für die eigentlich EDA.
\item [\textbf{D} \msquare] Die Funktionsreihenfolge ist wie folgt: \texttt{read\_excel()} ->  \texttt{mutate()} -> \text{ggplot()}. Dabei ist bei der Transformation der Daten darauf zu achten, dass die Faktoren richtig erstellt werden.
\item [\textbf{E} \msquare] Wir lesen als erstes die Daten über \texttt{read\_excel()} ein, transformieren die Spalten über \texttt{mutate()} in die richtige Form und können dann  über \text{ggplot()} uns die Abbildungen erstellen lassen. Wichtig ist, dass wir keine Faktoren sondern nur numerische Variablen vorliegen haben.
\end{enumerate} 
\section*{Deskriptive Statistik \& Explorative Datenanalyse}

\section{Aufgabe \hfill (2 Punkte)}

%% --------------------------------------------------------------------
\ifcollection
\begin{flushright}
\tiny\vspace{-2Ex}
\textbf{\examinhaltstart}
\exammodulemathstat $\;\bullet$
\exammodulestat $\;\bullet$
\exammodulestatbbv 
\vspace{-1Ex}
\end{flushright}
\fi
%% --------------------------------------------------------------------




Wie lautet der Mittelwert und Standardabweichung von $y$ mit 14, 19, 17, 16 und 23.



\begin{enumerate}
\item [\textbf{A} \msquare] Es berechnet sich 17.8 +/- 3.42
\item [\textbf{B} \msquare] Es berechnet sich 17.8 +/- 11.7
\item [\textbf{C} \msquare] Sie erhalten 17.8 +/- 1.71
\item [\textbf{D} \msquare] Es berechnet sich 18.8 +/- 11.7
\item [\textbf{E} \msquare] Es ergibt sich 16.8 +/- 5.85
\end{enumerate} 

\section{Aufgabe \hfill (2 Punkte)}

%% --------------------------------------------------------------------
\ifcollection
\begin{flushright}
\tiny\vspace{-2Ex}
\textbf{\examinhaltstart}
\exammodulemathstat $\;\bullet$
\exammodulestat $\;\bullet$
\exammodulestatbbv 
\vspace{-1Ex}
\end{flushright}
\fi
%% --------------------------------------------------------------------




Berechnen Sie den Median, das $1^{st}$ Quartile sowie das $3^{rd}$ Quartile von $y$ mit 37, 24, 13, 20, 21, 21 und 63.




\begin{enumerate}
\item [\textbf{A} \msquare] Sie erhalten 21 +/- 37
\item [\textbf{B} \msquare] Sie erhalten 21 [20; 37]
\item [\textbf{C} \msquare] Es berechnet sich 22 [21; 36]
\item [\textbf{D} \msquare] Es ergibt sich 21 +/- 20
\item [\textbf{E} \msquare] Sie erhalten 21 [18; 35]
\end{enumerate} 

\section{Aufgabe \hfill (2 Punkte)}

%% --------------------------------------------------------------------
\ifcollection
\begin{flushright}
\tiny\vspace{-2Ex}
\textbf{\examinhaltstart}
\exammodulemathstat $\;\bullet$
\exammodulestat $\;\bullet$
\exammodulestatbbv 
\vspace{-1Ex}
\end{flushright}
\fi
%% --------------------------------------------------------------------




Die \uline{Standardabweichung} ist eine bedeutende deskriptive Statistik für die Analyse von Daten. Wie müssen Sie vorgehen um die Standardabweichung zu berechnen?



\begin{enumerate}
\item [\textbf{A} \msquare] Wir berechnen erst den Mittelwert und dann die absoluten Abstände zu dem Mittelwert. Diese quadratischen Abstände summieren wir auf und teilen am Ende durch die Fallzahl $(n-1)$.
\item [\textbf{B} \msquare] Den Mittelwert berechen, dann die quadratischen Abstände zum Mittelwert aufsummieren und durch die Fallzahl $(n-1)$ teilen.
\item [\textbf{C} \msquare] Als erstes berechnen wir den Mittelwert. Dann bilden wir die Summe der quadratischen Abstände zu dem Mittelwert. Abschließend teilen wir durch die Fallzahl $(n-1)$. Nicht zu vergessen, am Ende dann noch die Wurzel zu ziehen.
\item [\textbf{D} \msquare] Den Mittelwert berechen, dann die absoluten Abstände zum Mittelwert aufsummieren. Die Fallzahl $(n-1)$ entsprechend gewichten.
\item [\textbf{E} \msquare] Als erstes berechnen wir den Mittelwert. Dann bilden wir die Summe der quadratischen Abstände zu dem Mittelwert. Abschließend subtrahieren wir die Fallzahl $(n-1)$.
\end{enumerate} 

\section{Aufgabe \hfill (2 Punkte)}

%% --------------------------------------------------------------------
\ifcollection
\begin{flushright}
\tiny\vspace{-2Ex}
\textbf{\examinhaltstart}
\exammodulemathstat $\;\bullet$
\exammodulestat $\;\bullet$
\exammodulestatbbv $\;\bullet$
\exammodulestatversuch $\;\bullet$
\exammodulebiostat
\vspace{-1Ex}
\end{flushright}
\fi
%% --------------------------------------------------------------------




Der \uline{Barplot} stellt folgende statistische Maßzahlen in einer Abbildung dar. Damit gehört der Barplot zu einem der am meisten genutzten statistischen Verfahren zur Visualisierung von Daten.

 



\begin{enumerate}
\item [\textbf{A} \msquare] Der Barplot stellt die Mittelwerte und die Standardabweichung dar.
\item [\textbf{B} \msquare] Der Barplot stellt den Median und die Streuung dar.
\item [\textbf{C} \msquare] Der Barplot stellt den Median und die Quartile dar.
\item [\textbf{D} \msquare] Durch die Abbildung des Barplot erhalten wir die Informationen über die Mittelwerte und die Varianz.
\item [\textbf{E} \msquare] Den Median und die Standardabweichung.
\end{enumerate} 

\section{Aufgabe \hfill (2 Punkte)}

%% --------------------------------------------------------------------
\ifcollection
\begin{flushright}
\tiny\vspace{-2Ex}
\textbf{\examinhaltstart}
\exammodulemathstat $\;\bullet$
\exammodulestat $\;\bullet$
\exammodulestatbbv $\;\bullet$
\exammodulestatversuch $\;\bullet$
\exammodulebiostat
\vspace{-1Ex}
\end{flushright}
\fi
%% --------------------------------------------------------------------




In Ihrer Abschlussarbeit zu Erdbeeren finden Sie aufeinmal seltsame Daten. Jedenfalls kommt Ihnen das so vor. Daher berechnen Sie den Mittelwert mit 11.8 t/ha und den Median mit 12.1 t/ha. Der Mittelwert wird mit $\bar{y}$ und der Median mit $\tilde{y}$ abgekürzt. Welche Aussage ist richtig?



\begin{enumerate}
\item [\textbf{A} \msquare] Wenn sich der Mittelwert und der Median \underline{unterscheiden}, liegen vermutlich Outlier in den Daten vor.
\item [\textbf{B} \msquare] Da sich der Mittelwert und der Median \underline{unterscheiden}, ist der Datensatz nicht zu verwenden. Mittelwert und Median müssen gleich sein.
\item [\textbf{C} \msquare] Nach Ihrer Betrachtung der Werte \underline{unterscheiden} sich Mittelwert und Median \underline{nicht}. Die Daten können also so nicht verwendet werden. Es gibt mindestens Outlier $n-1$ Ausreißer in den Daten.
\item [\textbf{D} \msquare] Da sich der Mittelwert und der Median \underline{unterscheiden}, liegen vermutlich keine Outlier in den Daten vor. Wir verweden den Datensatz so wie er ist.
\item [\textbf{E} \msquare] Nach der Betrachtung der Werte \underline{unterscheiden} sich Mittelwert und Median \underline{nicht}. Die Daten können so verwendet werden wie sie vorliegen. Es gibt keinen Outlier.
\end{enumerate}

\section{Aufgabe \hfill (2 Punkte)}

%% --------------------------------------------------------------------
\ifcollection
\begin{flushright}
\tiny\vspace{-2Ex}
\textbf{\examinhaltstart}
\exammodulestat $\;\bullet$
\exammodulestatbbv $\;\bullet$
\exammodulestatversuch $\;\bullet$
\exammodulebiostat
\vspace{-1Ex}
\end{flushright}
\fi
%% --------------------------------------------------------------------




Die Normalverteilung wird durch den Mittelwert und die Standardabweichung definiert. Welche Aussage im Bezug zur Standardabweichung in einer Normalverteilung ist richtig?



{\centering \includegraphics[width=\maxwidth]{img/mc-distribution-02-a-1} 

}







\begin{enumerate}
\item [\textbf{A} \msquare] Die Fläche links von $-2\sigma$ ist der p-Wert mit $Pr(D|H_0)$ in der obigen Abbildung.
\item [\textbf{B} \msquare] Die Fläche rechts von $2\sigma$ ist der p-Wert mit $Pr(D|H_0)$ in der obigen Abbildung.
\item [\textbf{C} \msquare] Die Fläche zwischen $-1\sigma$ und $1\sigma$ ist 0.68 und 68\% der Beobachtungen liegen somit zwischen $\bar{y}\pm\sigma$ in der obigen Abbildung.
\item [\textbf{D} \msquare] Dargestellt ist keine Standardnormalverteilung.
\item [\textbf{E} \msquare] Die Fläche zwischen $-2\sigma$ und $2\sigma$ ist 0.68 und 68\% der Beobachtungen liegen somit zwischen $\bar{y}\pm\sigma$ in der obigen Verteilung.
\end{enumerate} 

\section{Aufgabe \hfill (2 Punkte)}

%% --------------------------------------------------------------------
\ifcollection
\begin{flushright}
\tiny\vspace{-2Ex}
\textbf{\examinhaltstart}
\exammodulestatversuch $\;\bullet$
\exammodulebiostat
\vspace{-1Ex}
\end{flushright}
\fi
%% --------------------------------------------------------------------




Mit einem Histogramm  können Sie sehr gut die Verteilung von Daten visualisieren. Die empfohlene Mindestanzahl an Beobachtungen ist dabei?



\begin{enumerate}
\item [\textbf{A} \msquare] Wir sollten eine Beobachtung mindestens pro Gruppe vorliegen haben.
\item [\textbf{B} \msquare] Die opimale Anzahl ist größer als zwanzig Beobachtungen, wobei es gerne sehr viel mehr sein können.
\item [\textbf{C} \msquare] 1 Beobachtung.
\item [\textbf{D} \msquare] 10 Beobachtungen.
\item [\textbf{E} \msquare] Wir brauchen fünf oder mehr Beobachtungen.
\end{enumerate}

\section{Aufgabe \hfill (2 Punkte)}

%% --------------------------------------------------------------------
\ifcollection
\begin{flushright}
\tiny\vspace{-2Ex}
\textbf{\examinhaltstart}
\exammodulestatversuch $\;\bullet$
\exammodulebiostat
\vspace{-1Ex}
\end{flushright}
\fi
%% --------------------------------------------------------------------




Sie wollen in Ihrer Abschlussarbeit über eine explorative Datenanalyse überprüfen, ob Ihr gemessener Endpunkt einer Normalverteilung folgt. Welche drei Abbildungen eignen sich insbesondere für die Überprüfung?





\begin{enumerate}
\item [\textbf{A} \msquare] Scatterplot, Mosaicplot, Boxplot
\item [\textbf{B} \msquare] Boxplot, Violinplot, Mosaicplot
\item [\textbf{C} \msquare] Barplot, Mosaicplot, Violinplot
\item [\textbf{D} \msquare] Histogramm, Scatterplot, Boxplot
\item [\textbf{E} \msquare] Densityplot, Boxplot, Violinplot
\end{enumerate}

\section{Aufgabe \hfill (2 Punkte)}


%% --------------------------------------------------------------------
\ifcollection
\begin{flushright}
\tiny\vspace{-2Ex}
\textbf{\examinhaltstart}
\exammodulestatversuch $\;\bullet$
\exammodulebiostat
\vspace{-1Ex}
\end{flushright}
\fi
%% --------------------------------------------------------------------




Um zu Überprüfen, ob die Daten die Annahme einer Varianzhomogenität genügen, können wir folgende Visualisierung nutzen. Dabei kommt dann auch die entsprechende Regel zur Abschätzung der Annahme einer Varianzhomogenität zur Anwendung.



\begin{enumerate}
\item [\textbf{A} \msquare] In einer explorativen Datanalyse nutzen wir den Boxplot. Dabei sollte der Median als dicke Linie in der Mitte der Box liegen. Dann können wir von einer Varianzhomogenität ausgehen.
\item [\textbf{B} \msquare] Einen Boxplot. Das IQR muss über alle Behandlungen zusammen mit den Whiskers ungefähr gleich aussehen.
\item [\textbf{C} \msquare] Wir erstellen uns für jede Behandlung einen Dotplot und schauen, ob die Dots und damit die Varianz für jede Behandlung gleich groß sind.
\item [\textbf{D} \msquare] Einen Dotplot. Die Punkte müssen sich wie an einer Perlenschnurr audreihen. Eine Abweichung führt zur Ablehnung der Annahme einer Varianzhomogenität.
\item [\textbf{E} \msquare] Nach dem Einlesen der Daten nutzen wir einen Barplot um zu schauen, ob alle Mittelwerte über alle Behandlungen in etwa gleich groß sind. Damit ist dann auch die Varianz in allen Behandlungen in etwa gleich.
\end{enumerate}
\section*{Statistische Testtheorie}  

\section{Aufgabe \hfill (2 Punkte)}

%% --------------------------------------------------------------------
\ifcollection
\begin{flushright}
\tiny\vspace{-2Ex}
\textbf{\examinhaltstart}
\exammodulemathstat $\;\bullet$
\exammodulestat $\;\bullet$
\exammodulestatbbv $\;\bullet$
\exammodulestatversuch $\;\bullet$
\exammodulebiostat
\vspace{-1Ex}
\end{flushright}
\fi
%% --------------------------------------------------------------------




Sie haben den mathematischen Ausdruck $Pr(D|H_0)$ vorliegen, welche Aussage ist richtig?



\begin{enumerate}
\item [\textbf{A} \msquare] $Pr(D|H_0)$ ist die Wahrscheinlichkeit die Daten $D$ zu beobachten, wenn die Nullhypothese wahr ist.
\item [\textbf{B} \msquare] $Pr(D|H_0)$ ist die Wahrscheinlichkeit der Alternativehypothese und somit $1 - Pr(H_A)$
\item [\textbf{C} \msquare] $Pr(D|H_0)$ ist die Wahrscheinlichkeit nicht die Daten $D$ zu beobachten sondern die Nullhypothese, wenn diese wahr ist.
\item [\textbf{D} \msquare] Die Wahrscheinlichkeit der Daten unter der Nullhypothese in der Grundgesamtheit.
\item [\textbf{E} \msquare] $Pr(D|H_0)$ stellt die Wahrscheinlichkeit die Teststatistik $T$ zu beobachten dar, wenn die Nullhypothese falsch ist.
\end{enumerate} 

\section{Aufgabe \hfill (2 Punkte)}

%% --------------------------------------------------------------------
\ifcollection
\begin{flushright}
\tiny\vspace{-2Ex}
\textbf{\examinhaltstart}
\exammodulemathstat $\;\bullet$
\exammodulestat $\;\bullet$
\exammodulestatbbv $\;\bullet$
\exammodulestatversuch $\;\bullet$
\exammodulebiostat
\vspace{-1Ex}
\end{flushright}
\fi
%% --------------------------------------------------------------------




Das Signifikanzniveau $\alpha$ wird auch Fehler 1. Art genannt und liegt bei 5\%. Warum wurde der Grenzwert von 5\% als Signifikanzschwelle gewählt?



\begin{enumerate}
\item [\textbf{A} \msquare] Da Wissenschaftler eine Schwelle für die statistische Testentscheidung benötigen wurde $\alpha$ in einer großen Konferenz 1945 gewählt. Damit ist $\alpha = 5\%$ eine Kulturkonstante mit einem Rank einer Naturkonstante.
\item [\textbf{B} \msquare] Auf einer Statistikkonferenz in Genf im Jahre 1942 wurde dieser Cut-Off nach langen Diskussionen festgelegt. Bis heute ist der Cut Off aber umstritten, da wegen dem 2. Weltkrieg viele Wissenschaftler nicht teilnehmen konnten.
\item [\textbf{C} \msquare] Der Begründer der modernen Statistik, R. Fischer, hat die Grenze simuliert und berechnet. Dadurch ergibt sich dieser optimale Cut-Off.
\item [\textbf{D} \msquare] Der Wert ergab sich aus einer Auswertung von 1042 wissenschaftlichen Veröffentlichungen zwischen 1914 und 1948. Der Wert $5\%$ wurde in $28\%$ der Veröffentlichungen genutzt. Daher legte man sich auf diese Zahl fest.
\item [\textbf{E} \msquare] Da Wissenschaftler eine Schwelle für die statistische Testentscheidung benötigen wurde $\alpha$ historisch gewählt. Damit ist $\alpha = 5\%$ eine Kulturkonstante.
\end{enumerate} 

\section{Aufgabe \hfill (2 Punkte)}

%% --------------------------------------------------------------------
\ifcollection
\begin{flushright}
\tiny\vspace{-2Ex}
\textbf{\examinhaltstart}
\exammodulemathstat $\;\bullet$
\exammodulestat $\;\bullet$
\exammodulestatbbv $\;\bullet$
\exammodulestatversuch $\;\bullet$
\exammodulebiostat
\vspace{-1Ex}
\end{flushright}
\fi
%% --------------------------------------------------------------------




Das statistische Testen basiert auf dem Falsifikationsprinzip. Es besagt,



\begin{enumerate}
\item [\textbf{A} \msquare] ... dass ein schlechtes Modell durch das Falsifikationsprinzip durch ein weniger schlechtes Modell ersetzt wird.
\item [\textbf{B} \msquare] ... dass ein minderwertes Modell durch ein minderwertiges Modell ersetzt wird. Es gilt das Verifikationsprinzip nach Karl Popper.
\item [\textbf{C} \msquare] ... dass Annahmen an statistische Modelle meist falsch sind.
\item [\textbf{D} \msquare] ... dass Fehlerterme in statistischen Modellen nicht verifiziert werden können.
\item [\textbf{E} \msquare] ... dass ein schlechtes Modell durch das Falsifikationsprinzip durch ein noch schlechteres Modell ersetzt wird. Die Wissenschaft lehnt ab und verifiziert nicht.
\end{enumerate}

\section{Aufgabe \hfill (2 Punkte)}

%% --------------------------------------------------------------------
\ifcollection
\begin{flushright}
\tiny\vspace{-2Ex}
\textbf{\examinhaltstart}
\exammodulemathstat $\;\bullet$
\exammodulestat $\;\bullet$
\exammodulestatbbv $\;\bullet$
\exammodulestatversuch $\;\bullet$
\exammodulebiostat
\vspace{-1Ex}
\end{flushright}
\fi
%% --------------------------------------------------------------------




Betrachten wir die Teststatistik aus einem abstrakteren Blickwinkel. Beim statistischen Testen wird das 	extit{signal} mit dem 	extit{noise} aus den Daten $D$ zu einer Teststatistik $T_D$ verrechnet. Welche der Formel berechnet korrekt die Teststatistik $T_D$?



\begin{enumerate}
\item [\textbf{A} \msquare] Es gilt $T_D = \sfrac{signal}{noise}$. Der Effekt $noise$ wird mit der Varianz $signal$ gewichtet.
\item [\textbf{B} \msquare] Es gilt $T_D = signal \cdot noise$. Der Effekt $signal$ wird mit der Varianz $noise$ gewichtet.
\item [\textbf{C} \msquare] Wir gewichten den Effekt $noise$ mit der Varianz $signal$ und erhalten $\sfrac{signal}{noise^2}$.
\item [\textbf{D} \msquare] Es gilt $T_D = \sfrac{noise}{signal}$. Der Effekt $noise$ wird mit der Varianz $signal$ gewichtet.
\item [\textbf{E} \msquare] Es gilt $T_D = \sfrac{signal}{noise}$. Der Effekt $signal$ wird mit der Varianz $noise$ gewichtet.
\end{enumerate}

%% ------------------------------------------------------------

\section{Aufgabe \hfill (2 Punkte)}

%% --------------------------------------------------------------------
\ifcollection
\begin{flushright}
\tiny\vspace{-2Ex}
\textbf{\examinhaltstart}
\exammodulemathstat $\;\bullet$
\exammodulestat $\;\bullet$
\exammodulestatbbv $\;\bullet$
\exammodulestatversuch $\;\bullet$
\exammodulebiostat
\vspace{-1Ex}
\end{flushright}
\fi
%% --------------------------------------------------------------------




Sie haben ein Signifikanzniveau $\alpha$ gleich 5\% vorliegen. Welche Aussage zusammen mit dem $p$-Wert ist richtig?



\begin{enumerate}
\item [\textbf{A} \msquare] Wir vergleichen mit dem $p$-Wert und dem Signifikanzniveau $\alpha$ Wahrscheinlichkeiten und damit die absoluten Werte auf einem Zahlenstrahl, wenn die $H_0$ gilt.
\item [\textbf{B} \msquare] Wir schauen, ob der $p$-Wert größer ist als das Signifikanzniveau $\alpha$ und vergleichen somit Wahrscheinlichkeiten. Die Wahrscheinlichkeiten werden als Flächen unter der Kurve der Teststaistik dargestellt, wenn die $H_A$ gilt.
\item [\textbf{C} \msquare] Wir machen eine Aussage über die indivduelle Wahrscheinlichkeit des Eintretens der Nullhypothese $H_0$. Der $p$-Wert wird mit dem Signifikanzniveau verglichen und bewertet.
\item [\textbf{D} \msquare] Wir machen ein Aussage über die Flächen und der Kurve der Teststatistik, wenn die $H_0$ gilt. Dabei werden Wahrscheinlichkeiten vergleichen, die durch die Flächen unter der Kurve repräsentiert werden.
\item [\textbf{E} \msquare] Wir vergleichen die Effekte des $p$-Wertes mit den Effekten der Signifikanzschwelle unter der Annahme der Nullhypothese. Dabei gilt, dass wir die Nullhypothese nur ablehnen können anhand des Falsifikationsprinzips.
\end{enumerate}

\section{Aufgabe \hfill (2 Punkte)}

%% --------------------------------------------------------------------
\ifcollection
\begin{flushright}
\tiny\vspace{-2Ex}
\textbf{\examinhaltstart}
\exammodulemathstat $\;\bullet$
\exammodulestat $\;\bullet$
\exammodulestatbbv $\;\bullet$
\exammodulestatversuch $\;\bullet$
\exammodulebiostat
\vspace{-1Ex}
\end{flushright}
\fi
%% --------------------------------------------------------------------




Die Ergebnisse der einer statistischen Analyse können in die Analogie einer Wettervorhersage gebracht werden. Welche Analogie für die Ergebnisse eines statistischen Tests trifft am besten zu?



\begin{enumerate}
\item [\textbf{A} \msquare] In der Analogie der Regenwahrscheinlichkeit in einem bestimmten Gebiet: ein statistischer Test gibt die Wahrscheinlichkeit für ein Ereignis in einem Experiment mit den Daten $D$ wieder und lässt sich kaum verallgemeinern.
\item [\textbf{B} \msquare] In der Analogie der Durchschnittstemperatur: Wie oft tritt ein Effekt durchschnittlich ein? Wir erhalten eine Wahrscheinlichkeit für die Effekte. Zum Beispiel, wie hoch ist die Wahrscheinlichkeit für einen Mittelwert als Durchschnitt.
\item [\textbf{C} \msquare] In der Analogie des Niederschlags oder Regenmenge: ein statistischer Test gibt nur die Stärke eines Effektes wieder. Zum Beispiel, wie hoch ist der Mittelwertsunterschied. Der Effekt wird nicht bewertet.
\item [\textbf{D} \msquare] Die Analogie der Regenwahrscheinlichkeit: der statistische Test erlaubt es die Wahrscheinlichkeit für Regen abzuschätzen jedoch nicht die Menge und somit den Effekt.
\item [\textbf{E} \msquare] In der Analogie der Maximaltemperatur: Was ist der maximale Unterschied zwischen zwei Gruppen. Wir erhalten hier eine Aussage über die Spannweite und den maximalen Effekt.
\end{enumerate}

\section{Aufgabe \hfill (2 Punkte)}

%% --------------------------------------------------------------------
\ifcollection
\begin{flushright}
\tiny\vspace{-2Ex}
\textbf{\examinhaltstart}
\exammodulemathstat $\;\bullet$
\exammodulestat $\;\bullet$
\exammodulestatbbv $\;\bullet$
\exammodulestatversuch $\;\bullet$
\exammodulebiostat
\vspace{-1Ex}
\end{flushright}
\fi
%% --------------------------------------------------------------------




In Ihrer Forschungsarbeit wollen Sie eine Aussage über die untersuchte Population treffen. Dazu nutzen Sie einen statistischen Test. Erhalten Sie eine valide Aussage aus einem statistischen Test?



\begin{enumerate}
\item [\textbf{A} \msquare] Nein, es ist nicht möglich die untersuchte Population mit einem statistischen Test auszuwerten. Wir erhalten dann leider keine Aussage zur Population.
\item [\textbf{B} \msquare] Nein, wir erhalten eine \textit{einfache} Aussage. Wir müssen daher das Individuum im Kontext der Population adjustieren.
\item [\textbf{C} \msquare] Weder eine Ausssage über die Population noch über das Individuum ist mit einem statistischen Test möglich. Wir erhalten eine Aussage über ein Experiment.
\item [\textbf{D} \msquare] Ja, wir können die untersuchte Population nicht mit einem t-Test auswerten. Wir erhalten keine Aussage zur Population. Wir können aber den Effekt als Quelle der Relevanz nutzen.
\item [\textbf{E} \msquare] Ja, die untersuchte Population können wir mit einem statistischen Test auswerten. Wir erhalten dann eine Aussage zur Population.
\end{enumerate}

\section{Aufgabe \hfill (2 Punkte)}

%% --------------------------------------------------------------------
\ifcollection
\begin{flushright}
\tiny\vspace{-2Ex}
\textbf{\examinhaltstart}
\exammodulemathstat $\;\bullet$
\exammodulestat $\;\bullet$
\exammodulestatbbv $\;\bullet$
\exammodulestatversuch $\;\bullet$
\exammodulebiostat
\vspace{-1Ex}
\end{flushright}
\fi
%% --------------------------------------------------------------------




Welche Aussage über den Effekt eines statistischen Tests ist richtig?



\begin{enumerate}
\item [\textbf{A} \msquare] Der Effekt eines statistischen Tests beschreibt die biologisch interpretierbare Ausgabe eines Tests. Moderen Algorithmen liefern keine Effekte mehr sondern nur noch bedingte Wahrscheinlichkeiten. Der Effekt spielt in der modernen Statistik keine Rollen mehr.
\item [\textbf{B} \msquare] Durch den Effekt erfahren wir die biologisch interpretierbare Ausgabe eines statistischen Tests. Zum Beispiel das $\eta^2$ aus einer ANOVA. Damit können wir die Relevanz direkt mit dem Effekt verbinden. Am Ende muss der Forschende aber entscheiden, ob der Effekt entsprechend seinen Erwartungen als bedeutet zu bewerten ist.
\item [\textbf{C} \msquare] Der Forschende muss am Anfang wissen, ob das Eregbnis eines Experiments relevant für seine Forschung ist. Dafür kann der Effekt eines statistischen Tests genutzt werden oder auch der Prähoc-Test. Damit beschreibt der Effekt den biologischen interpretierbaren Teil eines Experimnts vor der Durchführung. Zum Beispiel der Unterschied zwischen zwei Mittelwerten.
\item [\textbf{D} \msquare] Der Effekt eines statistischen Tests beschreibt den Output oder die Wiedergabe eines Tests in einem Computer.
\item [\textbf{E} \msquare] Der Effekt eines statistischen Tests beschreibt die mathematisch interpretierbare Ausgabe eines Tests. Damit ist der Effekt direkt mit dem Begriff der Signifikanz verbunden. Die Entscheidung über die Signifikanz trifft der Forschende unabhängig von der Relevanz eines statistsichen Tests.
\end{enumerate}

\section{Aufgabe \hfill (2 Punkte)}

%% --------------------------------------------------------------------
\ifcollection
\begin{flushright}
\tiny\vspace{-2Ex}
\textbf{\examinhaltstart}
\exammodulemathstat $\;\bullet$
\exammodulestat $\;\bullet$
\exammodulestatbbv $\;\bullet$
\exammodulestatversuch $\;\bullet$
\exammodulebiostat
\vspace{-1Ex}
\end{flushright}
\fi
%% --------------------------------------------------------------------



Roland Fischer entwickelte Anfang des letzten Jahrhunderts als Grundlage für das experimentelle Design in der Statistik die Randomisierung. Warum ist die Randomisierung für die Entscheidung anhand einer statistischen Auswertung so wichtig?



\begin{enumerate}
\item [\textbf{A} \msquare] Strukturgleichheit ist durch Randomisierung gegeben. Leider hilft die Randomisierung noch nicht um von der Stichprobe auf die Grundgesamtheit zu schließen. Deshalb wurde das Falsifikationsprinzip entwickelt.
\item [\textbf{B} \msquare] Durch eine Randomisierung können wir von Strukturgleichheit zwischen der Stichprobe und der Grundgesamtheit ausgehen.
\item [\textbf{C} \msquare] Randomisierung erlaubt erst die Varianzen zu schätzen. Ohne eine Randomisierung ist die Berechnung von Mittelwerten und Varianzen nicht möglich. Dadurch lässt sich erst ein Experiment auswerten.
\item [\textbf{D} \msquare] Randomisierung bringt starke Unstrukturiertheit in das Experiment und erlaubt erst von der Stichprobe auf die Grundgesamtheit zurückzuschliessen.
\item [\textbf{E} \msquare] Durch eine Randomisierung können wir nicht von Strukturgleichheit zwischen der Stichprobe und der Grundgesamtheit ausgehen.
\end{enumerate}

\section{Aufgabe \hfill (2 Punkte)}

%% --------------------------------------------------------------------
\ifcollection
\begin{flushright}
\tiny\vspace{-2Ex}
\textbf{\examinhaltstart}
\exammodulemathstat $\;\bullet$
\exammodulestat $\;\bullet$
\exammodulestatbbv $\;\bullet$
\exammodulestatversuch $\;\bullet$
\exammodulebiostat
\vspace{-1Ex}
\end{flushright}
\fi
%% --------------------------------------------------------------------




Ein statistischer Test benötigt für die richtige Durchführung Hypothesen $H$, sonst ist der Test nicht zu interpretieren. Welche Aussage ist richtig?



\begin{enumerate}
\item [\textbf{A} \msquare] Mit der Nullhypothese $H_0$ und der Alternativehypothese $H_A$ oder $H_1$ gibt es zwei Hypothesen.
\item [\textbf{B} \msquare] Es gibt ein Hypothesenset bestehend aus $k$ Hypothesen. Meistens wird die Nullhypothese $H_0$ und die Alternativhypothese $H_A$ verwendet. Wegen des Falsifikationsprinzips ist es wichtig, die bekannte falsche und unbekannte richtige Hypothese mit in das Set zu nehmen.
\item [\textbf{C} \msquare] Es gibt ein statistisches Hypothesenpaar mit der Hypothese für und gegen die wissenschaftliche Fragestellung. Die Hypothesen werden $H_{pro}$ und $H_{contra}$ bezeichnet.
\item [\textbf{D} \msquare] Ein statistisches Hypothesenpaare gibt es. Zum einen die Nullhypothese und zum anderen die Alternativehypothese. Es ist aber nur notwendig die Alternative anzugeben, da die Nullhypothese nicht beim Testen benötigt wird.
\item [\textbf{E} \msquare] Mit der Nullhypothese $H_A$ und der Alternativehypothese $H_0$ gibt es zwei Hypothesen, die aber selten genutzt werden.
\end{enumerate}

\section{Aufgabe \hfill (2 Punkte)}

%% --------------------------------------------------------------------
\ifcollection
\begin{flushright}
\tiny\vspace{-2Ex}
\textbf{\examinhaltstart}
\exammodulemathstat $\;\bullet$
\exammodulestat $\;\bullet$
\exammodulestatbbv $\;\bullet$
\exammodulestatversuch $\;\bullet$
\exammodulebiostat
\vspace{-1Ex}
\end{flushright}
\fi
%% --------------------------------------------------------------------




In der Theorie zur statistischen Testentscheidung kann folgende Aussage
in welche richtige Analogie gesetzt werden?

\begin{center}
\textit{$H_0$ beibehalten obwohl die $H_0$ falsch ist}
\end{center}



\begin{enumerate}
\item [\textbf{A} \msquare] \textit{Alarm without fire}, dem $\alpha$-Fehler in der Analogie eines Rauchmelders.
\item [\textbf{B} \msquare] In die Analogie eines Rauchmelders: \textit{Fire without alarm}, dem $\beta$-Fehler.
\item [\textbf{C} \msquare] \textit{Alarm with fire}, dem $\alpha$-Fehler in der Analogie von Feuer.
\item [\textbf{D} \msquare] In die Analogie eines Rauchmelders: \textit{Alarm without fire police}, dem $\alpha$-Fehler.
\item [\textbf{E} \msquare] In die Analogie eines Feuerwehrautos: \textit{Car without noise}.
\end{enumerate}

\section{Aufgabe \hfill (2 Punkte)}

%% --------------------------------------------------------------------
\ifcollection
\begin{flushright}
\tiny\vspace{-2Ex}
\textbf{\examinhaltstart}
\exammodulestat $\;\bullet$
\exammodulestatbbv $\;\bullet$
\exammodulestatversuch $\;\bullet$
\exammodulebiostat
\vspace{-1Ex}
\end{flushright}
\fi
%% --------------------------------------------------------------------




Welche statistische Maßzahl erlaubt es Relevanz mit Signifikanz zu verbinden? Welche Aussage ist richtig?



\begin{enumerate}
\item [\textbf{A} \msquare] Das $\Delta$. Durch die Effektstärke haben wir einen Wert für die Relevanz, die vom Anwender bewertet werden muss. Da $\Delta$ antiproportional zum p-Wert ist, bedeutet auch ein hohes $\Delta$ ein sehr kleinen p-Wert.
\item [\textbf{B} \msquare] Das OR. Als Chancenverhältnis gibt es das Verhältnis von Relevanz und Signifikanz wieder.
\item [\textbf{C} \msquare] Die Teststatistik. Durch den Vergleich von $T_c$ zu $T_k$ ist es m{"o}glich die $H_0$ abzulehnen. Die Relevanz ergibt sich aus der Fläche rechts vom dem $T_c$-Wert.
\item [\textbf{D} \msquare] Einem Konfidenzintervall. Das Konfidenzinterval bringt durch eine Visualisierung und zwei Intervallgrenzen die Möglichkeit mit, eine Relevanzschwelle neben der definierten Signifikanzschwelle zu definieren.
\item [\textbf{E} \msquare] Über das Konfidenzintervall. Das Konfidenzinterval inkludiert eine Entscheidung über die Relevanz und zusätzlich kann über die Visualizierung des Konfidenzintervals eine Signifikanzschwelle vom Forschenden definiert werden.
\end{enumerate}

\section{Aufgabe \hfill (2 Punkte)}

%% --------------------------------------------------------------------
\ifcollection
\begin{flushright}
\tiny\vspace{-2Ex}
\textbf{\examinhaltstart}
\exammodulestatversuch $\;\bullet$
\exammodulebiostat
\vspace{-1Ex}
\end{flushright}
\fi
%% --------------------------------------------------------------------




Sie werten in Ihrer Abschlussarbeit einen sehr großen Datensatz aus einer öffentlichen Datenbank aus. Nun stellen Sie fest, dass Sie ein Problem mit der Bewertung Ihrer Ergbnisse anhand der Signifikanz bekommen. Wie Sie herausfinden, scheint dies ein häufiges Problem in der Bio Data Science zu sein. Welche Aussage ist richtig?




\begin{enumerate}
\item [\textbf{A} \msquare] Riesige Datensätz haben mehr Fallzahl was zur $\alpha$-Inflation führt. Durch eine Adjustoerung kann dem Problem entgegengewirkt werden.
\item [\textbf{B} \msquare] Relevanz und Signifikanz haben nichts miteinander zu tun. Daher gibt es auch keinen Zusammenhang zwischen hoher Fahlzahl (n > 10000) und einem signifikanten Test. Ein Effekt ist immer relevant und somit signifikant.
\item [\textbf{C} \msquare] Mehr Fallzahl in Datensätzen bedeutet mehr signifikante Ergebnisse, da in mehr Daten auch mehr Informationen beinhaltet sind. Deshalb lohnen sich riesige Datensätze, die durch die vielen signifikanten Ergebnisse auch eine Menge an relevanten Erkenntnissen liefern.
\item [\textbf{D} \msquare] Aktuell werden immer größere Datensätze erhoben. Eine erhöhte Fallzahl führt automatisch auch zu mehr signifikanten Ergebnissen, selbst wenn die eigentlichen Effekte nicht relevant sind.
\item [\textbf{E} \msquare] Aktuell werden zu grosse Datensätze für die gänigige Statistik gemessen. Daher wendet man maschinelle Lernverfahren für kausale Modelle an. Hier ist die Relevanz gleich Signifikanz.
\end{enumerate}

\section{Aufgabe \hfill (2 Punkte)}

%% --------------------------------------------------------------------
\ifcollection
\begin{flushright}
\tiny\vspace{-2Ex}
\textbf{\examinhaltstart}
\exammodulestatversuch $\;\bullet$
\exammodulebiostat
\vspace{-1Ex}
\end{flushright}
\fi
%% --------------------------------------------------------------------




Die Leistung von Sauen soll auf einem Zuchtbetrieb gesteigert werden. Dafür werden die Ferkel verschiedener Sauen gemessen. Die Ferkel einer Muttersaue sind daher...



\begin{enumerate}
\item [\textbf{A} \msquare] Untereinander unabhängig. Sollten die Mütter verwandt sein, so ist die Varianzstruktur ähnlich und muss modelliert werden.
\item [\textbf{B} \msquare] Untereinander abhängig. Die Ferkel stammen von einem Muttertier und haben vermutliche eine ähnliche Varianzstruktur.
\item [\textbf{C} \msquare] Abhängig von der Stallanlage und des Experiments können die Ferkel abhängig oder unabhängig sein. Allgmein gilt, dass Ferkel von unterschiedlichen Sauen näher miteinander verwandt sind als Ferkel von gleichen Sauen. Das Fisher-Axiom.
\item [\textbf{D} \msquare] Die Ferkel stammen von der gleichen Sau und sind somit untereinander unabhängig.
\item [\textbf{E} \msquare] Je nach Stallanlage kommt eine andere Analyse in Betracht. Eine allgemeine Aussage über Ferkel und Sauen lässt sich statistisch nicht treffen.
\end{enumerate}

\section{Aufgabe \hfill (2 Punkte)}

%% --------------------------------------------------------------------
\ifcollection
\begin{flushright}
\tiny\vspace{-2Ex}
\textbf{\examinhaltstart}
\exammodulestatversuch $\;\bullet$
\exammodulebiostat
\vspace{-1Ex}
\end{flushright}
\fi
%% --------------------------------------------------------------------






Nachdem Sie einen t-Test gerechnet haben stehen Ihnen verschiedene statistische Maßzahlen zu Verfügung um eine Entscheidung gegen die Nullhypothese zu treffen. Sie entscheiden sich anhand des p-Wertes eine Entscheidung zu treffen.





\begin{enumerate}
\item [\textbf{A} \msquare] Das Signifikanzniveauintervall $\alpha$ ist gleich $5\%$ und damit muss der berrechnete Wert unter dem Signifikanzniveauintervall $\alpha$ liegen, dann kann die Nullhypothese nicht abgelehnt werden.
\item [\textbf{B} \msquare] Anhand des p-Wertes lässt sich wie folgt eine Entscheidung treffen. Ist die Null mit enthalten, dann kann die Nullhypothese nicht abgelehnt werden.
\item [\textbf{C} \msquare] Liegt der berrechnete Wert über dem kritische Wert $T_{\alpha = 5\%}$, dann kann die Nullhypothese nicht abgelehnt werden.
\item [\textbf{D} \msquare] Das Signifikanzniveau $\alpha$ ist gleich $5\%$. $Pr(D|H_0)$ muss kleiner sein als das Signifikanzniveau.
\item [\textbf{E} \msquare] Der kritische Wert $T_{\alpha = 5\%}$ ist tabelarisch festgelegt und gegeben. Ist $T_{D}$ größer als der kritische Wert, kann die Nullhypothese abgelehnt werden
\end{enumerate}

\section{Aufgabe \hfill (2 Punkte)}

%% --------------------------------------------------------------------
\ifcollection
\begin{flushright}
\tiny\vspace{-2Ex}
\textbf{\examinhaltstart}
\exammodulebiostat
\vspace{-1Ex}
\end{flushright}
\fi
%% --------------------------------------------------------------------




In der statistischen Testtheorie gibt es den Begriff \textit{Power}. Was sagt der statistische Begriff \textit{Power} aus?



\begin{enumerate}
\item [\textbf{A} \msquare] Alle statistischen Tests sind so konstruiert, dass die $H_A$ mit 20\% \textit{bewiesen wird}. Die Power ist $1-\beta$ mit $\beta$ gleich 80\% gesetzt.
\item [\textbf{B} \msquare] Die Power beschreibt die Wahrscheinlichkeit die $H_A$ abzulehnen. Wir testen die Power jedoch nicht.
\item [\textbf{C} \msquare] Die Power wird berechnet und ist keine Eigenschaft des Tests. Die Power wird auf $80\%$ gesetzt und beschreibt mit welcher Wahrscheinlichkeit $H_0$ \textit{bewiesen wird}
\item [\textbf{D} \msquare] Die Power ist nicht in der aktuellen Testthorie mehr vertreten. Wir rechnen nur noch mit dem Fehler 1. Art.
\item [\textbf{E} \msquare] Alle statistischen Tests sind so konstruiert, dass die $H_A$ mit 80\% \textit{bewiesen wird}. Die Power ist $1-\beta$ mit $\beta$ gleich 20\% gesetzt.
\end{enumerate}
\section*{Statistische Tests für Gruppenvergleiche} 

\section{Aufgabe \hfill (2 Punkte)}

%% --------------------------------------------------------------------
\ifcollection
\begin{flushright}
\tiny\vspace{-2Ex}
\textbf{\examinhaltstart}
\exammodulemathstat $\;\bullet$
\exammodulestat $\;\bullet$
\exammodulestatbbv 
\vspace{-1Ex}
\end{flushright}
\fi
%% --------------------------------------------------------------------




In Ihrer Abschlussarbeit rechnen Sie einen Student t-Test. Welche Aussage ist auch für den Welch t-Test richtig?



\begin{enumerate}
\item [\textbf{A} \msquare] Der t-Test vergleicht die Mittelwerte von zwei Gruppen unter der strikten Annahme von Varianzhomogenität. Sollte keine Varianzhomogenität vorliegen, so gibt es keine Möglichkeit den t-Test in einer Variante anzuwenden.
\item [\textbf{B} \msquare] Der t-Test berechnet die Differenz von zwei Mittelwerten als Effekt und gibt eine Entscheidung, ob sich die beiden Mittelwerte \textit{jeweils} von Null unterscheiden.
\item [\textbf{C} \msquare] Der t-Test vergleicht zwei Gruppen indem die Mittelwerte miteinander verglichen werden.
\item [\textbf{D} \msquare] Der t-Test ist ein Vortest der ANOVA und basiert daher auf dem Vergleich von Streuungsparametern
\item [\textbf{E} \msquare] Der t-Test testet generell zu einem erhöhten $\alpha$-Niveau von 20\%.
\end{enumerate}

\section{Aufgabe \hfill (2 Punkte)}

%% --------------------------------------------------------------------
\ifcollection
\begin{flushright}
\tiny\vspace{-2Ex}
\textbf{\examinhaltstart}
\exammodulestatversuch $\;\bullet$
\exammodulebiostat
\vspace{-1Ex}
\end{flushright}
\fi
%% --------------------------------------------------------------------





Welche Aussage über den gepaarten t-Test für verbundene Stichproben ist richtig?



\begin{enumerate}
\item [\textbf{A} \msquare] Beim gepaarten t-Test kombinieren wir die Vorteile des Student t-Test für Varianzhomogenität mit den Vorteilen des Welch t-Test für Varianzheterogenität. Wir bilden dafür die Differenz der Einzelbeobachtungen.
\item [\textbf{B} \msquare] Der gepaarte t-Test wird genutzt, wenn die Differenzen der Beobachtungen verbunden sind und wir dadurch die Unabhäängigkeit nicht mehr vorliegen haben.
\item [\textbf{C} \msquare] Wenn die Beobachtungen unabhängig voneinander sind, rechnen wir einen gepaarten t-Test. Messen wir wiederholt an dem gleichen Tier oder Pflanze dann bilden wir das Produkt zwischen den zwei Messpunkten.
\item [\textbf{D} \msquare] Der gepaarte t-Test nutzt die Varianz der Beobachtungen jeweils paarweise und bildet dafür eine verbundene Stichprobe. Dieser Datensatz $d$ dient dann zur Differenzbildung.
\item [\textbf{E} \msquare] Wenn die Beobachtungen nicht unabhängig voneinander sind, rechnen wir einen gepaarten t-Test. Messen wir wiederholt an dem gleichen Tier oder Pflanze dann bilden wir die Differenz zwischen den zwei Messpunkten.
\end{enumerate}

\section{Aufgabe \hfill (2 Punkte)}

%% --------------------------------------------------------------------
\ifcollection
\begin{flushright}
\tiny\vspace{-2Ex}
\textbf{\examinhaltstart}
\exammodulemathstat $\;\bullet$
\exammodulestat $\;\bullet$
\exammodulestat $\;\bullet$
\exammodulestatbbv $\;\bullet$
\exammodulestatversuch $\;\bullet$
\exammodulebiostat
\vspace{-1Ex}
\end{flushright}
\fi
%% --------------------------------------------------------------------




Die folgende Abbildung enthält die Daten aus einer Studie zur Bewertung der Wirkung des Mikronährstoff Sulfit auf den Ertrag in t/ha von Weizen im Vergleich zu einer Kontrolle. Der Versuch wurde in 7 Parzellen pro Gruppe durchgeführt. Welche Aussage im Bezug auf eine statistische Auswertung ist richtig?



{\centering \includegraphics[width=\maxwidth]{img/mc-testing-ttest-02-1} 

}







\begin{enumerate}
\item [\textbf{A} \msquare] Nach Betrachtung des Barplots liegt kein signifikanter Unterschied vor. Der Effekt kann nicht bei einem t-Test aus Barplots bestimmt werden.
\item [\textbf{B} \msquare] Der Effekt und die Signifikanz lassen sich nicht aus Barplots abschätzen. Höchtens der Effekt als relativer Unterschied zwischen der Höhe der Barplots. Standard ist der mediane Unterschied aus Boxplots.
\item [\textbf{C} \msquare] Die Barplots deuten auf kein signifikanten Unterschied. Der Effekt liegt vermutlich bei -7.
\item [\textbf{D} \msquare] Die Barplots deuten auf einen signifikanten Unterschied. Der Effekt liegt vermutlich bei -7 unter einer groben Abschätzung.
\item [\textbf{E} \msquare] Die Barplots deuten auf einen signifikanten Unterschied. Der Effekt liegt vermutlich bei -7 unter einer groben Abschätzung. Wir müssen aber eine ANOVA rechnen um den Effekt wirklich bestimmen zu können.
\end{enumerate}

\section{Aufgabe \hfill (2 Punkte)}

%% --------------------------------------------------------------------
\ifcollection
\begin{flushright}
\tiny\vspace{-2Ex}
\textbf{\examinhaltstart}
\exammodulestatversuch $\;\bullet$
\exammodulebiostat
\vspace{-1Ex}
\end{flushright}
\fi
%% --------------------------------------------------------------------




Nach einem Experiment mit vier Weizensorten ergibt eine ANOVA ($p = 0.049$) einen signifikanten Unterschied für den Ertrag. Sie führen anschließend die paarweisen t-Tests für alle Vergleiche der verschiedenen Weizensorten durch. Nach der Adjustierung für multiples Testen ist kein p-Wert unter der $\alpha$-Schwelle. Sie schauen sich auch die rohen, unadjustierten p-Werte an und finden hier als niedrigsten p-Wert $p_{3-2} = 0.053$. Welche Aussage ist richtig?




\begin{enumerate}
\item [\textbf{A} \msquare] Hier kommt der Effekt der stiegenden Fallzahl auf die Anzahl an signifikante Ergebnisse zu tragen. Da die ANOVA auf mehr Fallzahl testet als die einzelnen paarweisen t-Tests, kann die ANOVA leichter einen signifikanten Unterscheid nachweisen. Die p-Werte sind immer etwas kleiner als bei den t-Tests.
\item [\textbf{B} \msquare] Der Fehler liegt in den t-Tests. Wenn eine ANOVA signifikant ist, dann muss zwangsweise auch ein t-Test signifikant sein.
\item [\textbf{C} \msquare] Das Beispiel kann so nicht auftreten, da die ANOVA und die t-Tests algorithmisch miteinander verschränkt sind.
\item [\textbf{D} \msquare] Es gibt einen Fehler in der Varianzstruktur. Daher kann die ANOVA nicht richtig sein und paarweise t-Tests liefern das richtige Ergebnis.
\item [\textbf{E} \msquare] Die ANOVA testet auf der gesamten Fallzahl. Es wäre besser die ANOVA auf der gleichen Fallzahl wie die einzelnen t-Tests zu rechnen.
\end{enumerate}
\section*{ANOVA}

\section{Aufgabe \hfill (2 Punkte)}

%% --------------------------------------------------------------------
\ifcollection
\begin{flushright}
\tiny\vspace{-2Ex}
\textbf{\examinhaltstart}
\exammodulemathstat $\;\bullet$
\exammodulestat $\;\bullet$
\exammodulestatbbv $\;\bullet$
\exammodulestatversuch $\;\bullet$
\exammodulebiostat
\vspace{-1Ex}
\end{flushright}
\fi
%% --------------------------------------------------------------------






Nach der Berechnung einer einfaktoriellen ANOVA ergibt sich ein $\eta^2 = 0.78$. Welche Aussage ist richtig?



\begin{enumerate}
\item [\textbf{A} \msquare] Das $\eta^2$ beschreibt den Anteil der globalen Varianz, der von den Umweltbedingungen erklärt wird.
\item [\textbf{B} \msquare] Das $\eta^2$ beschreibt den Anteil der Varianz, der von den Behandlungsbedingungen nicht erklärt wird. Somit der Rest an nicht erklärbarer Varianz.
\item [\textbf{C} \msquare] Das $\eta^2$ wird genutzt um zu erfahren welchen Anteil der Varianz die Behandlungsbedingungen erklären.
\item [\textbf{D} \msquare] Die Berechnung von $\eta^2$ ist ein Wert für die Interaktion in der einfaktoriellen ANOVA.
\item [\textbf{E} \msquare] Das $\eta^2$ ist die Korrelation der ANOVA. Mit der Ausnahme, dass $\eta^2 = 0$ der beste Wert ist.
\end{enumerate}

\section{Aufgabe \hfill (2 Punkte)}

%% --------------------------------------------------------------------
\ifcollection
\begin{flushright}
\tiny\vspace{-2Ex}
\textbf{\examinhaltstart}
\exammodulemathstat $\;\bullet$
\exammodulestat $\;\bullet$
\exammodulestatbbv $\;\bullet$
\exammodulestatversuch $\;\bullet$
\exammodulebiostat
\vspace{-1Ex}
\end{flushright}
\fi
%% --------------------------------------------------------------------







Sie führen ein Feldexperiment durch um das Gewicht von Erdbeeren zu steigern. Die Pflanzen wachsen unter einer Kontrolle und zwei verschiedenen Behandlungsbedingungen. Nach der Berechnung einer einfaktoriellen ANOVA ergibt sich ein $\eta^2 = 0.26$. Welche Aussage ist richtig?



\begin{enumerate}
\item [\textbf{A} \msquare] Das $\eta^2$ beschreibt den Anteil der Varianz, der durch den Forschenden entsteht. Es gilt die Regel, dass ca. 70\% der Varianz eines Versuches durch die Versuchsdurchführung entstehen sollen.
\item [\textbf{B} \msquare] Mit dem $\eta^2$ lässt sich auf die Qualität der Randomisierung und damit der Strukturgleichheit zwischen der Grundgesamtheit und der Stichprobe schließen. Es gilt dabei die Regel, dass ein $\eta^2$-Wert von 1 zu bevorzugen ist.
\item [\textbf{C} \msquare] Das $\eta^2$ beschreibt den Anteil der Varianz, der von den Behandlungsbedingungen erklärt wird. Daher werden 26\% der Varianz durch die Behandlungsgruppen erklärt.
\item [\textbf{D} \msquare] Es werden 26\% der Varianz durch den Versuch erklärt. Das $\eta^2$ beschreibt den Anteil der Varianz, der durch Fehler in der Versuchsdurchführung entsteht.
\item [\textbf{E} \msquare] Es werden 74\% der Varianz durch die Behandlung erklärt. Das $\eta^2$ beschreibt den Anteil der Varianz, der von den unterschiedlichen Behandlungsbedingungen nicht erklärt wird.
\end{enumerate}

\section{Aufgabe \hfill (2 Punkte)}

%% --------------------------------------------------------------------
\ifcollection
\begin{flushright}
\tiny\vspace{-2Ex}
\textbf{\examinhaltstart}
\exammodulemathstat $\;\bullet$
\exammodulestat $\;\bullet$
\exammodulestatbbv $\;\bullet$
\exammodulestatversuch $\;\bullet$
\exammodulebiostat
\vspace{-1Ex}
\end{flushright}
\fi
%% --------------------------------------------------------------------




Eine einfaktorielle ANOVA berechnet eine Teststatistik um zu die Nullhypothese abzulehnen. Welche Aussage über die Teststatistik der ANOVA ist richtig?



\begin{enumerate}
\item [\textbf{A} \msquare] Wenn die F-Statistik kleiner als der kritische Wert ist kann die Nullhypothese nicht abgelehnt werden. Die F-Statistik ist der Quotient der MS der Behandlung durch die MS des Fehlers.
\item [\textbf{B} \msquare] Die ANOVA berechnet die F-Statistik indem die MS des Fehlers durch die MS der Behandlung geteilt werden. Wenn die F-Statistik sich der 1 ann{"a}hert kann die Nullhypothese nicht abgelehnt werden.
\item [\textbf{C} \msquare] Die ANOVA berechnt die F-Statistik aus den SS Behandlung geteilt durch die SS Fehler.
\item [\textbf{D} \msquare] Die ANOVA berechnet die T-Statistik aus der Multiplikation der MS Behandlung mit der MS der Fehler. Wenn die F-Statistik genau 0 ist, kann die Nullhypothese nicht abgelehnt werden.
\item [\textbf{E} \msquare] Wenn die F-Statistik höher ist als der kritische Wert kann die Nullhypothese nicht abgelehnt werden. Die F-Statistik ist die Differenz der MS der Behandlung durch die MS des Fehlers.
\end{enumerate}

\section{Aufgabe \hfill (2 Punkte)}

%% --------------------------------------------------------------------
\ifcollection
\begin{flushright}
\tiny\vspace{-2Ex}
\textbf{\examinhaltstart}
\exammodulemathstat $\;\bullet$
\exammodulestat $\;\bullet$
\exammodulestatbbv $\;\bullet$
\exammodulestatversuch $\;\bullet$
\exammodulebiostat
\vspace{-1Ex}
\end{flushright}
\fi
%% --------------------------------------------------------------------




Die ANOVA ist ein statistisches Verfahren welches häufig in den Auswertungen von Experimenten in den Agrarwissenschaften angewendet
wird. Dabei wird die ANOVA als ein erstes statistischen Werkzeug für die
Übersicht über die Daten benutzt. Eine ANOVA testet dabei...



\begin{enumerate}
\item [\textbf{A} \msquare] ... den Unterschied zwischen der Varianz durch verschiedene Behandlungsguppen $MS_{Treatment}$ inverse der Varianz über alle Behandlungsgruppen $MS_{1/Error}$. Wenn die ANOVA signifikant ist, kann kein Effekt $\eta^2$ bestimmt werden.
\item [\textbf{B} \msquare] ... den Unterschied zwischen zwei paarweisen Mittelwerten aus verschiedenen Behandlungsguppen $MS_{Treatment}$. Wenn die signifikant ist, ist daher bekannt welcher Vergleich konkret unterschiedlich ist.
\item [\textbf{C} \msquare] ... den Unterschied zwischen der Varianz aus verschiedenen Behandlungsguppen $MS_{Error}$ und der Varianz über alle Behandlungsgruppen $MS_{Treatment}$. Wenn die ANOVA signifikant ist, muss über einen Posthoc-Test nachgedacht werden um den signifikanten Unterschied in den Gruppen exakt zu bestimmen.
\item [\textbf{D} \msquare] ... den Unterschied zwischen der Varianz über alle Behandlungsgruppen oder der Varianz aus verschiedenen Behandlungsguppen. Wenn die ANOVA signifikant ist, muss sich zwischen einem der beiden Varianzquellen entschieden werden.
\item [\textbf{E} \msquare] ... den Unterschied zwischen der Mittelwerte und der Varianz aus verschiedenen Behandlungsguppen. Wenn die ANOVA signifikant ist, ist bekannt welcher Vergleich konkret unterschiedlich ist.
\end{enumerate}

\section{Aufgabe \hfill (2 Punkte)}

%% --------------------------------------------------------------------
\ifcollection
\begin{flushright}
\tiny\vspace{-2Ex}
\textbf{\examinhaltstart}
\exammodulestatversuch $\;\bullet$
\exammodulebiostat
\vspace{-1Ex}
\end{flushright}
\fi
%% --------------------------------------------------------------------




Ein Versuch wurde an 53 Tieren durchgeführt, wobei jedes Tier eine von drei Vitamin-C-Dosen (0.5, 1 und 1.5 mg/Tag) über eine von zwei Verabreichungsmethoden erhielt. Die folgende Abbildung enthält die Daten aus diesem Versuch zur Bewertung der Wirkung von Vitamin E auf das Zahnwachstum bei Hasen.  Welche Aussage ist richtig, wenn Sie eine zweifaktorielle ANOVA rechnen?



{\centering \includegraphics[width=\maxwidth]{img/mc-anova-02-a-1} 

}







\begin{enumerate}
\item [\textbf{A} \msquare] Keine Interaktion liegt vor $(p > 0.05)$.
\item [\textbf{B} \msquare] Die Koeffizienten sind negativ $(\beta_0 < 0; \beta_1 < 0)$.
\item [\textbf{C} \msquare] Die Koeffizienten sind positiv $(\beta_0 > 0; \beta_1 > 0)$.
\item [\textbf{D} \msquare] Es liegt eine mittlere bis starke Interaktion vor $(p \leq 0.05)$.
\item [\textbf{E} \msquare] Eine negative Interaktion liegt vor $(\rho \geq 0.5)$.
\end{enumerate}
\section*{Multiple Gruppenvergleiche} 

\section{Aufgabe \hfill (2 Punkte)}

%% --------------------------------------------------------------------
\ifcollection
\begin{flushright}
\tiny\vspace{-2Ex}
\textbf{\examinhaltstart}
\exammodulestatversuch $\;\bullet$
\exammodulebiostat
\vspace{-1Ex}
\end{flushright}
\fi
%% --------------------------------------------------------------------




Sie haben folgende unadjustierten p-Werte gegeben: 0.27, 0.89, 0.02, 0.38, 0.01 und 0.001. Sie adjustieren die p-Werte nach
Bonferroni. Welche Aussage ist richtig?



\begin{enumerate}
\item [\textbf{A} \msquare] Nach der Bonferroni-Adjustierung ergeben sich die adjustierten p-Werte von 1, 1, 0.12, 1, 0.06 und 0.006. Die adjustierten p-Werte werden zu einem $\alpha$-Niveau von 5\% verglichen.
\item [\textbf{B} \msquare] Nach der Bonferroni-Adjustierung ergeben sich die adjustierten p-Werte von 0.045, 0.1483, 0.0033, 0.0633, 0.0017 und 2e-04. Die adjustierten p-Werte werden zu einem $\alpha$-Niveau von 0.83\% verglichen.
\item [\textbf{C} \msquare] Nach der Bonferroni-Adjustierung ergeben sich die adjustierten p-Werte von 0.045, 0.1483, 0.0033, 0.0633, 0.0017 und 2e-04. Die adjustierten p-Werte werden zu einem $\alpha$-Niveau von 5\% verglichen.
\item [\textbf{D} \msquare] Nach der Bonferroni-Adjustierung ergeben sich die adjustierten p-Werte von 1.62, 5.34, 0.12, 2.28, 0.06 und 0.006. Die adjustierten p-Werte werden zu einem $\alpha$-Niveau von 5\% verglichen.
\item [\textbf{E} \msquare] Nach der Bonferroni-Adjustierung ergeben sich die adjustierten p-Werte von 1, 1, 0.12, 1, 0.06 und 0.006. Die adjustierten p-Werte werden zu einem $\alpha$-Niveau von 0.83\% verglichen.
\end{enumerate}

\section{Aufgabe \hfill (2 Punkte)}

%% --------------------------------------------------------------------
\ifcollection
\begin{flushright}
\tiny\vspace{-2Ex}
\textbf{\examinhaltstart}
\exammodulestatversuch $\;\bullet$
\exammodulebiostat
\vspace{-1Ex}
\end{flushright}
\fi
%% --------------------------------------------------------------------




Sie rechnen einen PostHoc-Test. Nun sollen Sie ein \textit{CLD} erstellen. Was bedeutet dieser Fachbegriff und welche folgende Beschreibung der Interpretation ist korrekt?



\begin{enumerate}
\item [\textbf{A} \msquare] Contrast letter display. Unterschiede in den Behandlungen werden durch den gleichen Buchstaben oder Symbol dargestellt. Die Interpretation des CLD führt häufig in die Irre.
\item [\textbf{B} \msquare] Compact letter display. Das CLD ist umstritten, da es die Gleichheit der Behandlungen durch gleiche Buchstaben darstellt. Dadurch ist das CLD nicht mehr sauber auf einer Linie mit dem statistischen Testen. Wir lehnen die Nullhypothese ab und zeigen keine Gleichheit im statistischen Testen.
\item [\textbf{C} \msquare] Compact letter detection. Gleichheit in den Behandlungen wird durch den gleichen Buchstaben oder Symbol dargestellt.
\item [\textbf{D} \msquare] Compact line display. Gleichheit in den Behandlungen wird durch den gleichen Buchstaben oder Symbol dargestellt. Früher wurden keine Buchstaben sondern eine durchgezogene Linie verwendet. Bei mehr als drei Gruppen funktioniert die Linie aber graphisch nicht mehr.
\item [\textbf{E} \msquare] Compact letter display. Gleiche Buchstaben bedeuten, dass sich die Behandlungen unterscheiden. Daher ist das CLD sehr unintuitiv. Es wäre besser, wenn gleiche Buchstaben Gleichheit anzeigen würden. Dies ist aber leider in der statistischen Testtheorie nicht möglich.
\end{enumerate}

\section{Aufgabe \hfill (2 Punkte)}

%% --------------------------------------------------------------------
\ifcollection
\begin{flushright}
\tiny\vspace{-2Ex}
\textbf{\examinhaltstart}
\exammodulestatversuch $\;\bullet$
\exammodulebiostat
\vspace{-1Ex}
\end{flushright}
\fi
%% --------------------------------------------------------------------




Sie haben eine zweifaktorielle ANOVA gerechnet und wollen nach einem signifikanten Ergebnis in dem Gruppenfaktor einen Posthoc-Test rechnen. Welches R Paket nutzen Sie dafür und welche Eigenschaften des Paktes sind korrekt?



\begin{enumerate}
\item [\textbf{A} \msquare] Das R Paket \{emmeans\} erlaubt die Durchführung eines multiplen Gruppenvergleichs. Aus einem emmeans Objekt lässt sich leider kein CLD erstellen. Dennoch ist das Paket einfach zu bedienen und wird deshalb genutzt. Die Interpretation der statistischen Auswertung wird über einen Barplot abgebildet.
\item [\textbf{B} \msquare] Das R Paket \{ggplot\}. Wir erhalten hier sofort eine Visualisierung der Daten. Anhand der Visualisierung lässt sich eine explorative Datenanalyse durchführen, die gleichwertig zu einem Posthoc-Test ist.
\item [\textbf{C} \msquare] Das R Paket \{lm\}. Das Paket \{lm\} erstellt selbstständig Konfidenzintervalle und entsprechende p-Werte. Da wir in dem Paket nicht adjustieren müssen, ist es bei Anwendern sehr beliebt.
\item [\textbf{D} \msquare] Das R Paket \{hmisc\} erlaubt die Durchführung eines multiplen Gruppenvergleichs aus verschiedenen Modellen heraus. Aus einem hmisc Objekt lässt sich recht einfach das CLD erstellen und so über Barplots eine schnelle Interpration der statistischen Auswertung durchführen.
\item [\textbf{E} \msquare] Da Sie für Ihre Bachelorarbeit einen Barplot mit CLD brauchen nutzen Sie das R Paket \{emmeans\} welches Ihnen schnell die notwenidigen Informationen liefert um einen Barplot zu erstelen. Die Berechnung eines CLD ist hierbei auch einfach.
\end{enumerate}

\section{Aufgabe \hfill (2 Punkte)}

%% --------------------------------------------------------------------
\ifcollection
\begin{flushright}
\tiny\vspace{-2Ex}
\textbf{\examinhaltstart}
\exammodulestatversuch $\;\bullet$
\exammodulebiostat
\vspace{-1Ex}
\end{flushright}
\fi
%% --------------------------------------------------------------------




In den Humanwissenschaften werden multiple Vergleiche häufig anders behandelt als in den Agrarwissenschaften. In beiden Bereichen tritt jedoch das gleiche Phänomen bei multiplen Testen auf. Wie muss mit dem Phänomen umgegangen werden und wie ist es benannt?



\begin{enumerate}
\item [\textbf{A} \msquare] Das globale Signifikanzniveau liegt nicht mehr bei $5\%$ sondern sehr viel niedriger, bei ca. $1\%$. Es kommt zu einer $\alpha$-Hyperinflation. Dagegen kann mit der Adjustierung der p-Werte nach Bonferroni vorgegangen werden.
\item [\textbf{B} \msquare] Das globale Signifikanzniveau explodiert und erreicht Werte größer als Eins. Es kommt zu einer $\alpha$-Inflation. Dagegen kann mit der Adjustierung der $\alpha$-Werte nach Bonferroni vorgegangen werden.
\item [\textbf{C} \msquare] Die Adjustierung der p-Werte nach Bonferroni erlaubt es gegen die $\beta$-Inflation vorzugehen, die häufig beim multiplen Testen auftritt. Das globale Powerniveau liegt nicht mehr bei $80\%$ sondern sehr viel niedriger.
\item [\textbf{D} \msquare] Beim multiplen Testen kann es zu Varianzheterogenität kommen. Das globale Signifikanzniveau liegt nicht mehr bei $5\%$. Daher müssen die p-Werte entsprechend adjustiert werden. Das Verfahren nach Welch, bekannt aus dem t-Test, ist hier häufig anzuwenden.
\item [\textbf{E} \msquare] Beim multiplen Testen kann es zu einer $\alpha$-Inflation kommen. Das globale Signifikanzniveau liegt nicht mehr bei $5\%$ sondern sehr viel höher. Daher müssen die p-Werte entsprechend adjustiert werden. Hierfür gibt es verschiedene Verfahren, wobei das Verfahren zur Adjustierung der p-Werte nach Bonferroni das bekanneste Verfahren ist.
\end{enumerate}

\section{Aufgabe \hfill (2 Punkte)}

%% --------------------------------------------------------------------
\ifcollection
\begin{flushright}
\tiny\vspace{-2Ex}
\textbf{\examinhaltstart}
\exammodulestatversuch $\;\bullet$
\exammodulebiostat
\vspace{-1Ex}
\end{flushright}
\fi
%% --------------------------------------------------------------------




In Ihrer Bachelorarbeit werten Sie einen einfaktoriellen Versuch aus. Dafür rechnen Sie in \Rlogo zunächst eine ANOVA und schließen dann dann einen multiplen vergleich mit t-Tests an. Welche Aussage über die Effekte in Ihrem versuch ist richtig?



\begin{enumerate}
\item [\textbf{A} \msquare] Beim multiplen Testen werden die Effekte der paarweisen Vergleiche ignoriert. Der Nachteil des multiplen Testens ist ja auch, dass wir am Ende keine Effekte mehr vorliegen haben. Eine ANOVA liefert hier bessere Informationen.
\item [\textbf{B} \msquare] Wenn ein multipler Test gerechnet wird, dann muss der Effekt $\Delta$ nach Bonferroni adjustiert werden. Dafür wird der Effekt mit der Anzahl an Vergleichen $k$ multipliziert. Dies geschiet analog zu den p-Werten.
\item [\textbf{C} \msquare] Beim multiplen Testen kann es zu einer Effektüberschätzung ($\Delta$-Inflation) kommen. Daher müssen die Effekte angepasst werden. Dies geschieht nicht händisch sondern intern in den angewendeten Algorithmen.
\item [\textbf{D} \msquare] Wenn ein multipler Test gerechnet wird, dann muss der Effekt $\Delta$ nicht adjustiert werden im Gegensatz zu den p-Werten.
\item [\textbf{E} \msquare] Beim multiplen Testen kann es zu einer $\Delta$-Deflation kommen. Das globale Relevanzniveau liegt nicht mehr bei $5\%$ sondern weit darunter. Daher müssen die $\Delta$-Werte entsprechend adjustiert werden. Hierfür gibt es verschiedene Verfahren, wobei das Verfahren zur Adjustierung der $\Delta$-Werte nach Bonferroni das bekanneste Verfahren ist. Die $\Delta$-Werte werden durch die Anzahl an Vergleichen geteilt.
\end{enumerate}
\section*{Lineare Regression \& Korrelation}

\section{Aufgabe \hfill (2 Punkte)}

%% --------------------------------------------------------------------
\ifcollection
\begin{flushright}
\tiny\vspace{-2Ex}
\textbf{\examinhaltstart}
\exammodulestatversuch $\;\bullet$
\exammodulebiostat
\vspace{-1Ex}
\end{flushright}
\fi
%% --------------------------------------------------------------------




In Ihrer Abschlussarbeit wollen Sie ein kausales Modell rechnen. Jetzt stellt sich die Frage, was diese Entscheidung für Ihre Auswertung bedeutet. Welche Aussage ist richtig?



\begin{enumerate}
\item [\textbf{A} \msquare] Ein kausales Modell schliesst grundsätzlich lineare Modell aus. Es muss ein Graph gefunden werden, der alle Punkte beinhaltet. Erst dann kann das $R^2$ berechnet werden.
\item [\textbf{B} \msquare] Ein kausales Modell basiert auf einem Traingsdatensatz und einem Testdatensatz. Auf dem Trainingsdatensatz wird das Modell trainiert und auf dem Testdatensatz validiert.
\item [\textbf{C} \msquare] Wenn ein kausales Modell gerechnet werden soll, dann muss zum einen ein Traingsdatensatz sowie ein Testdatensatz definiert werden. Dabei ist der Trainingsdatensatz meist 1/10 und der Testdatensatz 1/3 der Fallzahl groß. Der Testdatensatz dient zur Validierung.
\item [\textbf{D} \msquare] Es wird ein Trainingsdatensatz zum Modellieren des Trainingsmodells benötigt. Der Testdatensatz dient rein zur Visualisierung. Dies gilt vor allem für ein kausales Modell.
\item [\textbf{E} \msquare] Wenn ein kausales Modell gerechnet werden soll dann kann dies auf dem gesamten Datensatz geschehen. Das Ziel ist es einen Zusammenhang von $X$ auf $Y$ zu modellieren. Wie wirken sich die Einflussvariablen $X$ auf den gemessenen Endpunkt $Y$ aus?
\end{enumerate}

\section{Aufgabe \hfill (2 Punkte)}

%% --------------------------------------------------------------------
\ifcollection
\begin{flushright}
\tiny\vspace{-2Ex}
\textbf{\examinhaltstart}
\exammodulestat $\;\bullet$
\exammodulestatbbv $\;\bullet$
\exammodulestatversuch $\;\bullet$
\exammodulebiostat
\vspace{-1Ex}
\end{flushright}
\fi
%% --------------------------------------------------------------------




Nach einer Regressions sollten die Residuen normalverteilt sein. Was bei einer simplen Regression noch relativ einfach visuell in einem Scatterplot zu überprüfen ist. Für komplexere Modell liefert der QQ-Plot die notwendigen Informationen über die Normalverteilung. Welche Aussage ist richtig?



{\centering \includegraphics[width=\maxwidth]{img/mc-regression-05-a-1} 

}







\begin{enumerate}
\item [\textbf{A} \msquare] Die Annahme der normalverteilten Residuen ist erfüllt. Die Punkte liegen zum überwiegenden Teil auf der Geraden.
\item [\textbf{B} \msquare] Wir betrachten insbesondere die beiden Enden der Gerade. Der Rest ist mehr oder minder egal, dann ist die Annahme an die Normalverteilung der Residuen erfüllt.
\item [\textbf{C} \msquare] Wir betrachten die Gerade. Wenn die Punkte einigermaßen gleichmäßig um die Gerade verteilt liegen, dann gehen wir von normalverteilten Residuen aus. Dies ist hier nicht der Fall. Wir haben keine normalverteilten Residuen vorliegen.
\item [\textbf{D} \msquare] Wir betrachten die Punkte. Wenn die Punkte einigermaßen gleichmäßig verteilt liegen, dann gehen wir von normalen Residuen aus.
\item [\textbf{E} \msquare] Die Annahme der normalverteilten Residuen ist erfüllt. Die Punkte liegen zum überwiegenden Teil nicht auf der Geraden und Korrelation ist negativ.
\end{enumerate}

\section{Aufgabe \hfill (2 Punkte)}

%% --------------------------------------------------------------------
\ifcollection
\begin{flushright}
\tiny\vspace{-2Ex}
\textbf{\examinhaltstart}
\exammodulestat $\;\bullet$
\exammodulestatbbv $\;\bullet$
\exammodulestatversuch $\;\bullet$
\exammodulebiostat
\vspace{-1Ex}
\end{flushright}
\fi
%% --------------------------------------------------------------------




Sie berechnen in Ihrer Abschlussarbeit den Korrelationskoeffizienten $\rho$. Welche Aussage über den Korrelationskoeffizienten $\rho$ ist richtig?




\begin{enumerate}
\item [\textbf{A} \msquare] Der Korrelationskoeffizienten $\rho$ zeigt keinen Zusammenhang zwischen zwei Variablen $x$ und $y$ bei einem Wert von 0. Einen maximalen negativen Zusammenhang bei -1 und somit auch einen maximalen positiven Zusammenhang bei 1. Korrelationskoeffizienten $\rho$ ist einheitslos.
\item [\textbf{B} \msquare] Der Korrelationskoeffizienten $\rho$ ist eine standardisierte, statistische Maßzahl, die zwischen 0 und 1 liegt. Dabei ist Korrelationskoeffizienten $\rho$ einheitslos. Eine Signifikanz kann nicht nachgewiesen werden.
\item [\textbf{C} \msquare] Der Korrelationskoeffizienten $\rho$ zeigt keinen Zusammenhang zwischen zwei Variablen $x$ und $y$ bei einem Wert von 0. Einen negativen Zusammenhang Richtung -1 und somit auch einen positiven Zusammenhang Richtung 1. Je größer die Zahl allgemein, desto stärker der Effekt.
\item [\textbf{D} \msquare] Der Korrelationskoeffizienten $\rho$ liegt zwischen -1 und 1. Darüber hinaus ist der Korrelationskoeffizienten $\rho$ als standardisierte Steigung zu verstehen, wenn eine Standardisierung durchgeführt wurde. Diese Adjustierung nach Fischer muss am Anschluß der Berechnung der Korrelation durchgeführt werden.
\item [\textbf{E} \msquare] Korrelationskoeffizienten $\rho$ liegt zwischen 0 und 1. Darüber hinaus ist der Korrelationskoeffizienten $\rho$ einheitslos und kann als Standardisierung verstanden werden.
\end{enumerate}

\section{Aufgabe \hfill (2 Punkte)}

%% --------------------------------------------------------------------
\ifcollection
\begin{flushright}
\tiny\vspace{-2Ex}
\textbf{\examinhaltstart}
\exammodulestatversuch $\;\bullet$
\exammodulebiostat
\vspace{-1Ex}
\end{flushright}
\fi
%% --------------------------------------------------------------------




Sie rechnen eine linearen Regression und erhalten folgende Abbildung der Residuen (\texttt{.resid}). Welche Aussage ist richtig?



{\centering \includegraphics[width=\maxwidth]{img/mc-regression-06-a-1} 

}







\begin{enumerate}
\item [\textbf{A} \msquare] Die Punkte müssen gleichmäßig, mit ähnlichen Abständen, in dem positiven wie auch negativen Bereich liegen. Dies ist hier klar nicht der Fall. Einzelne Ausreißer können beobachtet werden. Wir können mit dem Model so nicht rechnen und müssen erst die auffälligen Werte gesondert betrachten.
\item [\textbf{B} \msquare] Die Annahme der normalverteilten Residuen ist erfüllt. Die Punkte liegen zum überwiegenden Teil auf der Diagonalen. Damit ist das Modell erfolgreich geschätzt worden.
\item [\textbf{C} \msquare] Die Annahme der normalverteilten Residuen ist nicht erfüllt. Vereinzelte Punkte liegen oberhalb bzw. unterhalb der Geraden um die 0 Linie weiter entfernt. Ein klares Muster ist zu erkennen.
\item [\textbf{D} \msquare] Die Annahme der normalverteilten Residuen ist erfüllt. Kein Muster ist zu erkennen und keine Outlier zu beobachten.
\item [\textbf{E} \msquare] Die Annahme der normalverteilten Residuen ist erfüllt. Es ist ein Muster zu erkennen und wir können damit auf die Signifkanz von $x_1, ..., x_p$ schließen.
\end{enumerate}

\section{Aufgabe \hfill (2 Punkte)}

%% --------------------------------------------------------------------
\ifcollection
\begin{flushright}
\tiny\vspace{-2Ex}
\textbf{\examinhaltstart}
\exammodulebiostat
\vspace{-1Ex}
\end{flushright}
\fi
%% --------------------------------------------------------------------




In einer lineren Regression kann es vorkommen, dass der Effekt repräsentiert durch den $\beta$ Koeffizienten nicht so richtig von der Größenordnung zu dem p-Wert passen will. So liefert eine Untersuchung des Einflusses von der $NO_3$-Konzentration in [$\mu g$] im Wasser auf das Wachstum in [$kg$] an Brokkoli folgende Effekte und p-Werte: $2e-04$ als p-Wert und einen $\beta_{NO_3}$ Koeffizienten von $2.3\times 10^{-9}$. Welche Aussage ist richtig?




\begin{enumerate}
\item [\textbf{A} \msquare] Manchmal ist die Einheit der Einflussvariable $X$ zu groß gewählt, so dass der Ansteig von 1 Einheit in $X$ zu einer zu großen Änderung in $y$ führt. Daher kann der Effekt $\beta_{NO_3}$ sehr klein wirken, da der p-Wert wird auf einer einheitslosen Teststatistik bestimmt wird.
\item [\textbf{B} \msquare] Manchmal ist die Einheit der Einflussvariable $X$ zu klein gewählt, so dass der Ansteig von 1 Einheit in $X$ zu einer zu kleinen Änderung in $y$ führt. Daher kann der Effekt $\beta_{NO_3}$ sehr klein wirken, aber auf einer anderen Einheit sehr viel größer sein. Der p-Wert wird auf einer einheitslosen Teststatistik bestimmt.
\item [\textbf{C} \msquare] Die Einheit der $NO_3$-Konzentration ist zu klein gewählt. Dadurch sehen wir den sehr kleinen $p$-Wert. Der $p$-Wert und die Einheit von der $NO_3$-Konzentration hängen antiproportional zusammen.
\item [\textbf{D} \msquare] Wenn der Effekt $\beta_{NO_3}$ winzig ist, dann kann es an einer falsch gewählten Einheit liegen. Der Anstieg von einer Einheit in $X$ führt ja zu einer Änderung von $\beta_{NO_3}$ in $x$. Wir müssen daher die Einheit von $y$ entsprechend anpassen.
\item [\textbf{E} \msquare] Die Fallzahl ist zu klein angesetzt. Je kleiner die Fallzahl ist, desto höher ist die Teststatsitik und damit auch der $p$-Wert kleiner. Wir brauchen also mehr Fallzahl um den geringen Effekt noch signifikant zu krigen.
\end{enumerate}

\section{Aufgabe \hfill (2 Punkte)}

%% --------------------------------------------------------------------
\ifcollection
\begin{flushright}
\tiny\vspace{-2Ex}
\textbf{\examinhaltstart}
\exammodulestatversuch $\;\bullet$
\exammodulebiostat
\vspace{-1Ex}
\end{flushright}
\fi
%% --------------------------------------------------------------------




Nachdem Sie Ihr Experiment abgeschlossen haben, stehen Sie vor der Frage wie Sie Ihre Daten modellieren sollen. In der Beispielauswertung von Ihrem Betreuenden finden Sie die Funktion \texttt{lm()} in \Rlogo. Welche Aussage ist richtig?





\begin{enumerate}
\item [\textbf{A} \msquare] Die Funktion \texttt{lm()} in \Rlogo wird klassischerweise für die nicht-lineare Regression genutzt. Ist die Einflussvariable $X$ numerisch so werden die Gruppenmittelwerte geschätzt.
\item [\textbf{B} \msquare] Neben der klassichen Verwendung der Funktion \texttt{lm()} in der linearen Regression kann auch ein Gruppenvergleich gerechnet werden. Dafür müssen aber alle Faktoren aus den Daten entfernt und numerishc umgewandelt werden. Dann kann das R Paket \{emmeans\} genutzt werden um die Korrelation zu berechnen. Eine Adjustierung ist dann nicht mehr notwendig.
\item [\textbf{C} \msquare] Ist die Einflussvariable $X$ ein Faktor so werden die Gruppenmittelwerte geschätzt und eine anschließende ANOVA sowie multipler Gruppenvergleich mit \{emmeans\} ist möglich. Die Funktion \texttt{lm()} kann dabei eigentlich weggelassen werden, wird aber traditionell gerechnet.
\item [\textbf{D} \msquare] Ist die Einflussvariable $X$ numerisch so werden die Gruppenmittelwerte geschätzt und eine anschließende ANOVA sowie multipler Gruppenvergleich mit \{emmeans\} ist möglich.
\item [\textbf{E} \msquare] Die Funktion \texttt{lm()} in \Rlogo wird klassischerweise für die lineare Regression genutzt. Ist die Einflussvariable $X$ ein Faktor so werden die Gruppenmittelwerte geschätzt und eine anschließende ANOVA sowie multipler Gruppenvergleich mit \{emmeans\} ist möglich.
\end{enumerate}

\section{Aufgabe \hfill (2 Punkte)}

%% --------------------------------------------------------------------
\ifcollection
\begin{flushright}
\tiny\vspace{-2Ex}
\textbf{\examinhaltstart}
\exammodulebiostat
\vspace{-1Ex}
\end{flushright}
\fi
%% --------------------------------------------------------------------




Wenn Ihr gemessener Endpunkt nicht einer Normalverteilung folgt, so können Sie dennoch Ihre Daten modellieren. Hierzu nutzen Sie dann das \textit{generalisierte lineare Modell (GLM)}. Welche Aussage ist richtig?




\begin{enumerate}
\item [\textbf{A} \msquare] Das \textit{generalisierte lineare Modell (GLM)} erlaubt auch weitere Verteilungsfamilien für das $Y$ bzw. das Outcome in einer linearen Regression zu wählen.
\item [\textbf{B} \msquare] Das GLM ist ein faktisch maschineller Lernalgorithmus, der selstständig die Verteilungsfamilie für Y wählt.
\item [\textbf{C} \msquare] Das GLM ist eine allgemeine Erweiterung der linearen Regression auf die Normalverteilung.
\item [\textbf{D} \msquare] Das \textit{generalisierte lineare Modell (GLM)} erlaubt auch weitere Verteilungsgruppen für das $X$ bzw. die Einflussvariablen in einer linearen Regression zu wählen.
\item [\textbf{E} \msquare] Das GLM ist eine Vereinfachung des LM in R. Mit dem GLM lassen sich polygonale Regressionen rechnen. Somit stehen neben der Normalverteilung noch weitere Verteilungen zu Verfügung.
\end{enumerate}

\section{Aufgabe \hfill (2 Punkte)}

%% --------------------------------------------------------------------
\ifcollection
\begin{flushright}
\tiny\vspace{-2Ex}
\textbf{\examinhaltstart}
\exammodulebiostat
\vspace{-1Ex}
\end{flushright}
\fi
%% --------------------------------------------------------------------




Neben der Mittelwertsdifferenz als Effektschätzer bei normalverteilten Endpunkten wird auch häufig der Effektschätzer Risk ratio bei binären Endpunkten verwendet. Welche Aussage über den Effektschätzer Risk ratio ist im folgenden Beispiel zur Behandlung von Klaueninfektionen bei Schafe richtig? Dabei sind 6 Tiere krank und 12 Tiere sind gesund.



\begin{enumerate}
\item [\textbf{A} \msquare] Das Verhältnis der Chancen Risk ratio ergibt ein Chancenverhältnis von 0.33. Wir sind an der Chance krank zu sein interessiert.
\item [\textbf{B} \msquare] Es ergibt sich ein Risk ratio von 0.33, da es sich um eine Chancenverhältnis handelt.
\item [\textbf{C} \msquare] Das Verhältnis der Anteile Risk ratio ergibt ein Anteilsverhältnis von 0.33. Wir sind am Anteil der Kranken interessiert.
\item [\textbf{D} \msquare] Es ergibt sich ein Risk ratio von 0.5, da es sich um eine Chancenverhältnis handelt
  
\item [\textbf{E} \msquare] Da es sich um ein Chancenverhältnis handelt ergibt sich ein Risk ratio von 3.
\end{enumerate}
    
% -----------------------------------------------------------------------
\clearpage
% -----------------------------------------------------------------------
\part{Programmieren in R}
% -----------------------------------------------------------------------

\section{Aufgabe \hfill (9 Punkte)}



 
%% --------------------------------------------------------------------
\ifcollection
\begin{flushright}
\tiny\vspace{-3Ex}
\textbf{\examinhaltstart}
\exammodulemathstat $\;\bullet$
\exammodulelanddaten $\;\bullet$
\exammodulestat 
\vspace{-4Ex}
\end{flushright}
\begin{minipage}[t]{0.5\textwidth}
\includegraphics[width = 1.3cm]{/Users/kruppajo/work/GitHub/exam/avatare/Nilufar.png}
\end{minipage}
\begin{minipage}[t]{0.5\textwidth}
\hfill
\href{https://www.youtube.com/playlist?list=PLe51bCp9JvEFUnFqaJG5aRmON9i1ZbOYC}{\includegraphics[width = 2cm]{img/youtube}}
\end{minipage}
\vspace{-3ex}
\fi
%% --------------------------------------------------------------------



\ifcollection
\paragraph{Grundlegende Kenntnisse der Programierung in \Rlogo}
\fi

'Unter den Blinden ist der Einäuge König!', ruft Ihnen Nilufar entgegen. Leider kennt sich Nilufar überhaupt nicht mit den Grundlagen in \Rlogo aus aber ihr Betreuer möchte gerne, dass die Auswertung in \Rlogo gemacht wird. Da müssen Sie dann wohl mal ran und helfen.\\[1Ex]

Nilufar: \textit{Ich habe den Namen der Funktion, die intern Daten speichert, vergessen. Was waren da nochmal die Vorteile?} \textbf{(1 Punkt)}\\[1ex]
Sie antworten:\\[3Ex]

Nilufar: \textit{Der Zuweisungs-Operator wird sehr häufig genutzt. Wie sieht der aus und wie funktioniert der an einem Beispiel?} \textbf{(1 Punkt)}\\[1ex]
Sie antworten:\\[3Ex]

Nilufar: \textit{Was ist der Unterschied zwischen \texttt{library()} und \texttt{Packages} und warum brauche ich sowas in \Rlogo?} \textbf{(1 Punkt)}\\[1ex]
Sie antworten:\\[3Ex]

Nilufar: \textit{Warum gibt es eigentlich Objekte, Wörter und Funktionen in R? Wie unterscheiden sich diese überhaupt?} \textbf{(1 Punkt)}\\[1ex]
Sie antworten:\\[3Ex]

Nilufar: \textit{Jetzt lese ich hier von einem Faktor. Was ist ein Faktor in \Rlogo?} \textbf{(1 Punkt)}\\[1ex]
Sie antworten:\\[3Ex]

Nilufar: \textit{Was macht überhaupt dieses \texttt{c()} hier überall im \Rlogo Code?} \textbf{(1 Punkt)}\\[1ex]
Sie antworten:\\[3Ex]

Nilufar: \textit{Wenn ich was in \Rlogo machen möchte, dann lade ich nochmal welche zwei \Rlogo Pakete sehr häufig?} \textbf{(1 Punkt)}\\[1ex]
Sie antworten:\\[3Ex]

Nilufar: \textit{Ich habe gehört, dass es Vorteile gibt \Rlogo zu nutzen. Nenne mir mal einen Vorteil!} \textbf{(1 Punkt)}\\[1ex]
Sie antworten:\\[3Ex]

Nilufar: \textit{Ich verstehe den Pipe-Operator nicht. Wie sieht der aus und was macht der? Gebe mal ein Beispiel!} \textbf{(1 Punkt)}\\[1ex]
Sie antworten:\\[3Ex] 
\clearpage
% -----------------------------------------------------------------------

\section{Aufgabe \hfill (9 Punkte)}



 
%% --------------------------------------------------------------------
\ifcollection
\begin{flushright}
\tiny\vspace{-3Ex}
\textbf{\examinhaltstart}
\exammodulelanddaten $\;\bullet$
\exammodulestatversuch $\;\bullet$
\exammodulebiostat
\vspace{-4Ex}
\end{flushright}
\begin{minipage}[t]{0.5\textwidth}
\includegraphics[width = 1.3cm]{/Users/kruppajo/work/GitHub/exam/avatare/Yuki.png}
\end{minipage}
\begin{minipage}[t]{0.5\textwidth}
\hfill
\href{https://www.youtube.com/playlist?list=PLe51bCp9JvEFUnFqaJG5aRmON9i1ZbOYC}{\includegraphics[width = 2cm]{img/youtube}}
\end{minipage}
\vspace{1ex}
\fi
%% --------------------------------------------------------------------



\ifcollection
\paragraph{Fortgeschrittene Kenntnisse der Programierung in \Rlogo}
\fi

'Unter den Blinden ist der Einäuge König!', ruft Ihnen Yuki entgegen. Das können Sie schon nicht mehr hören. Nur weil Sie einmal gesagt haben, dass Sie sich schonmal mit \Rlogo beschäftigt haben, stehen hier alle Schlange. Aber gut, das hat Sie dann doch vorangebracht. Leider kennt sich Yuki auch wieder überhaupt nicht mit \Rlogo aus aber sein Betreuer möchte gerne, dass die Auswertung in \Rlogo gemacht wird. Da müssen Sie dann wohl mal nochmal ran und helfen.\\[1Ex]

Yuki fragt: \textit{ANOVA in R ist ja nicht so kompliziert. Welche beiden Funktionen brauche ich nochmal in welcher Reihenfolge? \textbf{(1 Punkt)}}\\[1ex]
Sie antworten:\\[6.5Ex]

Yuki fragt: \textit{Wenn ich die Spalten in Excel benenne, was muss ich da beachten? \textbf{(1 Punkt)}}\\[1ex]
Sie antworten:\\[6.5Ex]

Yuki fragt: \textit{Ich möchte in der Funktion \texttt{emmeans()} den Faktor $f_1$ getrennt in jedem Level des Faktors $f_2$ auswerten. Was muss ich da in de Funktion \texttt{emmeans()} angeben? \textbf{(1 Punkt)}}\\[1ex]
Sie antworten:\\[6.5Ex]

Yuki fragt: \textit{Ich möchte ein CLD erstellen. Welche Funktionen muss ich in welcher Reihenfolge nutzen? \textbf{(2 Punkte)}}\\[1ex]
Sie antworten:\\[6.5Ex]

Yuki fragt: \textit{Wenn ich Daten in R mit Gruppen eingelesen habe, welche Funktion nutze ich dann meistens als erstes und warum muss ich das machen? Was muss ich da machen? \textbf{(2 Punkte)}}\\[1ex]
Sie antworten:\\[6.5Ex]

Yuki fragt: \textit{Was muss ich bei der Eingabe eines Datums in Excel beachten, wenn ich später die Exceldatei in R einlesen will? Wie lautet das Format? \textbf{(1 Punkt)}}\\[1ex]
Sie antworten:\\[6.5Ex]

Yuki fragt: \textit{Nach der EDA zu urteilen, habe ich Varianzheterogenität zwischen den Gruppen vorliegen. Wie kann ich da die Funktion \texttt{emmeans()} umstellen. \textbf{(1 Punkt)}}\\[1ex]
Sie antworten:\\[6.5Ex]



 
\clearpage
% -----------------------------------------------------------------------
\part{Deskriptive Statistik \& Explorative Datenanalyse}
% -----------------------------------------------------------------------

\section{Aufgabe \hfill (9 Punkte)}

\textit{Geben Sie grundsätzlich Formeln und Rechenweg zur Lösung der Teilaufgaben mit an!} \\[1Ex]
 

 
%% --------------------------------------------------------------------
\ifcollection
\begin{flushright}
\tiny\vspace{-3Ex}
\textbf{\examinhaltstart}
\exammodulemathstat $\;\bullet$
\exammodulestat $\;\bullet$
\exammodulestatbbv $\;\bullet$
\exammodulestatversuch\\
\exammodulelanddaten $\;\bullet$
\exammodulebiostat
\vspace{-4Ex}
\end{flushright}
\begin{minipage}[t]{0.5\textwidth}
\includegraphics[width = 1.3cm]{/Users/kruppajo/work/GitHub/exam/avatare/Jessica.png}
\end{minipage}
\begin{minipage}[t]{0.5\textwidth}
\hfill
\href{https://youtu.be/GNoORO7GfAE}{\includegraphics[width = 2cm]{img/youtube}}
\end{minipage}
\vspace{-3ex}
\fi
%% --------------------------------------------------------------------



\ifcollection
\paragraph{Zerforschen des Barplots}
\fi

Barplots sind bedeutend in der Darstellung von wissenschaftlichen Ergebnissen. Leider hat sich Jessica nicht gemerkt, welche statistischen Maßzahlen für einen Barplot erhoben werden müssen. Besser wäre was anderes gewesen. Warhammer. Ein wunderbares Hobby um sich drin zu verlieren und Abstand zu bekommen. Jessica denkt gerne über Warhammer nach. Das ist in soweit doof, da nach ihrer Betreuer erstmal ein Barplot nachgebaut werden soll, bevor es mit ihrer Abschlussarbeit losgeht. Dann hat sie schonmal den \Rlogo Code vorliegen und nachher geht dann alles schneller. Na dann mal los. Jessica schafft sich die nötige Stimmung. Jessica streichelt liebevoll die Hündin. Der Kopf ist in ihrem Schloß vergraben um den Klang von David Bowie zu dämpfen. In der Behandlung für Erdbeeren werden verschiedene Lüftungssysteme und Folientunnel ($ctrl$, $storm$ und $tornado$) sein. Erfasst wird als Messwert ($Y$) \textit{Trockengewicht}. Jessica soll dann \textit{drymatter} in ihrer Exceldatei eintragen.



{\centering \includegraphics[width=\maxwidth]{img/barplot-02-1} 

}




Leider kennt sich Jessica mit der Erstellung von Barplots in \Rlogo nicht aus. Deshalb braucht sie bei der Visualisierung Ihre Hilfe!

\begin{enumerate}
\item Formulieren Sie die wissenschaftliche Fragestellung! \textbf{(1 Punkt)}
\item Erstellen Sie eine Tabelle mit den statistischen Maßzahlen der drei Barplots! \textit{Beachten Sie die korrekte Darstellungsform der statistischen Maßzahlen!} \textbf{(4 Punkte)}
\item Erstellen Sie einen beispielhaften Datensatz im \Rlogo üblichen Format, aus dem die drei Barplots \textit{möglicherweise} erstellt wurden! \textbf{(2 Punkte)}
\item Kann Jessica einen Unterschied zwischen den Behandlungen erwarten? Begründen Sie Ihre Antwort! \textbf{(2 Punkte)}
\end{enumerate} 
\clearpage
% -----------------------------------------------------------------------

\section{Aufgabe \hfill (8 Punkte)}

\textit{Geben Sie grundsätzlich Formeln und Rechenweg zur Lösung der Teilaufgaben mit an!} \\[1Ex]
 

 
%% --------------------------------------------------------------------
\ifcollection
\begin{flushright}
\tiny\vspace{-3Ex}
\textbf{\examinhaltstart}
\exammodulemathstat $\;\bullet$
\exammodulestat $\;\bullet$
\exammodulelanddaten $\;\bullet$
\exammodulestatbbv 
\vspace{-4Ex}
\end{flushright}
\begin{minipage}[t]{0.5\textwidth}
\includegraphics[width = 1.3cm]{/Users/kruppajo/work/GitHub/exam/avatare/Paula.png}
\end{minipage}
\begin{minipage}[t]{0.5\textwidth}
\hfill
\href{https://youtu.be/UabXe73G-7w}{\includegraphics[width = 2cm]{img/youtube}}
\end{minipage}
\vspace{-3ex}
\fi
%% --------------------------------------------------------------------



\ifcollection
\paragraph{Visualisierung des Barplots}
\fi

Eine echte Herausforderung für sie war schon immer der Perfektionismus gewesen. Ein leidiges Lied. Deshalb gilt anschauen, was andere vor einem gemacht haben. Für Paula ist es eine Möglichkeit schneller ans Ziel zu gelangen. Deshalb hat sich Paula viele Poster in der Fakultät angeschaut und ist zum Schluß gekommen, dass Barplots eine häufig genutzte Abbildung sind. Paula soll nun in ihrem Projektbericht Lauch untersuchen. Die Behandlung in ihrem Projektbericht sind verschiedene Genotypen ($AA$, $AB$ und $BB$). Erhoben wurden von Paula als Messwert ($Y$) \textit{Frischegewicht} benannt als \textit{freshmatter} in ihrer Exceldatei. Erwartungsgemäß erhält sie von ihrer Betreuerin den Auftrag die erhobenen Daten als Barplots darzustellen. Dann kann Paula auch schonmal abschätzen, was bei einem statistischen Test rauskommen könnte. Na dann mal los. Paula schafft sich die nötige Stimmung. Wenn White Lies ertönt, dann sucht die Ratte schleunigst Schutz unter dem Sofa. Paula schüttelt den Kopf.

\begin{table}[!h]
\centering
\begin{tabular}{cc}
\toprule
treatment & freshmatter\\
\midrule
BB & 29.1\\
AA & 34.2\\
AB & 43.6\\
AB & 45.0\\
AA & 34.2\\
\addlinespace
AB & 38.9\\
BB & 35.4\\
BB & 25.7\\
AA & 41.1\\
BB & 21.7\\
\bottomrule
\end{tabular}
\end{table}



Leider kennt sich Paula mit der Erstellung von Barplots nicht aus. Deshalb braucht sie bei der Visualisierung Ihre Hilfe!

\begin{enumerate}
\item Formulieren Sie die wissenschaftliche Fragestellung! \textbf{(1 Punkt)}
\item Zeichnen Sie in \underline{einer} Abbildung die Barplots für die Behandlung von Lauch! Beschriften Sie die Achsen entsprechend! \textbf{(4 Punkte)}
\item Beschriften Sie \underline{einen} Barplot mit den gängigen statistischen Maßzahlen! \textbf{(2 Punkte)}
\item Wenn Paula \underline{keinen} Effekt zwischen den Behandlungen von Lauch erwarten würde, wie sehen dann die Barplots aus? \textit{Antworten Sie mit einer Skizze der Barplots!}
  \textbf{(1 Punkt)}
\end{enumerate} 
\clearpage
% -----------------------------------------------------------------------

\section{Aufgabe \hfill (9 Punkte)}

\textit{Geben Sie grundsätzlich Formeln und Rechenweg zur Lösung der Teilaufgaben mit an!} \\[1Ex]
 

 
%% --------------------------------------------------------------------
\ifcollection
\begin{flushright}
\tiny\vspace{-3Ex}
\textbf{\examinhaltstart}
\exammodulemathstat $\;\bullet$
\exammodulestat $\;\bullet$
\exammodulestatbbv $\;\bullet$
\exammodulestatversuch\\
\exammodulelanddaten $\;\bullet$
\exammodulebiostat
\vspace{-4Ex}
\end{flushright}
\begin{minipage}[t]{0.5\textwidth}
\includegraphics[width = 1.3cm]{/Users/kruppajo/work/GitHub/exam/avatare/Paula.png}
\end{minipage}
\begin{minipage}[t]{0.5\textwidth}
\hfill
\href{https://youtu.be/MyrNZ46i3uY}{\includegraphics[width = 2cm]{img/youtube}}
\end{minipage}
\vspace{-3ex}
\fi
%% --------------------------------------------------------------------



\ifcollection
\paragraph{Zerforschen des Boxplots}
\fi

Wenn der Perfektionismus nicht wäre, ja dann wäre wohl vieles möglich für Paula! Aber so.. Deshalb gilt anschauen, was andere vor einem gemacht haben. Für Paula ist es eine Möglichkeit schneller ans Ziel zu gelangen. Paula soll in ihrer Hausarbeit Erbsen untersuchen. Die Behandlung in ihrer Hausarbeit werden verschiedene Bewässerungstypen ($low$, $mid$ und $high$) sein. Erheben wird Paula als Endpunkt ($Y$) \textit{Proteingehalt} benannt als \textit{protein} in ihrer Exceldatei. Von ihrem Betreuer erhält sie nun folgende Abbildung von Boxplots, die sie erstmal zur Übung nachbauen soll, bevor sie mit dem eigentlichen Versuch beginnt. Aber nur in passender Atmospäre! Auf seinem Second Screen läuft Jagd auf roter Oktober und Paula schaufelt Smarties. Nicht effizient, aber gut.



{\centering \includegraphics[width=\maxwidth]{img/boxplot-02-zer-1} 

}




Leider kennt sich Paula mit der Erstellung von Boxplots in \Rlogo nicht aus. Deshalb braucht sie bei der Visualisierung Ihre Hilfe!

\begin{enumerate}
\item Formulieren Sie die wissenschaftliche Fragestellung! \textbf{(1 Punkt)}
\item Erstellen Sie eine Tabelle mit den statistischen Maßzahlen aus der obigen Abbildung der drei Boxplots! \textit{Beachten Sie die korrekte Darstellungsform der statistischen Maßzahlen!} \textbf{(2 Punkte)}
\item Beschriften Sie \textit{einen} der Boxplots mit den gängigen statistischen Maßzahlen! \textbf{(2 Punkte)}
\item Erstellen Sie einen beispielhaften Datensatz, aus dem die drei Boxplots \textit{möglicherweise} erstellt wurden, im \Rlogo üblichen Format! \textbf{(2 Punkte)}
\item Kann Paula einen Unterschied zwischen den Behandlungen erwarten? Begründen Sie Ihre Antwort! \textbf{(2 Punkte)}
\end{enumerate} 
\clearpage
% -----------------------------------------------------------------------

\section{Aufgabe \hfill (9 Punkte)}

\textit{Geben Sie grundsätzlich Formeln und Rechenweg zur Lösung der Teilaufgaben mit an!} \\[1Ex]
 

 
%% --------------------------------------------------------------------
\ifcollection
\begin{flushright}
\tiny\vspace{-3Ex}
\textbf{\examinhaltstart}
\exammodulemathstat $\;\bullet$
\exammodulestat $\;\bullet$
\exammodulelanddaten $\;\bullet$
\exammodulestatbbv 
\vspace{-4Ex}
\end{flushright}
\begin{minipage}[t]{0.5\textwidth}
\includegraphics[width = 1.3cm]{/Users/kruppajo/work/GitHub/exam/avatare/Mark.png}
\end{minipage}
\begin{minipage}[t]{0.5\textwidth}
\hfill
\href{https://youtu.be/ZbrvpCuZl_k}{\includegraphics[width = 2cm]{img/youtube}}
\end{minipage}
\vspace{-3ex}
\fi
%% --------------------------------------------------------------------



\ifcollection
\paragraph{Visualisierung des Boxplots}
\fi

Mark steht vor einem ersten Problem, denn wenn es nach seinem Betreuer geht, soll er in einem einem Gewächshausexperiment Erdbeeren auswertet. Soweit eigentlich alles passend. Besser wäre was anderes gewesen. Mark liebt Geocaching. Darin kann er sich wirklich verlieren und immer wieder neu begeistern. Die Behandlung waren verschiedene Genotypen ($AA$ und $BB$). In seiner Exceldatei hat er den Endpunkt ($Y$) \textit{Trockengewicht} als \textit{drymatter} aufgenommen. Nun soll Mark die Daten eimal als Boxplots in einer Präsentation visualisieren, damit seinem Betreuer wieder klar wird, was er eigentlich nochmal gemacht hat und was für ein Ergbnis in einem statistischen Test zu erwarten wäre. Anhand von Boxplots lässt sich eine Aussage über die Varianzhomogenität über die Behandlungsgruppen treffen. Wäre da nicht noch etwas. Wenn die Unsicherheit nicht wäre, ja dann wäre wohl vieles möglich für Mark! Aber so.. Aber egal. Mark will später nochmal raus um zu Reiten. Druck ablassen, dass muss er auch.

\begin{table}[!h]
\centering
\begin{tabular}{cc}
\toprule
Behandlung & Messwert\\
\midrule
AA & 37.1\\
BB & 26.1\\
BB & 29.1\\
AA & 43.3\\
BB & 30.8\\
\addlinespace
BB & 54.7\\
AA & 49.6\\
AA & 49.0\\
BB & 44.3\\
BB & 35.8\\
\addlinespace
BB & 38.9\\
AA & 32.5\\
BB & 36.2\\
AA & 42.4\\
AA & 48.3\\
\addlinespace
AA & 41.1\\
BB & 42.0\\
BB & 41.5\\
AA & 41.3\\
\bottomrule
\end{tabular}
\end{table}



Leider kennt sich Mark mit der Erstellung von Boxplots nicht aus. Deshalb braucht er bei der Visualisierung Ihre Hilfe!

\begin{enumerate}
\item Formulieren Sie die wissenschaftliche Fragestellung! \textbf{(1 Punkt)}
\item Zeichnen Sie in \textit{einer} Abbildung die beiden Boxplots für die zwei Behandlungen von Erdbeeren! Beschriften Sie die Achsen der wissenschaftlichen Fragestellung entsprechend! \textbf{(4 Punkte)} 
\item Wie ist Ihr Vorgehen, wenn Sie eine \textit{gerade} Anzahl an
  Beobachtungen pro Gruppe haben? \textbf{(1 Punkt)}
\item Beschriften Sie \textit{einen} der beiden Boxplots mit den gängigen
  statistischen Maßzahlen! \textbf{(2 Punkte)}
\item Wenn Sie \underline{einen} Effekt zwischen den Behandlungen von Erdbeeren erwarten würden, wie sehen dann die beiden Boxplots aus? \textit{Antworten Sie mit einer Skizze der Boxplots!} \textbf{(1 Punkt)}
\end{enumerate} 
\clearpage
% -----------------------------------------------------------------------

\section{Aufgabe \hfill (10 Punkte)}

\textit{Geben Sie grundsätzlich Formeln und Rechenweg zur Lösung der Teilaufgaben mit an!} \\[1Ex]
 

 
%% --------------------------------------------------------------------
\ifcollection
\begin{flushright}
\tiny\vspace{-3Ex}
\textbf{\examinhaltstart}
\exammodulemathstat $\;\bullet$
\exammodulestat $\;\bullet$
\exammodulestatbbv $\;\bullet$
\exammodulestatversuch $\;\bullet$
\exammodulebiostat
\vspace{-4Ex}
\end{flushright}
\begin{minipage}[t]{0.5\textwidth}
\includegraphics[width = 1.3cm]{/Users/kruppajo/work/GitHub/exam/avatare/Steffen.png}
\end{minipage}
\begin{minipage}[t]{0.5\textwidth}
\hfill
\href{https://youtu.be/md9I_UV5_lE}{\includegraphics[width = 2cm]{img/youtube}}
\end{minipage}
\vspace{-3ex}
\fi
%% --------------------------------------------------------------------




\ifcollection
\paragraph{Visualisierung des Scatterplots}
\fi

Steffen schmeißt noch eine Handvoll Oreos in seinen Rachen. Im Hintergrund klirrt leise der Spiegel zum Sound von Taylor Swift. Jetzt heißt es aber erstmal auf was anderes konzentrieren. Steffen möchte gerne den Zusammenhang zwischen durschnittlichen Niederschlag [ml/w] und Trockengewicht [kg/ha] im Kontext von Spargel herausfinden. Hierfür hat Steffen ein Gewächshausexperiment im Oldenburger Land durchgeführt. Nach einigen unvorgesehenen Ereignissen hat er es geschafft folgende Datentabelle zu erstellen. Eine echte Herausforderung für ihn war schon immer die Romantik gewesen. Ein leidiges Lied. Aber das steht auch nicht im Zentrum. Nun stellt sich die Frage für ihn, ob es überhaupt einen Zusammenhang zwischen den gemessenen Variablen gibt. Deshalb möchte Steffen als erstes eine explorative Datenanalyse durchführen. Dann was anderes. Das Verrückte ist, dass die Schlange Harry Potter wirklich liebt. Das ist Steffen sehr recht, denn er braucht Entspannung.

\begin{table}[!h]
\centering
\begin{tabular}{cc}
\toprule
Trockengewicht [kg/ha] & Durschnittlichen Niederschlag [ml/w]\\
\midrule
7.0 & 9.4\\
2.5 & 3.3\\
4.0 & 3.7\\
0.0 & 1.8\\
1.5 & 2.8\\
\addlinespace
10.5 & 13.8\\
6.5 & 8.2\\
12.0 & 12.7\\
\bottomrule
\end{tabular}
\end{table}



Leider kennt sich Steffen mit der Erstellung einer explorativen Datenanalyse für kontinuierliche Daten überhaupt nicht aus. Deshalb braucht er bei der Erstellung Ihre Hilfe!

\begin{enumerate}
\item Erstellen Sie eine Visualisierung für die Datentabelle. Beschriften Sie
  die Achsen entsprechend! \textbf{(4 Punkte)}
\item Schätzen Sie eine Grade durch die Punkte! \textbf{(1 Punkt)}
\item Beschriften Sie die Grade mit den gängigen statistischen Maßzahlen! Geben Sie die numerischen Zahlenwerte mit an! \textbf{(3 Punkte)}
\item Wenn \textit{kein} Effekt von $x$ auf $y$ vorhanden wäre, wie würde die Grade verlaufen und welche Werte würden die statistischen Maßzahlen annehmen? \textbf{(2 Punkt)}
\end{enumerate} 
\clearpage
% -----------------------------------------------------------------------

\section{Aufgabe \hfill (10 Punkte)}

\textit{Geben Sie grundsätzlich Formeln und Rechenweg zur Lösung der Teilaufgaben mit an!} \\[1Ex]
 

 
%% --------------------------------------------------------------------
\ifcollection
\begin{flushright}
\tiny\vspace{-3Ex}
\textbf{\examinhaltstart}
\exammodulestat $\;\bullet$
\exammodulestatbbv 
\vspace{-4Ex}
\end{flushright}
\begin{minipage}[t]{0.5\textwidth}
\includegraphics[width = 1.3cm]{/Users/kruppajo/work/GitHub/exam/avatare/Steffen.png}
\end{minipage}
\begin{minipage}[t]{0.5\textwidth}
\hfill
\href{https://youtu.be/L4nSytEHcaE}{\includegraphics[width = 2cm]{img/youtube}}
\end{minipage}
\vspace{-3ex}
\fi
%% --------------------------------------------------------------------



\ifcollection
\paragraph{Visualisierung des Mosaicplots}
\fi

Zwei kategoriale Variablen darzustellen ist nicht so einfach. Steffen hatte erst über einen Mittelwert nachgedacht, dann aber die Idee verworfen. Wäre da nicht noch was anderes. Wenn die Romantik nicht wäre, ja dann wäre wohl vieles möglich für Steffen! Aber so.. Dabei hatte er sich in ein Freilandversuch im Oldenburger Land zum einen die Behandlung KI-gesteuert [ja/nein] und zum anderen die Messung Trockengewicht über Zielwert [ja/nein] im Kontext von Erbsen angeschaut. Jetzt möchte seine Betreuerin erstmal die langen Tabellen mit ja/nein in einer explorativen Datenanalyse zusammengefasst bekommen. Sonst geht es bei seiner Abschlussarbeit nicht weiter. Was super nervig ist. Um zu Ringen geht Steffen dann später nochmal raus. Echte Entspannung.



\vspace{1Ex}

\begin{center}
\begin{minipage}[t]{0.45\textwidth}
%\small
\begin{center}

\begin{tabular}{p{2.5cm}p{2.5cm}p{2.5cm}p{2.5cm}}
\toprule
Trockengewicht über Zielwert & KI-gesteuert\\
\midrule
nein & ja\\
nein & ja\\
nein & ja\\
nein & nein\\
ja & nein\\
\addlinespace
nein & nein\\
ja & ja\\
nein & ja\\
ja & nein\\
nein & nein\\
\addlinespace
ja & nein\\
ja & nein\\
nein & ja\\
ja & nein\\
ja & nein\\
\bottomrule
\end{tabular}


\end{center}
\end{minipage}
\begin{minipage}[t]{0.45\textwidth}
%\small
\begin{center}

\begin{tabular}{p{2.5cm}p{2.5cm}p{2.5cm}p{2.5cm}}
\toprule
Trockengewicht über Zielwert & KI-gesteuert\\
\midrule
nein & ja\\
nein & ja\\
nein & ja\\
nein & ja\\
nein & ja\\
\addlinespace
nein & ja\\
ja & ja\\
ja & nein\\
nein & ja\\
ja & nein\\
\addlinespace
ja & nein\\
ja & nein\\
nein & ja\\
ja & ja\\
ja & nein\\
\bottomrule
\end{tabular}


\end{center}
\end{minipage}
\end{center}

\vspace{2Ex}

Leider kennt sich Steffen mit der Erstellung einer explorativen Datenanalyse für kategoriale Daten überhaupt nicht aus. Deshalb braucht er bei der Erstellung Ihre Hilfe!

\begin{enumerate}
\item Stellen Sie den Zusammenhang zwischen den beiden kategorialen Variablen in einer zusammenfassenden Tabelle dar! \textbf{(3 Punkte)}
\item Berechnen Sie die Verhältnisse in der zusammenfassenden Tabelle! Welche Annahme haben Sie getroffen? \textbf{(2 Punkte)}
\item Visualisieren Sie den Zusammenhang zwischen den beiden kategorialen Variablen! \textbf{(3 Punkte)}
\item Wenn \textit{kein} Effekt von der Behandlung vorliegen würde, wie würde die Tabelle und die Visualisierung aussehen? \textbf{(2 Punkt)}
\end{enumerate} 
\clearpage
% -----------------------------------------------------------------------

\section{Aufgabe \hfill (8 Punkte)}

\textit{Geben Sie grundsätzlich Formeln und Rechenweg zur Lösung der Teilaufgaben mit an!} \\[1Ex]
 

 
%% --------------------------------------------------------------------
\ifcollection
\begin{flushright}
\tiny\vspace{-3Ex}
\textbf{\examinhaltstart}
\exammodulestatversuch $\;\bullet$
\exammodulebiostat
\vspace{-4Ex}
\end{flushright}
\begin{minipage}[t]{0.5\textwidth}
\includegraphics[width = 1.3cm]{/Users/kruppajo/work/GitHub/exam/avatare/Nilufar.png}
\end{minipage}
\begin{minipage}[t]{0.5\textwidth}
\hfill
\href{https://youtu.be/n0Wr7VaN3Ao}{\includegraphics[width = 2cm]{img/youtube}}
\end{minipage}
\vspace{-3ex}
\fi
%% --------------------------------------------------------------------



\ifcollection
\paragraph{Visualisierung des Histogramm für kategoriale Daten}
\fi

In einem Gespräch mit ihrer Betreuerin wird Nilufar gebeten seine Daten aus einem Versuch in einer Klimakammer mit Erdbeeren in einem Histogramm darzustellen. 'Hm...', Takis Blue Heat und Deichkind. Das ist und bleibt die beste Kombination zum Nachdenken für Nilufar. In ihrem Experiment hat er die Knötchen erst fotographiert und dann ausgezählt. Laut ihrer Betreuerin soll das Histogramm helfen, die Verteilung der die Knötchen zu bestimmen. Es wäre einfacher, wenn da nicht noch was wäre. Eine echte Herausforderung für sie war schon immer die Erwartung gewesen. Ein leidiges Lied. Nilufar nickt im Takt von Deichkind und bemerkt dabei gar nicht was das Huhn schon wieder anstellt.

\begin{center}
Die Knötchen: 5, 6, 4, 6, 5, 2, 3, 4, 3, 2, 3, 3, 3, 7, 1, 3, 4, 0, 3, 1, 5, 3, 5, 4, 6, 1, 4, 8, 5, 4
\end{center}

Leider kennt sich Nilufar mit der Erstellung von Histogrammen überhaupt nicht aus. Deshalb braucht sie bei der Erstellung Ihre Hilfe!

\begin{enumerate}
\item Zeichen Sie ein Histogramm um die Verteilung der Daten zu visualisieren! (\textbf{3 Punkte})
\item Beschriften Sie die Achsen der Abbildung! (\textbf{2 Punkte})
\item Ergänzen Sie die absoluten und relativen Häufigkeiten in der
  Abbildung! \textbf{(1 Punkt)}
\item Berechnen Sie aus den Daten die \textit{Wahrscheinlichkeit}
  mehr als die Anzahl 4 zu beobachten! \textbf{(1
    Punkt)}
\item Berechnen Sie aus den Daten die \textit{Chance} mehr
  als die Anzahl 4 zu beobachten! \textbf{(1 Punkt)}
\end{enumerate}

 
\clearpage
% -----------------------------------------------------------------------

\section{Aufgabe \hfill (8 Punkte)}

\textit{Geben Sie grundsätzlich Formeln und Rechenweg zur Lösung der Teilaufgaben mit an!} \\[1Ex]
 

 
%% --------------------------------------------------------------------
\ifcollection
\begin{flushright}
\tiny\vspace{-3Ex}
\textbf{\examinhaltstart}
\exammodulestatversuch $\;\bullet$
\exammodulebiostat
\vspace{-4Ex}
\end{flushright}
\begin{minipage}[t]{0.5\textwidth}
\includegraphics[width = 1.3cm]{/Users/kruppajo/work/GitHub/exam/avatare/Steffen.png}
\end{minipage}
\begin{minipage}[t]{0.5\textwidth}
\hfill
\href{https://youtu.be/Q-kyhucwIXs}{\includegraphics[width = 2cm]{img/youtube}}
\end{minipage}
\vspace{-3ex}
\fi
%% --------------------------------------------------------------------



\ifcollection
\paragraph{Visualisierung des Histogramm für kontinuierliche Daten}
\fi

In seiner Abschlussarbeit möchte Steffen gerne die Daten aus einem Leistungssteigerungsversuch mit Fleischrindern in einem Histogramm darstellen. Das Histogramm erlaubt ihm dabei Rückschlüsse auf die Verteilung über den Messwert ($Y$) zu treffen Steffen schmeißt noch eine Handvoll Oreos in seinen Rachen. Im Hintergrund klirrt leise der Spiegel zum Sound von Taylor Swift. In seinem Experiment hat Steffen die mittleren auffälligen Hautflecken gezählt. Es wäre einfacher, wenn da nicht noch was wäre. Wenn die Romantik nicht wäre, ja dann wäre wohl vieles möglich für Steffen! Aber so.. Steffen nickt im Takt von Taylor Swift und bemerkt dabei gar nicht was die Schlange schon wieder anstellt.

\begin{center}
Die mittleren auffälligen Hautflecken: 11.1, 9.8, 7.8, 10.7, 12, 7.6, 9, 9.4, 12.8, 11.4, 10.6, 11.8, 12.7, 8.9, 15.2, 9.3, 10, 8.1, 9.4, 12.8, 8.7, 10
\end{center}

Leider kennt sich Steffen mit der Erstellung von Histogrammen überhaupt nicht aus. Deshalb braucht er bei der Erstellung Ihre Hilfe!

\begin{enumerate}
\item Zeichen Sie ein Histogramm um die Verteilung der Daten zu visualisieren! (\textbf{3 Punkte})
 \item Erläutern Sie Ihr Vorgehen um ein Histogramm für kontinuierliche Daten zu zeichnen!  (\textbf{2 Punkte})
\item Beschriften Sie die Achsen der Abbildung! (\textbf{2 Punkte})
\item Ergänzen Sie die relativen Häufigkeiten in der Abbildung! \textbf{(1 Punkt)}  
\end{enumerate}

 
\clearpage
% -----------------------------------------------------------------------

\section{Aufgabe \hfill (10 Punkte)}

\textit{Geben Sie grundsätzlich Formeln und Rechenweg zur Lösung der Teilaufgaben mit an!} \\[1Ex]
 

 
%% --------------------------------------------------------------------
\ifcollection
\begin{flushright}
\tiny\vspace{-3Ex}
\textbf{\examinhaltstart}
\exammodulestatversuch $\;\bullet$
\exammodulebiostat
\vspace{-4Ex}
\end{flushright}
\begin{minipage}[t]{0.5\textwidth}
\includegraphics[width = 1.3cm]{/Users/kruppajo/work/GitHub/exam/avatare/Alex.png}\hspace{-4mm}\includegraphics[width = 1.3cm]{/Users/kruppajo/work/GitHub/exam/avatare/Yuki.png}
\end{minipage}
\begin{minipage}[t]{0.5\textwidth}
\hfill
\href{https://youtu.be/A4E9zBOltuU}{\includegraphics[width = 2cm]{img/youtube}}
\end{minipage}
\fi
%% --------------------------------------------------------------------



\ifcollection
\paragraph{Visualisierung von Verteilungen}
\fi

'Was hast du dir denn da hingeklebt? \textit{Frei ist, wer missfallen kann.}\footnote{Oschmann, A. (2024) Mädchen stärken: Stärken fördern, Selbstwert erhöhen und liebevoll durch Krisen begleiten. Goldegg Verlag}', liest Yuki vom Kühlschrank vor. Yuki und Alex sitzen zusammen in der Küche und versuchen zu verhindern, dass das Minischwein den Biomüll mampft. 'Können wir uns auf die etwas kryptische Aufgabe konzentrieren?', nöhlt Alex. Die beiden schauen angestrengt auf die drei Boxplots. Das Ziel ist es zu verstehen, wie eine Verteilung anhand eines Boxplots bewertet werden kann. Alex und die Gefälligkeit machen die Sache nicht einfacher.



{\centering \includegraphics[width=\maxwidth]{img/desc-stat-11-1} 

}




Jetzt brauchen Yuki und Alex Ihre Hilfe bei der Abschätzung einer Verteilung anhand von Boxplots um ihre Arbeit dann in diesem Semester noch abschließen zu können.

\begin{enumerate}
\item Zeichnen Sie über die Boxplots die entsprechende zugehörige Verteilung! \textbf{(3 Punkte)} 
\item Zeichnen Sie unter die Boxplots die entsprechende zugehörige Beobachtungen als Striche! \textbf{(3 Punkte)}
\item Wie viel Prozent der Beobachtungen fallen in das IQR? Ergänzen Sie die Abbildung entsprechend um den Bereich! \textbf{(2 Punkte)}
\item Wie viel Prozent der Beobachtungen fallen in $\bar{y} \pm 1s$ und $\bar{y} \pm 2s$  unter der Annahme einer Normalverteilung? \textbf{(2 Punkte)}
\end{enumerate} 
\clearpage
% -----------------------------------------------------------------------
\part{Statistisches Testen \& statistische Testtheorie}
% -----------------------------------------------------------------------  

\section{Aufgabe \hfill (9 Punkte)}


 
%% --------------------------------------------------------------------
\ifcollection
\begin{flushright}
\tiny
\textbf{\examinhaltstart}
\exammodulemathstat $\;\bullet$
\exammodulestat $\;\bullet$
\exammodulestatbbv $\;\bullet$
\exammodulestatversuch $\;\bullet$
\exammodulebiostat
\vspace{-4Ex}
\end{flushright}
\begin{minipage}[t]{0.5\textwidth}
\includegraphics[width = 1.3cm]{/Users/kruppajo/work/GitHub/exam/avatare/Mark.png}\hspace{-4mm}\includegraphics[width = 1.3cm]{/Users/kruppajo/work/GitHub/exam/avatare/Tina.png}
\end{minipage}
\begin{minipage}[t]{0.5\textwidth}
\hfill
\href{https://youtu.be/KxDklbFEEHw}{\includegraphics[width = 2cm]{img/youtube}}
\end{minipage}
\fi
%% --------------------------------------------------------------------



\ifcollection
\paragraph{Grundgesamtheit und experimentelle Stichprobe}
\fi

An einem schwülem Sommernachmittag sitzen Tina und Mark in einem Eiskaffee und wollen sich auf die Klausur vorbereiten. In fast allen Fragen geht es ja um die Interpretation eines statistischen Tests. Daher wollen die beiden jetzt nochmal nacharbeiten, was die Grundlagen der Stichprobe (eng. \textit{sample}) und der Grundgesamtheit (eng. \textit{population} oder \textit{ground truth}) sind. Tina hat sich Katjes Eisbecher bestellt und Mark bleibt lieber bei einem Marzipankugeln Eis. 'Irre, was die Lebensmittelindustrie alles auf die Beine kriegt', merk Mark an und Tina schüttelt anerkennend den Kopf.

\vspace{1ex}

Leider kennen sich Tina und Mark mit der Grundgesamtheit und der Stichprobe überhaupt nicht aus. Daher sind Sie gefragt!

\begin{enumerate}
\item Erklären Sie den Zusammenhang zwischen Stichprobe und Grundgesamtheit an einem Schaubild! Beschriften Sie das Schaubild entsprechend! \textit{Nutzen Sie hierfür als Veranschaulichung ein aussagekräftiges Beispiel!}  \textbf{(3 Punkte)}
\begin{enumerate}
\item Nennen Sie das statistische Verfahren um von einer Grundgesamtheit auf eine Stichprobe zu gelangen!  \textbf{(1 Punkt)}
\item Nennen Sie ein konkretes Beispiel zur Durchführung um von einer Grundgesamtheit auf eine Stichprobe zu gelangen! \textbf{(1 Punkt)}
\item Benennen Sie die statistische Eigenschaft, die zwischen Grundgesamtheit und Stichprobe vorliegen muss! \textbf{(1 Punkt)}
\end{enumerate}
\item Erweitern Sie das Schaubild um die Entstehung von $Pr(D|H_0)$! \textit{Nutzen Sie hierfür als Veranschaulichung ein aussagekräftiges Beispiel!}  \textbf{(3 Punkte)}
\end{enumerate}

 
\clearpage
% -----------------------------------------------------------------------

\section{Aufgabe \hfill (9 Punkte)}


 
%% --------------------------------------------------------------------
\ifcollection
\begin{flushright}
\tiny
\textbf{\examinhaltstart}
\exammodulestat $\;\bullet$
\exammodulestatbbv $\;\bullet$
\exammodulestatversuch $\;\bullet$
\exammodulebiostat
\vspace{-4Ex}
\end{flushright}
\begin{minipage}[t]{0.5\textwidth}
\includegraphics[width = 1.3cm]{/Users/kruppajo/work/GitHub/exam/avatare/Steffen.png}\hspace{-4mm}\includegraphics[width = 1.3cm]{/Users/kruppajo/work/GitHub/exam/avatare/Tina.png}
\end{minipage}
\begin{minipage}[t]{0.5\textwidth}
\hfill
\href{https://youtu.be/XAE8BwNmCQ0}{\includegraphics[width = 2cm]{img/youtube}}
\end{minipage}
\fi
%% --------------------------------------------------------------------



\ifcollection
\paragraph{Das Nullritual - Die statistische Testtheorie}
\fi

'Das Känguruh hat mir gerade zugezwinkert. Das macht mir Angst', bemerkt Steffen. Es reicht ja schon ein Problem. Steffen und die Romantik, eine unendliche Geschichte mit kniffeligen Wendungen. Da braucht es nicht noch Wahnvorstellungen. 'Ich glaube nicht, das Känguruhs zwinkern. Aber was Wichtigeres. Wo hast du eigentlich meine Schnapspralinen hingetan? Wir haben nur noch Oreos von dir. Bäh!', antwortet Tina. Beide sind im Zoo und wollen sich von der statistische Testheorie ablenken lassen. Eigentlich wollte ja Steffen stoppen wie lange Tiere pinkeln\footnote{Yang, P. J., et al. (2014). Duration of urination does not change with body size. Proceedings of the National Academy of Sciences, 111(33), 11932-11937.}, scheiterte aber an einer Oma mit Stock, die die beiden beschimpfte.

\vspace{1ex}

Leider kennen sich Steffen und Tina mit statistischen Testtheorie, auch Null-Ritual genannt, überhaupt nicht aus. Geschweige denn mit der Visualisierung als Kreuztabelle.  

\begin{enumerate}
\item Tragen Sie folgende statistische Fachbegriffe zur statistischen Testtheorie korrekt eine selbst erstellte Kreuztabelle ein! \textbf{(3 Punkte)}
  \begin{center}
  \begin{tabular}{cccc}
  Testentscheidung & Richtige Entscheidung & H$_0$ wahr & 20\% \\
  \end{tabular}
  \end{center}
\item Ergänzen Sie Ihre erstellte Kreuztabelle um vier weitere, passende Fachbegriffe zur statistischen Testtheorie! \textbf{(2 Punkte)}
\end{enumerate}

Die Entscheidungsfindung durch einen statistischen Test kann auch durch die Analogie zu einem Feuermelder abgebildet werden. Dabei symbolisiert der Feuermelder den statistischen Test und es soll getestet werden, ob ein Feuer ausgebrochen ist.

\begin{enumerate}
  \setcounter{enumi}{2}    
\item In der Analogie des Feuermelders, wie lautet der $\alpha$-Fehler? \textbf{(1 Punkt)}
\item In der Analogie des Feuermelders, wie lautet der $\beta$-Fehler? \textbf{(1 Punkt)}
\item Wenn der Feuermelder einmal pro Tag messen würde, wie oft würde der Feuermelder mit einem $\alpha$ von 5\% in einem Monat Alarm schlagen? Begründen Sie Ihre Antwort! \textbf{(2 Punkte)}
\end{enumerate}



 
\clearpage
% -----------------------------------------------------------------------

\section{Aufgabe \hfill (9 Punkte)}

\textit{Geben Sie grundsätzlich Formeln und Rechenweg zur Lösung der Teilaufgaben mit an!} \\[1Ex]


 
%% --------------------------------------------------------------------
\ifcollection
\begin{flushright}
\tiny\vspace{-3Ex}
\textbf{\examinhaltstart}
\exammodulemathstat $\;\bullet$
\exammodulestat $\;\bullet$
\exammodulestatbbv $\;\bullet$
\exammodulestatversuch $\;\bullet$
\exammodulebiostat
\vspace{-4Ex}
\end{flushright}
\begin{minipage}[t]{0.5\textwidth}
\includegraphics[width = 1.3cm]{/Users/kruppajo/work/GitHub/exam/avatare/Nilufar.png}\hspace{-4mm}\includegraphics[width = 1.3cm]{/Users/kruppajo/work/GitHub/exam/avatare/Tina.png}
\end{minipage}
\begin{minipage}[t]{0.5\textwidth}
\hfill
\href{https://youtu.be/2LQzyQCy2FI}{\includegraphics[width = 2cm]{img/youtube}}
\end{minipage}
\fi
%% --------------------------------------------------------------------



\ifcollection
\paragraph{Visualisierung der Teststatistik $\boldsymbol{T_D}$ und dem p-Wert}
\fi

'Kannst du mir nochmal an einer Visualisierung erklären, wie der Zusammenhang zwischen der Teststatistik aus den Daten $T_D$ und dem p-Wert ist? Ich habe hier zig Fachbegriffe, kriege die abr nicht zusammen...', fragt Tina nachdrücklich  Nilufar. Das hilft aber nur bedingt, denn Nilufar hat wenig geschlafen und träumt zu den Klängen von Tocotronic. Tina hatte den ganzen Abend mit Nilufar über die Wut diskutiert und nun sind beide voll neben der Spur. So wird es nichts mit der Klausur. Tina mampft noch ein paar Katjes und nickt ein. Jetzt brauchen die beiden gesondert Hilfe!

\vspace{1ex}

Leider kennen sich Tina und Nilufar mit der Visualisierung der Teststatistik $T_D$ und dem p-Wert überhaupt nicht aus und brauchen dahr Ihre Hilfe!

\vspace{1ex}

\textit{Beachten Sie, dass im Folgenden \underline{keine numerisch korrekte Darstellung} verlangt wird! Es gilt Erkennbarkeit vor Genauigkeit!}

\begin{enumerate}
\item Ergänzen Sie eine beschriftete $x$-Achse! \textbf{(1 Punkt)}
\item Ergänzen Sie "`$\bar{y}_1 = \bar{y}_2$"'! \textbf{(1 Punkt)} 
\item Ergänzen Sie "`$95\%$"'! \textbf{(1 Punkt)}
\item Zeichnen Sie $T_{\alpha=5\%}$ in die Abbildung! \textbf{(1 Punkt)} 
\item Zeichnen Sie das Signifikanzniveau $\alpha$ in die Abbildung! Begründen Sie Ihre Antwort! \textbf{(2 Punkte)} 
\item Zeichnen Sie $+T_{D}$ in die Abbildung! \textbf{(1 Punkt)}
\item Zeichnen Sie einen nicht signifikant p-Wert in die Abbildung! Begründen Sie Ihre Antwort! \textbf{(2 Punkte)}   
\end{enumerate}



{\centering \includegraphics[width=\maxwidth]{img/statistisches-testen-3-1} 

}


 
\clearpage
% -----------------------------------------------------------------------

\section{Aufgabe \hfill (10 Punkte)}


 
%% --------------------------------------------------------------------
\ifcollection
\begin{flushright}
\tiny
\textbf{\examinhaltstart}
\exammodulestatversuch $\;\bullet$
\exammodulebiostat
\vspace{-4Ex}
\end{flushright}
\begin{minipage}[t]{0.5\textwidth}
\includegraphics[width = 1.3cm]{/Users/kruppajo/work/GitHub/exam/avatare/Nilufar.png}\hspace{-4mm}\includegraphics[width = 1.3cm]{/Users/kruppajo/work/GitHub/exam/avatare/Steffen.png}
\end{minipage}
\begin{minipage}[t]{0.5\textwidth}
\hfill
\href{https://youtu.be/6w_KH7t6bp0}{\includegraphics[width = 2cm]{img/youtube}}
\end{minipage}
\fi
%% --------------------------------------------------------------------



\ifcollection
\paragraph{Visualisierung des 95\% Konfidenzintervalls}
\fi

'So, was haben wir gemacht? Wir haben einen t-test für den Vergleich der Mittelwerte gerechnet.', meint Steffen. Nilufar schaut fragend. 'Hatten wir nicht alles zu einer Kontrolle verglichen? Das war doch so!', ruft Nilufar laut aus. 'Wir haben doch als Messwert \textit{Wasserverbrauch der Bewässerung} erhoben.', stellt Steffen fest. Jetzt haben beide das Problem, die möglichen 95\% Konfidenzintervalle zu interpretieren.

\vspace{1ex}

Leider kennen sich Steffen und Nilufar mit der Visualisierung des 95\% Konfidenzintervall überhaupt nicht aus. 

\begin{enumerate}
\item Beschriften Sie die untenstehende Abbildung mit der Signifikanzschwelle! Begründen Sie Ihre Antwort! \textbf{(2 Punkte)}
\item Ergänzen Sie eine \textit{in den Kontext passende} Relevanzschwelle! Begründen Sie Ihre Antwort! \textbf{(2 Punkte)} 
\item Skizzieren Sie in die untenstehende Abbildung sechs einzelne Konfidenzintervalle (a-f) mit den
  jeweiligen Eigenschaften! \textbf{(6 Punkte)}
  \begin{itemize}
  \item[(a)] Ein 95\% Konfidenzintervall mit niedriger Fallzahl $n$ in der Stichprobe als der Rest 95\% der Konfidenzintervalle 	
  \item[(b)] Ein signifikantes, nicht relevantes 95\% Konfidenzintervall 	
  \item[(c)] Ein signifikantes, relevantes 95\% Konfidenzintervall 	
  \item[(d)] Ein nicht signifikantes, nicht relevantes 95\% Konfidenzintervall 
  \item[(e)] Ein 95\% Konfidenzintervall mit h{"o}herer Fallzahl $n$ in der Stichprobe als der Rest der 95\% Konfidenzintervalle
  \item[(f)] Ein signifikantes, relevantes 99\% Konfidenzintervall.
  \end{itemize}
\end{enumerate}

\begin{center}
  \includegraphics[height = 10cm]{/Users/kruppajo/work/GitHub/exam/question/img/statistisches-testen-04}
\end{center}


 
\clearpage
% -----------------------------------------------------------------------

\section{Aufgabe \hfill (10 Punkte)}

\textit{Geben Sie grundsätzlich Formeln und Rechenweg zur Lösung der Teilaufgaben mit an!} \\[1Ex]


 
%% --------------------------------------------------------------------
\ifcollection
\begin{flushright}
\tiny\vspace{-3Ex}
\textbf{\examinhaltstart}
\exammodulestatversuch $\;\bullet$
\exammodulebiostat
\vspace{-4Ex}
\end{flushright}
\begin{minipage}[t]{0.5\textwidth}
\includegraphics[width = 1.3cm]{/Users/kruppajo/work/GitHub/exam/avatare/Mark.png}\hspace{-4mm}\includegraphics[width = 1.3cm]{/Users/kruppajo/work/GitHub/exam/avatare/Paula.png}
\end{minipage}
\begin{minipage}[t]{0.5\textwidth}
\hfill
\href{https://youtu.be/SejyJ0K20jA}{\includegraphics[width = 2cm]{img/youtube}}
\end{minipage}
\fi
%% --------------------------------------------------------------------



\ifcollection
\paragraph{Zusammenhang zwischen dem Effekt, der Streuung sowie der Fallzahl}
\fi

An einem heißen Juliabend haben sich Paula und Mark zum Lernen verabredet. Eine große Kanne Eistee und Berge von Smarties schmelzen in der Sonne  und warten darauf gegessen zu werden. Paula liest laut vor:\begin{quote}
                 \textit{
                 Beim statistischen Testen gibt es einen Zusammenhang zwischen dem Effekt, der Streuung sowie der Fallzahl. Gegeben sei die Formel für den Student t-Test auf den die folgenden Überlegungen basieren sollen. Welche Auswirkung hat die Änderungen der jeweiligen statistischen Maßzahl des Effekts $\Delta$, der Streuung $s$ und der Fallzahl $n$ auf die Teststistik $T_{D}$, den p-Wert $Pr(D|H_0)$ sowie dem Konfidenzintervall $KI_{1-\alpha}$?
                 }
                 \end{quote}Mark hebt die Augenbraue. 'Ich bleibe dabei. Wir sollten erstmal Columbo schauen, bis es kühler ist. Den Film habe ich doch extra mitgebracht!' Paula ist der Idee nicht abgeneigt und auch die Ratte kommt auf die Terasse um dabei zu sein.

\vspace{1ex}

Leider kennen sich Paula und Mark mit dem Zusammenhang zwischen dem Effekt, der Streuung sowie der Fallzahl überhaupt nicht aus. 


\begin{enumerate}
\item Visualisieren Sie den Zusammenhang zwischen der Teststatiatik $T_{D}$ und dem p-Wert $Pr(D|H_0)$ für sich verändernde $T_{D}$-Werte!\textit{Geben Sie dafür ein numerisches Beispiel in dem Sie drei $T_{D}$-Werte und deren Einfluss auf den p-Wert vergleichen!} \textbf{(3 Punkte)}  
\item  Füllen Sie die untenstehende Tabelle aus in dem Sie die Änderung der statistischen Maßzahlen auf die Teststatistik, den p-Wert sowie das Konfidenzintervall in \textit{einem} Wort oder Symbol beschreiben! \textbf{(4 Punkte)}
\begin{center}
  \large
  \begin{tabular}[c]{l|c|c|c|l|c|c|c}
    & $T_{D}$ & $Pr(D|H_0)$ & $KI_{1-\alpha}$ & & $T_{D}$ & $Pr(D|H_0)$ & $KI_{1-\alpha}$\strut\\ 
    \hline
    \textbf{$\Delta\; \uparrow$} & \hspace{1.8cm} & \hspace{1.8cm}  & \hspace{1.8cm} & \textbf{
                                                          $\Delta\; \downarrow$} &
                                                                          \hspace{1.8cm} & \hspace{1.8cm}  & \hspace{1.8cm}\strut\\
    \hline
        \textbf{$s\; \uparrow$} & \hspace{1.8cm} & \hspace{1.8cm}  & \hspace{1.8cm} & \textbf{
                                                          $s\; \downarrow$} &
                                                                          \hspace{1.8cm}
                                                & \hspace{1.8cm}  & \hspace{1.8cm}\strut\\
    \hline
        \textbf{$n\; \uparrow$} & \hspace{1.8cm} & \hspace{1.8cm}  & \hspace{1.8cm} & \textbf{
                                                          $n\; \downarrow$} &
                                                                          \hspace{1.8cm}
                                                & \hspace{1.8cm}  & \hspace{1.8cm}\strut\\
    \hline
  \end{tabular}
\end{center}
\item Visualisieren Sie ein 95\%-iges Konfidenzintervall im Vergleich zu einem 90\%-igen Konfidenzintervall! Begründen Sie Ihre Visualisierung anhand der Formel des Konfidenzintervalls des t-Tests mathematisch! \textbf{(3 Punkte)} 
\end{enumerate} 
\clearpage
% -----------------------------------------------------------------------
\part{Der t-Test}
% -----------------------------------------------------------------------

\section{Aufgabe \hfill (9 Punkte)}

\textit{Geben Sie grundsätzlich Formeln und Rechenweg zur Lösung der Teilaufgaben mit an!} \\[1Ex]
 

 
%% --------------------------------------------------------------------
\ifcollection
\begin{flushright}
\tiny\vspace{-3Ex}
\textbf{\examinhaltstart}
\exammodulemathstat $\;\bullet$
\exammodulestat $\;\bullet$
\exammodulestatbbv 
\vspace{-4Ex}
\end{flushright}
\begin{minipage}[t]{0.5\textwidth}
\includegraphics[width = 1.3cm]{/Users/kruppajo/work/GitHub/exam/avatare/Yuki.png}
\end{minipage}
\begin{minipage}[t]{0.5\textwidth}
\hfill
\href{https://youtu.be/zgpw9GC0plk}{\includegraphics[width = 2cm]{img/youtube}}
\end{minipage}
\vspace{-3ex}
\fi
%% --------------------------------------------------------------------



\ifcollection
\paragraph{Berechnung des Student t-Test \underline{oder} Welch t-Test}
\fi

Yuki ist im Wendland für einen Pilotexperiment mit sehr geringer Fallzahl $(n_1 = n_2 = 3)$ mit Spargel. Allein diese Tatsache ist für sie eine Erzählung wert. Wenn die Faulheit nicht wäre, ja dann wäre wohl vieles möglich für Yuki! Aber so..  Für ihrer Hausarbeit musste sie ein Gewächshausexperiment mit Spargel durchführen und das sollte laut ihrem Betreuer an diesem Ort besonders gut gelingen, da man hier gut neue technische Anlagen und Behandlungen fernab der Bevölkerung testen könne. Zeugen gibt es hier jedenfalls keine. Gar keine.  Alleine sein hilft jetzt aber nur bedingt, denn ihre Behandlung Bewässerungstypen ($low$ und $high$) und der Messwert Trockengewicht [kg/ha] sollen mit einem t-Test ausgewertet werden. Immerhin weiß sie, dass ihr Messwert einer Normalverteilung folgt. Hm..., was entspannendes wäre gut. Yuki will später nochmal raus um zu Boldern. Druck ablassen, dass muss sie auch.

\begin{table}[!h]
\centering
\begin{tabular}{cc}
\toprule
Bewässerungstypen & Trockengewicht\\
\midrule
high & 24.4\\
high & 20.3\\
low & 22.7\\
high & 10.2\\
low & 23.3\\
\addlinespace
low & 21.8\\
\bottomrule
\end{tabular}
\end{table}



Leider kennt sich Yuki mit der Berechnung eines Welch t-Tests überhaupt nicht aus. Deshalb braucht sie bei der Berechnung Ihre Hilfe!

\begin{enumerate}
  \item Formulieren Sie das statistische Hypothesenpaar! \textbf{(1 Punkt)}
  \item Bestimmen Sie die Teststatistik $T_{D}$ eines Welch t-Tests! \textbf{(3 Punkte)}
  \item Treffen Sie mit $T_{\alpha = 5\%} = 2.68$ eine Aussage zur Nullhypothese! Begründen Sie Ihre Antwort! \textbf{(2 Punkte)}
  \item Berechnen Sie den Effekt des Welch t-Tests! \textbf{(1 Punkt)}
  \item Formulieren Sie eine Antwort an Yuki über das Ergebnis Ihrer statistischen Analyse! \textbf{(2 Punkte)}
\end{enumerate} 
\clearpage
% -----------------------------------------------------------------------

\section{Aufgabe \hfill (12 Punkte)}

\textit{Geben Sie grundsätzlich Formeln und Rechenweg zur Lösung der Teilaufgaben mit an!} \\[1Ex]
 

 
%% --------------------------------------------------------------------
\ifcollection
\begin{flushright}
\tiny\vspace{-3Ex}
\textbf{\examinhaltstart}
\exammodulemathstat $\;\bullet$
\exammodulestat $\;\bullet$
\exammodulestatbbv 
\vspace{-4Ex}
\end{flushright}
\begin{minipage}[t]{0.5\textwidth}
\includegraphics[width = 1.3cm]{/Users/kruppajo/work/GitHub/exam/avatare/Steffen.png}
\end{minipage}
\begin{minipage}[t]{0.5\textwidth}
\hfill
\href{https://youtu.be/eGoLNgE-WCo}{\includegraphics[width = 2cm]{img/youtube}}
\end{minipage}
\vspace{-3ex}
\fi
%% --------------------------------------------------------------------



\ifcollection
\paragraph{Berechnung des Student t-Test}
\fi

Der t-Test. Steffen erschaudert. Steffen und die Romantik, eine unendliche Geschichte mit kniffeligen Wendungen. Ein mächtiges Werkzeug ist der t-Test in den Händen desjenigen, der einen normalverteilten Endpunkt ($Y$) hat. Aber erstmal überhaupt den t-Test rechnen können. Wie sah das Experiment von Steffen überhaupt aus? Aus den Boxen wummert Taylor Swift und sein Mund ist verklebt von Oreos. 'Herrlich', denkt Steffen. Steffen hat ein Stallexperiment mit Schweinen durchgeführt. Dabei wurde die Behandlung Genotypen ($AA$ und $BB$) an den Schweinen getestet. Gemessen hat Steffen dann als Messwert Gewichtszuwachs in der 1LW [\%/kg]. Warum der Versuch im Emsland für seiner Hausarbeit stattfinden musste, ist ihm bis heute ein Rätsel. Egal. Gibt es jetzt einen Zusammenhang zwischen der Behandlung und Gewichtszuwachs in der 1LW [\%/kg]?

\begin{table}[!h]
\centering
\begin{tabular}{cc}
\toprule
Genotypen & Gewichtszuwachs\\
\midrule
AA & 40.6\\
AA & 37.5\\
AA & 32.4\\
BB & 22.2\\
BB & 26.7\\
\addlinespace
BB & 22.4\\
AA & 30.5\\
BB & 22.2\\
BB & 18.9\\
AA & 24.9\\
\addlinespace
BB & 27.0\\
\bottomrule
\end{tabular}
\end{table}



Leider kennt sich Steffen mit der Berechnung eines t-Tests überhaupt nicht aus. Deshalb braucht er bei der Berechnung Ihre Hilfe!

\begin{enumerate}
  \item Formulieren Sie die wissenschaftliche Fragestellung! \textbf{(1 Punkt)}
  \item Formulieren Sie das statistische Hypothesenpaar! \textbf{(1 Punkt)}
  \item Bestimmen Sie die Teststatistik $T_{D}$ eines Student t-Tests! \textbf{(3 Punkte)}
\item Treffen Sie mit $T_{\alpha = 5\%} = 1.64$ eine Aussage zur Nullhypothese! Begründen Sie Ihre Antwort! \textbf{(2 Punkte)}
\item Berechnen Sie den Effekt des Student t-Tests! \textbf{(1 Punkt)}
\item Wenn Sie \textit{einen} Unterschied zwischen den Behandlungsgruppen erwarten würden, wie groß wäre dann der \textit{mindeste} Effekt? Begründen Sie Ihre Antwort! \textbf{(2 Punkte)}
\item Formulieren Sie eine Antwort an Steffen über das Ergebnis Ihrer statistischen Analyse! \textbf{(2 Punkte)}
\end{enumerate} 
\clearpage
% -----------------------------------------------------------------------

\section{Aufgabe \hfill (12 Punkte)}

\textit{Geben Sie grundsätzlich Formeln und Rechenweg zur Lösung der Teilaufgaben mit an!} \\[1Ex]
 

 
%% --------------------------------------------------------------------
\ifcollection
\begin{flushright}
\tiny\vspace{-3Ex}
\textbf{\examinhaltstart}
\exammodulemathstat $\;\bullet$
\exammodulestat $\;\bullet$
\exammodulestatbbv 
\vspace{-4Ex}
\end{flushright}
\begin{minipage}[t]{0.5\textwidth}
\includegraphics[width = 1.3cm]{/Users/kruppajo/work/GitHub/exam/avatare/Nilufar.png}
\end{minipage}
\begin{minipage}[t]{0.5\textwidth}
\hfill
\href{https://youtu.be/uQDOuSlHaqw}{\includegraphics[width = 2cm]{img/youtube}}
\end{minipage}
\vspace{-3ex}
\fi
%% --------------------------------------------------------------------



\ifcollection
\paragraph{Berechnung des Welch t-Test}
\fi


Das Oldenburger Land, unendliche Weiten. Wir schreiben das Jahr 2024. Dies sind die Abenteuer von Nilufar, die mit ihrer 1 Frau starken Besatzung 12 Wochen lang unterwegs ist, um neue Welten zu erforschen, neues Leben und neue Zivilisationen. 'Oder nennen wir es Ödnis und Verzweiflung', denkt Nilufar. Für ihrer Hausarbeit ist Nilufar ins Nichts gezogen. Nilufar und die Erwartung, eine unendliche Geschichte mit kniffeligen Wendungen. Was macht sie nun? Nilufar hat ein Kreuzungsexperiment mit Zandern durchgeführt. Die Behandlung Elterlinie ($Standard$ und $Xray$) wurde an Zandern getestet. Gemessen hat sie dann als einen normalverteilten Endpunkt ($Y$) Schlachtgewicht [kg]. Jetzt soll sie ihrem Betreuer nach testen, ob die Behandlung Elterlinie ($Standard$ und $Xray$) ein signifikantes Ergebnis liefert. Hm..., was entspannendes wäre gut. 'Hm...', Takis Blue Heat und Deichkind. Das ist und bleibt die beste Kombination zum Nachdenken für Nilufar.

\begin{table}[!h]
\centering
\begin{tabular}{cc}
\toprule
Elterlinie & Schlachtgewicht\\
\midrule
Standard & 48.6\\
Xray & 46.3\\
Standard & 44.8\\
Standard & 46.3\\
Xray & 55.6\\
\addlinespace
Xray & 31.7\\
Standard & 38.4\\
Standard & 40.8\\
Xray & 38.7\\
Standard & 38.9\\
\addlinespace
Xray & 38.1\\
\bottomrule
\end{tabular}
\end{table}



Leider kennt sich Nilufar mit der Berechnung eines t-Tests überhaupt nicht aus. Deshalb braucht sie bei der Berechnung Ihre Hilfe!

\begin{enumerate}
  \item Formulieren Sie die wissenschaftliche Fragestellung! \textbf{(1 Punkt)}
  \item Formulieren Sie das statistische Hypothesenpaar! \textbf{(1 Punkt)}
  \item Bestimmen Sie die Teststatistik $T_{D}$ eines  Welch t-Tests! \textbf{(3 Punkte)}
  \item Treffen Sie mit $T_{\alpha = 10\%} = 2.86$ eine Aussage zur Nullhypothese! Begründen Sie Ihre Antwort! \textbf{(2 Punkte)}
\item Berechnen Sie das 90\% Konfidenzintervall. Welche Annahmen haben Sie getroffen? \textbf{(2 Punkte)}
\item Nennen Sie den statistischen Grund, warum Sie sich zwischen einem Student t-Test und einem Welch t-Test entscheiden müssen! \textbf{(1 Punkt)}
\item Formulieren Sie eine Antwort an Nilufar über das Ergebnis Ihrer statistischen Analyse! Beziehen Sie dabei das 90\% Konfidenzintervall mit in Ihre Antwort ein! \textbf{(2 Punkte)}
\end{enumerate} 
\clearpage
% -----------------------------------------------------------------------

\section{Aufgabe \hfill (10 Punkte)}

\textit{Geben Sie grundsätzlich Formeln und Rechenweg zur Lösung der Teilaufgaben mit an!} \\[1Ex]
 

 
%% --------------------------------------------------------------------
\ifcollection
\begin{flushright}
\tiny\vspace{-3Ex}
\textbf{\examinhaltstart}
\exammodulemathstat $\;\bullet$
\exammodulestat $\;\bullet$
\exammodulestatbbv $\;\bullet$
\exammodulestatversuch $\;\bullet$
\exammodulebiostat
\vspace{-4Ex}
\end{flushright}
\begin{minipage}[t]{0.5\textwidth}
\includegraphics[width = 1.3cm]{/Users/kruppajo/work/GitHub/exam/avatare/Nilufar.png}\hspace{-4mm}\includegraphics[width = 1.3cm]{/Users/kruppajo/work/GitHub/exam/avatare/Steffen.png}\hspace{-4mm}\includegraphics[width = 1.3cm]{/Users/kruppajo/work/GitHub/exam/avatare/Yuki.png}
\end{minipage}
\begin{minipage}[t]{0.5\textwidth}
\hfill
\href{https://youtu.be/K3LGigegKIg}{\includegraphics[width = 2cm]{img/youtube}}
\end{minipage}
\fi
%% --------------------------------------------------------------------



\ifcollection
\paragraph{Interpretation des t-Tests in \Rlogo - die Teststatistik und der p-Wert}
\fi

'Wir waren im Teuteburgerwald um Zandern in einem Kreuzungsexperiment zu messen.', Steffen legt das Dokument auf den Tisch und schaut Nilufar und Yuki fragend an. Beide schauen fragend zurück. Gäbe es die Faulheit nicht, dann wäre es für Yuki irgendwie einfacher hier zu helfen. Echt unangenehm. Die beiden sind zu Steffen gekommen, da sie sich nicht mit \Rlogo auskennen und daher Hilfe bei der Interpretation des t-Tests brauchen. Im Hintergrund wummert Taylor Swift und leere Oreos Packungen stappeln sich auf dem Boden. 'Kein Problem', sagt Steffen und streichelt langsam die Schlange. 'Aber worum es in dem Versuch geht, lässt sich nur aus dem Text in seiner Hand erahnen.' merkt er an. Vielleicht hilft da ja die Ausgabe des t-Tests in R weiter. Draußen geht blutrot die Sonne unter.

\begin{knitrout}
\definecolor{shadecolor}{rgb}{0.969, 0.969, 0.969}\color{fgcolor}\begin{kframe}
\begin{verbatim}
## 
## 	Two Sample t-test
## 
## data:  Protein/Fettrate by Bestandsdichte
## t = -0.33241, df = 15, p-value = 0.7442
## alternative hypothesis: true  is not equal to [condensed]
## 95 percent confidence interval:
##  -10.122902   7.391474
## sample estimates:
## mean in group Verordnung mean in group Gesteigert 
##                 33.41429                 34.78000
\end{verbatim}
\end{kframe}
\end{knitrout}

Helfen Sie Steffen bei der Interpretation des t-Tests! Sonst geht es auch für Nilufar und Yuki nicht weiter.
  
\begin{enumerate}
  \item Formulieren Sie die wissenschaftliche Fragestellung! \textbf{(1 Punkt)}
  \item Formulieren Sie das statistische Hypothesenpaar! \textbf{(1 Punkt)}
\item Liegt ein signifikanter Unterschied zwischen den Gruppen vor? Begründen Sie Ihre Antwort! \textbf{(2 Punkte)}
\item Berechnen Sie den Effekt des t-Tests! \textbf{(1 Punkt)}
\item Skizzieren Sie eine Abbildung in der Sie $T_{D}$, $Pr(D|H_0)$, $A=0.95$, sowie $T_{\alpha=5\%} = |2.13|$ einzeichnen! \textbf{(4 Punkte)}
\item Beschriften Sie die Abbildung! \textbf{(1 Punkt)}  
\end{enumerate} 
\clearpage
% -----------------------------------------------------------------------

\section{Aufgabe \hfill (10 Punkte)}

\textit{Geben Sie grundsätzlich Formeln und Rechenweg zur Lösung der Teilaufgaben mit an!} \\[1Ex]
 

 
%% --------------------------------------------------------------------
\ifcollection
\begin{flushright}
\tiny\vspace{-3Ex}
\textbf{\examinhaltstart}
\exammodulestatversuch $\;\bullet$
\exammodulebiostat
\vspace{-4Ex}
\end{flushright}
\begin{minipage}[t]{0.5\textwidth}
\includegraphics[width = 1.3cm]{/Users/kruppajo/work/GitHub/exam/avatare/Jessica.png}\hspace{-4mm}\includegraphics[width = 1.3cm]{/Users/kruppajo/work/GitHub/exam/avatare/Nilufar.png}\hspace{-4mm}\includegraphics[width = 1.3cm]{/Users/kruppajo/work/GitHub/exam/avatare/Tina.png}
\end{minipage}
\begin{minipage}[t]{0.5\textwidth}
\hfill
\href{https://youtu.be/zppl5eq4ioU}{\includegraphics[width = 2cm]{img/youtube}}
\end{minipage}
\fi
%% --------------------------------------------------------------------



\ifcollection
\paragraph{Interpretation des t-Tests in \Rlogo - das 95\% Konfidenzintervall}
\fi


'Wir waren im Teuteburgerwald um Zandern in einem Stallexperiment zu messen.', Tina legt das Dokument auf den Tisch und schaut Nilufar und Jessica fragend an. Beide schauen fragend zurück. Gäbe es der Mangel nicht, dann wäre es für Jessica irgendwie einfacher hier zu helfen. Echt unangenehm. Die beiden sind zu Tina gekommen, da sie sich nicht mit \Rlogo auskennen und daher Hilfe bei der Interpretation des t-Tests brauchen. Im Hintergrund wummert Tocotronic und leere Katjes Packungen stappeln sich auf dem Boden. 'Kein Problem', sagt Tina und streichelt langsam die Spinne. 'Aber worum es in dem Versuch geht, lässt sich nur aus dem Text in seiner Hand erahnen.' merkt sie an. Vielleicht hilft da ja die Ausgabe des t-Tests in R weiter. Draußen geht blutrot die Sonne unter.

\begin{knitrout}
\definecolor{shadecolor}{rgb}{0.969, 0.969, 0.969}\color{fgcolor}\begin{kframe}
\begin{verbatim}
## 
## 	Two Sample t-test
## 
## data:  Fettgehalt by Elterlinie
## t = 2.0298, df = 15, p-value = 0.0605
## alternative hypothesis: true  is not equal to [condensed]
## 95 percent confidence interval:
##  -0.2831269 11.5914603
## sample estimates:
## mean in group Standard     mean in group Xray 
##               35.36667               29.71250
\end{verbatim}
\end{kframe}
\end{knitrout}

Helfen Sie Tina bei der Interpretation des t-Tests! Sonst geht es auch für Nilufar und Jessica nicht weiter.

\begin{enumerate}
  \item Formulieren Sie die wissenschaftliche Fragestellung! \textbf{(1 Punkt)}
  \item Formulieren Sie das statistische Hypothesenpaar! \textbf{(1 Punkt)}
\item Liegt ein signifikanter Unterschied zwischen den Gruppen vor? Begründen Sie Ihre Antwort! \textbf{(2 Punkte)}
\item Skizieren Sie das sich ergebende 95\% Konifidenzintervall! \textbf{(2 Punkte)}
\item Beschriften Sie die Abbildung und das 95\% Konfidenzintervall entsprechend! \textbf{(2 Punkte)}  
\item Interpretieren Sie den Effekt des 95\% Konifidenzintervalls! \textbf{(2 Punkte)}
\end{enumerate} 
\clearpage
% -----------------------------------------------------------------------

\section{Aufgabe \hfill (9 Punkte)}

\textit{Geben Sie grundsätzlich Formeln und Rechenweg zur Lösung der Teilaufgaben mit an!} \\[1Ex]
 

 
%% --------------------------------------------------------------------
\ifcollection
\begin{flushright}
\tiny\vspace{-3Ex}
\textbf{\examinhaltstart}
\exammodulemathstat $\;\bullet$
\exammodulestat $\;\bullet$
\exammodulestatbbv $\;\bullet$
\exammodulestatversuch $\;\bullet$
\exammodulebiostat
\vspace{-4Ex}
\end{flushright}
\begin{minipage}[t]{0.5\textwidth}
\includegraphics[width = 1.3cm]{/Users/kruppajo/work/GitHub/exam/avatare/Paula.png}\hspace{-4mm}\includegraphics[width = 1.3cm]{/Users/kruppajo/work/GitHub/exam/avatare/Tina.png}\hspace{-4mm}\includegraphics[width = 1.3cm]{/Users/kruppajo/work/GitHub/exam/avatare/Yuki.png}
\end{minipage}
\begin{minipage}[t]{0.5\textwidth}
\hfill
\href{https://youtu.be/02hibcRs5Wc}{\includegraphics[width = 2cm]{img/youtube}}
\end{minipage}
\fi
%% --------------------------------------------------------------------



\ifcollection
\paragraph{Interpretation des t-Tests in \Rlogo - die Visualisierung}
\fi

'Wir waren im Oldenburger Land um Lamas in einem Stallexperiment zu messen.', Tina legt das Dokument auf den Tisch und schaut Yuki und Paula fragend an. Beide schauen fragend zurück. Gäbe es der Perfektionismus nicht, dann wäre es für Paula irgendwie einfacher hier zu helfen. Echt unangenehm. Die beiden sind zu Tina gekommen, da sie sich nicht mit \Rlogo auskennen und daher Hilfe bei der Interpretation des t-Tests brauchen. Im Hintergrund wummert Tocotronic und leere Katjes Packungen stappeln sich auf dem Boden. 'Kein Problem', sagt Tina und streichelt langsam die Spinne. 'Aber worum es in dem Versuch geht, lässt sich nur aus dem Text in seiner Hand erahnen.' merkt sie an. Vielleicht hilft da ja die Ausgabe des t-Tests in R weiter. Draußen geht blutrot die Sonne unter.

\begin{knitrout}
\definecolor{shadecolor}{rgb}{0.969, 0.969, 0.969}\color{fgcolor}\begin{kframe}
\begin{verbatim}
## 
## 	Two Sample t-test
## 
## data:  Gewichtszuwachs by Flüssignahrung
## t = 1.1979, df = 15, p-value = 0.2495
## alternative hypothesis: true  is not equal to [condensed]
## 95 percent confidence interval:
##  -3.973356 14.170499
## sample estimates:
## mean in group ctrl mean in group flOw 
##           38.82857           33.73000
\end{verbatim}
\end{kframe}
\end{knitrout}

Helfen Sie Tina bei der Interpretation des t-Tests! Sonst geht es auch für Yuki und Paula nicht weiter.
  
\begin{enumerate}
  \item Formulieren Sie die wissenschaftliche Fragestellung! \textbf{(1 Punkt)}
  \item Formulieren Sie das statistische Hypothesenpaar! \textbf{(1 Punkt)}
\item Liegt ein signifikanter Unterschied zwischen den Gruppen vor? Begründen Sie Ihre Antwort! \textbf{(2 Punkte)}
\item Berechnen Sie den Effekt des t-Tests! \textbf{(1 Punkt)}
\item Skizzieren Sie die sich ergebenden Barplots! Begründen Sie Ihre Antwort! \textbf{(2 Punkte)}
\item Skizzieren Sie die sich ergebenden Boxplot! Welche Annahmen an die Daten haben Sie getroffen? Begründen Sie Ihre Antwort! \textbf{(2 Punkte)} 
\end{enumerate}
 
\clearpage
% -----------------------------------------------------------------------

\section{Aufgabe \hfill (11 Punkte)}

\textit{Geben Sie grundsätzlich Formeln und Rechenweg zur Lösung der Teilaufgaben mit an!} \\[1Ex]
 

 
%% --------------------------------------------------------------------
\ifcollection
\begin{flushright}
\tiny\vspace{-3Ex}
\textbf{\examinhaltstart}
\exammodulestat $\;\bullet$
\exammodulestatversuch $\;\bullet$
\exammodulebiostat
\vspace{-4Ex}
\end{flushright}
\begin{minipage}[t]{0.5\textwidth}
\includegraphics[width = 1.3cm]{/Users/kruppajo/work/GitHub/exam/avatare/Alex.png}\hspace{-4mm}\includegraphics[width = 1.3cm]{/Users/kruppajo/work/GitHub/exam/avatare/Yuki.png}
\end{minipage}
\begin{minipage}[t]{0.5\textwidth}
\hfill
\href{https://youtu.be/QR90zyn0Iz8}{\includegraphics[width = 2cm]{img/youtube}}
\end{minipage}
\fi
%% --------------------------------------------------------------------



\ifcollection
\paragraph{Berechnung des gepaarten t-Test}
\fi

Alles voll mit Lauch. Das haben Yuki und Alex gemeinsam in einem Projektbericht gemacht! Worum ging es aber konkret? Beide haben als einen normalverteilten Endpunkt ($Y$) Trockengewicht [kg/ha] von Lauch bestimmt. Die Daten haben beide zusammen in einem Gewächshausexperiment erhoben. In dem Experiment ging es um eine vorher/nachher Untersuchung an den gleichen Lauch. Als Behandlung wurde Beschattung ($7d$ und $14d$) eingesetzt. Nach der Meinung der Betreuerin muss hier ein gepaarter t-Test gerechnet werden. Yuki schaut nachdenklich zu Alex. Wenn die Faulheit nicht wäre, ja dann wäre wohl vieles möglich für Yuki! Aber so... Steffen denkt derweil angestrengt an Gummibärchen. Im Hintergrund wummert London Grammar.

\begin{table}[!h]
\centering
\begin{tabular}{ccc}
\toprule
ID & treatment & freshmatter\\
\midrule
6 & 14d & 30.8\\
8 & 14d & 36.8\\
11 & 7d & 40.6\\
5 & 14d & 44.2\\
1 & 7d & 43.5\\
\addlinespace
6 & 7d & 39.9\\
1 & 14d & 27.3\\
9 & 7d & 43.9\\
5 & 7d & 48.9\\
2 & 14d & 35.3\\
\addlinespace
7 & 7d & 43.3\\
3 & 7d & 40.0\\
4 & 7d & 46.7\\
7 & 14d & 35.8\\
9 & 14d & 22.3\\
\addlinespace
10 & 7d & 46.5\\
4 & 14d & 39.3\\
3 & 14d & 28.4\\
8 & 7d & 44.0\\
2 & 7d & 49.3\\
\bottomrule
\end{tabular}
\end{table}



Leider kennen sich Yuki und Alex mit der Berechnung eines gepaarten t-Tests überhaupt nicht aus. Deshalb brauchen sie beide bei der Berechnung Ihre Hilfe!

\begin{enumerate}
  \item Formulieren Sie die wissenschaftliche Fragestellung! \textbf{(1 Punkt)}
  \item Formulieren Sie das statistische Hypothesenpaar! \textbf{(1 Punkt)}
  \item Bestimmen Sie die Teststatistik $T_{D}$ eines gepaarten t-Tests! \textbf{(3 Punkte)}
  \item Treffen Sie mit $T_{\alpha = 5\%} = 1.84$ eine Aussage zur Nullhypothese! Begründen Sie Ihre Antwort! \textbf{(2 Punkte)}
\item Schätzen Sie den $p$-Wert des gepaarten t-Tests ab! Begründen Sie Ihre Antwort mit einer Skizze! \textbf{(2 Punkte)}
\item Formulieren Sie eine Antwort an Yuki über das Ergebnis Ihrer statistischen Analyse! \textbf{(2 Punkte)}
\end{enumerate}


 
\clearpage
% -----------------------------------------------------------------------

\section{Aufgabe \hfill (10 Punkte)}

\textit{Geben Sie grundsätzlich Formeln und Rechenweg zur Lösung der Teilaufgaben mit an!} \\[1Ex]
 

 
%% --------------------------------------------------------------------
\ifcollection
\begin{flushright}
\tiny\vspace{-3Ex}
\textbf{\examinhaltstart}
\exammodulestat $\;\bullet$
\exammodulestatversuch $\;\bullet$
\exammodulebiostat
\vspace{-4Ex}
\end{flushright}
\begin{minipage}[t]{0.5\textwidth}
\includegraphics[width = 1.3cm]{/Users/kruppajo/work/GitHub/exam/avatare/Nilufar.png}\hspace{-4mm}\includegraphics[width = 1.3cm]{/Users/kruppajo/work/GitHub/exam/avatare/Paula.png}
\end{minipage}
\begin{minipage}[t]{0.5\textwidth}
\hfill
\href{https://youtu.be/kHmfEmU6lrk}{\includegraphics[width = 2cm]{img/youtube}}
\end{minipage}
\fi
%% --------------------------------------------------------------------



\ifcollection
\paragraph{Interpretation des gepaarten t-Tests in \Rlogo}
\fi

Alles voll mit Puten. Aber das haben Nilufar und Paula eben gemeinsam in einem Projektbericht gemacht! Worum ging es aber konkret? Beide haben als einen normalverteilten Endpunkt ($Y$) Fettgehalt [\%/kg] von Puten bestimmt. Die Daten haben beide zusammen in einem Kreuzungsexperiment erhoben. In dem Experiment ging es um eine vorher/nachher Untersuchung an den gleichen Puten. Als Behandlung wurde Flüssignahrung ($1l/d$ und $5l/d$) eingesetzt. Nach der Meinung der Betreuerin muss hier ein gepaarter t-Test gerechnet werden. Leider kennen sich beide nicht sehr gut in \Rlogo aus. Nilufar hat einiges an Takis Blue Heat geholt, so dass beide die Zeit gut durchbringen werden. Dann geht Nilufar nochmal zum Sport. Einfach mal raus um zu Kicken. Ohne Ziel und Uhr. Das ist was für Nilufar.

\begin{knitrout}
\definecolor{shadecolor}{rgb}{0.969, 0.969, 0.969}\color{fgcolor}\begin{kframe}
\begin{verbatim}
## 
## 	Paired t-test
## 
## data:  Fettgehalt by Flüssignahrung
## t = 1.596, df = 8, p-value = 0.1492
## alternative hypothesis: true  is not equal to [condensed]
## 95 percent confidence interval:
##  -2.268797 12.468797
## sample estimates:
## mean difference 
##             5.1
\end{verbatim}
\end{kframe}
\end{knitrout}

Jetzt brauchen Nilufar und Paula Ihre Hilfe bei der Berechnung eines gepaarten t-Tests in \Rlogo um ihre Arbeit dann in diesem Semester noch abschließen zu können.

\begin{enumerate}
  \item Formulieren Sie die wissenschaftliche Fragestellung! \textbf{(1 Punkt)}
  \item Formulieren Sie das statistische Hypothesenpaar! \textbf{(1 Punkt)}
\item Liegt ein signifikanter Unterschied zwischen den Gruppen vor?
  Begründen Sie Ihre Antwort! \textbf{(2 Punkte)}
\item Skizzieren Sie das sich ergebende 95\% Konfidenzintervall! \textbf{(2 Punkte)}
\item Interpretieren Sie den Effekt des gepaarten t-Tests! \textbf{(2 Punkte)}
\item Skizzieren Sie den sich ergebenden Boxplot der Differenzen! Welche Annahmen an die Daten haben Sie getroffen? Begründen Sie Ihre Antwort! \textbf{(2 Punkte)} 
\end{enumerate}
 
\clearpage
% -----------------------------------------------------------------------
\part{Die einfaktorielle \& zweifaktorielle ANOVA}
% -----------------------------------------------------------------------

\section{Aufgabe \hfill (11 Punkte)}

\textit{Geben Sie grundsätzlich Formeln und Rechenweg zur Lösung der Teilaufgaben mit an!} \\[1Ex]
 

 
%% --------------------------------------------------------------------
\ifcollection
\begin{flushright}
\tiny\vspace{-3Ex}
\textbf{\examinhaltstart}
\exammodulemathstat $\;\bullet$
\exammodulestat $\;\bullet$
\exammodulestatbbv 
\vspace{-4Ex}
\end{flushright}
\begin{minipage}[t]{0.5\textwidth}
\includegraphics[width = 1.3cm]{/Users/kruppajo/work/GitHub/exam/avatare/Jonas.png}\hspace{-4mm}\includegraphics[width = 1.3cm]{/Users/kruppajo/work/GitHub/exam/avatare/Yuki.png}
\end{minipage}
\begin{minipage}[t]{0.5\textwidth}
\hfill
\href{https://youtu.be/wEePzcwwti8}{\includegraphics[width = 2cm]{img/youtube}}
\end{minipage}
\fi
%% --------------------------------------------------------------------



\ifcollection
\paragraph{Visualisierung der einfaktoriellen ANOVA}
\fi

'Uff... die einfaktorielle ANOVA. Und wir können jetzt anhand der Visualisuierung sehen, ob da schon was signifikant ist?', Yuki hebt die Augenbraue. 'Ja, können wir. Dafür müssen wir aber erstmal in \texttt{\{ggplot\}} uns die Daten anschauen. Oder wir zeichnen es flott mit der Hand. Geht auch.', meint Jonas dazu. Die beiden hatten sich auf einem Konzert von Iron Maiden kennengelernt. Yuki hatte sich in ein Gewächshausexperiment verschiedene Kartoffeln angeschaut. Dabei ging es herauszufinden, ob es einen Zusammenhang zwischen der Behandlung Bewässerungstypen ($low$, $mid$ und $high$) und dem Messwert Chlorophyllgehalt (SPAD-502Plus) [SPAD] gibt. Später wird noch Mission Impossible geguckt. Jonas befürwortet das!

\begin{knitrout}
\definecolor{shadecolor}{rgb}{0.969, 0.969, 0.969}\color{fgcolor}\begin{table}[!h]
\centering
\begin{tabular}{cc}
\toprule
Bewässerungstypen & Chlorophyllgehalt\\
\midrule
low & 30\\
mid & 44\\
mid & 43\\
low & 29\\
mid & 46\\
\addlinespace
low & 32\\
high & 32\\
high & 38\\
low & 32\\
low & 29\\
\addlinespace
low & 31\\
low & 32\\
high & 34\\
high & 35\\
high & 34\\
\addlinespace
high & 33\\
mid & 44\\
high & 37\\
mid & 45\\
mid & 45\\
\bottomrule
\end{tabular}
\end{table}

\end{knitrout}

Leider kennen sich Yuki und Jonas mit Darstellung einer einfaktoriellen ANOVA überhaupt nicht aus. 

\begin{enumerate}
\item Erstellen  Sie  eine  Visualisierung  der  Datentabelle! Beschriften  Sie  die  Abbildung! \textbf{(2 Punkte)}
\item Benennen Sie die Visualisierung mit dem korrekten, statistischen Fachbegriff! \textbf{(1 Punkt)}
\item Zeichnen Sie folgende statistischen Maßzahlen passend ein! 
  \begin{itemize}
  \item Den globalen Mittelwert $\beta_0$ \textbf{(1 Punkt)}
  \item Die Mittelwerte der einzelnen Behandlungsstufen \textbf{(1 Punkt)}
  \item Die Mittelwertsdifferenz der einzelnen Behandlungsstufen mit $\beta_{low}$, $\beta_{mid}$ und $\beta_{high}$ \textbf{(1 Punkt)}
  \item Die Residuen oder Fehler mit $\epsilon$ \textbf{(1 Punkt)}
  \end{itemize}
\item Liegt ein \textit{vermutlicher} signifikanter Unterschied vor? Begründen Sie Ihre Antwort! \textbf{(2 Punkte)}
\item Schätzen Sie die Effekte der Behandlungsstufen! \textbf{(2 Punkte)}
\end{enumerate}
 
\clearpage
% -----------------------------------------------------------------------

\section{Aufgabe \hfill (10 Punkte)}

\textit{Geben Sie grundsätzlich Formeln und Rechenweg zur Lösung der Teilaufgaben mit an!} \\[1Ex]
 

 
%% --------------------------------------------------------------------
\ifcollection
\begin{flushright}
\tiny\vspace{-3Ex}
\textbf{\examinhaltstart}
\exammodulemathstat $\;\bullet$
\exammodulestat $\;\bullet$
\exammodulestatbbv 
\vspace{-4Ex}
\end{flushright}
\begin{minipage}[t]{0.5\textwidth}
\includegraphics[width = 1.3cm]{/Users/kruppajo/work/GitHub/exam/avatare/Nilufar.png}\hspace{-4mm}\includegraphics[width = 1.3cm]{/Users/kruppajo/work/GitHub/exam/avatare/Paula.png}
\end{minipage}
\begin{minipage}[t]{0.5\textwidth}
\hfill
\href{https://youtu.be/25bvWshX-Gw}{\includegraphics[width = 2cm]{img/youtube}}
\end{minipage}
\fi
%% --------------------------------------------------------------------



\ifcollection
\paragraph{Ergebnistabelle der einfaktoriellen ANOVA}
\fi

'Als erstes bauen wir uns aus unsere Daten die ANOVA Tabelle dann sehen wir schon, ob unser Gruppenvergleich in der ANOVA signifikant ist.', Nilufar schaut Paula fragend an und hofft auf eine positive Regung im Gesicht. Wird aber enttäuscht. Da hilft die Ratte von Paula auch nur bedingt. Paula tut sich auch sehr schwer mit der einfaktoriellen ANOVA. Beide waren im Emsland um einen Leistungssteigerungsversuch mit Puten durchzuführen. Dabei ging es herauszufinden, ob es einen Zusammenhang zwischen der Behandlung Elterlinie ($ctrl$, $Standard$, $Yray$ und $Xray$) und dem Messwert Fettgehalt [\%/kg] gibt. Nachher wollen sich beide noch mit dem Hobby Harry Potter von Paula beschäftigen. Kennt Nilufar noch nicht, klingt aber interessant.



\vspace{1ex}

Leider kennen sich Nilufar und Paula mit Berechnung einer einfaktoriellen ANOVA überhaupt nicht aus. Deshalb brauchen beide bei der Erstellung Ihre Hilfe, die Ratte reicht als Hilfe nicht aus! 

\begin{enumerate}
  \item Formulieren Sie die wissenschaftliche Fragestellung! \textbf{(1 Punkt)}
  \item Formulieren Sie das statistische Hypothesenpaar! \textbf{(2 Punkte)}
\item Füllen Sie die unterstehende einfaktorielle ANOVA Ergebnistabelle aus! \textbf{(3 Punkte)}
\end{enumerate}

\vspace{1Ex}

\begin{center}
  \Large
  \begin{tabular}{lccccp{3cm}}
\toprule
     & \textbf{Df} & \textbf{Sum Sq} & \textbf{Mean Sq} & \textbf{F value} & \textbf{Pr(>F)} \strut\\
    \midrule
   \textbf{Elterlinie}  & 3 & 107.94 &  &  &  \strut\\
   \textbf{error}  & 25 & 343.3 &  &  &  \strut\\
   \textbf{Total}  & 28 &  &  &  &  \strut\\
\bottomrule
  \end{tabular}
\end{center}

\vspace{1Ex}

\begin{enumerate}
  \setcounter{enumi}{3}
\item Schätzen Sie den p-Wert der Tabelle mit $F_{\alpha = 5\%} = 2.99$ ab. Begründen Sie Ihre Antwort! \textbf{(2 Punkte)}
\item Berechen Sie den Effektschätzer $\eta^2$. Was sagt Ihnen der Wert von $\eta^2$ aus? \textbf{(2 Punkte)}
\end{enumerate}



 
\clearpage
% -----------------------------------------------------------------------

\section{Aufgabe \hfill (10 Punkte)}

\textit{Geben Sie grundsätzlich Formeln und Rechenweg zur Lösung der Teilaufgaben mit an!} \\[1Ex]
 

 
%% --------------------------------------------------------------------
\ifcollection
\begin{flushright}
\tiny\vspace{-3Ex}
\textbf{\examinhaltstart}
\exammodulemathstat $\;\bullet$
\exammodulestat $\;\bullet$
\exammodulestatbbv
\vspace{-4Ex}
\end{flushright}
\begin{minipage}[t]{0.5\textwidth}
\includegraphics[width = 1.3cm]{/Users/kruppajo/work/GitHub/exam/avatare/Tina.png}
\end{minipage}
\begin{minipage}[t]{0.5\textwidth}
\hfill
\href{https://youtu.be/UOzpeBYWgVQ}{\includegraphics[width = 2cm]{img/youtube}}
\end{minipage}
\vspace{-3Ex}
\fi
%% --------------------------------------------------------------------



\ifcollection
\paragraph{Die einfaktorielle ANOVA in \Rlogo}
\fi

Tina schaut entnervt auf und klappt den Laptop zu. Tina dreht Tocotronic auf, so dass sich die Nachbarn beschweren werden. Nun möchte ihr Betreuer ihrem Projektbericht erstmal eine ANOVA sehen und \textit{dann} die Ergebnisse präsentiert bekommen bevor es überhaupt mit der Abschlussarbeit weitergeht. Dabei war sie extra im Emsland um ein Freilandversuch mit Maiss durchzuführen. Und dort was es wirklich nicht schön geschweige denn spannend wie bei ihren Kommilitonen, die in Almería waren. Hätte sie es vorher gewusst, dann hätte sie die Abschlussarbeit bei wem anders geschrieben. Aber gut, jetzt als die ANOVA in \Rlogo.

\begin{knitrout}
\definecolor{shadecolor}{rgb}{0.969, 0.969, 0.969}\color{fgcolor}\begin{kframe}
\begin{verbatim}
## Analysis of Variance Table
## 
## Response: Proteingehalt
##             Df Sum Sq Mean Sq F value    Pr(>F)
## Bewässerung  3 4335.7 1445.23  50.645 1.956e-11
## Residuals   28  799.0   28.54
\end{verbatim}
\end{kframe}
\end{knitrout}

\vspace{1ex}

Leider kennen sich Tina mit Berechnung einer einfaktoriellen ANOVA überhaupt nicht aus. Deshalb braucht sie bei der Erstellung Ihre Hilfe! 

\begin{enumerate}
  \item Formulieren Sie die wissenschaftliche Fragestellung! \textbf{(1 Punkt)}
  \item Formulieren Sie das statistische Hypothesenpaar! \textbf{(2 Punkte)}
\item Interpretieren Sie das Ergebnis der einfaktoriellen ANOVA! \textbf{(2 Punkte)} 
\item Berechnen Sie den Effektschätzer $\eta^2$. Was sagt Ihnen der Wert von $\eta^2$ aus? \textbf{(2 Punkte)}
\item Skizzieren Sie eine Abbildung, der dem obigen Ergebnis der
  einfaktoriellen ANOVA näherungsweise entspricht! \textbf{(3 Punkte)}
\end{enumerate}

 
\clearpage
% -----------------------------------------------------------------------

\section{Aufgabe \hfill (12 Punkte)}

\textit{Geben Sie grundsätzlich Formeln und Rechenweg zur Lösung der Teilaufgaben mit an!} \\[1Ex]
 

 
%% --------------------------------------------------------------------
\ifcollection
\begin{flushright}
\tiny\vspace{-3Ex}
\textbf{\examinhaltstart}
\exammodulestatversuch $\;\bullet$
\exammodulebiostat
\vspace{-4Ex}
\end{flushright}
\begin{minipage}[t]{0.5\textwidth}
\includegraphics[width = 1.3cm]{/Users/kruppajo/work/GitHub/exam/avatare/Alex.png}\hspace{-4mm}\includegraphics[width = 1.3cm]{/Users/kruppajo/work/GitHub/exam/avatare/Jonas.png}
\end{minipage}
\begin{minipage}[t]{0.5\textwidth}
\hfill
\href{https://youtu.be/G-AW_--5V0w}{\includegraphics[width = 2cm]{img/youtube}}
\end{minipage}
\fi
%% --------------------------------------------------------------------



\ifcollection
\paragraph{Die einfaktoriellen ANOVA und der Student t-Test}
\fi

'Uff... die einfaktorielle ANOVA. Und wie füllen wir jetzt 	extit{genau} die Tabelle der ANOVA aus und schauen, ob da was signifikant ist?', Jonas hebt die Augenbraue. 'Das ist eine sehr gute Frage. Ich glaube man kann alles in der Tabelle relativ einfach mit wenigen Informationen berechnen.', meint Alex dazu und schmiß sich noch ein paar Gummibärchen in den Rachen. Jonas hatte sich in ein Stallexperiment verschiedene Milchvieh angeschaut. Dabei ging es herauszufinden, ob es einen Zusammenhang zwischen der Behandlung Genotypen ($AA$, $AB$ und $BB$) und dem Messwert Schlachtgewicht [kg] gibt. Nun möchte erstmal sein Betreuer eine ANOVA Tabelle sehen. Was immer da auch drin zu erkennen sein mag. Später wollen die beiden dann noch raus um zu Laufen.



\vspace{1ex}

Leider kennen sich Jonas und Alex mit Berechnung einer einfaktoriellen ANOVA überhaupt nicht aus. Deshalb brauchen beide bei der Erstellung Ihre Hilfe! 

\begin{enumerate}
  \item Formulieren Sie die wissenschaftliche Fragestellung! \textbf{(1 Punkt)}
  \item Formulieren Sie das statistische Hypothesenpaar! \textbf{(1 Punkt)}
\item Füllen Sie die unterstehende einfaktorielle ANOVA Ergebnistabelle aus! \textbf{(3 Punkte)}
\end{enumerate}

\vspace{1Ex}

\begin{center}
  \Large
  \begin{tabular}{lccccp{3cm}}
\toprule
     & \textbf{Df} & \textbf{Sum Sq} & \textbf{Mean Sq} & \textbf{F value} & \textbf{Pr(>F)} \strut\\
    \midrule
   \textbf{Genotypen}  & 2 & 53.4 &  &  &  \strut\\
   \textbf{Error}  & 21 & 175.56 &  &  &  \strut\\
\bottomrule
  \end{tabular}
\end{center}

\vspace{1Ex}

\begin{enumerate}
  \setcounter{enumi}{3}
\item Schätzen Sie den p-Wert der Tabelle mit $F_{\alpha = 5\%} = 3.47$ ab. Begründen Sie Ihre Antwort! \textbf{(2 Punkte)}
\item Was bedeutet ein signifikantes Ergebnis in einer einfaktoriellen ANOVA? \textbf{(1 Punkt)}
\item Berechnen Sie \textit{einen} Student t-Test für den \textit{vermutlich} signifikantesten Gruppenvergleich anhand der untenstehenden Tabelle mit $T_{\alpha = 5\%} = 2.03$. Begründen Sie Ihre Auswahl! \textbf{(3 Punkte)}
\end{enumerate}


\begin{knitrout}
\definecolor{shadecolor}{rgb}{0.969, 0.969, 0.969}\color{fgcolor}\begin{table}[!h]
\centering\begingroup\fontsize{11}{13}\selectfont

\begin{tabular}{cccc}
\toprule
\textbf{Genotypen} & \textbf{Fallzahl (n)} & \textbf{Mittelwert} & \textbf{Standardabweichung}\\
\midrule
AA & 9 & 1.56 & 3.00\\
AB & 6 & 3.33 & 1.03\\
BB & 9 & 5.00 & 3.50\\
\bottomrule
\end{tabular}
\endgroup{}
\end{table}

\end{knitrout}


\begin{enumerate}
  \setcounter{enumi}{6}
\item Gegebenen der ANOVA Tabelle war das Ergebnis des Student t-Tests zu erwarten? Begründen Sie Ihre Antwort! \textbf{(2 Punkte)}
\end{enumerate}

 
\clearpage
% -----------------------------------------------------------------------

\section{Aufgabe \hfill (12 Punkte)}

\textit{Geben Sie grundsätzlich Formeln und Rechenweg zur Lösung der Teilaufgaben mit an!} \\[1Ex]
 

 
%% --------------------------------------------------------------------
\ifcollection
\begin{flushright}
\tiny\vspace{-3Ex}
\textbf{\examinhaltstart}
\exammodulestatversuch $\;\bullet$
\exammodulebiostat
\vspace{-4Ex}
\end{flushright}
\begin{minipage}[t]{0.5\textwidth}
\includegraphics[width = 1.3cm]{/Users/kruppajo/work/GitHub/exam/avatare/Paula.png}
\end{minipage}
\begin{minipage}[t]{0.5\textwidth}
\hfill
\href{https://youtu.be/1uZsKAipACI}{\includegraphics[width = 2cm]{img/youtube}}
\end{minipage}
\vspace{-3Ex}
\fi
%% --------------------------------------------------------------------



\ifcollection
\paragraph{Ergebnistabelle der zweifaktoriellen ANOVA}
\fi

Wie absolut ärgerlich. Jetzt stellt sich tatsächlich heraus, dass ihrem Betreuer keine Ahnung von der zweifaktoriellen ANOVA hat. Woher soll Paula jetzt das Wissen nehmen? Paula mampft aus Frust noch eine Handvoll Smarties. Immerhin muss sie ja noch mit ihrer Abschlussarbeit dieses Jahr fertig werden. In einen Versuch in einer Klimakammer hatte sie Spargel mit der Behandlung Lichtstufen ($none$, $200lm$, $400lm$ und $600lm$) sowie der Behandlung Düngestufen ($ctrl$, und $high$) im Emsland untersucht. Es wurde als Messwert Trockengewicht [kg/ha] bestimmt. Jetzt muss sie erstmal die zweifaktorielle ANOVA verstehen. Und eigentlich wollte Paula doch noch zum Sport! Um zu Fechten geht Paula dann später nochmal raus. Echte Entspannung.



\vspace{1ex}

Leider kennen sich Paula mit Berechnung einer zweifaktoriellen ANOVA überhaupt nicht aus. Deshalb braucht sie bei der Erstellung Ihre Hilfe! 

\begin{enumerate}
  \item Formulieren Sie die wissenschaftliche Fragestellung für beide Faktoren separat! \textbf{(2 Punkte)}
  \item Formulieren Sie die statistischen Hypothesenpaare für beide Faktoren separat! \textbf{(2 Punkte)}
\item Füllen Sie die unterstehende zweifaktorielle ANOVA Ergebnistabelle aus! \textbf{(3 Punkte)}
\end{enumerate}

\vspace{1Ex}

\begin{center}
  \Large
  \begin{tabular}{lccccc}
  \toprule
     & \textbf{Df} & \textbf{Sum Sq} & \textbf{Mean Sq} & \textbf{F value} & \textbf{Pr(>F)} \strut\\
    \midrule
   \textbf{Lichtstufen}  & 3 & 353.65 &  &  &  \strut\\
    \textbf{Düngestufen}  & 1 & 14.87 &  &  &  \strut\\
    \textbf{Lichtstufen:Düngestufen}  & 3 & 84.81 &  &  &  \strut\\
   \textbf{Error}  & 18 & 448.63 &  &  &  \strut\\
\bottomrule
  \end{tabular}
\end{center}

\vspace{1Ex}

\begin{enumerate}
  \setcounter{enumi}{3}
\item Schätzen Sie den p-Wert der Tabelle ab. Begründen Sie Ihre Antwort! \textbf{(3 Punkte)}
\end{enumerate}
  
\begin{center}
    \Large
\begin{tabular}{lc}
  \toprule
     & $\boldsymbol{F_{\alpha = 5\%}}$ \\
\midrule
  \textbf{Lichtstufen} & $4.26$ \\
  \textbf{Düngestufen} & $3.40$ \\
  \textbf{Lichtstufen:Düngestufen} & $5.23$ \\
  \bottomrule
  \end{tabular}
\end{center}

\begin{enumerate}
  \setcounter{enumi}{4}
\item Was sagt der Term \textit{Lichtstufen:Düngestufen} aus? Interpretieren Sie das Ergebnis! \textbf{(2 Punkte)}
\end{enumerate}
 
\clearpage
% -----------------------------------------------------------------------

\section{Aufgabe \hfill (12 Punkte)}

\textit{Geben Sie grundsätzlich Formeln und Rechenweg zur Lösung der Teilaufgaben mit an!} \\[1Ex]
 

 
%% --------------------------------------------------------------------
\ifcollection
\begin{flushright}
\tiny\vspace{-3Ex}
\textbf{\examinhaltstart}
\exammodulestatversuch $\;\bullet$
\exammodulebiostat
\vspace{-4Ex}
\end{flushright}
\begin{minipage}[t]{0.5\textwidth}
\includegraphics[width = 1.3cm]{/Users/kruppajo/work/GitHub/exam/avatare/Jessica.png}
\end{minipage}
\begin{minipage}[t]{0.5\textwidth}
\hfill
\href{https://youtu.be/i7TadGcyn8c}{\includegraphics[width = 2cm]{img/youtube}}
\end{minipage}
\vspace{-3Ex}
\fi
%% --------------------------------------------------------------------



\ifcollection
\paragraph{Die zweifaktorielle ANOVA in \Rlogo}
\fi

'Mit der zweifaktoriellen ANOVA lässt sich die Interaktion zwischen den beiden Behandlungen nachweisen!', ihr Betreuer scheint die zweifaktoriellen ANOVA zu verstehen. Warum jetzt sie jetzt nochmal alles wiederkäuen muss, wird Jessica echt nicht so klar. Wenn es doch so klar ist? Jessica war im Teuteburgerwald und hatte dort ein Freilandversuch mit Kartoffeln durchgeführt. Die Komune wo sie untergekommen war, war cool gewesen. Nur jetzt muss eben das Experiment fertig ausgewertet werden. Jessica hatte zwei Behandlungen auf Kartoffeln angewendet. Einmal Bewässerungstypen ($ctrl$, $low$, $mid$ und $high$) sowie als zweite Behandlung Düngestufen ($ctrl$, und $high$). Gemessen wurde der Messwert ($Y$) Chlorophyllgehalt (SPAD-502Plus) [SPAD]. Jetzt muss das hier zu einem Ende kommen! Eigentlich wollte Jessica nachher noch einen Film schauen. Wenn Herr der Ringe läuft, dann ist die Hündin nicht mehr da. Aber jetzt braucht sie mal Entspannung!

\begin{knitrout}
\definecolor{shadecolor}{rgb}{0.969, 0.969, 0.969}\color{fgcolor}\begin{kframe}
\begin{verbatim}
## Analysis of Variance Table
## 
## Response: Chlorophyllgehalt
##                               Df  Sum Sq Mean Sq F value    Pr(>F)
## Bewässerungstypen              3 240.755 120.378 18.2591 1.513e-05
## Düngestufen                    1   7.011   7.011  1.0635    0.3127
## Bewässerungstypen:Düngestufen  3 247.543 123.772 18.7739 1.236e-05
## Residuals                     24 158.226   6.593
\end{verbatim}
\end{kframe}
\end{knitrout}

\vspace{1ex}

Leider kennt sich Jessica mit Berechnung einer zweifaktoriellen ANOVA überhaupt nicht aus. Deshalb braucht sie bei der Erstellung Ihre Hilfe! 

\begin{enumerate}
  \item Formulieren Sie die wissenschaftliche Fragestellung für beide Faktoren separat! \textbf{(2 Punkte)}
  \item Formulieren Sie die statistischen Hypothesenpaare für beide Faktoren separat! \textbf{(2 Punkte)}
\item Interpretieren Sie das Ergebnis der zweifaktoriellen ANOVA! Begründen Sie Ihre Antwort! \textbf{(3 Punkte)} 
\item Zeichnen Sie eine Abbildung, der dem obigen Ergebnis der
  zweifaktoriellen ANOVA näherungsweise entspricht! \textbf{(5 Punkte)}
\end{enumerate}
 
\clearpage
% -----------------------------------------------------------------------

\section{Aufgabe \hfill (11 Punkte)}

\textit{Geben Sie grundsätzlich Formeln und Rechenweg zur Lösung der Teilaufgaben mit an!} \\[1Ex]
 

 
%% --------------------------------------------------------------------
\ifcollection
\begin{flushright}
\tiny\vspace{-3Ex}
\textbf{\examinhaltstart}
\exammodulestatversuch $\;\bullet$
\exammodulebiostat
\vspace{-4Ex}
\end{flushright}
\begin{minipage}[t]{0.5\textwidth}
\includegraphics[width = 1.3cm]{/Users/kruppajo/work/GitHub/exam/avatare/Steffen.png}
\end{minipage}
\begin{minipage}[t]{0.5\textwidth}
\hfill
\href{https://youtu.be/-yJWhI3Uy5g}{\includegraphics[width = 2cm]{img/youtube}}
\end{minipage}
\vspace{-3Ex}
\fi
%% --------------------------------------------------------------------



\ifcollection
\paragraph{Interaktion in der zweifaktoriellen ANOVA}
\fi

Es ist schon kurz nach fünf und Steffen wird langsam nervös. Steffen wollte heute Abend noch seine E-Sport Qualifikation schauen und dann zum Sport. Stattdessen versucht seine Betreuerin die Ausgabe der zweifaktoriellen ANOVA zu visualieren und zu überprüfen, ob es mit der Visualisierung der Daten als Boxplots zusammenpasst. Es liegt anscheinend eine signifikante Interaktion vor? Steffen hatte im Teuteburgerwald ein Kreuzungsexperiment mit Schweinen durchgeführt. Es gab dabei zwei Behandlungen. Einmal Elterlinie ($ctrl$, $Standard$, $Yray$ und $Xray$) sowie als zweite Behandlung Genotypen ($AA$ und $BB$). Gemessen wurde der Messwert ($Y$) Gewichtszuwachs in der 1LW. So kompliziert kann das jetzt doch nicht sein! Eigentlich wollte Steffen nachher noch zum Sport. Um zu Ringen geht Steffen dann später nochmal raus. Echte Entspannung.

\vspace{1ex}

Leider kennen sich Steffen und seine Betreuerin mit der zweifaktoriellen ANOVA überhaupt nicht aus. Geschweige denn mit der Interpretation einer Interaktion. Deshalb braucht er bei der Erstellung Ihre Hilfe, sonst wird es heute Abend mit seinem Hobby Klemmbausteine nichts mehr! 

\begin{enumerate}
\item Visualisieren Sie folgende mögliche Interaktionen zwischen den Behandlungen! Heben Sie die Positionen der auftretenden Interaktion (farblich) hervor! Beschriften Sie die Abbildung! \textbf{(4 Punkte)}
\begin{enumerate}
\item \underline{Keine} Interaktion liegt vor.
\item Eine \underline{schwache} Interaktion liegt vor. 
\item Eine \underline{starke} Interaktion liegt vor. 
\end{enumerate}
\item Erklären Sie den Unterschied zwischen den verschiedenen Interaktionen! \textbf{(2 Punkte)}
\item Welche statistische Maßzahl betrachten Sie für die Bewertung der Interaktion? \textbf{(1 Punkt)}
\item Schreiben Sie das Modell mit Interaktionsterm in \Rlogo für eine zweifaktorielle ANOVA! \textbf{(1 Punkt)} 
\item Skizzieren Sie die Funktion in \Rlogo für eine Post-hoc Analyse in \texttt{\{emmeans\}}! \textbf{(1 Punkt)} 
\item Wenn eine signifikante Interaktion in den Daten vorliegt, wie ist dann das weitere Vorgehen? Berücksichtigen Sie dabei die Verwendung von \texttt{\{emmeans\}}! \textbf{(2 Punkte)}
\end{enumerate}

 
\clearpage
% -----------------------------------------------------------------------

\section{Aufgabe \hfill (10 Punkte)}

\textit{Geben Sie grundsätzlich Formeln und Rechenweg zur Lösung der Teilaufgaben mit an!} \\[1Ex]
 

 
%% --------------------------------------------------------------------
\ifcollection
\begin{flushright}
\tiny\vspace{-3Ex}
\textbf{\examinhaltstart}
\exammodulebiostat
\vspace{-4Ex}
\end{flushright}
\begin{minipage}[t]{0.5\textwidth}
\includegraphics[width = 1.3cm]{/Users/kruppajo/work/GitHub/exam/avatare/Yuki.png}
\end{minipage}
\begin{minipage}[t]{0.5\textwidth}
\hfill
\href{https://youtu.be/X0-2he7ByMA}{\includegraphics[width = 2cm]{img/youtube}}
\end{minipage}
\vspace{-3Ex}
\fi
%% --------------------------------------------------------------------



\ifcollection
\paragraph{Zusammenhang zwischen der ANOVA und dem Post-hoc-Test}
\fi

'Mit der einfaktoriellen ANOVA lassen sich flott die Gruppen in einer Behandlungen vergleichen, wenn wir normalverteilte Daten und Varianzhomogenität vorliegen haben!', ihre Betreuerin scheint die einfaktoriellen ANOVA zu verstehen. Warum jetzt sie jetzt nochmal alles wiederkäuen muss, wird Yuki echt nicht so klar. Wenn es doch so klar ist? 'Wir haben jetzt bei der ANOVA einen p-Wert mit 0.071 raus sowie eine F-Statistik $F_D$ mit 1.2 berechnet. Nach den Boxplots müsste sich eigentlich ein Unterschied zwischen $thunder$ und $keins$ ergeben. Der Unterschied ist in \texttt{\{emmeans\}} auch signifikant mit einem p-Wert von 0.021. Wie kann das sein?', fragt Yuki etwas provokant und dreht London Grammar leiser. Yuki war im Wendland und hatte dort ein Stallexperiment mit Lamas durchgeführt. Die Komune wo sie untergekommen war, war cool gewesen. Dort gab es selbstgemachte Reese's Peanut Butter Cups aus Vollkorn! Nur jetzt muss eben das Experiment fertig ausgewertet werden. Yuki hatte eine Behandlungen Lüftungssystem ($keins$, $storm$, $tornado$ und $thunder$) auf Lamas angewendet. Gemessen wurde der Messwert ($Y$) Schlachtgewicht [kg]. Dabei wurden die Daten $D$ erhoben. Jetzt muss das hier zu einem Ende kommen! Yuki hat schon genug Probleme. Wenn die Faulheit nicht wäre, dann wäre es einfacher.

\begin{graybox}{Gegebene Formeln}
\begin{center}
  \begin{tabular}{ccc}
    $MS_{treatment} = \cfrac{SS_{treatment}}{df_{treatment}}$ &
    $MS_{error} = \cfrac{SS_{error}}{df_{error}}$ &
    $F_{D} = \cfrac{MS_{treatment}}{MS_{error}}$ \\
  \end{tabular}
\end{center}
\end{graybox}

Leider kennen sich Yuki und ihre Betreuerin mit der Interpretation einer ANOVA überhaupt nicht aus. Deshalb braucht sie bei der Erstellung Ihre Hilfe und die Zeit wird knapp. 

\begin{enumerate}
\item Formulieren Sie die wissenschaftliche Fragestellung! \textbf{(1 Punkt)}
\item Formulieren Sie das statistische Hypothesenpaar! \textbf{(1 Punkt)}
\item Was bedeutet eine signifikante ANOVA für die beobachteten Daten $D$? \textbf{(1 Punkt)}
\item Visualisieren Sie den Unterschied zwischen Varianzhomogenität und Varianzheterogenität anhand der Daten $D$! Beschriften Sie die Abbildung! \textbf{(2 Punkte)} 
\item Welche Auswirkung hat die Verletzung der Annahme der Varianzhomogenität für die Teststatistik $F_D$ der ANOVA? Begründen Sie Ihre Antwort! \textbf{(3 Punkte)}
\item Erklären Sie abschließend die Diskrepanz zwischen den Ergebnis der ANOVA und dem paarweisen Gruppenvergleich in \texttt{\{emmeans\}}! \textbf{(2 Punkte)}
\end{enumerate}

 
\clearpage
% -----------------------------------------------------------------------
\part{Multiple Gruppenvergleiche}
% ----------------------------------------------------------------------- 

\section{Aufgabe \hfill (12 Punkte)}

\textit{Geben Sie grundsätzlich Formeln und Rechenweg zur Lösung der Teilaufgaben mit an!} \\[1Ex]
 

 
%% --------------------------------------------------------------------
\ifcollection
\begin{flushright}
\tiny\vspace{-3Ex}
\textbf{\examinhaltstart}
\exammodulestatversuch $\;\bullet$
\exammodulebiostat
\vspace{-4Ex}
\end{flushright}
\begin{minipage}[t]{0.5\textwidth}
\includegraphics[width = 1.3cm]{/Users/kruppajo/work/GitHub/exam/avatare/Nilufar.png}\hspace{-4mm}\includegraphics[width = 1.3cm]{/Users/kruppajo/work/GitHub/exam/avatare/Paula.png}
\end{minipage}
\begin{minipage}[t]{0.5\textwidth}
\hfill
\href{https://youtu.be/uXwlbuHOj7Y}{\includegraphics[width = 2cm]{img/youtube}}
\end{minipage}
\fi
%% --------------------------------------------------------------------



\ifcollection
\paragraph{Adjustierung multipler Vergleiche}
\fi

'Moment, die haben ja das Gleiche gemacht wie wir!', ruft Nilufar laut aus. Paula schaut etwas verwundert. 'Das glaube ich eher nicht. Lass uns mal unsere Daten mit den Ergebnissen von Meyer et al. (2021) vergleichen.', antwortet Paula. In ein Stallexperiment mit Milchvieh wurde die Behandlung Ernährungszusatz ($ctrl$, $fedX$, $proteinX$ und $getIt$) auf den Messwert Gewichtszuwachs in der 1LW untersucht. Jetzt müssen die beiden mal schauen, ob sie wirklich was Neues gefunden haben oder ob die Ergebnisse alle die gleichen sind wie schon bei Meyer et al. (2021). Es ergab sich dann die folgende Tabelle der rohen p-Werte für die Vergleiche zu Meyer et al. (2021).

\begin{knitrout}
\definecolor{shadecolor}{rgb}{0.969, 0.969, 0.969}\color{fgcolor}\begin{table}[!h]
\centering\begingroup\fontsize{10}{12}\selectfont

\begin{tabular}{ccc}
\toprule
\textbf{Rohen p-Werte} & \textbf{Adjustierte p-Werte} & \textbf{Nullhypothese ablehnen?}\\
\midrule
0.0200 &  & \\
0.0012 &  & \\
0.0700 &  & \\
0.0010 &  & \\
\bottomrule
\end{tabular}
\endgroup{}
\end{table}

\end{knitrout}

Leider kennen sich Nilufar und Paula mit der Adjustierung von $p$-Werten und dem Signifikanzniveau $\alpha$ überhaupt nicht aus. Deshalb brauchen die beiden bei der Erstellung Ihre Hilfe!

\begin{enumerate}
  \item Formulieren Sie die wissenschaftliche Fragestellung! \textbf{(1 Punkt)}
  \item Formulieren Sie die statistischen Hypothesen! \textbf{(1 Punkt)}
\item Füllen Sie die Spalte \textit{Adjustierte p-Werte} nach der Bonferoni-Methode aus! \textbf{(2 Punkte)}
\item Entscheiden Sie, ob nach der Adjustierung die Nullhypothese abgelehnt werden kann! Begründen Sie Ihre Antwort! \textbf{(2 Punkte)}
\item Wie ist Ihr Vorgehen, wenn Sie anstatt der $p$-Werte das Signifikanzniveau $\alpha$ adjustieren? \textbf{(2 Punkte)}
\item Erklären Sie warum die $p$-Werte oder das Signifikanzniveau $\alpha$ bei multiplen Vergleichen adjustiert werden müssen! \textbf{(2 Punkte)}
\item Welche Adjustierung wird im Allgemeinen vorgezogen? Die Adjustierung der $p$-Werte oder die Adjustierung des Signifikanzniveaus $\alpha$? Begründen Sie Ihre Antwort! \textbf{(2 Punkte)}
\end{enumerate}


 
\clearpage
% ----------------------------------------------------------------------- 

\section{Aufgabe \hfill (10 Punkte)}

\textit{Geben Sie grundsätzlich Formeln und Rechenweg zur Lösung der Teilaufgaben mit an!} \\[1Ex]
 

 
%% --------------------------------------------------------------------
\ifcollection
\begin{flushright}
\tiny\vspace{-3Ex}
\textbf{\examinhaltstart}
\exammodulestat $\;\bullet$
\exammodulestatbbv $\;\bullet$
\exammodulestatversuch $\;\bullet$
\exammodulebiostat
\vspace{-4Ex}
\end{flushright}
\begin{minipage}[t]{0.5\textwidth}
\includegraphics[width = 1.3cm]{/Users/kruppajo/work/GitHub/exam/avatare/Mark.png}
\end{minipage}
\begin{minipage}[t]{0.5\textwidth}
\hfill
\href{https://youtu.be/VNz7nDuK148}{\includegraphics[width = 2cm]{img/youtube}}
\end{minipage}
\vspace{-3ex}
\fi
%% --------------------------------------------------------------------



\ifcollection
\paragraph{Visualisierung des Compact Letter Displays (CLD)}
\fi

Mark betrachtet in sich gekehrt die Poster vor dem Büro von seinem Betreuer. Viele der explorativen Abbildungen sagen ihm etwas. Die Barplots und die Boxplots könnte er dann schon nachbauen. Das macht ihn dann zuversichtlich die Abschlussarbeit auch hinzukriegen. Etwas komischer sind die seltsamen Buchstaben über den Barplots. Mark betrachtet ein Poster das sich mit Zandern beschäftigt. Lüftungssystem ($keins$, $storm$, $tornado$ und $thunder$) und Gewichtszuwachs in der 1LW wurden dort bestimmt. So richtig schlau, wird er daraus nicht.

\begin{knitrout}
\definecolor{shadecolor}{rgb}{0.969, 0.969, 0.969}\color{fgcolor}\begin{table}[!h]
\centering\begingroup\fontsize{10}{12}\selectfont

\begin{tabular}{cc}
\toprule
\textbf{Behandlung} & \textbf{Compact letter display}\\
\midrule
keins & AB\\
storm & B\\
tornado & A\\
thunder & A\\
\bottomrule
\end{tabular}
\endgroup{}
\end{table}

\end{knitrout}

Leider kennen sich Mark mit dem \textit{Compact letter display (CLD)} überhaupt nicht aus. Deshalb braucht er bei der Erstellung Ihre Hilfe!

\begin{enumerate}
  \item Formulieren Sie die wissenschaftliche Fragestellung! \textbf{(1 Punkt)}
  \item Formulieren Sie die statistischen Hypothesen! \textbf{(1 Punkt)}
\item Zeichnen Sie die sich anhand des \textit{Compact letter display (CLD)} ergebenden Barplots! \textbf{(2 Punkte)}
\item Ergänzen Sie das \textit{Compact letter display (CLD)} zu den Barplots! \textbf{(1 Punkt)}
\item Erklären Sie \textit{einen} Vorteil und \textit{einen} Nachteil des \textit{Compact letter display (CLD)}! \textbf{(2 Punkte)}
\item Erstellen Sie eine Matrix mit den paarweisen $p$-Werten eines Student t-Tests, die sich näherungsweise aus dem \textit{Compact letter display (CLD)} ergeben würde! Begründen Sie Ihre Antwort! \textbf{(3 Punkte)}
\end{enumerate}

 
\clearpage
% ----------------------------------------------------------------------- 

\section{Aufgabe \hfill (12 Punkte)}

\textit{Geben Sie grundsätzlich Formeln und Rechenweg zur Lösung der Teilaufgaben mit an!} \\[1Ex]
 

 
%% --------------------------------------------------------------------
\ifcollection
\begin{flushright}
\tiny\vspace{-3Ex}
\textbf{\examinhaltstart}
\exammodulestatversuch $\;\bullet$
\exammodulebiostat
\vspace{-4Ex}
\end{flushright}
\begin{minipage}[t]{0.5\textwidth}
\includegraphics[width = 1.3cm]{/Users/kruppajo/work/GitHub/exam/avatare/Alex.png}
\end{minipage}
\begin{minipage}[t]{0.5\textwidth}
\hfill
\href{https://youtu.be/951MNUXdqwA}{\includegraphics[width = 2cm]{img/youtube}}
\end{minipage}
\vspace{-3ex}
\fi
%% --------------------------------------------------------------------



\ifcollection
\paragraph{Berechnung des Compact Letter Displays (CLD) anhand von t-Tests}
\fi

Alex betrachtet in sich gekehrt die Poster vor dem Büro von seine Betreuerin. Viele der explorativen Abbildungen sagen ihm etwas. Die Barplots und die Boxplots könnte er dann schon nachbauen. Das macht ihn dann zuversichtlich die Abschlussarbeit auch hinzukriegen. Etwas komischer sind die seltsamen Buchstaben über den Barplots. Alex betrachtet ein Poster das sich mit Milchvieh beschäftigt. Bestandsdichte ($effizient$, $standard$, $eng$ und $kontakt$) und Fettgehalt [\%/kg] wurden dort bestimmt. So richtig schlau, wird er daraus nicht. Als erstes müsse müsse man die Gruppen nach absteigender Effektstärke sortieren, liest Alex im Methodenteil und ist dann noch verwirrter als vorher schon.

\begin{knitrout}
\definecolor{shadecolor}{rgb}{0.969, 0.969, 0.969}\color{fgcolor}\begin{table}[!h]
\centering\begingroup\fontsize{10}{12}\selectfont

\begin{tabular}{cccc}
\toprule
\textbf{Bestandsdichte} & \textbf{Fallzahl (n)} & \textbf{Mittelwert} & \textbf{Standardabweichung}\\
\midrule
effizient & 8 & 15.75 & 2.23\\
standard & 9 & 16.09 & 1.84\\
eng & 7 & 15.44 & 2.12\\
kontakt & 9 & 3.84 & 4.02\\
\bottomrule
\end{tabular}
\endgroup{}
\end{table}

\end{knitrout}

Leider kennen sich Alex mit dem \textit{Compact letter display (CLD)} überhaupt nicht aus. Deshalb braucht er bei der Erstellung Ihre Hilfe!

\begin{enumerate}
  \item Formulieren Sie die wissenschaftliche Fragestellung! \textbf{(1 Punkt)}
  \item Formulieren Sie die statistischen Hypothesen! \textbf{(1 Punkt)}
\item Zeichnen Sie die sich ergebenden Barplots! \textbf{(1 Punkt)}
\item Berechnen Sie die Matrix der $p$-Werte anhand von Student t-Tests! \textit{Nutzen Sie hierfür ein globales $s_p$ sowie eine gemittelte Fallzahl $n$ für die Berechnung der Teststatistik! Nehmen Sie einen kritischen Wert $T_{\alpha = 5\%}$ von 1.96 an!} \textbf{(4 Punkte)}
\item Ergänzen Sie das \textit{Compact letter display (CLD)} zu den gezeichneten Barplots! Begründen Sie Ihre Antwort! \textbf{(4 Punkte)}
\item Interpretieren Sie das \textit{Compact letter display (CLD)} für Alex! \textbf{(1 Punkt)} 
\end{enumerate}

 
\clearpage
% -----------------------------------------------------------------------
\part{Der Chi-Quadrat-Test}
% -----------------------------------------------------------------------

\section{Aufgabe \hfill (10 Punkte)}

\textit{Geben Sie grundsätzlich Formeln und Rechenweg zur Lösung der Teilaufgaben mit an!} \\[1Ex]
 

 
%% --------------------------------------------------------------------
\ifcollection
\begin{flushright}
\tiny\vspace{-3Ex}
\textbf{\examinhaltstart}
\exammodulestat $\;\bullet$
\exammodulestatbbv 
\vspace{-4Ex}
\end{flushright}
\begin{minipage}[t]{0.5\textwidth}
\includegraphics[width = 1.3cm]{/Users/kruppajo/work/GitHub/exam/avatare/Nilufar.png}
\end{minipage}
\begin{minipage}[t]{0.5\textwidth}
\hfill
\href{https://www.youtube.com/1y_X_TI-Gm4}{\includegraphics[width = 2cm]{img/youtube}}
\end{minipage}
\vspace{-3Ex}
\fi
%% --------------------------------------------------------------------



\ifcollection
\paragraph{Den Chi-Quadrat-Test berechnen}
\fi

Nilufar hat sich ein Herz gefasst und war für ihrer Hausarbeit in die Niederlande gegangen. Das war eine super Zeit in der sie viel gelernt hat. Klar gab es auch die ein oder andere Besonderheit, aber das gehört hier eher nicht hin. Dann noch schnell Deichkind auf das Ohr und los gehts. Nilufar ist schon eine ganze Zeit im Büro, da ihr Betreuer möchte, dass sie jetzt auf ihren Daten mit $n = 124$ Beobachtungen von Lamas einen $\mathcal{X}^2$-Test rechnet. Das ginge, da sie als Behandlung \textit{Außenklimakontakt [ja/nein]} bestimmt und zum anderen die Variable \textit{Protein/Fettrate im Zielbereich [ja/nein]} ermittelt hat. Wie genau, das ist jetzt eine andere Frage. Eigentlich wollte Nilufar nachher noch einen Film schauen. Das Verrückte ist, dass das Huhn Star Trek wirklich liebt. Das ist Nilufar sehr recht, denn sie braucht Entspannung.

\vspace{5Ex}

\begin{center}
  \huge
  \begin{tabular}{c|l|l|c}
     & \phantom{\textbf{Erkrankt (ja)}} & \phantom{\textbf{Erkrankt (ja)}} & \phantom{\textbf{Erkrankt (ja)}} \strut\\
    \hline
    \phantom{\textbf{Pestizid (ja)}} & 38  & 11  &     \strut\\
    \hline
    \phantom{\textbf{Pestizid (ja)}} & 23  & 52  &      \strut\\
    \hline
     \phantom{100} & \phantom{100}  & \phantom{100}  &  \phantom{100}  \strut\\
  \end{tabular}
\end{center}

\vspace{5Ex}

Leider kennt sich Nilufar mit der Berechnung eines $\mathcal{X}^2$-Test für kategoriale Daten überhaupt nicht aus. Deshalb braucht sie bei der Erstellung Ihre Hilfe!

\begin{enumerate}
\item Formulieren Sie die wissenschaftliche Fragestellung! \textbf{(1 Punkt)}
\item Ergänzen Sie die Tabelle um die fehlenden Informationen! \textbf{(1 Punkt)} 
\item Visualisieren Sie den Zusammenhang zwischen den beiden kategorialen Variablen! \textbf{(2 Punkte)}
\item Berechnen Sie die Teststatistik eines Chi-Quadrat-Test! \textbf{(2 Punkte)}
\item Treffen Sie eine Entscheidung im Bezug zu der Nullhypothese gegeben
  einem $\mathcal{X}^2_{\alpha = 5\%} = 6.21$! Begründen Sie Ihre Antwort!
  \textbf{(2 Punkte)}
\item Skizzieren Sie in einer Abbildung die $\mathcal{X}^2$-Verteilung, wenn die $H_0$ wahr ist! Ergänzen Sie  $\mathcal{X}^2_{\alpha = 5\%}$ und $\mathcal{X}^2_{D}$ in der Abbildung! \textit{Beachten Sie folgenden Informationen zur $\mathcal{X}^2$-Verteilung. Die $\mathcal{X}^2$-Verteilung hat ein Maxima bei $\mathcal{X}^2 = 5$ sowie ein Minima bei $\mathcal{X}^2 = 8$.} \textbf{(2 Punkte)}
\end{enumerate} 
\clearpage
% -----------------------------------------------------------------------

\section{Aufgabe \hfill (11 Punkte)}

\textit{Geben Sie grundsätzlich Formeln und Rechenweg zur Lösung der Teilaufgaben mit an!} \\[1Ex]
 

 
%% --------------------------------------------------------------------
\ifcollection
\begin{flushright}
\tiny\vspace{-3Ex}
\textbf{\examinhaltstart}
\exammodulestat $\;\bullet$
\exammodulestatbbv 
\vspace{-4Ex}
\end{flushright}
\begin{minipage}[t]{0.5\textwidth}
\includegraphics[width = 1.3cm]{/Users/kruppajo/work/GitHub/exam/avatare/Alex.png}
\end{minipage}
\begin{minipage}[t]{0.5\textwidth}
\hfill
\href{https://youtu.be/IcjUIoEo0rY}{\includegraphics[width = 2cm]{img/youtube}}
\end{minipage}
\vspace{-3Ex}
\fi
%% --------------------------------------------------------------------



\ifcollection
\paragraph{Den Chi-Quadrat-Test mit Effektmaß berechnen}
\fi

Am Ende hätte Alex dann doch einen normalverteilten Endpunkt in seinem Projektbericht nehmen sollen. Vor ihm liegen jetzt die Daten von zwei Variablen als Kategorien oder wie es in \Rlogo so schön heißt, als Faktoren. Aber immerhin, hofft er das was bei den Daten rausgekommen ist. Dann noch schnell Abba auf das Ohr und los gehts. Gezählt hat Alex einiges mit $n = 148$ Beobachtungen von Erdbeeren. Zum einen hat er als Behandlung \textit{Pestizideinsatz [ja/nein]} bestimmt und zum anderen die Variable \textit{Trockengewicht über Zielwert [ja/nein]} ermittelt. Nun möchte sein Betreuer gerne einen $\mathcal{X}^2$-Test auf einer $2x2$-Kreuztabelle berechnet bekommen. Eigentlich wollte Alex nachher noch einen Film schauen. Wenn Alien läuft, dann ist die Katze nicht mehr da. Aber jetzt braucht er mal Entspannung!

\vspace{5Ex}

\begin{center}
  \huge
  \begin{tabular}{c|l|l|c}
     & \phantom{\textbf{Erkrankt (ja)}} & \phantom{\textbf{Erkrankt (ja)}} & \phantom{\textbf{Erkrankt (ja)}} \strut\\
    \hline
    \phantom{\textbf{Pestizid (ja)}} & 24  & 19  &     \strut\\
    \hline
    \phantom{\textbf{Pestizid (ja)}} & 61  & 44  &      \strut\\
    \hline
     \phantom{100} & \phantom{100}  & \phantom{100}  &  \phantom{100}  \strut\\
  \end{tabular}
\end{center}

\vspace{5Ex}

Leider kennt sich Alex mit der Berechnung eines $\mathcal{X}^2$-Test für kategoriale Daten überhaupt nicht aus. Deshalb braucht er bei der Erstellung Ihre Hilfe!

\begin{enumerate}
\item Formulieren Sie die wissenschaftliche Fragestellung! \textbf{(1 Punkt)}
\item Ergänzen Sie die Tabelle um die fehlenden Informationen! \textbf{(1 Punkt)} 
\item Berechnen Sie die Teststatistik eines Chi-Quadrat-Test! \textbf{(3 Punkte)}
\item Treffen Sie eine Entscheidung im Bezug zu der Nullhypothese gegeben
  einem $\mathcal{X}^2_{\alpha = 5\%} = 5.61$! Begründen Sie Ihre Antwort!
  \textbf{(2 Punkte)}
\item Berechnen Sie den Effektschätzer $Cramers\; V$ auf der 2x2 Kreuztabelle! \textbf{(1 Punkt)}
\item Welchen Wertebereich kann der Effektschätzer $Cramers\; V$ annehmen? Wann liegt kein Effekt und wann ein starker Effekt vor? \textbf{(2 Punkte)}
\item Interpretieren Sie den berechneten Effektschätzer $Cramers\; V$ unter Berücksichtigung der Fragestellung! \textbf{(1 Punkt)}
\end{enumerate} 
\clearpage
% -----------------------------------------------------------------------

\section{Aufgabe \hfill (10 Punkte)}

\textit{Geben Sie grundsätzlich Formeln und Rechenweg zur Lösung der Teilaufgaben mit an!} \\[1Ex]
 

 
%% --------------------------------------------------------------------
\ifcollection
\begin{flushright}
\tiny\vspace{-3Ex}
\textbf{\examinhaltstart}
\exammodulestat $\;\bullet$
\exammodulestatbbv 
\vspace{-4Ex}
\end{flushright}
\begin{minipage}[t]{0.5\textwidth}
\includegraphics[width = 1.3cm]{/Users/kruppajo/work/GitHub/exam/avatare/Steffen.png}
\end{minipage}
\begin{minipage}[t]{0.5\textwidth}
\hfill
\href{https://youtu.be/LzAAdJPH4ZU}{\includegraphics[width = 2cm]{img/youtube}}
\end{minipage}
\vspace{-3Ex}
\fi
%% --------------------------------------------------------------------



\ifcollection
\paragraph{Den Chi-Quadrat-Test in einem Fragebogen berechnen}
\fi

Steffen hatte sich gleich von Beginn an in seinem Projektbericht für eine Umfrage im Marketing interessiert. Jetzt geht es um den Haupt- und Nebenerwerb von Erlebnishöfen in Norddeutschland. Viele Höfe haben angefangen auch Großkatzen zu halten, damit mehr Kunden auf die Höfe kommen. Für den Verband der Großkatzenbesitzer e.V. möchte er nun einen Fragebogen zur Zukunftsfähigkeit Schritt für Schritt auswerten. Dabei teilt er zuerst die Antwortenden in die beiden Gruppen 'Höfe mit Großkatzen [ja]' und 'Höfe mit Großkatzen [nein]' ein. Daraufhin möchte er für folgende Frage \textit{f3verband} einmal auswerten, ob es einen Unterschied zwischen den beiden Höfen mit oder ohne Großkatzen gibt. \begin{center}\textit{Halten Sie die Verbandsarbeit für die Verbreitung von Großkatzen im ländlichen Raum für sinnvoll?}\end{center}Steffen krazt sich an seinem Kopf. Wie soll man eine Tabelle mit so vielen Zahlen sinnvoll auswerten? Schnell noch ein paar Oreos einwerfen und los gehts!

\vspace{5Ex}

\begin{center}
  \Large
  \begin{tabular}{c|l|l|l|l|l|c}
\textit{f3verband}     & \textbf{trifft gar} & \textbf{trifft} & \textbf{weder} & \textbf{trifft} & \textbf{trifft} & \\
 \phantom{\textbf{Pestizid}}    & \textbf{nicht zu} & \textbf{nicht zu} & \textbf{noch} & \textbf{zu} & \textbf{voll zu} & \\
    \hline
   \textbf{ja}  & 11  & 25  &  8  & 4  & 3  &   \strut\\[4Ex]
    \hline
    \textbf{nein} & 9  & 21  & 11  & 9  & 5  &      \strut\\[4Ex]
    \hline
     \phantom{\textbf{Pestizid}} & \phantom{\textbf{trifft gar}}  & \phantom{\textbf{trifft gar}}  &  \phantom{\textbf{trifft gar}} &  \phantom{\textbf{trifft gar}} &  \phantom{\textbf{trifft gar}} &  \phantom{100} \strut\\
  \end{tabular}
\end{center}

\vspace{5Ex}

Leider kennt sich Steffen mit der Berechnung eines $\mathcal{X}^2$-Test auf einer Frage in einem Fragebogen überhaupt nicht aus. Deshalb braucht er bei der Erstellung Ihre Hilfe!

\begin{enumerate}
\item Formulieren Sie die wissenschaftliche Fragestellung! \textbf{(1 Punkt)}
\item Ergänzen Sie die Tabelle um die fehlenden Informationen! \textbf{(1 Punkt)} 
\item Berechnen Sie die Teststatistik $\mathcal{X}^2_{D}$ eines Chi-Quadrat-Test! \textit{Ignorieren Sie Zellbelegungen \underline{kleiner gleich} fünf in der Berechnung von $\mathcal{X}^2_{D}$ und Runden Sie den erwarteten Wert $E$ ganzzahlig!} \textbf{(2 Punkte)}
\item Treffen Sie eine Entscheidung im Bezug zu der Nullhypothese gegeben
  einem $\mathcal{X}^2_{\alpha = 5\%} = 23.83$! Begründen Sie Ihre Antwort!
  \textbf{(2 Punkte)}
\item Visualisieren Sie die 2x5 Kreuztabelle \textit{ohne} die Berücksichtigung der Antwortkategorie 'weder noch'! \textbf{(2 Punkte)}
\item Berechnen Sie den Effektschätzer $Cramers\; V$ auf der 2x5 Kreuztabelle! \textbf{(1 Punkt)}
\item Interpretieren Sie den berechneten Effektschätzer $Cramers\; V$ unter Berücksichtigung der Fragestellung! \textbf{(1 Punkt)}
\end{enumerate} 
\clearpage
% -----------------------------------------------------------------------
\part{Lineare Regression \& Korrelation}
% -----------------------------------------------------------------------

\section{Aufgabe \hfill (10 Punkte)}

\textit{Geben Sie grundsätzlich Formeln und Rechenweg zur Lösung der Teilaufgaben mit an!} \\[1Ex]
 

 
%% --------------------------------------------------------------------
\ifcollection
\begin{flushright}
\tiny\vspace{-3Ex}
\textbf{\examinhaltstart}
\exammodulestat $\;\bullet$
\exammodulestatbbv $\;\bullet$
\exammodulestatversuch 
\vspace{-4Ex}
\end{flushright}
\begin{minipage}[t]{0.5\textwidth}
\includegraphics[width = 1.3cm]{/Users/kruppajo/work/GitHub/exam/avatare/Jonas.png}\hspace{-4mm}\includegraphics[width = 1.3cm]{/Users/kruppajo/work/GitHub/exam/avatare/Tina.png}
\end{minipage}
\begin{minipage}[t]{0.5\textwidth}
\hfill
\href{https://youtu.be/LtbC_K9V5wU}{\includegraphics[width = 2cm]{img/youtube}}
\end{minipage}
\fi
%% --------------------------------------------------------------------



\ifcollection
\paragraph{Visualisierung der linearen Regression}
\fi

'Hä? Hatten wir das als Aufgabe nicht schon mal, das wir aus kontinuierlichen Daten eine Abbildung bauen sollten?', fragt Jonas. Tina schaut fragend zurück. 'Kann mich wie immer an nichts erinnern. Können wir trotzdem jetzt erstmal die Daten auswerten? Columbo?', antwortet Tina leicht angespannt. Die beiden hatten einen Versuch in einer Klimakammer im Wendland mit Erdbeeren durchgeführt. Dabei wurden die beiden folgenden Variablen gemessen: durchschnittliche UV Einstrahlung [UV/d] und Proteingehalt [g/kg]. Jetzt haben die beiden eigentlich alles zusammen. \textit{Eigentlich...}

\begin{table}[!h]
\centering
\begin{tabular}{cc}
\toprule
Durchschnittliche UV Einstrahlung [UV/d] & Proteingehalt [g/kg]\\
\midrule
20.3 & 10.4\\
20.5 & 10.2\\
19.2 & 7.0\\
20.9 & 15.7\\
22.3 & 12.0\\
\addlinespace
20.5 & 13.5\\
21.2 & 10.4\\
22.8 & 9.9\\
20.9 & 14.1\\
17.9 & 10.3\\
\addlinespace
23.0 & 16.7\\
\bottomrule
\end{tabular}
\end{table}



Leider kennen sich Jonas und Tina mit der linearen Regression für kontinuierliche Daten überhaupt nicht aus. Deshalb brauchen beide bei der Erstellung Ihre Hilfe!

\begin{enumerate}
\item Formulieren Sie die wissenschaftliche Fragestellung! \textbf{(1 Punkt)}
\item Erstellen  Sie  eine  Visualisierung  für  die  Datentabelle.  Beschriften  Sie  die  Achsen! \textbf{(2 Punkte)}
\item Schätzen Sie die Regressionsgleichung aus der obigen Abbildung ab! Begründen Sie Ihre Antwort mit einer Skizze der Methodik an ausgewählten Punkten! \textbf{(2 Punkte)}
\item Beschriften Sie die Grade mit den statistischen Maßzahlen der linearen Regressionsgleichung! \textbf{(2 Punkte)}
\item Liegt ein Zusammenhang zwischen $x$ und $y$ vor? Begründen Sie Ihre Antwort! \textbf{(2 Punkte)}
\item Wenn kein Zusammenhang zu beobachten wäre, wie würde die Grade aussehen? \textit{Antworten Sie mit einer Skizze der Geraden!} \textbf{(1 Punkt)}
\end{enumerate} 
\clearpage
% -----------------------------------------------------------------------

\section{Aufgabe \hfill (12 Punkte)}

\textit{Geben Sie grundsätzlich Formeln und Rechenweg zur Lösung der Teilaufgaben mit an!} \\[1Ex]
 

 
%% --------------------------------------------------------------------
\ifcollection
\begin{flushright}
\tiny\vspace{-3Ex}
\textbf{\examinhaltstart}
\exammodulestat $\;\bullet$
\exammodulestatbbv $\;\bullet$
\exammodulestatversuch 
\vspace{-4Ex}
\end{flushright}
\begin{minipage}[t]{0.5\textwidth}
\includegraphics[width = 1.3cm]{/Users/kruppajo/work/GitHub/exam/avatare/Jessica.png}\hspace{-4mm}\includegraphics[width = 1.3cm]{/Users/kruppajo/work/GitHub/exam/avatare/Steffen.png}
\end{minipage}
\begin{minipage}[t]{0.5\textwidth}
\hfill
\href{https://youtu.be/5eP9VjecZmk}{\includegraphics[width = 2cm]{img/youtube}}
\end{minipage}
\fi
%% --------------------------------------------------------------------



\ifcollection
\paragraph{Interpretation der Ergebnisse einer linearen Regression}
\fi

'Ich glaube du bringst da was durcheinander. Wir nutzen zwar auch für die ANOVA die Funktion \texttt{lm()} aber hier wollen wir, glaube ich, eine Gerade durch die Punkte zeichnen.', merkt Jessica an. 'Ich sehe keine Punkte... ich sehe nur zwei Zeilen einer Tabelle und ich glaube du hast gerade was gelöscht.', antwortet Steffen sichtlich übernächtigt. 'Wir müssen die Koeffizienten der linearen Regression ja auch erst interpretieren!', spricht Jessica sehr deutlich und langsam. Die beiden hatten ein Kreuzungsexperiment im Emsland mit Fleischrindern durchgeführt. Dabei wurden die beiden folgenden Variablen gemessen: durchschnittliche Tagestemperatur [C/d] und Protein/Fettrate [\%/kg]. Das Bestimmtheitsmaß $R^2$ hatten die beiden mit 0.2 bestimmt.  Jetzt wollen sie erstmal schauen, ob es einen Zusammenhang gibt und das soll mit der \Rlogo Ausgabe möglich sein.

\begin{table}[!h]
\centering\begingroup\fontsize{11}{13}\selectfont

\begin{tabular}{ccccc}
\toprule
term & estimate & std.error & t statistic & p-value\\
\midrule
(Intercept) & 1.80 & 1.68 &  & \\
Durchschnittliche Tagestemperatur & 0.62 & 0.16 &  & \\
\bottomrule
\end{tabular}
\endgroup{}
\end{table}



Leider kennen sich Jessica und Steffen mit der linearen Regression für kontinuierliche Daten in \Rlogo überhaupt nicht aus. Deshalb brauchen beide bei der Erstellung Ihre Hilfe!

\begin{enumerate}
\item Formulieren Sie die wissenschaftliche Fragestellung! \textbf{(1 Punkt)}
\item Formulieren Sie die Regressionsgleichung! \textbf{(1 Punkt)}
\item Erstellen  Sie  eine  Visualisierung  der \texttt{lm()}-Ausgabe. \textit{Beachten Sie die Informationen zum Bestimmtheitsmaß $R^2$ aus dem Aufgabentext!} Beschriften  Sie  die  Achsen! \textbf{(2 Punkte)}
\item Beschriften Sie die Visualisierung mit den statistischen Maßzahlen der der \texttt{lm()}-Ausgabe! \textbf{(2 Punkte)}
\item Ergänzen Sie die t-Statistik in der \texttt{lm()}-Ausgabe! \textbf{(2 Punkte)}
\item Ergänzen Sie den $p$-Wert in der \texttt{lm()}-Ausgabe mit $T_{\alpha = 5\%} = 1.96$!  \textbf{(2 Punkte)}
\item Interpretieren Sie den $p$-Wert im Kontext der wissenschaftlichen Fragestellung! \textbf{(1 Punkt)}  
\item Wie groß ist der Effekt im Kontext der wissenschaftlichen Fragestellung? \textbf{(1 Punkt)}
\end{enumerate} 
\clearpage
% -----------------------------------------------------------------------

\section{Aufgabe \hfill (11 Punkte)}

\textit{Geben Sie grundsätzlich Formeln und Rechenweg zur Lösung der Teilaufgaben mit an!} \\[1Ex]
 

 
%% --------------------------------------------------------------------
\ifcollection
\begin{flushright}
\tiny\vspace{-3Ex}
\textbf{\examinhaltstart}
\exammodulestat $\;\bullet$
\exammodulestatbbv $\;\bullet$
\exammodulestatversuch $\;\bullet$
\exammodulebiostat
\vspace{-4Ex}
\end{flushright}
\begin{minipage}[t]{0.5\textwidth}
\includegraphics[width = 1.3cm]{/Users/kruppajo/work/GitHub/exam/avatare/Paula.png}\hspace{-4mm}\includegraphics[width = 1.3cm]{/Users/kruppajo/work/GitHub/exam/avatare/Yuki.png}
\end{minipage}
\begin{minipage}[t]{0.5\textwidth}
\hfill
\href{https://youtu.be/I48edRN4bAU}{\includegraphics[width = 2cm]{img/youtube}}
\end{minipage}
\fi
%% --------------------------------------------------------------------



\ifcollection
\paragraph{Interpretation der Ergebnisse einer linearen Regression in \Rlogo}
\fi

'Ich glaube du bringst da was durcheinander. Wir nutzen zwar auch für die ANOVA die Funktion \texttt{lm()} aber hier wollen wir, glaube ich, eine Gerade durch die Punkte zeichnen.', merkt Paula an. 'Ich sehe keine Punkte... das ist doch eine Ausgabe in \Rlogo. Überhaupt, darum geht es doch gar nicht in unserem Versuch. Wir wollen doch keine Gerade zeichnen?.', antwortet Yuki sichtlich übernächtigt. 'Doch wir müssen nur die Koeffizienten der linearen Regression erst richtig interpretieren und vor unserem geistigen Auge erscheint eine Gerade!', spricht Paula sehr deutlich und langsam. Die beiden hatten ein Stallexperiment im Emsland mit Fleischrindern durchgeführt. Dabei wurden die beiden folgenden Variablen gemessen: mittlere Anzahl an weißen Blutkörperchen [LEU/ml] und Fettgehalt [\%/kg]. Jetzt wollen sie erstmal schauen, ob es einen Zusammenhang gibt und das soll mit der \Rlogo Ausgabe möglich sein.

\begin{knitrout}
\definecolor{shadecolor}{rgb}{0.969, 0.969, 0.969}\color{fgcolor}\begin{kframe}
\begin{verbatim}
## 
## Call:
## Fettgehalt ~ Mittlere_Anzahl
## 
## Residuals:
##     Min      1Q  Median      3Q     Max 
## -2.7234 -1.1770 -0.3044  1.0514  3.9454 
## 
## Coefficients:
##                 Estimate Std. Error t value Pr(>|t|)
## (Intercept)       0.6134     2.0735   0.296  0.76914
## Mittlere_Anzahl   0.6193     0.2049   3.023  0.00473
## 
## Residual standard error: 1.618 on 34 degrees of freedom
## Multiple R-squared:  0.2118,	Adjusted R-squared:  0.1887 
## F-statistic: 9.139 on 1 and 34 DF,  p-value: 0.004734
\end{verbatim}
\end{kframe}
\end{knitrout}

Leider kennen sich Paula und Yuki mit der linearen Regression für kontinuierliche Daten in \Rlogo überhaupt nicht aus. Deshalb brauchen beide bei der Erstellung Ihre Hilfe!


\begin{enumerate}
\item Formulieren Sie die wissenschaftliche Fragestellung! \textbf{(1 Punkt)}
\item Wie groß ist der Effekt im Kontext der wissenschaftlichen Fragestellung? \textbf{(2 Punkte)} 
\item Interpretieren Sie die $p$-Werte im Kontext der wissenschaftlichen Fragestellung! \textbf{(2 Punkte)}
\item Visualisieren Sie die Verteilung der Residuen! \textbf{(2 Punkte)} 
\item Ist die Annahme der Normalverteilung erfüllt? Begründen Sie die Antwort! \textbf{(2 Punkte)}
\item Erklären Sie \textit{kurz} den Begriff \texttt{R-squared}! Was sagt Ihnen der Wert aus? \textbf{(2 Punkte)}
\end{enumerate}
 
\clearpage
% -----------------------------------------------------------------------

\section{Aufgabe \hfill (10 Punkte)}

\textit{Geben Sie grundsätzlich Formeln und Rechenweg zur Lösung der Teilaufgaben mit an!} \\[1Ex]
 

 
%% --------------------------------------------------------------------
\ifcollection
\begin{flushright}
\tiny\vspace{-3Ex}
\textbf{\examinhaltstart}
\exammodulestat $\;\bullet$
\exammodulestatbbv $\;\bullet$
\exammodulestatversuch $\;\bullet$
\exammodulebiostat
\vspace{-4Ex}
\end{flushright}
\begin{minipage}[t]{0.5\textwidth}
\includegraphics[width = 1.3cm]{/Users/kruppajo/work/GitHub/exam/avatare/Alex.png}
\end{minipage}
\begin{minipage}[t]{0.5\textwidth}
\hfill
\href{https://youtu.be/C9skfFRTHhI}{\includegraphics[width = 2cm]{img/youtube}}
\end{minipage}
\vspace{-3ex}
\fi
%% --------------------------------------------------------------------



\ifcollection
\paragraph{Interpretation der Ergebnisse einer Korrelationsanalyse in \Rlogo}
\fi

'Hä? Was ist denn das? Das wird ja immer wilder! Hatten wir als Aufgabe eine Korrelation zu berechnen? Wir bauen aus kontinuierlichen Daten eine Abbildung und interpretieren diese dann?', fragt sich Alex laut. 'Keine Ahnung... das ist jetzt jedenfalls keine Abbildung von irgendwas sondern eine \Rlogo Ausgabe mit ganz wilden Bezeichnungen...', denkt er. Aber das hilft jetzt auch nicht. Das Verrückte ist, dass die Katze Alien wirklich liebt. Das ist Alex sehr recht, denn er braucht Entspannung. Alex hatte ein Feldexperiment im Teuteburgerwald mit Maiss durchgeführt. Dabei wurden die beiden folgenden Variablen gemessen: durchschnittliche Regenwurmdichte [Anzahl/l] und Chlorophyllgehalt (SPAD-502Plus) [SPAD]. Jetzt hat er eigentlich alles zusammen. \textit{Eigentlich...}, denn mit der \Rlogo Ausgabe hat Alex jetzt ein Problem. Wenn die Gefälligkeit nicht wäre, ja dann wäre wohl vieles möglich für Alex! Aber so.. 


\begin{knitrout}
\definecolor{shadecolor}{rgb}{0.969, 0.969, 0.969}\color{fgcolor}\begin{kframe}
\begin{verbatim}
## 
## 	Pearson's correlation
## 
## data:  Durchschnittliche Regenwurmdichte and Chlorophyllgehalt
## t = 3.0716, df = 8, p-value = 0.01531
## alternative hypothesis: true correlation is not equal to 0
## 95 percent confidence interval:
##  0.1974517 0.9330783
## sample estimates:
##       cor 
## 0.7356246
\end{verbatim}
\end{kframe}
\end{knitrout}

Leider kennt sich Alex mit der Korrelationsanalyse in \Rlogo überhaupt nicht aus. Deshalb braucht er bei der Erstellung Ihre Hilfe!

\begin{enumerate}
  \item Formulieren Sie die wissenschaftliche Fragestellung! \textbf{(1 Punkt)}
  \item Formulieren Sie das statistische Hypothesenpaar! \textbf{(1 Punkt)}
\item Erstellen Sie eine Visualisierung für den Korrelationskoeffizienten! Beschriften Sie die Abbildung! \textbf{(2 Punkte)}
\item Nennen Sie die zwei Eigenschaften des Korrelationskoeffizienten! \textbf{(2 Punkte)}
\item Interpretieren Sie den Korrelationskoefizienten hinsichtlich des
  Effekts und der Signifikanz! Begründen Sie Ihre Antwort! \textbf{(2 Punkte)}
\item Visualisieren Sie das 95\% Konfidenzintervall! Beschriften Sie die Abbildung! \textbf{(2 Punkte)} 
\end{enumerate} 
\clearpage
% -----------------------------------------------------------------------

\section{Aufgabe \hfill (12 Punkte)}

\textit{Geben Sie grundsätzlich Formeln und Rechenweg zur Lösung der Teilaufgaben mit an!} \\[1Ex]
 

 
%% --------------------------------------------------------------------
\ifcollection
\begin{flushright}
\tiny\vspace{-3Ex}
\textbf{\examinhaltstart}
\exammodulestat $\;\bullet$
\exammodulestatbbv $\;\bullet$
\exammodulestatversuch $\;\bullet$
\exammodulebiostat
\vspace{-4Ex}
\end{flushright}
\begin{minipage}[t]{0.5\textwidth}
\includegraphics[width = 1.3cm]{/Users/kruppajo/work/GitHub/exam/avatare/Jonas.png}
\end{minipage}
\begin{minipage}[t]{0.5\textwidth}
\hfill
\href{https://youtu.be/g7imB3A5_zw}{\includegraphics[width = 2cm]{img/youtube}}
\end{minipage}
\vspace{-3ex}
\fi
%% --------------------------------------------------------------------



\ifcollection
\paragraph{Visualisierung der Korrelation und des Bestimmtheitsmaßes}
\fi

'Hm..., drei leere Abbildungen. Was soll ich da jetzt machen?', fragt sich Jonas. Jonas kennt sich nur begrenzt bis gar nicht mit der linearen Regresion und Korrelation aus. Dafür mit etwas anderem. Am Ende dann doch besser Stricken. Wunderbar. Eine echte Ablenkung für Jonas. Aber das hilft hier auch nur so halb, die Aufgabe zu lösen und mehr über den Korrelationskoeffizienten zu erfahren.
\vspace{2Ex}



{\centering \includegraphics[width=\maxwidth]{img/correlation-01-1} 

}




\vspace{2Ex}

Leider kennt sich Jonas mit der Korrelationsanalyse und der linearen Regression überhaupt nicht aus. Deshalb braucht er bei der Auswertung Ihre Hilfe!

\begin{enumerate}
\item Zeichnen Sie für die $\rho$-Werte eine Gerade in die entsprechende Abbildung! \textbf{(3 Punkte)}
\item Zeichnen Sie für die $R^2$-Werte die entsprechende Punktewolke um die Gerade! \textbf{(3 Punkte)}
\item Nennen Sie die zwei Eigenschaften des Korrelationskoeffizienten! \textbf{(2 Punkte)}
\item Interpretieren Sie die $R^2$-Werte für die jeweilige Gerade! \textbf{(2 Punkte)}
\item Warum müssen Sie ein $R^2$-Wert berechnen, wenn Sie die einfachere Möglichkeit der visuellen Überprüfung haben? Begründen Sie Ihre Antwort! \textbf{(2 Punkte)}
\end{enumerate}
 
\clearpage
% -----------------------------------------------------------------------

\section{Aufgabe \hfill (12 Punkte)}

\textit{Geben Sie grundsätzlich Formeln und Rechenweg zur Lösung der Teilaufgaben mit an!} \\[1Ex]
 

 
%% --------------------------------------------------------------------
\ifcollection
\begin{flushright}
\tiny\vspace{-3Ex}
\textbf{\examinhaltstart}
\exammodulestat $\;\bullet$
\exammodulestatbbv $\;\bullet$
\exammodulestatversuch $\;\bullet$
\exammodulebiostat
\vspace{-4Ex}
\end{flushright}
\begin{minipage}[t]{0.5\textwidth}
\includegraphics[width = 1.3cm]{/Users/kruppajo/work/GitHub/exam/avatare/Nilufar.png}
\end{minipage}
\begin{minipage}[t]{0.5\textwidth}
\hfill
\href{https://youtu.be/2QJa19ZwLls}{\includegraphics[width = 2cm]{img/youtube}}
\end{minipage}
\vspace{-3ex}
\fi
%% --------------------------------------------------------------------



\ifcollection
\paragraph{Schätzen der Korrelation und des Bestimmtheitsmaßes}
\fi

Heute hat leider Nilufar Lerngruppe keine Zeit, so dass dann Nilufar alles hier alleine machen muss. Das ist schon recht nervig. Nilufar versteht nicht viel von $\rho$-Werten oder gar $R^2$-Werten. Geschweige denn hat sie eine Idee, wie sie die Werte abschätzen sollte. Müde dreht sie Deichkind auf und lässt sich in Gedanken treiben. Gestern war sie viel zu lange auf.
\vspace{2Ex}



{\centering \includegraphics[width=\maxwidth]{img/correlation-02-1} 

}




Leider kennt sich Nilufar mit der Korrelationsanalyse und der linearen Regression überhaupt nicht aus. Deshalb braucht sie bei der Auswertung Ihre Hilfe!

\begin{enumerate}
\item Schätzen Sie die $\rho$-Werte in den Abbildungen! \textbf{(2 Punkte)}
\item Schätzen Sie die $R^2$-Werte in den Abbildungen! \textbf{(2 Punkte)}
\item Interpretieren Sie die $R^2$-Werte für die jeweilige Gerade! \textbf{(2 Punkte)}
\item Was ist der optimale $R^2$-Wert im Kontext einer wissenschaftlichen Fragestellung? Begründen Sie Ihre Antwort an einem Beispiel! \textbf{(2 Punkte)}
\item Was ist der optimale $\rho$-Wert im Kontext einer wissenschaftlichen Fragestellung? Begründen Sie Ihre Antwort an einem Beispiel! \textbf{(2 Punkte)}
\item Erklären Sie die Aussage \textit{"Correlation does not imply causation!"} an einem Beispiel! \textbf{(2 Punkte)}
\end{enumerate} 
\clearpage
% -----------------------------------------------------------------------

\section{Aufgabe \hfill (12 Punkte)}

\textit{Geben Sie grundsätzlich Formeln und Rechenweg zur Lösung der Teilaufgaben mit an!} \\[1Ex]
 

 
%% --------------------------------------------------------------------
\ifcollection
\begin{flushright}
\tiny\vspace{-3Ex}
\textbf{\examinhaltstart}
\exammodulebiostat
\vspace{-4Ex}
\end{flushright}
\begin{minipage}[t]{0.5\textwidth}
\includegraphics[width = 1.3cm]{/Users/kruppajo/work/GitHub/exam/avatare/Alex.png}\hspace{-4mm}\includegraphics[width = 1.3cm]{/Users/kruppajo/work/GitHub/exam/avatare/Jessica.png}
\end{minipage}
\begin{minipage}[t]{0.5\textwidth}
\hfill
\href{https://youtu.be/kHmfEmU6lrk}{\includegraphics[width = 2cm]{img/youtube}}
\end{minipage}
\fi
%% --------------------------------------------------------------------



\ifcollection
\paragraph{Visualisierung des Regressionskreuzes}
\fi

Jessica hat ein Kreuzungsexperiment mit Schweinen duchgeführt. Soweit so gut. Dann war sie bei ihrem Betreuer. Leider war der Schritt nicht so hilfreich.  Jessica und der Mangel, eine unendliche Geschichte mit kniffeligen Wendungen. Aber es muss ja weitergehen. Jessica hatte dann in ihrer Abschlusarbeit einfach zu viele Endpunkte gemessen und ist jetzt vollkommen durcheinander, welche Analyse sie nun wie rechnen soll. Naja, dann heißt es jetzt eben David Bowie aufdrehen und darüber nachdenken, was hier eigentlich gemacht wurde. Jessica fängt einfach an und nimmt den ersten Endpunkt Anzahl dunkle Pigmentstörungen. Dann kann sie sich voran arbeiten. Später dann noch raus um Rad zu fahren um mal zu entspannen und vielleicht ist Alex auch da. Wäre toll.

\vspace{1Ex}

Leider kennt sich Jessica mit dem Kontext der linearen Regression überhaupt nicht aus. Deshalb braucht sie bei der Auswertung Ihre Hilfe!

\begin{enumerate}
  \item Formulieren Sie die wissenschaftliche Fragestellung! \textbf{(1 Punkt)}
\item Zeichen Sie die Zeile des Regressionskreuzes für den Endpunkt mit \underline{drei} Feldern! Beschriften Sie die Abbildung! \textbf{(4 Punkte)}
\item Ergänzen Sie die entsprechenden statistische Methoden zur Analyse in jedem Feld! \textbf{(2 Punkte)}
\item Formulieren Sie die Nullhypothese für die statistische Methode in jedem Feld! \textbf{(2 Punkte)}
\item Ergänzen Sie die entsprechenden Funktionen in \Rlogo zur Analyse in jedem Feld! \textbf{(2 Punkte)}
\item Welchen Effekt erhalten Sie in jedem Feld? Geben Sie ein Beispiel! \textbf{(2 Punkte)}
\end{enumerate} 
\clearpage
% -----------------------------------------------------------------------

\section{Aufgabe \hfill (12 Punkte)}

\textit{Geben Sie grundsätzlich Formeln und Rechenweg zur Lösung der Teilaufgaben mit an!} \\[1Ex]
 

 
%% --------------------------------------------------------------------
\ifcollection
\begin{flushright}
\tiny\vspace{-3Ex}
\textbf{\examinhaltstart}
\exammodulelanddaten $\;\bullet$
\exammodulebiostat
\vspace{-4Ex}
\end{flushright}
\begin{minipage}[t]{0.5\textwidth}
\includegraphics[width = 1.3cm]{/Users/kruppajo/work/GitHub/exam/avatare/Steffen.png}
\end{minipage}
\begin{minipage}[t]{0.5\textwidth}
\hfill
\href{https://youtu.be/C9skfFRTHhI}{\includegraphics[width = 2cm]{img/caution}}
\end{minipage}
\vspace{-3ex}
\fi
%% --------------------------------------------------------------------



\ifcollection
\paragraph{Interpretation der Ergebnisse eines gemischten Modells in \Rlogo}
\fi

'Hä? Was ist denn das? Seid wann verfolgen mich Gummibären? Das wird ja immer wilder!', ruft Steffen laut. 'Keine Ahnung... was das jetzt wieder soll.', denkt er. Dabei war Steffen in tiefen Wäldern unterwegs um an Gummibären die Lesefähigkeit zu testen. Dafür hatte er von der Hochschule extra einen Fragebogen mit den notwendigen Informationen erhalten. Als Einflussvariable hat Steffen dann die Sprunghöhe bestimmt. Warum die Gummibären jetzt so aggressiv sind ist ihm auch nicht klar. Die Messungen wurden an drei Orten \textit{nearby}, \textit{midway} und \textit{far far away} pro Wald durchgeführt. Vielleicht hätte er nicht so viel Taylor Swift laut hören sollen.


\begin{knitrout}
\definecolor{shadecolor}{rgb}{0.969, 0.969, 0.969}\color{fgcolor}\begin{kframe}
\begin{verbatim}
## Linear mixed model fit by REML ['lmerMod']
## Formula: reading_score ~ jump_height + (1 | forest/direction)
##    Data: dragon_small_tbl
## 
## REML criterion at convergence: 2250.2
## 
## Scaled residuals: 
##      Min       1Q   Median       3Q      Max 
## -2.40552 -0.66015 -0.06551  0.65880  3.01072 
## 
## Random effects:
##  Groups           Name        Variance Std.Dev.
##  direction:forest (Intercept)  13.33    3.651  
##  forest           (Intercept) 267.61   16.359  
##  Residual                     220.86   14.861  
## Number of obs: 270, groups:  direction:forest, 18; forest, 6
## 
## Fixed effects:
##             Estimate Std. Error t value
## (Intercept) 25.40258   24.95601   1.018
## jump_height  0.08945    0.11894   0.752
## 
## Correlation of Fixed Effects:
##             (Intr)
## jump_height -0.962
\end{verbatim}
\end{kframe}
\end{knitrout}

Leider kennt sich Steffen mit der Analyse eines gemischten Modells in \Rlogo überhaupt nicht aus. Deshalb braucht er bei der Analyse Ihre Hilfe!

\begin{enumerate}
  \item Formulieren Sie die wissenschaftliche Fragestellung! \textbf{(1 Punkt)}
  \item Formulieren Sie das statistische Hypothesenpaar für den festen Effekt! \textbf{(1 Punkt)}
  \item Erklären Sie den Begriff \underline{fester Effekt} im Kontext des gemischten Modells! \textbf{(1 Punkt)}
  \item Erklären Sie den Begriff \underline{zufälliger Effekt} im Kontext des gemischten Modells! \textbf{(1 Punkt)}
  \item Skizzieren Sie das Versuchsdesign! \textit{Nutzen Sie eine reduzierte Auswahl an Faktorkombinationen!} \textbf{(1 Punkt)} 
  \item Visualisieren Sie die Ausgabe des linearen gemischten Modells! \textit{Nutzen Sie eine reduzierte Auswahl an Faktorkombinationen!} \textbf{(1 Punkt)} 
  \item Folgt der Messwert $y$ einer annährenden Normalverteilung? Begründen Sie Ihre Antwort! \textbf{(1 Punkte)}
\item Treffen Sie eine Aussage mit $T_{\alpha = 5\%} = 2.04$ zu Ihrer aufgestellten Nullhypothese! Begründen Sie Ihre Antwort \textbf{(2 Punkte)}
\item Interpretieren Sie den Effekt des festen Effekts im Kontext der wissenschaftlichen Fragestellung! \textbf{(1 Punkt)}
\item Wieviel Varianz erklärt der zufällige Haupteffekt? \textbf{(1 Punkt)}
\end{enumerate} 
\clearpage
% -----------------------------------------------------------------------
\part{Experimentelles Design}
% -----------------------------------------------------------------------

\section{Aufgabe \hfill (16 Punkte)}


 
%% --------------------------------------------------------------------
\ifcollection
\begin{flushright}
\tiny\vspace{-3Ex}
\textbf{\examinhaltstart}
\exammodulestatversuch $\;\bullet$
\exammodulebiostat
\vspace{-4Ex}
\end{flushright}
\begin{minipage}[t]{0.5\textwidth}
\includegraphics[width = 1.3cm]{/Users/kruppajo/work/GitHub/exam/avatare/Jessica.png}\hspace{-4mm}\includegraphics[width = 1.3cm]{/Users/kruppajo/work/GitHub/exam/avatare/Mark.png}\hspace{-4mm}\includegraphics[width = 1.3cm]{/Users/kruppajo/work/GitHub/exam/avatare/Nilufar.png}
\end{minipage}
\begin{minipage}[t]{0.5\textwidth}
\hfill
\href{https://youtu.be/wJqsNV1hOW8}{\includegraphics[width = 2cm]{img/caution}}
\end{minipage}
\fi
%% --------------------------------------------------------------------



\ifcollection
\paragraph{Einfache experimentelle Designs}
\fi

Nilufar und Mark sind bei Jessica um sich Hilfe für eine Versuchsplanung in \Rlogo zu holen. Im Hintergrund läuft viel zu laut David Bowie. Dabei geht es um den Zusammenhang zwischen der Behandlung Lichtstufen ($none$, $200lm$, $400lm$ und $600lm$) und dem Messwert Frischegewicht [kg/ha] in Erbsen. Der Versuch soll in einem Freilandversuch im Oldenburger Land durchgeführt werden. Nach der Dozentin ist der Messwert Frischegewicht [kg/ha] normalverteilt. Die beiden entschieden sich für ein faktorielles Versuchsdesign. Im ersten Schritt überlegt Jessica ein einfaches experimentelles Design zu probieren. Daher entscheiden sich alle drei für ein \textit{Complete randomized design (CRD) mit drei Wiederholungen}. Das sollte für den Anfang erstmal reichen. 'Und jetzt, was machen wir jetzt?', Mark schaut die anderen beiden mit großen Augen an. Die zucken mit der Schulter. Alle mampfen Schokobons.\\

Leider kennen sich Jessica, Nilufar und Mark mit dem \textit{Complete randomized design (CRD) mit drei Wiederholungen} überhaupt nicht aus. Deshalb brauchen die Drei bei der Erstellung Ihre Hilfe!

\begin{enumerate}
  \setcounter{enumi}{0}
  \item Formulieren Sie die wissenschaftliche Fragestellung! \textbf{(1 Punkt)}
  \item Formulieren Sie das statistische Hypothesenpaar! \textbf{(1 Punkt)}
  \item Skizzieren Sie das faktorielle Versuchsdesign! \textbf{(3 Punkte)}
  \item Skizzieren Sie eine Datentabelle für das faktorielle Versuchsdesign in \Rlogo! \textbf{(2 Punkte)}
  \item Erstellen Sie das statistische Modell in der in \Rlogo üblichen Schreibweise für eine ANOVA! Skizzieren Sie die notwendige Funktionen in \Rlogo! \textbf{(3 Punkte)}
  \item Skizzieren Sie die weitere Datenanalyse hinsichtlich eines multiplen Gruppenvergleiches in \texttt{\{emmeans\}}! \textbf{(2 Punkte)}
  \item Skizzieren Sie eine mögliche Abbildung im Kontext der wissenschaftlichen Fragestellung! Beschriften Sie die Abbildung! \textbf{(2 Punkte)}
  \item Ergänzen Sie zu der Abbildung ein mögliches Ergebnis des multiplen Gruppenvergleichs! Begründen Sie Ihre Antwort! \textbf{(2 Punkte)}
\end{enumerate}


 
\clearpage
% -----------------------------------------------------------------------

\section{Aufgabe \hfill (20 Punkte)}


 
%% --------------------------------------------------------------------
\ifcollection
\begin{flushright}
\tiny\vspace{-3Ex}
\textbf{\examinhaltstart}
\exammodulebiostat
\vspace{-4Ex}
\end{flushright}
\begin{minipage}[t]{0.5\textwidth}
\includegraphics[width = 1.3cm]{/Users/kruppajo/work/GitHub/exam/avatare/Jessica.png}\hspace{-4mm}\includegraphics[width = 1.3cm]{/Users/kruppajo/work/GitHub/exam/avatare/Tina.png}\hspace{-4mm}\includegraphics[width = 1.3cm]{/Users/kruppajo/work/GitHub/exam/avatare/Yuki.png}
\end{minipage}
\begin{minipage}[t]{0.5\textwidth}
\hfill
\href{https://youtu.be/wJqsNV1hOW8}{\includegraphics[width = 2cm]{img/caution}}
\end{minipage}
\fi
%% --------------------------------------------------------------------



\ifcollection
\paragraph{Fortgeschrittene experimentelle Designs}
\fi

Neuer Versuch neues Glück! Yuki und Tina sind bei Jessica um sich Hilfe für eine Versuchsplanung in \Rlogo zu holen. Im Hintergrund läuft viel zu laut Herr der Ringe. Daher hat die Hündin schon lange reißaus genommen. In dem neuen Versuch geht es um den Zusammenhang zwischen der Behandlung Flüssignahrung ($ctrl$, $superIn$ und $flOw$) sowie Ernährungszusatz ($ctrl$ und $getIt$) sowie vier Blöcken und dem Messwert Fettgehalt [\%/kg] in Lamas. Der Versuch soll in einem Kreuzungsexperiment im Emsland durchgeführt werden. Immerhin ist der Messswert normalverteilt, was einges einfacher macht. Was es nicht so einfacher macht ist, dass Yuki noch als zusätzliche Herausforderung etwas anderes umtreibt: die Faulheit. Im ersten Schritt überlegt Jessica ein komplexeres experimentelles Design zu probieren. Daher entscheiden sich alle drei für ein \textit{Strip plot design oder auch Streifenanlage}. Ob es das jetzt einfacher macht?\\

Leider kennen sich Jessica, Yuki und Tina mit dem \textit{Strip plot design oder auch Streifenanlage} überhaupt nicht aus. Deshalb brauchen die Drei bei der Erstellung Ihre Hilfe!

\begin{enumerate}
  \setcounter{enumi}{0}
  \item Formulieren Sie die wissenschaftliche Fragestellung! \textbf{(1 Punkt)}
  \item Formulieren Sie die statistische Hypothesenpaare! \textbf{(2 Punkte)}
  \item Skizzieren Sie das faktorielle Versuchsdesign! \textbf{(3 Punkte)}
  \item Skizzieren Sie eine Datentabelle für das faktorielle Versuchsdesign in \Rlogo! \textbf{(2 Punkte)}
  \item Erstellen Sie das statistische Modell in der in \Rlogo üblichen Schreibweise für eine ANOVA! Skizzieren Sie die notwendige Funktionen in \Rlogo! \textbf{(4 Punkte)}
  \item Skizzieren Sie die weitere Datenanalyse hinsichtlich eines multiplen Gruppenvergleiches in \texttt{\{emmeans\}}! \textbf{(2 Punkte)}
  \item Skizzieren Sie eine mögliche Abbildung im Kontext der wissenschaftlichen Fragestellung! Beschriften Sie die Abbildung! \textbf{(3 Punkte)}
  \item Ergänzen Sie zu der Abbildung ein mögliches Ergebnis des multiplen Gruppenvergleichs! Welche Annahme hinsichtlich der Modellierung haben Sie getroffen? Begründen Sie Ihre Antwort! \textbf{(3 Punkte)}
\end{enumerate} 
\clearpage
% -----------------------------------------------------------------------
\part{Forschendes Lernen}
% -----------------------------------------------------------------------

\section{Aufgabe \hfill (20 Punkte)}

\textit{Geben Sie grundsätzlich Formeln und Rechenweg zur Lösung der Teilaufgaben mit an!} \\[1Ex]
 

 
%% --------------------------------------------------------------------
\ifcollection
\begin{flushright}
\tiny\vspace{-3Ex}
\textbf{\examinhaltstart}
\exammodulebiostat
\vspace{-4Ex}
\end{flushright}
\begin{minipage}[t]{0.5\textwidth}
\includegraphics[width = 1.3cm]{/Users/kruppajo/work/GitHub/exam/avatare/Nilufar.png}
\end{minipage}
\begin{minipage}[t]{0.5\textwidth}
\hfill
\href{https://youtu.be/C9skfFRTHhI}{\includegraphics[width = 2cm]{img/caution}}
\end{minipage}
\fi
%% --------------------------------------------------------------------



\ifcollection

\begin{graybox}{}
\small
Die folgende Aufgabe basiert auf einer der zwei folgenden wissenschaftlichen Veröffentlichungen. Für die Prüfung wird die vertiefende Kenntnis der beiden Veröffentlichungen vorausgesetzt.

\begin{itemize}[noitemsep]
\item Sánchez, M., et al. (2022). Hoverfly pollination enhances yield and fruit quality in mango under protected cultivation. Scientia Horticulturae, 304, 111320. [\href{https://www.sciencedirect.com/science/article/pii/S0304423822004411}{Link}]
\item Wu, G., et al. (2004). Arginine nutrition in neonatal pigs. The Journal of Nutrition, 134(10), 2783S-2790S. [\href{https://www.sciencedirect.com/science/article/pii/S0022316623031279}{Link}]
\end{itemize}

In der Prüfung erhalten Sie \underline{keinen Auszug} der wissenschaftlichen Veröffentlichung! Die Veröffentlichungen werden als \underline{bekannt} in der Prüfung vorgesetzt. Sie haben sich vorab Notizen und Anmerkungen auf Ihrem Spickzettel gemacht.
\end{graybox}

\paragraph{Zerforschen einer wissenschaftlichen Veröffentlichung}
\fi

Vor dem Start der eigenen Arbeit möchte ihre Betreuerin, dass Nilufar einmal die wissenschaftliche Veröffentlichung \textit{Sánchez, M., et al. (2022). Hoverfly pollination enhances yield and fruit quality in mango under protected cultivation} sinnvoll zusammenfasst. Dann würde die eigene Arbeit auch leichter von der Hand gehen und Nilufar hätte dann schon eine Vorlage. 'Das ist jetzt aber umfangreicher als gedacht!', schnauft sie und runzelt die Stirn. Im Hintergrund spielt viel zu leise Deichkind. Das wird dann vermutlich heute Abend nichts mehr mit ihrem Hobby Hip Hop. Das Huhn schaut mitleidig.\\

Leider kennt sich Nilufar mit dem Lesen einer wissenschaftlichen Veröffentlichung mit Fokus auf die Statistik überhaupt nicht aus. Deshalb braucht sie bei der Erstellung Ihre Hilfe! Glücklicherweise kennen Sie die wissenschaftliche Veröffentlichung schon im Detail und können sofort helfen.

\begin{enumerate}
  \setcounter{enumi}{0}
  \item Erläutern Sie die wissenschaftliche Fragestellung der wissenschaftlichen Veröffentlichung anhand des OCAR Prinzips nach Schimel (2012)\footnote{Schimel, J. (2012). Writing science: how to write papers that get cited and proposals that get funded. OUP USA.} \textbf{(4 Punkte)}
  \item Nennen Sie die untersuchten Endpunkte in der wissenschaftlichen Veröffentlichung! Wie lautet der primäre Endpunkt? \textbf{(2 Punkte)} 
\item Erstellen Sie das statistische Modell in der in \Rlogo üblichen Schreibweise! \textbf{(2 Punkte)}
  \item Nennen Sie eine Auswahl an bedeutenden statistischen Maßzahlen in der wissenschaftlichen Veröffentlichung! \textbf{(1 Punkt)}
  \item Interpretieren Sie die Hauptaussage der wissenschaftlichen Veröffentlichung hinsichtlich der Signifikanz für den primären Endpunkt! \textbf{(2 Punkte)}
  \item Interpretieren Sie die Hauptaussage der wissenschaftlichen Veröffentlichung hinsichtlich der Effektstärke für den primären Endpunkt! \textbf{(2 Punkte)}
  \item Diskutieren Sie die ökonomische Relevanz der Hauptaussage der wissenschaftlichen Veröffentlichung im Bezug auf Signifikanz und Effektstärke für den primären Endpunkt! \textbf{(1 Punkt)}
  \item Skizzieren Sie für den primären Endpunkt den sich ergebenden Datensatz in \Rlogo für eine ausgewählte Abbildung! \textbf{(2 Punkte)}
\item Skizzieren Sie einen möglichen Versuchsplan für den primären Endpunkt! \textbf{(2 Punkte)}
  \item Schätzen Sie die benötigte Fallzahl für ein zukünftiges Experiment anhand der Ergebnisse in der wissenschaftlichen Veröffentlichung für den primären Endpunkt! \textbf{(2 Punkte)}
\end{enumerate} 
\clearpage
% -----------------------------------------------------------------------

\section{Aufgabe \hfill (20 Punkte)}

\textit{Geben Sie grundsätzlich Formeln und Rechenweg zur Lösung der Teilaufgaben mit an!} \\[1Ex]
 

 
%% --------------------------------------------------------------------
\ifcollection
\begin{flushright}
\tiny\vspace{-3Ex}
\textbf{\examinhaltstart}
\exammodulebiostat
\vspace{-4Ex}
\end{flushright}
\begin{minipage}[t]{0.5\textwidth}
\includegraphics[width = 1.3cm]{/Users/kruppajo/work/GitHub/exam/avatare/Steffen.png}
\end{minipage}
\begin{minipage}[t]{0.5\textwidth}
\hfill
\href{https://youtu.be/0upJd61FI-c}{\includegraphics[width = 2cm]{img/youtube}}
\end{minipage}
\fi
%% --------------------------------------------------------------------



\ifcollection
\paragraph{Zerforschen eines wissenschaftlichen Datensatzes}
\fi

Vor dem Start der eigenen Arbeit möchte sein Betreuer, dass Steffen einmal die wissenschaftlichen Daten \textit{in der Tabelle 1} sinnvoll zusammenfasst. Insgesamt wurden $n$ Beobachtungen erhoben. Dann würde die eigene Arbeit auch leichter von der Hand gehen und Steffen hätte dann schon eine Vorlage um die eigenen erhobenen Daten in eine Tabelle eintragen zu können. 'Das ist jetzt aber umfangreicher als gedacht!', schnauft er und runzelt die Stirn als er in seinen Laptop starrt. Dabei isst er noch ein paar Oreos. Das wird dann vermutlich heute Abend nichts mehr mit Harry Potter.

\begin{table}[h]
\centering
\Large
  \begin{tabular}{ccccc}
  \toprule
   $\boldsymbol{f_1}$  & $\boldsymbol{f_2}$ & $\boldsymbol{x_1}$ & $\boldsymbol{y_1}$ & $\boldsymbol{y_2}$ \\[2pt]
     <\phantom{xxxxx}>  & <\phantom{xxxxx}> & <\phantom{xxxxx}> & <\phantom{xxxxx}> & <\phantom{xxxxx}> \\[2pt] 
  \midrule
  1  & 1 & 2.3 & 10.1 & 0 \\  
  1  & 1 & 4.1 & 13.1 & 0 \\ 
  1  & 1 & 5.7 & 16.5 & 1 \\ 
  1  & 1 & 3.4 & 14.6 & 0 \\
  1  & 2 & 2.8 & 12.1 & 1 \\
  1  & 2 & 6.1 & 13.4 & 1 \\
  $\vdots$  & $\vdots$ & $\vdots$ & $\vdots$ & $\vdots$\\
  5 & 2  & 1.9 & 9.6 &  0\\
  \bottomrule
  \end{tabular}
\end{table}

Steffen füllt sich mit der Analyse der Daten in der Tabelle 1 überfordert. Deshalb braucht er bei der Auswertung Ihre Hilfe! Glücklicherweise kennen Sie den wissenschaftlichen Datensatz aus Ihren eigenen Analysen schon im Detail und können sofort helfen.

\begin{graybox}{}
\textit{Beantworten Sie die folgenden Fragen anhand eines selbst gewählten Beispiels!}
\end{graybox}

\paragraph{Allgemeiner Aufgabenteil} 

\begin{enumerate}
  \setcounter{enumi}{0}
  \item Ergänzen Sie die Eigenschaften der Spalten in der Form eines \texttt{tibbles}! \textbf{(2 Punkte)}
  \item Skizzieren Sie zwei übergeordnete Analysebereiche der Statistik! \textit{Nutzen Sie hierfür die Variablennamen der obigen Datentabelle.} Beschriften Sie die Abbildungen! \textbf{(4 Punkte)}
  \item Formulieren Sie zwei mögliche wissenschaftliche Fragestellungen in Form einer PowerPoint Folie aus der obigen Datentabelle! \textbf{(2 Punkte)}
\end{enumerate}

\paragraph{Spezieller Aufgabenteil für die Variablen $\boldsymbol{f_1}$, $\boldsymbol{f_2}$ und $\boldsymbol{y_1}$}

\begin{enumerate}
  \setcounter{enumi}{3}
  \item In welchen der übergeordneten Analysebereiche der Statistik gehört die Auswertung Ihres Endpunktes? Begründen Sie Ihre Antwort! \textbf{(2 Punkte)}
  \item Skizzieren Sie eine beispielhafte Abbildung für Ihren Endpunkt im Kontext der wissenschaftlichen Fragestellung! \textbf{(2 Punkte)}
  \item Erstellen Sie das statistische Modell in der in \Rlogo üblichen Schreibweise! \textbf{(1 Punkt)}
  \item Skizzieren Sie die Datenanalyse für Ihren Endpunkt! \textbf{(4 Punkte)}
  \item Auf welche Eigenschaften der Daten müssen Sie für Ihre statistische Analyse im Besonderen achten? \textbf{(2 Punkte)}
  \item Welche statistische Maßzahl können Sie aus Ihrer Datenanalyse berichten? \textbf{(1 Punkt)}
\end{enumerate}

 
\clearpage
% -----------------------------------------------------------------------
\part{Mathematik}
% -----------------------------------------------------------------------  

\section{Aufgabe \hfill (10 Punkte)}

\textit{Geben Sie grundsätzlich Formeln und Rechenweg zur Lösung der Teilaufgaben mit an!} \\[1Ex]
 

 
%% --------------------------------------------------------------------
\ifcollection
\begin{flushright}
\tiny\vspace{-3Ex}
\textbf{\examinhaltstart}
\exammodulemathstat
\vspace{-4Ex}
\end{flushright}
\begin{minipage}[t]{0.5\textwidth}
\includegraphics[width = 1.3cm]{/Users/kruppajo/work/GitHub/exam/avatare/Mark.png}\hspace{-4mm}\includegraphics[width = 1.3cm]{/Users/kruppajo/work/GitHub/exam/avatare/Nilufar.png}\hspace{-4mm}\includegraphics[width = 1.3cm]{/Users/kruppajo/work/GitHub/exam/avatare/Steffen.png}\hspace{-4mm}\includegraphics[width = 1.3cm]{/Users/kruppajo/work/GitHub/exam/avatare/Tina.png}
\end{minipage}
\begin{minipage}[t]{0.5\textwidth}
\hfill
\href{https://youtu.be/Fu8kN0Uj13Y}{\includegraphics[width = 2cm]{img/youtube}}
\end{minipage}
{\tiny\textbf{Stichworte:} The Great Horse-Manure Crisis of 1894\footnote{Weitere Informationen: \href{https://fee.org/articles/the-great-horse-manure-crisis-of-1894/}{The Great Horse-Manure Crisis of 1894} und \href{https://www.gregorystrachta.com/18.html}{The Great Horse Manure Crisis of 1894 was never mentioned before 2004}} $\bullet$ Napoleon $\bullet$ Ivenacker Eichen}
\fi
%% --------------------------------------------------------------------





\ifcollection
\paragraph{Herodot – der Schimmel aus Ivenack}
\fi

Die Lerngruppe \textit{Die Pantoffeltieren} bestehend aus Steffen, Nilufar, Tina und Mark waren auf Exkursion in Brandenburg und haben dort Folgendes erarbeitet. Während der Besetzung Mecklenburgs durch die Franzosen kamen Napoleon die Geschichten des berühmten Apfelschimmels Herodot aus Ivenack zu Gehör. Herodot lief zwar niemals Rennen, war aber eines der berühmtesten Pferde dieser Zeit. Napoleon selbst gab den Auftrag, diesen Schimmel durch die Armee nach Frankreich zu bringen. Der Legende nach sollen Arbeiter den Schimmel im hohlen Stamm einer 1000-jährigen Eiche aus Ivenack vor den Franzosen versteckt haben. Doch Herodot verriet sein Versteck durch lautes Wiehern, woraufhin die französische Armee den Schimmel beschlagnahmte und nach Frankreich führte\footnote{Die Quelle der Inspiration  für die Aufgabe war eine Fahrt an die Ostsee und folgender Artikel:
  \href{https://www.wald-mv.de/landingpage/ivenacker-eichen/}{Entdecke das erste Nationale Naturmonument Deutschlands - Ivenacker Eichen und Hutewald}}. Jetzt wollen die vier herausfinden: \textit{"Konnten die Ivenacker den Apfelschimmel Herodot vor dem Zugriff von Napoleon in der 1000-jährigen Eiche verstecken?"} 



\vspace{1Ex}

Helfen Sie der Lerngruppe \textit{Die Pantoffeltieren} bei der Beantwortung der Forschungsfrage! Gehen Sie von einem radialen Wachstum der 1000-jährigen Eiche von $2.25mm$ pro Jahr aus. Es ist bekannt, dass die Eiche im Jahr 2022 einen Umfang von $12m$ in Brusthöhe hatte.

\begin{enumerate}
\item Wie groß war der Durchmesser in $m$ der Eiche im Jahr $1820$ als Herodot in der Eiche versteckt werden sollte? \textbf{(2 Punkte)}
\item Erstellen Sie die lineare Gradengleichung für das radiale Wachstum! \textbf{(1 Punkt)}
\item Skizzieren Sie in einer Abbildung einen linearen Zusammenhang und einen exponentiellen Zusammenhang für das Wachstum der 1000-jährigen Eiche! \textbf{(1 Punkt)}
\end{enumerate}
 
Herodot hatte eine Schulterhöhe von $180$cm, eine Breite von $95$cm sowie eine Länge von  $230$cm.

\begin{enumerate}
  \setcounter{enumi}{2}
\item Berechnen Sie das effektive Volumen von Herodot in $m^3$, welches Herodot in der 1000-jährigen Eiche einnehmen würde! \textbf{(2 Punkte)}
\end{enumerate}

Es wurde berichtet, dass sich Herodot in der 1000-jährigen Eiche $sehr bequem mit Abstand$ um die eigene Achse drehen konnte.

\begin{enumerate}
  \setcounter{enumi}{3}
\item Berechnen Sie die Dicke der Eichenwand in $cm$! Verdeutlichen Sie Ihre Berechnungen an einer aussagekräftigen Skizze für Pferd und Eiche! \textbf{(2 Punkte)} 
\item Unter einer Dicke der Eichenwand von $20cm$ bricht die Eiche zusammen. Beantworten Sie die Forschungsfrage! Begründen Sie Ihre Antwort! \textbf{(2 Punkte)} 
\end{enumerate}


 
\clearpage
% ----------------------------------------------------------------------- 

\section{Aufgabe \hfill (10 Punkte)}

\textit{Geben Sie grundsätzlich Formeln und Rechenweg zur Lösung der Teilaufgaben mit an!} \\[1Ex]
 

 
%% --------------------------------------------------------------------
\ifcollection
\begin{flushright}
\tiny\vspace{-3Ex}
\textbf{\examinhaltstart}
\exammodulemathstat
\vspace{-4Ex}
\end{flushright}
\begin{minipage}[t]{0.5\textwidth}
\includegraphics[width = 1.3cm]{/Users/kruppajo/work/GitHub/exam/avatare/Jonas.png}\hspace{-4mm}\includegraphics[width = 1.3cm]{/Users/kruppajo/work/GitHub/exam/avatare/Mark.png}\hspace{-4mm}\includegraphics[width = 1.3cm]{/Users/kruppajo/work/GitHub/exam/avatare/Steffen.png}\hspace{-4mm}\includegraphics[width = 1.3cm]{/Users/kruppajo/work/GitHub/exam/avatare/Yuki.png}
\end{minipage}
\begin{minipage}[t]{0.5\textwidth}
\hfill
\href{https://youtu.be/57B-yYoFSk0}{\includegraphics[width = 2cm]{img/youtube}}
\end{minipage}
\fi
%% --------------------------------------------------------------------



\ifcollection
\paragraph{Von Töpfen auf Tischen}
\fi



Die Projektgruppe \textit{A} bestehend aus Steffen, Jonas, Mark und Yuki hat sich zusammengefunden um den ersten Versuch zu planen. In einem Experiment wollen sie die Wuchshöhe von 180 Maispflanzen bestimmen. Bevor die Vier überhaupt mit dem Experiment beginnen können, gibt es aber ein paar Abschätzungen über die Kosten und den Aufwand zu treffen. Zum einen müssen sie die Maispflanzen einpflanzen und müssen dafür Substrat bestellen. Zum anderen muss die Projektgruppe die Maispflanzen auch bewegen und in ein Gewächshaus auf rechteckigen Tischen platzieren. Die schmale Tischseite fast ohne Randpflanzen 5 Pflanzen. Die Töpfe für die Keimung haben
einen Durchmesser von 8cm und eine Höhe von 8cm. Der Kubikmeterpreis für Torf liegt bei 310 EUR. 

\vspace{1Ex}

Helfen Sie der Projektgruppe \textit{A} bei der Planung des Versuches! 

\begin{enumerate}
\item Skizzieren Sie den Versuchsplan auf \textit{drei} Tischen im Gewächshaus! \textit{Skizzieren Sie einen Tisch detailierter und die restlichen Tische schematisch!} \textbf{(2 Punkte)}
\item Berechnen Sie die benötigte Anzahl an Pflanztöpfen, wenn Sie Randpflanzen mit berücksichtigen wollen! \textbf{(1 Punkt)}
\item Berechnen Sie die benötigte \textbf{(a)} \underline{Pflanztopf}fläche in $m^2$ sowie die \textbf{(b)} \underline{Tisch}fläche in $m^2$ \textit{für einen Tisch} gegeben der Anzahl an Pflanztöpfen inklusive Randpflanzen am Anfang der Keimungsphase! \textbf{(4 Punkte)}
\item Berechnen Sie die benötigte Menge an Torf in Liter $l$, die Sie für das Befüllen der Pflanztöpfe benötigen! Gehen Sie von \textit{einem Zylinder} für die Pflanztöpfe aus!  \textbf{(2 Punkte)}
\item Berechnen Sie die Kosten in EUR für Ihre Torfbestellung! \textbf{(1 Punkt)}
\end{enumerate}




 
\clearpage
% ----------------------------------------------------------------------- 

\section{Aufgabe \hfill (10 Punkte)}

\textit{Geben Sie grundsätzlich Formeln und Rechenweg zur Lösung der Teilaufgaben mit an!} \\[1Ex]
 

 
%% --------------------------------------------------------------------
\ifcollection
\begin{flushright}
\tiny\vspace{-3Ex}
\textbf{\examinhaltstart}
\exammodulemathstat
\vspace{-4Ex}
\end{flushright}
\begin{minipage}[t]{0.5\textwidth}
\includegraphics[width = 1.3cm]{/Users/kruppajo/work/GitHub/exam/avatare/Alex.png}\hspace{-4mm}\includegraphics[width = 1.3cm]{/Users/kruppajo/work/GitHub/exam/avatare/Jonas.png}\hspace{-4mm}\includegraphics[width = 1.3cm]{/Users/kruppajo/work/GitHub/exam/avatare/Mark.png}\hspace{-4mm}\includegraphics[width = 1.3cm]{/Users/kruppajo/work/GitHub/exam/avatare/Nilufar.png}
\end{minipage}
\begin{minipage}[t]{0.5\textwidth}
\hfill
\href{https://youtu.be/aBxLkdF-c4M}{\includegraphics[width = 2cm]{img/youtube}}
\end{minipage}
\fi
%% --------------------------------------------------------------------




\ifcollection
\paragraph{Solar- \& Biogasanlagen}
\fi



Mark bringt ein neues, tolles Projekt mit in die Lerngruppe \textit{Die Kühe auf dem Deich} bestehend aus ihm, Jonas, Nilufar sowie Alex. Um die Energiekosten seines Betriebes zu senken, will er eine Solaranlage auf den Hühnerstall montieren lassen. Dafür hat er seinen Stall ausgemessen und findet folgende Maße wieder. Die vordere Seite des Hühnerstall hat eine Höhe $h_v$ von $6.5m$. Die hintere Seite des Hühnerstall hat eine Höhe $h_b$ von $8m$. Der Hühnerstall hat eine Tiefe $t$ von $12m$ und eine Breite $b$ von $40m$. 'Sag mal Mark, ist das eine Matheaufgabe oder rechnen wir hier gerade für dich kostenlos als menschliche Computer Sachen für deinen Betrieb?', fragt Nilufar mit erhobenenen Augenbrauen. Alex und Jonas nicken zustimmend.

\vspace{1Ex}

Wenn die Lerngruppe nicht will, dann müssen Sie bei der Planung helfen!

\begin{enumerate}
\item Skizzieren Sie den Hühnerstall auf dem die Solaranlage montiert werden soll! Ergänzen Sie die Angaben für die Höhen $h_v$, $h_b$, die Tiefe $t$ und die Breite $b$ des Stalls!  \textbf{(2 Punkte)}
\item Berechnen Sie die Fläche der schrägen, neuen Solaranlage auf dem Hühnerstall! \textbf{(3 Punkte)}
\end{enumerate}

Ebenfalls plant Mark eine neue Biogasanlage für seinen Betrieb. Der neue Methantank hat einen Radius $r$ von $1.8m$. Leider gibt es ein paar bauliche Beschränkungen auf dem Grundstück. Das Fundament des zylindrischen Methantanks kann nur ein Gewicht von maximal $14t$ aushalten bevor der Tank wegbricht. Mark rechnen eine Sicherheitstoleranz von $15\%$ ein beinhaltend das Gewicht des Methantanks. In flüssiger Form hat Methan bei $-80^\circ\text{C}$ eine Dichte von $220kg/m^3$. Bei $-100^\circ\text{C}$ hat Methan eine Dichte von $280kg/m^3$. Mark betreibt seine Anlage bei $-88^\circ\text{C}$.

\begin{enumerate}
  \setcounter{enumi}{2}
\item Extrapolieren Sie die effektive Dichte des Methans in dem Methantank! Welche Annahme haben Sie getroffen? \textbf{(1 Punkt)}
\item Berechnen Sie wie viel Kubikmeter $m^3$ in den Methantank gefüllen werden können, bevor das Fundament nachgibt! \textbf{(2 Punkte)}
\item Berechnen Sie die maximale Höhe $h_{max}$ in $m$ für den gefüllten Methantank mit dem Radius $r$, bevor das Fundament wegbricht! \textbf{(2 Punkte)}
\end{enumerate}

 
\clearpage
% -----------------------------------------------------------------------

\section{Aufgabe \hfill (10 Punkte)}

\textit{Geben Sie grundsätzlich Formeln und Rechenweg zur Lösung der Teilaufgaben mit an!} \\[1Ex]
 

 
%% --------------------------------------------------------------------
\ifcollection
\begin{flushright}
\tiny\vspace{-3Ex}
\textbf{\examinhaltstart}
\exammodulemathstat
\vspace{-4Ex}
\end{flushright}
\begin{minipage}[t]{0.5\textwidth}
\includegraphics[width = 1.3cm]{/Users/kruppajo/work/GitHub/exam/avatare/Mark.png}
\end{minipage}
\begin{minipage}[t]{0.5\textwidth}
\hfill
\href{https://youtu.be/https://youtu.be/k2G52hMIfqk}{\includegraphics[width = 2cm]{img/youtube}}
\end{minipage}
{\tiny\textbf{Stichworte:} Riesenfaultier $\bullet$ Evolution der Avocado $\bullet$ Bluetooth $ \bullet$ Blauzahn $\bullet$ Colonia Dignidad $\bullet$ ODESSA $\bullet$ Rattenlinie $\bullet$ Adolf Eichmann}
\fi
%% --------------------------------------------------------------------



\ifcollection
\paragraph{Aligatorenbirnen und Blaubeeren}
\fi



"'Sind Sie ein Riesenfautier oder warum kaufen Sie so viele Aligatorenbirnen?"', spricht es hinter Ihnen. Irritiert drehen Sie sich um und blicken in das puderrote Gesicht von Mark. "'Wieso?"', entfährt es Ihnen und Sie bereuen sogleich die Frage. Sofort werden Sie zu einem Whiteboard voller roter Schnüre geschliffen und müssen folgenden mathematischen untermauerten Argumenten im Netto über sich ergehen lassen. Da kommen Sie nicht mehr raus, also können Sie auch gleich mitmachen. Das Problem liegt in Chile\footnote{Die Quelle der Inspiration für die Aufgabe waren folgende Reportagen: \href{https://www.amnesty.ch/de/ueber-amnesty/publikationen/magazin-amnesty/2021-3/bis-zum-letzten-tropfen}{"`Bis zum letzten Tropfen"' in AMNESTY – Magazin der Menschenrechte vom August 2021} und \href{https://www.welthungerhilfe.de/welternaehrung/rubriken/klima-ressourcen/wassernot-in-chile-eine-folge-der-privatisierung}{"`Wasserknappheit in Chile: Eine Folge der Privatisierung?"' in Die Welternährung dem Fachjournal der Welthungerhilfe vom April 2022.}}. Tja, die Deutschen und Südamerika.\\

Zuerst werden Ihre Fähigkeiten getestet, der Mathematik folgen zu können. Oder berechnen Sie gerade den Einkauf von Mark?\\

\begin{enumerate}
\item Wenn 6 Blaubeerschalen 11.94 Euro kosten,  wie viel kosten 9 Schalen? \textbf{(2 Punkte)}
\item Wenn Sie die 9 Blaubeerschalen gekauft haben, wie viele Aligatorbirnen zu je 2.89 EUR können Sie sich dann noch für 50 EUR leisten? \textbf{(1 Punkt)}
\end{enumerate}

Das Whiteboard beinhaltet folgende Liste mit Informationen zum Wasserverbrauch bei der Produktion von Gemüse aus Chile. Seltsam, was man so alles in einem Netto über Gemüse erfährt.
  
\begin{itemize}[noitemsep]
\item Ein Kilo Strauchtomaten benötigt 170l Wasser. Eine Strauchtomate wiegt 90 - 115g.
\item Ein Kilo Salat benötigt 100l Wasser. Ein Salatkopf wiegt 310 - 520g.
\item Ein Kilo Avocado benötigt 1100l Wasser. Eine Avocado wiegt 130 - 410g.
\item Ein Kilo Blaubeeren benötigt 820l Wasser. Eine Blaubeere wiegt 3 - 3.8g.
\end{itemize}

\begin{enumerate}
  \setcounter{enumi}{2}
\item Berechnen Sie den Wasserverbrauch für die Produktion für jeweils eine Strauchtomate, einem Salat, einer Avocado und einer Blaubeeren. Stellen Sie das Ergebnis als Tabelle dar! \textbf{(3 Punkte)}
\end{enumerate}

Chile exportiert im großem Ausmaß Blaubeeren und Avocados. In dem Exportjahr 2023 blieben die Erträge von Blaubeeren mit \ensuremath{7.5\times 10^{4}}t in dem prognostizierten Rahmen. Die Menge \underline{steigerte} sich um 5.1\%. Die Exporte für Avocados \underline{fielen} in dem gleichen Zeitraum um 17.3\% auf \ensuremath{2.3\times 10^{5}}t.

\begin{enumerate}
  \setcounter{enumi}{3}
\item Wie viele Kubikmeter Wasser hat Chile in dem Exportjahr 2022 exportiert? \textbf{(2 Punkte)}
\end{enumerate}

Chile ist eines der wenigen Länder der Welt, die ihr Wasser komplett privatisiert haben. Derzeit sind nur zwei Prozent des Wassers des Landes für den häuslichen Verbrauch vorgesehen. In den Dörfern der Anbauregionen versorgen Tankwagen die Bevölkerung jede Woche mit Wasser, es gibt etwa 61 Liter Wasser pro Kopf für den täglichen Bedarf. In \textit{Deutschland} liegt der Verbrauch bei 10 - 15 Liter pro Minute Duschen und 9 - 14 Liter pro Sp{"u}lgang.

\begin{enumerate}
  \setcounter{enumi}{4}
\item Mit der rationierten Wassermenge aus Chiles Anbaugebieten können Sie in \textit{Deutschland} wie oft Ihren Bedarf stillen? \textbf{(1 Punkt)}
\end{enumerate}

Das alles hätten Sie nicht von Mark erwartet. Ganz schön viele Informationen wurden da zusammengetragen.

\begin{enumerate}
  \setcounter{enumi}{5}  
  \item Nennen Sie eine \textit{Daten}quelle im Internet, wo Sie mehr Informationen zu landwirtschaftlichen Daten oder klimatischen, wirtschaftlichen und gesellschaftlichen Daten erhalten! \textbf{(1 Punkt)}
\end{enumerate}



 
\clearpage
% ----------------------------------------------------------------------- 

\section{Aufgabe \hfill (10 Punkte)}

\textit{Geben Sie grundsätzlich Formeln und Rechenweg zur Lösung der Teilaufgaben mit an!} \\[1Ex]
 

 
%% --------------------------------------------------------------------
\ifcollection
\begin{flushright}
\tiny\vspace{-3Ex}
\textbf{\examinhaltstart}
\exammodulemathstat
\vspace{-4Ex}
\end{flushright}
\begin{minipage}[t]{0.5\textwidth}
\includegraphics[width = 1.3cm]{/Users/kruppajo/work/GitHub/exam/avatare/Paula.png}
\end{minipage}
\begin{minipage}[t]{0.5\textwidth}
\hfill
\href{https://youtu.be/WZSxntiNF8s}{\includegraphics[width = 2cm]{img/youtube}}
\end{minipage}
\fi
%% --------------------------------------------------------------------

%% --------------------------------------------------------------------
{\tiny\textbf{Stichworte:} Kardaschow-Skala $\bullet$ Dyson-Sphäre $\bullet$ Hohlerde $\bullet$ Entropie $\bullet$ Proton $r_P = 1.7 \times 10e-15$ $\bullet$ Wasserstoff $r_H = 5.3\times 10e-11$}
%% --------------------------------------------------------------------

\ifcollection
\paragraph{Die Dampfnudelerde}
\fi



"'Was für einen Unsinn!"', rufen Sie. Jetzt haben Sie auf Empfehlung von von Paula kostbaren Schlaf prokrastiniert um einem Ernährungswissenschaftler auf YouTube über die Erde als Dampfnudel zu lauschen. Irgendwie passt es dann doch mit der Analogie. Übermüdet müssen Sie darüber nachdenken, warum vor 66 Millionen Jahren die Dinosaurier - so groß sie auch waren - nicht von der Schwerkraft zu Boden gerissen wurden. Hat der Dampfplauderer etwa recht und war die Schwerkraft vor Millionen von Jahren eine andere?  Sind deshalb alle Lebewesen auf der Erde \textit{heutzutage} so viel kleiner, weil die Schwerkraft größer ist als damals? War die Erde kleiner und hatte weniger Masse? Oder ist es nur ein Rechenfehler wie bei der Theorie der Hohlerde von Edmond Halley aus dem 17.–18. Jahrhundert? Müde reiben Sie sich die Augen. So wird es nichts mehr mit dem Schlafen, dann können Sie auch mal etwas rechnen\footnote{Die Quelle der Inspiration
  für die Aufgabe war folgender Artikel:
  \href{https://hpd.de/artikel/erde-dampfnudel-22236}{"Skeptische Anmerkungen --- Die Erde als Dampfnudel" in Der Humanistische Pressedienst}}.  \\

Betrachten wir die Schwerkraft oder Gewichtskraft, die auf Lebewesen damals und heute gewirkt haben soll. Nehmen Sie für die Fallbeschleunigung $g$ der Erde \textit{heutzutage} einen Wert von 9.87m/s$^2$ an. Im Weiteren hat die Erde einen ungefähren Durchmesser von \ensuremath{1.2742\times 10^{4}}km und eine mittlere Dichte $\rho$ von 5.21g/cm$^3$. Das Gewicht von einem heute lebenden Waldelefanten mit 2.7t liegt bei 6t und das Gewicht von einem Tyrannosaurus rex (T. rex) bei 4.5t bis 8t.

\begin{enumerate}
\item Welchen Durchmesser müsste die Erde vor 66 Millionen Jahren gehabt haben, wenn Dinosaurier und Elefanten die gleiche Gewichtskraft $\overrightarrow{F_G}$ damals und heute erfahren hätten? \textit{Beantworten Sie die Frage anhand der folgenden Teilaufgaben!}
\begin{enumerate}
\item Berechnen Sie die Fallbeschleunigung von vor 66 Millionen Jahren unter der obigen Annahme gleich wirkender Gewichtskraft $\overrightarrow{F_G}$ auf Elefant und Dinosaurier! \textbf{(1 Punkt)}
\item Berechnen Sie Masse der heutigen Erde! \textbf{(2 Punkte)}
\item Schließen Sie über die Masse auf den Durchmesser der Erde vor 66 Millionen Jahren! \textbf{(2 Punkte)}
\end{enumerate}
\item Beantworten Sie die Eingangsfrage mit 1-2 Antwortsätzen! \textbf{(1 Punkt)}
\end{enumerate}

Die Distanz zwischen Sonne und Erde entspricht 0.99 astronomische Einheiten ($AE$). Die Einheit 1 AE wird mit \ensuremath{1.52\times 10^{8}}km angegeben. Der \textit{massebehaftete} Sonnenwind besteht aus 84\% Wasserstoffkernen mit einer molaren Masse von 1.01g/mol, 12\% Heliumkernen mit 3.92g/mol sowie 4\% weiteren Atomkernen mit  69.18g/mol. Die Teilchendichte bei Eintritt in die Erdatmosphäre liegt zwischen 0.4 bis 100 Teilchen cm$^{-3}$ pro Sekunde mit einer mittleren Teilchendichte von 7cm$^{-3}$ pro Sekunde. \\

\textit{Lösen Sie den folgenden Aufgabenteil mit einer aussagekräftigen Skizze!}

\begin{enumerate}
  \setcounter{enumi}{2}
\item Berechnen Sie die Anzahl an massebehafteten Teilchen des Sonnenwindes, die die gesamte Erde pro Sekunde treffen! \textbf{(2 Punkte)}
%\item Berechnen Sie die Anzahl an massebehafteten Teilchen des Sonnenwindes, die die Sonne pro Sekunde in alle Richtungen aussendet! \textbf{(2 Punkte)}
\item Berechnen Sie die Masse, die die Erde pro Jahr durch die \textit{massebehafteten} Teilchen des Sonnenwind zunimmt! \textbf{(2 Punkte)}
\end{enumerate}

%\blfootnote{\tiny\textbf{Stichworte:} Kardaschow-Skala $\bullet$ Dyson-Sphäre $\bullet$ Hohlerde $\bullet$ Entropie}

% https://de.wikipedia.org/wiki/Kardaschow-Skala

 
\clearpage
% ----------------------------------------------------------------------- 

\section{Aufgabe \hfill (10 Punkte)}

\textit{Geben Sie grundsätzlich Formeln und Rechenweg zur Lösung der Teilaufgaben mit an!} \\[1Ex]
 

 
%% --------------------------------------------------------------------
\ifcollection
\begin{flushright}
\tiny\vspace{-3Ex}
\textbf{\examinhaltstart}
\exammodulemathstat
\vspace{-4Ex}
\end{flushright}
\begin{minipage}[t]{0.5\textwidth}
\includegraphics[width = 1.3cm]{/Users/kruppajo/work/GitHub/exam/avatare/Jonas.png}\hspace{-4mm}\includegraphics[width = 1.3cm]{/Users/kruppajo/work/GitHub/exam/avatare/Mark.png}\hspace{-4mm}\includegraphics[width = 1.3cm]{/Users/kruppajo/work/GitHub/exam/avatare/Steffen.png}\hspace{-4mm}\includegraphics[width = 1.3cm]{/Users/kruppajo/work/GitHub/exam/avatare/Tina.png}
\end{minipage}
\begin{minipage}[t]{0.5\textwidth}
\hfill
\href{https://youtu.be/n451XnhtSh4}{\includegraphics[width = 2cm]{img/youtube}}
\end{minipage}
\fi
%% --------------------------------------------------------------------



\ifcollection
\paragraph{'Entschuldigung, ist das Ihre Feder in meinem Auge?'}
\fi



Man hört schon häufig vieles Geschnatter von höflichen Enten in Mastställen. Enge ist natürlich etwas ungünstig, den dann kommt es zu Picken und Kannibalismus. Denn wenn der Nachbar nervt, dann muss zu Maßnahmen gegriffen werden. Kennt jeder aus einer mittelmäßigen Wohngemeinschaft. Das wollen Steffen, Tina, Jonas und Mark aber als vorsorgliche Enten-Halter:innen nicht\footnote{Die Quelle der Inspiration für die Aufgabe war der folgende wissenschaftliche Artikel: \href{https://www.efsa.europa.eu/en/efsajournal/pub/7788}{EFSA Panel on Animal Health and Welfare, et al. (2023) Welfare of broilers on farm. EFSA Journal 21.2}.}. Gemeinsam sind die Vier in einer Projektgruppe gelandet. Betrachten wir also gemeinsam einmal das Platzangebot (eng. \textit{space allowance}, abk. \textit{SA}) der Enten für vier Tätigkeiten und versuchen die notwendige Fläche zu optimieren. Wie immer gibt es dafür eine mathematische Formel:

\begin{center}
  \begin{tabular}{cc}
    $SA = \sum^n_{i = 1} (A_i \times PB_i)$ & $A_i = \pi \times (r_i + R_i)^2$\\
  \end{tabular}
\end{center}

\vspace{-2Ex}

mit

\begin{itemize}[noitemsep]
\item $SA$ dem benötigten Platzangebot aller aufsummierten Verhalten $i$.
\item $A_i$ dem benötigten Platz für ein Verhalten $i$. 
\item $PB_i$ dem Anteil des Auftretens eines Verhaltens $i$.
\item $r_i$ dem Radius Ente plus dem benötigten Radius für das Verhalten $i$.
\item $R_i$ dem notwendigen Abstand zu den Nachbarn für das Verhalten $i$.    
\item $i$ dem Verhalten: (1) wingflapping, (2) standing, (3)
  preening und (4) sitting.
\end{itemize}

In der folgenden Tabelle 1 sind die Werte für $r_i$, $R_i$ und $PB_i$ für ein spezifisches Verhalten $i$ aus drei wissenschaftlichen Veröffentlichungen dargestellt.

\vspace{-1Ex}

{\small
\begin{knitrout}
\definecolor{shadecolor}{rgb}{0.969, 0.969, 0.969}\color{fgcolor}\begin{table}[!h]
\centering
\begin{tabular}{llll}
\toprule
  & Aldridge et al. (2021) & Baxter et al. (2022) & Jabcobs et al. (2019)\\
\midrule
wingflapping & 19cm; -4cm; 3.5\% & 4cm; 17cm; 12.1\% & 55cm; 46cm; 3.5\%\\
standing & 43cm; 45cm; 50.1\% & 46cm; 47cm; 64.1\% & 3cm; 48cm; 25.4\%\\
preening & 40cm; 32cm; 8.1\% & 40cm; 55cm; 5.2\% & 32cm; 8cm; 5.1\%\\
sitting & 14cm; 13cm; 1\% & 47cm; 24cm; 0.8\% & 24cm; 23cm; 1\%\\
\bottomrule
\end{tabular}
\end{table}

\end{knitrout}
}

Leider kennen sich die Vier nicht so gut mit der Berechnung aus! Daher brauchen die Vier Ihre Hilfe!

\begin{enumerate}
\item Skizzieren Sie die Werte $r_i$, $R_i$ und $A_i$ für zwei nebeneinander agierende Enten für ein Verhalten $i$. Nutzen Sie hierfür vereinfachte geometrische Formen! \textbf{(2 Punkte)}
\item Erstellen Sie eine zusammenfassende Tabelle mit den mittleren Werten für $r$, $R$ und $PB$ aus der obigen Tabelle 1 für die jeweiligen Verhalten! \textbf{(3 Punkte)}
\item Ergänzen Sie eine Spalte mit dem benötigten Platz $A$ für das jeweilige Verhalten, welches sich aus den mittleren Werten ergibt! \textbf{(1 Punkt)}
\item Berechnen Sie das benötigte Platzangebot $SA$ für alle betrachteten Verhalten! \textbf{(1 Punkt)}
\item Sie entnehmen der Literatur folgende Aussage zur Verteilung der Enten in der Fläche $A$: \textit{"`Assuming, that the animals will optimally and equally distribute in an area $A$, we observe a
    small part, which is not covered. This area is called $\omega$ and is calculated with $\omega = \tfrac{A}{0.9069}$."'} Veranschaulichen Sie die Fläche $\omega$ in einer aussagekräftigen Abbildung!  \textbf{(1 Punkt)}
\item Ein Tier braucht Platz für sich selbst. Berechnen Sie nun die Körperfläche $a$, die ein Tier einnimmt. Welche Annahmen haben Sie für die Berechnung der Körperfläche getroffen? \textbf{(2 Punkte)}
\end{enumerate}



 
\clearpage
% ----------------------------------------------------------------------- 

\section{Aufgabe \hfill (10 Punkte)}

\textit{Geben Sie grundsätzlich Formeln und Rechenweg zur Lösung der Teilaufgaben mit an!} \\[1Ex]
 

 
%% --------------------------------------------------------------------
\ifcollection
\begin{flushright}
\tiny\vspace{-3Ex}
\textbf{\examinhaltstart}
\exammodulemathstat
\vspace{-4Ex}
\end{flushright}
\begin{minipage}[t]{0.5\textwidth}
\includegraphics[width = 1.3cm]{/Users/kruppajo/work/GitHub/exam/avatare/Alex.png}\hspace{-4mm}\includegraphics[width = 1.3cm]{/Users/kruppajo/work/GitHub/exam/avatare/Jonas.png}
\end{minipage}
\begin{minipage}[t]{0.5\textwidth}
\hfill
\href{https://youtu.be/1B53cVFIU7Q}{\includegraphics[width = 2cm]{img/youtube}}
\end{minipage}
\fi
%% --------------------------------------------------------------------



\ifcollection
\paragraph{Nelken von den Molukken}
\fi



Alex und Jonas waren gemeinsam in Berlin und sitzen nun im IC nach Amsterdam um zurück nach Osnabrück zu fahren. 'Weißt du was ich mich frage?', entfährt es Alex ziemlich plötzlich, so dass Jonas die Snickers aus dem Mund fallen. 'Nein, und ehrlich gesagt bin ich auch ziemlich müde...'. Das ist jetzt aber Alex egal, denn er möchte folgende Sachlage diskutieren. Und Alex hat jetzt 3 Stunden Zeit. Plus Verspätung. In der Ausstellung \textit{Europa und das Meer} im Deutschen Historischen Museum in Berlin gab es folgendes Zitat über die Probleme der frühen Hochseeschifffahrt.

\begin{quote}
  >>Ohne ausreichende Zufuhr von Vitamin C stellen sich nach 40 Tagen die ersten Symptome ein; die ersten Toten sind nach 60 Tagen zu beklagen; nach 105 Tagen rafft die Skorbut eine ganze Schiffsbesatzung dahin<<
\end{quote}

Ferdinand Magellan stach im Jahre 1519 in See um eine Passage durch den südamerikanischen Kontinent zu finden. Zu seiner Flotte gehörten fünf Schiffe - das Flaggschiff Trinidad, die San Antonio, die Victoria, die Concepciön und die Santiago - mit einer Besatzung von insgesamt 245 Mann. 

\begin{enumerate}
\item Stellen Sie den Verlauf der Anzahl an Matrosen auf einem Schiff der Flotte in der Form einer Überlebenszeitkurve dar! Beschriften Sie die Achsen entsprechend! \textbf{(2 Punkte)} 
\item Was ist die Besonderheit der Überlebenszeitkurve? Begründen Sie Ihre Antwort! \textbf{(2 Punkte)} 
\item Schätzen Sie die Überlebenswahrscheinlichkeit nach 95 Tagen aus Ihrer Abbildung ab! \textbf{(1 Punkt)} 
\end{enumerate}

Der Chronist an Bord der Trinidad, Antonio Pigafetta, schrieb in seinem Bericht '[...] Um nicht Hungers zu sterben, aßen wir das Leder, mit dem die große Rahe zum Schutz der Taue umwunden war.' Insbesondere die Mannschaft der Concepciön erlitt große Verluste durch die Skrobut bei der Überquerung des Pazifiks, da durch Erkundungsfahrten weniger Zeit blieb, um wilden Sellerie aufzunehmen. Wilder Sellerie enthält $7000\mu g/100g$ Vitamin C. Der Bedarf liegt bei $110mg$ pro Tag für Männer.

\begin{enumerate}
  \setcounter{enumi}{2}
\item Berechnen Sie die notwendige Menge in $t$ an aufzunehmenden wilden Sellerie auf die Concepciön für die ununterbrochene Fahrt von drei Monate und 24 Tage über den Pazifik! \textbf{(3 Punkte)}
\item Skizzieren Sie die Überlebenszeitkurve für die Concepciön im Vergleich zu der Überlebenszeitkurve der Trinidad! Beschriften Sie die Achsen! \textbf{(2 Punkte)}
\end{enumerate}

 
\clearpage
% ----------------------------------------------------------------------- 

\section{Aufgabe \hfill (10 Punkte)}

\textit{Geben Sie grundsätzlich Formeln und Rechenweg zur Lösung der Teilaufgaben mit an!} \\[1Ex]
 

 
%% --------------------------------------------------------------------
\ifcollection
\begin{flushright}
\tiny\vspace{-3Ex}
\textbf{\examinhaltstart}
\exammodulemathstat
\vspace{-4Ex}
\end{flushright}
\begin{minipage}[t]{0.5\textwidth}
\includegraphics[width = 1.3cm]{/Users/kruppajo/work/GitHub/exam/avatare/Tina.png}\hspace{-4mm}\includegraphics[width = 1.3cm]{/Users/kruppajo/work/GitHub/exam/avatare/Yuki.png}
\end{minipage}
\begin{minipage}[t]{0.5\textwidth}
\hfill
\href{https://youtu.be/q-qYK4Chslg}{\includegraphics[width = 2cm]{img/youtube}}
\end{minipage}
\fi
%% --------------------------------------------------------------------



\ifcollection
\paragraph{Event Horizon -- Am Rande des Universums}
\fi



Yuki ist bei Tina um gemeinsam \textit{Event Horizon -- Am Rande des Universums} zu streamen. Das war jetzt nicht die beste Idee. Denn Yuki kann Horror überhaupt nicht ab. Deshalb flüchtet er sich in Logik um seine Emotionen zu bändigen. Tina mampft ungerührt Katjes. Folgenden Gedankengang nutzt Yuki um dem Film zu entkommen. Die Sonne hat eine aktuelle, angenommene Masse von $\ensuremath{2\times 10^{31}}$kg. Wenn die Sonne nun am Ende ihrer Lebenszeit zu einem schwarzen Loch mit dem Radius von $4000$m kollabiert, wird die Sonne $20$\% der aktuellen Masse verloren haben. Ein Lichtteilchen mit der Masse $m_f$ und der Fluchtgeschwindigkeit $v_f$ will dem schwarzen Loch entkommen. An folgende Formeln erinnert sich Yuki für die kinetische Energie des Lichtteilchens $E_{kin}$ und der Graviationsenergie des schwarzen Lochs $E_{grav}$\footnote{Die Quelle der Inspiration für die Aufgabe war ein Montagnachtfilm: \href{https://de.wikipedia.org/wiki/Event_Horizon_–_Am_Rande_des_Universums}{Event Horizon – Am Rande des Universums}}.

\begin{center}
  \begin{tabular}{cc}
    $E_{kin} = \cfrac{1}{2}m_fv_f^2$ & $E_{grav} = \cfrac{Gm_sm_f}{r_s}$\\
  \end{tabular}
\end{center}

mit

\begin{itemize}[noitemsep]
\item $m_f$, gleich der Masse [kg] des fliehenden Objektes
\item $m_s$, gleich der Masse [kg] des stationären Objekts
\item $r_s$, gleich dem Radius [m] des stationären Objekts  
\item $G$, gleich der Gravitationskonstante mit $6.165 \cdot 10^{-11} m^3(kg \cdot s^2)^{-1}$ 
\end{itemize}

Im Folgenden wollen wir Yuki bei der Ablenkung helfen und uns mit der Frage beschäftigen, ob das Lichtteilchen der Gravitation des schwarzen Lochs entkommen kann.

\begin{enumerate}
\item Geben Sie die Formel für die Fluchtgeschwindigkeit $v_f$ an! \textbf{(2 Punkte)}
\item Überprüfen Sie Ihre umgestellte Formel nach $v_f$ anhand der Einheiten! \textbf{(1 Punkt)} 
\item Berechnen Sie die notwendige Fluchtgeschwindigkeit $v_f$ des Lichtteilchens mit den angegebenen Informationen! \textbf{(2 Punkte)}
\item Gehen Sie von einer Lichtgeschwindigkeit von $\ensuremath{2.9\times 10^{8}}m/s$ aus. Kann das Lichtteilchen der Gravitation des schwarzen Lochs entkommen? Begründen Sie Ihre Antwort! \textbf{(2 Punkte)}
\item Stellen Sie den Zusammenhang zwischen dem sich verringernden Radius $r$ des schwarzen Lochs bei gleichbleibender Masse $m_s$ und der notwendigen Fluchtgeschwindigkeit $v_f$ in einer Abbildung dar! \textit{Erstellen Sie dafür eine Datentabelle mit fünf Werten für den Radius $r$!} \textbf{(3 Punkte)}
\end{enumerate}

 
\clearpage
% -----------------------------------------------------------------------

\section{Aufgabe \hfill (10 Punkte)}

\textit{Geben Sie grundsätzlich Formeln und Rechenweg zur Lösung der Teilaufgaben mit an!} \\[1Ex]
 

 
%% --------------------------------------------------------------------
\ifcollection
\begin{flushright}
\tiny\vspace{-3Ex}
\textbf{\examinhaltstart}
\exammodulemathstat
\vspace{-4Ex}
\end{flushright}
\begin{minipage}[t]{0.5\textwidth}
\includegraphics[width = 1.3cm]{/Users/kruppajo/work/GitHub/exam/avatare/Nilufar.png}\hspace{-4mm}\includegraphics[width = 1.3cm]{/Users/kruppajo/work/GitHub/exam/avatare/Tina.png}
\end{minipage}
\begin{minipage}[t]{0.5\textwidth}
\hfill
\href{https://youtu.be/iCQogS6KhPM}{\includegraphics[width = 2cm]{img/youtube}}
\end{minipage}
\fi
%% --------------------------------------------------------------------

%% --------------------------------------------------------------------
{\tiny\textbf{Stichworte:} Great filter $\bullet$ SETI $\bullet$ WOW-Signal $\bullet$ 5-Sigma $\bullet$ Voyager 1 $\bullet$ Voyager 2}
%% --------------------------------------------------------------------




\ifcollection
\paragraph{Das Fermi Paradoxon}
\fi



Tina und Nilufar wandern durch den Teuteburgerwald um mal vom Studium runterzukommen. 'Kennst du eigentlich Enrico Fermi?', fragt Tina und fährt ohne die Antwort abzuwarten fort, 'Er war ein berümter Kernphysiker! Enrico Fermi diskutierte 1950 auf dem Weg zum Mittagessen im Los Alamos National Laboratory mit seinen Kollegen angebliche UFO-Sichtungen und fragte schließlich: >>Where is everybody?<<. Warum seien weder Raumschiffe anderer Weltraumbewohner noch andere Spuren extraterrestrischer Technik zu beobachten?'. Nilufar schaut sie irritiert und interessiert an. Die beiden hat das Problem gepackt. Deshalb wollen Tina und Nilufar das Paradoxon mal mathematisch untersuchen! Wie lange würde eine außerirdische Zivilisation benötigen um die gesamte Milchstraße zu besuchen, wenn das maximale Reisetempo die Geschwindigkeit der Voyager 1 Sonde wäre?\footnote{Die Quelle der Inspiration für die Aufgabe war folgender Wikipediaeintrag: \href{https://de.wikipedia.org/wiki/Fermi-Paradoxon}{Fermi-Paradoxon}}\\[-1ex]

Die beiden treffen folgende Annahmen. Eine außerirdische Zivilisation schickt $drei$ Voyager 1 ähnliche Sonden mit der Geschwindigkeit von $\ensuremath{6.3587\times 10^{4}}km/h$ los um sich auf den erreichten Planeten selbst zu replizieren. Nach $500$ Jahren ist die Replikation abgeschlossen und wiederum $drei$ Sonden werden ausgesendet. Gehen Sie von $5.16$ Lichtjahren als mittlerer Abstand der Sterne in der Milchstraße aus. Es gibt $\ensuremath{1.5\times 10^{11}}$ Sterne in der Milchstraße. Nehmen Sie eine Lichtgeschwindigkeit von $\ensuremath{2.8\times 10^{8}}m/s$ an.

\begin{enumerate}
\item Skizzieren Sie in einer Abbildung die ersten vier Schritte der Vervielfältigung der Sonden in der Galaxie! Beschriften Sie die Abbildung mit der Dauer und der Anzahl an Sonden für jeden Schritt der Vervielfältigung! \textbf{(4 Punkte)}
\item Berechnen Sie die theoretische Anzahl an Vervielfältigungsschritten die benötigt werden um mit \textit{einem einzigen Vervielfältigungsschritt} die gesamten Sterne der Milchstraße mit Sonden zu besuchen! \textbf{(2 Punkte)}
\item Berechnen Sie die Dauer, die eine außerirdische Zivilisation annährungsweise benötigt um die gesamten Sterne der Milchstraße mit Sonden zu besuchen! \textbf{(2 Punkte)}
\item Bei einem vermutetet Alter der Erde von $\ensuremath{4.3\times 10^{9}}$ Jahren, wie oft war dann eine Sonde einer außerirdischen Zivilisation schon zu Besuch? Korrigieren Sie Ihre Antwort mit dem Wissen, dass sich die Kontinentalplatten einmal alle $\ensuremath{8\times 10^{7}}$ Jahre vollständig im Erdinneren umgewandelt haben! \textbf{(2 Punkte)}
\end{enumerate}


 
\clearpage
% -----------------------------------------------------------------------

\section{Aufgabe \hfill (10 Punkte)}

\textit{Geben Sie grundsätzlich Formeln und Rechenweg zur Lösung der Teilaufgaben mit an!} \\[1Ex]
 

 
%% --------------------------------------------------------------------
\ifcollection
\begin{flushright}
\tiny\vspace{-3Ex}
\textbf{\examinhaltstart}
\exammodulemathstat
\vspace{-4Ex}
\end{flushright}
\begin{minipage}[t]{0.5\textwidth}
\includegraphics[width = 1.3cm]{/Users/kruppajo/work/GitHub/exam/avatare/Nilufar.png}
\end{minipage}
\begin{minipage}[t]{0.5\textwidth}
\hfill
\href{https://youtu.be/tDgr6fpkkYA}{\includegraphics[width = 2cm]{img/youtube}}
\end{minipage}
\fi
%% --------------------------------------------------------------------



\ifcollection
\paragraph{Pyramiden bauen}
\fi



Es stehen die mecklemburgischen Pyramidentage an! Sie und Nilufar sind auf abenteuerlichen Wegen für den Bau der Pyramiden zuständig. Zu allem Überfluss handelt es sich auch noch eine \textit{Reenactment} Veranstaltung. Thema der diesjährigen Pyramidentage sind die Pyramiden von Meroe, die den Königen und Königinnen des historischen Reiches von Kusch in Nubien, dem heutigen Sudan, als Grabstätten dienten. Die Pyramiden in Meroe fallen durch ihren steilen Winkel von 73 Grad im Vergleich zu den ägyptischen Pyramiden mit 55 Grad auf. Die durchschnittliche Seitenlänge der Grundfläche einer Pyramide beträgt 44 Königsellen. Eine Königselle misst 52.4cm.\\

\textit{Lösen Sie diese Aufgabe mit Hilfe einer Skizze der Pyramide. Bezeichnen Sie Seiten und die Winkel der Pyramide entsprechend!}

\begin{enumerate}
\item Bei der Königspyramide von Meroe soll eine Seitenlänge der Grundfläche 44 Königsellen lang sein. Einer der Skla'angestellten hat eine Höhe der Königspyramide von 37.61$m$ berechnet. Dem müssen Sie auf jeden Fall nachgehen. Überprüfen Sie die Berechnung! \textbf{(1 Punkt)}
\item Die Außenflächen der Königspyramide soll begrünt werden. Für die Bepflanzung muss eine 7cm dicke Torfschicht auf die Königspyramide aufgebracht werden. Berechnen Sie die ungefähre Menge an benötigten Torf in $m^3$! \textbf{(2 Punkte)}
\end{enumerate}

Wie in jedem guten \textit{Reenactment} gibt es viel Oberschicht, aber nur 5 Sklaven, die Ihnen und Nilufar bei dem Befüllen der Königspyramide mit Schutt zu Seite stehen. Leider haben Ihre Sklaven zu allem Überfluss auch noch chronische Rückenschmerzen entwickelt, als die Sklaven von der anstehenden Aufgabe erfahren haben. Gehen Sie daher von einer Effizienz der Sklaven von 75\% aus. In eine Schubkarre passen 110 Liter.

\begin{enumerate}
  \setcounter{enumi}{2}
\item Wie oft müssen Ihre maladen Sklaven die Rampe mit der Schubkarre zur Spitze der Königspyramide hochfahren um die Pyramide mit Schutt zu füllen? \textbf{(1 Punkt)}
\item Berechnen Sie die Länge der Rampe zur Spitze der Königspyramide mit einem Anstellwinkel von $12^\circ$! \textbf{(2 Punkte)}
\item Wie weit reicht Ihre Rampe vom Fuß der Königspyramide in die mecklemburgische Landschaft?  \textbf{(2 Punkte)}
\end{enumerate}

Bei der Besichtigung der Königspyramide teilt Ihnen der leicht übergewichtige Pharao (Nebenberuf \textit{Finanzbeamter}) mit, das die Pyramide \underline{zu flach sei} und somit nicht in die mecklemburgische Landschaft passen würde. Sie müssen nochmal ran.

\begin{enumerate}
  \setcounter{enumi}{5}
\item Die Grundfläche der Königspyramide ändert sich nicht. Berechnen Sie die Änderung der Höhe in \underline{Königsellen}, wenn sich der Anstellwinkel der Königspyramide um $5^\circ$ ändert!  \textbf{(2 Punkte)}
\end{enumerate}



% https://de.wikipedia.org/wiki/Rechtwinkliges_Dreieck
% https://www.matheretter.de/wiki/pyramide
% https://de.wikipedia.org/wiki/Knickpyramide 
\clearpage
% -----------------------------------------------------------------------

\section{Aufgabe \hfill (10 Punkte)}

\textit{Geben Sie grundsätzlich Formeln und Rechenweg zur Lösung der Teilaufgaben mit an!} \\[1Ex]
 

 
%% --------------------------------------------------------------------
\ifcollection
\begin{flushright}
\tiny\vspace{-3Ex}
\textbf{\examinhaltstart}
\exammodulemathstat
\vspace{-4Ex}
\end{flushright}
\begin{minipage}[t]{0.5\textwidth}
\includegraphics[width = 1.3cm]{/Users/kruppajo/work/GitHub/exam/avatare/Alex.png}\hspace{-4mm}\includegraphics[width = 1.3cm]{/Users/kruppajo/work/GitHub/exam/avatare/Jonas.png}
\end{minipage}
\begin{minipage}[t]{0.5\textwidth}
\hfill
\href{https://youtu.be/3LAq3R0rS14}{\includegraphics[width = 2cm]{img/youtube}}
\end{minipage}
\fi
%% --------------------------------------------------------------------



\ifcollection
\paragraph{Geocaching -- Von Satelliten und Plastikdosen}
\fi



Es ist Wochenende und das Wetter ist \textit{sweet}. Jonas und Alex schwingen sich auf ihre Cachermobile um mit 19km/h, geleitet von modernster Satellietentechnologie und einem Supercompter aus dem Jahr 2000 in den Händen, Plastikdosen in der Natur und an sehenswerten Orten zu finden. Jonas und Alex wollen diesmal endlich die abwärts Schwierigkeitschallenge durchführen. Die Reihenfolge der Caches nach Schwierigkeitswertung gibt daher die von den beiden abzufahrenden Orte vor. Die Terrain- und Schwierigkeitswertungen laufen von 1 (leichteste Wertung) bis 5 (schwierigste Wertung) in 0.5 Schritten. Folgende Informationen zu den Orten und den entsprechenden Caches stehen Jonas und Alex für die Planung der Route zu Verfügung\footnote{Die Quelle der Inspiration für die Aufgabe war folgende Tätigkeit: \href{https://www.geocaching.com/play}{Geocaching -- Mach mit bei der weltweit größten Schatzsuche.}}.

\begin{center}
  \begin{tabular}{ ccc }
    \toprule
    Ort & Cache & Wertung (S|T|G) \\
    \midrule
    A & GCOBBET & 4.5 | 2.0 | Normal \\
    B & GCUR8SL & 5.0 | 1.0 | Normal \\ 
    C & GCUWLL7 & 1.5 | 4.5 | Normal \\ 
    D & GCHZSNO & 1.0 | 3.0 | Mikro \\ 
    E & GCH25BD & 2.5 | 1.5 | Klein \\     
 \bottomrule
\end{tabular}
\end{center}

Im Weiteren sind den beiden folgende Informationen zu den Entfernungen der Orte zugänglich. Der Entfernungsvektor $\overrightarrow{AB}$ ist $3$km. Im Weiteren ist Ihnen der Entfernungsvektor $\overrightarrow{CB}$ mit $6.5$km bekannt. Der Entfernungsvektor $\overrightarrow{BE}$ ist das $1.5$-fache des Entfernungsvektor $\overrightarrow{CB}$. Wenn Sie von dem Ort A den Ort C anpeilen, so liegt der Ort B ungefähr $40^\circ$ südlich. Wenn Sie von dem Ort C den Ort B anpeilen, so liegt der Ort D ungefähr $50^\circ$ östlich. Vom Ort B betrachtet, bilden die Orte C und D einen rechten Winkel am Ort B. Der Ort B liegt auf gerader Linie zwischen den Orten C und E. Somit liegt der Ort E südlich von B. Die Strecke zwischen A und E ist nicht passierbar. Sie starten an dem Ort D Ihre Cachertour. \\

Leider sind die beiden sehr schlecht im Navigieren und Entfernungen ausrechnen. Die beiden brauchen Ihre Hilfe!
  
\begin{enumerate}
\item Lösen Sie diese Aufgabe mit Hilfe einer aussagekräftigen Skizze der Orte und Caches. Bezeichnen Sie die Strecken und die Winkel Ihrer Skizze entsprechend! \textbf{(5 Punkte)}
\item Welche Strecke in $km$ legen Sie bei der Bewältigung der abwärts Schwierigkeitschallenge zurück? \textbf{(2 Punkte)}
\item Gehen Sie von einer zusätzlichen Suchzeit in Stunden für die Caches an den jeweiligen Orten zur reinen Reisezeit mit Ihrem Cachermobil aus. Die Suchzeit in Stunden für \textit{jeden einzelnen} Cache wird durch die Funktion  
  \begin{equation*}
    Suchzeit = 0.15 + 0.18 \cdot Schwierigkeit
  \end{equation*}  
  beschreiben.  Wie lange in Stunden benötigen Sie um die abwärts Schwierigkeitschallenge zu erfüllen? \textbf{(3 Punkte)}
%\item An der höchsten Schwierigkeit müssen Sie angeln. Ihre Angel ist ausgefahren l_angelm lang. Erreichen Sie einen Cache in der Höhe von h_angelm?  Berechnen Sie dazu Ihre maximale mögliche Angelhöhe! Welche Annahmen mussten Sie treffen um die Aufgabe zu lösen? \textbf{(2 Punkte)} 
\end{enumerate}

 
\clearpage
% -----------------------------------------------------------------------

\section{Aufgabe \hfill (10 Punkte)}

\textit{Geben Sie grundsätzlich Formeln und Rechenweg zur Lösung der Teilaufgaben mit an!} \\[1Ex]
 

 
%% --------------------------------------------------------------------
\ifcollection
\begin{flushright}
\tiny\vspace{-3Ex}
\textbf{\examinhaltstart}
\exammodulemathstat
\vspace{-4Ex}
\end{flushright}
\begin{minipage}[t]{0.5\textwidth}
\includegraphics[width = 1.3cm]{/Users/kruppajo/work/GitHub/exam/avatare/Jonas.png}
\end{minipage}
\begin{minipage}[t]{0.5\textwidth}
\hfill
\href{https://youtu.be/dcMhg0iaHNI}{\includegraphics[width = 2cm]{img/youtube}}
\end{minipage}
\fi
%% --------------------------------------------------------------------

%% --------------------------------------------------------------------
{\tiny\textbf{Stichworte:} Brot aus Luft $\bullet$ Walöl $\bullet$ Haber-Bosch-Verfahren $\bullet$ 1. Weltkrieg $\bullet$ 40\% N im menschlichen Körper $\bullet$ Positivist}
%% --------------------------------------------------------------------



\ifcollection
\paragraph{Die atmende Wand und Brot aus Luft}
\fi



Als Kellerkind\footnote{\href{https://www.youtube.com/watch?v=54H0HAJexVI}{Tocotronic - Electric Guitar} als passende Untermalung für diese Aufgabe.} vom Dorf will Jonas das Ausmaß der Radonbelastung in seinem Kellerzimmer bestimmen und lüften daher nicht. Passt schon. Spart dann auch Energie und lüften wird sowieso überschätzt. Während einer Messperiode von 7:00 Uhr bis 15:00 bestimmt er dreimal automatisch die Radonbelastung in seinem Kellerraum in $Bq/m^3$. Es ergibt sich folgende Abbildung\footnote{Die Quelle der Inspiration für die Aufgabe war folgender Artikel: \href{https://de.wikipedia.org/wiki/Atmende_Wand}{Atmende Wand}}. Leider helfen die Messwerte Jonas überhaupt nicht weiter. Sie müssen also helfen! 

\begin{knitrout}
\definecolor{shadecolor}{rgb}{0.969, 0.969, 0.969}\color{fgcolor}

{\centering \includegraphics[width=\maxwidth]{img/math-10-1} 

}


\end{knitrout}

\vspace{-0.75cm}

\begin{enumerate}
\item Wie lange dauert es in Stunden bis Jonas eine kritische Belastung von 400$Bq/m^3$ in dem ungelüfteten Kellerraum erreicht hat? \textbf{(2 Punkte)}
\end{enumerate}

Radon zerfällt mit einer Halbwertszeit von 4.1d zu Polonium. Polonium wiederum zerfällt mit einer Halbwertszeit von 143d zu Blei. Nur Radon und Polonium tragen zur radioaktiven Strahlenbelastung bei.

\begin{enumerate}
  \setcounter{enumi}{1}
\item Wie lange dauert es in Stunden bis die kritische Radonbelastung von
  400$Bq/m^3$ auf unter 110$Bq/m^3$ gefallen ist?
  \textbf{(4 Punkte)}
\end{enumerate}

Folgende Tabelle enthält die Informationen zur Zusammensetzung der normalen Umgebungsluft.

\begin{center}
  \begin{tabular}{ c|c|c|c }
     & Vol-\% & M [g/mol] & ppm \\
    \hline
    Stickstoff & 79.7 & 28.4 &
                                                    \phantom{1000000000000}\strut\\
        \hline
    Sauerstoff & 20.45 & 15.8 &
                                                    \phantom{10000000}\strut\\
        \hline
    Kohlenstoffdioxid & 0.029 & 11.8 & \phantom{10000000}\strut\\     
     \hline
\end{tabular}
\end{center}

\begin{enumerate}
   \setcounter{enumi}{2}
\item Rechnen Sie die Volumenprozente (Vol-\%) der Umgebungsluft in die entsprechenden ppm-Werte um und ergänzen Sie die berechneten ppm-Werte in die Tabelle!  \textbf{(1 Punkt)}
\end{enumerate}

Während Jonas sein etwas pappiges Toastbrot mampfen kommt Jonas die Dokumentation über Brot aus Luft in den Sinn. Jonas denkt darüber ein wenig nach. Für die Umwandlung von Stickstoff $N_2$ mit Wasserstoff $H_2$ zu Ammoniak $NH_3$ gilt folgende Reaktionsgleichung\footnote{Die Quelle der Inspiration für die Aufgabe war folgender Artikel: \href{https://www.gdch.de/netzwerk-strukturen/fachstrukturen/ag-chemie-und-gesellschaft/projekte-und-veranstaltungen/cartoons/haber-bosch-verfahren-brot-aus-luft.html}{Haber-Bosch-Verfahren – Brot aus Luft}}:

\begin{equation*}
  N_2 + 3H_2 \rightarrow 2NH_3
\end{equation*}  

Ein Mol eines beliebigen Gases hat bei normalen Umweltbedingungen ein Volumen von 22.4 Liter. % Ein Mol enthält $6 \cdot 10^{23}$ Teilchen.

\begin{enumerate}
  \setcounter{enumi}{3}
\item Welche Masse an Ammoniak $2NH_3$ in Kilogramm $kg$ können Sie aus einem Kubikmeter $m^3$ Luft unter normalen Umweltbedingungen gewinnen? \textit{Beachten Sie die Reaktionsgleichung und gehen Sie von einem molaren Gewicht von Wasserstoff $H_2$ mit $2g/mol$ aus.}
  \textbf{(3 Punkte)}
\end{enumerate}

 
\clearpage
% -----------------------------------------------------------------------

\section{Aufgabe \hfill (10 Punkte)}

\textit{Geben Sie grundsätzlich Formeln und Rechenweg zur Lösung der Teilaufgaben mit an!} \\[1Ex]
 

 
%% --------------------------------------------------------------------
\ifcollection
\begin{flushright}
\tiny\vspace{-3Ex}
\textbf{\examinhaltstart}
\exammodulemathstat
\vspace{-4Ex}
\end{flushright}
\begin{minipage}[t]{0.5\textwidth}
\includegraphics[width = 1.3cm]{/Users/kruppajo/work/GitHub/exam/avatare/Jonas.png}
\end{minipage}
\begin{minipage}[t]{0.5\textwidth}
\hfill
\href{https://youtu.be/Bbu6n8MXxQk}{\includegraphics[width = 2cm]{img/youtube}}
\end{minipage}
\fi
%% --------------------------------------------------------------------



\ifcollection
\paragraph{Armee der Finsternis}
\fi



Der Studentenjob von Jonas war nach Ladenschluss bei Aldi die Regale einzuräumen. Dabei ist Jonas in der Auslage der Sonderangebote das Necronomicon\footnote{Ein wirklich gefährliches Buch ist: \textit{Du bist genug: Vom Mut, glücklich zu sein} von Fumitake Koga und Ichiro Kishimi} in die Hände gefallen. Nun ist er ein Magier der Zeichen geworden! Also eigentlich kann Jonas nur Mathe und das dämliche Necronomicon hat ihn in die Vergangenheit geschleudert... aber gut, was tut man nicht alles im Jahr 246 n. Chr. für den neuen Lehnsherren Fürsten Arthur. Jonas baut natürlich einen Schrottkugelturm um sich den
Horden der Finsternis mit genug Schrott erwehren zu können! Jonas stehen zwei mächtige magische Formeln zur Unterstützung zu Verfügung. Leider wird das nicht reichen, deshalb müssen Sie hier auch noch durch Zeit und Raum helfen!

\begin{center}
  \begin{tabular}{cc}
    $E_{kin} = \cfrac{1}{2}\cdot m \cdot v^2$ & $E_{pot} = m \cdot g \cdot h$\\
  \end{tabular}
\end{center}

mit

\begin{itemize}[noitemsep]
\item $m$, gleich der Masse [kg] des Objekts
\item $h$, gleich der Höhe [m] des ruhenden Objekts
\item $v$, gleich der Geschwindigkeit [m/s] des Objekts
\item $g$, gleich der Erdbeschleunigung mit $9.81 \tfrac{m}{s^2}$ 
\end{itemize}

Als erstes müssen Sie die Höhe des zu bauenden Schrottkugelturmes bestimmen. Hierfür ist wichtig zu wissen, dass sich die Blei\textit{tropfen} mit einem Gewicht von $20mg$ zu gleichförmigen Blei\textit{kugeln} bei einer Geschwindigkeit von $14m/s$ bilden.

\begin{enumerate}
\item Wie hoch müssen Sie den Schrottkugelturm bauen lassen, damit sich runde Bleikugeln durch die Fallgeschwindigkeit von $14m/s$ bilden? \textbf{(3 Punkte)}
\end{enumerate}

Ihre erstellten Schrottkugeln sind leider zu gro{\ss} und somit sind zu wenige Schrottkugeln in einer Ladung. Damit können Sie die Armee der Finsternis nicht aufhalten. Die Sachlage müssen Sie einmal mathematisch untersuchen.

\begin{enumerate}
  \setcounter{enumi}{1}
\item Nennen Sie die beiden geometrischen Formen aus denen sich näherungsweise ein Tropfen zusammensetzt! Erstellen Sie eine beschriftete Skizze des Tropfens! \textbf{(2 Punkte)}
\item Sie messen eine Länge des Tropfens von 3.1mm. Die Löcher im Sieb erlauben ein Tropfendurchmesser von 2mm. Welchen Durchmesser in mm haben Ihre produzierten Blei\textit{kugeln}?  \textbf{(3 Punkte)}
\end{enumerate}

Sie haben jetzt die \ensuremath{1.2\times 10^{6}} Bleikugeln zusammen. Blei hat eine Dichte
von $11.34g/cm^3$.

\begin{enumerate}
  \setcounter{enumi}{3}
\item Wie schwer in Kilogramm $kg$ sind die \ensuremath{1.2\times 10^{6}} produzierten
  Bleikugeln, die Sie jetzt auf die Burgmauer transportieren müssen?
  \textbf{(1 Punkt)}
\end{enumerate}

Am Ende müssen Sie noch die Produktion von dem Bleischrott im Turm optimieren.

\begin{enumerate}
  \setcounter{enumi}{4}
\item Wie gro{\ss} in $cm^2$ ist Ihr quadratisches Sieb am oberen Ende des Turms,
  wenn Sie pro Fall ca. 700 Bleikugeln produzieren wollen und
  die Bleikugel im Fall 0.5cm Abstand haben müssen?  \textbf{(1
    Punkt)}
\end{enumerate}
 
\clearpage
% -----------------------------------------------------------------------

\section{Aufgabe \hfill (10 Punkte)}

\textit{Geben Sie grundsätzlich Formeln und Rechenweg zur Lösung der Teilaufgaben mit an!} \\[1Ex]
 

 
%% --------------------------------------------------------------------
\ifcollection
\begin{flushright}
\tiny\vspace{-3Ex}
\textbf{\examinhaltstart}
\exammodulemathstat
\vspace{-4Ex}
\end{flushright}
\begin{minipage}[t]{0.5\textwidth}
\includegraphics[width = 1.3cm]{/Users/kruppajo/work/GitHub/exam/avatare/Mark.png}
\end{minipage}
\begin{minipage}[t]{0.5\textwidth}
\hfill
\href{https://youtu.be/Mr6eslls4J0}{\includegraphics[width = 2cm]{img/youtube}}
\end{minipage}
\fi
%% --------------------------------------------------------------------



\ifcollection
\paragraph{Armee der Kaninchen}
\fi



Leider hat es bei Mark mit der Surfschule in Down Under nicht geklappt. War vielleicht auch nicht \textit{so} die beste Idee... aber dafür hat Mark eine neue Eingebung! Oder wie es Mike Tyson zugeschrieben wird: >>Ich wurde nie niedergeschlagen, ich war immer am Aufstehen!<<. Daher macht Mark jetzt einen Großhandel mit Kaninchenfleisch und damit dem teuersten Fleisch in Australien auf. Moment, hopsen hier nicht, seit Thomas Austin im Jahr 1859 ungefähr 28 Kaninchen entlassen hat, Millionen von Kaninchen rum? Wieso ist das Kaninchenfleisch dann so exklusiv? Mark wird stutzig und frag Sie, dem mal mathematisch nachzugehen!\footnote{Die Quelle der Inspiration für die Aufgabe war der folgendes YouTube Video: \href{https://youtu.be/38fuOr3tdgc?si=Li7NL_FoByML8JtT}{ Incredible Stories -- Why don't they eat wild rabbits in Australia? They have    millions of them! The reason is surprising...}} \\

Forscherinnen fand folgende Sättigungsfunktion für das jährliche Wachstum der gesamten Kaninchenpopulation im westlichen Australien.
\begin{equation*}
  f(t) = \ensuremath{10^{10}} - \ensuremath{1.4\times 10^{9}} \cdot 2.3^{-0.2 \cdot t + 3.1}
\end{equation*}

\begin{enumerate}
\item Skizzieren Sie die Sättigungsfunktion \textit{annäherungsweise} in einer Abbildung! \textbf{(1 Punkt)}
\item Wie viele Kaninchen können nach der Sättigungsfunktion maximal im westlichen Australien leben? Ergänzen Sie den Wert in Ihrer Abbildung! \textbf{(2 Punkte)}
\item Wie viele Millionen Kaninchen leben nach der Sättigungsfunktion nach 6 Jahren auf dem australischen Kontinent? \textbf{(1 Punkt)}
\end{enumerate}

Um den Kaninchen Einhalt zu gebieten wurde das Myxoma Virus und das Rabbit Haemorrhagic Disease Virus (RHDV) in 16 Kaninchen ausgebracht. Da die Kaninchen keine Maßnahmen gegen die Ausbreitung vornehmen können, verläuft die Ausbreitung mit einem wöchentlichen Wachstumsfakor von 1.6 nach folgender Formel.

\begin{equation*}
  N(t) = N(0) \cdot a^t
\end{equation*}

\begin{enumerate}
  \setcounter{enumi}{2}
\item Wie viele Wochen benötigen die Viren um theoretisch die gesamte Kaninchenpopulation nach 4 Jahren Wachstum zu durchseuchen? \textbf{(1 Punkt)}
\end{enumerate}

Das Myxoma Virus und das RHDV töten 98.5\% der Kaninchenpopulation innerhalb weniger Wochen.

\begin{enumerate}
  \setcounter{enumi}{3}  
\item Wie lange in Jahren dauert es bis eine Kaninchenpopulation nach einer Viruspandemie wieder auf 70\% der gesättigten Kaninchenpopulation angewachsen ist?  \textbf{(2 Punkte)}
\end{enumerate}

Thomas Austin entließ die Kaninchen im äußersten Norden von Australien. Australien hat eine West-Ost-Ausdehnung von 4200km und eine Nord-Süd-Ausdehnung von knapp 3500km. Die Kaninchen breiten sich radial mit einer Geschwindigkeit von 7.8km pro Jahr aus.

\begin{enumerate}
  \setcounter{enumi}{4}
\item Wie lange dauert es in Jahren bis die Kaninchen jeden Ort in Australien erreicht haben? \textit{Lösen Sie die Aufgabe unter der Verwendung einer schematischen Skizze!} \textbf{(2 Punkte)}
\end{enumerate}

Eine jährliche Impfung gegen das Myxoma Virus und das Rabbit Haemorrhagic Disease Virus (RHDV) kosten 12\$ pro Tier und der durchführende Arzt verlangt ca. 45\$ pro Tier.

\begin{enumerate}
  \setcounter{enumi}{5}
\item In Ihrem Stall leben 1000 Mastkaninchen. Mit welchen jährlichen Zusatzkosten für die Impfungen der Kaninchen müssen Sie daher kalkulieren? \textbf{(1 Punkt)}
\end{enumerate}
 
\clearpage
% -----------------------------------------------------------------------

\section{Aufgabe \hfill (12 Punkte)}

\textit{Geben Sie grundsätzlich Formeln und Rechenweg zur Lösung der Teilaufgaben mit an!} \\[1Ex]
 

 
%% --------------------------------------------------------------------
\ifcollection
\begin{flushright}
\tiny\vspace{-3Ex}
\textbf{\examinhaltstart}
\exammodulemathstat
\vspace{-4Ex}
\end{flushright}
\begin{minipage}[t]{0.5\textwidth}
\includegraphics[width = 1.3cm]{/Users/kruppajo/work/GitHub/exam/avatare/Yuki.png}
\end{minipage}
\begin{minipage}[t]{0.5\textwidth}
\hfill
\href{https://youtu.be/fK34Yl9Caas}{\includegraphics[width = 2cm]{img/caution}}
\end{minipage}
\fi
%% --------------------------------------------------------------------






\ifcollection
\paragraph{Uckermark. Unendliche Weiten.}
\fi



Wir schreiben das Jahr 2026. Dies sind die Abenteuer des Esels Fridolin und Yuki. Grünes Gras unter Yukis Füßen und ein strammer Wind im Gesicht, egal wohin er schaut. Ein schmatzendes Geräusch ertönt unter Yuki. Yuki sinniert, sollte er seine weiten Graslandschaften jetzt schon düngen? Dafür benötigt Yuki die \textit{Grünlandtemperatur}! Die Grünlandtemperatur (GLT) ist die Summe aller positiven Tagesmitteltemperaturen seit Jahresbeginn. Ab einer GLT von 200$^\circ C$ kann mit der Stickstoffdüngung begonnen werden. Yuki sieht nicht ein, Geld für einen Agrarmetrologen zu bezahlen, wenn auch Sie mitrechnen können. Also rechnen Sie beide mit folgenden Informationen zu Monatsmultiplikatoren des GLT-Wertes: Januar mit $0.7\times$, Februar mit $0.75\times$ und März mit
$1.05\times$. Sie haben noch im letzten Jahr folgende Temperaturen an den entsprechenden Tagen gemessen.

\begin{center}
\begin{tabular}{cc}
  \toprule
  Datum & C$^\circ$ \\
  \midrule
  01. Jan 2025 & 1.6\\
  01. Feb 2025 & 2.5\\
  01. Mrz 2025 & 5.1\\
  01. Apr 2025 & 6.9\\
  \bottomrule
\end{tabular}
\end{center}

\textit{Ignorieren Sie ein eventuelles Schaltjahr in Ihrer Berechnung.}

\begin{enumerate}
\item Erstellen Sie eine Skizze aus den Informationen aus der Temperaturtabelle! Beschriften Sie die Achsen entsprechend! \textit{Gehen Sie von einem linearen Zusammenhang innerhalb der Monate aus.} \textbf{(1 Punkt)}
\item Stellen Sie die linearen Funktionen $f_1(t)$, $f_2(t)$ und $f_3(t)$ aus der obigen Temperaturtabelle auf! \textit{Nehmen Sie zur Vereinfachung den 01. Januar 2025 als y-Achsenabschnitt für $f_1(t)$ an.}  \textbf{(3 Punkte)}
\item Bestimmen Sie die Stammfunktionen $F_1(t)$, $F_2(t)$ und $F_3(t)$ für die linearen Funktionen aus der obigen Temperaturtabelle!  \textbf{(2 Punkte)}
\end{enumerate}

Nachdem Yuki die Stammfunktionen vorliegen hat, kann jetzt mit den GLT-Werten als Fläche unter der Kurve gerechnet werden. 

\begin{enumerate}
  \setcounter{enumi}{3} 
\item Am 10. März 2026 hat Yuki Geburtstag. Welchen GLT-Wert wird Yuki am 10. März 2026 vermutlich erreicht haben? \textbf{(3 Punkte)}
\item Yuki liebt Osterglocken.  Osterglocken beginnen ab einer GLT von 76$^\circ C$ zu blühen. An welchem Tag im 1. Quartal des Jahres 2025 war dies der Fall?  \textbf{(3 Punkte)}
\end{enumerate}

%Auf dem Weg zu names apfellantage werden Sie beide auf dem Trecker von einer Gruppe elektrifizierter Rentner abgedrängt. Der Trecker muss wieder aus dem Graben! name_animal und die elektrifizierten Rentner ziehen an zwei, separaten Seilen. Dabei zieht name_animal mit $n_fridN$. Die elektrifizierter Rentner bringen eine Kraft von $n_rentN$ auf.\\

%\textit{Lösen Sie diese Aufgabe mit Hilfe einer aussagekräftigen Skizze der Kraftvektoren. Bezeichnen Sie die Kraftvektoren und die Winkel Ihrer Skizze entsprechend!}

%\begin{enumerate}
%  \setcounter{enumi}{4}  
%\item Im ersten Versuch legen Sie das Seil für name_animal lotrecht über einen Ast oberhalb des Treckers. Die Rentner ziehen in einer geraden Linie über die Böschung hinweg am anderen Seil. Welche Kraft wird aufgebracht?  \textbf{(2 Punkte)}
%\item Im zweiten Versuch ziehen name_animal und die Rentner mit einem $gamma^\circ$ Winkel mit ihrem Seil an dem Trecker. Welche Kraft wird aufgebracht? \textbf{(2 Punkte)}
%\end{enumerate}

 
\clearpage
% -----------------------------------------------------------------------

\section{Aufgabe \hfill (10 Punkte)}

\textit{Geben Sie grundsätzlich Formeln und Rechenweg zur Lösung der Teilaufgaben mit an!} \\[1Ex]
 

 
%% --------------------------------------------------------------------
\ifcollection
\begin{flushright}
\tiny\vspace{-3Ex}
\textbf{\examinhaltstart}
\exammodulemathstat
\vspace{-4Ex}
\end{flushright}
\begin{minipage}[t]{0.5\textwidth}
\includegraphics[width = 1.3cm]{/Users/kruppajo/work/GitHub/exam/avatare/Jessica.png}\hspace{-4mm}\includegraphics[width = 1.3cm]{/Users/kruppajo/work/GitHub/exam/avatare/Jonas.png}\hspace{-4mm}\includegraphics[width = 1.3cm]{/Users/kruppajo/work/GitHub/exam/avatare/Paula.png}\hspace{-4mm}\includegraphics[width = 1.3cm]{/Users/kruppajo/work/GitHub/exam/avatare/Tina.png}
\end{minipage}
\begin{minipage}[t]{0.5\textwidth}
\hfill
\href{https://youtu.be/RuzMjwvwT-4}{\includegraphics[width = 2cm]{img/youtube}}
\end{minipage}
\fi
%% --------------------------------------------------------------------



\ifcollection
\paragraph{In der Kartonagenfabrik}
\fi



Tina, Jonas, Paula und Jessica sitzen im Bus. Wenn man sich zu spät anmeldet, dann ist die Exkursion nicht so toll. Tina hatte den Anderen in der Lerngruppe zu spät Bescheid gesagt. 'Was denn, bin ich eure Nanny oder was?!', entfährt es Tina nachdem die vorwurfsvollen Blicke schon eine Weile auf ihr lasten. Also geht es eben mit Rektor Skinner und Mrs. Krabappel in die Kartonagenfabrik. Wie schon im vorherigen Semester... In der Kartonagenfabrik angekommen erfahren die Vier, dass die Kartons zum Versand von Nägeln nicht hier zusammengebautwerden sondern das sich die Endfertigung in Flint, Michigan befindet. Unter anderem wird dort der berühmte \textit{Doppelt gewellte, 5-mal-gefaltete, 0.5mm, 40-cm-Karton} durch
Falzung hergestellt. Beim letzten Mal war Rektor Skinner die Stimmung zu schlecht und deshalb geht es erst nach Hause, wenn ein paar Aufgaben gelöst sind. Martin gefällt das. An dem Vorrat an Zigaretten von Mrs. Krabappel meinen alle wenig Zuversicht zu erkennen.\\

Jetzt heißt es Kartons optimieren, wenn Sie auch nochmal nach Hause wollen. Warum jetzt \textit{Sie} mit dabei sind, lassen wir mal weg. Der nun zu optimierende, flache Karton hat eine Länge von 40cm und eine Breite von 18cm. Die Kartonagenmaschine in Flint soll dann einen quadratischen Eckenausschnitt der Länge $x$ falzen.

\begin{enumerate}
\item Erstellen Sie eine Skizze des Karton\textit{blatt}rohlings! Beschriften Sie die Skizze mit den entsprechenden Längenangaben \textbf{(1 Punkt)}
\item Berechnen Sie die Falztiefe $x$ für ein maximales Volumen des flachen Kartons! \textbf{(3 Punkte)}
\item Welches Volumen in Liter ergibt sich mit der von Ihnen berechneten Falztiefe $x$?  \textbf{(1 Punkt)}
\item Sie wollen noch einen bündig mit dem Boden abschließenden Deckel für den Karton stanzen lassen. Wie groß ist die Fläche des Kartondeckel\textit{blatt}rohlings in $cm^2$? \textbf{(2 Punkte)}
\end{enumerate}

Rektor Skinner möchte sich gerne wieder in seinem Vorgarten aufhalten und nicht die ganze Zeit von Bart mit Erdnüssen beworfen werden. Deshalb möchte er einen geräumigen Teil seines Vorgartens einzäunen. Ein Teil der Umzäunung bildet seine Vorderhauswand. Wegen Lieferschwierigkeiten stehen Rektor Skinner nur 110m Zaun zu Verfügung. Auch hier sollen Sie mal helfen, sonst fährt der Bus Sie nicht nach Hause. Sie wollen nun die maximale Fläche des abgeschirmten Vorgartens in Abhängigkeit der Seitenlängen bei der Verwendung von 110m Zaun bestimmen!

\begin{enumerate}
  \setcounter{enumi}{4}  
\item  Welche Seitenlängen für den Zaun ergeben sich für die maximale Fläche des abgeschirmten Vorgartens? \textbf{(2 Punkte)}
\item Berechnen Sie die Fläche des abgeschirmten Vorgartens! \textbf{(1 Punkt)}
\end{enumerate}

 
\clearpage
% -----------------------------------------------------------------------

\section{Aufgabe \hfill (10 Punkte)}

\textit{Geben Sie grundsätzlich Formeln und Rechenweg zur Lösung der Teilaufgaben mit an!} \\[1Ex]
 

 
%% --------------------------------------------------------------------
\ifcollection
\begin{flushright}
\tiny\vspace{-3Ex}
\textbf{\examinhaltstart}
\exammodulemathstat
\vspace{-4Ex}
\end{flushright}
\begin{minipage}[t]{0.5\textwidth}
\includegraphics[width = 1.3cm]{/Users/kruppajo/work/GitHub/exam/avatare/Alex.png}\hspace{-4mm}\includegraphics[width = 1.3cm]{/Users/kruppajo/work/GitHub/exam/avatare/Jonas.png}\hspace{-4mm}\includegraphics[width = 1.3cm]{/Users/kruppajo/work/GitHub/exam/avatare/Nilufar.png}\hspace{-4mm}\includegraphics[width = 1.3cm]{/Users/kruppajo/work/GitHub/exam/avatare/Yuki.png}
\end{minipage}
\begin{minipage}[t]{0.5\textwidth}
\hfill
\href{https://youtu.be/OhyuH6hzEhY}{\includegraphics[width = 2cm]{img/youtube}}
\end{minipage}
\fi
%% --------------------------------------------------------------------







\ifcollection
\paragraph{Ein Pfund Insekten, bitte!} 
\fi

'Das wird wohl häufiger gehört werden, wenn wir die Menschheit mit Proteinen ausreichend ernähren wollen\footnote{Die Quelle der Inspiration für die Aufgabe war der folgende Artikel aus dem Spiegel: \href{https://www.spiegel.de/ausland/ueberbevoelkerung-acht-milliarden-sind-wir-bald-zu-viele-menschen-auf-der-erde-a-3f20c7bc-3d60-4440-9f52-eb338db207f5}{Acht Milliarden - sind wir bald zu viele Menschen auf der Erde?}}.', merkt Alex an. Die Lerngruppe um Yuki, Jonas und Nilufar sind bei Alex um mal was außergewöhnliches zu essen. Um den Sinn der Nahrungsumstellung zu verdeutlichen, vergleicht Alex einmal Deutschland mit Nigeria. Nigeria hat eine der am schnellsten wachsenden Bevölkerungen der Welt und wird vermutlich im Jahr 2100 zu den Top 5 der bevölkerungsreichsten Länder zählen. Im Jahr 2023 leben ca. \ensuremath{8\times 10^{7}} Menschen in Deutschland und ca. \ensuremath{1.79\times 10^{8}} Menschen in Nigeria. Mit den Informationen wollen Sie und Alex mit der Überzeugungsarbeit anfangen und dann eine Prognose für den Fleischkonsum im Jahr 2050 zu treffen. \\ 

Im Folgenden ist Abbildung des Fleischkonsums im Jahr 2023 in Deutschland und Nigeria in [kg] einmal dargestellt. 

\begin{knitrout}
\definecolor{shadecolor}{rgb}{0.969, 0.969, 0.969}\color{fgcolor}

{\centering \includegraphics[width=\maxwidth]{img/math-14-a-1} 

}


\end{knitrout}
  
\begin{enumerate}
\item Stellen Sie den Fleischkonsum in Deutschland und Nigeria im Jahr 2023 \textit{pro Kopf} in einer aussagekräftigen Tabelle dar! \textbf{(2 Punkte)}
\item Ergänzen Sie in der Tabelle eine Spalte in der Sie für den Fleischkonsum in Nigeria auf Deutschland normieren, daher ins Verhältnis Nigeria/Deutschland, setzen! \textbf{(1 Punkt)}
\end{enumerate}

In der nächsten Abbildung finden Sie die CO$_2$-Emission in [kg] nach Lebensmittel, die durch die Produktion entsteht, abgebildet.

\begin{knitrout}
\definecolor{shadecolor}{rgb}{0.969, 0.969, 0.969}\color{fgcolor}

{\centering \includegraphics[width=\maxwidth]{img/math-14-b-1} 

}


\end{knitrout}

\begin{enumerate}
  \setcounter{enumi}{2}  
\item Stellen Sie in einer Tabelle die Treibhausgasemissionen an $CO_2$ pro Kopf, die durch den Fleischkonsum in Deutschland und Nigeria im Jahr 2023 entstehen, dar! Ergänzen Sie auch hier das Verhältnis Nigeria zu Deutschland! \textbf{(2 Punkte)}
\end{enumerate}

\newpage

In der folgenden Abbildung sehen Sie die Bevölkerungsentwicklung [Millionen] in Nigeria von 1950 bis ins Jahr 2030 fortgeführt.

\begin{knitrout}
\definecolor{shadecolor}{rgb}{0.969, 0.969, 0.969}\color{fgcolor}

{\centering \includegraphics[width=\maxwidth]{img/math-14-c-1} 

}


\end{knitrout}

\begin{enumerate}
  \setcounter{enumi}{3}  
\item Schätzen Sie graphisch die zu erwartende Bevölkerung [Millionen] in Nigeria im Jahr 2050, die sich anhand der Informationen aus der Abbildung ergibt!
\begin{enumerate}
\item Ohne Berücksichtigung der Covid-19-Pandemie! \textbf{(1 Punkt)}
\item Unter Berücksichtigung der Covid-19-Pandemie! \textbf{(1 Punkt)}
\end{enumerate}
\item Berechnen Sie den geschätzten Fleischkonsum von Nigeria im Jahr 2050 unter der Annahme 80\%-iger Angleichung der Lebensbedingungen zu Deutschland im Jahr 2023! \textbf{(1 Punkt)}
\item Berechnen Sie die prozentuale Steigerung der Treibhausgasemissionen an $CO_2$ in Nigeria im Jahr 2050 im Vergleich zum Jahr 2023, der sich durch den angeglichenen Fleischkonsum ergibt! \textbf{(1 Punkt)}
\item Berechnen Sie die prozentuale Steigerung der Treibhausgasemissionen an $CO_2$ in Nigeria, wenn die gesamte Proteinaufnahme durch Insekten ersetzt würde! \textbf{(1 Punkt)}
\end{enumerate}







 
\clearpage
% -----------------------------------------------------------------------

\section{Aufgabe \hfill (12 Punkte)}

\textit{Geben Sie grundsätzlich Formeln und Rechenweg zur Lösung der Teilaufgaben mit an!} \\[1Ex]
 

 
%% --------------------------------------------------------------------
\ifcollection
\begin{flushright}
\tiny\vspace{-3Ex}
\textbf{\examinhaltstart}
\exammodulemathstat
\vspace{-4Ex}
\end{flushright}
\begin{minipage}[t]{0.5\textwidth}
\includegraphics[width = 1.3cm]{/Users/kruppajo/work/GitHub/exam/avatare/Mark.png}\hspace{-4mm}\includegraphics[width = 1.3cm]{/Users/kruppajo/work/GitHub/exam/avatare/Nilufar.png}
\end{minipage}
\begin{minipage}[t]{0.5\textwidth}
\hfill
\href{https://youtu.be/flRBo1FWQy0}{\includegraphics[width = 2cm]{img/youtube}}
\end{minipage}
\fi
%% --------------------------------------------------------------------

%% --------------------------------------------------------------------
{\tiny\textbf{Stichworte:} Immunsystem -- Muskel vs. Interpol $\bullet$ Inzidenz $\bullet$ Prävalenz}
%% --------------------------------------------------------------------




\ifcollection
\paragraph{Tödliche Seuche AIDS -- Die rätselhafte Krankheit}
\fi



Irritiert legt Nilufar die historische Ausgabe des Spiegels aus den 80zigern beiseite. Nilufar und Mark sind bei ihrem Augenarzt und wollen einen AIDS-Test machen lassen. Woanders leider keinen Termin gekriegt... Immerhin denken die beiden über Nachwuchs nach und da geht es eben nur durch ungeschützten Sex. Was wissen Nilufar und Mark nun aber über AIDS und dem diagnostischen AIDS-Test, den die beiden nun machen werden? Leider zu wenig. Da brauchen dann Nilufar und Mark mal wieder Ihre Hilfe bei der Interpretation eines diagnostischen Tests!\\

Die Prävalenz von AIDS bei einem Menschen in Europa wird mit 0.01\% angenommen. In 92\% der Fälle ist ein HIV-Test positiv, wenn der Patient erkrankt ist. In 2.5\% der Fälle ist ein HIV-Test positiv, wenn der Patient \textit{nicht} erkrankt ist und somit gesund ist. Sie stutzen. Wie wahrscheinlich ist es denn eigentlich an AIDS erkrankt zu sein $(K^+)$, wenn Sie einen positiven AIDS-Test vorliegen haben $(T^+)$? Gehen Sie für die folgenden Berechnungen von $n = \ensuremath{10^{4}}$ Patienten mit einem diagnostischen Test für AIDS aus. Sie nehmen sich also einen Kuli und fangen an auf der historischen Ausgabe des Spiegels zu rechnen\footnote{Die Quelle der Inspiration für die Aufgabe war der folgende wissenschaftlicher Artikel: \textit{Binder et al. (2022) Von Baumdiagrammen über Doppelbäume zu Häufigkeitsnetzen -- kognitive überlastung oder didaktische Unterstützung? Journal für Mathematik-Didaktik, 1-33}}.


\begin{enumerate}
\item Welche Wahrscheinlichkeit $Pr$ wollen Sie berechnen? \textbf{(1 Punkt)}
\item Zeichnen Sie einen Häufigkeitsdoppelbaum zur Bestimmung der gesuchten Wahrscheinlichkeit $Pr$! \textbf{(2 Punkte)} \item Beschriften Sie den Häufigkeitsdoppelbaum, mit denen Ihnen bekannten Informationen zu der AIDS Erkrankung und dem AIDS-Test! \textbf{(1 Punkt)}
\item Füllen Sie den Häufigkeitsdoppelbaum mit den sich ergebenden, absoluten Patientenzahlen $n$ aus! \textbf{(2 Punkte)}
\item Berechnen Sie die gesuchte Wahrscheinlichkeit $Pr$! \textbf{(1 Punkt)}
\end{enumerate}

Bei dem folgenden Arztgespräch erfahren Nilufar und Mark, dass beim diagnostischen Testen \textit{True Positives (TP)}, \textit{True Negatives (TN)}, \textit{False Positives (FP)} und \textit{False Negatives (FN)} auftreten. Das verstehen beiden so noch nicht und deshalb stellen Sie für Nilufar und Mark den Zusammenhang in einer 2x2 Kreuztabelle dar.

\begin{enumerate}
  \setcounter{enumi}{5}
\item Tragen Sie \textit{TP}, \textit{TN}, \textit{FP} und \textit{FN} in eine 2x2 Kreuztablle ein. Beschriften Sie die Tabelle entsprechend! \textbf{(1 Punkt)}
\item Berechnen Sie die Sensitivität und Spezifität des diagnostischen Tests für AIDS! Füllen Sie dafür die 2x2 Kreuztabelle mit den Informationen aus dem Häufigkeitsdoppelbaum aus! \textbf{(2 Punkte)}
\item Was beschreibt die Sensitivität und die Spezifität im Bezug auf die Gesunden und Kranken? Stellen Sie beide diagnostische Maßzahlen als Wahrscheinlichkeiten $Pr$ dar! \textbf{(2 Punkte)} 
\end{enumerate}



 
\clearpage
% -----------------------------------------------------------------------

\section{Aufgabe \hfill (12 Punkte)}

\textit{Geben Sie grundsätzlich Formeln und Rechenweg zur Lösung der Teilaufgaben mit an!} \\[1Ex]
 

 
%% --------------------------------------------------------------------
\ifcollection
\begin{flushright}
\tiny\vspace{-3Ex}
\textbf{\examinhaltstart}
\exammodulemathstat
\vspace{-4Ex}
\end{flushright}
\begin{minipage}[t]{0.5\textwidth}
\includegraphics[width = 1.3cm]{/Users/kruppajo/work/GitHub/exam/avatare/Jessica.png}\hspace{-4mm}\includegraphics[width = 1.3cm]{/Users/kruppajo/work/GitHub/exam/avatare/Paula.png}\hspace{-4mm}\includegraphics[width = 1.3cm]{/Users/kruppajo/work/GitHub/exam/avatare/Tina.png}\hspace{-4mm}\includegraphics[width = 1.3cm]{/Users/kruppajo/work/GitHub/exam/avatare/Yuki.png}
\end{minipage}
\begin{minipage}[t]{0.5\textwidth}
\hfill
\href{https://youtu.be/SZqp_vy3rgI}{\includegraphics[width = 2cm]{img/youtube}}
\end{minipage}
\fi
%% --------------------------------------------------------------------



\ifcollection
\paragraph{Network-Marketing oder Schneeballschlacht!}
\fi



Yuki, Jessica und Paula sitzen bei Tina und hören sich etwas über Network-Marketing an. Tina ist jetzt im Network-Marketing tätig. 'Jetzt reicht es. Wir sind eine Lerngruppe und du versuchst uns hier abzuziehen!', poltert Yuki und fährt fort, 'Ich erklär dir mal, wie falsch du liegst!'. Eine Möglichkeit, leicht Geld zu verdienen, ist es anderen Menschen für Geld zu versprechen, wie man leicht reich werden kann. Am besten natürlich ohne viel Aufwand und ortsunabhängig. Schnell ein YouTube-Werbevideo gedreht und auf geht es mit unserem Network-Marketing. Aber Moment, wie funktioniert Network-Marketing eigentlich und was hat das alles mit einer Schneeballschlacht zu tun? Yuki und Sie wollen hier einmal in die Untiefen des >>passiven Einkommens<< abtauchen und die Lerngruppe vor Schlimmeren bewahren\footnote{Die Quellen der Inspiration für die Aufgabe waren folgendes YouTube Video: \href{https://youtu.be/UOKkZF_qK9M?si=uf4foJVFKfeQMwSw}{Simplicissimus -- Die meistgesuchte Betrügerin der Welt} und der Artikel: \href{https://www.deutschlandfunkkultur.de/netzwerk-marketing-die-illusion-schnell-reich-zu-werden-100.html}{Deutschlandfunk Kultur -- Die Illusion, schnell reich zu werden}}!\\

Das Jahr 2024 war das erfolgreichste Jahr in der Geschichte von Up/Down Systems and Networking (UDSysNet). Das Unternehmen steigerte den Umsatz um rund 10 Prozent von 280 Millionen Euro im Jahr 2023. Doch wie viel kommt bei den Partnern an? Laut UDSysNet habe das Unternehmen \ensuremath{3.3\times 10^{5}} aktive Partner.

\begin{enumerate}
\item Berechnen Sie zuerst den Umsatz der Firma UDSysNet im Jahr 2024! \textbf{(1 Punkt)}
\item Wie viel von dem Umsatz im Jahr 2024 wird im Durchschnitt von jedem aktiven Partner erwirtschaftet? \textbf{(1 Punkt)}
\item Welche \textit{monatlicher} Umsatz ergibt sich dadurch im Durchschnitt für jeden aktiven Partner bei einer direkten Provision von 40\%? \textbf{(1 Punkt)}
\end{enumerate}

Das von Tina zu vermarkende Produkt, hinter dem Tina voll steht, kostet 200EUR pro Einheit im Direktverkauf. Die direkte Provision für die erste Stufe beträgt 20\%. Für die zweite, dritte und vierte Stufe betragen die indirekten Provisionen jeweils 3\%, 2\% und 1.5\%. Jeder von Tina angeworbener >>Partner<< wirbt wiederum fünf Partner für sich selbst an. Pro Monat werden im Schnitt zwei Einheiten vom Produkt verkauft. Tina will nun 2100EUR im Monat \textit{passiv} -- also durch indirekte Provisionen -- erwirtschaften. Kann das klappen? Sie sind zusammen mit Yuki skeptisch.

\begin{enumerate}
  \setcounter{enumi}{3}
\item Ergänzen Sie die folgende Tabelle mit den obigen Informationen! \textbf{(2 Punkte)}
\end{enumerate}

\begin{center}
\begin{tabular}{c|c|c|c}
  \toprule
  \textbf{Stufe} & \textbf{Anzahl Partner}  & \textbf{Umsatz/Stufe} & \textbf{Provision}\\
  \midrule
  1 & Sie selber  &  & \\ \midrule
  2 &   &  &  \\ \midrule
  3 &   &  &  \\ \midrule
  4 &   &  &  \\
  \bottomrule
\end{tabular}
\end{center}

\begin{enumerate}
  \setcounter{enumi}{4}
\item Wie viele Partner müssen Sie auf der 2 Stufe anwerben um Ihr passives Einkommen durch indirekte Provision zu erreichen? Wie viele Menschen arbeiten am Ende indirekt für Sie? Stellen Sie den Zusammenhang graphisch dar!  \textbf{(3 Punkte)}
\end{enumerate}

Tina musste zum Einstieg bei UDSysNet Einheiten des Produkts für 7000EUR kaufen. Diese Einheiten kann Tina nur direkt verkaufen. Das ganze Wohnzimmer ist voll davon. Leider musste Tina den Kauf über einen Kredit über 6.2\% p.a. über 72 Monate finanzieren. Sie schütteln den Kopf und klären Tina über Zinsen auf.

\begin{enumerate}
  \setcounter{enumi}{5}
\item Berechnen Sie die Gesamtsumme, die Sie als Kredit abbezahlen müssen! \textbf{(2 Punkte)}
\item Wie viele Einheiten müssen Sie pro Monat verkaufen um die anfallenden Zinsen durch die direkte Provision zu erwirtschaften? \textbf{(1 Punkt)}
\item Wie lange in Monaten benötigen Sie um den Kredit durch die direkte Provision abzubezahlen? \textbf{(1 Punkt)}
\end{enumerate} 
\clearpage
% -----------------------------------------------------------------------

\section{Aufgabe \hfill (10 Punkte)}

\textit{Geben Sie grundsätzlich Formeln und Rechenweg zur Lösung der Teilaufgaben mit an!} \\[1Ex]
 

 
%% --------------------------------------------------------------------
\ifcollection
\begin{flushright}
\tiny\vspace{-3Ex}
\textbf{\examinhaltstart}
\exammodulemathstat
\vspace{-4Ex}
\end{flushright}
\begin{minipage}[t]{0.5\textwidth}
\includegraphics[width = 1.3cm]{/Users/kruppajo/work/GitHub/exam/avatare/Jessica.png}\hspace{-4mm}\includegraphics[width = 1.3cm]{/Users/kruppajo/work/GitHub/exam/avatare/Mark.png}\hspace{-4mm}\includegraphics[width = 1.3cm]{/Users/kruppajo/work/GitHub/exam/avatare/Paula.png}\hspace{-4mm}\includegraphics[width = 1.3cm]{/Users/kruppajo/work/GitHub/exam/avatare/Steffen.png}
\end{minipage}
\begin{minipage}[t]{0.5\textwidth}
\hfill
\href{https://youtu.be/8Pb2sKUIMyk}{\includegraphics[width = 2cm]{img/youtube}}
\end{minipage}
\fi
%% --------------------------------------------------------------------




\ifcollection
\paragraph{Höhlen \& Drachen}
\fi



Jessica, Paula und Mark sitzen bei Steffen nachdem sich alle begeistert in der Serie \textit{Stranger Thinks} verloren haben. Alle drei wollen jetzt einmal bei Steffen \textit{Höhlen \& Drachen} ausprobieren. Um Geld zu sparen, das Zeug kostet echt, wurde etwas an den Regeln gebastelt. Schnell stellen die Drei fest, dass hier ganz schön viele unterschiedliche Würfel durch die Gegend fliegen. Daher müssen Sie sich jetzt einiges an Fragen klären damit Steffen nicht so alleine ist. \\%[-1ex]

In dem Spiel hat Jessica nun auf einmal 6 sechseitige Würfel (6d6) zum würfeln in der Hand. Wenn Jessica eine 6 würfelt, hat Jessica einen Erfolg.

\begin{enumerate}
\item Berechnen Sie die Wahrscheinlichkeit \textit{genau} 5 Erfolge zu erzielen!  \textbf{(2 Punkte)}
\item Berechnen Sie die Wahrscheinlichkeit keinen Erfolg zu erzielen! \textbf{(1 Punkt)}
\end{enumerate}

Paula betrachtet nun aufmerksam die ausufernden Ausrüstungstabellen. Paula wird aber geholfen und muss sich jetzt nur zwischen der Axt oder dem Schwert entscheiden.

\begin{enumerate}
  \setcounter{enumi}{2}
\item Würden Sie die Axt mit zwei zwölfseitigen Würfeln (2d12) als Schaden oder das Schwert mit einem zwölfseitigen Würfel plus 6 (1d12+6) als Schaden bevorzugen? Begründen Sie Ihre Antwort mathematisch! \textbf{(1 Punkt)}
\end{enumerate}

Jetzt wird es immer wilder, da Paula und Mark sich jetzt überlegen müssen, wie wahrscheinlich es ist, dass der Rettungswurf gegen den zaubernden Hexer funktioniert. Paula und Mark haben folgende Wahrscheinlichkeiten gegeben. Die Wahrscheinlichkeit für das Ereignis $A$, der Rettungswurf ist erfolgreich, ist $Pr(A) = 0.65$, die Wahrscheinlichkeit für das Ereignis $B$, der Zauberwurf des Hexers ist erfolgreich, ist $Pr(B) = 0.7$. Sie haben aber mitgezählt und festgestellt, dass in $50$ von 100 Fällen der Rettungswurf bei einem erfolgeichen Zauber funktioniert hat.  

\begin{enumerate}
  \setcounter{enumi}{3}
\item Erstellen Sie eine 2x2 Kreuztabelle mit den Ereignissen $A$ und $B$ sowie den Gegenereignissen $\bar{A}$ und $\bar{B}$ mit einen $\Omega = 100$. Beachten Sie hierbei die entsprechenden Wahrscheinlichkeiten für die Ereignisse $A$ und $B$! \textbf{(2 Punkte)}
\item Bestimmen Sie $Pr(A \cap B)$! \textbf{(1 Punkt)}
\item Erstellen Sie ein Baumdiagramm mit den passenden Informationen aus der 2x2 Kreuztabelle! \textbf{(2 Punkte)}
\item Bestimmen Sie Wahrscheinlichkeit $Pr(A|B)$, dass Ihr Rettungswurf gelingt, wenn der Hexer erfolgreich gezaubert hat! \textbf{(1 Punkt)}
\end{enumerate}

  
\clearpage
% -----------------------------------------------------------------------

\section{Aufgabe \hfill (12 Punkte)}

\textit{Geben Sie grundsätzlich Formeln und Rechenweg zur Lösung der Teilaufgaben mit an!} \\[1Ex]
 

 
%% --------------------------------------------------------------------
\ifcollection
\begin{flushright}
\tiny\vspace{-3Ex}
\textbf{\examinhaltstart}
\exammodulemathstat
\vspace{-4Ex}
\end{flushright}
\begin{minipage}[t]{0.5\textwidth}
\includegraphics[width = 1.3cm]{/Users/kruppajo/work/GitHub/exam/avatare/Jonas.png}\hspace{-4mm}\includegraphics[width = 1.3cm]{/Users/kruppajo/work/GitHub/exam/avatare/Mark.png}\hspace{-4mm}\includegraphics[width = 1.3cm]{/Users/kruppajo/work/GitHub/exam/avatare/Yuki.png}
\end{minipage}
\begin{minipage}[t]{0.5\textwidth}
\hfill
\href{https://youtu.be/8Pb2sKUIMyk}{\includegraphics[width = 2cm]{img/youtube}}
\end{minipage}
\fi
%% --------------------------------------------------------------------



\ifcollection
\paragraph{Retrocheck im TV}
\fi




Das war zu viel für Jonas gestern. Die Lerngruppe mit Mark und Yuki ging viel zu lang. Während er wegdämmert, kommen in ihm seltsame Bilder hoch. 'Und hier ist sie wieder, die Show der fantastischen Preise. Seien Sie mit dabei, wenn es wieder heißt: Der Preis ist heiß!', ertönt es und Jonas fragt sich, ob er nicht doch lieber bezahlter Gast bei Barbara Salesch hätten sein sollten. Aber Jonas braucht das Geld und jetzt heißt es Spielschows farmen! Erstmal eine Kaffemaschine von Mitropa gewinnen. Ein Kandidat gewinnt die Kaffeemaschine von Mitropa, wenn nicht alle Kandidaten überbieten (eng. \textit{outbid}). Mit Ihnen bilden Mark und Yuki das Team der drei Kandidaten. Jonas braucht dringend Ihre Hilfe in seinen Wahnträumen. Sie gehen wie in \textit{Interception} rein!\footnote{\href{https://www.youtube.com/watch?v=3RhxyHMs-w8}{South Park Inception Spoof} -- Wunderbare South Park Folge}

\begin{center}
\begin{tabular}{ccc}
  \toprule
  Name & $P(win)$ & $P(outbid)$\\
  \midrule
  Mark & 0.2 & 0.12\\
  Yuki & 0.3 & 0.11 \\
  \bottomrule
\end{tabular}
\end{center}

\begin{enumerate}
\item Mit welcher Wahrscheinlichkeit gewinnen Sie die Kaffeemaschine von Mitropa, wenn keiner der Kandidaten überbietet? \textbf{(1 Punkt)}
\item Wenn Ihre überbietungswahrscheinlichkeit $P(outbid)$ bei 0.043 liegt, mit welcher Wahrscheinlichkeit gewinnt \textit{keiner} die Kaffeemaschine von Mitropa? \textbf{(1 Punkt)}
\end{enumerate}

Glücksrad für Arme auf der Kirmes! Leider hat es für Maren Gilzer nicht gereicht. Deshalb sind Sie jetzt mit Jonas auf der Kirmes und spielen mit einem einäugen Piraten um das große Geld. Das Glücksrad hat 22 Felder. Sie beide drehen das Glücksrad zweimal. Auf 6 Feldern gewinnen Jonas und Sie 3000EUR sonst 2000EUR. Ganz schön viel Geld und ganz schön zwielichtig hier...

\begin{enumerate}
  \setcounter{enumi}{2}  
\item Skizzieren Sie das Glücksrad und ergänzen Sie die Wahrscheinlichkeiten! \textbf{(1 Punkt)}
\item Zeichnen Sie das zugehörige Baumdiagramm für das zweimalige Drehen! Ergänzen Sie die Wahrscheinlichkeiten und die entsprechenden Ereignisse \textbf{(2 Punkte)}
\item Mir welcher Wahrscheinlichkeit gewinnen Sie 5000EUR? \textbf{(1 Punkt)}
\end{enumerate}

Im Fiebertraum von Jonas reisen sie beide im Zug nach Köln um bei >>Geh aufs Ganze!<< mitzuspielen. Jonas und Sie schaffen es tatsächlich ins Finale und können als Hauptgewinn ein Auto hinter einer der drei Türen gewinnen. Und was braucht man mehr als ein Auto in einem Fiebertraum? 

\begin{enumerate}
  \setcounter{enumi}{5}  
\item Bevor die Show beginnt, wird das Auto hinter eine zufällig bestimmte Tür gestellt. Mit welcher Wahrscheinlichkeit wird jeweils eine der drei Türen ausgewählt? Zeichnen Sie ein Baumdiagramm! \textbf{(1 Punkt)}
\item Mit welcher Wahrscheinlichkeit wählen Sie sofort die Tür mit dem Auto? Erweitere Sie das Baumdiagramm entsprechend! \textbf{(1 Punkt)}
\item Der Moderator öffnet nun eine der nicht gewählten Türen, aber natürlich nicht die mit dem Auto. Mit welcher Wahrscheinlichkeit steht das Auto hinter der anderen Tür? Erweitern Sie das Baumdiagramm entsprechend! \textbf{(2 Punkte)}
\item Lösen Sie nun das >>Ziegenproblem<<! Berechne Sie dazu die Wahrscheinlichkeiten der einzelnen Pfade. Lohnt sich ein Wechsel der anfangs gewählte Tür? Begründen Sie Ihre Antwort mathematisch! \textbf{(2 Punkte)}
\end{enumerate}
 

%https://unterrichten.zum.de/wiki/Laplace-Wahrscheinlichkeit_wiederholen_und_vertiefen/Ziegen

%Buchstabenhäufigkeit im Deutschen

%%https://www.mathelike.de/abiturloesungen-mathematik-bayern/mathematik-abitur-bayern-2020/pruefungsteil-a/stochastik-2/teilaufgabe-a.html

%%https://de.wikipedia.org/wiki/Buchstabenh%C3%A4ufigkeit


 
\clearpage
% -----------------------------------------------------------------------
\part{Angewandte Nutztier- und Pflanzenwissenschaften (M.Sc.)}
% -----------------------------------------------------------------------  

\section{Aufgabe \hfill (6 Punkte)}
Vergleichen Sie die Standardabweichung mit dem Standardfehler und grenzen
Sie die beiden Kennzahlen voneinander ab.



\section{Aufgabe \hfill (8 Punkte)}
Ihnen liegt folgendes Varianzanalysemodell mit der {\"u}blichen Beschreibung
zur Auswertung des Merkmals fett- und eiwei{\ss}korrigierte Milchleistung pro
Kuh und Jahr in kg vor:

\begin{equation*}
  Y_{ijkl} = \mu + Var_i + EKA_j + VarEKA_{ij} + V_k + b(L_{ij} - L) + e_{ijkl}
\end{equation*}

mit

\begin{itemize}
\item $Y_{ijkl}$: l-te Beobachtung
\item $\mu$: Populationsmittel 
\item $Var_i$: fixer Effekt der i-ten Variante (i: Kontrolle,
  Versuchsgruppe 1, Versuchsgruppe 2)
\item $EKA_j$: fixer Effekt der j-ten Erstkalbealtergruppe (j: EKA $\leq$ 25
  Monate, EKA > 25 Monate)
\item $VarEKA_{ij}$: fixer Effekt der Interaktion Variante x
  Erstkalbealtergruppe 
\item $V_k$: zuf{\"a}lliger Effekt des Vaters 
\item $b(L_{ij} - L)$: lineare Kovariable Laktationsnummer 
\item $e_{ijkl}$: zuf{\"a}lliger
  Restfehler
\end{itemize}
Erl{\"a}utern Sie anhand dieses Beispiels die Begriffe fixer Effekt,
Interaktion, zuf{\"a}lliger Effekt und Kovariable und grenzen Sie diese
Begriffe voneinander ab.



\section{Aufgabe \hfill (6 Punkte)}
Wie bestimmen Sie die richtige Stichprobengr{\"o}{\ss}e? Welche Kennzahlen /
statistische Ma{\ss}zahlen ben{\"o}tigen Sie dabei und nennen Sie die
Voraussetzungen.

 
\clearpage
% -----------------------------------------------------------------------
\part{Platzhalter}

\section{Hausarbeit \hfill (10 Punkte)}

%% --------------------------------------------------------------------
\ifcollection
\begin{flushright}
\tiny\vspace{-3Ex}
\textbf{\examinhaltstart}
\exammodulestat $\;\bullet$
\exammodulestatversuch $\;\bullet$
\exammodulelanddaten 
\end{flushright}
\hfill
\href{https://youtu.be/GNoORO7GfAE}{\includegraphics[width = 2cm]{img/caution}}
\fi
%% --------------------------------------------------------------------

\begin{graybox}{}
Teil der Prüfungsleistung in dem Modul ist eine Hausarbeit zur Datenanalyse in R.\\

Diese Aufgabe dient in der Klausur als Platzhalter für die erreichten Punkte in der Hausarbeit.\\

Zum Bestehen des Moduls ist die Abgabe des Berichts \underline{nicht} notwendig.\\

Alle Informationen zu der Hausarbeit sowie dem Stichtag der Abgabe finden Sie auf ILIAS.
\end{graybox}
 
\clearpage
% -----------------------------------------------------------------------
\part{Lösungen}

Die folgenden Lösungen sind absolut ein Entwurf und können Fehler enthalten. Daher immer mit Vorsicht genießen. Im Weiteren ist das Runden so eine Sache. Deshalb wird es vermutlich auch Abweichungen zu deiner händischen Berechnung geben. Damit leben wir dann aber alle.

% ----------------------------------------------------------------------- 
\section*{Statistik}
% ----------------------------------------------------------------------- 

\begin{verbatim}
library(tidyverse)
data_tbl <- read.table(file = pipe("pbpaste"), sep = " ", header = TRUE)
\end{verbatim}

% ----------------------------------------------------------------------- 
\section*{Mathematik}
% ----------------------------------------------------------------------- 
\subsection*{Herodot – der Schimmel aus Ivenack}
\begin{knitrout}
\definecolor{shadecolor}{rgb}{0.969, 0.969, 0.969}\color{fgcolor}\begin{table}[!h]
\centering\begingroup\fontsize{10}{12}\selectfont

\begin{tabular}{ccccc}
\toprule
\textbf{1)} & \textbf{2)} & \textbf{3)} & \textbf{4)} & \textbf{5)}\\
\midrule
$d_{1820} = 291.07cm$ & r = 0 + 2.25mm * Baumalter & $V_{eff} = 1.97m^3$ & $s_{Eiche} = 30.54cm$ & $s_{Platz} = 10.54cm$\\
\bottomrule
\end{tabular}
\endgroup{}
\end{table}

\end{knitrout}
% ----------------------------------------------------------------------- 
\subsection*{Von Töpfen auf Tischen}
\begin{knitrout}
\definecolor{shadecolor}{rgb}{0.969, 0.969, 0.969}\color{fgcolor}\begin{table}[!h]
\centering\begingroup\fontsize{10}{12}\selectfont

\begin{tabular}{ccccc}
\toprule
\textbf{2)} & \textbf{3a)} & \textbf{3b)} & \textbf{4)} & \textbf{5)}\\
\midrule
$n = 294$ & $A_{Tisch} = 0.79m^2$ & $A_{Topf} = 0.49m^2$ & $V_{Topf} = 118.22cm^3$ & $36.65EUR$\\
\bottomrule
\end{tabular}
\endgroup{}
\end{table}

\end{knitrout}
% ----------------------------------------------------------------------- 
\subsection*{Solar- \& Biogasanlagen}
\begin{knitrout}
\definecolor{shadecolor}{rgb}{0.969, 0.969, 0.969}\color{fgcolor}\begin{table}[!h]
\centering\begingroup\fontsize{10}{12}\selectfont

\begin{tabular}{cccc}
\toprule
\textbf{2)} & \textbf{3)} & \textbf{4)} & \textbf{5)}\\
\midrule
$A_{Dach} = 483.74m^2$ & $\rho = 244kg/m^3$ & $V = 48.77m^3$ & $h = 4.79m$\\
\bottomrule
\end{tabular}
\endgroup{}
\end{table}

\end{knitrout}
% ----------------------------------------------------------------------- 
\subsection*{Aligatorenbirnen und Blaubeere}
\begin{knitrout}
\definecolor{shadecolor}{rgb}{0.969, 0.969, 0.969}\color{fgcolor}\begin{table}[!h]
\centering\begingroup\fontsize{10}{12}\selectfont

\begin{tabular}{cc}
\toprule
\textbf{1)} & \textbf{2)}\\
\midrule
$17.91EUR$ & $n = 28$\\
\bottomrule
\end{tabular}
\endgroup{}
\end{table}

\begin{table}[!h]
\centering\begingroup\fontsize{10}{12}\selectfont

\begin{tabular}{cccc}
\toprule
\textbf{3)} & \textbf{Wasserverbrauch} & \textbf{Mittleres Gewicht} & \textbf{Liter pro Stück}\\
\midrule
Strauchtomate & $0.17l/g$ & $102.5g$ & $17.425l$\\
Salat & $0.1l/g$ & $415g$ & $41.5l$\\
Avocado & $1.1l/g$ & $270g$ & $297l$\\
Blaubeere & $0.82l/g$ & $3.4g$ & $2.788l$\\
\bottomrule
\end{tabular}
\endgroup{}
\end{table}

\begin{table}[!h]
\centering\begingroup\fontsize{10}{12}\selectfont

\begin{tabular}{ccccc}
\toprule
\textbf{4a)} & \textbf{4b)} & \textbf{5a)} & \textbf{5b)} & \textbf{6)}\\
\midrule
$V = 64636500m^3$ & $V = 209231000m^3$ & 4.88$\times$ & 5.3$\times$ & FAO Stat, Our World in Data\\
\bottomrule
\end{tabular}
\endgroup{}
\end{table}

\end{knitrout}
% ----------------------------------------------------------------------- 
\subsection*{Die Dampfnudelerde}
\begin{knitrout}
\definecolor{shadecolor}{rgb}{0.969, 0.969, 0.969}\color{fgcolor}\begin{table}[!h]
\centering\begingroup\fontsize{10}{12}\selectfont

\begin{tabular}{cccccc}
\toprule
\textbf{1)} & \textbf{2)} & \textbf{3)} & \textbf{4)} & \textbf{5)} & \textbf{6)}\\
\midrule
$h = 3.27 
                                            \cdot 11.5 = 37.61m$ & $V_{T} = 27.14m^3$ & $n = 2102$ & $r = 180.89m$ & $b' = 165.44m$ & $17.46Ellen$\\
\bottomrule
\end{tabular}
\endgroup{}
\end{table}

\end{knitrout}
% ----------------------------------------------------------------------- 
\subsection*{Pyramiden bauen}
\begin{knitrout}
\definecolor{shadecolor}{rgb}{0.969, 0.969, 0.969}\color{fgcolor}\begin{table}[!h]
\centering\begingroup\fontsize{10}{12}\selectfont

\begin{tabular}{cccccc}
\toprule
\textbf{1)} & \textbf{2)} & \textbf{3)} & \textbf{4)} & \textbf{5)} & \textbf{6)}\\
\midrule
$h = 3.27 
                                            \cdot 11.5 = 37.61m$ & $V_{T} = 27.14m^3$ & $n = 2102$ & $r = 180.89m$ & $b' = 165.44m$ & $17.46Ellen$\\
\bottomrule
\end{tabular}
\endgroup{}
\end{table}

\end{knitrout}
% ----------------------------------------------------------------------- 
\end{document}
% -----------------------------------------------------------------------


  
