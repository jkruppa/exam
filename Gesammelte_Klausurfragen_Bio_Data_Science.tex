\documentclass[a4paper, 9pt]{scrartcl}\usepackage[]{graphicx}\usepackage[]{xcolor}
% maxwidth is the original width if it is less than linewidth
% otherwise use linewidth (to make sure the graphics do not exceed the margin)
\makeatletter
\def\maxwidth{ %
  \ifdim\Gin@nat@width>\linewidth
    \linewidth
  \else
    \Gin@nat@width
  \fi
}
\makeatother

\definecolor{fgcolor}{rgb}{0.345, 0.345, 0.345}
\newcommand{\hlnum}[1]{\textcolor[rgb]{0.686,0.059,0.569}{#1}}%
\newcommand{\hlsng}[1]{\textcolor[rgb]{0.192,0.494,0.8}{#1}}%
\newcommand{\hlcom}[1]{\textcolor[rgb]{0.678,0.584,0.686}{\textit{#1}}}%
\newcommand{\hlopt}[1]{\textcolor[rgb]{0,0,0}{#1}}%
\newcommand{\hldef}[1]{\textcolor[rgb]{0.345,0.345,0.345}{#1}}%
\newcommand{\hlkwa}[1]{\textcolor[rgb]{0.161,0.373,0.58}{\textbf{#1}}}%
\newcommand{\hlkwb}[1]{\textcolor[rgb]{0.69,0.353,0.396}{#1}}%
\newcommand{\hlkwc}[1]{\textcolor[rgb]{0.333,0.667,0.333}{#1}}%
\newcommand{\hlkwd}[1]{\textcolor[rgb]{0.737,0.353,0.396}{\textbf{#1}}}%
\let\hlipl\hlkwb

\usepackage{framed}
\makeatletter
\newenvironment{kframe}{%
 \def\at@end@of@kframe{}%
 \ifinner\ifhmode%
  \def\at@end@of@kframe{\end{minipage}}%
  \begin{minipage}{\columnwidth}%
 \fi\fi%
 \def\FrameCommand##1{\hskip\@totalleftmargin \hskip-\fboxsep
 \colorbox{shadecolor}{##1}\hskip-\fboxsep
     % There is no \\@totalrightmargin, so:
     \hskip-\linewidth \hskip-\@totalleftmargin \hskip\columnwidth}%
 \MakeFramed {\advance\hsize-\width
   \@totalleftmargin\z@ \linewidth\hsize
   \@setminipage}}%
 {\par\unskip\endMakeFramed%
 \at@end@of@kframe}
\makeatother

\definecolor{shadecolor}{rgb}{.97, .97, .97}
\definecolor{messagecolor}{rgb}{0, 0, 0}
\definecolor{warningcolor}{rgb}{1, 0, 1}
\definecolor{errorcolor}{rgb}{1, 0, 0}
\newenvironment{knitrout}{}{} % an empty environment to be redefined in TeX

\usepackage{alltt}
\usepackage[ngerman]{babel}

% -----------------------------------------------------------------------


% -----------------------------------------------------------------------
%% ------------------------------------------------------------
%% by J.Kruppa on Friday, February 11, 2022 (11:31)
%% \def\mainDir{\Sexpr{exam_path}}
\def\source{/Users/jokruppa/source/tex}
\usepackage[margin=2cm, includefoot]{geometry}
\setlength{\parindent}{0cm}
\usepackage{booktabs}
\usepackage{amsmath}
\usepackage{scalerel,amssymb}
\usepackage{setspace}
\def\csquare{{\Large $\boxtimes$}}
\def\msquare{{\Large $\square$}}
\usepackage[normalem]{ulem}
\usepackage{array}
\usepackage{xcolor}
\usepackage{float}
\usepackage{currfile}
\usepackage{tikz}
\usepackage[nomessages]{fp}

%% beamer defs
\def\lecture{Klausurfragen der Bio Data Science}

%% exam defs
\def\examtitle{\lecture}
\def\exammodule{
\vspace{-1.75cm}  
\begin{graybox}{}
\vspace{2Ex}
\textbf{\large Name:} \rule[0ex]{16.75em}{.4pt}
\hfill \textnormal{\textit{Nicht bestanden:}} \msquare \\[2.5Ex]
\textbf{\large Vorname:} \rule[0ex]{15em}{.4pt} \\[2.5Ex]
\textbf{\large Matrikelnummer:} \rule[0ex]{10.8em}{.4pt}
\hfill Endnote: \rule[0ex]{7em}{.4pt} 
\end{graybox}
\vspace{3Ex}
\phantom{text}
}
\def\examsemester{Sommersemester \& Wintersemester}
\def\examdate{\today}
%% ------------------------------------------------------------
\definecolor{darkblue}{rgb}{0,0,.5}
\definecolor{darkpurple}{rgb}{0.4117, 0.2, 0.4117}
\definecolor{uni}{rgb}{0,0.3137,0.6078}
\definecolor{gray}{gray}{0.7}

\usepackage{tcolorbox}
\definecolor{logo1}{RGB}{0, 158, 227}
\definecolor{gray5}{RGB}{247, 247, 247}
\definecolor{gray2}{RGB}{102, 102, 102}

\newtcolorbox{graybox}[1]{
  colback=gray5,%%red!5!white,
  colframe=gray2,%%red!75!black,
  fonttitle=\bfseries\Large,
  %%valign=center,
  fontupper=\large,
  before skip=10pt plus 2pt,
  after skip=20pt plus 4pt,
  title=#1}

\newtcolorbox{takehomebox}[1]{
  colback=gray5,%%red!5!white,
  colframe=logo1,%%red!75!black,
  fonttitle=\bfseries\Large,
  %%valign=center,
  fontupper=\large,
  before skip=10pt plus 2pt,
  after skip=10pt plus 2pt,
  title=#1}

\def\Rlogo{\includegraphics[width = 0.5cm]{\string~/Documents/GitHub/exam/img/Rlogo}\;}

\usepackage[scaled=.90]{helvet} 
\usepackage{fancyhdr}
\usepackage{lastpage}
\usepackage{hyperref}
\hypersetup{
    colorlinks=true,       % false: boxed links; true: colored links
    linkcolor=black,          % color of internal links 
    urlcolor=magenta           % color of external links
}
\renewcommand{\familydefault}{\sfdefault}

\title{
\large \exammodule \\[5Ex]
\Huge \examtitle \\[2Ex] 
\Large Hochschule Osnabr{\"u}ck
}
\author{Pr{\"u}fer: Prof. Dr. Jochen Kruppa \\
Fakult{\"a}t f{\"u}r Agrarwissenschaften und Landschaftsarchitektur \\ 
j.kruppa@hs-osnabrueck.de}
\date{Version vom \examdate}

%% ------------------------------------------------------------
%% by J.Kruppa on Tuesday, September 23, 2014 (12:50)
%% Header
\renewcommand{\headrulewidth}{0pt}
\renewcommand{\footrulewidth}{0pt}
\pagestyle{fancy}

\fancyhf{}
\fancyhead[L]{}
\fancyhead[R]{}
\fancyfoot[R]{\thepage}
\fancyfoot[L]{\footnotesize \examtitle}

\fancypagestyle{empty}{
 \fancyhf{}
 \fancyhead[L]{}
 \fancyhead[R]{}
 \fancyfoot[R]{\thepage}
 \fancyfoot[L]{\footnotesize \examtitle}
}

\usepackage{arevtext,arevmath}

\newcommand\Tstrut{\rule{0pt}{2.6ex}}         % = `top' strut
\newcommand\Bstrut{\rule[-0.9ex]{0pt}{0pt}}   % = `bottom' strut
\def\strut{\Tstrut\Bstrut}

% -----------------------------------------------------------------------
\IfFileExists{upquote.sty}{\usepackage{upquote}}{}
\begin{document}
\date{
\vspace{1.5Ex}
\textbf{\Large\textcolor{red}{>>\underline{\underline{\examsemester}}<<}}
\vfill
\begin{center}
\includegraphics[width = 1.9cm]{avatare/Alex}\hspace{-8mm}
\includegraphics[width = 1.9cm]{avatare/Jessica}\hspace{-8mm}
\includegraphics[width = 1.9cm]{avatare/Jonas}\hspace{-8mm}
\includegraphics[width = 1.9cm]{avatare/Mark}\hspace{-8mm}
\includegraphics[width = 1.9cm]{avatare/Nilufar}\hspace{-8mm}
\includegraphics[width = 1.9cm]{avatare/Paula}\hspace{-8mm}
\includegraphics[width = 1.9cm]{avatare/Steffen}\hspace{-8mm}
\includegraphics[width = 1.9cm]{avatare/Tina}\hspace{-8mm}
\includegraphics[width = 1.9cm]{avatare/Yuki}\\
\small
\vspace{1.5Ex}
\textit{"`The test of a student is not how much she or he knows,\\ but how much the student wants to know."'\\ --- Alice W. Rollins}
\end{center}}
% -----------------------------------------------------------------------
\maketitle
\fancypagestyle{empty}{
  \fancyfoot[L]{\tiny $\square\!\square\!\square\!\blacksquare\!\square\!\blacksquare\!\blacksquare\!\blacksquare\!\blacksquare\!\square\!\blacksquare\!\square\!\square\!\blacksquare\!\blacksquare\!\blacksquare\!\blacksquare\!\square\!\blacksquare\!\blacksquare$}
}
\thispagestyle{empty}
\clearpage
% -----------------------------------------------------------------------
\begin{minipage}[c]{0.125\textwidth}
\includegraphics[width = 1.9cm]{avatare/Alex}
\end{minipage}
\begin{minipage}[c]{0.875\textwidth}
\textit{Alex studiert im 3. Semester und wiederholt das Modul, da er im ersten Jahr andere Prioritäten für sich gesetzt hat. Das musste sein, da er sich ziemlich im Abitur verausgabt hat. Darüber hinaus war die WG auch eher auf Party angelegt. Alex hofft jetzt über Pünktlichkeit wieder in die Bahn zu kommen. Dafür steht er jetzt immer um 5 Uhr auf! Freunde von ihm beschreiben ihn eher als extrovertiert. Er kennt Paula noch aus der Schulzeit und er überlegt, ob nicht beide Mal nach Mallorca sollten.} 
\end{minipage}\\[2.75Ex]
% -----------------------------------------------------------------------
\begin{minipage}[c]{0.875\textwidth}
\textit{Nach zwei Semestern Studium an der Universität Osnabrück war es dann Jessica doch viel zu theoretisch. Etwas angewandtes sollte es sein, wo sie auch eine Fähigkeit lernt, die frau nutzen kann. Deshalb hat sich Jessica an der Hochschule eingeschrieben. Hoffentlich lernt sie etwas nützliches, wo andere für Geld geben würden. Wer nützlich ist, ist wertvoll. Ihr Traum ist ja eine Hundeschule aufzumachen. Die großen Parties hat sie immer gemieden. Sie ist lieber mit ihrer Hündin im Teuteburgerwald.}
\end{minipage}
\begin{minipage}[c]{0.125\textwidth}
\includegraphics[width = 1.9cm]{avatare/Jessica}
\end{minipage}\\[2.75Ex]
% -----------------------------------------------------------------------
\begin{minipage}[c]{0.125\textwidth}
\includegraphics[width = 1.9cm]{avatare/Jonas}
\end{minipage}
\begin{minipage}[c]{0.875\textwidth}
\textit{Das ist jetzt der letzte Versuch mit einem Studium. Wenn es nicht klappt dann überlegt Jonas das \href{https://www.ihk.de/osnabrueck/aus-und-weiterbildung/ausbildung/ausbildungsbetriebe/projekt-neustart-1087206}{Programm der IHK zu Ausbildungsvermittlung} zu nutzen. Aber eine Runde gibt er sich noch. Struktur ist eigentlich das Wichtigste und diesmal hat er sich alle Altklausuren der Fachschaft besorgt. Dann ist er auch noch gleich der Fachschaft beigetreten um mehr Informationen abzugreifen. Und er versucht nicht seine Zeit mit Alex zu verdaddeln oder in der Fachschaft bei einem Bier oder so...}
\end{minipage}\\[2.75Ex]
% -----------------------------------------------------------------------
\begin{minipage}[c]{0.875\textwidth}
\textit{Nächstes Semester geht es nach Kanada davon hat er schon auf der Berufsschule geträumt. Deshalb konzentriert er sich sehr auf die Prüfungen. Ein Schiff ist im Hafen sicher, aber dafür ist es nicht gebaut worden. Das \href{https://www.hs-osnabrueck.de/wir/fakultaeten/aul/international/}{International Faculty Office} der Fakultät Agrarwissenschaften und Landschaftsarchitektur hat super geholfen, aber es waren einiges an Unterlagen. Jetzt hofft er, dass Tina dann doch noch mitkommt. Aber sonst macht er das eben alleine. Obwohl das eher nicht so seine Art ist. Vielleicht sollte er sich mal einen Tipp bei Tina holen, sie wirkt sehr entschlossen.} 
\end{minipage}
\begin{minipage}[c]{0.125\textwidth}
\includegraphics[width = 1.9cm]{avatare/Mark}
\end{minipage}\\[2.75Ex]
% -----------------------------------------------------------------------
\begin{minipage}[c]{0.125\textwidth}
\includegraphics[width = 1.9cm]{avatare/Nilufar}
\end{minipage}
\begin{minipage}[c]{0.875\textwidth}
\textit{Nach der Ausbildung wollte Nilufar eigentlich gleich anfangen zu arbeiten, aber nach einem Praktikum und der Probezeit stellte sie fest, dass es ohne einen Hochschulabschluss schwer wird Führungsverantwortung zu übernehmen. Mit Menschen kann sie schon immer und dann auch eigene Projekte mit anderen verwirklichen, dass ist doch was. Mit dem notwendigen Abschluss sollte der Start um so einfacher sein. Dann ist sie keine Befehlsempfängerin mehr sondern gibt die Marschrichtung vor. Schon jetzt koordiniert Nilufar das Studium von anderen.}
\end{minipage}\\[2.75Ex]
% -----------------------------------------------------------------------
\begin{minipage}[c]{0.875\textwidth}
\textit{Paula möchte die Welt zu einem besseren Ort machen. Wenn da nicht die anderen Mitmenschen wären. Paula muss das Modul nochmal wiederholen, da es dann am Ende des Semesters zu viel für sie wurde. Eine Lerngruppe hätte geholfen, aber das ist dann gar nicht so einfach eine zu finden. Zwar kennt sie schon Nilufar, aber Nilufar ist ihr manchmal zu forsch. Ihr schwant aber, dass alleine das Studium sehr schwer werden wird. Das Abitur war schon so ein Lernhorror, das möchte sie nicht nochmal. Alex sieht sie da als Vorbild.}
\end{minipage}
\begin{minipage}[c]{0.125\textwidth}
\includegraphics[width = 1.9cm]{avatare/Paula}
\end{minipage}\\[2.75Ex]
% -----------------------------------------------------------------------
\begin{minipage}[c]{0.125\textwidth}
\includegraphics[width = 1.9cm]{avatare/Steffen}
\end{minipage}
\begin{minipage}[c]{0.875\textwidth}
\textit{Sommer, Sonne, Natur. Das ist es was Steffen mag. Raus in die Komune und die Natur genießen. Leider hat Steffen noch andere Bedürfnisse, die ein Einkommen benötigen. Da Studierte mehr verdienen, würde dann in Teilzeit auch mehr rausspringen. Wenn er dann privat was anbauen kann, dann spart er gleich doppelt. Leider sind viele seiner Kommilitonen total verkrampfte Karrieristen. Es geht nur ums Äußere. Dabei verliert sich Steffen gerne im Prozess. Das hat auch seinen Schulabschluss etwas verzögert. Steffen lässt sich eben Zeit.}
\end{minipage}\\[2.75Ex]
% -----------------------------------------------------------------------
\begin{minipage}[c]{0.875\textwidth}
\textit{Wille  war es, die es Tina hierher gebracht hat und Wille wird es sein, die Tina dann auch zum Abschluß treibt. Nach einem Rückschlag muss Tina jetzt einige Module wiederholen, damit sie dann auch fertig wird. Ab und zu ist sie schwach gewesen und das hat dann Zeit gekostet. Das Tina es dann manchmal übertreibt, weiß sie nur zu gut, aber irgendwie muss die Kontrolle ja erhalten bleiben? Insbesondere, wenn sie mal wieder die Nacht durchgefeiert hat, verachtet Tina sich. Dann baut Nilufar sie dann bei einem Tee wieder auf.}
\end{minipage}
\begin{minipage}[c]{0.125\textwidth}
\includegraphics[width = 1.9cm]{avatare/Tina}
\end{minipage}\\[2.75Ex]
% -----------------------------------------------------------------------
\begin{minipage}[c]{0.125\textwidth}
\includegraphics[width = 1.9cm]{avatare/Yuki}
\end{minipage}
\begin{minipage}[c]{0.875\textwidth}
\textit{Für Yuki war es nicht einfach. Teilweise war die Krankheit sehr hinderlich, dann war es Yuki selber. Dann muss man auch wieder auf die Beine kommen und es dauert eben seine Zeit. Aber immerhin hat Yuki es jetzt den Abschluss gekriegt und hat einen Studienplatz. Jetzt heißt es in den Rhythmus kommen und schauen, was noch so passiert. Immerhin hat Yuki schon eine kleine Gruppe gefunden, in der Yuki dann Hilfe findet. Ist aber auch sehr unübersichtlich so ein Studium. Steffen ist immer super entspannt.}
\end{minipage}
\clearpage
% -----------------------------------------------------------------------


\begin{graybox}{Erlaubte Hilfsmittel}
  \vspace{1Ex}
  \begin{itemize}
  \item Normaler Taschenrechner ohne Möglichkeit der Kommunikation mit anderen
    Geräten! Ausdrücklich kein Handy!
  \item Eine DIN A4-Seite als beidseitig, selbstgeschriebene,
    handschriftliche Formelsammlung. Keine digitalen Ausdrucke! 
  \item \textbf{\textcolor{red}{Die Verwendung eines roten Farbstiftes ist nicht gestattet! Korrekturfarbe!}}
  \item \textit{You can answer the questions in English without any consequences.}  
  \end{itemize}
\end{graybox}
\vfill

\begin{graybox}{Endnote}
  \vspace{1Ex}
  \begin{itemize}
  \item[] \rule[0ex]{3em}{.4pt}\, von 20\, Punkten sind aus den Multiple
    Choice Aufgaben erreicht.
  \item[] \rule[0ex]{3em}{.4pt}\, von 74 Punkten sind aus den Rechen- und
    Textaufgaben erreicht. 
  \item[] \rule[0ex]{3em}{.4pt}\, von 94 Punkten in Summe.
  \item[] Es wird folgender Notenschlüssel angewendet.   
  \end{itemize}
  \vspace{1ex}
\begin{center}
  \begin{tabular}[c]{cc}
    \toprule
    \textbf{Punkte}	&	\textbf{Note}	\\
    \midrule
    90.0 - 94.0	&	1,0	\\
    85.0 - 89.5	&	1,3	\\
    80.5 - 84.5	&	1,7	\\
    75.5 - 80.0	&	2,0	\\
    71.0 - 75.0	&	2,3	\\
    66.5 - 70.5	&	2,7	\\
    61.5 - 66.0	&	3,0	\\
    57.0 - 61.0	&	3,3	\\
    52.0 - 56.5	&	3,7	\\
    47.0 - 51.5	&	4,0	\\
    \bottomrule
  \end{tabular}
\end{center}
  \vspace{1ex}
\begin{itemize}
\item[] Es ergibt sich eine Endnote von \rule[0ex]{4em}{.4pt}.
\end{itemize}
  \vspace{1Ex}
\end{graybox}

% -----------------------------------------------------------------------
\newpage
% -----------------------------------------------------------------------

\begin{graybox}{Multiple Choice Aufgaben}
  \begin{itemize}
  \item Pro Multipe Choice Frage ist \emph{genau} eine Antwort richtig.
  \item \textbf{Übertragen Sie Ihre Kreuze in die Tabelle auf
      dieser Seite.}
  \end{itemize}

\begin{center}
  \large
  \begin{tabular}{|l|c|c|c|c|c?c|}
    \hline
    & \textbf{A} & \textbf{B} & \textbf{C} & \textbf{D} & \textbf{E} & $\boldsymbol{\checkmark}$\strut\\
    \hline
    \textbf{Aufgabe 1} &   &   &   &   &   & \strut\\
    \hline
    \textbf{Aufgabe 2} &   &   &   &   &   & \strut\\
    \hline
    \textbf{Aufgabe 3} &   &   &   &   &   & \strut\\
    \hline
    \textbf{Aufgabe 4} &   &   &   &   &   & \strut\\
    \hline
    \textbf{Aufgabe 5} &   &   &   &   &   & \strut\\
    \hline
    \textbf{Aufgabe 6} &   &   &   &   &   & \strut\\
    \hline
    \textbf{Aufgabe 7} &   &   &   &   &   & \strut\\
    \hline
    \textbf{Aufgabe 8} &   &   &   &   &   & \strut\\
    \hline
    \textbf{Aufgabe 9} &   &   &   &   &   & \strut\\
    \hline
    \textbf{Aufgabe 10} &   &   &   &   &   & \strut\\
    \hline
  \end{tabular}
\end{center}

\begin{itemize}
\item Es sind \rule[0ex]{2em}{.4pt}\, von 20 Punkten erreicht worden.
\end{itemize}
\end{graybox}

\vfill

\begin{graybox}{Rechen- und Textaufgaben}
  \begin{center}
    \large
    \begin{tabular}{|l|c|c|c|c|c|c|c|}
      \hline
      \textbf{Aufgabe} & \textbf{11} & \textbf{12} & \textbf{13} & \textbf{14} & \textbf{15} & \textbf{16} & \textbf{17} \strut\\
      \hline
      \textbf{Punkte} & 
      \hspace{1Ex}\Large\textcolor{gray!70}{9}\hspace{1Ex}  & 
      \hspace{1Ex}\Large\textcolor{gray!70}{12}\hspace{1Ex}  & 
      \hspace{1Ex}\Large\textcolor{gray!70}{10}\hspace{1Ex}  & 
      \hspace{1Ex}\Large\textcolor{gray!70}{11}\hspace{1Ex}  & 
      \hspace{1Ex}\Large\textcolor{gray!70}{12}\hspace{1Ex}  & 
      \hspace{1Ex}\Large\textcolor{gray!70}{10}\hspace{1Ex}  & 
      \hspace{1Ex}\Large\textcolor{gray!70}{10}\hspace{1Ex} \strut\\
      \hline
  \end{tabular}
\end{center}
\begin{itemize}
\item Es sind \rule[0ex]{2em}{.4pt}\, von 74 Punkten erreicht worden.
\end{itemize}
\end{graybox}

% -----------------------------------------------------------------------
\clearpage
% -----------------------------------------------------------------------
\begin{graybox}{Multiple Choice Aufgaben}
Die Multiple Choice Aufgaben \textcolor{red}{unterliegen dem Zufall}. Die Reihenfolge der Antworten ist zufällig. Die Fragen und Antworten sind semantisch zufällig und haben somit \textcolor{red}{verschiedene Textvarianten}. Insbesondere die reinen Textaufgaben haben verschiedene Textvarianten. Die Semeantik mag sich unterscheiden, die Inhalte sind aber gleich.
\end{graybox}
\section*{Programmieren in R} 

\section{Aufgabe \hfill (2 Punkte)}

%% --------------------------------------------------------------------
\ifcollection
\begin{flushright}
\tiny\vspace{-2Ex}
\textbf{\examinhaltstart}
\exammodulemathstat $\;\bullet$
\exammodulestat $\;\bullet$
\exammodulestatversuch $\;\bullet$
\exammodulebiostat
\vspace{-1Ex}
\end{flushright}
\fi
%% --------------------------------------------------------------------




Sie wollen Ihren Datensatz in \Rlogo einlesen und stehen nun vor einem Problem. Sie stellen fest, dass die Hilfeseiten alle in englischer Sprache verfasst sind. Warum mag die Nutzung von Deutsch problematisch sein?



\begin{enumerate}
\item [\textbf{A} \msquare] Programmiersprachen haben Probleme mit Umlauten und Sonderzeichen der deutschen Sprache. Daher ist die Nutzung in Deutsch in den AGBs von \Rlogo untersagt.
\item [\textbf{B} \msquare] Es gibt keinen Grund nicht auch deutsche Wörter zu verwenden. Es ist ein Stilmittel.
\item [\textbf{C} \msquare] Programmiersprachen können nur englische Begriffe verarbeiten. Zusätzliche Pakete können zwar geladen werden, aber meist funktionieren diese Pakete nicht richtig. Deutsch ist International nicht bedeutend genug.
\item [\textbf{D} \msquare] \Rlogo Pakete sind nur in englischer Sprache verfasst. Es macht keinen Sinn \Rlogo daher in Deutsch zu bedienen.
\item [\textbf{E} \msquare] Programmiersprachen haben Probleme mit Umlauten und Sonderzeichen der deutschen Sprache. Die Nutzung von englischer Sprache umgeht dieses Problem in eleganter Art.
\end{enumerate} 

\section{Aufgabe \hfill (2 Punkte)}

%% --------------------------------------------------------------------
\ifcollection
\begin{flushright}
\tiny\vspace{-2Ex}
\textbf{\examinhaltstart}
\exammodulemathstat $\;\bullet$
\exammodulestat $\;\bullet$
\exammodulestatversuch $\;\bullet$
\exammodulebiostat
\vspace{-1Ex}
\end{flushright}
\fi
%% --------------------------------------------------------------------




In \Rlogo müssen die Daten eines Experiments in einem bestimmten Format vorliegen. Wir sprechen auch von dem log-Format der Daten. Welche Aussage zu dem Long-Format ist richtig?



\begin{enumerate}
\item [\textbf{A} \msquare] In den Zeilen finden sich die experimentellen Faktoren ($X$) sowie die Messwerte ($Y$). In den Spalten finden sich dann die einzelnen Beobachtungen.
\item [\textbf{B} \msquare] In den Spalten sind die Beobachtungen in den Zeilen die Variablen, wie die Messwerte und experiementellen Faktoren.
\item [\textbf{C} \msquare] Das Long-Format beschreibt in den Spalten die Beobachtungen sowie in den Zeilen die \textit{unabhängigen} Beobachtungen.
\item [\textbf{D} \msquare] In den Spalten finden sich die experimentellen Faktoren ($X$) sowie die Messwerte ($Y$). In den Zeilen finden sich dann die einzelnen Beobachtungen.
\item [\textbf{E} \msquare] In den Spalten finden sich die einzelnen Beobachtungen und in den Zeilen die gemessenen Variablen ($Y$) sowie die experimentellen Faktoren ($X$).
\end{enumerate} 

\section{Aufgabe \hfill (2 Punkte)}

%% --------------------------------------------------------------------
\ifcollection
\begin{flushright}
\tiny\vspace{-2Ex}
\textbf{\examinhaltstart}
\exammodulemathstat $\;\bullet$
\exammodulestat $\;\bullet$
\exammodulestatversuch $\;\bullet$
\exammodulebiostat
\vspace{-1Ex}
\end{flushright}
\fi
%% --------------------------------------------------------------------



In Ihrer Abschlussarbeit wollen Sie zu Beginn eine explorativen Datenanalyse (EDA) in \Rlogo rechnen. Dafür gibt es eine generelle Abfolge von Prozessschritten. Welche ist hierbei die richtige Reihenfolge?



\begin{enumerate}
\item [\textbf{A} \msquare] Wir lesen als erstes die Daten über \texttt{read\_excel()} ein, transformieren die Spalten über \texttt{mutate()} in die richtige Form und können dann  über \text{ggplot()} uns die Abbildungen erstellen lassen. Wichtig ist, dass wir keine Faktoren sondern nur numerische Variablen vorliegen haben.
\item [\textbf{B} \msquare] Für eine explorativen Datenanalyse (EDA) in \Rlogo müssen wir als erstes die Daten über \texttt{read\_excel()} einlesen. Danach müssen wir schauen, dass wir die Zeilen richtig über \texttt{mutate()} transformiert haben. Insbesondere müssen Variablen mit kontinuierlichen Werten in einen Faktor umgewandelt werden. Am Ende nutzen wir die Funktion \text{ggplot()} für die eigentlich EDA.
\item [\textbf{C} \msquare] Für eine explorativen Datenanalyse (EDA) in \Rlogo müssen wir als erstes die Daten über \texttt{read\_excel()} einlesen. Danach müssen wir schauen, dass wir die Spalten richtig über \texttt{mutate()} transformiert haben. Insbesondere müssen Variablen mit Kategorien in einen Faktor umgewandelt werden. Am Ende nutzen wir die Funktion \text{ggplot()} für die eigentlich EDA.
\item [\textbf{D} \msquare] Wir lesen die Daten über eine generische Funktion \texttt{read()} ein und müssen dann die Funktion \texttt{ggplot()} nur noch installieren. Dann haben wir die Abbildungen als \texttt{*.png} vorliegen.
\item [\textbf{E} \msquare] Die Funktionsreihenfolge ist wie folgt: \texttt{read\_excel()} ->  \texttt{mutate()} -> \text{ggplot()}. Dabei ist bei der Transformation der Daten darauf zu achten, dass keine Faktoren erstellt werden.
\end{enumerate} 
\section*{Deskriptive Statistik \& Explorative Datenanalyse}

\section{Aufgabe \hfill (2 Punkte)}

%% --------------------------------------------------------------------
\ifcollection
\begin{flushright}
\tiny\vspace{-2Ex}
\textbf{\examinhaltstart}
\exammodulemathstat $\;\bullet$
\exammodulestat $\;\bullet$
\exammodulestatbbv 
\vspace{-1Ex}
\end{flushright}
\fi
%% --------------------------------------------------------------------




Gegeben ist $y$ mit 14, 20, 11, 15 und 14. Berechnen Sie den Mittelwert und Standardabweichung.



\begin{enumerate}
\item [\textbf{A} \msquare] Es berechnet sich 15.8 +/- 10.7
\item [\textbf{B} \msquare] Es ergibt sich 15.8 +/- 1.635
\item [\textbf{C} \msquare] Es berechnet sich 14.8 +/- 10.7
\item [\textbf{D} \msquare] Sie erhalten 14.8 +/- 1.81
\item [\textbf{E} \msquare] Sie erhalten 14.8 +/- 3.27
\end{enumerate} 

\section{Aufgabe \hfill (2 Punkte)}

%% --------------------------------------------------------------------
\ifcollection
\begin{flushright}
\tiny\vspace{-2Ex}
\textbf{\examinhaltstart}
\exammodulemathstat $\;\bullet$
\exammodulestat $\;\bullet$
\exammodulestatbbv 
\vspace{-1Ex}
\end{flushright}
\fi
%% --------------------------------------------------------------------




Berechnen Sie den Median, das $1^{st}$ Quartile sowie das $3^{rd}$ Quartile von $y$ mit 12, 22, 22, 17, 26, 28 und 63.




\begin{enumerate}
\item [\textbf{A} \msquare] Sie erhalten 22 [15; 26]
\item [\textbf{B} \msquare] Es ergibt sich 22 +/- 17
\item [\textbf{C} \msquare] Sie erhalten 22 +/- 28
\item [\textbf{D} \msquare] Es ergibt sich 22 [17; 28]
\item [\textbf{E} \msquare] Es berechnet sich 23 [18; 27]
\end{enumerate} 

\section{Aufgabe \hfill (2 Punkte)}

%% --------------------------------------------------------------------
\ifcollection
\begin{flushright}
\tiny\vspace{-2Ex}
\textbf{\examinhaltstart}
\exammodulemathstat $\;\bullet$
\exammodulestat $\;\bullet$
\exammodulestatbbv 
\vspace{-1Ex}
\end{flushright}
\fi
%% --------------------------------------------------------------------




Die \uline{Standardabweichung} ist eine bedeutende deskriptive Statistik für die Analyse von Daten. Wie müssen Sie vorgehen um die Standardabweichung zu berechnen?



\begin{enumerate}
\item [\textbf{A} \msquare] Wir berechnen erst den Mittelwert und dann die absoluten Abstände zu dem Mittelwert. Diese quadratischen Abstände summieren wir auf und teilen am Ende durch die Fallzahl $(n-1)$.
\item [\textbf{B} \msquare] Den Mittelwert berechen, dann die absoluten Abstände zum Mittelwert aufsummieren. Die Fallzahl $(n-1)$ entsprechend gewichten.
\item [\textbf{C} \msquare] Den Mittelwert berechnen und die Abstände quadrieren. Die Summe mit der Fallzahl $(n-1)$ multiplizieren.
\item [\textbf{D} \msquare] Wir berechnen erst den Mittelwert und dann die quadratischen Abstände zu dem Mittelwert. Diese quadratischen Abstände summieren wir auf und teilen am Ende durch die Fallzahl $(n-1)$. Als letzten Schritt ziehen wir die quadratische Wurzel.
\item [\textbf{E} \msquare] Den Mittelwert berechen, dann die quadratischen Abstände zum Mittelwert aufsummieren und durch die Fallzahl $(n-1)$ teilen.
\end{enumerate} 

\section{Aufgabe \hfill (2 Punkte)}

%% --------------------------------------------------------------------
\ifcollection
\begin{flushright}
\tiny\vspace{-2Ex}
\textbf{\examinhaltstart}
\exammodulemathstat $\;\bullet$
\exammodulestat $\;\bullet$
\exammodulestatbbv $\;\bullet$
\exammodulestatversuch $\;\bullet$
\exammodulebiostat
\vspace{-1Ex}
\end{flushright}
\fi
%% --------------------------------------------------------------------




Der \uline{Barplot} stellt folgende statistische Maßzahlen in einer Abbildung dar. Damit gehört der Barplot zu einem der am meisten genutzten statistischen Verfahren zur Visualisierung von Daten.

 



\begin{enumerate}
\item [\textbf{A} \msquare] Der Barplot stellt die Mittelwerte und die Varianz dar.
\item [\textbf{B} \msquare] Den Median und die Standardabweichung.
\item [\textbf{C} \msquare] Der Barplot stellt die Mittelwerte und die Standardabweichung dar.
\item [\textbf{D} \msquare] Durch die Abbildung des Barplot erhalten wir die Informationen über den Median und die Quartile.
\item [\textbf{E} \msquare] Den Mittelwert und die Varianz.
\end{enumerate} 

\section{Aufgabe \hfill (2 Punkte)}

%% --------------------------------------------------------------------
\ifcollection
\begin{flushright}
\tiny\vspace{-2Ex}
\textbf{\examinhaltstart}
\exammodulemathstat $\;\bullet$
\exammodulestat $\;\bullet$
\exammodulestatbbv $\;\bullet$
\exammodulestatversuch $\;\bullet$
\exammodulebiostat
\vspace{-1Ex}
\end{flushright}
\fi
%% --------------------------------------------------------------------




In Ihrer Abschlussarbeit zu Erbsen finden Sie aufeinmal seltsame Daten. Jedenfalls kommt Ihnen das so vor. Daher berechnen Sie den Mittelwert mit 9.7 t/ha und den Median mit 10.1 t/ha. Der Mittelwert wird mit $\bar{y}$ und der Median mit $\tilde{y}$ abgekürzt. Welche Aussage ist richtig?



\begin{enumerate}
\item [\textbf{A} \msquare] Wenn sich der Mittelwert und der Median \underline{unterscheiden}, liegen vermutlich Outlier in den Daten vor.
\item [\textbf{B} \msquare] Da sich der Mittelwert und der Median \underline{nicht unterscheiden}, liegen vermutlich Outlier in den Daten vor. Wir untersuchen den Datensatz nach auffälligen Beobachtungen.
\item [\textbf{C} \msquare] Da sich der Mittelwert und der Median \underline{nicht unterscheiden}, liegen vermutlich keine Outlier in den Daten vor. Wir verweden den Datensatz so wie er ist.
\item [\textbf{D} \msquare] Nach Ihrer Betrachtung der Werte \underline{unterscheiden} sich Mittelwert und Median \underline{nicht}. Die Daten können also so nicht verwendet werden. Es gibt mindestens Outlier $n-1$ Ausreißer in den Daten.
\item [\textbf{E} \msquare] Da sich der Mittelwert und der Median \underline{unterscheiden}, ist der Datensatz nicht zu verwenden. Mittelwert und Median müssen gleich sein.
\end{enumerate}

\section{Aufgabe \hfill (2 Punkte)}

%% --------------------------------------------------------------------
\ifcollection
\begin{flushright}
\tiny\vspace{-2Ex}
\textbf{\examinhaltstart}
\exammodulestat $\;\bullet$
\exammodulestatbbv $\;\bullet$
\exammodulestatversuch $\;\bullet$
\exammodulebiostat
\vspace{-1Ex}
\end{flushright}
\fi
%% --------------------------------------------------------------------




Im Folgenden sehen Sie ein Normalverteilung dargestellt. Welche Aussage zu der Normalvererteilung und der Standardabweichung $\sigma$ ist richtig?



{\centering \includegraphics[width=\maxwidth]{img/mc-distribution-02-a-1} 

}







\begin{enumerate}
\item [\textbf{A} \msquare] Die Fläche links von $-2\sigma$ ist der p-Wert mit $Pr(D|H_0)$ in der obigen Abbildung.
\item [\textbf{B} \msquare] Die Fläche unter der Kurve entspricht dem Signifikanzniveau $\alpha$ von 5\%. Damit ist die Standardabweichung $\sigma$ gleich 1 in der obigen Abbildung.
\item [\textbf{C} \msquare] Es liegen 68\% der Beobachtungen zwischen $\bar{y}\pm 1 \sigma$. Angezeigt durch die Fläche zwischen $-1\sigma$ und $1\sigma$ in der obigen Abbildung.
\item [\textbf{D} \msquare] Die Fläche zwischen $-2\sigma$ und $2\sigma$ ist 0.68 und 68\% der Beobachtungen liegen somit zwischen $\bar{y}\pm\sigma$ in der obigen Verteilung.
\item [\textbf{E} \msquare] Die Fläche zwischen $-1\sigma$ und $1\sigma$ ist 0.95 und 95\% der Beobachtungen liegen somit zwischen $\bar{y}\pm\sigma$ in der obigen Verteilung.
\end{enumerate} 

\section{Aufgabe \hfill (2 Punkte)}

%% --------------------------------------------------------------------
\ifcollection
\begin{flushright}
\tiny\vspace{-2Ex}
\textbf{\examinhaltstart}
\exammodulestatversuch $\;\bullet$
\exammodulebiostat
\vspace{-1Ex}
\end{flushright}
\fi
%% --------------------------------------------------------------------




Die empfohlene Mindestanzahl an Beobachtungen für die Visualisierung mit einem Histogramm sind...



\begin{enumerate}
\item [\textbf{A} \msquare] Die Mindestanzahl liegt bei fünf Beobachtungen.
\item [\textbf{B} \msquare] 2-5 Beobachtungen.
\item [\textbf{C} \msquare] Wir sollten eine Beobachtung mindestens pro Gruppe vorliegen haben.
\item [\textbf{D} \msquare] Damit wir hier sauber eine Abbilung von einem Histogramm erhalten, sollten wir mindestens zwanzig Beobachtungen haben.
\item [\textbf{E} \msquare] Die opimale Anzahl ist größer als hundert Beobachtungen, wobei es gerne sehr viel mehr sein können.
\end{enumerate}

\section{Aufgabe \hfill (2 Punkte)}

%% --------------------------------------------------------------------
\ifcollection
\begin{flushright}
\tiny\vspace{-2Ex}
\textbf{\examinhaltstart}
\exammodulestatversuch $\;\bullet$
\exammodulebiostat
\vspace{-1Ex}
\end{flushright}
\fi
%% --------------------------------------------------------------------




Sie wollen in Ihrer Abschlussarbeit über eine explorative Datenanalyse überprüfen, ob Ihr gemessener Endpunkt einer Normalverteilung folgt. Welche drei Abbildungen eignen sich insbesondere für die Überprüfung?





\begin{enumerate}
\item [\textbf{A} \msquare] Boxplot, Violinplot, Mosaicplot
\item [\textbf{B} \msquare] Densityplot, Boxplot, Violinplot
\item [\textbf{C} \msquare] Violinplot, Scatterplot, Barplot
\item [\textbf{D} \msquare] Histogramm, Densityplot, Dotplot
\item [\textbf{E} \msquare] Barplot, Mosaicplot, Violinplot
\end{enumerate}

\section{Aufgabe \hfill (2 Punkte)}


%% --------------------------------------------------------------------
\ifcollection
\begin{flushright}
\tiny\vspace{-2Ex}
\textbf{\examinhaltstart}
\exammodulestatversuch $\;\bullet$
\exammodulebiostat
\vspace{-1Ex}
\end{flushright}
\fi
%% --------------------------------------------------------------------




Um zu Überprüfen, ob die Daten die Annahme einer Varianzhomogenität genügen, können wir folgende Visualisierung nutzen. Dabei kommt dann auch die entsprechende Regel zur Abschätzung der Annahme einer Varianzhomogenität zur Anwendung.



\begin{enumerate}
\item [\textbf{A} \msquare] In einer explorativen Datanalyse nutzen wir den Violinplot. Dabei sollte der Bauch am Rand liegen. Dann können wir von einer Varianzhomogenität ausgehen.
\item [\textbf{B} \msquare] Nach der Erstellung eines Boxplots schauen wir, ob der Median in der Mitte der Box liegt. Dabei ist der Median als dicke Linie dargestellt und die Box ist das IQR.
\item [\textbf{C} \msquare] Einen Boxplot. Das IQR muss über alle Behandlungen zusammen mit den Whiskers ungefähr gleich aussehen.
\item [\textbf{D} \msquare] Wir erstellen uns für jede Behandlung einen Dotplot und schauen, ob die Dots und damit die Varianz für jede Behandlung gleich groß sind.
\item [\textbf{E} \msquare] Einen Barplot. Die Mittelwerte müssen alle auf einer Höhe liegen. Die Fehlerbalken haben hier keine Informationen.
\end{enumerate}
\section*{Statistische Testtheorie}  

\section{Aufgabe \hfill (2 Punkte)}

%% --------------------------------------------------------------------
\ifcollection
\begin{flushright}
\tiny\vspace{-2Ex}
\textbf{\examinhaltstart}
\exammodulemathstat $\;\bullet$
\exammodulestat $\;\bullet$
\exammodulestatbbv $\;\bullet$
\exammodulestatversuch $\;\bullet$
\exammodulebiostat
\vspace{-1Ex}
\end{flushright}
\fi
%% --------------------------------------------------------------------




Welche Aussage zum mathematische Ausdruck $Pr(D|H_0)$ ist richtig?



\begin{enumerate}
\item [\textbf{A} \msquare] $Pr(D|H_0)$ ist die Wahrscheinlichkeit der Alternativehypothese und somit $1 - Pr(H_A)$
\item [\textbf{B} \msquare] $Pr(D|H_0)$ ist die Wahrscheinlichkeit nicht die Daten $D$ zu beobachten sondern die Nullhypothese, wenn diese wahr ist.
\item [\textbf{C} \msquare] Die Inverse der Wahrscheinlichkeit unter der die Nullhypothese nicht mehr die Alternativehypothese überdeckt.
\item [\textbf{D} \msquare] Die Wahrscheinlichkeit der Daten unter der Nullhypothese in der Grundgesamtheit.
\item [\textbf{E} \msquare] $Pr(D|H_0)$ stellt die Wahrscheinlichkeit die Daten $D$ und somit die Teststatistik $T_D$ zu beobachten dar, wenn die Nullhypothese wahr ist.
\end{enumerate} 

\section{Aufgabe \hfill (2 Punkte)}

%% --------------------------------------------------------------------
\ifcollection
\begin{flushright}
\tiny\vspace{-2Ex}
\textbf{\examinhaltstart}
\exammodulemathstat $\;\bullet$
\exammodulestat $\;\bullet$
\exammodulestatbbv $\;\bullet$
\exammodulestatversuch $\;\bullet$
\exammodulebiostat
\vspace{-1Ex}
\end{flushright}
\fi
%% --------------------------------------------------------------------




Der Fehler 1. Art oder auch Signifikanzniveau $\alpha$ genannt, liegt bei
5\%. Welcher der folgenden Gründe für diese Festlegeung auf 5\% als Signifikanzschwelle ist richtig?



\begin{enumerate}
\item [\textbf{A} \msquare] Im Rahmen eines langen Disputs zwischen Neyman und Fischer wurde $\alpha = 5\%$ festgelegt. Leider werden die Randbedingungen und Voraussetzungen an statistsiche Modelle heute immer wieder ignoriert.
\item [\textbf{B} \msquare] Auf einer Statistikkonferenz in Genf im Jahre 1942 wurde dieser Cut-Off nach langen Diskussionen festgelegt. Bis heute ist der Cut Off aber umstritten, da wegen dem 2. Weltkrieg viele Wissenschaftler nicht teilnehmen konnten.
\item [\textbf{C} \msquare] Der Begründer der modernen Statistik, R. Fischer, hat die Grenze simuliert und berechnet. Dadurch ergibt sich dieser optimale Cut-Off.
\item [\textbf{D} \msquare] Als Kulturkonstante hat $\alpha = 5\%$ den Rang einer Naturkonstante und wurde nach langer Diskussion in der UN im Jahre 1983 festgesetzt. Damals auch schon mit der Zustimmung der UdSSR.
\item [\textbf{E} \msquare] Da Wissenschaftler eine Schwelle für die statistische Testentscheidung benötigen wurde $\alpha$ historisch gewählt. Damit ist $\alpha = 5\%$ eine Kulturkonstante.
\end{enumerate} 

\section{Aufgabe \hfill (2 Punkte)}

%% --------------------------------------------------------------------
\ifcollection
\begin{flushright}
\tiny\vspace{-2Ex}
\textbf{\examinhaltstart}
\exammodulemathstat $\;\bullet$
\exammodulestat $\;\bullet$
\exammodulestatbbv $\;\bullet$
\exammodulestatversuch $\;\bullet$
\exammodulebiostat
\vspace{-1Ex}
\end{flushright}
\fi
%% --------------------------------------------------------------------




Die Testtheorie hat einen philosophischen Unterbau. Eins der Prinzipien ist das Falsifikationsprinzip. Das Falsifikationsprinzip besagt,



\begin{enumerate}
\item [\textbf{A} \msquare] ... dass ein schlechtes Modell durch ein schlechteres Modell ersetzt wird. Die Wissenschaft lehnt ab und verifiziert nicht.
\item [\textbf{B} \msquare] ... dass ein minderwertes Modell durch ein weniger minderwertiges Modell ersetzt wird. Es gilt das Falsifikationsprinzip nach Karl Popper.
\item [\textbf{C} \msquare] ... dass Modelle meist falsch sind und selten richtig.
\item [\textbf{D} \msquare] ... dass ein minderwertes Modell durch ein minderwertiges Modell ersetzt wird. Es gilt das Verifikationsprinzip nach Karl Popper.
\item [\textbf{E} \msquare] ... dass Annahmen an statistische Modelle meist falsch sind.
\end{enumerate}

\section{Aufgabe \hfill (2 Punkte)}

%% --------------------------------------------------------------------
\ifcollection
\begin{flushright}
\tiny\vspace{-2Ex}
\textbf{\examinhaltstart}
\exammodulemathstat $\;\bullet$
\exammodulestat $\;\bullet$
\exammodulestatbbv $\;\bullet$
\exammodulestatversuch $\;\bullet$
\exammodulebiostat
\vspace{-1Ex}
\end{flushright}
\fi
%% --------------------------------------------------------------------




Betrachten wir die Teststatistik aus einem abstrakteren Blickwinkel. Beim statistischen Testen wird das 	extit{signal} mit dem 	extit{noise} aus den Daten $D$ zu einer Teststatistik $T_D$ verrechnet. Welche der Formel berechnet korrekt die Teststatistik $T_D$?



\begin{enumerate}
\item [\textbf{A} \msquare] Bei der Berechnung der Teststatistik $T_D$ gewichten wir den Effekt $signal$ mit der Varianz $noise$. Wir können verallgemeinert $T_D = \sfrac{signal}{noise^2}$ schreiben.
\item [\textbf{B} \msquare] Es gilt $T_D = signal \cdot noise$. Der Effekt $signal$ wird mit der Varianz $noise$ gewichtet.
\item [\textbf{C} \msquare] Wir gewichten den Effekt $noise$ mit der Varianz $signal$ und erhalten $\sfrac{signal}{noise^2}$.
\item [\textbf{D} \msquare] Bei der Berechnung der Teststatistik $T_D$ gewichten wir den Effekt $signal$ mit der Varianz $noise$. Wir können verallgemeinert $T_D = \sfrac{signal}{noise}$ schreiben.
\item [\textbf{E} \msquare] Es gilt $T_D = \sfrac{noise}{signal}$. Der Effekt $noise$ wird mit der Varianz $signal$ gewichtet.
\end{enumerate}

%% ------------------------------------------------------------

\section{Aufgabe \hfill (2 Punkte)}

%% --------------------------------------------------------------------
\ifcollection
\begin{flushright}
\tiny\vspace{-2Ex}
\textbf{\examinhaltstart}
\exammodulemathstat $\;\bullet$
\exammodulestat $\;\bullet$
\exammodulestatbbv $\;\bullet$
\exammodulestatversuch $\;\bullet$
\exammodulebiostat
\vspace{-1Ex}
\end{flushright}
\fi
%% --------------------------------------------------------------------




Ein statistischer Test produziert für einen Gruppenvergleich einen $p$-Wert. Welche Aussage zusammen mit dem Signifikanzniveau $\alpha$ gleich 5\% stimmt?



\begin{enumerate}
\item [\textbf{A} \msquare] Wir vergleichen mit dem $p$-Wert und dem Signifikanzniveau $\alpha$ absolute Werte auf einem Zahlenstrahl und damit den Unterschied der Teststatistiken, wenn die $H_0$ gilt.
\item [\textbf{B} \msquare] Wir schauen, ob der $p$-Wert kleiner ist als das Signifikanzniveau $\alpha$. Wir sprechen von einem signifikanten Ergebnis. Dabei vergleichen wir somit Wahrscheinlichkeiten. Die Wahrscheinlichkeiten werden als Flächen unter der Kurve der Teststaistik dargestellt, wenn die $H_0$ gilt.
\item [\textbf{C} \msquare] Wir vergleichen die Effekte des $p$-Wertes mit den Effekten der Signifikanzschwelle unter der Annahme der Nullhypothese. Dabei gilt, dass wir die Nullhypothese nur ablehnen können anhand des Falsifikationsprinzips.
\item [\textbf{D} \msquare] Wir schauen, ob der $p$-Wert größer ist als das Signifikanzniveau $\alpha$ und vergleichen somit Wahrscheinlichkeiten. Die Wahrscheinlichkeiten werden als Flächen unter der Kurve der Teststaistik dargestellt, wenn die $H_A$ gilt.
\item [\textbf{E} \msquare] Wir machen ein Aussage über die Flächen unter der Kurve der Teststatistik der Hypothese $H_0$ und über die Flächen unter den Kurve der Teststatistik der Hypothese $H_A$. Dabei werden Wahrscheinlichkeiten vergleichen, die durch die Flächen unter der Kurve der beiden Testverteilungen repräsentiert werden.
\end{enumerate}

\section{Aufgabe \hfill (2 Punkte)}

%% --------------------------------------------------------------------
\ifcollection
\begin{flushright}
\tiny\vspace{-2Ex}
\textbf{\examinhaltstart}
\exammodulemathstat $\;\bullet$
\exammodulestat $\;\bullet$
\exammodulestatbbv $\;\bullet$
\exammodulestatversuch $\;\bullet$
\exammodulebiostat
\vspace{-1Ex}
\end{flushright}
\fi
%% --------------------------------------------------------------------




Die Ergebnisse der einer statistischen Analyse können in die Analogie einer Wettervorhersage gebracht werden. Welche Analogie für die Ergebnisse eines statistischen Tests trifft am besten zu?



\begin{enumerate}
\item [\textbf{A} \msquare] In der Analogie des Niederschlags oder Regenmenge: ein statistischer Test gibt die Stärke eines Effektes wieder. Zum Beispiel, wie hoch ist der Mittelwertsunterschied.
\item [\textbf{B} \msquare] In der Analogie der Durchschnittstemperatur: Wie oft tritt ein Effekt durchschnittlich ein? Wir erhalten eine Wahrscheinlichkeit für die Effekte. Zum Beispiel, wie hoch ist die Wahrscheinlichkeit für einen Mittelwert als Durchschnitt.
\item [\textbf{C} \msquare] In der Analogie der Regenwahrscheinlichkeit in einem bestimmten Gebiet: ein statistischer Test gibt die Wahrscheinlichkeit für ein Ereignis in einem Experiment mit den Daten $D$ wieder und lässt sich kaum verallgemeinern.
\item [\textbf{D} \msquare] In der Analogie der Wahrscheinlichkeit für Regen: ein statistischer Test erlaubt die Wahrscheinlichkeit für ein Ereignis abzuschätzen. Die Stärke des Effektes können wir nicht bestimmen.
\item [\textbf{E} \msquare] In der Analogie der Maximaltemperatur: Was ist der maximale Unterschied zwischen zwei Gruppen. Wir erhalten hier eine Aussage über die Spannweite und den maximalen Effekt.
\end{enumerate}

\section{Aufgabe \hfill (2 Punkte)}

%% --------------------------------------------------------------------
\ifcollection
\begin{flushright}
\tiny\vspace{-2Ex}
\textbf{\examinhaltstart}
\exammodulemathstat $\;\bullet$
\exammodulestat $\;\bullet$
\exammodulestatbbv $\;\bullet$
\exammodulestatversuch $\;\bullet$
\exammodulebiostat
\vspace{-1Ex}
\end{flushright}
\fi
%% --------------------------------------------------------------------




In Ihrer Abschlussarbeit wollen Sie eine Aussage über die untersuchte Population treffen. Dazu nutzen Sie einen statistischen Test. Können Sie eine valide Aussage treffen?



\begin{enumerate}
\item [\textbf{A} \msquare] Weder eine Ausssage über die Population noch über das Individuum ist mit einem statistischen Test möglich. Wir erhalten eine Aussage über ein Experiment.
\item [\textbf{B} \msquare] Ja, wir erhalten nur eine Aussage zu zwei Individuen. Ein statistischer Test liefert Informationen zu einem Individuum im Vergleich zu einem anderen Individuum.
\item [\textbf{C} \msquare] Nein, wir können die untersuchte Population nicht mit einer statistischen Analyse auswerten. Wir erhalten keine Aussage zur Population.
\item [\textbf{D} \msquare] Ja, es ist möglich die untersuchte Population mit einem statistischen Test wie einem t-Test auszuwerten. Wir erhalten dann eine Aussage zur Population.
\item [\textbf{E} \msquare] Nein, wir erhalten eine \textit{einfache} Aussage. Wir müssen daher das Individuum im Kontext der Population adjustieren.
\end{enumerate}

\section{Aufgabe \hfill (2 Punkte)}

%% --------------------------------------------------------------------
\ifcollection
\begin{flushright}
\tiny\vspace{-2Ex}
\textbf{\examinhaltstart}
\exammodulemathstat $\;\bullet$
\exammodulestat $\;\bullet$
\exammodulestatbbv $\;\bullet$
\exammodulestatversuch $\;\bullet$
\exammodulebiostat
\vspace{-1Ex}
\end{flushright}
\fi
%% --------------------------------------------------------------------




Welche Aussage über den Effekt eines statistischen Tests ist richtig?



\begin{enumerate}
\item [\textbf{A} \msquare] Der Effekt eines statistischen Tests beschreibt die biologisch interpretierbare Ausgabe eines Tests. Damit ist der Effekt direkt mit dem Begriff der Signifikanz verbunden. Die Entscheidung über die Signifikanz trifft der Forschende unabhängig von der Relevanz eines statistsichen Tests.
\item [\textbf{B} \msquare] Der Forschende muss am Ende wissen, ob das Eregbnis eines Experiments relevant für seine Forschung ist. Dafür kann der Effekt eines statistischen Tests genutzt werden. Damit beschreibt der Effekt den biologischen interpretierbaren Teil einer Ausgabe eines Tests. Zum Beispiel der Unterschied zwischen zwei Anteilen.
\item [\textbf{C} \msquare] Durch den Effekt erfahren wir die statistische interpretierbare Ausgabe eines statistischen Tests. Zum Beispiel das $\eta^2$ aus einer ANOVA. Damit können wir die Signifikanz direkt mit dem Effekt verbinden. Am Ende muss der Forschende aber entscheiden, ob der Effekt entsprechend seinen Erwartungen als bedeutet zu bewerten ist.
\item [\textbf{D} \msquare] Der Effekt eines statistischen Tests beschreibt die biologisch interpretierbare Ausgabe eines Tests. Moderen Algorithmen liefern keine Effekte mehr sondern nur noch bedingte Wahrscheinlichkeiten. Der Effekt spielt in der modernen Statistik keine Rollen mehr.
\item [\textbf{E} \msquare] Der Forschende muss am Anfang wissen, ob das Eregbnis eines Experiments relevant für seine Forschung ist. Dafür kann der Effekt eines statistischen Tests genutzt werden oder auch der Prähoc-Test. Damit beschreibt der Effekt den biologischen interpretierbaren Teil eines Experimnts vor der Durchführung. Zum Beispiel der Unterschied zwischen zwei Mittelwerten.
\end{enumerate}

\section{Aufgabe \hfill (2 Punkte)}

%% --------------------------------------------------------------------
\ifcollection
\begin{flushright}
\tiny\vspace{-2Ex}
\textbf{\examinhaltstart}
\exammodulemathstat $\;\bullet$
\exammodulestat $\;\bullet$
\exammodulestatbbv $\;\bullet$
\exammodulestatversuch $\;\bullet$
\exammodulebiostat
\vspace{-1Ex}
\end{flushright}
\fi
%% --------------------------------------------------------------------



Die Randomisierung von Beobachtungen zu den Versuchseinheiten ist bedeutend in der Versuchsplanung. Welche der folgenden Aussagen ist richtig?



\begin{enumerate}
\item [\textbf{A} \msquare] Randomisierung ist die direkte Folge von Strukturgleichheit. Die Strukturgleichheit erlaubt es erst von der Stichprobe auf die Grundgesamtheit zurückzuschliessen.
\item [\textbf{B} \msquare] Strukturgleichheit ist durch Randomisierung gegeben. Somit kann von der Stichprobe auf die Grundgesamtheit geschlossen werden
\item [\textbf{C} \msquare] Strukturgleichheit ist durch Randomisierung gegeben. Leider hilft die Randomisierung noch nicht um von der Stichprobe auf die Grundgesamtheit zu schließen. Deshalb wurde das Falsifikationsprinzip entwickelt.
\item [\textbf{D} \msquare] Randomisierung erlaubt erst die Mittelwerte zu schätzen. Ohne Randomisierung keine Mittelwerte. Ohne Mittelwerte keine Varianz und somit auch kein statistischer Test.
\item [\textbf{E} \msquare] Randomisierung bringt starke Unstrukturiertheit in das Experiment und erlaubt erst von der Stichprobe auf die Grundgesamtheit zurückzuschliessen.
\end{enumerate}

\section{Aufgabe \hfill (2 Punkte)}

%% --------------------------------------------------------------------
\ifcollection
\begin{flushright}
\tiny\vspace{-2Ex}
\textbf{\examinhaltstart}
\exammodulemathstat $\;\bullet$
\exammodulestat $\;\bullet$
\exammodulestatbbv $\;\bullet$
\exammodulestatversuch $\;\bullet$
\exammodulebiostat
\vspace{-1Ex}
\end{flushright}
\fi
%% --------------------------------------------------------------------




Ein statistischer Test benötigt für die richtige Durchführung Hypothesen $H$, sonst ist der Test nicht zu interpretieren. Welche Aussage ist richtig?



\begin{enumerate}
\item [\textbf{A} \msquare] Es gibt - bedingt durch das das Falsifikationsprinzip - ein Set von $k$ Nullhypothesen, die iterative gegen $k-1$ Alternativhypothesen getestet werden.
\item [\textbf{B} \msquare] Die Hypothesen $H_0$ und $H_A$ sind rein prosarischer Natur und bilden keinen mathematischen Hintergrund ab. In der Statistik wird die wissenschaftliche Fragestellung getestet. Daher stehen auch die verständlichen Hypothesen im Mittelpunkt der biologischen Interpretation.
\item [\textbf{C} \msquare] Es gibt ein statistisches Hypothesenpaar mit der Hypothese für und gegen die wissenschaftliche Fragestellung. Die Hypothesen werden $H_{pro}$ und $H_{contra}$ bezeichnet.
\item [\textbf{D} \msquare] Ein statistisches Hypothesenpaare gibt es. Zum einen die Nullhypothese und zum anderen die Alternativehypothese. Es ist aber nur notwendig die Alternative anzugeben, da die Nullhypothese nicht beim Testen benötigt wird.
\item [\textbf{E} \msquare] Mit der Nullhypothese $H_0$ und der Alternativehypothese $H_A$ oder $H_1$ gibt es zwei Hypothesen.
\end{enumerate}

\section{Aufgabe \hfill (2 Punkte)}

%% --------------------------------------------------------------------
\ifcollection
\begin{flushright}
\tiny\vspace{-2Ex}
\textbf{\examinhaltstart}
\exammodulemathstat $\;\bullet$
\exammodulestat $\;\bullet$
\exammodulestatbbv $\;\bullet$
\exammodulestatversuch $\;\bullet$
\exammodulebiostat
\vspace{-1Ex}
\end{flushright}
\fi
%% --------------------------------------------------------------------




In der Theorie zur statistischen Testentscheidung kann folgende Aussage
in welche richtige Analogie gesetzt werden?

\begin{center}
\textit{$H_0$ beibehalten obwohl die $H_0$ falsch ist}
\end{center}



\begin{enumerate}
\item [\textbf{A} \msquare] Dem $\alpha$-Fehler in der Analogie eines Rauchmelder: \textit{Alarm without fire}.
\item [\textbf{B} \msquare] \textit{Alarm with fire}, dem $\alpha$-Fehler in der Analogie von Feuer.
\item [\textbf{C} \msquare] In die Analogie eines Rauchmelders: \textit{Alarm with fire}.
\item [\textbf{D} \msquare] In die Analogie eines brennenden Hauses ohne Rauchmelder: \textit{House without noise}.
\item [\textbf{E} \msquare] Dem $\beta$-Fehler mit der Analogie eines Rauchmelders: \textit{Fire without alarm}.
\end{enumerate}

\section{Aufgabe \hfill (2 Punkte)}

%% --------------------------------------------------------------------
\ifcollection
\begin{flushright}
\tiny\vspace{-2Ex}
\textbf{\examinhaltstart}
\exammodulestat $\;\bullet$
\exammodulestatbbv $\;\bullet$
\exammodulestatversuch $\;\bullet$
\exammodulebiostat
\vspace{-1Ex}
\end{flushright}
\fi
%% --------------------------------------------------------------------




Welche statistische Maßzahl erlaubt es Relevanz mit Signifikanz zu verbinden? Welche Aussage ist richtig?



\begin{enumerate}
\item [\textbf{A} \msquare] Das $\Delta$. Durch die Effektstärke haben wir einen Wert für die Relevanz, die vom Anwender bewertet werden muss. Da $\Delta$ antiproportional zum p-Wert ist, bedeutet auch ein hohes $\Delta$ ein sehr kleinen p-Wert.
\item [\textbf{B} \msquare] Über das Konfidenzintervall. Das Konfidenzinterval beitet eine Entscheidung über die Signifikanz und zusätzlich kann über die Visualizierung des Konfidenzintervals eine Relevanzschwelle definiert werden.
\item [\textbf{C} \msquare] Das OR. Als Chancenverhältnis gibt es das Verhältnis von Relevanz und Signifikanz wieder.
\item [\textbf{D} \msquare] Über das Konfidenzintervall. Das Konfidenzinterval inkludiert eine Entscheidung über die Relevanz und zusätzlich kann über die Visualizierung des Konfidenzintervals eine Signifikanzschwelle vom Forschenden definiert werden.
\item [\textbf{E} \msquare] Der p-Wert. Durch den Vergleich mit $\alpha$ lässt sich über die Signifikanz entscheiden und der $\beta$-Fehler erlaubt über die Power eine Einschätzung der Relevanz.
\end{enumerate}

\section{Aufgabe \hfill (2 Punkte)}

%% --------------------------------------------------------------------
\ifcollection
\begin{flushright}
\tiny\vspace{-2Ex}
\textbf{\examinhaltstart}
\exammodulestatversuch $\;\bullet$
\exammodulebiostat
\vspace{-1Ex}
\end{flushright}
\fi
%% --------------------------------------------------------------------




Sie werten in Ihrer Abschlussarbeit einen sehr großen Datensatz aus einer öffentlichen Datenbank aus. Nun stellen Sie fest, dass Sie ein Problem mit der Bewertung Ihrer Ergbnisse anhand der Signifikanz bekommen. Wie Sie herausfinden, scheint dies ein häufiges Problem in der Bio Data Science zu sein. Welche Aussage ist richtig?




\begin{enumerate}
\item [\textbf{A} \msquare] Aktuell werden zu grosse Datensätze für die gänigige Statistik gemessen. Daher wendet man maschinelle Lernverfahren für kausale Modelle an. Hier ist die Relevanz gleich Signifikanz.
\item [\textbf{B} \msquare] Relevanz und Signifikanz haben nichts miteinander zu tun. Daher gibt es auch keinen Zusammenhang zwischen hoher Fahlzahl (n > 10000) und einem signifikanten Test. Ein Effekt ist immer relevant und somit signifikant.
\item [\textbf{C} \msquare] Eine große Fallzahl führt zu mehr signifikanten Ergebnissen auch bei kleinen Effekten. Daher werden fast alle Vergleich esignifikant, wenn die Fallzahl nur groß genug wird.
\item [\textbf{D} \msquare] Riesige Datensätz haben mehr Fallzahl was zur $\alpha$-Inflation führt. Durch eine Adjustoerung kann dem Problem entgegengewirkt werden.
\item [\textbf{E} \msquare] Mehr Fallzahl in Datensätzen bedeutet mehr signifikante Ergebnisse, da in mehr Daten auch mehr Informationen beinhaltet sind. Deshalb lohnen sich riesige Datensätze, die durch die vielen signifikanten Ergebnisse auch eine Menge an relevanten Erkenntnissen liefern.
\end{enumerate}

\section{Aufgabe \hfill (2 Punkte)}

%% --------------------------------------------------------------------
\ifcollection
\begin{flushright}
\tiny\vspace{-2Ex}
\textbf{\examinhaltstart}
\exammodulestatversuch $\;\bullet$
\exammodulebiostat
\vspace{-1Ex}
\end{flushright}
\fi
%% --------------------------------------------------------------------




Die Leistung von Sauen soll auf einem Zuchtbetrieb gesteigert werden. Dafür werden die Ferkel verschiedener Sauen gemessen. Die Ferkel einer Muttersaue sind daher...



\begin{enumerate}
\item [\textbf{A} \msquare] Untereinander unabhängig. Sollten die Mütter verwandt sein, so ist die Varianzstruktur ähnlich und muss modelliert werden.
\item [\textbf{B} \msquare] Untereinander stark korreliert. Die Ferkel sind von einer Mutter und sommit miteinander korreliert. Dies wird in der Statistik jedoch meist nicht modelliert.
\item [\textbf{C} \msquare] Abhängig von der Stallanlage und des Experiments können die Ferkel abhängig oder unabhängig sein. Allgmein gilt, dass Ferkel von unterschiedlichen Sauen näher miteinander verwandt sind als Ferkel von gleichen Sauen. Das Fisher-Axiom.
\item [\textbf{D} \msquare] Untereinander abhängig, wenn die Mütter ebenfalls miteinander verwandt sind. Erst die Abhängigkeit 2. Grades wird in der Statistik modelliert.
\item [\textbf{E} \msquare] Die Ferkel stammen vom gleichen Muttertier und haben vermutlich eine ähnlichere Varianzstruktur als die Ferkel von anderen Sauen. Die Ferkel sind untereinander über die Mutter abhängig.
\end{enumerate}

\section{Aufgabe \hfill (2 Punkte)}

%% --------------------------------------------------------------------
\ifcollection
\begin{flushright}
\tiny\vspace{-2Ex}
\textbf{\examinhaltstart}
\exammodulestatversuch $\;\bullet$
\exammodulebiostat
\vspace{-1Ex}
\end{flushright}
\fi
%% --------------------------------------------------------------------






Sie rechnen einen statistischen Test und wollen anhand der berechneten Teststatistik eine Entscheidung gegen die Nullhypothese treffen. Welche Aussage ist richtig?





\begin{enumerate}
\item [\textbf{A} \msquare] Das Signifikanzniveau $\alpha$ ist gleich $5\%$. $Pr(D|H_0)$ muss kleiner sein als das Signifikanzniveau.
\item [\textbf{B} \msquare] Wir betrachten das ganze Intervall des 95\%-Konfidenzintervalls. Liegt die Null mit in dem Intervall, dann kann die Nullhypothese nicht abgelehnt werden.
\item [\textbf{C} \msquare] Der kritische Wert $T_{\alpha = 5\%}$ ist tabelarisch festgelegt und gegeben. Ist $T_{D}$ größer als der kritische Wert, kann die Nullhypothese abgelehnt werden
\item [\textbf{D} \msquare] Anhand der berechneten Teststatistik lässt sich wie folgt eine Entscheidung treffen. Liegt der Wert in dem Signifikanzniveauintervall $\alpha$ dann kann die Nullhypothese abgelehnt werden.
\item [\textbf{E} \msquare] Das Signifikanzniveauintervall $\alpha$ ist gleich $5\%$ und damit muss der berrechnete Wert unter dem Signifikanzniveauintervall $\alpha$ liegen, dann kann die Nullhypothese nicht abgelehnt werden.
\end{enumerate}

\section{Aufgabe \hfill (2 Punkte)}

%% --------------------------------------------------------------------
\ifcollection
\begin{flushright}
\tiny\vspace{-2Ex}
\textbf{\examinhaltstart}
\exammodulebiostat
\vspace{-1Ex}
\end{flushright}
\fi
%% --------------------------------------------------------------------




In der statistischen Testtheorie gibt es den Begriff \textit{Power}. Was sagt der statistische Begriff \textit{Power} aus?



\begin{enumerate}
\item [\textbf{A} \msquare] Die Power wird nicht berechnet sondern ist eine Eigenschaft des Tests. Die Power wird auf $80\%$ gesetzt und beschreibt mit welcher Wahrscheinlichkeit $H_A$ \textit{bewiesen wird}
\item [\textbf{B} \msquare] Die Power beschreibt die Wahrscheinlichkeit die $H_A$ abzulehnen. Wir testen die Power jedoch nicht.
\item [\textbf{C} \msquare] Es gilt $\alpha + \beta = 1$ und somit liegt $\beta$ meist bei 95\%.
\item [\textbf{D} \msquare] Die Power ist nicht in der aktuellen Testthorie mehr vertreten. Wir rechnen nur noch mit dem Fehler 1. Art.
\item [\textbf{E} \msquare] Die Power wird berechnet und ist keine Eigenschaft des Tests. Die Power wird auf $80\%$ gesetzt und beschreibt mit welcher Wahrscheinlichkeit $H_0$ \textit{bewiesen wird}
\end{enumerate}
\section*{Statistische Tests für Gruppenvergleiche} 

\section{Aufgabe \hfill (2 Punkte)}

%% --------------------------------------------------------------------
\ifcollection
\begin{flushright}
\tiny\vspace{-2Ex}
\textbf{\examinhaltstart}
\exammodulemathstat $\;\bullet$
\exammodulestat $\;\bullet$
\exammodulestatbbv 
\vspace{-1Ex}
\end{flushright}
\fi
%% --------------------------------------------------------------------




Welche Aussage über den t-Test im Allgmeinen ist richtig? Berücksichtigen Sie den Welch t-Test wie auch den Student t-Test!



\begin{enumerate}
\item [\textbf{A} \msquare] Der t-Test ist ein Vortest der ANOVA und basiert daher auf dem Vergleich von Streuungsparametern
\item [\textbf{B} \msquare] Der t-Test testet generell zu einem erhöhten $\alpha$-Niveau von 20\%.
\item [\textbf{C} \msquare] Der t-Test vergleicht zwei Gruppen indem die Mittelwerte miteinander verglichen werden.
\item [\textbf{D} \msquare] Der t-Test vergleicht die Mittelwerte von zwei Gruppen unter der strikten Annahme von Varianzhomogenität. Sollte keine Varianzhomogenität vorliegen, so gibt es keine Möglichkeit den t-Test in einer Variante anzuwenden.
\item [\textbf{E} \msquare] Der t-Test vergleicht zwei oder mehr Gruppen indem die Mittelwerte miteinander verglichen werden.
\end{enumerate}

\section{Aufgabe \hfill (2 Punkte)}

%% --------------------------------------------------------------------
\ifcollection
\begin{flushright}
\tiny\vspace{-2Ex}
\textbf{\examinhaltstart}
\exammodulestatversuch $\;\bullet$
\exammodulebiostat
\vspace{-1Ex}
\end{flushright}
\fi
%% --------------------------------------------------------------------





Welche Aussage über den gepaarten t-Test für verbundene Stichproben ist richtig?



\begin{enumerate}
\item [\textbf{A} \msquare] Der gepaarte t-Test nutzt die Varianz der Beobachtungen jeweils paarweise und bildet dafür eine verbundene Stichprobe. Dieser Datensatz $d$ dient dann zur Differenzbildung.
\item [\textbf{B} \msquare] Der gepaarte t-Test wird genutzt, wenn die Differenzen der Beobachtungen verbunden sind und wir dadurch die Unabhäängigkeit nicht mehr vorliegen haben.
\item [\textbf{C} \msquare] Wenn die Beobachtungen nicht unabhängig voneinander sind, rechnen wir einen gepaarten t-Test. Messen wir wiederholt an dem gleichen Tier oder Pflanze dann bilden wir die Differenz zwischen den zwei Messpunkten.
\item [\textbf{D} \msquare] Wenn die Beobachtungen unabhängig voneinander sind, rechnen wir einen gepaarten t-Test. Messen wir wiederholt an dem gleichen Tier oder Pflanze dann bilden wir das Produkt zwischen den zwei Messpunkten.
\item [\textbf{E} \msquare] Beim gepaarten t-Test kombinieren wir die Vorteile des Student t-Test für Varianzhomogenität mit den Vorteilen des Welch t-Test für Varianzheterogenität. Wir bilden dafür die Differenz der Einzelbeobachtungen.
\end{enumerate}

\section{Aufgabe \hfill (2 Punkte)}

%% --------------------------------------------------------------------
\ifcollection
\begin{flushright}
\tiny\vspace{-2Ex}
\textbf{\examinhaltstart}
\exammodulemathstat $\;\bullet$
\exammodulestat $\;\bullet$
\exammodulestat $\;\bullet$
\exammodulestatbbv $\;\bullet$
\exammodulestatversuch $\;\bullet$
\exammodulebiostat
\vspace{-1Ex}
\end{flushright}
\fi
%% --------------------------------------------------------------------




Ein Versuch wurde in 7 Parzellen pro Gruppe durchgeführt. Die folgende Abbildung enthält die Daten aus diesem Versuch zur Bewertung der Wirkung des Mikronährstoff Sulfit auf den Ertrag in t/ha von Mango im Vergleich zu einer Kontrolle.  Welche Aussage ist richtig, wenn Sie einen t-Test rechnen?



{\centering \includegraphics[width=\maxwidth]{img/mc-testing-ttest-02-1} 

}







\begin{enumerate}
\item [\textbf{A} \msquare] Es liegt ein signifikanter Unterschied vor. Der Effekt liegt bei 0.1.
\item [\textbf{B} \msquare] Der Test deutet auf keinen signifikanten Unterschied hin. Der Effekt liegt vermutlich bei 1.
\item [\textbf{C} \msquare] Die Barplots deuten auf ein signifikanten Unterschied. Der Effekt liegt vermutlich bei 1.
\item [\textbf{D} \msquare] Die Barplots deuten auf keinen signifikanten Unterschied. Der Effekt liegt vermutlich bei 1 unter einer groben Abschätzung. Wir müssen aber eine ANOVA rechnen um den Effekt wirklich bestimmen zu können.
\item [\textbf{E} \msquare] Nach Betrachtung des Barplots liegt kein signifikanter Unterschied vor. Der Effekt kann nicht bei einem t-Test aus Barplots bestimmt werden.
\end{enumerate}

\section{Aufgabe \hfill (2 Punkte)}

%% --------------------------------------------------------------------
\ifcollection
\begin{flushright}
\tiny\vspace{-2Ex}
\textbf{\examinhaltstart}
\exammodulestatversuch $\;\bullet$
\exammodulebiostat
\vspace{-1Ex}
\end{flushright}
\fi
%% --------------------------------------------------------------------




In Ihrer Abschlussarbeit passen die Ergebnisse einer ANOVA und eines multiplen Vergleiches nicht zusammen. Nach einem Experiment mit drei Maissorten ergibt eine ANOVA ($p = 0.045$). Sie führen anschließend die paarweisen t-Tests für alle Vergleiche durch. Nach der Adjustierung für multiples Testen ist kein p-Wert unter der $\alpha$-Schwelle. Sie schauen sich auch die rohen, unadjustierten p-Werte an und finden hier als niedrigsten p-Wert $p_{3-2} = 0.052$. Welche Aussage ist richtig?




\begin{enumerate}
\item [\textbf{A} \msquare] Hier kommt der Effekt der stiegenden Fallzahl auf die Anzahl an signifikante Ergebnisse zu tragen. Da die ANOVA auf weniger Fallzahl testet als die paarweisen t-Tests, kann die ANOVA schwerer einen signifikanten Unterscheid nachweisen.
\item [\textbf{B} \msquare] Das ist kein Wunder. Die ANOVA testet nicht auf der gesamten Fallzahl und die paarweisen t-Tests gewinnen immer eine oder mehr Gruppen als Fallzahl dazu. Mit steigender Fallzahl sind mehr signifikante Unterschiede zu erwarten. Die p-Werte unterscheiden sich numerisch auch kaum.
\item [\textbf{C} \msquare] Das Beispiel kann so nicht auftreten, da die ANOVA und die t-Tests algorithmisch miteinander verschränkt sind.
\item [\textbf{D} \msquare] Es gibt einen Fehler in der Varianzstruktur. Daher kann die ANOVA nicht richtig sein und paarweise t-Tests liefern das richtige Ergebnis.
\item [\textbf{E} \msquare] Hier kommt der Effekt der stiegenden Fallzahl auf die Anzahl an signifikante Ergebnisse zu tragen. Da die ANOVA auf mehr Fallzahl testet als die einzelnen paarweisen t-Tests, kann die ANOVA leichter einen signifikanten Unterscheid nachweisen. Die p-Werte sind immer etwas kleiner als bei den t-Tests.
\end{enumerate}
\section*{ANOVA}

\section{Aufgabe \hfill (2 Punkte)}

%% --------------------------------------------------------------------
\ifcollection
\begin{flushright}
\tiny\vspace{-2Ex}
\textbf{\examinhaltstart}
\exammodulemathstat $\;\bullet$
\exammodulestat $\;\bullet$
\exammodulestatbbv $\;\bullet$
\exammodulestatversuch $\;\bullet$
\exammodulebiostat
\vspace{-1Ex}
\end{flushright}
\fi
%% --------------------------------------------------------------------






Aus einem Feldversuch ergibt sich die Notwendigkeit der Berechnung einer einfaktoriellen ANOVA. Es ergibt sich ein $\eta^2 = 0.31$. Welche Aussage ist richtig?



\begin{enumerate}
\item [\textbf{A} \msquare] Die Berechnung von $\eta^2$ ist ein Wert für die Interaktion in der einfaktoriellen ANOVA.
\item [\textbf{B} \msquare] Das $\eta^2$ beschreibt den Anteil der globalen Varianz, der von den Umweltbedingungen erklärt wird.
\item [\textbf{C} \msquare] Der Anteil der Varianz, der von den Behandlungsbedingungen erklärt wird, wird durch das $\eta^2$ beschrieben.
\item [\textbf{D} \msquare] Der Anteil der Varianz, der von den Behandlungsbedingungen erklärt wird, wird durch das $1-\eta^2$ beschrieben.
\item [\textbf{E} \msquare] Das $\eta^2$ beschreibt den Anteil der Varianz, der von den Behandlungsbedingungen nicht erklärt wird. Somit der Rest an nicht erklärbarer Varianz.
\end{enumerate}

\section{Aufgabe \hfill (2 Punkte)}

%% --------------------------------------------------------------------
\ifcollection
\begin{flushright}
\tiny\vspace{-2Ex}
\textbf{\examinhaltstart}
\exammodulemathstat $\;\bullet$
\exammodulestat $\;\bullet$
\exammodulestatbbv $\;\bullet$
\exammodulestatversuch $\;\bullet$
\exammodulebiostat
\vspace{-1Ex}
\end{flushright}
\fi
%% --------------------------------------------------------------------







In Ihrem Feldexperiment wollen Sie den Proteingehalt von Erbsen steigern. Die Pflanzen wachsen unter einer Kontrolle und zwei verschiedenen Behandlungsbedingungen. Um die Daten auszuwerten führen Sie eine einfaktoriellen ANOVA durch. Es ergibt sich ein $\eta^2 = 0.28$. Welche Aussage ist richtig?



\begin{enumerate}
\item [\textbf{A} \msquare] Es werden 28\% der Varianz durch den Versuch erklärt. Das $\eta^2$ beschreibt den Anteil der Varianz, der durch Fehler in der Versuchsdurchführung entsteht.
\item [\textbf{B} \msquare] Mit dem $\eta^2$ lässt sich auf die Qualität der Randomisierung und damit der Strukturgleichheit zwischen der Grundgesamtheit und der Stichprobe schließen. Es gilt dabei die Regel, dass ein $\eta^2$-Wert von 1 zu bevorzugen ist.
\item [\textbf{C} \msquare] Das $\eta^2$ beschreibt den Anteil der Varianz, der von den Umweltbedingungen erklärt wird. Daher werden 28\% der Varianz durch die Umweltbedingungen erklärt. Der Anteil der Varianz durch die Behandlungsgruppen ist dann 72\%.
\item [\textbf{D} \msquare] Das $\eta^2$ beschreibt den Anteil der Varianz, der von den Behandlungsbedingungen erklärt wird. Daher werden 28\% der Varianz durch die Behandlungsgruppen erklärt.
\item [\textbf{E} \msquare] Es werden 72\% der Varianz durch die Behandlung erklärt. Das $\eta^2$ beschreibt den Anteil der Varianz, der von den unterschiedlichen Behandlungsbedingungen nicht erklärt wird.
\end{enumerate}

\section{Aufgabe \hfill (2 Punkte)}

%% --------------------------------------------------------------------
\ifcollection
\begin{flushright}
\tiny\vspace{-2Ex}
\textbf{\examinhaltstart}
\exammodulemathstat $\;\bullet$
\exammodulestat $\;\bullet$
\exammodulestatbbv $\;\bullet$
\exammodulestatversuch $\;\bullet$
\exammodulebiostat
\vspace{-1Ex}
\end{flushright}
\fi
%% --------------------------------------------------------------------




Die einfaktorielle ANOVA ist ein Standardverfahren in der agrawissenschaftlichen Forschung wenn es um den Vergleich von Behandlungsgruppen geht. Welche der folgenden Aussage zu der Berechnung der Teststatistik der einfaktoriellen ANOVA ist richtig?



\begin{enumerate}
\item [\textbf{A} \msquare] Die ANOVA berechnet die F-Statistik indem die MS des Fehlers durch die MS der Behandlung geteilt werden. Wenn die F-Statistik sich der 0 ann{"a}hert kann die Nullhypothese abgelehnt werden.
\item [\textbf{B} \msquare] Die ANOVA berechnet die T-Statistik aus der Multiplikation der MS Behandlung mit der MS der Fehler. Wenn die F-Statistik genau 0 ist, kann die Nullhypothese nicht abgelehnt werden.
\item [\textbf{C} \msquare] Die ANOVA berechnet die F-Statistik indem die MS des Fehlers durch die MS der Behandlung geteilt werden. Wenn die F-Statistik sich der 1 ann{"a}hert kann die Nullhypothese nicht abgelehnt werden.
\item [\textbf{D} \msquare] Die ANOVA berechnet die T-Statistik indem den Mittelwertsunterschied der Gruppen simultan durch die Standardabweichung der Gruppen teilt. Wenn die T-Statistik h{"o}her als 1.96 ist, kann die Nullhypothese abgelehnt werden.
\item [\textbf{E} \msquare] Die ANOVA berechnet die F-Statistik indem die MS der Behandlung durch die MS des Fehlers geteilt werden. Wenn die F-Statistik sich der 0 ann{"a}hert kann die Nullhypothese nicht abgelehnt werden.
\end{enumerate}

\section{Aufgabe \hfill (2 Punkte)}

%% --------------------------------------------------------------------
\ifcollection
\begin{flushright}
\tiny\vspace{-2Ex}
\textbf{\examinhaltstart}
\exammodulemathstat $\;\bullet$
\exammodulestat $\;\bullet$
\exammodulestatbbv $\;\bullet$
\exammodulestatversuch $\;\bullet$
\exammodulebiostat
\vspace{-1Ex}
\end{flushright}
\fi
%% --------------------------------------------------------------------




Die ANOVA ist ein statistisches Verfahren welches häufig in den Auswertungen von Experimenten in den Agrarwissenschaften angewendet
wird. Dabei wird die ANOVA als ein erstes statistischen Werkzeug für die
Übersicht über die Daten benutzt. Eine ANOVA testet dabei...



\begin{enumerate}
\item [\textbf{A} \msquare] ... den Unterschied zwischen der Varianz in den verschiedenen Behandlungsguppen und der Varianz in einer der Behandlungsgruppen. Wenn die ANOVA signifikant ist, muss über einen Posthoc-Test nachgedacht werden um den signifikanten Unterschied in einer der Gruppen exakt zu bestimmen.
\item [\textbf{B} \msquare] ... den Unterschied zwischen der Varianz über alle Behandlungsgruppen und der Varianz aus verschiedenen Behandlungsguppen. Wenn die ANOVA signifikant ist, muss ein Posthoc-Test angeschlossen werden.
\item [\textbf{C} \msquare] ... den Unterschied zwischen der Mittelwerte und der Varianz aus verschiedenen Behandlungsguppen. Wenn die ANOVA signifikant ist, ist bekannt welcher Vergleich konkret unterschiedlich ist.
\item [\textbf{D} \msquare] ... den Unterschied zwischen mehreren Varianzen aus verschiedenen Behandlungsguppen. Wenn die ANOVA signifikant ist, ist bekannt welcher Vergleich konkret unterschiedlich ist.
\item [\textbf{E} \msquare] ... den Unterschied zwischen der F-Statistik anhand der Varianz der Gruppen. Wenn die F-Statistik exakt 0 ist, kann die Nullhypothese abgelehnt werden.
\end{enumerate}

\section{Aufgabe \hfill (2 Punkte)}

%% --------------------------------------------------------------------
\ifcollection
\begin{flushright}
\tiny\vspace{-2Ex}
\textbf{\examinhaltstart}
\exammodulestatversuch $\;\bullet$
\exammodulebiostat
\vspace{-1Ex}
\end{flushright}
\fi
%% --------------------------------------------------------------------




Ein Versuch wurde an 61 Tieren durchgeführt, wobei jedes Tier eine von drei Vitamin-C-Dosen (0.5, 1 und 1.5 mg/Tag) über eine von zwei Verabreichungsmethoden erhielt. Die folgende Abbildung enthält die Daten aus diesem Versuch zur Bewertung der Wirkung von Vitamin D auf das Zahnwachstum bei Hasen.  Welche Aussage ist richtig, wenn Sie eine zweifaktorielle ANOVA rechnen?



{\centering \includegraphics[width=\maxwidth]{img/mc-anova-02-a-1} 

}







\begin{enumerate}
\item [\textbf{A} \msquare] Es liegt keine Interaktion vor $(p > 0.05)$.
\item [\textbf{B} \msquare] Das Bestimmtheitsmaß $R^2$ ist klein.
\item [\textbf{C} \msquare] Eine Korrelation liegt vor $(p \leq 0.05)$.
\item [\textbf{D} \msquare] Eine negative Interaktion liegt vor $(\rho \geq 0.5)$.
\item [\textbf{E} \msquare] Eine mittlere bis starke Interaktion liegt vor $(p \leq 0.05)$
\end{enumerate}
\section*{Multiple Gruppenvergleiche} 

\section{Aufgabe \hfill (2 Punkte)}

%% --------------------------------------------------------------------
\ifcollection
\begin{flushright}
\tiny\vspace{-2Ex}
\textbf{\examinhaltstart}
\exammodulestatversuch $\;\bullet$
\exammodulebiostat
\vspace{-1Ex}
\end{flushright}
\fi
%% --------------------------------------------------------------------




Sie haben folgende unadjustierten p-Werte gegeben: 0.02, 0.03, 0.42, 0.001 und 0.89. Sie adjustieren die p-Werte nach
Bonferroni. Welche Aussage ist richtig?



\begin{enumerate}
\item [\textbf{A} \msquare] Nach der Bonferroni-Adjustierung ergeben sich die adjustierten p-Werte von 0.1, 0.15, 1, 0.005 und 1. Die adjustierten p-Werte werden zu einem $\alpha$-Niveau von 5\% verglichen.
\item [\textbf{B} \msquare] Nach der Bonferroni-Adjustierung ergeben sich die adjustierten p-Werte von 0.1, 0.15, 1, 0.005 und 1. Die adjustierten p-Werte werden zu einem $\alpha$-Niveau von 1\% verglichen.
\item [\textbf{C} \msquare] Nach der Bonferroni-Adjustierung ergeben sich die adjustierten p-Werte von 0.004, 0.006, 0.084, 2e-04 und 0.178. Die adjustierten p-Werte werden zu einem $\alpha$-Niveau von 1\% verglichen.
\item [\textbf{D} \msquare] Nach der Bonferroni-Adjustierung ergeben sich die adjustierten p-Werte von 0.1, 0.15, 2.1, 0.005 und 4.45. Die adjustierten p-Werte werden zu einem $\alpha$-Niveau von 5\% verglichen.
\item [\textbf{E} \msquare] Nach der Bonferroni-Adjustierung ergeben sich die adjustierten p-Werte von 0.004, 0.006, 0.084, 2e-04 und 0.178. Die adjustierten p-Werte werden zu einem $\alpha$-Niveau von 5\% verglichen.
\end{enumerate}

\section{Aufgabe \hfill (2 Punkte)}

%% --------------------------------------------------------------------
\ifcollection
\begin{flushright}
\tiny\vspace{-2Ex}
\textbf{\examinhaltstart}
\exammodulestatversuch $\;\bullet$
\exammodulebiostat
\vspace{-1Ex}
\end{flushright}
\fi
%% --------------------------------------------------------------------




Auf wissenschaftlichen Postern finden Sie unter Abbildungen häufig die Abbkürzung \textit{CLD}. Für welchen statistischen Fachbegriff steht die Abbkürzung und wie interpretieren Sie ein \textit{CLD}?



\begin{enumerate}
\item [\textbf{A} \msquare] Contrast letter display. Unterschiede in den Behandlungen werden durch den gleichen Buchstaben oder Symbol dargestellt. Die Interpretation des CLD führt häufig in die Irre.
\item [\textbf{B} \msquare] Compound letter display. Gleichheit in dem Outcomes wird durch den gleichen Buchstaben oder Symbol dargestellt. Teilweise ist die Interpretation des Verbunds (eng. compound) herausfordernd, da wir ja nach dem Unterschied suchen.
\item [\textbf{C} \msquare] Compact letter display. Teilweise ist die Interpretation des CLD schwierig, da wir ja nach Unterschieden suchen aber nur Gleichheit in den Buchstaben sehen. Die Gleichheit der Behandlungen wird durch gleiche Buchstaben dargestellt.
\item [\textbf{D} \msquare] Compact line display. Gleichheit in den Behandlungen wird durch den gleichen Buchstaben oder Symbol dargestellt. Früher wurden keine Buchstaben sondern eine durchgezogene Linie verwendet. Bei mehr als drei Gruppen funktioniert die Linie aber graphisch nicht mehr.
\item [\textbf{E} \msquare] Compact letter detection. Gleichheit in den Behandlungen wird durch den gleichen Buchstaben oder Symbol dargestellt.
\end{enumerate}

\section{Aufgabe \hfill (2 Punkte)}

%% --------------------------------------------------------------------
\ifcollection
\begin{flushright}
\tiny\vspace{-2Ex}
\textbf{\examinhaltstart}
\exammodulestatversuch $\;\bullet$
\exammodulebiostat
\vspace{-1Ex}
\end{flushright}
\fi
%% --------------------------------------------------------------------




Der multiple Vergleich als Posthoc-Test nach einer ANOVA ist in den Agrarwissenschaften heutzutage Standard. Welches R Paket wird häufig für den multiplen Vergleich genutzt? Welche Beschreibung der Eigenschaften ist korrekt?



\begin{enumerate}
\item [\textbf{A} \msquare] Das R Paket \{emmeans\} erlaubt die Durchführung eines multiplen Gruppenvergleichs. Aus einem emmeans Objekt lässt sich leider kein CLD erstellen. Dennoch ist das Paket einfach zu bedienen und wird deshalb genutzt. Die Interpretation der statistischen Auswertung wird über einen Barplot abgebildet.
\item [\textbf{B} \msquare] Das R Paket \{lm\}. Das Paket \{lm\} erstellt selbstständig Konfidenzintervalle und entsprechende p-Werte. Da wir in dem Paket nicht adjustieren müssen, ist es bei Anwendern sehr beliebt.
\item [\textbf{C} \msquare] Da Sie für Ihre Bachelorarbeit einen Barplot mit CLD brauchen nutzen Sie das R Paket \{emmeans\} welches Ihnen schnell die notwenidigen Informationen liefert um einen Barplot zu erstelen. Die Berechnung eines CLD ist hierbei auch einfach.
\item [\textbf{D} \msquare] Das R Paket \{hmisc\} erlaubt die Durchführung eines multiplen Gruppenvergleichs aus verschiedenen Modellen heraus. Aus einem hmisc Objekt lässt sich recht einfach das CLD erstellen und so über Barplots eine schnelle Interpration der statistischen Auswertung durchführen.
\item [\textbf{E} \msquare] Das R Paket \{ggplot\}. Wir erhalten hier sofort eine Visualisierung der Daten. Anhand der Visualisierung lässt sich eine explorative Datenanalyse durchführen, die gleichwertig zu einem Posthoc-Test ist.
\end{enumerate}

\section{Aufgabe \hfill (2 Punkte)}

%% --------------------------------------------------------------------
\ifcollection
\begin{flushright}
\tiny\vspace{-2Ex}
\textbf{\examinhaltstart}
\exammodulestatversuch $\;\bullet$
\exammodulebiostat
\vspace{-1Ex}
\end{flushright}
\fi
%% --------------------------------------------------------------------




In den Humanwissenschaften werden multiple Vergleiche häufig anders behandelt als in den Agrarwissenschaften. In beiden Bereichen tritt jedoch das gleiche Phänomen bei multiplen Testen auf. Wie muss mit dem Phänomen umgegangen werden und wie ist es benannt?



\begin{enumerate}
\item [\textbf{A} \msquare] Beim multiplen Testen kann es zu einer $\alpha$-Inflation kommen. Das globale Signifikanzniveau liegt nicht mehr bei $5\%$ sondern sehr viel höher. Daher müssen die p-Werte entsprechend adjustiert werden. Hierfür gibt es verschiedene Verfahren, wobei das Verfahren zur Adjustierung der p-Werte nach Bonferroni das bekanneste Verfahren ist.
\item [\textbf{B} \msquare] Die Adjustierung der p-Werte nach Bonferroni erlaubt es gegen die $\beta$-Inflation vorzugehen, die häufig beim multiplen Testen auftritt. Das globale Powerniveau liegt nicht mehr bei $80\%$ sondern sehr viel niedriger.
\item [\textbf{C} \msquare] Beim multiplen Testen kann es zu einer $\alpha$-Inflation kommen. Das globale Signifikanzniveau liegt nicht mehr bei $5\%$ sondern weit darunter. Daher müssen die p-Werte entsprechend adjustiert werden. Hierfür gibt es verschiedene Verfahren, wobei das Verfahren zur Adjustierung der p-Werte nach Welch das bekanneste Verfahren ist.
\item [\textbf{D} \msquare] Das globale Signifikanzniveau liegt nicht mehr bei $5\%$ sondern sehr viel niedriger, bei ca. $1\%$. Es kommt zu einer $\alpha$-Hyperinflation. Dagegen kann mit der Adjustierung der p-Werte nach Bonferroni vorgegangen werden.
\item [\textbf{E} \msquare] Beim multiplen Testen kann es zu Varianzheterogenität kommen. Das globale Signifikanzniveau liegt nicht mehr bei $5\%$. Daher müssen die p-Werte entsprechend adjustiert werden. Das Verfahren nach Welch, bekannt aus dem t-Test, ist hier häufig anzuwenden.
\end{enumerate}

\section{Aufgabe \hfill (2 Punkte)}

%% --------------------------------------------------------------------
\ifcollection
\begin{flushright}
\tiny\vspace{-2Ex}
\textbf{\examinhaltstart}
\exammodulestatversuch $\;\bullet$
\exammodulebiostat
\vspace{-1Ex}
\end{flushright}
\fi
%% --------------------------------------------------------------------




Sie rechnen mehrere t-Tests für einen multiplen Vergleich nachdem eine einfaktorielle ANOVA sich als signifikant herausgestellt hat. Welche Aussage im Bezug auf den Effekt ist richtig? 



\begin{enumerate}
\item [\textbf{A} \msquare] Wenn ein multipler Test gerechnet wird, dann muss der Effekt $\Delta$ nicht adjustiert werden. Bei einem Effekt im multiplen Testen handelt es sich um eine Wahrscheinlichkeit für das Auftreten der Nullhypothese.
\item [\textbf{B} \msquare] Wenn ein multipler Test gerechnet wird, dann muss der Effekt $\Delta$ adjustiert werden im Gegensatz zu den p-Werten.
\item [\textbf{C} \msquare] Beim multiplen Testen kann es zu einer $\Delta$-Inflation kommen. Das globale Effektniveau liegt nicht mehr bei $20\%$. Daher müssen die Effekte entsprechend adjustiert werden. Hierfür gibt es verschiedene Verfahren, wobei das Verfahren zur Adjustierung der Effekte nach Bonferroni das bekanneste Verfahren ist.
\item [\textbf{D} \msquare] Beim multiplen Testen kann es zu einer Effektüberschätzung ($\Delta$-Inflation) kommen. Daher müssen die Effekte angepasst werden. Dies geschieht nicht händisch sondern intern in den angewendeten Algorithmen.
\item [\textbf{E} \msquare] Wenn ein multipler Test gerechnet wird, dann muss der Effekt $\Delta$ nicht adjustiert werden im Gegensatz zu den p-Werten.
\end{enumerate}
\section*{Lineare Regression \& Korrelation}

\section{Aufgabe \hfill (2 Punkte)}

%% --------------------------------------------------------------------
\ifcollection
\begin{flushright}
\tiny\vspace{-2Ex}
\textbf{\examinhaltstart}
\exammodulestatversuch $\;\bullet$
\exammodulebiostat
\vspace{-1Ex}
\end{flushright}
\fi
%% --------------------------------------------------------------------




Im Allgemeinen gibt es zwei mögliche Ziele für ein Regressionsmodell. Wir können eine Vorhersagemodell oder ein kausales Modell rechnen. Welche Aussage ist für ein prädiktives Modell richtig?



\begin{enumerate}
\item [\textbf{A} \msquare] Wenn ein prädiktives Modell gerechnet werden soll, dann muss zum einen ein Traingsdatensatz sowie ein Testdatensatz definiert werden. Dabei ist der Trainingsdatensatz meist 1/10 und der Testdatensatz 1/3 der Fallzahl groß. Der Testdatensatz dient zur Validierung.
\item [\textbf{B} \msquare] Ein prädiktives Modell wird auf einem Trainingsdatensatz trainiert und anschliessend über eine explorative Datenanalyse validiert. Signifikanzen über $\beta_i$ können hier nicht festgestellt werden.
\item [\textbf{C} \msquare] Es wird ein Trainingsdatensatz zum Trainieren des Modells benötigt. Der Testdatensatz dient zur Validierung. Dies gilt insbesondere für ein prädiktives Modell.
\item [\textbf{D} \msquare] Wir modellieren den Zusammenhang zwischen $X$ und $Y$ wenn ein prädiktives Modell rerechnet wird. Dabei kann nicht der gesamte Datensatz genutzt werden. Es wird ein Trainingsdatensatz zum Trainieren des Modells benötigt.
\item [\textbf{E} \msquare] Wenn ein prädiktives Modell gerechnet werden soll dann kann dies auf dem gesamten Datensatz geschehen. Das Ziel ist es einen Zusammenhang von $X$ auf $Y$ zu modellieren. Wie wirken sich die Einflussvariablen $X$ auf den gemessenen Endpunkt $Y$ aus?
\end{enumerate}

\section{Aufgabe \hfill (2 Punkte)}

%% --------------------------------------------------------------------
\ifcollection
\begin{flushright}
\tiny\vspace{-2Ex}
\textbf{\examinhaltstart}
\exammodulestat $\;\bullet$
\exammodulestatbbv $\;\bullet$
\exammodulestatversuch $\;\bullet$
\exammodulebiostat
\vspace{-1Ex}
\end{flushright}
\fi
%% --------------------------------------------------------------------




Sie rechnen in eine linearen Regression und erhalten folgenden QQ Plot um die Annahme der normalverteilten Residuen zu überprüfen. Welche Aussage ist richtig?



{\centering \includegraphics[width=\maxwidth]{img/mc-regression-05-a-1} 

}







\begin{enumerate}
\item [\textbf{A} \msquare] Wir betrachten die Gerade und dabei insbesondere die beiden Enden der Gerade in dem IQR, also dem ersten und dritten Quartile. Hier sollten die Punkte auf der Geraden liegen, dann ist die Annahme an die Normalverteilung der Residuen erfüllt.
\item [\textbf{B} \msquare] Wir betrachten die Gerade. Wenn die Punkte einigermaßen gleichmäßig um die Gerade verteilt liegen, dann gehen wir von normalverteilten Residuen aus. Dies ist hier nicht der Fall. Wir haben keine normalverteilten Residuen vorliegen.
\item [\textbf{C} \msquare] Die Annahme der normalverteilten Residuen ist nicht erfüllt. Die Punkte liegen zum überwiegenden Teil auf der Geraden.
\item [\textbf{D} \msquare] Die Annahme der normalverteilten Residuen ist nicht erfüllt. Die Punkte liegen zum überwiegenden Teil nicht auf der Geraden.
\item [\textbf{E} \msquare] Wir betrachten die Punkte auf der Geraden. Wenn die Punkte einigermaßen auf der Geraden liegen, dann gehen wir von normalverteilten Residuen aus. Wir können hier von normalverteilten Residuen ausgehen.
\end{enumerate}

\section{Aufgabe \hfill (2 Punkte)}

%% --------------------------------------------------------------------
\ifcollection
\begin{flushright}
\tiny\vspace{-2Ex}
\textbf{\examinhaltstart}
\exammodulestat $\;\bullet$
\exammodulestatbbv $\;\bullet$
\exammodulestatversuch $\;\bullet$
\exammodulebiostat
\vspace{-1Ex}
\end{flushright}
\fi
%% --------------------------------------------------------------------




Sie berechnen in Ihrer Abschlussarbeit den Korrelationskoeffizienten $\rho$. Welche Aussage über den Korrelationskoeffizienten $\rho$ ist richtig?




\begin{enumerate}
\item [\textbf{A} \msquare] Der Korrelationskoeffizienten $\rho$ ist eine standardisierte, statistische Maßzahl, die zwischen 0 und 1 liegt. Dabei ist Korrelationskoeffizienten $\rho$ einheitslos. Eine Signifikanz kann nicht nachgewiesen werden.
\item [\textbf{B} \msquare] Der Korrelationskoeffizienten $\rho$ ist eine standardisierte, statistische Maßzahl, die zwischen -1 und 1 liegt. Dabei ist Korrelationskoeffizienten $\rho$ einheitslos.
\item [\textbf{C} \msquare] Korrelationskoeffizienten $\rho$ liegt zwischen 0 und 1. Darüber hinaus ist der Korrelationskoeffizienten $\rho$ einheitslos und kann als Standardisierung verstanden werden.
\item [\textbf{D} \msquare] Der Korrelationskoeffizienten $\rho$ wird wie das $\eta^2$ aus der ANOVA interpretiert. Der Korrelationskoeffizienten $\rho$ beschreibt den Anteil an erklärter Varianz durch die Regression. Dabei gibt er jedoch eine Richtung an und kann auch negativ werden.
\item [\textbf{E} \msquare] Der Korrelationskoeffizienten $\rho$ ist eine veraltete Darstellungsform von Effekten in der linearen Regression und wird wie das $\eta^2$ aus der ANOVA interpretiert. Der Korrelationskoeffizienten $\rho$ beschreibt den Anteil an erklärter Varianz durch die Regression.
\end{enumerate}

\section{Aufgabe \hfill (2 Punkte)}

%% --------------------------------------------------------------------
\ifcollection
\begin{flushright}
\tiny\vspace{-2Ex}
\textbf{\examinhaltstart}
\exammodulestatversuch $\;\bullet$
\exammodulebiostat
\vspace{-1Ex}
\end{flushright}
\fi
%% --------------------------------------------------------------------




Nach der Modellierung einer Regression stellt sich die Frage, ob die Residuen (\texttt{.resid}) gleichmäßig um die gefitte Gerade liegen. Sie können folgende Abbildung für die visuelle Überprüfung der Residuen nutzen. Welche Aussage ist richtig?



{\centering \includegraphics[width=\maxwidth]{img/mc-regression-06-a-1} 

}







\begin{enumerate}
\item [\textbf{A} \msquare] Die Annahme der normalverteilten Residuen ist erfüllt. Die Punkte liegen zum überwiegenden Teil auf der Diagonalen. Damit ist das Modell erfolgreich geschätzt worden.
\item [\textbf{B} \msquare] Wenn wir die Nulllinie betrachten so liegen die Punkte nicht gleichmäßig über und unter der Nulllinie. Unser Modell erfüllt nicht die Annahme von normalverteilten Residuen mit einem Mittelwert von 0 und einer Streuung von $s^2$.
\item [\textbf{C} \msquare] Die Punkte müssen gleichmäßig in dem negativen Bereich liegen. Dies ist hier klar nicht der Fall. Einzelne Ausreißer können beobachtet werden. Die Analyse ist gescheitert.
\item [\textbf{D} \msquare] Wir betrachten die Nulllinie und alle Punkte sollten ohne Muster gleichmäßig um die Nulllinie liegen. Da dies der Fal ist, gehen wir von keinen Ausreißern aus.
\item [\textbf{E} \msquare] Die Annahme der normalverteilten Residuen ist erfüllt. Es ist ein Muster zu erkennen und wir können damit auf die Signifkanz von $x_1, ..., x_p$ schließen.
\end{enumerate}

\section{Aufgabe \hfill (2 Punkte)}

%% --------------------------------------------------------------------
\ifcollection
\begin{flushright}
\tiny\vspace{-2Ex}
\textbf{\examinhaltstart}
\exammodulebiostat
\vspace{-1Ex}
\end{flushright}
\fi
%% --------------------------------------------------------------------




Sie haben ein Feldexperiment mit Spitzkohl durchgeführt und wollen nun in einer simplen linearen Regression den Einfluss der $PO_2$-Konzentration in [$\mu g$] im Wasser auf das Trockengewicht in [$kg$] untersuchen. Sie erhalten einen $\beta_{PO_2}$ Koeffizienten von $1.1\times 10^{-5}$ und einen $p$-Wert mit $1e-04$. Welche Aussage zu der Signifikanz und dem Effekt ist richtig?




\begin{enumerate}
\item [\textbf{A} \msquare] Wenn der Effekt $\beta_{PO_2}$ sehr klein ist, dann kann es an einer falsch gewählten Einheit liegen. Der Anstieg von einer Einheit in $X$ führt ja zu einer Änderung von $\beta_{PO_2}$ in $y$. Daher ist hier mit einer anderen Einheit in den Daten zu rechnen, so dass wir hier einen besser formatierten Effekt sehen. Der p-Wert stammt aus einer einheitslosen Teststatistik.
\item [\textbf{B} \msquare] Das Gewicht und die $PO_2$-Konzentration korrelieren sehr stark, deshalb wird der $\beta_{PO_2}$ Koeffizient sehr klein. Mit einer ANOVA kann für die Korrelation korrigiert werden und der Effektschätzer passt dann zum p-Wert.
\item [\textbf{C} \msquare] Manchmal ist die Einheit der Einflussvariable $X$ zu groß gewählt, so dass der Ansteig von 1 Einheit in $X$ zu einer zu großen Änderung in $y$ führt. Daher kann der Effekt $\beta_{PO_2}$ sehr klein wirken, da der p-Wert wird auf einer einheitslosen Teststatistik bestimmt wird.
\item [\textbf{D} \msquare] Wenn der Effekt $\beta_{PO_2}$ winzig ist, dann kann es an einer falsch gewählten Einheit liegen. Der Anstieg von einer Einheit in $X$ führt ja zu einer Änderung von $\beta_{PO_2}$ in $x$. Wir müssen daher die Einheit von $y$ entsprechend anpassen.
\item [\textbf{E} \msquare] Die Einheit der $PO_2$-Konzentration ist zu klein gewählt. Dadurch sehen wir den sehr kleinen $p$-Wert. Der $p$-Wert und die Einheit von der $PO_2$-Konzentration hängen antiproportional zusammen.
\end{enumerate}

\section{Aufgabe \hfill (2 Punkte)}

%% --------------------------------------------------------------------
\ifcollection
\begin{flushright}
\tiny\vspace{-2Ex}
\textbf{\examinhaltstart}
\exammodulestatversuch $\;\bullet$
\exammodulebiostat
\vspace{-1Ex}
\end{flushright}
\fi
%% --------------------------------------------------------------------




Neben der klassischen Regression kann die Funktion \texttt{lm()} in \Rlogo auch für welche andere Art von Anwendung genutzt werden?





\begin{enumerate}
\item [\textbf{A} \msquare] Neben der klassichen Verwendung der Funktion \texttt{lm()} in der linearen Regression kann auch ein Gruppenvergleich gerechnet werden. Dafür müssen aber alle Faktoren aus den Daten entfernt und numerishc umgewandelt werden. Dann kann das R Paket \{emmeans\} genutzt werden um die Korrelation zu berechnen. Eine Adjustierung ist dann nicht mehr notwendig.
\item [\textbf{B} \msquare] Die Funktion \texttt{lm()} in \Rlogo ist der erste Schritt für einen Gruppenvergleich. Danach kann eine ANOVA oder aber ein multipler Vergleich in \{emmeans\} gerechnet werden. In der Funktion  \texttt{lm()} werden die Gruppenmittelwerte bestimmt.
\item [\textbf{C} \msquare] Die Funktion \texttt{lm()} berechnet die Varianzstruktur für eine ANOVA. Dannach kann dann über eine explorative Datenalayse nochmal eine Signifikanz berechnet werden. Sollte vor der Verwendung der Funktion \texttt{lm()} schon eine EDA gerechnet worden sein, so ist die Analyse wertlos.
\item [\textbf{D} \msquare] Ist die Einflussvariable $X$ numerisch so werden die Gruppenmittelwerte geschätzt und eine anschließende ANOVA sowie multipler Gruppenvergleich mit \{emmeans\} ist möglich.
\item [\textbf{E} \msquare] Ist die Einflussvariable $X$ ein Faktor so werden die Gruppenmittelwerte geschätzt und eine anschließende ANOVA sowie multipler Gruppenvergleich mit \{emmeans\} ist möglich. Die Funktion \texttt{lm()} kann dabei eigentlich weggelassen werden, wird aber traditionell gerechnet.
\end{enumerate}

\section{Aufgabe \hfill (2 Punkte)}

%% --------------------------------------------------------------------
\ifcollection
\begin{flushright}
\tiny\vspace{-2Ex}
\textbf{\examinhaltstart}
\exammodulebiostat
\vspace{-1Ex}
\end{flushright}
\fi
%% --------------------------------------------------------------------




Wenn Ihr gemessener Endpunkt nicht einer Normalverteilung folgt, so können Sie dennoch Ihre Daten modellieren. Hierzu nutzen Sie dann das \textit{generalisierte lineare Modell (GLM)}. Welche Aussage ist richtig?




\begin{enumerate}
\item [\textbf{A} \msquare] Das GLM erlaubt auch nicht normalverteilte Residuen in der Schätzung der Regressionsgrade.
\item [\textbf{B} \msquare] Dank dem \textit{generalisierten linearen Modell (GLM)} können auch andere Verteilungsfamilien -- außer die Normalverteilung -- mit einer linearen Regression modelliert werden. Dafür werden alle Verteilungen in eine Normalverteilung überführt und anschließend standardisiert.
\item [\textbf{C} \msquare] Das \textit{generalisierte lineare Modell (GLM)} erlaubt auch weitere Verteilungsfamilien für das $Y$ bzw. das Outcome in einer linearen Regression zu wählen.
\item [\textbf{D} \msquare] Das GLM ist eine allgemeine Erweiterung der linearen Regression auf die Normalverteilung.
\item [\textbf{E} \msquare] Das GLM ist eine Vereinfachung des LM in R. Mit dem GLM lassen sich polygonale Regressionen rechnen. Somit stehen neben der Normalverteilung noch weitere Verteilungen zu Verfügung.
\end{enumerate}

\section{Aufgabe \hfill (2 Punkte)}

%% --------------------------------------------------------------------
\ifcollection
\begin{flushright}
\tiny\vspace{-2Ex}
\textbf{\examinhaltstart}
\exammodulebiostat
\vspace{-1Ex}
\end{flushright}
\fi
%% --------------------------------------------------------------------




Neben der Mittelwertsdifferenz als Effektschätzer bei normalverteilten Endpunkten wird auch häufig der Effektschätzer Odds ratio bei binären Endpunkten verwendet. Welche Aussage über den Effektschätzer Odds ratio ist im folgenden Beispiel zur Behandlung von Klaueninfektionen bei Schafe richtig? Dabei sind 3 Tiere krank und 7 Tiere sind gesund.



\begin{enumerate}
\item [\textbf{A} \msquare] Der Anteil der Kranken wird berechnet. Da es sich um ein Anteil handelt ergibt sich ein Odds ratio von 0.3.
\item [\textbf{B} \msquare] Da es sich um ein Chancenverhältnis handelt ergibt sich ein Odds ratio von 3.33.
\item [\textbf{C} \msquare] Es ergibt sich ein Odds ratio von 0.3, da es sich um eine Chancenverhältnis handelt.
\item [\textbf{D} \msquare] Das Verhältnis der Chancen Odds ratio ergibt ein Chancenverhältnis von 0.3. Wir sind an der Chance krank zu sein interessiert.
\item [\textbf{E} \msquare] Es ergibt sich ein Odds ratio von 0.43, da es sich um eine Chancenverhältnis handelt
  
\end{enumerate}
    
% -----------------------------------------------------------------------
\clearpage
% -----------------------------------------------------------------------
\part{Programmieren in R}
% -----------------------------------------------------------------------

\section{Aufgabe \hfill (9 Punkte)}



 
%% --------------------------------------------------------------------
\ifcollection
\begin{flushright}
\tiny\vspace{-3Ex}
\textbf{\examinhaltstart}
\exammodulemathstat $\;\bullet$
\exammodulelanddaten $\;\bullet$
\exammodulestat 
\vspace{-4Ex}
\end{flushright}
\begin{minipage}[t]{0.5\textwidth}
\includegraphics[width = 1.3cm]{/Users/kruppajo/work/GitHub/exam/avatare/Nilufar.png}
\end{minipage}
\begin{minipage}[t]{0.5\textwidth}
\hfill
\href{https://www.youtube.com/playlist?list=PLe51bCp9JvEFUnFqaJG5aRmON9i1ZbOYC}{\includegraphics[width = 2cm]{img/youtube}}
\end{minipage}
\vspace{-3ex}
\fi
%% --------------------------------------------------------------------



\ifcollection
\paragraph{Grundlegende Kenntnisse der Programierung in \Rlogo}
\fi

'Hm. \Rlogo ist eigentlich gar nicht so schwer, wenn man die Grundlagen kann.', meint  Nilufar ganz zuversichtlich. Nur leider kennt sie sich überhaupt nicht mit \Rlogo aus! Das heißt, Sie müssen hier einmal Rede und Antwort stehen und helfen.\\[1Ex]

Nilufar: \textit{Jetzt lese ich hier von einem Faktor. Was ist ein Faktor in \Rlogo?} \textbf{(1 Punkt)}\\[1ex]
Sie antworten:\\[3Ex]

Nilufar: \textit{Was war eigentlich nochmal ein Vorteil von der Nutzng von \Rlogo?} \textbf{(1 Punkt)}\\[1ex]
Sie antworten:\\[3Ex]

Nilufar: \textit{Warum gibt es eigentlich das RStudio und R? Wie unterscheiden sich beide voneinander?} \textbf{(1 Punkt)}\\[1ex]
Sie antworten:\\[3Ex]

Nilufar: \textit{Ich verstehe den Zuweisungs-Operator nicht. Wie sieht der aus und was macht der? Gebe mal ein Beispiel!} \textbf{(1 Punkt)}\\[1ex]
Sie antworten:\\[3Ex]

Nilufar: \textit{Ich verstehe den Pipe-Operator nicht. Wie sieht der aus und was macht der? Gebe mal ein Beispiel!} \textbf{(1 Punkt)}\\[1ex]
Sie antworten:\\[3Ex]

Nilufar: \textit{In R gibt es Objekte, Wörter und Funktionen. Wie unterscheiden sich diese voneinander?} \textbf{(1 Punkt)}\\[1ex]
Sie antworten:\\[3Ex]

Nilufar: \textit{Jetzt sehe ich wieder diese Tilde ($\sim$) in R. Wo brauchen wir diese denn nochmal?} \textbf{(1 Punkt)}\\[1ex]
Sie antworten:\\[3Ex]

Nilufar: \textit{Was macht überhaupt dieses \texttt{c()} hier überall im \Rlogo Code?} \textbf{(1 Punkt)}\\[1ex]
Sie antworten:\\[3Ex]

Nilufar: \textit{Ich verstehe den Unterschied zwischen \texttt{library()} und \texttt{Packages} nicht. Warum gibt es die?} \textbf{(1 Punkt)}\\[1ex]
Sie antworten:\\[3Ex] 
\clearpage
% -----------------------------------------------------------------------

\section{Aufgabe \hfill (9 Punkte)}



 
%% --------------------------------------------------------------------
\ifcollection
\begin{flushright}
\tiny\vspace{-3Ex}
\textbf{\examinhaltstart}
\exammodulelanddaten $\;\bullet$
\exammodulestatversuch $\;\bullet$
\exammodulebiostat
\vspace{-4Ex}
\end{flushright}
\begin{minipage}[t]{0.5\textwidth}
\includegraphics[width = 1.3cm]{/Users/kruppajo/work/GitHub/exam/avatare/Alex.png}
\end{minipage}
\begin{minipage}[t]{0.5\textwidth}
\hfill
\href{https://www.youtube.com/playlist?list=PLe51bCp9JvEFUnFqaJG5aRmON9i1ZbOYC}{\includegraphics[width = 2cm]{img/youtube}}
\end{minipage}
\vspace{1ex}
\fi
%% --------------------------------------------------------------------



\ifcollection
\paragraph{Fortgeschrittene Kenntnisse der Programierung in \Rlogo}
\fi

'Hm...am Ende ist dann \Rlogo eigentlich gar nicht so schwer, wenn ich Hilfe habe.', meint  Alex stolz und lacht Sie an. Nur leider kennt er sich überhaupt nicht mit \Rlogo aus! Das heißt, Sie müssen hier einmal Rede und Antwort stehen und helfen. Sonst wird es für Alex dann in seiner Hausarbeit nichts mit der Auswertung und Abgabe. Das kann auch keine Lösung für Alex und Sie sein. Immerhin haben Sie schon so viel gelernt.\\[1Ex]

Alex fragt: \textit{Wie nennt sich das Datenformat in R? Bitte mit kurzem Beispiel! \textbf{(1 Punkt)}}\\[1ex]
Sie antworten:\\[6.5Ex]

Alex fragt: \textit{Man kann doch die Funktion \texttt{emmeans()} von Varianzhomogenität auf Varianzheterogenität umstellen. Wie ging das noch gleich? \textbf{(1 Punkt)}}\\[1ex]
Sie antworten:\\[6.5Ex]

Alex fragt: \textit{Datumsangaben sind schwierig, da es nur ein gültiges Format gibt, was zwischen Programmen funktioniert. Wie lautet das Format? \textbf{(1 Punkt)}}\\[1ex]
Sie antworten:\\[6.5Ex]

Alex fragt: \textit{Ich baue mir ja meinen Datensatz in Excel. Was muss ich da im Bezug auf die Namen der Spalten beachten? \textbf{(1 Punkt)}}\\[1ex]
Sie antworten:\\[6.5Ex]

Alex fragt: \textit{Wie ist noch gleich die Reihenfolge der Funktionen für die Erstellung eines CLD in R. Welche Funktionen brauche ich da? \textbf{(2 Punkte)}}\\[1ex]
Sie antworten:\\[6.5Ex]

Alex fragt: \textit{Ich habe hier einen Gruppenvergleich als Expriment durchgeführt. Wenn ich jetzt die Daten in R reingeladen habe, was muss ich dann als erstes machen und warum nochmal? \textbf{(2 Punkte)}}\\[1ex]
Sie antworten:\\[6.5Ex]

Alex fragt: \textit{Ich habe den Faktor $f_1$ und $f_2$ und möchte den Faktor $f_1$ getrennt in jedem Level des Faktors $f_2$ auszuwerten. Wie geht das in der Funktion \texttt{emmeans()}? \textbf{(1 Punkt)}}\\[1ex]
Sie antworten:\\[6.5Ex]



 
\clearpage
% -----------------------------------------------------------------------
\part{Deskriptive Statistik \& Explorative Datenanalyse}
% -----------------------------------------------------------------------

\section{Aufgabe \hfill (9 Punkte)}

\textit{Geben Sie grundsätzlich Formeln und Rechenweg zur Lösung der Teilaufgaben mit an!} \\[1Ex]
 

 
%% --------------------------------------------------------------------
\ifcollection
\begin{flushright}
\tiny\vspace{-3Ex}
\textbf{\examinhaltstart}
\exammodulemathstat $\;\bullet$
\exammodulestat $\;\bullet$
\exammodulestatbbv $\;\bullet$
\exammodulestatversuch\\
\exammodulelanddaten $\;\bullet$
\exammodulebiostat
\vspace{-4Ex}
\end{flushright}
\begin{minipage}[t]{0.5\textwidth}
\includegraphics[width = 1.3cm]{/Users/kruppajo/work/GitHub/exam/avatare/Jonas.png}
\end{minipage}
\begin{minipage}[t]{0.5\textwidth}
\hfill
\href{https://youtu.be/GNoORO7GfAE}{\includegraphics[width = 2cm]{img/youtube}}
\end{minipage}
\vspace{-3ex}
\fi
%% --------------------------------------------------------------------



\ifcollection
\paragraph{Zerforschen des Barplots}
\fi

Jonas steht vor einem ersten Problem, denn wenn es nach seiner Betreuerin geht, soll er in einem einem Freilandversuch Erdbeeren auswertet. Soweit eigentlich alles passend. Besser wäre was anderes gewesen. Stricken. Ein wunderbares Hobby um sich drin zu verlieren und Abstand zu bekommen. Jonas denkt gerne über Stricken nach. Das heißt erstmal überlegen für Jonas. Aus den Boxen wummert Iron Maiden und sein Mund ist verklebt von Snickers. 'Herrlich', denkt Jonas. Die Behandlung werden verschiedene Düngestufen ($ctrl$, $low$ und $high$) sein. In seiner Exceldatei wird er den Outcome ($Y$) \textit{Proteingehalt} als \textit{protein} aufnehmen. Vorab soll Jonas aber eimal die folgenden Barplots seiner Betreuerin nachbauen, damit er den \Rlogo Code schonmal für später vorliegen hat. Damit geht das Problem schon los. Wenn die Erschöpfung nicht wäre, ja dann wäre wohl vieles möglich für Jonas! Aber so..



{\centering \includegraphics[width=\maxwidth]{img/barplot-02-1} 

}




Leider kennt sich Jonas mit der Erstellung von Barplots in \Rlogo nicht aus. Deshalb braucht er bei der Visualisierung Ihre Hilfe!

\begin{enumerate}
\item Formulieren Sie die wissenschaftliche Fragestellung! \textbf{(1 Punkt)}
\item Erstellen Sie eine Tabelle mit den statistischen Maßzahlen der drei Barplots! \textit{Beachten Sie die korrekte Darstellungsform der statistischen Maßzahlen!} \textbf{(4 Punkte)}
\item Erstellen Sie einen beispielhaften Datensatz im \Rlogo üblichen Format, aus dem die drei Barplots \textit{möglicherweise} erstellt wurden! \textbf{(2 Punkte)}
\item Kann Jonas einen Unterschied zwischen den Behandlungen erwarten? Begründen Sie Ihre Antwort! \textbf{(2 Punkte)}
\end{enumerate} 
\clearpage
% -----------------------------------------------------------------------

\section{Aufgabe \hfill (8 Punkte)}

\textit{Geben Sie grundsätzlich Formeln und Rechenweg zur Lösung der Teilaufgaben mit an!} \\[1Ex]
 

 
%% --------------------------------------------------------------------
\ifcollection
\begin{flushright}
\tiny\vspace{-3Ex}
\textbf{\examinhaltstart}
\exammodulemathstat $\;\bullet$
\exammodulestat $\;\bullet$
\exammodulelanddaten $\;\bullet$
\exammodulestatbbv 
\vspace{-4Ex}
\end{flushright}
\begin{minipage}[t]{0.5\textwidth}
\includegraphics[width = 1.3cm]{/Users/kruppajo/work/GitHub/exam/avatare/Yuki.png}
\end{minipage}
\begin{minipage}[t]{0.5\textwidth}
\hfill
\href{https://youtu.be/UabXe73G-7w}{\includegraphics[width = 2cm]{img/youtube}}
\end{minipage}
\vspace{-3ex}
\fi
%% --------------------------------------------------------------------



\ifcollection
\paragraph{Visualisierung des Barplots}
\fi

Yuki steht vor einem ersten Problem, denn wenn es nach seiner Betreuerin geht, soll er in einem einem Feldexperiment Erdbeeren auswertet. Soweit eigentlich alles passend. Besser wäre was anderes gewesen. Am Ende dann doch besser Orchideen. Wunderbar. Eine echte Ablenkung für Yuki. Die Behandlung waren verschiedene Substrattypen ($torf$, $40p60n$ und $70p30n$). In seiner Exceldatei hat er den Outcome ($Y$) \textit{Trockengewicht} als \textit{drymatter} aufgenommen. Nun soll Yuki die Daten eimal als Barplots in einer Präsentation visualisieren, damit seiner Betreuerin wieder klar wird, was er eigentlich nochmal gemacht hat und was für ein Ergbnis in einem statistischen Test zu erwarten wäre. Wäre da nicht noch etwas. Eine echte Herausforderung für ihn war schon immer die Faulheit gewesen. Ein leidiges Lied. Aber egal. Einfach mal raus um zu Boldern. Ohne Ziel und Uhr. Das ist was für Yuki.

\begin{table}[!h]
\centering
\begin{tabular}{cc}
\toprule
treatment & drymatter\\
\midrule
torf & 29.4\\
40p60n & 34.4\\
70p30n & 27.6\\
70p30n & 27.3\\
70p30n & 19.5\\
\addlinespace
70p30n & 22.7\\
40p60n & 28.5\\
40p60n & 33.3\\
40p60n & 51.5\\
torf & 37.4\\
\addlinespace
40p60n & 33.2\\
torf & 43.2\\
\bottomrule
\end{tabular}
\end{table}



Leider kennt sich Yuki mit der Erstellung von Barplots nicht aus. Deshalb braucht er bei der Visualisierung Ihre Hilfe!

\begin{enumerate}
\item Formulieren Sie die wissenschaftliche Fragestellung! \textbf{(1 Punkt)}
\item Zeichnen Sie in \underline{einer} Abbildung die Barplots für die Behandlung von Erdbeeren! Beschriften Sie die Achsen entsprechend! \textbf{(4 Punkte)}
\item Beschriften Sie \underline{einen} Barplot mit den gängigen statistischen Maßzahlen! \textbf{(2 Punkte)}
\item Wenn Yuki \underline{keinen} Effekt zwischen den Behandlungen von Erdbeeren erwarten würde, wie sehen dann die Barplots aus? \textit{Antworten Sie mit einer Skizze der Barplots!}
  \textbf{(1 Punkt)}
\end{enumerate} 
\clearpage
% -----------------------------------------------------------------------

\section{Aufgabe \hfill (9 Punkte)}

\textit{Geben Sie grundsätzlich Formeln und Rechenweg zur Lösung der Teilaufgaben mit an!} \\[1Ex]
 

 
%% --------------------------------------------------------------------
\ifcollection
\begin{flushright}
\tiny\vspace{-3Ex}
\textbf{\examinhaltstart}
\exammodulemathstat $\;\bullet$
\exammodulestat $\;\bullet$
\exammodulestatbbv $\;\bullet$
\exammodulestatversuch\\
\exammodulelanddaten $\;\bullet$
\exammodulebiostat
\vspace{-4Ex}
\end{flushright}
\begin{minipage}[t]{0.5\textwidth}
\includegraphics[width = 1.3cm]{/Users/kruppajo/work/GitHub/exam/avatare/Tina.png}
\end{minipage}
\begin{minipage}[t]{0.5\textwidth}
\hfill
\href{https://youtu.be/MyrNZ46i3uY}{\includegraphics[width = 2cm]{img/youtube}}
\end{minipage}
\vspace{-3ex}
\fi
%% --------------------------------------------------------------------



\ifcollection
\paragraph{Zerforschen des Boxplots}
\fi

Eine echte Herausforderung für sie war schon immer die Wut gewesen. Ein leidiges Lied. Deshalb gilt anschauen, was andere vor einem gemacht haben. Für Tina ist es eine Möglichkeit schneller ans Ziel zu gelangen. Tina soll in ihrer Hausarbeit Brokoli untersuchen. Die Behandlung in ihrer Hausarbeit werden verschiedene Genotypen ($AA$, $AB$ und $BB$) sein. Erheben wird Tina als Messwert ($Y$) \textit{Trockengewicht} benannt als \textit{drymatter} in ihrer Exceldatei. Von ihrer Betreuerin erhält sie nun folgende Abbildung von Boxplots, die sie erstmal zur Übung nachbauen soll, bevor sie mit dem eigentlichen Versuch beginnt. Aber nur in passender Atmospäre! Schon dutzende Male gesehen: Indiana Jones. Aber immer noch großartig zusammen mit Katjes.



{\centering \includegraphics[width=\maxwidth]{img/boxplot-02-zer-1} 

}




Leider kennt sich Tina mit der Erstellung von Boxplots in \Rlogo nicht aus. Deshalb braucht sie bei der Visualisierung Ihre Hilfe!

\begin{enumerate}
\item Formulieren Sie die wissenschaftliche Fragestellung! \textbf{(1 Punkt)}
\item Erstellen Sie eine Tabelle mit den statistischen Maßzahlen aus der obigen Abbildung der drei Boxplots! \textit{Beachten Sie die korrekte Darstellungsform der statistischen Maßzahlen!} \textbf{(2 Punkte)}
\item Beschriften Sie \textit{einen} der Boxplots mit den gängigen statistischen Maßzahlen! \textbf{(2 Punkte)}
\item Erstellen Sie einen beispielhaften Datensatz, aus dem die drei Boxplots \textit{möglicherweise} erstellt wurden, im \Rlogo üblichen Format! \textbf{(2 Punkte)}
\item Kann Tina einen Unterschied zwischen den Behandlungen erwarten? Begründen Sie Ihre Antwort! \textbf{(2 Punkte)}
\end{enumerate} 
\clearpage
% -----------------------------------------------------------------------

\section{Aufgabe \hfill (9 Punkte)}

\textit{Geben Sie grundsätzlich Formeln und Rechenweg zur Lösung der Teilaufgaben mit an!} \\[1Ex]
 

 
%% --------------------------------------------------------------------
\ifcollection
\begin{flushright}
\tiny\vspace{-3Ex}
\textbf{\examinhaltstart}
\exammodulemathstat $\;\bullet$
\exammodulestat $\;\bullet$
\exammodulelanddaten $\;\bullet$
\exammodulestatbbv 
\vspace{-4Ex}
\end{flushright}
\begin{minipage}[t]{0.5\textwidth}
\includegraphics[width = 1.3cm]{/Users/kruppajo/work/GitHub/exam/avatare/Steffen.png}
\end{minipage}
\begin{minipage}[t]{0.5\textwidth}
\hfill
\href{https://youtu.be/ZbrvpCuZl_k}{\includegraphics[width = 2cm]{img/youtube}}
\end{minipage}
\vspace{-3ex}
\fi
%% --------------------------------------------------------------------



\ifcollection
\paragraph{Visualisierung des Boxplots}
\fi

Steffen steht vor einem ersten Problem, denn wenn es nach seinem Betreuer geht, soll er in einem einem Freilandversuch Kartoffeln auswertet. Soweit eigentlich alles passend. Besser wäre was anderes gewesen. Klemmbausteine. Ein wunderbares Hobby um sich drin zu verlieren und Abstand zu bekommen. Steffen denkt gerne über Klemmbausteine nach. Die Behandlung waren verschiedene Lichtstufen ($none$ und $600lm$). In seiner Exceldatei hat er den Endpunkt ($Y$) \textit{Proteingehalt} als \textit{protein} aufgenommen. Nun soll Steffen die Daten eimal als Boxplots in einer Präsentation visualisieren, damit seinem Betreuer wieder klar wird, was er eigentlich nochmal gemacht hat und was für ein Ergbnis in einem statistischen Test zu erwarten wäre. Anhand von Boxplots lässt sich eine Aussage über die Normalverteilung von $Y$ treffen. Wäre da nicht noch etwas. Wenn die Romantik nicht wäre, ja dann wäre wohl vieles möglich für Steffen! Aber so.. Aber egal. Steffen will später nochmal raus um zu Ringen. Druck ablassen, dass muss er auch.

\begin{table}[!h]
\centering
\begin{tabular}{cc}
\toprule
Behandlung & Messwert\\
\midrule
600lm & 26.0\\
none & 13.0\\
none & 15.9\\
none & 25.9\\
600lm & 44.8\\
\addlinespace
600lm & 34.3\\
none & 31.8\\
none & 19.2\\
none & 20.0\\
none & 18.7\\
\addlinespace
600lm & 36.5\\
600lm & 33.6\\
none & 27.9\\
none & 18.8\\
600lm & 39.7\\
\addlinespace
600lm & 18.7\\
600lm & 34.7\\
600lm & 36.8\\
none & 22.7\\
600lm & 37.9\\
\bottomrule
\end{tabular}
\end{table}



Leider kennt sich Steffen mit der Erstellung von Boxplots nicht aus. Deshalb braucht er bei der Visualisierung Ihre Hilfe!

\begin{enumerate}
\item Formulieren Sie die wissenschaftliche Fragestellung! \textbf{(1 Punkt)}
\item Zeichnen Sie in \textit{einer} Abbildung die beiden Boxplots für die zwei Behandlungen von Kartoffeln! Beschriften Sie die Achsen der wissenschaftlichen Fragestellung entsprechend! \textbf{(4 Punkte)} 
\item Wie ist Ihr Vorgehen, wenn Sie eine \textit{gerade} Anzahl an
  Beobachtungen pro Gruppe haben? \textbf{(1 Punkt)}
\item Beschriften Sie \textit{einen} der beiden Boxplots mit den gängigen
  statistischen Maßzahlen! \textbf{(2 Punkte)}
\item Wenn Sie \underline{einen} Effekt zwischen den Behandlungen von Kartoffeln erwarten würden, wie sehen dann die beiden Boxplots aus? \textit{Antworten Sie mit einer Skizze der Boxplots!} \textbf{(1 Punkt)}
\end{enumerate} 
\clearpage
% -----------------------------------------------------------------------

\section{Aufgabe \hfill (10 Punkte)}

\textit{Geben Sie grundsätzlich Formeln und Rechenweg zur Lösung der Teilaufgaben mit an!} \\[1Ex]
 

 
%% --------------------------------------------------------------------
\ifcollection
\begin{flushright}
\tiny\vspace{-3Ex}
\textbf{\examinhaltstart}
\exammodulemathstat $\;\bullet$
\exammodulestat $\;\bullet$
\exammodulestatbbv $\;\bullet$
\exammodulestatversuch $\;\bullet$
\exammodulebiostat
\vspace{-4Ex}
\end{flushright}
\begin{minipage}[t]{0.5\textwidth}
\includegraphics[width = 1.3cm]{/Users/kruppajo/work/GitHub/exam/avatare/Mark.png}
\end{minipage}
\begin{minipage}[t]{0.5\textwidth}
\hfill
\href{https://youtu.be/md9I_UV5_lE}{\includegraphics[width = 2cm]{img/youtube}}
\end{minipage}
\vspace{-3ex}
\fi
%% --------------------------------------------------------------------




\ifcollection
\paragraph{Visualisierung des Scatterplots}
\fi

Wenn es nach Mark ginge, wäre er schon längst fertig mit seiner Abschlussarbeit. Geht es aber nicht. Aus den Boxen wummert Andrea Berg und sein Mund ist verklebt von Marzipankugeln. 'Herrlich', denkt Mark. In seiner Abschlussarbeit hatte er ein Kreuzungsexperiment im Wendland durchgeführt. Nach der Meinung seiner Betreuerin sieht das jedoch etwas anders aus. Jetzt soll er doch noch eine explorative Datenanalyse für den Zusammenhang zwischen mittlerer Anzahl an weißen Blutkörperchen [LEU/ml] und Gewichtszuwachs in der 1LW in Milchvieh durchführen. Wie nervig! Eine echte Herausforderung für ihn war schon immer die Unsicherheit gewesen. Ein leidiges Lied. Da zwei kontinuierliche Variablen vorliegen, geht die explorative Datenanalyse leider nicht mit Boxplots oder Barplots. Dann was anderes. Wenn Columbo läuft, dann ist der Hamster nicht mehr da. Aber jetzt braucht er mal Entspannung!

\begin{table}[!h]
\centering
\begin{tabular}{cc}
\toprule
Gewichtszuwachs in der 1LW & Mittlerer Anzahl an weißen Blutkörperchen [LEU/ml]\\
\midrule
1.0 & 3.9\\
5.5 & 6.3\\
0.5 & 0.5\\
4.0 & 4.7\\
10.5 & 8.0\\
\addlinespace
9.0 & 6.9\\
3.5 & 4.3\\
12.0 & 11.4\\
\bottomrule
\end{tabular}
\end{table}



Leider kennt sich Mark mit der Erstellung einer explorativen Datenanalyse für kontinuierliche Daten überhaupt nicht aus. Deshalb braucht er bei der Erstellung Ihre Hilfe!

\begin{enumerate}
\item Erstellen Sie eine Visualisierung für die Datentabelle. Beschriften Sie
  die Achsen entsprechend! \textbf{(4 Punkte)}
\item Schätzen Sie eine Grade durch die Punkte! \textbf{(1 Punkt)}
\item Beschriften Sie die Grade mit den gängigen statistischen Maßzahlen! Geben Sie die numerischen Zahlenwerte mit an! \textbf{(3 Punkte)}
\item Wenn \textit{ein} Effekt von $x$ auf $y$ vorhanden wäre, wie würde die Grade verlaufen und welche Werte würden die statistischen Maßzahlen annehmen? \textbf{(2 Punkt)}
\end{enumerate} 
\clearpage
% -----------------------------------------------------------------------

\section{Aufgabe \hfill (10 Punkte)}

\textit{Geben Sie grundsätzlich Formeln und Rechenweg zur Lösung der Teilaufgaben mit an!} \\[1Ex]
 

 
%% --------------------------------------------------------------------
\ifcollection
\begin{flushright}
\tiny\vspace{-3Ex}
\textbf{\examinhaltstart}
\exammodulestat $\;\bullet$
\exammodulestatbbv 
\vspace{-4Ex}
\end{flushright}
\begin{minipage}[t]{0.5\textwidth}
\includegraphics[width = 1.3cm]{/Users/kruppajo/work/GitHub/exam/avatare/Steffen.png}
\end{minipage}
\begin{minipage}[t]{0.5\textwidth}
\hfill
\href{https://youtu.be/L4nSytEHcaE}{\includegraphics[width = 2cm]{img/youtube}}
\end{minipage}
\vspace{-3ex}
\fi
%% --------------------------------------------------------------------



\ifcollection
\paragraph{Visualisierung des Mosaicplots}
\fi

Das Verrückte ist, dass die Schlange Harry Potter wirklich liebt. Das ist Steffen sehr recht, denn er braucht Entspannung. Aber Ablenkung hilft nur begrenzt. 'Uff!', denkt sich Steffen. Jetzt hat er doch tatsächlich zwei kategoriale Variablen in seiner Hausarbeit gemessen. Zum einen die Behandlung Herbizideinsatz [ja/nein] und zum anderen die Messung Frischegewicht über Zielwert [ja/nein] im Kontext von Maiss. Hierfür hat er ein Feldexperiment im Oldenburger Land durchgeführt. Jetzt möchte Steffen die Daten einmal in einer explorativen Datenanalyse darstellen. Danach kann er dann über den passenden statistischen Test nachdenken. Dabei unterstützt sein Betreuer diesen Ansatz bevor es in der Datenanalyse weiter geht. So schön wie so gut. Eine echte Herausforderung für ihn war schon immer die Romantik gewesen. Ein leidiges Lied.



\vspace{1Ex}

\begin{center}
\begin{minipage}[t]{0.45\textwidth}
%\small
\begin{center}

\begin{tabular}{p{2.5cm}p{2.5cm}p{2.5cm}p{2.5cm}}
\toprule
Herbizideinsatz & Frischegewicht über Zielwert\\
\midrule
ja & nein\\
ja & ja\\
ja & ja\\
ja & ja\\
nein & ja\\
\addlinespace
nein & ja\\
ja & ja\\
ja & nein\\
ja & ja\\
nein & ja\\
\addlinespace
ja & ja\\
nein & ja\\
nein & ja\\
ja & nein\\
ja & ja\\
\bottomrule
\end{tabular}


\end{center}
\end{minipage}
\begin{minipage}[t]{0.45\textwidth}
%\small
\begin{center}

\begin{tabular}{p{2.5cm}p{2.5cm}p{2.5cm}p{2.5cm}}
\toprule
Herbizideinsatz & Frischegewicht über Zielwert\\
\midrule
nein & ja\\
nein & ja\\
nein & ja\\
nein & ja\\
ja & ja\\
\addlinespace
nein & ja\\
nein & ja\\
ja & nein\\
ja & ja\\
ja & ja\\
\addlinespace
nein & ja\\
ja & ja\\
nein & ja\\
ja & ja\\
nein & ja\\
\bottomrule
\end{tabular}


\end{center}
\end{minipage}
\end{center}

\vspace{2Ex}

Leider kennt sich Steffen mit der Erstellung einer explorativen Datenanalyse für kategoriale Daten überhaupt nicht aus. Deshalb braucht er bei der Erstellung Ihre Hilfe!

\begin{enumerate}
\item Stellen Sie den Zusammenhang zwischen den beiden kategorialen Variablen in einer zusammenfassenden Tabelle dar! \textbf{(3 Punkte)}
\item Berechnen Sie die Verhältnisse in der zusammenfassenden Tabelle! Welche Annahme haben Sie getroffen? \textbf{(2 Punkte)}
\item Visualisieren Sie den Zusammenhang zwischen den beiden kategorialen Variablen! \textbf{(3 Punkte)}
\item Wenn \textit{ein} Effekt von der Behandlung vorliegen würde, wie würde die Tabelle und die Visualisierung aussehen? \textbf{(2 Punkt)}
\end{enumerate} 
\clearpage
% -----------------------------------------------------------------------

\section{Aufgabe \hfill (8 Punkte)}

\textit{Geben Sie grundsätzlich Formeln und Rechenweg zur Lösung der Teilaufgaben mit an!} \\[1Ex]
 

 
%% --------------------------------------------------------------------
\ifcollection
\begin{flushright}
\tiny\vspace{-3Ex}
\textbf{\examinhaltstart}
\exammodulestatversuch $\;\bullet$
\exammodulebiostat
\vspace{-4Ex}
\end{flushright}
\begin{minipage}[t]{0.5\textwidth}
\includegraphics[width = 1.3cm]{/Users/kruppajo/work/GitHub/exam/avatare/Paula.png}
\end{minipage}
\begin{minipage}[t]{0.5\textwidth}
\hfill
\href{https://youtu.be/n0Wr7VaN3Ao}{\includegraphics[width = 2cm]{img/youtube}}
\end{minipage}
\vspace{-3ex}
\fi
%% --------------------------------------------------------------------



\ifcollection
\paragraph{Visualisierung des Histogramm für kategoriale Daten}
\fi

In einem Gespräch mit ihrer Betreuerin wird Paula gebeten seine Daten aus einem Stallexperiment mit Milchvieh in einem Histogramm darzustellen. Paula schmeißt noch eine Handvoll Smarties in ihren Rachen. Im Hintergrund klirrt leise der Spiegel zum Sound von White Lies. In ihrem Experiment hat er die dunklen Pigmentstörungen erst fotographiert und dann ausgezählt. Laut ihrer Betreuerin soll das Histogramm helfen, die Verteilung der die dunklen Pigmentstörungen zu bestimmen. Es wäre einfacher, wenn da nicht noch was wäre. Paula und der Perfektionismus, eine unendliche Geschichte mit kniffeligen Wendungen. Paula nickt im Takt von White Lies und bemerkt dabei gar nicht was die Ratte schon wieder anstellt.

\begin{center}
Die dunklen Pigmentstörungen: 5, 2, 5, 11, 10, 5, 2, 4, 3, 3, 4, 4, 3, 4, 5, 3, 4, 5, 5, 6, 4, 2, 5, 3, 7, 2, 2, 8, 8, 6
\end{center}

Leider kennt sich Paula mit der Erstellung von Histogrammen überhaupt nicht aus. Deshalb braucht sie bei der Erstellung Ihre Hilfe!

\begin{enumerate}
\item Zeichen Sie ein Histogramm um die Verteilung der Daten zu visualisieren! (\textbf{3 Punkte})
\item Beschriften Sie die Achsen der Abbildung! (\textbf{2 Punkte})
\item Ergänzen Sie die absoluten und relativen Häufigkeiten in der
  Abbildung! \textbf{(1 Punkt)}
\item Berechnen Sie aus den Daten die \textit{Wahrscheinlichkeit}
  mehr als die Anzahl 4 zu beobachten! \textbf{(1
    Punkt)}
\item Berechnen Sie aus den Daten die \textit{Chance} mehr
  als die Anzahl 4 zu beobachten! \textbf{(1 Punkt)}
\end{enumerate}

 
\clearpage
% -----------------------------------------------------------------------

\section{Aufgabe \hfill (8 Punkte)}

\textit{Geben Sie grundsätzlich Formeln und Rechenweg zur Lösung der Teilaufgaben mit an!} \\[1Ex]
 

 
%% --------------------------------------------------------------------
\ifcollection
\begin{flushright}
\tiny\vspace{-3Ex}
\textbf{\examinhaltstart}
\exammodulestatversuch $\;\bullet$
\exammodulebiostat
\vspace{-4Ex}
\end{flushright}
\begin{minipage}[t]{0.5\textwidth}
\includegraphics[width = 1.3cm]{/Users/kruppajo/work/GitHub/exam/avatare/Tina.png}
\end{minipage}
\begin{minipage}[t]{0.5\textwidth}
\hfill
\href{https://youtu.be/Q-kyhucwIXs}{\includegraphics[width = 2cm]{img/youtube}}
\end{minipage}
\vspace{-3ex}
\fi
%% --------------------------------------------------------------------



\ifcollection
\paragraph{Visualisierung des Histogramm für kontinuierliche Daten}
\fi

In ihrer Hausarbeit möchte Tina gerne die Daten aus einem Gewächshausexperiment mit Kartoffeln in einem Histogramm darstellen. Das Histogramm erlaubt ihr dabei Rückschlüsse auf die Verteilung über den Endpunkt ($Y$) zu treffen Aus den Boxen wummert Tocotronic und ihr Mund ist verklebt von Katjes. 'Herrlich', denkt Tina. In seinem Experiment hat Tina die mittleren Läsionen auf den Blättern gezählt. Es wäre einfacher, wenn da nicht noch was wäre. Tina und die Wut, eine unendliche Geschichte mit kniffeligen Wendungen. Wenn Tocotronic ertönt, dann sucht die Spinne schleunigst Schutz unter dem Sofa. Tina schüttelt den Kopf.

\begin{center}
Die mittleren Läsionen auf den Blättern: 9.9, 12.8, 9.3, 9.9, 8.6, 9.5, 5.3, 9.4, 10.6, 11, 13.3, 8.7, 9, 5.7, 9.6, 10.1, 5.9, 9.9, 8.1, 12.1, 12.3, 11.3
\end{center}

Leider kennt sich Tina mit der Erstellung von Histogrammen überhaupt nicht aus. Deshalb braucht sie bei der Erstellung Ihre Hilfe!

\begin{enumerate}
\item Zeichen Sie ein Histogramm um die Verteilung der Daten zu visualisieren! (\textbf{3 Punkte})
 \item Erläutern Sie Ihr Vorgehen um ein Histogramm für kontinuierliche Daten zu zeichnen!  (\textbf{2 Punkte})
\item Beschriften Sie die Achsen der Abbildung! (\textbf{2 Punkte})
\item Ergänzen Sie die relativen Häufigkeiten in der Abbildung! \textbf{(1 Punkt)}  
\end{enumerate}

 
\clearpage
% -----------------------------------------------------------------------

\section{Aufgabe \hfill (10 Punkte)}

\textit{Geben Sie grundsätzlich Formeln und Rechenweg zur Lösung der Teilaufgaben mit an!} \\[1Ex]
 

 
%% --------------------------------------------------------------------
\ifcollection
\begin{flushright}
\tiny\vspace{-3Ex}
\textbf{\examinhaltstart}
\exammodulestatversuch $\;\bullet$
\exammodulebiostat
\vspace{-4Ex}
\end{flushright}
\begin{minipage}[t]{0.5\textwidth}
\includegraphics[width = 1.3cm]{/Users/kruppajo/work/GitHub/exam/avatare/Alex.png}\hspace{-4mm}\includegraphics[width = 1.3cm]{/Users/kruppajo/work/GitHub/exam/avatare/Tina.png}
\end{minipage}
\begin{minipage}[t]{0.5\textwidth}
\hfill
\href{https://youtu.be/A4E9zBOltuU}{\includegraphics[width = 2cm]{img/youtube}}
\end{minipage}
\fi
%% --------------------------------------------------------------------



\ifcollection
\paragraph{Visualisierung von Verteilungen}
\fi

'Ich glaube, dass es sich hier wieder um so ein kryptisches Lernziel handelt, was nicht so gleich klar ist.', meint Alex und streichelt sanft die Katze. Das Tier versucht dem strammen Griff zu entkommen, gibt aber auf. Tina sieht sich sehr genau die drei liegenden Boxplots an. 'Du weißt doch wie es heißt, \textit{Frei ist, wer missfallen kann.}\footnote{Oschmann, A. (2024) Mädchen stärken: Stärken fördern, Selbstwert erhöhen und liebevoll durch Krisen begleiten. Goldegg Verlag}', merkt Alex nickend an. Das Ziel ist es zu verstehen, wie eine Verteilung anhand eines Boxplots bewertet werden kann. Alex und die Gefälligkeit machen die Sache nicht einfacher.



{\centering \includegraphics[width=\maxwidth]{img/desc-stat-11-1} 

}




Jetzt brauchen Alex und Tina Ihre Hilfe bei der Abschätzung einer Verteilung anhand von Boxplots um ihre Arbeit dann in diesem Semester noch abschließen zu können.

\begin{enumerate}
\item Zeichnen Sie über die Boxplots die entsprechende zugehörige Verteilung! \textbf{(3 Punkte)} 
\item Zeichnen Sie unter die Boxplots die entsprechende zugehörige Beobachtungen als Striche! \textbf{(3 Punkte)}
\item Wie viel Prozent der Beobachtungen fallen in das IQR? Ergänzen Sie die Abbildung entsprechend um den Bereich! \textbf{(2 Punkte)}
\item Wie viel Prozent der Beobachtungen fallen in $\bar{y} \pm 1s$ und $\bar{y} \pm 2s$  unter der Annahme einer Normalverteilung? \textbf{(2 Punkte)}
\end{enumerate} 
\clearpage
% -----------------------------------------------------------------------
\part{Statistisches Testen \& statistische Testtheorie}
% -----------------------------------------------------------------------  

\section{Aufgabe \hfill (9 Punkte)}


 
%% --------------------------------------------------------------------
\ifcollection
\begin{flushright}
\tiny
\textbf{\examinhaltstart}
\exammodulestat $\;\bullet$
\exammodulestatbbv $\;\bullet$
\exammodulestatversuch $\;\bullet$
\exammodulebiostat
\vspace{-4Ex}
\end{flushright}
\begin{minipage}[t]{0.5\textwidth}
\includegraphics[width = 1.3cm]{/Users/kruppajo/work/GitHub/exam/avatare/Jonas.png}\hspace{-4mm}\includegraphics[width = 1.3cm]{/Users/kruppajo/work/GitHub/exam/avatare/Tina.png}
\end{minipage}
\begin{minipage}[t]{0.5\textwidth}
\hfill
\href{https://youtu.be/KxDklbFEEHw}{\includegraphics[width = 2cm]{img/youtube}}
\end{minipage}
\fi
%% --------------------------------------------------------------------



\ifcollection
\paragraph{Grundgesamtheit und experimentelle Stichprobe}
\fi

'Grundlage des statistischen Testen ist das Verständnis von der Grundgesamtheit (eng. \textit{population} oder \textit{ground truth}) und der experimentellen Stichprobe (eng. \textit{sample}). ', liest Tina laut aus dem Skript vor. Jonas war kurz eingenickt und wird mit einem Stoß geweckt. 'Reiz dich zusammen und iss noch ein paar Katjes das hilft mir immer. Alleine komme ich hier nicht weiter.', tadelt Tina Jonas etwas zu forsch. 'War ne lange Nacht', mault Jonas. Beide sollen in ihrer Abschlussarbeit einen statistischen Test interpretieren und versuchen die Grundlagen zu wiederholen. Jonas war auf einem Konzert von Iron Maiden.

\vspace{1ex}

Leider kennen sich Tina und Jonas mit der Grundgesamtheit und der Stichprobe überhaupt nicht aus. Daher sind Sie gefragt!

\begin{enumerate}
\item Erklären Sie den Zusammenhang zwischen Stichprobe und Grundgesamtheit an einem Schaubild! Beschriften Sie das Schaubild entsprechend! \textit{Nutzen Sie hierfür als Veranschaulichung ein aussagekräftiges Beispiel!}  \textbf{(3 Punkte)}
\begin{enumerate}
\item Nennen Sie das statistische Verfahren um von einer Grundgesamtheit auf eine Stichprobe zu gelangen!  \textbf{(1 Punkt)}
\item Nennen Sie ein konkretes Beispiel zur Durchführung um von einer Grundgesamtheit auf eine Stichprobe zu gelangen! \textbf{(1 Punkt)}
\item Benennen Sie die statistische Eigenschaft, die zwischen Grundgesamtheit und Stichprobe vorliegen muss! \textbf{(1 Punkt)}
\end{enumerate}
\item Erweitern Sie das Schaubild um die Entstehung von $Pr(D|H_0)$! \textit{Nutzen Sie hierfür als Veranschaulichung ein aussagekräftiges Beispiel!}  \textbf{(3 Punkte)}
\end{enumerate}

 
\clearpage
% -----------------------------------------------------------------------

\section{Aufgabe \hfill (9 Punkte)}


 
%% --------------------------------------------------------------------
\ifcollection
\begin{flushright}
\tiny
\textbf{\examinhaltstart}
\exammodulestat $\;\bullet$
\exammodulestatbbv $\;\bullet$
\exammodulestatversuch $\;\bullet$
\exammodulebiostat
\vspace{-4Ex}
\end{flushright}
\begin{minipage}[t]{0.5\textwidth}
\includegraphics[width = 1.3cm]{/Users/kruppajo/work/GitHub/exam/avatare/Alex.png}\hspace{-4mm}\includegraphics[width = 1.3cm]{/Users/kruppajo/work/GitHub/exam/avatare/Nilufar.png}
\end{minipage}
\begin{minipage}[t]{0.5\textwidth}
\hfill
\href{https://youtu.be/XAE8BwNmCQ0}{\includegraphics[width = 2cm]{img/youtube}}
\end{minipage}
\fi
%% --------------------------------------------------------------------



\ifcollection
\paragraph{Das Nullritual - Die statistische Testtheorie}
\fi

'Das Känguruh hat mir gerade zugezwinkert. Das macht mir Angst', bemerkt Alex. Es reicht ja schon ein Problem. Alex und die Gefälligkeit, eine unendliche Geschichte mit kniffeligen Wendungen. Da braucht es nicht noch Wahnvorstellungen. 'Ich glaube nicht, das Känguruhs zwinkern. Aber was Wichtigeres. Wo hast du eigentlich meine Schnapspralinen hingetan? Wir haben nur noch Gummibärchen von dir. Bäh!', antwortet Nilufar. Beide sind im Zoo und wollen sich von der statistische Testheorie ablenken lassen. Eigentlich wollte ja Alex stoppen wie lange Tiere pinkeln\footnote{Yang, P. J., et al. (2014). Duration of urination does not change with body size. Proceedings of the National Academy of Sciences, 111(33), 11932-11937.}, scheiterte aber an einer Oma mit Stock, die die beiden beschimpfte.

\vspace{1ex}

Leider kennen sich Alex und Nilufar mit statistischen Testtheorie, auch Null-Ritual genannt, überhaupt nicht aus. Geschweige denn mit der Visualisierung als Kreuztabelle.  

\begin{enumerate}
\item Tragen Sie folgende statistische Fachbegriffe zur statistischen Testtheorie korrekt eine selbst erstellte Kreuztabelle ein! \textbf{(3 Punkte)}
  \begin{center}
  \begin{tabular}{cccc}
  H$_0$ falsch & H$_0$ wahr & $\alpha$-Fehler & $\beta$-Fehler \\
  \end{tabular}
  \end{center}
\item Ergänzen Sie Ihre erstellte Kreuztabelle um vier weitere, passende Fachbegriffe zur statistischen Testtheorie! \textbf{(2 Punkte)}
\end{enumerate}

Die Entscheidungsfindung durch einen statistischen Test kann auch durch die Analogie zu einem Feuermelder abgebildet werden. Dabei symbolisiert der Feuermelder den statistischen Test und es soll getestet werden, ob ein Feuer ausgebrochen ist.

\begin{enumerate}
  \setcounter{enumi}{2}    
\item In der Analogie des Feuermelders, wie lautet der $\alpha$-Fehler? \textbf{(1 Punkt)}
\item In der Analogie des Feuermelders, wie lautet der $\beta$-Fehler? \textbf{(1 Punkt)}
\item Wenn der Feuermelder einmal pro Tag messen würde, wie oft würde der Feuermelder mit einem $\alpha$ von 5\% in einem halben Jahr Alarm schlagen? Begründen Sie Ihre Antwort! \textbf{(2 Punkte)}
\end{enumerate}



 
\clearpage
% -----------------------------------------------------------------------

\section{Aufgabe \hfill (9 Punkte)}

\textit{Geben Sie grundsätzlich Formeln und Rechenweg zur Lösung der Teilaufgaben mit an!} \\[1Ex]


 
%% --------------------------------------------------------------------
\ifcollection
\begin{flushright}
\tiny\vspace{-3Ex}
\textbf{\examinhaltstart}
\exammodulemathstat $\;\bullet$
\exammodulestat $\;\bullet$
\exammodulestatbbv $\;\bullet$
\exammodulestatversuch $\;\bullet$
\exammodulebiostat
\vspace{-4Ex}
\end{flushright}
\begin{minipage}[t]{0.5\textwidth}
\includegraphics[width = 1.3cm]{/Users/kruppajo/work/GitHub/exam/avatare/Jessica.png}\hspace{-4mm}\includegraphics[width = 1.3cm]{/Users/kruppajo/work/GitHub/exam/avatare/Steffen.png}
\end{minipage}
\begin{minipage}[t]{0.5\textwidth}
\hfill
\href{https://youtu.be/2LQzyQCy2FI}{\includegraphics[width = 2cm]{img/youtube}}
\end{minipage}
\fi
%% --------------------------------------------------------------------



\ifcollection
\paragraph{Visualisierung der Teststatistik $\boldsymbol{T_D}$ und dem p-Wert}
\fi

Jessica und Steffen wollten eigentlich einen Flug nach Mallorca buchen, sind jetzt aber dann doch dazu übergegangen nochmal die Aufgaben für die Statistikklausur durchzugehen. 'Kannst du mir nochmal an einer Visualisierung erklären, wie der Zusammenhang zwischen der Teststatistik aus den Daten $T_D$ und dem p-Wert ist? Ich habe hier zig Fachbegriffe, kriege die abr nicht zusammen...', fragt Jessica. Steffen zuckt mit den Schultern. So genau hatte Steffen jetzt auch nicht aufgepasst. Da hilft dann eventuell das YouTube Video weiter. Jessica mapmft Schokobons und fragt sich, was das alles soll.

\vspace{1ex}

Leider kennen sich Jessica und Steffen mit der Visualisierung der Teststatistik $T_D$ und dem p-Wert überhaupt nicht aus und brauchen dahr Ihre Hilfe!

\vspace{1ex}

\textit{Beachten Sie, dass im Folgenden \underline{keine numerisch korrekte Darstellung} verlangt wird! Es gilt Erkennbarkeit vor Genauigkeit!}

\begin{enumerate}
\item Ergänzen Sie eine beschriftete $x$-Achse! \textbf{(1 Punkt)}
\item Ergänzen Sie "`$\bar{y}_1 = \bar{y}_2$"'! \textbf{(1 Punkt)} 
\item Ergänzen Sie "`$95\%$"'! \textbf{(1 Punkt)}
\item Zeichnen Sie $T_{\alpha=5\%}$ in die Abbildung! \textbf{(1 Punkt)} 
\item Zeichnen Sie das Signifikanzniveau $\alpha$ in die Abbildung! Begründen Sie Ihre Antwort! \textbf{(2 Punkte)} 
\item Zeichnen Sie $+T_{D}$ in die Abbildung! \textbf{(1 Punkt)}
\item Zeichnen Sie einen signifikant p-Wert in die Abbildung! Begründen Sie Ihre Antwort! \textbf{(2 Punkte)}   
\end{enumerate}



{\centering \includegraphics[width=\maxwidth]{img/statistisches-testen-3-1} 

}


 
\clearpage
% -----------------------------------------------------------------------

\section{Aufgabe \hfill (10 Punkte)}


 
%% --------------------------------------------------------------------
\ifcollection
\begin{flushright}
\tiny
\textbf{\examinhaltstart}
\exammodulestatversuch $\;\bullet$
\exammodulebiostat
\vspace{-4Ex}
\end{flushright}
\begin{minipage}[t]{0.5\textwidth}
\includegraphics[width = 1.3cm]{/Users/kruppajo/work/GitHub/exam/avatare/Jessica.png}\hspace{-4mm}\includegraphics[width = 1.3cm]{/Users/kruppajo/work/GitHub/exam/avatare/Steffen.png}
\end{minipage}
\begin{minipage}[t]{0.5\textwidth}
\hfill
\href{https://youtu.be/6w_KH7t6bp0}{\includegraphics[width = 2cm]{img/youtube}}
\end{minipage}
\fi
%% --------------------------------------------------------------------



\ifcollection
\paragraph{Visualisierung des 95\% Konfidenzintervalls}
\fi

'So, was haben wir gemacht? Wir haben einen t-test für den Vergleich der Mittelwerte gerechnet.', meint Jessica. Steffen schaut fragend. 'Hatten wir nicht alles zu einer Kontrolle verglichen? Das war doch so!', ruft Steffen laut aus. 'Wir haben doch als Messwert \textit{Wasserverbrauch der Bewässerung} erhoben.', stellt Jessica fest. Jetzt haben beide das Problem, die möglichen 95\% Konfidenzintervalle zu interpretieren.

\vspace{1ex}

Leider kennen sich Jessica und Steffen mit der Visualisierung des 95\% Konfidenzintervall überhaupt nicht aus. 

\begin{enumerate}
\item Beschriften Sie die untenstehende Abbildung mit der Signifikanzschwelle! Begründen Sie Ihre Antwort! \textbf{(2 Punkte)}
\item Ergänzen Sie eine \textit{in den Kontext passende} Relevanzschwelle! Begründen Sie Ihre Antwort! \textbf{(2 Punkte)} 
\item Skizzieren Sie in die untenstehende Abbildung sechs einzelne Konfidenzintervalle (a-f) mit den
  jeweiligen Eigenschaften! \textbf{(6 Punkte)}
  \begin{itemize}
  \item[(a)] Ein signifikantes, relevantes 90\% Konfidenzintervall. 	
  \item[(b)] Ein nicht signifikantes, nicht relevantes 95\% Konfidenzintervall 	
  \item[(c)] Ein signifikantes, relevantes 95\% Konfidenzintervall 	
  \item[(d)] Ein signifikantes, nicht relevantes 95\% Konfidenzintervall 
  \item[(e)] Ein 95\% Konfidenzintervall mit h{"o}herer Varianz $s_p$ in der Stichprobe als der Rest der 95\% Konfidenzintervalle
  \item[(f)] Ein 95\% Konfidenzintervall mit niedriger Varianz $s_p$ in der Stichprobe als der Rest 95\% der Konfidenzintervalle
  \end{itemize}
\end{enumerate}

\begin{center}
  \includegraphics[height = 10cm]{/Users/kruppajo/work/GitHub/exam/question/img/statistisches-testen-04}
\end{center}


 
\clearpage
% -----------------------------------------------------------------------

\section{Aufgabe \hfill (10 Punkte)}

\textit{Geben Sie grundsätzlich Formeln und Rechenweg zur Lösung der Teilaufgaben mit an!} \\[1Ex]


 
%% --------------------------------------------------------------------
\ifcollection
\begin{flushright}
\tiny\vspace{-3Ex}
\textbf{\examinhaltstart}
\exammodulestatversuch $\;\bullet$
\exammodulebiostat
\vspace{-4Ex}
\end{flushright}
\begin{minipage}[t]{0.5\textwidth}
\includegraphics[width = 1.3cm]{/Users/kruppajo/work/GitHub/exam/avatare/Mark.png}\hspace{-4mm}\includegraphics[width = 1.3cm]{/Users/kruppajo/work/GitHub/exam/avatare/Nilufar.png}
\end{minipage}
\begin{minipage}[t]{0.5\textwidth}
\hfill
\href{https://youtu.be/SejyJ0K20jA}{\includegraphics[width = 2cm]{img/youtube}}
\end{minipage}
\fi
%% --------------------------------------------------------------------



\ifcollection
\paragraph{Zusammenhang zwischen dem Effekt, der Streuung sowie der Fallzahl}
\fi

An einem kalten Dezembermorgen haben sich Mark und Nilufar zum Lernen verabredet. Eine große Kanne Tee und Berge von Marzipankugeln warten darauf gegessen zu werden. Mark liest laut vor:\begin{quote}
                 \textit{
                 Beim statistischen Testen gibt es einen Zusammenhang zwischen dem Effekt, der Streuung sowie der Fallzahl. Gegeben sei die Formel für den Student t-Test auf den die folgenden Überlegungen basieren sollen. Welche Auswirkung hat die Änderungen der jeweiligen statistischen Maßzahl des Effekts $\Delta$, der Streuung $s$ und der Fallzahl $n$ auf die Teststistik $T_{D}$, den p-Wert $Pr(D|H_0)$ sowie dem Konfidenzintervall $KI_{1-\alpha}$?
                 }
                 \end{quote}Nilufar hebt die Augenbraue. 'Irgendwie sagt mir die Aufgabe jetzt mal so gar nichts. Was soll da gemacht werden?', merkt Nilufar an und ergänzt: 'Schauen wir doch erstmal zur Entspannung Star Trek, den Film habe ich extra nochmal mitgebracht.' Mark ist der Idee nicht abgeneigt und auch der Hamster kommt unter dem Sofa hervor um mitzuschauen. 

\vspace{1ex}

Leider kennen sich Mark und Nilufar mit dem Zusammenhang zwischen dem Effekt, der Streuung sowie der Fallzahl überhaupt nicht aus. 


\begin{enumerate}
\item Visualisieren Sie den Zusammenhang zwischen der Teststatiatik $T_{D}$ und dem p-Wert $Pr(D|H_0)$ für sich verändernde $T_{D}$-Werte!\textit{Geben Sie dafür ein numerisches Beispiel in dem Sie drei $T_{D}$-Werte und deren Einfluss auf den p-Wert vergleichen!} \textbf{(3 Punkte)}  
\item  Füllen Sie die untenstehende Tabelle aus in dem Sie die Änderung der statistischen Maßzahlen auf die Teststatistik, den p-Wert sowie das Konfidenzintervall in \textit{einem} Wort oder Symbol beschreiben! \textbf{(4 Punkte)}
\begin{center}
  \large
  \begin{tabular}[c]{l|c|c|c|l|c|c|c}
    & $T_{D}$ & $Pr(D|H_0)$ & $KI_{1-\alpha}$ & & $T_{D}$ & $Pr(D|H_0)$ & $KI_{1-\alpha}$\strut\\ 
    \hline
    \textbf{$\Delta\; \uparrow$} & \hspace{1.8cm} & \hspace{1.8cm}  & \hspace{1.8cm} & \textbf{
                                                          $\Delta\; \downarrow$} &
                                                                          \hspace{1.8cm} & \hspace{1.8cm}  & \hspace{1.8cm}\strut\\
    \hline
        \textbf{$s\; \uparrow$} & \hspace{1.8cm} & \hspace{1.8cm}  & \hspace{1.8cm} & \textbf{
                                                          $s\; \downarrow$} &
                                                                          \hspace{1.8cm}
                                                & \hspace{1.8cm}  & \hspace{1.8cm}\strut\\
    \hline
        \textbf{$n\; \uparrow$} & \hspace{1.8cm} & \hspace{1.8cm}  & \hspace{1.8cm} & \textbf{
                                                          $n\; \downarrow$} &
                                                                          \hspace{1.8cm}
                                                & \hspace{1.8cm}  & \hspace{1.8cm}\strut\\
    \hline
  \end{tabular}
\end{center}
\item Visualisieren Sie ein 95\%-iges Konfidenzintervall im Vergleich zu einem 90\%-igen Konfidenzintervall! Begründen Sie Ihre Visualisierung anhand der Formel des Konfidenzintervalls des t-Tests mathematisch! \textbf{(3 Punkte)} 
\end{enumerate} 
\clearpage
% -----------------------------------------------------------------------
\part{Der t-Test}
% -----------------------------------------------------------------------

\section{Aufgabe \hfill (9 Punkte)}

\textit{Geben Sie grundsätzlich Formeln und Rechenweg zur Lösung der Teilaufgaben mit an!} \\[1Ex]
 

 
%% --------------------------------------------------------------------
\ifcollection
\begin{flushright}
\tiny\vspace{-3Ex}
\textbf{\examinhaltstart}
\exammodulemathstat $\;\bullet$
\exammodulestat $\;\bullet$
\exammodulestatbbv 
\vspace{-4Ex}
\end{flushright}
\begin{minipage}[t]{0.5\textwidth}
\includegraphics[width = 1.3cm]{/Users/kruppajo/work/GitHub/exam/avatare/Mark.png}
\end{minipage}
\begin{minipage}[t]{0.5\textwidth}
\hfill
\href{https://youtu.be/zgpw9GC0plk}{\includegraphics[width = 2cm]{img/youtube}}
\end{minipage}
\vspace{-3ex}
\fi
%% --------------------------------------------------------------------



\ifcollection
\paragraph{Berechnung des Student t-Test \underline{oder} Welch t-Test}
\fi

Der t-Test. Mark erschaudert. Mark und die Unsicherheit, eine unendliche Geschichte mit kniffeligen Wendungen. Ein mächtiges Werkzeug ist der t-Test in den Händen desjenigen, der einen normalverteilten Endpunkt ($Y$) hat. Aber erstmal überhaupt den t-Test rechnen können. Wie sah das Experiment von Mark überhaupt aus? Aus den Boxen wummert Andrea Berg und sein Mund ist verklebt von Marzipankugeln. 'Herrlich', denkt Mark. Mark hat ein Kreuzungsexperiment mit Schweinen durchgeführt um eine neue technische Versuchsanlage zu testen. Bei dem Pilotexperiment mit sehr geringer Fallzahl $(n_1 = n_2 = 3)$ wurde die Behandlung Lüftungssystem ($keins$ und $vorhanden$) an den Schweinen getestet und dabei wurde geschaut, ob der Versuch überhaupt technisch klappen könnte. Gemessen hat Mark dann als Messwert Gewichtszuwachs in der 1LW [\%/kg]. Warum der Versuch im Emsland für seine Abschlussarbeit stattfinden musste, ist ihm bis heute ein Rätsel. Egal. Gibt es jetzt einen Zusammenhang zwischen der Behandlung und Gewichtszuwachs in der 1LW [\%/kg]?

\begin{table}[!h]
\centering
\begin{tabular}{cc}
\toprule
Lüftungssystem & Gewichtszuwachs\\
\midrule
keins & 18.1\\
vorhanden & 21.9\\
keins & 17.2\\
keins & 16.3\\
vorhanden & 18.8\\
\addlinespace
vorhanden & 18.1\\
\bottomrule
\end{tabular}
\end{table}



Leider kennt sich Mark mit der Berechnung eines Welch t-Tests überhaupt nicht aus. Deshalb braucht er bei der Berechnung Ihre Hilfe!

\begin{enumerate}
  \item Formulieren Sie das statistische Hypothesenpaar! \textbf{(1 Punkt)}
  \item Bestimmen Sie die Teststatistik $T_{D}$ eines Welch t-Tests! \textbf{(3 Punkte)}
  \item Treffen Sie mit $T_{\alpha = 5\%} = 1.96$ eine Aussage zur Nullhypothese! Begründen Sie Ihre Antwort! \textbf{(2 Punkte)}
  \item Berechnen Sie den Effekt des Welch t-Tests! \textbf{(1 Punkt)}
  \item Formulieren Sie eine Antwort an Mark über das Ergebnis Ihrer statistischen Analyse! \textbf{(2 Punkte)}
\end{enumerate} 
\clearpage
% -----------------------------------------------------------------------

\section{Aufgabe \hfill (12 Punkte)}

\textit{Geben Sie grundsätzlich Formeln und Rechenweg zur Lösung der Teilaufgaben mit an!} \\[1Ex]
 

 
%% --------------------------------------------------------------------
\ifcollection
\begin{flushright}
\tiny\vspace{-3Ex}
\textbf{\examinhaltstart}
\exammodulemathstat $\;\bullet$
\exammodulestat $\;\bullet$
\exammodulestatbbv 
\vspace{-4Ex}
\end{flushright}
\begin{minipage}[t]{0.5\textwidth}
\includegraphics[width = 1.3cm]{/Users/kruppajo/work/GitHub/exam/avatare/Jessica.png}
\end{minipage}
\begin{minipage}[t]{0.5\textwidth}
\hfill
\href{https://youtu.be/eGoLNgE-WCo}{\includegraphics[width = 2cm]{img/youtube}}
\end{minipage}
\vspace{-3ex}
\fi
%% --------------------------------------------------------------------



\ifcollection
\paragraph{Berechnung des Student t-Test}
\fi

Der t-Test. Jessica erschaudert. Wenn der Mangel nicht wäre, ja dann wäre wohl vieles möglich für Jessica! Aber so.. Ein mächtiges Werkzeug ist der t-Test in den Händen desjenigen, der einen normalverteilten Endpunkt ($Y$) hat. Aber erstmal überhaupt den t-Test rechnen können. Wie sah das Experiment von Jessica überhaupt aus? 'Hm...', Schokobons und David Bowie. Das ist und bleibt die beste Kombination zum Nachdenken für Jessica. Jessica hat einen Versuch in einer Klimakammer mit Erbsen durchgeführt. Dabei wurde die Behandlung Bewässerungstypen ($low$ und $high$) an den Erbsen getestet. Gemessen hat Jessica dann als Messwert Chlorophyllgehalt (SPAD-502Plus) [SPAD]. Warum der Versuch im Emsland für ihrer Hausarbeit stattfinden musste, ist ihr bis heute ein Rätsel. Egal. Gibt es jetzt einen Zusammenhang zwischen der Behandlung und Chlorophyllgehalt (SPAD-502Plus) [SPAD]?

\begin{table}[!h]
\centering
\begin{tabular}{cc}
\toprule
Bewässerungstypen & Chlorophyllgehalt\\
\midrule
high & 39.0\\
high & 41.9\\
low & 27.3\\
high & 50.2\\
low & 27.0\\
\addlinespace
high & 46.3\\
low & 31.7\\
low & 22.1\\
low & 26.5\\
\bottomrule
\end{tabular}
\end{table}



Leider kennt sich Jessica mit der Berechnung eines t-Tests überhaupt nicht aus. Deshalb braucht sie bei der Berechnung Ihre Hilfe!

\begin{enumerate}
  \item Formulieren Sie die wissenschaftliche Fragestellung! \textbf{(1 Punkt)}
  \item Formulieren Sie das statistische Hypothesenpaar! \textbf{(1 Punkt)}
  \item Bestimmen Sie die Teststatistik $T_{D}$ eines Student t-Tests! \textbf{(3 Punkte)}
\item Treffen Sie mit $T_{\alpha = 5\%} = 1.84$ eine Aussage zur Nullhypothese! Begründen Sie Ihre Antwort! \textbf{(2 Punkte)}
\item Berechnen Sie den Effekt des Student t-Tests! \textbf{(1 Punkt)}
\item Wenn Sie \textit{einen} Unterschied zwischen den Behandlungsgruppen erwarten würden, wie groß wäre dann der \textit{mindeste} Effekt? Begründen Sie Ihre Antwort! \textbf{(2 Punkte)}
\item Formulieren Sie eine Antwort an Jessica über das Ergebnis Ihrer statistischen Analyse! \textbf{(2 Punkte)}
\end{enumerate} 
\clearpage
% -----------------------------------------------------------------------

\section{Aufgabe \hfill (12 Punkte)}

\textit{Geben Sie grundsätzlich Formeln und Rechenweg zur Lösung der Teilaufgaben mit an!} \\[1Ex]
 

 
%% --------------------------------------------------------------------
\ifcollection
\begin{flushright}
\tiny\vspace{-3Ex}
\textbf{\examinhaltstart}
\exammodulemathstat $\;\bullet$
\exammodulestat $\;\bullet$
\exammodulestatbbv 
\vspace{-4Ex}
\end{flushright}
\begin{minipage}[t]{0.5\textwidth}
\includegraphics[width = 1.3cm]{/Users/kruppajo/work/GitHub/exam/avatare/Tina.png}
\end{minipage}
\begin{minipage}[t]{0.5\textwidth}
\hfill
\href{https://youtu.be/uQDOuSlHaqw}{\includegraphics[width = 2cm]{img/youtube}}
\end{minipage}
\vspace{-3ex}
\fi
%% --------------------------------------------------------------------



\ifcollection
\paragraph{Berechnung des Welch t-Test}
\fi


Tina ist in der Uckermark für einen Versuch mit Erdbeeren. Allein diese Tatsache ist für sie eine Erzählung wert. Tina und die Wut, eine unendliche Geschichte mit kniffeligen Wendungen. Für ihre Abschlussarbeit musste sie ein Feldexperiment mit Erdbeeren durchführen und das sollte laut ihrer Betreuerin an diesem Nichtort besonders gut gelingen. Ablenkung gibt es jedenfalls keine. Gar keine. Alleine sein hilft jetzt aber nur bedingt, denn ihre Behandlung Substrattypen ($torf$ und $70p30n$) und der Messwert Frischegewicht [kg/ha] sollen mit einem t-Test ausgewertet werden. Immerhin weiß sie, dass ihr Messwert einer Normalverteilung folgt. Hm..., was entspannendes wäre gut. Tina will später nochmal raus um zu Boxen. Druck ablassen, dass muss sie auch.

\begin{table}[!h]
\centering
\begin{tabular}{cc}
\toprule
Substrattypen & Frischegewicht\\
\midrule
70p30n & 40.2\\
70p30n & 37.6\\
70p30n & 40.0\\
torf & 43.9\\
70p30n & 46.9\\
\addlinespace
torf & 42.9\\
torf & 43.4\\
\bottomrule
\end{tabular}
\end{table}



Leider kennt sich Tina mit der Berechnung eines t-Tests überhaupt nicht aus. Deshalb braucht sie bei der Berechnung Ihre Hilfe!

\begin{enumerate}
  \item Formulieren Sie die wissenschaftliche Fragestellung! \textbf{(1 Punkt)}
  \item Formulieren Sie das statistische Hypothesenpaar! \textbf{(1 Punkt)}
  \item Bestimmen Sie die Teststatistik $T_{D}$ eines  Welch t-Tests! \textbf{(3 Punkte)}
  \item Treffen Sie mit $T_{\alpha = 5\%} = 2.86$ eine Aussage zur Nullhypothese! Begründen Sie Ihre Antwort! \textbf{(2 Punkte)}
\item Berechnen Sie das 95\% Konfidenzintervall. Welche Annahmen haben Sie getroffen? \textbf{(2 Punkte)}
\item Nennen Sie den statistischen Grund, warum Sie sich zwischen einem Student t-Test und einem Welch t-Test entscheiden müssen! \textbf{(1 Punkt)}
\item Formulieren Sie eine Antwort an Tina über das Ergebnis Ihrer statistischen Analyse! Beziehen Sie dabei das 95\% Konfidenzintervall mit in Ihre Antwort ein! \textbf{(2 Punkte)}
\end{enumerate} 
\clearpage
% -----------------------------------------------------------------------

\section{Aufgabe \hfill (10 Punkte)}

\textit{Geben Sie grundsätzlich Formeln und Rechenweg zur Lösung der Teilaufgaben mit an!} \\[1Ex]
 

 
%% --------------------------------------------------------------------
\ifcollection
\begin{flushright}
\tiny\vspace{-3Ex}
\textbf{\examinhaltstart}
\exammodulemathstat $\;\bullet$
\exammodulestat $\;\bullet$
\exammodulestatbbv $\;\bullet$
\exammodulestatversuch $\;\bullet$
\exammodulebiostat
\vspace{-4Ex}
\end{flushright}
\begin{minipage}[t]{0.5\textwidth}
\includegraphics[width = 1.3cm]{/Users/kruppajo/work/GitHub/exam/avatare/Jonas.png}\hspace{-4mm}\includegraphics[width = 1.3cm]{/Users/kruppajo/work/GitHub/exam/avatare/Steffen.png}\hspace{-4mm}\includegraphics[width = 1.3cm]{/Users/kruppajo/work/GitHub/exam/avatare/Yuki.png}
\end{minipage}
\begin{minipage}[t]{0.5\textwidth}
\hfill
\href{https://youtu.be/K3LGigegKIg}{\includegraphics[width = 2cm]{img/youtube}}
\end{minipage}
\fi
%% --------------------------------------------------------------------



\ifcollection
\paragraph{Interpretation des t-Tests in \Rlogo - die Teststatistik und der p-Wert}
\fi

'Programmieren ist wie eine Sprache lernen. Man muss es nur machen, dann wird man mit der Zeit immer besser!', gibt Yuki zwinkernd zu Protokoll. Ein paar Mal hat sie schon die Faulheit gehindert weiterzumachen. Das hilft jetzt Steffen und Jonas nur bedingt, da beide jetzt die \Rlogo Ausgabe interpretieren müssen und nicht vor drei Wochen, wo noch Zeit gewesen wäre. Beide mampfen konzentriert Oreos und Snickers in sich hinein. Die beiden hatten in der Uckermark einen Versuch mit Lamas in einem Leistungssteigerungsversuch durchgeführt. Das war schon anstrengend genug! 'Wir haben Gewichtszuwachs in der 1LW [\%/kg] gemessen, vielleicht hilft das ja...', meint Steffen leicht genervt. Alle starren auf die \Rlogo Ausgabe des t-Tests. Im Hintergrund wummert London Grammar und man versteht kaum sein eigenes Wort. Jonas hofft, dass das Minischwein von Yuki beruhigend wirkt.

\begin{knitrout}
\definecolor{shadecolor}{rgb}{0.969, 0.969, 0.969}\color{fgcolor}\begin{kframe}
\begin{verbatim}
## 
## 	Two Sample t-test
## 
## data:  Gewichtszuwachs by Ernährungszusatz
## t = -0.9939, df = 19, p-value = 0.3328
## alternative hypothesis: true  is not equal to [condensed]
## 95 percent confidence interval:
##  -13.420174   4.778355
## sample estimates:
## mean in group ctrl mean in group fedX 
##           39.37000           43.69091
\end{verbatim}
\end{kframe}
\end{knitrout}

Helfen Sie Yuki bei der Interpretation des t-Tests! Sonst geht es auch für Steffen und Jonas nicht weiter.
  
\begin{enumerate}
  \item Formulieren Sie die wissenschaftliche Fragestellung! \textbf{(1 Punkt)}
  \item Formulieren Sie das statistische Hypothesenpaar! \textbf{(1 Punkt)}
\item Liegt ein signifikanter Unterschied zwischen den Gruppen vor? Begründen Sie Ihre Antwort! \textbf{(2 Punkte)}
\item Berechnen Sie den Effekt des t-Tests! \textbf{(1 Punkt)}
\item Skizzieren Sie eine Abbildung in der Sie $T_{D}$, $Pr(D|H_0)$, $A=0.95$, sowie $T_{\alpha=5\%} = |2.09|$ einzeichnen! \textbf{(4 Punkte)}
\item Beschriften Sie die Abbildung! \textbf{(1 Punkt)}  
\end{enumerate} 
\clearpage
% -----------------------------------------------------------------------

\section{Aufgabe \hfill (10 Punkte)}

\textit{Geben Sie grundsätzlich Formeln und Rechenweg zur Lösung der Teilaufgaben mit an!} \\[1Ex]
 

 
%% --------------------------------------------------------------------
\ifcollection
\begin{flushright}
\tiny\vspace{-3Ex}
\textbf{\examinhaltstart}
\exammodulestatversuch $\;\bullet$
\exammodulebiostat
\vspace{-4Ex}
\end{flushright}
\begin{minipage}[t]{0.5\textwidth}
\includegraphics[width = 1.3cm]{/Users/kruppajo/work/GitHub/exam/avatare/Steffen.png}\hspace{-4mm}\includegraphics[width = 1.3cm]{/Users/kruppajo/work/GitHub/exam/avatare/Tina.png}\hspace{-4mm}\includegraphics[width = 1.3cm]{/Users/kruppajo/work/GitHub/exam/avatare/Yuki.png}
\end{minipage}
\begin{minipage}[t]{0.5\textwidth}
\hfill
\href{https://youtu.be/zppl5eq4ioU}{\includegraphics[width = 2cm]{img/youtube}}
\end{minipage}
\fi
%% --------------------------------------------------------------------



\ifcollection
\paragraph{Interpretation des t-Tests in \Rlogo - das 95\% Konfidenzintervall}
\fi


'Programmieren ist wie eine Sprache lernen. Man muss es nur machen, dann wird man mit der Zeit immer besser!', gibt Tina zwinkernd zu Protokoll. Ein paar Mal hat sie schon die Wut gehindert weiterzumachen. Das hilft jetzt Steffen und Yuki nur bedingt, da beide jetzt die \Rlogo Ausgabe interpretieren müssen und nicht vor drei Wochen, wo noch Zeit gewesen wäre. Beide mampfen konzentriert Oreos und Reese's Peanut Butter Cups in sich hinein. Die beiden hatten im Teuteburgerwald einen Versuch mit Lamas in einem Leistungssteigerungsversuch durchgeführt. Das war schon anstrengend genug! 'Wir haben Schlachtgewicht [kg] gemessen, vielleicht hilft das ja...', meint Yuki leicht genervt. Alle starren auf die \Rlogo Ausgabe des t-Tests. Im Hintergrund wummert Tocotronic und man versteht kaum sein eigenes Wort. Yuki hofft, dass die Spinne von Tina beruhigend wirkt.

\begin{knitrout}
\definecolor{shadecolor}{rgb}{0.969, 0.969, 0.969}\color{fgcolor}\begin{kframe}
\begin{verbatim}
## 
## 	Two Sample t-test
## 
## data:  Schlachtgewicht by Ernährungszusatz
## t = 7.2532, df = 15, p-value = 2.817e-06
## alternative hypothesis: true  is not equal to [condensed]
## 95 percent confidence interval:
##  11.15800 20.44486
## sample estimates:
## mean in group ctrl mean in group fedX 
##           48.27143           32.47000
\end{verbatim}
\end{kframe}
\end{knitrout}

Helfen Sie Tina bei der Interpretation des t-Tests! Sonst geht es auch für Steffen und Yuki nicht weiter.

\begin{enumerate}
  \item Formulieren Sie die wissenschaftliche Fragestellung! \textbf{(1 Punkt)}
  \item Formulieren Sie das statistische Hypothesenpaar! \textbf{(1 Punkt)}
\item Liegt ein signifikanter Unterschied zwischen den Gruppen vor? Begründen Sie Ihre Antwort! \textbf{(2 Punkte)}
\item Skizieren Sie das sich ergebende 95\% Konifidenzintervall! \textbf{(2 Punkte)}
\item Beschriften Sie die Abbildung und das 95\% Konfidenzintervall entsprechend! \textbf{(2 Punkte)}  
\item Interpretieren Sie den Effekt des 95\% Konifidenzintervalls! \textbf{(2 Punkte)}
\end{enumerate} 
\clearpage
% -----------------------------------------------------------------------

\section{Aufgabe \hfill (9 Punkte)}

\textit{Geben Sie grundsätzlich Formeln und Rechenweg zur Lösung der Teilaufgaben mit an!} \\[1Ex]
 

 
%% --------------------------------------------------------------------
\ifcollection
\begin{flushright}
\tiny\vspace{-3Ex}
\textbf{\examinhaltstart}
\exammodulemathstat $\;\bullet$
\exammodulestat $\;\bullet$
\exammodulestatbbv $\;\bullet$
\exammodulestatversuch $\;\bullet$
\exammodulebiostat
\vspace{-4Ex}
\end{flushright}
\begin{minipage}[t]{0.5\textwidth}
\includegraphics[width = 1.3cm]{/Users/kruppajo/work/GitHub/exam/avatare/Alex.png}\hspace{-4mm}\includegraphics[width = 1.3cm]{/Users/kruppajo/work/GitHub/exam/avatare/Mark.png}\hspace{-4mm}\includegraphics[width = 1.3cm]{/Users/kruppajo/work/GitHub/exam/avatare/Paula.png}
\end{minipage}
\begin{minipage}[t]{0.5\textwidth}
\hfill
\href{https://youtu.be/02hibcRs5Wc}{\includegraphics[width = 2cm]{img/youtube}}
\end{minipage}
\fi
%% --------------------------------------------------------------------



\ifcollection
\paragraph{Interpretation des t-Tests in \Rlogo - die Visualisierung}
\fi

'Programmieren ist wie eine Sprache lernen. Man muss es nur machen, dann wird man mit der Zeit immer besser!', gibt Paula zwinkernd zu Protokoll. Ein paar Mal hat sie schon der Perfektionismus gehindert weiterzumachen. Das hilft jetzt Alex und Mark nur bedingt, da beide jetzt die \Rlogo Ausgabe interpretieren müssen und nicht vor drei Wochen, wo noch Zeit gewesen wäre. Beide mampfen konzentriert Gummibärchen und Marzipankugeln in sich hinein. Die beiden hatten im Emsland einen Versuch mit Kartoffeln in einem Versuch in einer Klimakammer durchgeführt. Das war schon anstrengend genug! 'Wir haben Trockengewicht [kg/ha] gemessen, vielleicht hilft das ja...', meint Alex leicht genervt. Alle starren auf die \Rlogo Ausgabe des t-Tests. Im Hintergrund wummert White Lies und man versteht kaum sein eigenes Wort. Mark hofft, dass die Ratte von Paula beruhigend wirkt.

\begin{knitrout}
\definecolor{shadecolor}{rgb}{0.969, 0.969, 0.969}\color{fgcolor}\begin{kframe}
\begin{verbatim}
## 
## 	Two Sample t-test
## 
## data:  Trockengewicht by Lichtstufen
## t = -3.9426, df = 19, p-value = 0.0008735
## alternative hypothesis: true  is not equal to [condensed]
## 95 percent confidence interval:
##  -23.555912  -7.218634
## sample estimates:
##  mean in group none mean in group 600lm 
##            31.77273            47.16000
\end{verbatim}
\end{kframe}
\end{knitrout}

Helfen Sie Paula bei der Interpretation des t-Tests! Sonst geht es auch für Alex und Mark nicht weiter.
  
\begin{enumerate}
  \item Formulieren Sie die wissenschaftliche Fragestellung! \textbf{(1 Punkt)}
  \item Formulieren Sie das statistische Hypothesenpaar! \textbf{(1 Punkt)}
\item Liegt ein signifikanter Unterschied zwischen den Gruppen vor? Begründen Sie Ihre Antwort! \textbf{(2 Punkte)}
\item Berechnen Sie den Effekt des t-Tests! \textbf{(1 Punkt)}
\item Skizzieren Sie die sich ergebenden Barplots! Begründen Sie Ihre Antwort! \textbf{(2 Punkte)}
\item Skizzieren Sie die sich ergebenden Boxplot! Welche Annahmen an die Daten haben Sie getroffen? Begründen Sie Ihre Antwort! \textbf{(2 Punkte)} 
\end{enumerate}
 
\clearpage
% -----------------------------------------------------------------------

\section{Aufgabe \hfill (11 Punkte)}

\textit{Geben Sie grundsätzlich Formeln und Rechenweg zur Lösung der Teilaufgaben mit an!} \\[1Ex]
 

 
%% --------------------------------------------------------------------
\ifcollection
\begin{flushright}
\tiny\vspace{-3Ex}
\textbf{\examinhaltstart}
\exammodulestat $\;\bullet$
\exammodulestatversuch $\;\bullet$
\exammodulebiostat
\vspace{-4Ex}
\end{flushright}
\begin{minipage}[t]{0.5\textwidth}
\includegraphics[width = 1.3cm]{/Users/kruppajo/work/GitHub/exam/avatare/Jonas.png}\hspace{-4mm}\includegraphics[width = 1.3cm]{/Users/kruppajo/work/GitHub/exam/avatare/Nilufar.png}
\end{minipage}
\begin{minipage}[t]{0.5\textwidth}
\hfill
\href{https://youtu.be/QR90zyn0Iz8}{\includegraphics[width = 2cm]{img/youtube}}
\end{minipage}
\fi
%% --------------------------------------------------------------------



\ifcollection
\paragraph{Berechnung des gepaarten t-Test}
\fi

Es gibt ja immer die Möglichkeit sich Hilfe zu holen. Das geht natürlich auch immer in einem Projektbericht. Deshalb arbeiten Jonas und Nilufar gemeinsam an einem Projektbericht. Das macht dann auch die Analyse ihres Hauptversuches einfacher. Zwar hat jeder von ihnen noch ein Subthema, aber auch da kann man sich ja helfen. In dem Hauptversuch wurde Folgendes von den beiden gemacht. Jonas und Nilufar haben sich Spargel angeschaut. Dabei geht um Zusammenhang zwischen Düngung ($vorher$ und $nachher$) und Frischegewicht [kg/ha]. Jetzt sollen beide einen gepaarten t-Test rechnen. Es würde auch besser funktionieren, wenn Jonas nicht die Erschöpfung im Weg stehen würde und Nilufar nicht das Problem hätte die Erwartung zu händeln. Gott sei Dank haben beide genug Snickers und Takis Blue Heat auf dem Tisch aufgetürmt.

\begin{table}[!h]
\centering
\begin{tabular}{ccc}
\toprule
ID & treatment & freshmatter\\
\midrule
3 & vorher & 51.9\\
7 & nachher & 42.5\\
1 & vorher & 39.5\\
2 & nachher & 19.4\\
1 & nachher & 44.1\\
\addlinespace
8 & nachher & 38.8\\
4 & nachher & 43.5\\
5 & vorher & 44.0\\
4 & vorher & 48.8\\
6 & nachher & 39.6\\
\addlinespace
2 & vorher & 35.2\\
6 & vorher & 39.5\\
5 & nachher & 38.1\\
7 & vorher & 40.9\\
3 & nachher & 21.2\\
\bottomrule
\end{tabular}
\end{table}



Leider kennen sich Jonas und Nilufar mit der Berechnung eines gepaarten t-Tests überhaupt nicht aus. Deshalb brauchen sie beide bei der Berechnung Ihre Hilfe!

\begin{enumerate}
  \item Formulieren Sie die wissenschaftliche Fragestellung! \textbf{(1 Punkt)}
  \item Formulieren Sie das statistische Hypothesenpaar! \textbf{(1 Punkt)}
  \item Bestimmen Sie die Teststatistik $T_{D}$ eines gepaarten t-Tests! \textbf{(3 Punkte)}
  \item Treffen Sie mit $T_{\alpha = 5\%} = 1.64$ eine Aussage zur Nullhypothese! Begründen Sie Ihre Antwort! \textbf{(2 Punkte)}
\item Schätzen Sie den $p$-Wert des gepaarten t-Tests ab! Begründen Sie Ihre Antwort mit einer Skizze! \textbf{(2 Punkte)}
\item Formulieren Sie eine Antwort an Jonas über das Ergebnis Ihrer statistischen Analyse! \textbf{(2 Punkte)}
\end{enumerate}


 
\clearpage
% -----------------------------------------------------------------------

\section{Aufgabe \hfill (10 Punkte)}

\textit{Geben Sie grundsätzlich Formeln und Rechenweg zur Lösung der Teilaufgaben mit an!} \\[1Ex]
 

 
%% --------------------------------------------------------------------
\ifcollection
\begin{flushright}
\tiny\vspace{-3Ex}
\textbf{\examinhaltstart}
\exammodulestat $\;\bullet$
\exammodulestatversuch $\;\bullet$
\exammodulebiostat
\vspace{-4Ex}
\end{flushright}
\begin{minipage}[t]{0.5\textwidth}
\includegraphics[width = 1.3cm]{/Users/kruppajo/work/GitHub/exam/avatare/Jessica.png}\hspace{-4mm}\includegraphics[width = 1.3cm]{/Users/kruppajo/work/GitHub/exam/avatare/Jonas.png}
\end{minipage}
\begin{minipage}[t]{0.5\textwidth}
\hfill
\href{https://youtu.be/kHmfEmU6lrk}{\includegraphics[width = 2cm]{img/youtube}}
\end{minipage}
\fi
%% --------------------------------------------------------------------



\ifcollection
\paragraph{Interpretation des gepaarten t-Tests in \Rlogo}
\fi

Alles voll mit Maiss. Aber das haben Jessica und Jonas eben gemeinsam in einer Abschlussarbeit gemacht! Worum ging es aber konkret? Beide haben als einen normalverteilten Messwert ($Y$) Chlorophyllgehalt (SPAD-502Plus) [SPAD] von Maiss bestimmt. Die Daten haben beide zusammen in einem Feldexperiment erhoben. In dem Experiment ging es um eine vorher/nachher Untersuchung an den gleichen Maiss. Als Behandlung wurde Beschattung ($7d$ und $14d$) eingesetzt. Nach der Meinung der Betreuerin muss hier ein gepaarter t-Test gerechnet werden. Leider kennen sich beide nicht sehr gut in \Rlogo aus. Jessica hat einiges an Schokobons geholt, so dass beide die Zeit gut durchbringen werden. Dann geht Jessica nochmal zum Sport. Einfach mal raus um Rad zu fahren. Ohne Ziel und Uhr. Das ist was für Jessica.

\begin{knitrout}
\definecolor{shadecolor}{rgb}{0.969, 0.969, 0.969}\color{fgcolor}\begin{kframe}
\begin{verbatim}
## 
## 	Paired t-test
## 
## data:  Chlorophyllgehalt by Beschattung
## t = 1.2206, df = 8, p-value = 0.257
## alternative hypothesis: true  is not equal to [condensed]
## 95 percent confidence interval:
##  -4.495675 14.606787
## sample estimates:
## mean difference 
##        5.055556
\end{verbatim}
\end{kframe}
\end{knitrout}

Jetzt brauchen Jessica und Jonas Ihre Hilfe bei der Berechnung eines gepaarten t-Tests in \Rlogo um ihre Arbeit dann in diesem Semester noch abschließen zu können.

\begin{enumerate}
  \item Formulieren Sie die wissenschaftliche Fragestellung! \textbf{(1 Punkt)}
  \item Formulieren Sie das statistische Hypothesenpaar! \textbf{(1 Punkt)}
\item Liegt ein signifikanter Unterschied zwischen den Gruppen vor?
  Begründen Sie Ihre Antwort! \textbf{(2 Punkte)}
\item Skizzieren Sie das sich ergebende 95\% Konfidenzintervall! \textbf{(2 Punkte)}
\item Interpretieren Sie den Effekt des gepaarten t-Tests! \textbf{(2 Punkte)}
\item Skizzieren Sie den sich ergebenden Boxplot der Differenzen! Welche Annahmen an die Daten haben Sie getroffen? Begründen Sie Ihre Antwort! \textbf{(2 Punkte)} 
\end{enumerate}
 
\clearpage
% -----------------------------------------------------------------------
\part{Die einfaktorielle \& zweifaktorielle ANOVA}
% -----------------------------------------------------------------------

\section{Aufgabe \hfill (11 Punkte)}

\textit{Geben Sie grundsätzlich Formeln und Rechenweg zur Lösung der Teilaufgaben mit an!} \\[1Ex]
 

 
%% --------------------------------------------------------------------
\ifcollection
\begin{flushright}
\tiny\vspace{-3Ex}
\textbf{\examinhaltstart}
\exammodulemathstat $\;\bullet$
\exammodulestat $\;\bullet$
\exammodulestatbbv 
\vspace{-4Ex}
\end{flushright}
\begin{minipage}[t]{0.5\textwidth}
\includegraphics[width = 1.3cm]{/Users/kruppajo/work/GitHub/exam/avatare/Steffen.png}\hspace{-4mm}\includegraphics[width = 1.3cm]{/Users/kruppajo/work/GitHub/exam/avatare/Yuki.png}
\end{minipage}
\begin{minipage}[t]{0.5\textwidth}
\hfill
\href{https://youtu.be/wEePzcwwti8}{\includegraphics[width = 2cm]{img/youtube}}
\end{minipage}
\fi
%% --------------------------------------------------------------------



\ifcollection
\paragraph{Visualisierung der einfaktoriellen ANOVA}
\fi

'Uff... die einfaktorielle ANOVA. Und wir können jetzt anhand der Visualisuierung sehen, ob da schon was signifikant ist?', Yuki hebt die Augenbraue. 'Ja, können wir. Dafür müssen wir aber erstmal in \texttt{\{ggplot\}} uns die Daten anschauen. Oder wir zeichnen es flott mit der Hand. Geht auch.', meint Steffen dazu. Die beiden hatten sich auf einem Konzert von Taylor Swift kennengelernt. Yuki hatte sich in ein Gewächshausexperiment verschiedene Maiss angeschaut. Dabei ging es herauszufinden, ob es einen Zusammenhang zwischen der Behandlung Substrattypen ($torf$, $40p60n$ und $70p30n$) und dem Messwert Proteingehalt [g/kg] gibt. Später wird noch Harry Potter geguckt. Steffen befürwortet das!

\begin{knitrout}
\definecolor{shadecolor}{rgb}{0.969, 0.969, 0.969}\color{fgcolor}\begin{table}[!h]
\centering
\begin{tabular}{cc}
\toprule
Substrattypen & Proteingehalt\\
\midrule
70p30n & 42\\
torf & 32\\
70p30n & 40\\
70p30n & 40\\
70p30n & 40\\
\addlinespace
torf & 31\\
70p30n & 39\\
torf & 25\\
40p60n & 40\\
70p30n & 39\\
\addlinespace
40p60n & 40\\
40p60n & 39\\
40p60n & 40\\
40p60n & 41\\
torf & 30\\
\addlinespace
torf & 32\\
torf & 32\\
70p30n & 41\\
torf & 30\\
\bottomrule
\end{tabular}
\end{table}

\end{knitrout}

Leider kennen sich Yuki und Steffen mit Darstellung einer einfaktoriellen ANOVA überhaupt nicht aus. 

\begin{enumerate}
\item Erstellen  Sie  eine  Visualisierung  der  Datentabelle! Beschriften  Sie  die  Abbildung! \textbf{(2 Punkte)}
\item Benennen Sie die Visualisierung mit dem korrekten, statistischen Fachbegriff! \textbf{(1 Punkt)}
\item Zeichnen Sie folgende statistischen Maßzahlen passend ein! 
  \begin{itemize}
  \item Den globalen Mittelwert $\beta_0$ \textbf{(1 Punkt)}
  \item Die Mittelwerte der einzelnen Behandlungsstufen \textbf{(1 Punkt)}
  \item Die Mittelwertsdifferenz der einzelnen Behandlungsstufen mit $\beta_{torf}$, $\beta_{40p60n}$ und $\beta_{70p30n}$ \textbf{(1 Punkt)}
  \item Die Residuen oder Fehler mit $\epsilon$ \textbf{(1 Punkt)}
  \end{itemize}
\item Liegt ein \textit{vermutlicher} signifikanter Unterschied vor? Begründen Sie Ihre Antwort! \textbf{(2 Punkte)}
\item Schätzen Sie die Effekte der Behandlungsstufen! \textbf{(2 Punkte)}
\end{enumerate}
 
\clearpage
% -----------------------------------------------------------------------

\section{Aufgabe \hfill (10 Punkte)}

\textit{Geben Sie grundsätzlich Formeln und Rechenweg zur Lösung der Teilaufgaben mit an!} \\[1Ex]
 

 
%% --------------------------------------------------------------------
\ifcollection
\begin{flushright}
\tiny\vspace{-3Ex}
\textbf{\examinhaltstart}
\exammodulemathstat $\;\bullet$
\exammodulestat $\;\bullet$
\exammodulestatbbv 
\vspace{-4Ex}
\end{flushright}
\begin{minipage}[t]{0.5\textwidth}
\includegraphics[width = 1.3cm]{/Users/kruppajo/work/GitHub/exam/avatare/Alex.png}\hspace{-4mm}\includegraphics[width = 1.3cm]{/Users/kruppajo/work/GitHub/exam/avatare/Tina.png}
\end{minipage}
\begin{minipage}[t]{0.5\textwidth}
\hfill
\href{https://youtu.be/25bvWshX-Gw}{\includegraphics[width = 2cm]{img/youtube}}
\end{minipage}
\fi
%% --------------------------------------------------------------------



\ifcollection
\paragraph{Ergebnistabelle der einfaktoriellen ANOVA}
\fi

'Uff... die einfaktorielle ANOVA. Und wie füllen wir jetzt die Tabelle der ANOVA aus und schauen, ob da was signifikant ist?', Alex hebt die Augenbraue. 'Das ist eine sehr gute Frage. Ich glaube man kann alles in der Tabelle relativ einfach mit wenigen Informationen berechnen.', meint Tina dazu. Da hilft die Spinne von Tina auch nur bedingt. Alex hatte sich in ein Kreuzungsexperiment verschiedene Zandern angeschaut. Dabei ging es herauszufinden, ob es einen Zusammenhang zwischen der Behandlung Flüssignahrung ($ctrl$, $superIn$ und $flOw$) und dem Messwert Fettgehalt [\%/kg] gibt. Nachher wollen sich beide noch mit dem Hobby Astronomie von Tina beschäftigen. Kennt Alex noch nicht, klingt aber interessant.



\vspace{1ex}

Leider kennen sich Alex und Tina mit Berechnung einer einfaktoriellen ANOVA überhaupt nicht aus. Deshalb brauchen beide bei der Erstellung Ihre Hilfe, die Spinne reicht als Hilfe nicht aus! 

\begin{enumerate}
  \item Formulieren Sie die wissenschaftliche Fragestellung! \textbf{(1 Punkt)}
  \item Formulieren Sie das statistische Hypothesenpaar! \textbf{(2 Punkte)}
\item Füllen Sie die unterstehende einfaktorielle ANOVA Ergebnistabelle aus! \textbf{(3 Punkte)}
\end{enumerate}

\vspace{1Ex}

\begin{center}
  \Large
  \begin{tabular}{lccccp{3cm}}
\toprule
     & \textbf{Df} & \textbf{Sum Sq} & \textbf{Mean Sq} & \textbf{F value} & \textbf{Pr(>F)} \strut\\
    \midrule
   \textbf{Flüssignahrung}  & 2 & 6.24 &  &  &  \strut\\
   \textbf{error}  & 17 & 221.56 &  &  &  \strut\\
   \textbf{Total}  & 19 &  &  &  &  \strut\\
\bottomrule
  \end{tabular}
\end{center}

\vspace{1Ex}

\begin{enumerate}
  \setcounter{enumi}{3}
\item Schätzen Sie den p-Wert der Tabelle mit $F_{\alpha = 5\%} = 3.59$ ab. Begründen Sie Ihre Antwort! \textbf{(2 Punkte)}
\item Berechen Sie den Effektschätzer $\eta^2$. Was sagt Ihnen der Wert von $\eta^2$ aus? \textbf{(2 Punkte)}
\end{enumerate}



 
\clearpage
% -----------------------------------------------------------------------

\section{Aufgabe \hfill (10 Punkte)}

\textit{Geben Sie grundsätzlich Formeln und Rechenweg zur Lösung der Teilaufgaben mit an!} \\[1Ex]
 

 
%% --------------------------------------------------------------------
\ifcollection
\begin{flushright}
\tiny\vspace{-3Ex}
\textbf{\examinhaltstart}
\exammodulemathstat $\;\bullet$
\exammodulestat $\;\bullet$
\exammodulestatbbv
\vspace{-4Ex}
\end{flushright}
\begin{minipage}[t]{0.5\textwidth}
\includegraphics[width = 1.3cm]{/Users/kruppajo/work/GitHub/exam/avatare/Tina.png}
\end{minipage}
\begin{minipage}[t]{0.5\textwidth}
\hfill
\href{https://youtu.be/UOzpeBYWgVQ}{\includegraphics[width = 2cm]{img/youtube}}
\end{minipage}
\vspace{-3Ex}
\fi
%% --------------------------------------------------------------------



\ifcollection
\paragraph{Die einfaktorielle ANOVA in \Rlogo}
\fi

Tina schaut sich fragend in der Bibliothek um. Tina hatte gehofft, dass jemand hier sein würde, den sie kennt und sich mit \Rlogo auskennt. Wird aber enttäuscht. Tina war im Teuteburgerwald um einen Leistungssteigerungsversuch mit Milchvieh durchzuführen. Nun möchte ihre Betreuerin ihrem Projektbericht erstmal eine ANOVA sehen und die Ergebnisse präsentiert bekommen. Dabei ging es herauszufinden, ob es einen Zusammenhang zwischen der Behandlung Lüftungssystem ($keins$, $storm$, $tornado$ und $thunder$) und dem Messwert Fettgehalt [\%/kg] gibt. Und eigentlich wollte Tina doch noch zum Sport! Tina will später nochmal raus um zu Boxen. Druck ablassen, dass muss sie auch.

\begin{knitrout}
\definecolor{shadecolor}{rgb}{0.969, 0.969, 0.969}\color{fgcolor}\begin{kframe}
\begin{verbatim}
## Analysis of Variance Table
## 
## Response: Fettgehalt
##                Df Sum Sq Mean Sq F value    Pr(>F)
## Lüftungssystem  3 3678.5 1226.17  77.685 5.734e-14
## Residuals      29  457.7   15.78
\end{verbatim}
\end{kframe}
\end{knitrout}

\vspace{1ex}

Leider kennen sich Tina mit Berechnung einer einfaktoriellen ANOVA überhaupt nicht aus. Deshalb braucht sie bei der Erstellung Ihre Hilfe! 

\begin{enumerate}
  \item Formulieren Sie die wissenschaftliche Fragestellung! \textbf{(1 Punkt)}
  \item Formulieren Sie das statistische Hypothesenpaar! \textbf{(2 Punkte)}
\item Interpretieren Sie das Ergebnis der einfaktoriellen ANOVA! \textbf{(2 Punkte)} 
\item Berechnen Sie den Effektschätzer $\eta^2$. Was sagt Ihnen der Wert von $\eta^2$ aus? \textbf{(2 Punkte)}
\item Skizzieren Sie eine Abbildung, der dem obigen Ergebnis der
  einfaktoriellen ANOVA näherungsweise entspricht! \textbf{(3 Punkte)}
\end{enumerate}

 
\clearpage
% -----------------------------------------------------------------------

\section{Aufgabe \hfill (12 Punkte)}

\textit{Geben Sie grundsätzlich Formeln und Rechenweg zur Lösung der Teilaufgaben mit an!} \\[1Ex]
 

 
%% --------------------------------------------------------------------
\ifcollection
\begin{flushright}
\tiny\vspace{-3Ex}
\textbf{\examinhaltstart}
\exammodulestatversuch $\;\bullet$
\exammodulebiostat
\vspace{-4Ex}
\end{flushright}
\begin{minipage}[t]{0.5\textwidth}
\includegraphics[width = 1.3cm]{/Users/kruppajo/work/GitHub/exam/avatare/Jessica.png}\hspace{-4mm}\includegraphics[width = 1.3cm]{/Users/kruppajo/work/GitHub/exam/avatare/Steffen.png}
\end{minipage}
\begin{minipage}[t]{0.5\textwidth}
\hfill
\href{https://youtu.be/G-AW_--5V0w}{\includegraphics[width = 2cm]{img/youtube}}
\end{minipage}
\fi
%% --------------------------------------------------------------------



\ifcollection
\paragraph{Die einfaktoriellen ANOVA und der Student t-Test}
\fi

Steffen und Jessica schauen sich etwas entnervt an. Gemeinsam schreiben die beiden ihre Abschlussarbeit und sollen nun als erstes einmal die Daten mit eine einfaktoriellen ANOVA auswerten damit abgeschätzt werden kann, ob überhaupt signifikante Ergebnisse in den multipen Gruppenvergleichen zu erwarten sind. Deshalb erstmal Schokobons mampfen, die Jessica mitgebracht hat. Nun möchte erstmal ihre Betreuung der Arbeit eine ANOVA Tabelle sehen. Was immer da auch drin zu erkennen sein mag. Steffen schaut Jessica sehen erstmla gar nichts. Die beiden waren im Teuteburgerwald um ein Feldexperiment mit Erdbeeren durchzuführen. Dabei haben Steffen und Jessica den Messwert Chlorophyllgehalt (SPAD-502Plus) [SPAD] unter der Behandung Düngestufen ($ctrl$, $low$, $mid$ und $high$) ermittelt. Später wollen die beiden dann noch raus um Rad zu fahren.



\vspace{1ex}

Leider kennen sich Steffen und Jessica mit Berechnung einer einfaktoriellen ANOVA überhaupt nicht aus. Deshalb brauchen beide bei der Erstellung Ihre Hilfe! 

\begin{enumerate}
  \item Formulieren Sie die wissenschaftliche Fragestellung! \textbf{(1 Punkt)}
  \item Formulieren Sie das statistische Hypothesenpaar! \textbf{(1 Punkt)}
\item Füllen Sie die unterstehende einfaktorielle ANOVA Ergebnistabelle aus! \textbf{(3 Punkte)}
\end{enumerate}

\vspace{1Ex}

\begin{center}
  \Large
  \begin{tabular}{lccccp{3cm}}
\toprule
     & \textbf{Df} & \textbf{Sum Sq} & \textbf{Mean Sq} & \textbf{F value} & \textbf{Pr(>F)} \strut\\
    \midrule
   \textbf{Düngestufen}  & 3 & 7.09 &  &  &  \strut\\
   \textbf{Error}  & 30 & 191.52 &  &  &  \strut\\
\bottomrule
  \end{tabular}
\end{center}

\vspace{1Ex}

\begin{enumerate}
  \setcounter{enumi}{3}
\item Schätzen Sie den p-Wert der Tabelle mit $F_{\alpha = 5\%} = 2.92$ ab. Begründen Sie Ihre Antwort! \textbf{(2 Punkte)}
\item Was bedeutet ein signifikantes Ergebnis in einer einfaktoriellen ANOVA? \textbf{(1 Punkt)}
\item Berechnen Sie \textit{einen} Student t-Test für den \textit{vermutlich} signifikantesten Gruppenvergleich anhand der untenstehenden Tabelle mit $T_{\alpha = 5\%} = 2.03$. Begründen Sie Ihre Auswahl! \textbf{(3 Punkte)}
\end{enumerate}


\begin{knitrout}
\definecolor{shadecolor}{rgb}{0.969, 0.969, 0.969}\color{fgcolor}\begin{table}[!h]
\centering\begingroup\fontsize{11}{13}\selectfont

\begin{tabular}{cccc}
\toprule
\textbf{Düngestufen} & \textbf{Fallzahl (n)} & \textbf{Mittelwert} & \textbf{Standardabweichung}\\
\midrule
ctrl & 9 & 3.11 & 1.54\\
low & 7 & 4.14 & 0.90\\
mid & 9 & 4.22 & 2.39\\
high & 9 & 3.56 & 3.91\\
\bottomrule
\end{tabular}
\endgroup{}
\end{table}

\end{knitrout}


\begin{enumerate}
  \setcounter{enumi}{6}
\item Gegebenen der ANOVA Tabelle war das Ergebnis des Student t-Tests zu erwarten? Begründen Sie Ihre Antwort! \textbf{(2 Punkte)}
\end{enumerate}

 
\clearpage
% -----------------------------------------------------------------------

\section{Aufgabe \hfill (12 Punkte)}

\textit{Geben Sie grundsätzlich Formeln und Rechenweg zur Lösung der Teilaufgaben mit an!} \\[1Ex]
 

 
%% --------------------------------------------------------------------
\ifcollection
\begin{flushright}
\tiny\vspace{-3Ex}
\textbf{\examinhaltstart}
\exammodulestatversuch $\;\bullet$
\exammodulebiostat
\vspace{-4Ex}
\end{flushright}
\begin{minipage}[t]{0.5\textwidth}
\includegraphics[width = 1.3cm]{/Users/kruppajo/work/GitHub/exam/avatare/Nilufar.png}
\end{minipage}
\begin{minipage}[t]{0.5\textwidth}
\hfill
\href{https://youtu.be/1uZsKAipACI}{\includegraphics[width = 2cm]{img/youtube}}
\end{minipage}
\vspace{-3Ex}
\fi
%% --------------------------------------------------------------------



\ifcollection
\paragraph{Ergebnistabelle der zweifaktoriellen ANOVA}
\fi

In einen Versuch in einer Klimakammer wurden Erdbeeren mit dem Behandlung Düngestufen ($ctrl$, $low$, $mid$ und $high$) sowie der Behandlung Bewässerungstypen ($ctrl$, und $high$) untersucht. Es wurde als Messwert Chlorophyllgehalt (SPAD-502Plus) [SPAD] bestimmt. Nilufar ahnte schon, dass es komplexer wird, als sie mit ihrem Projektbericht angefangen hat. Das es jetzt aber so kompliziert wird, hätte sie jetzt aber auch nicht gedacht. Nilufar kratzt sich am Kopf. Nilufar mampft aus Frust noch eine Handvoll Takis Blue Heat. Eventuell muss sie dann doch nochmal Hilfe in der statistischen Beratung holen. Jetzt versucht sie es aber erstmal selber. Und eigentlich wollte Nilufar doch noch ihrem Hobby nachgehen! Nilufar liebt Hip Hop. Darin kann sie sich wirklich verlieren und immer wieder neu begeistern.



\vspace{1ex}

Leider kennen sich Nilufar mit Berechnung einer zweifaktoriellen ANOVA überhaupt nicht aus. Deshalb braucht sie bei der Erstellung Ihre Hilfe! 

\begin{enumerate}
  \item Formulieren Sie die wissenschaftliche Fragestellung für beide Faktoren separat! \textbf{(2 Punkte)}
  \item Formulieren Sie die statistischen Hypothesenpaare für beide Faktoren separat! \textbf{(2 Punkte)}
\item Füllen Sie die unterstehende zweifaktorielle ANOVA Ergebnistabelle aus! \textbf{(3 Punkte)}
\end{enumerate}

\vspace{1Ex}

\begin{center}
  \Large
  \begin{tabular}{lccccc}
  \toprule
     & \textbf{Df} & \textbf{Sum Sq} & \textbf{Mean Sq} & \textbf{F value} & \textbf{Pr(>F)} \strut\\
    \midrule
   \textbf{Düngestufen}  & 3 & 334.48 &  &  &  \strut\\
    \textbf{Bewässerungstypen}  & 1 & 2.77 &  &  &  \strut\\
    \textbf{Düngestufen:Bewässerungstypen}  & 3 & 403.18 &  &  &  \strut\\
   \textbf{Error}  & 18 & 232 &  &  &  \strut\\
\bottomrule
  \end{tabular}
\end{center}

\vspace{1Ex}

\begin{enumerate}
  \setcounter{enumi}{3}
\item Schätzen Sie den p-Wert der Tabelle ab. Begründen Sie Ihre Antwort! \textbf{(3 Punkte)}
\end{enumerate}
  
\begin{center}
    \Large
\begin{tabular}{lc}
  \toprule
     & $\boldsymbol{F_{\alpha = 5\%}}$ \\
\midrule
  \textbf{Düngestufen} & $4.26$ \\
  \textbf{Bewässerungstypen} & $3.40$ \\
  \textbf{Düngestufen:Bewässerungstypen} & $5.23$ \\
  \bottomrule
  \end{tabular}
\end{center}

\begin{enumerate}
  \setcounter{enumi}{4}
\item Was sagt der Term \textit{Düngestufen:Bewässerungstypen} aus? Interpretieren Sie das Ergebnis! \textbf{(2 Punkte)}
\end{enumerate}
 
\clearpage
% -----------------------------------------------------------------------

\section{Aufgabe \hfill (12 Punkte)}

\textit{Geben Sie grundsätzlich Formeln und Rechenweg zur Lösung der Teilaufgaben mit an!} \\[1Ex]
 

 
%% --------------------------------------------------------------------
\ifcollection
\begin{flushright}
\tiny\vspace{-3Ex}
\textbf{\examinhaltstart}
\exammodulestatversuch $\;\bullet$
\exammodulebiostat
\vspace{-4Ex}
\end{flushright}
\begin{minipage}[t]{0.5\textwidth}
\includegraphics[width = 1.3cm]{/Users/kruppajo/work/GitHub/exam/avatare/Steffen.png}
\end{minipage}
\begin{minipage}[t]{0.5\textwidth}
\hfill
\href{https://youtu.be/i7TadGcyn8c}{\includegraphics[width = 2cm]{img/youtube}}
\end{minipage}
\vspace{-3Ex}
\fi
%% --------------------------------------------------------------------



\ifcollection
\paragraph{Die zweifaktorielle ANOVA in \Rlogo}
\fi

'Mit der zweifaktoriellen ANOVA lässt sich die Interaktion zwischen den beiden Behandlungen nachweisen!', seine Betreuerin scheint die zweifaktoriellen ANOVA zu verstehen. Warum jetzt er jetzt nochmal alles wiederkäuen muss, wird Steffen echt nicht so klar. Wenn es doch so klar ist? Steffen war im Wendland und hatte dort ein Feldexperiment mit Kartoffeln durchgeführt. Die Komune wo er untergekommen war, war cool gewesen. Nur jetzt muss eben das Experiment fertig ausgewertet werden. Steffen hatte zwei Behandlungen auf Kartoffeln angewendet. Einmal Bewässerungstypen ($ctrl$, $low$, $mid$ und $high$) sowie als zweite Behandlung Lüftungssystemen und Folientunneln ($ctrl$, und $tornado$). Gemessen wurde der Messwert ($Y$) Trockengewicht [kg/ha]. Jetzt muss das hier zu einem Ende kommen! Eigentlich wollte Steffen nachher noch einen Film schauen. Irgendwie komisch, wenn er Harry Potter anmacht, dann ist die Schlange eigentlich sofort vor dem Bildschirm und starrt hinein.

\begin{knitrout}
\definecolor{shadecolor}{rgb}{0.969, 0.969, 0.969}\color{fgcolor}\begin{kframe}
\begin{verbatim}
## Analysis of Variance Table
## 
## Response: Trockengewicht
##                                    Df Sum Sq Mean Sq F value    Pr(>F)
## Bewässerungstypen                   3  99.69  49.844  5.1239  0.014027
## Lüftungssystemen                    1  96.70  96.699  9.9404  0.004304
## Bewässerungstypen:Lüftungssystemen  3 618.78 309.391 31.8045 1.786e-07
## Residuals                          24 233.47   9.728
\end{verbatim}
\end{kframe}
\end{knitrout}

\vspace{1ex}

Leider kennt sich Steffen mit Berechnung einer zweifaktoriellen ANOVA überhaupt nicht aus. Deshalb braucht er bei der Erstellung Ihre Hilfe! 

\begin{enumerate}
  \item Formulieren Sie die wissenschaftliche Fragestellung für beide Faktoren separat! \textbf{(2 Punkte)}
  \item Formulieren Sie die statistischen Hypothesenpaare für beide Faktoren separat! \textbf{(2 Punkte)}
\item Interpretieren Sie das Ergebnis der zweifaktoriellen ANOVA! Begründen Sie Ihre Antwort! \textbf{(3 Punkte)} 
\item Zeichnen Sie eine Abbildung, der dem obigen Ergebnis der
  zweifaktoriellen ANOVA näherungsweise entspricht! \textbf{(5 Punkte)}
\end{enumerate}
 
\clearpage
% -----------------------------------------------------------------------

\section{Aufgabe \hfill (11 Punkte)}

\textit{Geben Sie grundsätzlich Formeln und Rechenweg zur Lösung der Teilaufgaben mit an!} \\[1Ex]
 

 
%% --------------------------------------------------------------------
\ifcollection
\begin{flushright}
\tiny\vspace{-3Ex}
\textbf{\examinhaltstart}
\exammodulestatversuch $\;\bullet$
\exammodulebiostat
\vspace{-4Ex}
\end{flushright}
\begin{minipage}[t]{0.5\textwidth}
\includegraphics[width = 1.3cm]{/Users/kruppajo/work/GitHub/exam/avatare/Jonas.png}
\end{minipage}
\begin{minipage}[t]{0.5\textwidth}
\hfill
\href{https://youtu.be/-yJWhI3Uy5g}{\includegraphics[width = 2cm]{img/youtube}}
\end{minipage}
\vspace{-3Ex}
\fi
%% --------------------------------------------------------------------



\ifcollection
\paragraph{Interaktion in der zweifaktoriellen ANOVA}
\fi

Es ist schon kurz nach fünf und Jonas wird langsam nervös. Jonas wollte heute Abend noch seine E-Sport Qualifikation schauen und dann zum Sport. Stattdessen versucht sein Betreuer die Ausgabe der zweifaktoriellen ANOVA zu visualieren und zu überprüfen, ob es mit der Visualisierung der Daten als Boxplots zusammenpasst. Es liegt anscheinend eine signifikante Interaktion vor? Jonas hatte im Wendland ein Freilandversuch mit Spargel durchgeführt. Es gab dabei zwei Behandlungen. Einmal Düngestufen ($ctrl$, $low$, $mid$ und $high$) sowie als zweite Behandlung Lüftungssystemen und Folientunneln ($ctrl$, und $tornado$). Gemessen wurde der Messwert ($Y$) Proteingehalt [g/kg]. So kompliziert kann das jetzt doch nicht sein! Eigentlich wollte Jonas nachher noch zum Sport. Um zu Schwimmen geht Jonas dann später nochmal raus. Echte Entspannung.

\vspace{1ex}

Leider kennen sich Jonas und sein Betreuer mit der zweifaktoriellen ANOVA überhaupt nicht aus. Geschweige denn mit der Interpretation einer Interaktion. Deshalb braucht er bei der Erstellung Ihre Hilfe, sonst wird es heute Abend mit seinem Hobby Stricken nichts mehr! 

\begin{enumerate}
\item Visualisieren Sie folgende mögliche Interaktionen zwischen den Behandlungen! Heben Sie die Positionen der auftretenden Interaktion (farblich) hervor! Beschriften Sie die Abbildung! \textbf{(4 Punkte)}
\begin{enumerate}
\item \underline{Keine} Interaktion liegt vor.
\item Eine \underline{schwache} Interaktion liegt vor. 
\item Eine \underline{starke} Interaktion liegt vor. 
\end{enumerate}
\item Erklären Sie den Unterschied zwischen den verschiedenen Interaktionen! \textbf{(2 Punkte)}
\item Welche statistische Maßzahl betrachten Sie für die Bewertung der Interaktion? \textbf{(1 Punkt)}
\item Schreiben Sie das Modell mit Interaktionsterm in \Rlogo für eine zweifaktorielle ANOVA! \textbf{(1 Punkt)} 
\item Skizzieren Sie die Funktion in \Rlogo für eine Post-hoc Analyse in \texttt{\{emmeans\}}! \textbf{(1 Punkt)} 
\item Wenn eine signifikante Interaktion in den Daten vorliegt, wie ist dann das weitere Vorgehen? Berücksichtigen Sie dabei die Verwendung von \texttt{\{emmeans\}}! \textbf{(2 Punkte)}
\end{enumerate}

 
\clearpage
% -----------------------------------------------------------------------

\section{Aufgabe \hfill (10 Punkte)}

\textit{Geben Sie grundsätzlich Formeln und Rechenweg zur Lösung der Teilaufgaben mit an!} \\[1Ex]
 

 
%% --------------------------------------------------------------------
\ifcollection
\begin{flushright}
\tiny\vspace{-3Ex}
\textbf{\examinhaltstart}
\exammodulebiostat
\vspace{-4Ex}
\end{flushright}
\begin{minipage}[t]{0.5\textwidth}
\includegraphics[width = 1.3cm]{/Users/kruppajo/work/GitHub/exam/avatare/Tina.png}
\end{minipage}
\begin{minipage}[t]{0.5\textwidth}
\hfill
\href{https://youtu.be/X0-2he7ByMA}{\includegraphics[width = 2cm]{img/youtube}}
\end{minipage}
\vspace{-3Ex}
\fi
%% --------------------------------------------------------------------



\ifcollection
\paragraph{Zusammenhang zwischen der ANOVA und dem Post-hoc-Test}
\fi

Es ist schon kurz nach fünf und Tina wird langsam nervös. Tina wollte heute Abend noch ihre E-Sport Qualifikation schauen. Hoffentlich kommt sie noch rechtzeitig zum Streamen. Angestrengend krault sie die Spinne. Stattdessen versucht ihr Betreuer die Ausgabe der einfaktoriellen ANOVA zu visualieren und zu überprüfen, ob es mit der Visualisierung der Daten als Boxplots zusammenpasst. Anscheinend gibt es ein Problem mit der Annahme der Normalverteilung und der Varianzhomogenität der ANOVA in den Daten. 'Wir haben jetzt bei der ANOVA einen p-Wert mit 0.052 raus sowie eine F-Statistik $F_D$ mit 1.78 berechnet. Nach den Boxplots müsste sich eigentlich ein Unterschied zwischen $kontakt$ und $standard$ ergeben. Der Unterschied ist in \texttt{\{emmeans\}} auch signifikant mit einem p-Wert von 0.021. Wie kann das sein?', grummelt ihr Betreuer. Tina hatte im Wendland ein Stallexperiment mit Zandern durchgeführt. Dabei wurden die Daten $D$ erhoben. Es gab dabei eine Behandlungen Bestandsdichte ($standard$, $eng$, $weit$ und $kontakt$). Gemessen wurde der Messwert ($Y$) Gewichtszuwachs in der 1LW. So kompliziert kann das jetzt doch nicht sein! Tina hat schon genug Probleme. Wenn die Wut nicht wäre, dann wäre es einfacher.

\begin{graybox}{Gegebene Formeln}
\begin{center}
  \begin{tabular}{ccc}
    $MS_{treatment} = \cfrac{SS_{treatment}}{df_{treatment}}$ &
    $MS_{error} = \cfrac{SS_{error}}{df_{error}}$ &
    $F_{D} = \cfrac{MS_{treatment}}{MS_{error}}$ \\
  \end{tabular}
\end{center}
\end{graybox}

Leider kennen sich Tina und ihr Betreuer mit der Interpretation einer ANOVA überhaupt nicht aus. Deshalb braucht sie bei der Erstellung Ihre Hilfe und die Zeit wird knapp. 

\begin{enumerate}
\item Formulieren Sie die wissenschaftliche Fragestellung! \textbf{(1 Punkt)}
\item Formulieren Sie das statistische Hypothesenpaar! \textbf{(1 Punkt)}
\item Was bedeutet eine signifikante ANOVA für die beobachteten Daten $D$? \textbf{(1 Punkt)}
\item Visualisieren Sie den Unterschied zwischen Varianzhomogenität und Varianzheterogenität anhand der Daten $D$! Beschriften Sie die Abbildung! \textbf{(2 Punkte)} 
\item Welche Auswirkung hat die Verletzung der Annahme der Varianzhomogenität für die Teststatistik $F_D$ der ANOVA? Begründen Sie Ihre Antwort! \textbf{(3 Punkte)}
\item Erklären Sie abschließend die Diskrepanz zwischen den Ergebnis der ANOVA und dem paarweisen Gruppenvergleich in \texttt{\{emmeans\}}! \textbf{(2 Punkte)}
\end{enumerate}

 
\clearpage
% -----------------------------------------------------------------------
\part{Multiple Gruppenvergleiche}
% ----------------------------------------------------------------------- 

\section{Aufgabe \hfill (12 Punkte)}

\textit{Geben Sie grundsätzlich Formeln und Rechenweg zur Lösung der Teilaufgaben mit an!} \\[1Ex]
 

 
%% --------------------------------------------------------------------
\ifcollection
\begin{flushright}
\tiny\vspace{-3Ex}
\textbf{\examinhaltstart}
\exammodulestatversuch $\;\bullet$
\exammodulebiostat
\vspace{-4Ex}
\end{flushright}
\begin{minipage}[t]{0.5\textwidth}
\includegraphics[width = 1.3cm]{/Users/kruppajo/work/GitHub/exam/avatare/Jessica.png}\hspace{-4mm}\includegraphics[width = 1.3cm]{/Users/kruppajo/work/GitHub/exam/avatare/Mark.png}
\end{minipage}
\begin{minipage}[t]{0.5\textwidth}
\hfill
\href{https://youtu.be/uXwlbuHOj7Y}{\includegraphics[width = 2cm]{img/youtube}}
\end{minipage}
\fi
%% --------------------------------------------------------------------



\ifcollection
\paragraph{Adjustierung multipler Vergleiche}
\fi

Mark und Jessica untersuchen gemeinsam in ihrer Abschlussarbeit den Messwert Proteingehalt [g/kg] in Erbsen. Es ist einiges schiefgelaufen, wie es immer so passiert. Hauptsächlich waren es Würmer, auch wenn man erstmal bei dem Messwert nicht unbedingt an Würmer denken würde. Aber das ist eine andere Geschichte. Jetzt wollen Mark und Jessica ihre Ergebnisse nochmal mit einer Studie von Meyer et al. (2021) vergleichen und schauen, ob was ähnliches rauskommt. Angeschaut wurde sich als Behandlung Genotypen ($00$, $AA$, $AB$ und $BB$). Es ergab sich dann die folgende Tabelle der rohen p-Werte für die Vergleiche zu Meyer et al. (2021).

\begin{knitrout}
\definecolor{shadecolor}{rgb}{0.969, 0.969, 0.969}\color{fgcolor}\begin{table}[!h]
\centering\begingroup\fontsize{10}{12}\selectfont

\begin{tabular}{ccc}
\toprule
\textbf{Rohen p-Werte} & \textbf{Adjustierte p-Werte} & \textbf{Nullhypothese ablehnen?}\\
\midrule
0.070 &  & \\
0.012 &  & \\
0.340 &  & \\
0.060 &  & \\
\bottomrule
\end{tabular}
\endgroup{}
\end{table}

\end{knitrout}

Leider kennen sich Mark und Jessica mit der Adjustierung von $p$-Werten und dem Signifikanzniveau $\alpha$ überhaupt nicht aus. Deshalb brauchen die beiden bei der Erstellung Ihre Hilfe!

\begin{enumerate}
  \item Formulieren Sie die wissenschaftliche Fragestellung! \textbf{(1 Punkt)}
  \item Formulieren Sie die statistischen Hypothesen! \textbf{(1 Punkt)}
\item Füllen Sie die Spalte \textit{Adjustierte p-Werte} nach der Bonferoni-Methode aus! \textbf{(2 Punkte)}
\item Entscheiden Sie, ob nach der Adjustierung die Nullhypothese abgelehnt werden kann! Begründen Sie Ihre Antwort! \textbf{(2 Punkte)}
\item Wie ist Ihr Vorgehen, wenn Sie anstatt der $p$-Werte das Signifikanzniveau $\alpha$ adjustieren? \textbf{(2 Punkte)}
\item Erklären Sie warum die $p$-Werte oder das Signifikanzniveau $\alpha$ bei multiplen Vergleichen adjustiert werden müssen! \textbf{(2 Punkte)}
\item Welche Adjustierung wird im Allgemeinen vorgezogen? Die Adjustierung der $p$-Werte oder die Adjustierung des Signifikanzniveaus $\alpha$? Begründen Sie Ihre Antwort! \textbf{(2 Punkte)}
\end{enumerate}


 
\clearpage
% ----------------------------------------------------------------------- 

\section{Aufgabe \hfill (10 Punkte)}

\textit{Geben Sie grundsätzlich Formeln und Rechenweg zur Lösung der Teilaufgaben mit an!} \\[1Ex]
 

 
%% --------------------------------------------------------------------
\ifcollection
\begin{flushright}
\tiny\vspace{-3Ex}
\textbf{\examinhaltstart}
\exammodulestat $\;\bullet$
\exammodulestatbbv $\;\bullet$
\exammodulestatversuch $\;\bullet$
\exammodulebiostat
\vspace{-4Ex}
\end{flushright}
\begin{minipage}[t]{0.5\textwidth}
\includegraphics[width = 1.3cm]{/Users/kruppajo/work/GitHub/exam/avatare/Mark.png}
\end{minipage}
\begin{minipage}[t]{0.5\textwidth}
\hfill
\href{https://youtu.be/VNz7nDuK148}{\includegraphics[width = 2cm]{img/youtube}}
\end{minipage}
\vspace{-3ex}
\fi
%% --------------------------------------------------------------------



\ifcollection
\paragraph{Visualisierung des Compact Letter Displays (CLD)}
\fi

Mark betrachtet in sich gekehrt die Poster vor dem Büro von seiner Betreuerin. Viele der explorativen Abbildungen sagen ihm etwas. Die Barplots und die Boxplots könnte er dann schon nachbauen. Das macht ihn dann zuversichtlich die Abschlussarbeit auch hinzukriegen. Etwas komischer sind die seltsamen Buchstaben über den Barplots. Mark betrachtet ein Poster das sich mit Puten beschäftigt. Ernährungszusatz ($ctrl$, $fedX$, $proteinX$ und $getIt$) und Schlachtgewicht [kg] wurden dort bestimmt. So richtig schlau, wird er daraus nicht.

\begin{knitrout}
\definecolor{shadecolor}{rgb}{0.969, 0.969, 0.969}\color{fgcolor}\begin{table}[!h]
\centering\begingroup\fontsize{10}{12}\selectfont

\begin{tabular}{cc}
\toprule
\textbf{Behandlung} & \textbf{Compact letter display}\\
\midrule
ctrl & BC\\
fedX & A\\
proteinX & C\\
getIt & B\\
\bottomrule
\end{tabular}
\endgroup{}
\end{table}

\end{knitrout}

Leider kennen sich Mark mit dem \textit{Compact letter display (CLD)} überhaupt nicht aus. Deshalb braucht er bei der Erstellung Ihre Hilfe!

\begin{enumerate}
  \item Formulieren Sie die wissenschaftliche Fragestellung! \textbf{(1 Punkt)}
  \item Formulieren Sie die statistischen Hypothesen! \textbf{(1 Punkt)}
\item Zeichnen Sie die sich anhand des \textit{Compact letter display (CLD)} ergebenden Barplots! \textbf{(2 Punkte)}
\item Ergänzen Sie das \textit{Compact letter display (CLD)} zu den Barplots! \textbf{(1 Punkt)}
\item Erklären Sie \textit{einen} Vorteil und \textit{einen} Nachteil des \textit{Compact letter display (CLD)}! \textbf{(2 Punkte)}
\item Erstellen Sie eine Matrix mit den paarweisen $p$-Werten eines Student t-Tests, die sich näherungsweise aus dem \textit{Compact letter display (CLD)} ergeben würde! Begründen Sie Ihre Antwort! \textbf{(3 Punkte)}
\end{enumerate}

 
\clearpage
% ----------------------------------------------------------------------- 

\section{Aufgabe \hfill (12 Punkte)}

\textit{Geben Sie grundsätzlich Formeln und Rechenweg zur Lösung der Teilaufgaben mit an!} \\[1Ex]
 

 
%% --------------------------------------------------------------------
\ifcollection
\begin{flushright}
\tiny\vspace{-3Ex}
\textbf{\examinhaltstart}
\exammodulestatversuch $\;\bullet$
\exammodulebiostat
\vspace{-4Ex}
\end{flushright}
\begin{minipage}[t]{0.5\textwidth}
\includegraphics[width = 1.3cm]{/Users/kruppajo/work/GitHub/exam/avatare/Jonas.png}
\end{minipage}
\begin{minipage}[t]{0.5\textwidth}
\hfill
\href{https://youtu.be/951MNUXdqwA}{\includegraphics[width = 2cm]{img/youtube}}
\end{minipage}
\vspace{-3ex}
\fi
%% --------------------------------------------------------------------



\ifcollection
\paragraph{Berechnung des Compact Letter Displays (CLD) anhand von t-Tests}
\fi

Jonas hatte in seine Hausarbeit ein Stallexperiment durchgeführt. Soweit so gut. Dabei hat er sich mit Zandern beschäftigt. Angeblich der neueste heiße Kram... aber das ist wiederum was anderes. So richtig mitgenommen hat Jonas das Thema dann doch nicht. Hat er sich doch mit Ernährungszusatz ($ctrl$, $fedX$, $proteinX$ und $getIt$) und Fettgehalt [\%/kg] schon eine Menge an Daten angeschaut. Nach seine Betreuerin soll er nun ein CLD bestimmen. Weder weiß er was ein CLD ist, noch war sein erster Gedanke mit Köln und die LGBTQ Community richtig... Als erstes solle er die Gruppen nach absteigender Effektstärke sortieren. Was immer das jetzt bringen soll.

\begin{knitrout}
\definecolor{shadecolor}{rgb}{0.969, 0.969, 0.969}\color{fgcolor}\begin{table}[!h]
\centering\begingroup\fontsize{10}{12}\selectfont

\begin{tabular}{cccc}
\toprule
\textbf{Ernährungszusatz} & \textbf{Fallzahl (n)} & \textbf{Mittelwert} & \textbf{Standardabweichung}\\
\midrule
ctrl & 7 & 11.57 & 3.02\\
fedX & 9 & 10.28 & 2.79\\
proteinX & 7 & 6.23 & 1.45\\
getIt & 9 & 10.13 & 1.53\\
\bottomrule
\end{tabular}
\endgroup{}
\end{table}

\end{knitrout}

Leider kennen sich Jonas mit dem \textit{Compact letter display (CLD)} überhaupt nicht aus. Deshalb braucht er bei der Erstellung Ihre Hilfe!

\begin{enumerate}
  \item Formulieren Sie die wissenschaftliche Fragestellung! \textbf{(1 Punkt)}
  \item Formulieren Sie die statistischen Hypothesen! \textbf{(1 Punkt)}
\item Zeichnen Sie die sich ergebenden Barplots! \textbf{(1 Punkt)}
\item Berechnen Sie die Matrix der $p$-Werte anhand von Student t-Tests! \textit{Nutzen Sie hierfür ein globales $s_p$ sowie eine gemittelte Fallzahl $n$ für die Berechnung der Teststatistik! Nehmen Sie einen kritischen Wert $T_{\alpha = 5\%}$ von 1.96 an!} \textbf{(4 Punkte)}
\item Ergänzen Sie das \textit{Compact letter display (CLD)} zu den gezeichneten Barplots! Begründen Sie Ihre Antwort! \textbf{(4 Punkte)}
\item Interpretieren Sie das \textit{Compact letter display (CLD)} für Jonas! \textbf{(1 Punkt)} 
\end{enumerate}

 
\clearpage
% -----------------------------------------------------------------------
\part{Der Chi-Quadrat-Test}
% -----------------------------------------------------------------------

\section{Aufgabe \hfill (10 Punkte)}

\textit{Geben Sie grundsätzlich Formeln und Rechenweg zur Lösung der Teilaufgaben mit an!} \\[1Ex]
 

 
%% --------------------------------------------------------------------
\ifcollection
\begin{flushright}
\tiny\vspace{-3Ex}
\textbf{\examinhaltstart}
\exammodulestat $\;\bullet$
\exammodulestatbbv 
\vspace{-4Ex}
\end{flushright}
\begin{minipage}[t]{0.5\textwidth}
\includegraphics[width = 1.3cm]{/Users/kruppajo/work/GitHub/exam/avatare/Mark.png}
\end{minipage}
\begin{minipage}[t]{0.5\textwidth}
\hfill
\href{https://youtu.be/1y_X_TI-Gm4}{\includegraphics[width = 2cm]{img/youtube}}
\end{minipage}
\vspace{-3Ex}
\fi
%% --------------------------------------------------------------------



\ifcollection
\paragraph{Den Chi-Quadrat-Test berechnen}
\fi

Am Ende war es für Mark in seinem Projektbericht dann doch kein normalverteiltes Outcome. Das was jetzt etwas doof, da er sich auf eine ANOVA gefreut hatte. Dann noch schnell Andrea Berg auf das Ohr und los gehts. Prinzipiell ginge das auch irgendwie, aber nun möchte sein Betreuer gerne einen $\mathcal{X}^2$-Test auf einer $2x2$-Kreuztabelle berechnet bekommen. Mark hatte sich in ein Feldexperiment $n = 170$ Beobachtungen von Maiss angeschaut. Dabei hat er als Behandlung \textit{Pestizideinsatz [ja/nein]} bestimmt und zum anderen die Variable \textit{Frischegewicht über Zielwert [ja/nein]} ermittelt. Jetzt muss Mark mal schauen, wie er das jetzt rechnet. Eigentlich wollte Mark nachher noch zum Sport. Einfach mal raus um zu Reiten. Ohne Ziel und Uhr. Das ist was für Mark.

\vspace{5Ex}

\begin{center}
  \huge
  \begin{tabular}{c|l|l|c}
     & \phantom{\textbf{Erkrankt (ja)}} & \phantom{\textbf{Erkrankt (ja)}} & \phantom{\textbf{Erkrankt (ja)}} \strut\\
    \hline
    \phantom{\textbf{Pestizid (ja)}} & 38  & 51  &     \strut\\
    \hline
    \phantom{\textbf{Pestizid (ja)}} & 43  & 38  &      \strut\\
    \hline
     \phantom{100} & \phantom{100}  & \phantom{100}  &  \phantom{100}  \strut\\
  \end{tabular}
\end{center}

\vspace{5Ex}

Leider kennt sich Mark mit der Berechnung eines $\mathcal{X}^2$-Test für kategoriale Daten überhaupt nicht aus. Deshalb braucht er bei der Erstellung Ihre Hilfe!

\begin{enumerate}
\item Formulieren Sie die wissenschaftliche Fragestellung! \textbf{(1 Punkt)}
\item Ergänzen Sie die Tabelle um die fehlenden Informationen! \textbf{(1 Punkt)} 
\item Visualisieren Sie den Zusammenhang zwischen den beiden kategorialen Variablen! \textbf{(2 Punkte)}
\item Berechnen Sie die Teststatistik eines Chi-Quadrat-Test! \textbf{(2 Punkte)}
\item Treffen Sie eine Entscheidung im Bezug zu der Nullhypothese gegeben
  einem $\mathcal{X}^2_{\alpha = 5\%} = 3.83$! Begründen Sie Ihre Antwort!
  \textbf{(2 Punkte)}
\item Skizzieren Sie in einer Abbildung die $\mathcal{X}^2$-Verteilung, wenn die $H_0$ wahr ist! Ergänzen Sie  $\mathcal{X}^2_{\alpha = 5\%}$ und $\mathcal{X}^2_{D}$ in der Abbildung! \textit{Beachten Sie folgenden Informationen zur $\mathcal{X}^2$-Verteilung. Die $\mathcal{X}^2$-Verteilung hat ein Maxima bei $\mathcal{X}^2 = 5$ sowie ein Minima bei $\mathcal{X}^2 = 8$.} \textbf{(2 Punkte)}
\end{enumerate} 
\clearpage
% -----------------------------------------------------------------------

\section{Aufgabe \hfill (11 Punkte)}

\textit{Geben Sie grundsätzlich Formeln und Rechenweg zur Lösung der Teilaufgaben mit an!} \\[1Ex]
 

 
%% --------------------------------------------------------------------
\ifcollection
\begin{flushright}
\tiny\vspace{-3Ex}
\textbf{\examinhaltstart}
\exammodulestat $\;\bullet$
\exammodulestatbbv 
\vspace{-4Ex}
\end{flushright}
\begin{minipage}[t]{0.5\textwidth}
\includegraphics[width = 1.3cm]{/Users/kruppajo/work/GitHub/exam/avatare/Tina.png}
\end{minipage}
\begin{minipage}[t]{0.5\textwidth}
\hfill
\href{https://youtu.be/IcjUIoEo0rY}{\includegraphics[width = 2cm]{img/youtube}}
\end{minipage}
\vspace{-3Ex}
\fi
%% --------------------------------------------------------------------



\ifcollection
\paragraph{Den Chi-Quadrat-Test mit Effektmaß berechnen}
\fi

Am Ende hätte Tina dann doch einen normalverteilten Endpunkt in ihrem Projektbericht nehmen sollen. Vor ihr liegen jetzt die Daten von zwei Variablen als Kategorien oder wie es in \Rlogo so schön heißt, als Faktoren. Aber immerhin, hofft sie das was bei den Daten rausgekommen ist. Dann noch schnell Tocotronic auf das Ohr und los gehts. Gezählt hat Tina einiges mit $n = 122$ Beobachtungen von Puten. Zum einen hat sie als Behandlung \textit{Automatische Fütterung [ja/nein]} bestimmt und zum anderen die Variable \textit{Gewichtszuwachs erreicht [ja/nein]} ermittelt. Nun möchte ihre Betreuerin gerne einen $\mathcal{X}^2$-Test auf einer $2x2$-Kreuztabelle berechnet bekommen. Eigentlich wollte Tina nachher noch einen Film schauen. Wenn Indiana Jones läuft, dann ist die Spinne nicht mehr da. Aber jetzt braucht sie mal Entspannung!

\vspace{5Ex}

\begin{center}
  \huge
  \begin{tabular}{c|l|l|c}
     & \phantom{\textbf{Erkrankt (ja)}} & \phantom{\textbf{Erkrankt (ja)}} & \phantom{\textbf{Erkrankt (ja)}} \strut\\
    \hline
    \phantom{\textbf{Pestizid (ja)}} & 38  & 19  &     \strut\\
    \hline
    \phantom{\textbf{Pestizid (ja)}} & 27  & 38  &      \strut\\
    \hline
     \phantom{100} & \phantom{100}  & \phantom{100}  &  \phantom{100}  \strut\\
  \end{tabular}
\end{center}

\vspace{5Ex}

Leider kennt sich Tina mit der Berechnung eines $\mathcal{X}^2$-Test für kategoriale Daten überhaupt nicht aus. Deshalb braucht sie bei der Erstellung Ihre Hilfe!

\begin{enumerate}
\item Formulieren Sie die wissenschaftliche Fragestellung! \textbf{(1 Punkt)}
\item Ergänzen Sie die Tabelle um die fehlenden Informationen! \textbf{(1 Punkt)} 
\item Berechnen Sie die Teststatistik eines Chi-Quadrat-Test! \textbf{(3 Punkte)}
\item Treffen Sie eine Entscheidung im Bezug zu der Nullhypothese gegeben
  einem $\mathcal{X}^2_{\alpha = 5\%} = 5.61$! Begründen Sie Ihre Antwort!
  \textbf{(2 Punkte)}
\item Berechnen Sie den Effektschätzer $Cramers\; V$ auf der 2x2 Kreuztabelle! \textbf{(1 Punkt)}
\item Welchen Wertebereich kann der Effektschätzer $Cramers\; V$ annehmen? Wann liegt kein Effekt und wann ein starker Effekt vor? \textbf{(2 Punkte)}
\item Interpretieren Sie den berechneten Effektschätzer $Cramers\; V$ unter Berücksichtigung der Fragestellung! \textbf{(1 Punkt)}
\end{enumerate} 
\clearpage
% -----------------------------------------------------------------------

\section{Aufgabe \hfill (10 Punkte)}

\textit{Geben Sie grundsätzlich Formeln und Rechenweg zur Lösung der Teilaufgaben mit an!} \\[1Ex]
 

 
%% --------------------------------------------------------------------
\ifcollection
\begin{flushright}
\tiny\vspace{-3Ex}
\textbf{\examinhaltstart}
\exammodulestat $\;\bullet$
\exammodulestatbbv 
\vspace{-4Ex}
\end{flushright}
\begin{minipage}[t]{0.5\textwidth}
\includegraphics[width = 1.3cm]{/Users/kruppajo/work/GitHub/exam/avatare/Jonas.png}
\end{minipage}
\begin{minipage}[t]{0.5\textwidth}
\hfill
\href{https://youtu.be/LzAAdJPH4ZU}{\includegraphics[width = 2cm]{img/youtube}}
\end{minipage}
\vspace{-3Ex}
\fi
%% --------------------------------------------------------------------



\ifcollection
\paragraph{Den Chi-Quadrat-Test in einem Fragebogen berechnen}
\fi

Jonas hatte sich gleich von Beginn an in seiner Hausarbeit für eine Umfrage im Marketing interessiert. Jetzt geht es um den Haupt- und Nebenerwerb von Erlebnishöfen in Norddeutschland. Viele Höfe haben angefangen auch Großkatzen zu halten, damit mehr Kunden auf die Höfe kommen. Für den Verband der Großkatzenbesitzer e.V. möchte er nun einen Fragebogen zur Zukunftsfähigkeit Schritt für Schritt auswerten. Dabei teilt er zuerst die Antwortenden in die beiden Gruppen 'Höfe mit Großkatzen [ja]' und 'Höfe mit Großkatzen [nein]' ein. Daraufhin möchte er für folgende Frage \textit{f3verband} einmal auswerten, ob es einen Unterschied zwischen den beiden Höfen mit oder ohne Großkatzen gibt. \begin{center}\textit{Halten Sie die Verbandsarbeit für die Verbreitung von Großkatzen im ländlichen Raum für sinnvoll?}\end{center}Jonas krazt sich an seinem Kopf. Wie soll man eine Tabelle mit so vielen Zahlen sinnvoll auswerten? Schnell noch ein paar Snickers einwerfen und los gehts!

\vspace{5Ex}

\begin{center}
  \Large
  \begin{tabular}{c|l|l|l|l|l|c}
\textit{f3verband}     & \textbf{trifft gar} & \textbf{trifft} & \textbf{weder} & \textbf{trifft} & \textbf{trifft} & \\
 \phantom{\textbf{Pestizid}}    & \textbf{nicht zu} & \textbf{nicht zu} & \textbf{noch} & \textbf{zu} & \textbf{voll zu} & \\
    \hline
   \textbf{ja}  & 4  & 9  &  11  & 29  & 17  &   \strut\\[4Ex]
    \hline
    \textbf{nein} & 12  & 29  & 13  & 10  & 1  &      \strut\\[4Ex]
    \hline
     \phantom{\textbf{Pestizid}} & \phantom{\textbf{trifft gar}}  & \phantom{\textbf{trifft gar}}  &  \phantom{\textbf{trifft gar}} &  \phantom{\textbf{trifft gar}} &  \phantom{\textbf{trifft gar}} &  \phantom{100} \strut\\
  \end{tabular}
\end{center}

\vspace{5Ex}

Leider kennt sich Jonas mit der Berechnung eines $\mathcal{X}^2$-Test auf einer Frage in einem Fragebogen überhaupt nicht aus. Deshalb braucht er bei der Erstellung Ihre Hilfe!

\begin{enumerate}
\item Formulieren Sie die wissenschaftliche Fragestellung! \textbf{(1 Punkt)}
\item Ergänzen Sie die Tabelle um die fehlenden Informationen! \textbf{(1 Punkt)} 
\item Berechnen Sie die Teststatistik $\mathcal{X}^2_{D}$ eines Chi-Quadrat-Test! \textit{Ignorieren Sie Zellbelegungen \underline{kleiner gleich} fünf in der Berechnung von $\mathcal{X}^2_{D}$ und Runden Sie den erwarteten Wert $E$ ganzzahlig!} \textbf{(2 Punkte)}
\item Treffen Sie eine Entscheidung im Bezug zu der Nullhypothese gegeben
  einem $\mathcal{X}^2_{\alpha = 5\%} = 11.56$! Begründen Sie Ihre Antwort!
  \textbf{(2 Punkte)}
\item Visualisieren Sie die 2x5 Kreuztabelle \textit{ohne} die Berücksichtigung der Antwortkategorie 'weder noch'! \textbf{(2 Punkte)}
\item Berechnen Sie den Effektschätzer $Cramers\; V$ auf der 2x5 Kreuztabelle! \textbf{(1 Punkt)}
\item Interpretieren Sie den berechneten Effektschätzer $Cramers\; V$ unter Berücksichtigung der Fragestellung! \textbf{(1 Punkt)}
\end{enumerate} 
\clearpage
% -----------------------------------------------------------------------
\part{Lineare Regression \& Korrelation}
% -----------------------------------------------------------------------

\section{Aufgabe \hfill (10 Punkte)}

\textit{Geben Sie grundsätzlich Formeln und Rechenweg zur Lösung der Teilaufgaben mit an!} \\[1Ex]
 

 
%% --------------------------------------------------------------------
\ifcollection
\begin{flushright}
\tiny\vspace{-3Ex}
\textbf{\examinhaltstart}
\exammodulestat $\;\bullet$
\exammodulestatbbv $\;\bullet$
\exammodulestatversuch 
\vspace{-4Ex}
\end{flushright}
\begin{minipage}[t]{0.5\textwidth}
\includegraphics[width = 1.3cm]{/Users/kruppajo/work/GitHub/exam/avatare/Alex.png}\hspace{-4mm}\includegraphics[width = 1.3cm]{/Users/kruppajo/work/GitHub/exam/avatare/Paula.png}
\end{minipage}
\begin{minipage}[t]{0.5\textwidth}
\hfill
\href{https://youtu.be/LtbC_K9V5wU}{\includegraphics[width = 2cm]{img/youtube}}
\end{minipage}
\fi
%% --------------------------------------------------------------------



\ifcollection
\paragraph{Visualisierung der linearen Regression}
\fi

'Ich glaube du bringst da was durcheinander. Wir nutzen zwar auch für die ANOVA die Funktion \texttt{lm()} aber hier wollen wir, glaube ich, eine Gerade durch die Punkte zeichnen.', merkt Alex an. 'Ich sehe keine Punkte...', antwortet Paula sichtlich übernächtigt. 'Wir müssen die Daten ja auch erst visualisieren!', spricht Alex sehr deutlich und langsam. Die beiden hatten ein Stallexperiment im Wendland mit Lamas durchgeführt. Dabei wurden die beiden folgenden Variablen gemessen: mittlerer Eisenkonzentration [Fe/ml] und Schlachtgewicht [kg]. Jetzt wollen sie erstmal schauen, ob es einen Zusammenhang gibt.

\begin{table}[!h]
\centering
\begin{tabular}{cc}
\toprule
Mittlerer Eisenkonzentration [Fe/ml] & Schlachtgewicht [kg]\\
\midrule
23.3 & 22.7\\
23.8 & 26.2\\
22.8 & 23.9\\
14.9 & 19.1\\
26.9 & 28.5\\
\addlinespace
20.7 & 24.0\\
15.2 & 17.9\\
23.0 & 21.6\\
17.2 & 23.3\\
24.8 & 27.9\\
\addlinespace
11.0 & 13.3\\
\bottomrule
\end{tabular}
\end{table}



Leider kennen sich Alex und Paula mit der linearen Regression für kontinuierliche Daten überhaupt nicht aus. Deshalb brauchen beide bei der Erstellung Ihre Hilfe!

\begin{enumerate}
\item Formulieren Sie die wissenschaftliche Fragestellung! \textbf{(1 Punkt)}
\item Erstellen  Sie  eine  Visualisierung  für  die  Datentabelle.  Beschriften  Sie  die  Achsen! \textbf{(2 Punkte)}
\item Schätzen Sie die Regressionsgleichung aus der obigen Abbildung ab! Begründen Sie Ihre Antwort mit einer Skizze der Methodik an ausgewählten Punkten! \textbf{(2 Punkte)}
\item Beschriften Sie die Grade mit den statistischen Maßzahlen der linearen Regressionsgleichung! \textbf{(2 Punkte)}
\item Liegt ein Zusammenhang zwischen $x$ und $y$ vor? Begründen Sie Ihre Antwort! \textbf{(2 Punkte)}
\item Wenn kein Zusammenhang zu beobachten wäre, wie würde die Grade aussehen? \textit{Antworten Sie mit einer Skizze der Geraden!} \textbf{(1 Punkt)}
\end{enumerate} 
\clearpage
% -----------------------------------------------------------------------

\section{Aufgabe \hfill (12 Punkte)}

\textit{Geben Sie grundsätzlich Formeln und Rechenweg zur Lösung der Teilaufgaben mit an!} \\[1Ex]
 

 
%% --------------------------------------------------------------------
\ifcollection
\begin{flushright}
\tiny\vspace{-3Ex}
\textbf{\examinhaltstart}
\exammodulestat $\;\bullet$
\exammodulestatbbv $\;\bullet$
\exammodulestatversuch 
\vspace{-4Ex}
\end{flushright}
\begin{minipage}[t]{0.5\textwidth}
\includegraphics[width = 1.3cm]{/Users/kruppajo/work/GitHub/exam/avatare/Mark.png}\hspace{-4mm}\includegraphics[width = 1.3cm]{/Users/kruppajo/work/GitHub/exam/avatare/Nilufar.png}
\end{minipage}
\begin{minipage}[t]{0.5\textwidth}
\hfill
\href{https://youtu.be/5eP9VjecZmk}{\includegraphics[width = 2cm]{img/youtube}}
\end{minipage}
\fi
%% --------------------------------------------------------------------



\ifcollection
\paragraph{Interpretation der Ergebnisse einer linearen Regression}
\fi

'Ich glaube du bringst da was durcheinander. Wir nutzen zwar auch für die ANOVA die Funktion \texttt{lm()} aber hier wollen wir, glaube ich, eine Gerade durch die Punkte zeichnen.', merkt Nilufar an. 'Ich sehe keine Punkte... ich sehe nur zwei Zeilen einer Tabelle und ich glaube du hast gerade was gelöscht.', antwortet Mark sichtlich übernächtigt. 'Wir müssen die Koeffizienten der linearen Regression ja auch erst interpretieren!', spricht Nilufar sehr deutlich und langsam. Die beiden hatten einen Leistungssteigerungsversuch im Oldenburger Land mit Puten durchgeführt. Dabei wurden die beiden folgenden Variablen gemessen: durchschnittlicher Bewegungsscore [Movement/h] und Schlachtgewicht [kg]. Das Bestimmtheitsmaß $R^2$ hatten die beiden mit 0.2 bestimmt.  Jetzt wollen sie erstmal schauen, ob es einen Zusammenhang gibt und das soll mit der \Rlogo Ausgabe möglich sein.

\begin{table}[!h]
\centering\begingroup\fontsize{11}{13}\selectfont

\begin{tabular}{ccccc}
\toprule
term & estimate & std.error & t statistic & p-value\\
\midrule
(Intercept) & 3.87 & 1.97 &  & \\
Durchschnittlicher Bewegungsscore & 0.34 & 0.20 &  & \\
\bottomrule
\end{tabular}
\endgroup{}
\end{table}



Leider kennen sich Nilufar und Mark mit der linearen Regression für kontinuierliche Daten in \Rlogo überhaupt nicht aus. Deshalb brauchen beide bei der Erstellung Ihre Hilfe!

\begin{enumerate}
\item Formulieren Sie die wissenschaftliche Fragestellung! \textbf{(1 Punkt)}
\item Formulieren Sie die Regressionsgleichung! \textbf{(1 Punkt)}
\item Erstellen  Sie  eine  Visualisierung  der \texttt{lm()}-Ausgabe. \textit{Beachten Sie die Informationen zum Bestimmtheitsmaß $R^2$ aus dem Aufgabentext!} Beschriften  Sie  die  Achsen! \textbf{(2 Punkte)}
\item Beschriften Sie die Visualisierung mit den statistischen Maßzahlen der der \texttt{lm()}-Ausgabe! \textbf{(2 Punkte)}
\item Ergänzen Sie die t-Statistik in der \texttt{lm()}-Ausgabe! \textbf{(2 Punkte)}
\item Ergänzen Sie den $p$-Wert in der \texttt{lm()}-Ausgabe mit $T_{\alpha = 5\%} = 1.96$!  \textbf{(2 Punkte)}
\item Interpretieren Sie den $p$-Wert im Kontext der wissenschaftlichen Fragestellung! \textbf{(1 Punkt)}  
\item Wie groß ist der Effekt im Kontext der wissenschaftlichen Fragestellung? \textbf{(1 Punkt)}
\end{enumerate} 
\clearpage
% -----------------------------------------------------------------------

\section{Aufgabe \hfill (11 Punkte)}

\textit{Geben Sie grundsätzlich Formeln und Rechenweg zur Lösung der Teilaufgaben mit an!} \\[1Ex]
 

 
%% --------------------------------------------------------------------
\ifcollection
\begin{flushright}
\tiny\vspace{-3Ex}
\textbf{\examinhaltstart}
\exammodulestat $\;\bullet$
\exammodulestatbbv $\;\bullet$
\exammodulestatversuch $\;\bullet$
\exammodulebiostat
\vspace{-4Ex}
\end{flushright}
\begin{minipage}[t]{0.5\textwidth}
\includegraphics[width = 1.3cm]{/Users/kruppajo/work/GitHub/exam/avatare/Mark.png}\hspace{-4mm}\includegraphics[width = 1.3cm]{/Users/kruppajo/work/GitHub/exam/avatare/Tina.png}
\end{minipage}
\begin{minipage}[t]{0.5\textwidth}
\hfill
\href{https://youtu.be/I48edRN4bAU}{\includegraphics[width = 2cm]{img/youtube}}
\end{minipage}
\fi
%% --------------------------------------------------------------------



\ifcollection
\paragraph{Interpretation der Ergebnisse einer linearen Regression in \Rlogo}
\fi

'Hä? Was ist denn das? Das wird ja immer wilder! Hatten wir das als Aufgabe eine lineare Regression zu rechnen? Wir bauen aus kontinuierlichen Daten eine Abbildung und interpretieren diese dann?', fragt Tina. Mark schaut fragend zurück. 'Keine Ahnung... das ist jetzt jedenfalls keine Abbildung von irgendwas sondern eine \Rlogo Ausgabe mit ganz wilden Bezeichnungen...', antwortet Mark leicht angespannt. Die beiden hatten ein Freilandversuch in der Uckermark mit Spargel durchgeführt. Dabei wurden die beiden folgenden Variablen gemessen: durchschnittliche Regenwurmdichte [Anzahl/l] und Chlorophyllgehalt (SPAD-502Plus) [SPAD]. Jetzt haben die beiden eigentlich alles zusammen. \textit{Eigentlich...}, denn mit der \Rlogo Ausgabe haben beide jetzt ein Problem.

\begin{knitrout}
\definecolor{shadecolor}{rgb}{0.969, 0.969, 0.969}\color{fgcolor}\begin{kframe}
\begin{verbatim}
## 
## Call:
## Chlorophyllgehalt ~ Durchschnittliche_Regenwurmdichte
## 
## Residuals:
##     Min      1Q  Median      3Q     Max 
## -3.2149 -0.9110  0.1736  0.7860  3.0845 
## 
## Coefficients:
##                                   Estimate Std. Error t value Pr(>|t|)
## (Intercept)                         2.3844     1.6287   1.464    0.151
## Durchschnittliche_Regenwurmdichte   0.1054     0.1624   0.649    0.520
## 
## Residual standard error: 1.495 on 41 degrees of freedom
## Multiple R-squared:  0.01018,	Adjusted R-squared:  -0.01396 
## F-statistic: 0.4216 on 1 and 41 DF,  p-value: 0.5198
\end{verbatim}
\end{kframe}
\end{knitrout}

Leider kennen sich Tina und Mark mit der linearen Regression für kontinuierliche Daten in \Rlogo überhaupt nicht aus. Deshalb brauchen beide bei der Erstellung Ihre Hilfe!


\begin{enumerate}
\item Formulieren Sie die wissenschaftliche Fragestellung! \textbf{(1 Punkt)}
\item Wie groß ist der Effekt im Kontext der wissenschaftlichen Fragestellung? \textbf{(2 Punkte)} 
\item Interpretieren Sie die $p$-Werte im Kontext der wissenschaftlichen Fragestellung! \textbf{(2 Punkte)}
\item Visualisieren Sie die Verteilung der Residuen! \textbf{(2 Punkte)} 
\item Ist die Annahme der Normalverteilung erfüllt? Begründen Sie die Antwort! \textbf{(2 Punkte)}
\item Erklären Sie \textit{kurz} den Begriff \texttt{R-squared}! Was sagt Ihnen der Wert aus? \textbf{(2 Punkte)}
\end{enumerate}
 
\clearpage
% -----------------------------------------------------------------------

\section{Aufgabe \hfill (10 Punkte)}

\textit{Geben Sie grundsätzlich Formeln und Rechenweg zur Lösung der Teilaufgaben mit an!} \\[1Ex]
 

 
%% --------------------------------------------------------------------
\ifcollection
\begin{flushright}
\tiny\vspace{-3Ex}
\textbf{\examinhaltstart}
\exammodulestat $\;\bullet$
\exammodulestatbbv $\;\bullet$
\exammodulestatversuch $\;\bullet$
\exammodulebiostat
\vspace{-4Ex}
\end{flushright}
\begin{minipage}[t]{0.5\textwidth}
\includegraphics[width = 1.3cm]{/Users/kruppajo/work/GitHub/exam/avatare/Mark.png}
\end{minipage}
\begin{minipage}[t]{0.5\textwidth}
\hfill
\href{https://youtu.be/C9skfFRTHhI}{\includegraphics[width = 2cm]{img/youtube}}
\end{minipage}
\vspace{-3ex}
\fi
%% --------------------------------------------------------------------



\ifcollection
\paragraph{Interpretation der Ergebnisse einer Korrelationsanalyse in \Rlogo}
\fi

'Hä? Was ist denn das? Das wird ja immer wilder! Hatten wir als Aufgabe eine Korrelation zu berechnen? Wir bauen aus kontinuierlichen Daten eine Abbildung und interpretieren diese dann?', fragt sich Mark laut. 'Keine Ahnung... das ist jetzt jedenfalls keine Abbildung von irgendwas sondern eine \Rlogo Ausgabe mit ganz wilden Bezeichnungen...', denkt er. Aber das hilft jetzt auch nicht. Wenn Columbo läuft, dann ist der Hamster nicht mehr da. Aber jetzt braucht er mal Entspannung! Mark hatte ein Feldexperiment im Emsland mit Maiss durchgeführt. Dabei wurden die beiden folgenden Variablen gemessen: durchschnittliche Regenwurmdichte [Anzahl/l] und Trockengewicht [kg/ha]. Jetzt hat er eigentlich alles zusammen. \textit{Eigentlich...}, denn mit der \Rlogo Ausgabe hat Mark jetzt ein Problem. Wenn die Unsicherheit nicht wäre, ja dann wäre wohl vieles möglich für Mark! Aber so.. 


\begin{knitrout}
\definecolor{shadecolor}{rgb}{0.969, 0.969, 0.969}\color{fgcolor}\begin{kframe}
\begin{verbatim}
## 
## 	Pearson's correlation
## 
## data:  Durchschnittliche Regenwurmdichte and Trockengewicht
## t = 3.1246, df = 8, p-value = 0.01413
## alternative hypothesis: true correlation is not equal to 0
## 95 percent confidence interval:
##  0.2095603 0.9346933
## sample estimates:
##       cor 
## 0.7413671
\end{verbatim}
\end{kframe}
\end{knitrout}

Leider kennt sich Mark mit der Korrelationsanalyse in \Rlogo überhaupt nicht aus. Deshalb braucht er bei der Erstellung Ihre Hilfe!

\begin{enumerate}
  \item Formulieren Sie die wissenschaftliche Fragestellung! \textbf{(1 Punkt)}
  \item Formulieren Sie das statistische Hypothesenpaar! \textbf{(1 Punkt)}
\item Erstellen Sie eine Visualisierung für den Korrelationskoeffizienten! Beschriften Sie die Abbildung! \textbf{(2 Punkte)}
\item Nennen Sie die zwei Eigenschaften des Korrelationskoeffizienten! \textbf{(2 Punkte)}
\item Interpretieren Sie den Korrelationskoefizienten hinsichtlich des
  Effekts und der Signifikanz! Begründen Sie Ihre Antwort! \textbf{(2 Punkte)}
\item Visualisieren Sie das 95\% Konfidenzintervall! Beschriften Sie die Abbildung! \textbf{(2 Punkte)} 
\end{enumerate} 
\clearpage
% -----------------------------------------------------------------------

\section{Aufgabe \hfill (12 Punkte)}

\textit{Geben Sie grundsätzlich Formeln und Rechenweg zur Lösung der Teilaufgaben mit an!} \\[1Ex]
 

 
%% --------------------------------------------------------------------
\ifcollection
\begin{flushright}
\tiny\vspace{-3Ex}
\textbf{\examinhaltstart}
\exammodulestat $\;\bullet$
\exammodulestatbbv $\;\bullet$
\exammodulestatversuch $\;\bullet$
\exammodulebiostat
\vspace{-4Ex}
\end{flushright}
\begin{minipage}[t]{0.5\textwidth}
\includegraphics[width = 1.3cm]{/Users/kruppajo/work/GitHub/exam/avatare/Jonas.png}
\end{minipage}
\begin{minipage}[t]{0.5\textwidth}
\hfill
\href{https://youtu.be/g7imB3A5_zw}{\includegraphics[width = 2cm]{img/youtube}}
\end{minipage}
\vspace{-3ex}
\fi
%% --------------------------------------------------------------------



\ifcollection
\paragraph{Visualisierung der Korrelation und des Bestimmtheitsmaßes}
\fi

Wenn Mission Impossible läuft, dann ist das Meerschweinchen nicht mehr da. Aber jetzt braucht er mal Entspannung! Da hilft dann die Aufgabe auch nur bedingt. 'Hm..., drei leere Abbildungen. Was soll ich da jetzt machen?', fragt sich Jonas und mampft noch ein paar Snickers in sich hinein. Jonas kennt sich nur begrenzt bis gar nicht mit der linearen Regresion und Korrelation aus.
\vspace{2Ex}



{\centering \includegraphics[width=\maxwidth]{img/correlation-01-1} 

}




\vspace{2Ex}

Leider kennt sich Jonas mit der Korrelationsanalyse und der linearen Regression überhaupt nicht aus. Deshalb braucht er bei der Auswertung Ihre Hilfe!

\begin{enumerate}
\item Zeichnen Sie für die $\rho$-Werte eine Gerade in die entsprechende Abbildung! \textbf{(3 Punkte)}
\item Zeichnen Sie für die $R^2$-Werte die entsprechende Punktewolke um die Gerade! \textbf{(3 Punkte)}
\item Nennen Sie die zwei Eigenschaften des Korrelationskoeffizienten! \textbf{(2 Punkte)}
\item Interpretieren Sie die $R^2$-Werte für die jeweilige Gerade! \textbf{(2 Punkte)}
\item Warum müssen Sie ein $R^2$-Wert berechnen, wenn Sie die einfachere Möglichkeit der visuellen Überprüfung haben? Begründen Sie Ihre Antwort! \textbf{(2 Punkte)}
\end{enumerate}
 
\clearpage
% -----------------------------------------------------------------------

\section{Aufgabe \hfill (12 Punkte)}

\textit{Geben Sie grundsätzlich Formeln und Rechenweg zur Lösung der Teilaufgaben mit an!} \\[1Ex]
 

 
%% --------------------------------------------------------------------
\ifcollection
\begin{flushright}
\tiny\vspace{-3Ex}
\textbf{\examinhaltstart}
\exammodulestat $\;\bullet$
\exammodulestatbbv $\;\bullet$
\exammodulestatversuch $\;\bullet$
\exammodulebiostat
\vspace{-4Ex}
\end{flushright}
\begin{minipage}[t]{0.5\textwidth}
\includegraphics[width = 1.3cm]{/Users/kruppajo/work/GitHub/exam/avatare/Steffen.png}
\end{minipage}
\begin{minipage}[t]{0.5\textwidth}
\hfill
\href{https://youtu.be/2QJa19ZwLls}{\includegraphics[width = 2cm]{img/youtube}}
\end{minipage}
\vspace{-3ex}
\fi
%% --------------------------------------------------------------------



\ifcollection
\paragraph{Schätzen der Korrelation und des Bestimmtheitsmaßes}
\fi

Heute hat leider Steffen Lerngruppe keine Zeit, so dass dann Steffen alles hier alleine machen muss. Das ist schon recht nervig. Steffen versteht nicht viel von $\rho$-Werten oder gar $R^2$-Werten. Geschweige denn hat er eine Idee, wie er die Werte abschätzen sollte. Müde dreht er Taylor Swift auf und lässt sich in Gedanken treiben. Gestern war er viel zu lange auf.
\vspace{2Ex}



{\centering \includegraphics[width=\maxwidth]{img/correlation-02-1} 

}




Leider kennt sich Steffen mit der Korrelationsanalyse und der linearen Regression überhaupt nicht aus. Deshalb braucht er bei der Auswertung Ihre Hilfe!

\begin{enumerate}
\item Schätzen Sie die $\rho$-Werte in den Abbildungen! \textbf{(2 Punkte)}
\item Schätzen Sie die $R^2$-Werte in den Abbildungen! \textbf{(2 Punkte)}
\item Interpretieren Sie die $R^2$-Werte für die jeweilige Gerade! \textbf{(2 Punkte)}
\item Was ist der optimale $R^2$-Wert im Kontext einer wissenschaftlichen Fragestellung? Begründen Sie Ihre Antwort an einem Beispiel! \textbf{(2 Punkte)}
\item Was ist der optimale $\rho$-Wert im Kontext einer wissenschaftlichen Fragestellung? Begründen Sie Ihre Antwort an einem Beispiel! \textbf{(2 Punkte)}
\item Erklären Sie die Aussage \textit{"Correlation does not imply causation!"} an einem Beispiel! \textbf{(2 Punkte)}
\end{enumerate} 
\clearpage
% -----------------------------------------------------------------------

\section{Aufgabe \hfill (12 Punkte)}

\textit{Geben Sie grundsätzlich Formeln und Rechenweg zur Lösung der Teilaufgaben mit an!} \\[1Ex]
 

 
%% --------------------------------------------------------------------
\ifcollection
\begin{flushright}
\tiny\vspace{-3Ex}
\textbf{\examinhaltstart}
\exammodulebiostat
\vspace{-4Ex}
\end{flushright}
\begin{minipage}[t]{0.5\textwidth}
\includegraphics[width = 1.3cm]{/Users/kruppajo/work/GitHub/exam/avatare/Mark.png}\hspace{-4mm}\includegraphics[width = 1.3cm]{/Users/kruppajo/work/GitHub/exam/avatare/Paula.png}
\end{minipage}
\begin{minipage}[t]{0.5\textwidth}
\hfill
\href{https://youtu.be/kHmfEmU6lrk}{\includegraphics[width = 2cm]{img/youtube}}
\end{minipage}
\fi
%% --------------------------------------------------------------------



\ifcollection
\paragraph{Visualisierung des Regressionskreuzes}
\fi

'Ich sehe die Sinnhaftigkeit nicht. Eine Regression in einem Kreuz darzustellen ergibt für mich keinen Sinn.', meckert Paula so laut, dass die Cafébesucher an den anderen Tischen zu ihr rüberschauen. Paula hatte ein Feldexperiment mit Erdbeeren duchgeführt. Dabei hatte sie dann Anzahl Läsionen auf den Blättern gemessen. Jetzt war es an ihr die Auswertung für ihre Gruppe zu erledigen und dafür sollte Paula sich an dem Regressionskreuz orientieren.

\vspace{1Ex}

Leider kennt sich Paula mit dem Kontext der linearen Regression überhaupt nicht aus. Deshalb braucht sie bei der Auswertung Ihre Hilfe!

\begin{enumerate}
  \item Formulieren Sie die wissenschaftliche Fragestellung! \textbf{(1 Punkt)}
\item Zeichen Sie die Zeile des Regressionskreuzes für den Endpunkt mit \underline{drei} Feldern! Beschriften Sie die Abbildung! \textbf{(4 Punkte)}
\item Ergänzen Sie die entsprechenden statistische Methoden zur Analyse in jedem Feld! \textbf{(2 Punkte)}
\item Formulieren Sie die Nullhypothese für die statistische Methode in jedem Feld! \textbf{(2 Punkte)}
\item Ergänzen Sie die entsprechenden Funktionen in \Rlogo zur Analyse in jedem Feld! \textbf{(2 Punkte)}
\item Welchen Effekt erhalten Sie in jedem Feld? Geben Sie ein Beispiel! \textbf{(2 Punkte)}
\end{enumerate} 
\clearpage
% -----------------------------------------------------------------------

\section{Aufgabe \hfill (12 Punkte)}

\textit{Geben Sie grundsätzlich Formeln und Rechenweg zur Lösung der Teilaufgaben mit an!} \\[1Ex]
 

 
%% --------------------------------------------------------------------
\ifcollection
\begin{flushright}
\tiny\vspace{-3Ex}
\textbf{\examinhaltstart}
\exammodulelanddaten $\;\bullet$
\exammodulebiostat
\vspace{-4Ex}
\end{flushright}
\begin{minipage}[t]{0.5\textwidth}
\includegraphics[width = 1.3cm]{/Users/kruppajo/work/GitHub/exam/avatare/Jessica.png}
\end{minipage}
\begin{minipage}[t]{0.5\textwidth}
\hfill
\href{https://youtu.be/C9skfFRTHhI}{\includegraphics[width = 2cm]{img/caution}}
\end{minipage}
\vspace{-3ex}
\fi
%% --------------------------------------------------------------------



\ifcollection
\paragraph{Interpretation der Ergebnisse eines gemischten Modells in \Rlogo}
\fi

Jessica mag die Berge sehr. Also hatte sie sich sofort freiwillig gemeldet als es hieß, dass die Hochschule jemanden braucht um in den Bergen an Drachen was zu untersuchen. Drachen an sich sind ja recht friedfertig, aber die Schreibfähigkeit zu testen, ging den Wesen dann doch gehörig auf den Zeiger. Ein Schrei zerschneidet die Ruhe der Berge. Jessica hatte zusätzlich die Sprunghöhe bestimmt und wollte wissen, ob es hier einen Zusammenhang gibt. Die Messungen wurden an drei Orten \textit{nearby}, \textit{midway} und \textit{far far away} pro Berg durchgeführt. Vielleicht hätte sie nicht so offensiv viel Schokobons vor den Augen der Drachen essen sollen.


\begin{knitrout}
\definecolor{shadecolor}{rgb}{0.969, 0.969, 0.969}\color{fgcolor}\begin{kframe}
\begin{verbatim}
## Linear mixed model fit by REML ['lmerMod']
## Formula: literary_score ~ jump_height + (1 | mountain/direction)
##    Data: dragon_small_tbl
## 
## REML criterion at convergence: 2250.7
## 
## Scaled residuals: 
##     Min      1Q  Median      3Q     Max 
## -2.6549 -0.6262 -0.0072  0.6467  2.8057 
## 
## Random effects:
##  Groups             Name        Variance Std.Dev.
##  direction:mountain (Intercept)  14.39    3.794  
##  mountain           (Intercept) 352.91   18.786  
##  Residual                       196.83   14.030  
## Number of obs: 273, groups:  direction:mountain, 21; mountain, 7
## 
## Fixed effects:
##             Estimate Std. Error t value
## (Intercept) 38.35795   25.21992   1.521
## jump_height  0.04956    0.11988   0.413
## 
## Correlation of Fixed Effects:
##             (Intr)
## jump_height -0.958
\end{verbatim}
\end{kframe}
\end{knitrout}

Leider kennt sich Jessica mit der Analyse eines gemischten Modells in \Rlogo überhaupt nicht aus. Deshalb braucht sie bei der Analyse Ihre Hilfe!

\begin{enumerate}
  \item Formulieren Sie die wissenschaftliche Fragestellung! \textbf{(1 Punkt)}
  \item Formulieren Sie das statistische Hypothesenpaar für den festen Effekt! \textbf{(1 Punkt)}
  \item Erklären Sie den Begriff \underline{fester Effekt} im Kontext des gemischten Modells! \textbf{(1 Punkt)}
  \item Erklären Sie den Begriff \underline{zufälliger Effekt} im Kontext des gemischten Modells! \textbf{(1 Punkt)}
  \item Skizzieren Sie das Versuchsdesign! \textit{Nutzen Sie eine reduzierte Auswahl an Faktorkombinationen!} \textbf{(1 Punkt)} 
  \item Visualisieren Sie die Ausgabe des linearen gemischten Modells! \textit{Nutzen Sie eine reduzierte Auswahl an Faktorkombinationen!} \textbf{(1 Punkt)} 
  \item Folgt der Messwert $y$ einer annährenden Normalverteilung? Begründen Sie Ihre Antwort! \textbf{(1 Punkte)}
\item Treffen Sie eine Aussage mit $T_{\alpha = 5\%} = 2.04$ zu Ihrer aufgestellten Nullhypothese! Begründen Sie Ihre Antwort \textbf{(2 Punkte)}
\item Interpretieren Sie den Effekt des festen Effekts im Kontext der wissenschaftlichen Fragestellung! \textbf{(1 Punkt)}
\item Wieviel Varianz erklärt der zufällige Haupteffekt? \textbf{(1 Punkt)}
\end{enumerate} 
\clearpage
% -----------------------------------------------------------------------
\part{Experimentelles Design}
% -----------------------------------------------------------------------

\section{Aufgabe \hfill (16 Punkte)}


 
%% --------------------------------------------------------------------
\ifcollection
\begin{flushright}
\tiny\vspace{-3Ex}
\textbf{\examinhaltstart}
\exammodulestatversuch $\;\bullet$
\exammodulebiostat
\vspace{-4Ex}
\end{flushright}
\begin{minipage}[t]{0.5\textwidth}
\includegraphics[width = 1.3cm]{/Users/kruppajo/work/GitHub/exam/avatare/Alex.png}\hspace{-4mm}\includegraphics[width = 1.3cm]{/Users/kruppajo/work/GitHub/exam/avatare/Mark.png}\hspace{-4mm}\includegraphics[width = 1.3cm]{/Users/kruppajo/work/GitHub/exam/avatare/Yuki.png}
\end{minipage}
\begin{minipage}[t]{0.5\textwidth}
\hfill
\href{https://youtu.be/wJqsNV1hOW8}{\includegraphics[width = 2cm]{img/caution}}
\end{minipage}
\fi
%% --------------------------------------------------------------------



\ifcollection
\paragraph{Einfache experimentelle Designs}
\fi

Alex und Mark sind bei Yuki um sich Hilfe für eine Versuchsplanung in \Rlogo zu holen. Im Hintergrund läuft viel zu laut London Grammar. Dabei geht es um den Zusammenhang zwischen der Behandlung Elterlinie ($ctrl$, $Standard$, $Yray$ und $Xray$) und dem Messwert Gewichtszuwachs in der 1LW in Zandern. Der Versuch soll in einem Stallversuch im Teuteburgerwald durchgeführt werden. Nach dem Dozenten ist der Messwert Gewichtszuwachs in der 1LW normalverteilt. Die beiden entschieden sich für ein faktorielles Versuchsdesign. Im ersten Schritt überlegt Yuki ein einfaches experimentelles Design zu probieren. Daher entscheiden sich alle drei für ein \textit{Latin square design}. Das sollte für den Anfang erstmal reichen. 'Und jetzt, was machen wir jetzt?', Mark schaut die anderen beiden mit großen Augen an. Die zucken mit der Schulter. Alle mampfen Reese's Peanut Butter Cups.\\

Leider kennen sich Yuki, Alex und Mark mit dem \textit{Latin square design} überhaupt nicht aus. Deshalb brauchen die Drei bei der Erstellung Ihre Hilfe!

\begin{enumerate}
  \setcounter{enumi}{0}
  \item Formulieren Sie die wissenschaftliche Fragestellung! \textbf{(1 Punkt)}
  \item Formulieren Sie das statistische Hypothesenpaar! \textbf{(1 Punkt)}
  \item Skizzieren Sie das faktorielle Versuchsdesign! \textbf{(3 Punkte)}
  \item Skizzieren Sie eine Datentabelle für das faktorielle Versuchsdesign in \Rlogo! \textbf{(2 Punkte)}
  \item Erstellen Sie das statistische Modell in der in \Rlogo üblichen Schreibweise für eine ANOVA! Skizzieren Sie die notwendige Funktionen in \Rlogo! \textbf{(3 Punkte)}
  \item Skizzieren Sie die weitere Datenanalyse hinsichtlich eines multiplen Gruppenvergleiches in \texttt{\{emmeans\}}! \textbf{(2 Punkte)}
  \item Skizzieren Sie eine mögliche Abbildung im Kontext der wissenschaftlichen Fragestellung! Beschriften Sie die Abbildung! \textbf{(2 Punkte)}
  \item Ergänzen Sie zu der Abbildung ein mögliches Ergebnis des multiplen Gruppenvergleichs! Begründen Sie Ihre Antwort! \textbf{(2 Punkte)}
\end{enumerate}


 
\clearpage
% -----------------------------------------------------------------------

\section{Aufgabe \hfill (20 Punkte)}


 
%% --------------------------------------------------------------------
\ifcollection
\begin{flushright}
\tiny\vspace{-3Ex}
\textbf{\examinhaltstart}
\exammodulebiostat
\vspace{-4Ex}
\end{flushright}
\begin{minipage}[t]{0.5\textwidth}
\includegraphics[width = 1.3cm]{/Users/kruppajo/work/GitHub/exam/avatare/Jonas.png}\hspace{-4mm}\includegraphics[width = 1.3cm]{/Users/kruppajo/work/GitHub/exam/avatare/Paula.png}\hspace{-4mm}\includegraphics[width = 1.3cm]{/Users/kruppajo/work/GitHub/exam/avatare/Tina.png}
\end{minipage}
\begin{minipage}[t]{0.5\textwidth}
\hfill
\href{https://youtu.be/wJqsNV1hOW8}{\includegraphics[width = 2cm]{img/caution}}
\end{minipage}
\fi
%% --------------------------------------------------------------------



\ifcollection
\paragraph{Fortgeschrittene experimentelle Designs}
\fi

Jonas und Tina sind bei Paula um sich Hilfe für eine Versuchsplanung in \Rlogo zu holen. Im Hintergrund läuft viel zu laut White Lies. Dabei geht es um den Zusammenhang zwischen der Behandlung Lichtstufen ($none$, $200lm$, $400lm$ und $600lm$) sowie Genotypen ($AA$, und $BB$) sowie zwei Blöcken und dem Messwert Chlorophyllgehalt (SPAD-502Plus) [SPAD] in Erbsen. Der Versuch soll in einem Freilandversuch im Oldenburger Land durchgeführt werden. Nach dem Dozenten ist der Messwert Chlorophyllgehalt (SPAD-502Plus) [SPAD] normalverteilt. Die beiden entschieden sich für ein faktorielles Versuchsdesign. Im ersten Schritt überlegt Paula ein komplexeres experimentelles Design zu probieren. Daher entscheiden sich alle drei für ein \textit{Strip plot design oder auch Streifenanlage} mit Berücksichtigung einer Interaktion. Das sollte für den anfang erstmal reichen. 'Und jetzt, was machen wir jetzt?', Tina schaut die anderen beiden mit großen Augen an. Die zucken mit der Schulter. Alle mampfen Smarties.\\

Leider kennen sich Paula, Jonas und Tina mit dem \textit{Strip plot design oder auch Streifenanlage} überhaupt nicht aus. Deshalb brauchen die Drei bei der Erstellung Ihre Hilfe!

\begin{enumerate}
  \setcounter{enumi}{0}
  \item Formulieren Sie die wissenschaftliche Fragestellung! \textbf{(1 Punkt)}
  \item Formulieren Sie die statistische Hypothesenpaare! \textbf{(2 Punkte)}
  \item Skizzieren Sie das faktorielle Versuchsdesign! \textbf{(3 Punkte)}
  \item Skizzieren Sie eine Datentabelle für das faktorielle Versuchsdesign in \Rlogo! \textbf{(2 Punkte)}
  \item Erstellen Sie das statistische Modell in der in \Rlogo üblichen Schreibweise für eine ANOVA! Skizzieren Sie die notwendige Funktionen in \Rlogo! \textbf{(4 Punkte)}
  \item Skizzieren Sie die weitere Datenanalyse hinsichtlich eines multiplen Gruppenvergleiches in \texttt{\{emmeans\}}! \textbf{(2 Punkte)}
  \item Skizzieren Sie eine mögliche Abbildung im Kontext der wissenschaftlichen Fragestellung! Beschriften Sie die Abbildung! \textbf{(3 Punkte)}
  \item Ergänzen Sie zu der Abbildung ein mögliches Ergebnis des multiplen Gruppenvergleichs! Welche Annahme hinsichtlich der Modellierung haben Sie getroffen? Begründen Sie Ihre Antwort! \textbf{(3 Punkte)}
\end{enumerate} 
\clearpage
% -----------------------------------------------------------------------
\part{Forschendes Lernen}
% -----------------------------------------------------------------------

\section{Aufgabe \hfill (20 Punkte)}

\textit{Geben Sie grundsätzlich Formeln und Rechenweg zur Lösung der Teilaufgaben mit an!} \\[1Ex]
 

 
%% --------------------------------------------------------------------
\ifcollection
\begin{flushright}
\tiny\vspace{-3Ex}
\textbf{\examinhaltstart}
\exammodulebiostat
\vspace{-4Ex}
\end{flushright}
\begin{minipage}[t]{0.5\textwidth}
\includegraphics[width = 1.3cm]{/Users/kruppajo/work/GitHub/exam/avatare/Yuki.png}
\end{minipage}
\begin{minipage}[t]{0.5\textwidth}
\hfill
\href{https://youtu.be/C9skfFRTHhI}{\includegraphics[width = 2cm]{img/caution}}
\end{minipage}
\fi
%% --------------------------------------------------------------------



\ifcollection

\begin{graybox}{}
\small
Die folgende Aufgabe basiert auf einer der zwei folgenden wissenschaftlichen Veröffentlichungen. Für die Prüfung wird die vertiefende Kenntnis der beiden Veröffentlichungen vorausgesetzt.

\begin{itemize}[noitemsep]
\item Sánchez, M., et al. (2022). Hoverfly pollination enhances yield and fruit quality in mango under protected cultivation. Scientia Horticulturae, 304, 111320. [\href{https://www.sciencedirect.com/science/article/pii/S0304423822004411}{Link}]
\item Wu, G., et al. (2004). Arginine nutrition in neonatal pigs. The Journal of Nutrition, 134(10), 2783S-2790S. [\href{https://www.sciencedirect.com/science/article/pii/S0022316623031279}{Link}]
\end{itemize}

In der Prüfung erhalten Sie \underline{keinen Auszug} der wissenschaftlichen Veröffentlichung! Die Veröffentlichungen werden als \underline{bekannt} in der Prüfung vorgesetzt. Sie haben sich vorab Notizen und Anmerkungen auf Ihrem Spickzettel gemacht.
\end{graybox}

\paragraph{Zerforschen einer wissenschaftlichen Veröffentlichung}
\fi

Vor dem Start der eigenen Arbeit möchte ihre Betreuerin, dass Yuki einmal die wissenschaftliche Veröffentlichung \textit{Wu, G., et al. (2004). Arginine nutrition in neonatal pigs} sinnvoll zusammenfasst. Dann würde die eigene Arbeit auch leichter von der Hand gehen und Yuki hätte dann schon eine Vorlage. 'Das ist jetzt aber umfangreicher als gedacht!', schnauft sie und runzelt die Stirn. Im Hintergrund spielt viel zu leise London Grammar. Das wird dann vermutlich heute Abend nichts mehr mit ihrem Hobby Orchideen. Das Minischwein schaut mitleidig.\\

Leider kennt sich Yuki mit dem Lesen einer wissenschaftlichen Veröffentlichung mit Fokus auf die Statistik überhaupt nicht aus. Deshalb braucht sie bei der Erstellung Ihre Hilfe! Glücklicherweise kennen Sie die wissenschaftliche Veröffentlichung schon im Detail und können sofort helfen.

\begin{enumerate}
  \setcounter{enumi}{0}
  \item Erläutern Sie die wissenschaftliche Fragestellung der wissenschaftlichen Veröffentlichung anhand des OCAR Prinzips nach Schimel (2012)\footnote{Schimel, J. (2012). Writing science: how to write papers that get cited and proposals that get funded. OUP USA.} \textbf{(4 Punkte)}
  \item Nennen Sie die untersuchten Endpunkte in der wissenschaftlichen Veröffentlichung! Wie lautet der primäre Endpunkt? \textbf{(2 Punkte)} 
\item Erstellen Sie das statistische Modell in der in \Rlogo üblichen Schreibweise! \textbf{(2 Punkte)}
  \item Nennen Sie eine Auswahl an bedeutenden statistischen Maßzahlen in der wissenschaftlichen Veröffentlichung! \textbf{(1 Punkt)}
  \item Interpretieren Sie die Hauptaussage der wissenschaftlichen Veröffentlichung hinsichtlich der Signifikanz für den primären Endpunkt! \textbf{(2 Punkte)}
  \item Interpretieren Sie die Hauptaussage der wissenschaftlichen Veröffentlichung hinsichtlich der Effektstärke für den primären Endpunkt! \textbf{(2 Punkte)}
  \item Diskutieren Sie die ökonomische Relevanz der Hauptaussage der wissenschaftlichen Veröffentlichung im Bezug auf Signifikanz und Effektstärke für den primären Endpunkt! \textbf{(1 Punkt)}
  \item Skizzieren Sie für den primären Endpunkt den sich ergebenden Datensatz in \Rlogo für eine ausgewählte Abbildung! \textbf{(2 Punkte)}
\item Skizzieren Sie einen möglichen Versuchsplan für den primären Endpunkt! \textbf{(2 Punkte)}
  \item Schätzen Sie die benötigte Fallzahl für ein zukünftiges Experiment anhand der Ergebnisse in der wissenschaftlichen Veröffentlichung für den primären Endpunkt! \textbf{(2 Punkte)}
\end{enumerate} 
\clearpage
% -----------------------------------------------------------------------

\section{Aufgabe \hfill (20 Punkte)}

\textit{Geben Sie grundsätzlich Formeln und Rechenweg zur Lösung der Teilaufgaben mit an!} \\[1Ex]
 

 
%% --------------------------------------------------------------------
\ifcollection
\begin{flushright}
\tiny\vspace{-3Ex}
\textbf{\examinhaltstart}
\exammodulebiostat
\vspace{-4Ex}
\end{flushright}
\begin{minipage}[t]{0.5\textwidth}
\includegraphics[width = 1.3cm]{/Users/kruppajo/work/GitHub/exam/avatare/Tina.png}
\end{minipage}
\begin{minipage}[t]{0.5\textwidth}
\hfill
\href{https://youtu.be/0upJd61FI-c}{\includegraphics[width = 2cm]{img/youtube}}
\end{minipage}
\fi
%% --------------------------------------------------------------------



\ifcollection
\paragraph{Zerforschen eines wissenschaftlichen Datensatzes}
\fi

Vor dem Start der eigenen Arbeit möchte ihr Betreuer, dass Tina einmal die wissenschaftlichen Daten \textit{in der Tabelle 1} sinnvoll zusammenfasst. Insgesamt wurden $n$ Beobachtungen erhoben. Dann würde die eigene Arbeit auch leichter von der Hand gehen und Tina hätte dann schon eine Vorlage um die eigenen erhobenen Daten in eine Tabelle eintragen zu können. 'Das ist jetzt aber umfangreicher als gedacht!', schnauft sie und runzelt die Stirn als sie in ihren Laptop starrt. Dabei isst sie noch ein paar Katjes. Das wird dann vermutlich heute Abend nichts mehr mit Indiana Jones.

\begin{table}[h]
\centering
\Large
  \begin{tabular}{ccccc}
  \toprule
   $\boldsymbol{f_1}$  & $\boldsymbol{f_2}$ & $\boldsymbol{x_1}$ & $\boldsymbol{y_1}$ & $\boldsymbol{y_2}$ \\[2pt]
     <\phantom{xxxxx}>  & <\phantom{xxxxx}> & <\phantom{xxxxx}> & <\phantom{xxxxx}> & <\phantom{xxxxx}> \\[2pt] 
  \midrule
  1  & 1 & 2.3 & 10.1 & 0 \\  
  1  & 1 & 4.1 & 13.1 & 0 \\ 
  1  & 1 & 5.7 & 16.5 & 1 \\ 
  1  & 1 & 3.4 & 14.6 & 0 \\
  1  & 2 & 2.8 & 12.1 & 1 \\
  1  & 2 & 6.1 & 13.4 & 1 \\
  $\vdots$  & $\vdots$ & $\vdots$ & $\vdots$ & $\vdots$\\
  5 & 2  & 1.9 & 9.6 &  0\\
  \bottomrule
  \end{tabular}
\end{table}

Tina füllt sich mit der Analyse der Daten in der Tabelle 1 überfordert. Deshalb braucht sie bei der Auswertung Ihre Hilfe! Glücklicherweise kennen Sie den wissenschaftlichen Datensatz aus Ihren eigenen Analysen schon im Detail und können sofort helfen.

\begin{graybox}{}
\textit{Beantworten Sie die folgenden Fragen anhand eines selbst gewählten Beispiels!}
\end{graybox}

\paragraph{Allgemeiner Aufgabenteil} 

\begin{enumerate}
  \setcounter{enumi}{0}
  \item Ergänzen Sie die Eigenschaften der Spalten in der Form eines \texttt{tibbles}! \textbf{(2 Punkte)}
  \item Skizzieren Sie zwei übergeordnete Analysebereiche der Statistik! \textit{Nutzen Sie hierfür die Variablennamen der obigen Datentabelle.} Beschriften Sie die Abbildungen! \textbf{(4 Punkte)}
  \item Formulieren Sie zwei mögliche wissenschaftliche Fragestellungen in Form einer PowerPoint Folie aus der obigen Datentabelle! \textbf{(2 Punkte)}
\end{enumerate}

\paragraph{Spezieller Aufgabenteil für die Variablen $\boldsymbol{x_1}$ und $\boldsymbol{y_1}$}

\begin{enumerate}
  \setcounter{enumi}{3}
  \item In welchen der übergeordneten Analysebereiche der Statistik gehört die Auswertung Ihres Endpunktes? Begründen Sie Ihre Antwort! \textbf{(2 Punkte)}
  \item Skizzieren Sie eine beispielhafte Abbildung für Ihren Endpunkt im Kontext der wissenschaftlichen Fragestellung! \textbf{(2 Punkte)}
  \item Erstellen Sie das statistische Modell in der in \Rlogo üblichen Schreibweise! \textbf{(1 Punkt)}
  \item Skizzieren Sie die Datenanalyse für Ihren Endpunkt! \textbf{(4 Punkte)}
  \item Auf welche Eigenschaften der Daten müssen Sie für Ihre statistische Analyse im Besonderen achten? \textbf{(2 Punkte)}
  \item Welche statistische Maßzahl können Sie aus Ihrer Datenanalyse berichten? \textbf{(1 Punkt)}
\end{enumerate}

 
\clearpage
% -----------------------------------------------------------------------
\part{Mathematik}
% -----------------------------------------------------------------------  

\section{Aufgabe \hfill (10 Punkte)}

\textit{Geben Sie grundsätzlich Formeln und Rechenweg zur Lösung der Teilaufgaben mit an!} \\[1Ex]
 

 
%% --------------------------------------------------------------------
\ifcollection
\begin{flushright}
\tiny\vspace{-3Ex}
\textbf{\examinhaltstart}
\exammodulemathstat
\vspace{-4Ex}
\end{flushright}
\begin{minipage}[t]{0.5\textwidth}
\includegraphics[width = 1.3cm]{/Users/kruppajo/work/GitHub/exam/avatare/Jessica.png}\hspace{-4mm}\includegraphics[width = 1.3cm]{/Users/kruppajo/work/GitHub/exam/avatare/Jonas.png}\hspace{-4mm}\includegraphics[width = 1.3cm]{/Users/kruppajo/work/GitHub/exam/avatare/Mark.png}\hspace{-4mm}\includegraphics[width = 1.3cm]{/Users/kruppajo/work/GitHub/exam/avatare/Nilufar.png}
\end{minipage}
\begin{minipage}[t]{0.5\textwidth}
\hfill
\href{https://youtu.be/Fu8kN0Uj13Y}{\includegraphics[width = 2cm]{img/youtube}}
\end{minipage}
{\tiny\textbf{Stichworte:} The Great Horse-Manure Crisis of 1894\footnote{Weitere Informationen: \href{https://fee.org/articles/the-great-horse-manure-crisis-of-1894/}{The Great Horse-Manure Crisis of 1894} und \href{https://www.gregorystrachta.com/18.html}{The Great Horse Manure Crisis of 1894 was never mentioned before 2004}} $\bullet$ Napoleon $\bullet$ Ivenacker Eichen}
\fi
%% --------------------------------------------------------------------





\ifcollection
\paragraph{Herodot – der Schimmel aus Ivenack}
\fi

Die Lerngruppe \textit{Die Pantoffeltieren} bestehend aus Jonas, Nilufar, Mark und Jessica waren auf Exkursion in Brandenburg und haben dort Folgendes erarbeitet. Während der Besetzung Mecklenburgs durch die Franzosen kamen Napoleon die Geschichten des berühmten Apfelschimmels Herodot aus Ivenack zu Gehör. Herodot lief zwar niemals Rennen, war aber eines der berühmtesten Pferde dieser Zeit. Napoleon selbst gab den Auftrag, diesen Schimmel durch die Armee nach Frankreich zu bringen. Der Legende nach sollen Arbeiter den Schimmel im hohlen Stamm einer 1000-jährigen Eiche aus Ivenack vor den Franzosen versteckt haben. Doch Herodot verriet sein Versteck durch lautes Wiehern, woraufhin die französische Armee den Schimmel beschlagnahmte und nach Frankreich führte\footnote{Die Quelle der Inspiration  für die Aufgabe war eine Fahrt an die Ostsee und folgender Artikel:
  \href{https://www.wald-mv.de/landingpage/ivenacker-eichen/}{Entdecke das erste Nationale Naturmonument Deutschlands - Ivenacker Eichen und Hutewald}}. Jetzt wollen die vier herausfinden: \textit{"Konnten die Ivenacker den Apfelschimmel Herodot vor dem Zugriff von Napoleon in der 1000-jährigen Eiche verstecken?"} 



\vspace{1Ex}

Helfen Sie der Lerngruppe \textit{Die Pantoffeltieren} bei der Beantwortung der Forschungsfrage! Gehen Sie von einem radialen Wachstum der 1000-jährigen Eiche von $3mm$ pro Jahr aus. Es ist bekannt, dass die Eiche im Jahr 2022 einen Umfang von $17m$ in Brusthöhe hatte.

\begin{enumerate}
\item Wie groß war der Durchmesser in $m$ der Eiche im Jahr $1820$ als Herodot in der Eiche versteckt werden sollte? \textbf{(2 Punkte)}
\item Erstellen Sie die lineare Gradengleichung für das radiale Wachstum! \textbf{(1 Punkt)}
\item Skizzieren Sie in einer Abbildung einen linearen Zusammenhang und einen exponentiellen Zusammenhang für das Wachstum der 1000-jährigen Eiche! \textbf{(1 Punkt)}
\end{enumerate}
 
Herodot hatte eine Schulterhöhe von $190$cm, eine Breite von $85$cm sowie eine Länge von  $240$cm.

\begin{enumerate}
  \setcounter{enumi}{2}
\item Berechnen Sie das effektive Volumen von Herodot in $m^3$, welches Herodot in der 1000-jährigen Eiche einnehmen würde! \textbf{(2 Punkte)}
\end{enumerate}

Es wurde berichtet, dass sich Herodot in der 1000-jährigen Eiche $mit blutiger Nase$ um die eigene Achse drehen konnte.

\begin{enumerate}
  \setcounter{enumi}{3}
\item Berechnen Sie die Dicke der Eichenwand in $cm$! Verdeutlichen Sie Ihre Berechnungen an einer aussagekräftigen Skizze für Pferd und Eiche! \textbf{(2 Punkte)} 
\item Unter einer Dicke der Eichenwand von $15cm$ bricht die Eiche zusammen. Beantworten Sie die Forschungsfrage! Begründen Sie Ihre Antwort! \textbf{(2 Punkte)} 
\end{enumerate}


 
\clearpage
% ----------------------------------------------------------------------- 

\section{Aufgabe \hfill (10 Punkte)}

\textit{Geben Sie grundsätzlich Formeln und Rechenweg zur Lösung der Teilaufgaben mit an!} \\[1Ex]
 

 
%% --------------------------------------------------------------------
\ifcollection
\begin{flushright}
\tiny\vspace{-3Ex}
\textbf{\examinhaltstart}
\exammodulemathstat
\vspace{-4Ex}
\end{flushright}
\begin{minipage}[t]{0.5\textwidth}
\includegraphics[width = 1.3cm]{/Users/kruppajo/work/GitHub/exam/avatare/Jonas.png}\hspace{-4mm}\includegraphics[width = 1.3cm]{/Users/kruppajo/work/GitHub/exam/avatare/Paula.png}\hspace{-4mm}\includegraphics[width = 1.3cm]{/Users/kruppajo/work/GitHub/exam/avatare/Steffen.png}\hspace{-4mm}\includegraphics[width = 1.3cm]{/Users/kruppajo/work/GitHub/exam/avatare/Yuki.png}
\end{minipage}
\begin{minipage}[t]{0.5\textwidth}
\hfill
\href{https://youtu.be/57B-yYoFSk0}{\includegraphics[width = 2cm]{img/youtube}}
\end{minipage}
\fi
%% --------------------------------------------------------------------



\ifcollection
\paragraph{Von Töpfen auf Tischen}
\fi



Die Projektgruppe \textit{J} bestehend aus Paula, Yuki, Jonas und Steffen hat sich zusammengefunden um den ersten Versuch zu planen. In einem Experiment wollen sie die Wuchshöhe von 252 Sonnenblumen bestimmen. Bevor die Vier überhaupt mit dem Experiment beginnen können, gibt es aber ein paar Abschätzungen über die Kosten und den Aufwand zu treffen. Zum einen müssen sie die Sonnenblumen einpflanzen und müssen dafür Substrat bestellen. Zum anderen muss die Projektgruppe die Sonnenblumen auch bewegen und in ein Gewächshaus auf rechteckigen Tischen platzieren. Die schmale Tischseite fast ohne Randpflanzen 7 Pflanzen. Die Töpfe für die Keimung haben
einen Durchmesser von 10cm und eine Höhe von 8cm. Der Kubikmeterpreis für Torf liegt bei 260 EUR. 

\vspace{1Ex}

Helfen Sie der Projektgruppe \textit{J} bei der Planung des Versuches! 

\begin{enumerate}
\item Skizzieren Sie den Versuchsplan auf \textit{drei} Tischen im Gewächshaus! \textit{Skizzieren Sie einen Tisch detailierter und die restlichen Tische schematisch!} \textbf{(2 Punkte)}
\item Berechnen Sie die benötigte Anzahl an Pflanztöpfen, wenn Sie Randpflanzen mit berücksichtigen wollen! \textbf{(1 Punkt)}
\item Berechnen Sie die benötigte \textbf{(a)} \underline{Pflanztopf}fläche in $m^2$ sowie die \textbf{(b)} \underline{Tisch}fläche in $m^2$ \textit{für einen Tisch} gegeben der Anzahl an Pflanztöpfen inklusive Randpflanzen am Anfang der Keimungsphase! \textbf{(4 Punkte)}
\item Berechnen Sie die benötigte Menge an Torf in Liter $l$, die Sie für das Befüllen der Pflanztöpfe benötigen! Gehen Sie von \textit{einem Zylinder} für die Pflanztöpfe aus!  \textbf{(2 Punkte)}
\item Berechnen Sie die Kosten in EUR für Ihre Torfbestellung! \textbf{(1 Punkt)}
\end{enumerate}




 
\clearpage
% ----------------------------------------------------------------------- 

\section{Aufgabe \hfill (10 Punkte)}

\textit{Geben Sie grundsätzlich Formeln und Rechenweg zur Lösung der Teilaufgaben mit an!} \\[1Ex]
 

 
%% --------------------------------------------------------------------
\ifcollection
\begin{flushright}
\tiny\vspace{-3Ex}
\textbf{\examinhaltstart}
\exammodulemathstat
\vspace{-4Ex}
\end{flushright}
\begin{minipage}[t]{0.5\textwidth}
\includegraphics[width = 1.3cm]{/Users/kruppajo/work/GitHub/exam/avatare/Alex.png}\hspace{-4mm}\includegraphics[width = 1.3cm]{/Users/kruppajo/work/GitHub/exam/avatare/Steffen.png}\hspace{-4mm}\includegraphics[width = 1.3cm]{/Users/kruppajo/work/GitHub/exam/avatare/Tina.png}\hspace{-4mm}\includegraphics[width = 1.3cm]{/Users/kruppajo/work/GitHub/exam/avatare/Yuki.png}
\end{minipage}
\begin{minipage}[t]{0.5\textwidth}
\hfill
\href{https://youtu.be/aBxLkdF-c4M}{\includegraphics[width = 2cm]{img/youtube}}
\end{minipage}
\fi
%% --------------------------------------------------------------------




\ifcollection
\paragraph{Solar- \& Biogasanlagen}
\fi



Alex bringt ein neues, tolles Projekt mit in die Lerngruppe \textit{Die Blattläuse} bestehend aus ihm, Steffen, Yuki sowie Tina. Um die Energiekosten seines Betriebes zu senken, will er eine Solaranlage auf den Rinderstall montieren lassen. Dafür hat er seinen Stall ausgemessen und findet folgende Maße wieder. Die vordere Seite des Rinderstall hat eine Höhe $h_v$ von $5.5m$. Die hintere Seite des Rinderstall hat eine Höhe $h_b$ von $9m$. Der Rinderstall hat eine Tiefe $t$ von $15m$ und eine Breite $b$ von $40m$. 'Sag mal Alex, ist das eine Matheaufgabe oder rechnen wir hier gerade für dich kostenlos als menschliche Computer Sachen für deinen Betrieb?', fragt Yuki mit erhobenenen Augenbrauen. Tina und Steffen nicken zustimmend.

\vspace{1Ex}

Wenn die Lerngruppe nicht will, dann müssen Sie bei der Planung helfen!

\begin{enumerate}
\item Skizzieren Sie den Rinderstall auf dem die Solaranlage montiert werden soll! Ergänzen Sie die Angaben für die Höhen $h_v$, $h_b$, die Tiefe $t$ und die Breite $b$ des Stalls!  \textbf{(2 Punkte)}
\item Berechnen Sie die Fläche der schrägen, neuen Solaranlage auf dem Rinderstall! \textbf{(3 Punkte)}
\end{enumerate}

Ebenfalls plant Alex eine neue Biogasanlage für seinen Betrieb. Der neue Methantank hat einen Radius $r$ von $1m$. Leider gibt es ein paar bauliche Beschränkungen auf dem Grundstück. Das Fundament des zylindrischen Methantanks kann nur ein Gewicht von maximal $10t$ aushalten bevor der Tank wegbricht. Alex rechnen eine Sicherheitstoleranz von $10\%$ ein beinhaltend das Gewicht des Methantanks. In flüssiger Form hat Methan bei $-80^\circ\text{C}$ eine Dichte von $220kg/m^3$. Bei $-100^\circ\text{C}$ hat Methan eine Dichte von $270kg/m^3$. Alex betreibt seine Anlage bei $-90^\circ\text{C}$.

\begin{enumerate}
  \setcounter{enumi}{2}
\item Extrapolieren Sie die effektive Dichte des Methans in dem Methantank! Welche Annahme haben Sie getroffen? \textbf{(1 Punkt)}
\item Berechnen Sie wie viel Kubikmeter $m^3$ in den Methantank gefüllen werden können, bevor das Fundament nachgibt! \textbf{(2 Punkte)}
\item Berechnen Sie die maximale Höhe $h_{max}$ in $m$ für den gefüllten Methantank mit dem Radius $r$, bevor das Fundament wegbricht! \textbf{(2 Punkte)}
\end{enumerate}

 
\clearpage
% -----------------------------------------------------------------------

\section{Aufgabe \hfill (10 Punkte)}

\textit{Geben Sie grundsätzlich Formeln und Rechenweg zur Lösung der Teilaufgaben mit an!} \\[1Ex]
 

 
%% --------------------------------------------------------------------
\ifcollection
\begin{flushright}
\tiny\vspace{-3Ex}
\textbf{\examinhaltstart}
\exammodulemathstat
\vspace{-4Ex}
\end{flushright}
\begin{minipage}[t]{0.5\textwidth}
\includegraphics[width = 1.3cm]{/Users/kruppajo/work/GitHub/exam/avatare/Paula.png}
\end{minipage}
\begin{minipage}[t]{0.5\textwidth}
\hfill
\href{https://youtu.be/https://youtu.be/k2G52hMIfqk}{\includegraphics[width = 2cm]{img/youtube}}
\end{minipage}
{\tiny\textbf{Stichworte:} Riesenfaultier $\bullet$ Evolution der Avocado $\bullet$ Bluetooth $ \bullet$ Blauzahn $\bullet$ Colonia Dignidad $\bullet$ ODESSA $\bullet$ Rattenlinie $\bullet$ Adolf Eichmann}
\fi
%% --------------------------------------------------------------------



\ifcollection
\paragraph{Aligatorenbirnen und Blaubeeren}
\fi



"'Sind Sie ein Riesenfautier oder warum kaufen Sie so viele Aligatorenbirnen?"', spricht es hinter Ihnen. Irritiert drehen Sie sich um und blicken in das puderrote Gesicht von Paula. "'Wieso?"', entfährt es Ihnen und Sie bereuen sogleich die Frage. Sofort werden Sie zu einem Whiteboard voller roter Schnüre geschliffen und müssen folgenden mathematischen untermauerten Argumenten im Lidl über sich ergehen lassen. Da kommen Sie nicht mehr raus, also können Sie auch gleich mitmachen. Das Problem liegt in Chile\footnote{Die Quelle der Inspiration für die Aufgabe waren folgende Reportagen: \href{https://www.amnesty.ch/de/ueber-amnesty/publikationen/magazin-amnesty/2021-3/bis-zum-letzten-tropfen}{"`Bis zum letzten Tropfen"' in AMNESTY – Magazin der Menschenrechte vom August 2021} und \href{https://www.welthungerhilfe.de/welternaehrung/rubriken/klima-ressourcen/wassernot-in-chile-eine-folge-der-privatisierung}{"`Wasserknappheit in Chile: Eine Folge der Privatisierung?"' in Die Welternährung dem Fachjournal der Welthungerhilfe vom April 2022.}}. Tja, die Deutschen und Südamerika.\\

Zuerst werden Ihre Fähigkeiten getestet, der Mathematik folgen zu können. Oder berechnen Sie gerade den Einkauf von Paula?\\

\begin{enumerate}
\item Wenn 4 Blaubeerschalen 7.96 Euro kosten,  wie viel kosten 5 Schalen? \textbf{(2 Punkte)}
\item Wenn Sie die 5 Blaubeerschalen gekauft haben, wie viele Aligatorbirnen zu je 0.99 EUR können Sie sich dann noch für 50 EUR leisten? \textbf{(1 Punkt)}
\end{enumerate}

Das Whiteboard beinhaltet folgende Liste mit Informationen zum Wasserverbrauch bei der Produktion von Gemüse aus Chile. Seltsam, was man so alles in einem Lidl über Gemüse erfährt.
  
\begin{itemize}[noitemsep]
\item Ein Kilo Strauchtomaten benötigt 190l Wasser. Eine Strauchtomate wiegt 90 - 125g.
\item Ein Kilo Salat benötigt 130l Wasser. Ein Salatkopf wiegt 300 - 520g.
\item Ein Kilo Avocado benötigt 1100l Wasser. Eine Avocado wiegt 150 - 410g.
\item Ein Kilo Blaubeeren benötigt 820l Wasser. Eine Blaubeere wiegt 3.2 - 3.5g.
\end{itemize}

\begin{enumerate}
  \setcounter{enumi}{2}
\item Berechnen Sie den Wasserverbrauch für die Produktion für jeweils eine Strauchtomate, einem Salat, einer Avocado und einer Blaubeeren. Stellen Sie das Ergebnis als Tabelle dar! \textbf{(3 Punkte)}
\end{enumerate}

Chile exportiert im großem Ausmaß Blaubeeren und Avocados. In dem Exportjahr 2024 blieben die Erträge von Blaubeeren mit \ensuremath{7.9\times 10^{4}}t in dem prognostizierten Rahmen. Die Menge \underline{reduzierte} sich um 6.8\%. Die Exporte für Avocados \underline{stiegen} in dem gleichen Zeitraum um 18.1\% auf \ensuremath{1.9\times 10^{5}}t.

\begin{enumerate}
  \setcounter{enumi}{3}
\item Wie viele Hektoliter Wasser hat Chile in dem Exportjahr 2023 exportiert? \textbf{(2 Punkte)}
\end{enumerate}

Chile ist eines der wenigen Länder der Welt, die ihr Wasser komplett privatisiert haben. Derzeit sind nur ein Prozent des Wassers des Landes für den häuslichen Verbrauch vorgesehen. In den Dörfern der Anbauregionen versorgen Tankwagen die Bevölkerung jede Woche mit Wasser, es gibt etwa 50 Liter Wasser pro Kopf für den täglichen Bedarf. In \textit{Deutschland} liegt der Verbrauch bei 8 - 17 Liter pro Sp{"u}lmaschinenlauf und 35 - 115 Liter pro Waschgang einer Waschmaschine.

\begin{enumerate}
  \setcounter{enumi}{4}
\item Mit der rationierten Wassermenge aus Chiles Anbaugebieten können Sie in \textit{Deutschland} wie oft Ihren Bedarf stillen? \textbf{(1 Punkt)}
\end{enumerate}

Das alles hätten Sie nicht von Paula erwartet. Ganz schön viele Informationen wurden da zusammengetragen.

\begin{enumerate}
  \setcounter{enumi}{5}  
  \item Nennen Sie eine \textit{Daten}quelle im Internet, wo Sie mehr Informationen zu landwirtschaftlichen Daten oder klimatischen, wirtschaftlichen und gesellschaftlichen Daten erhalten! \textbf{(1 Punkt)}
\end{enumerate}



 
\clearpage
% ----------------------------------------------------------------------- 

\section{Aufgabe \hfill (10 Punkte)}

\textit{Geben Sie grundsätzlich Formeln und Rechenweg zur Lösung der Teilaufgaben mit an!} \\[1Ex]
 

 
%% --------------------------------------------------------------------
\ifcollection
\begin{flushright}
\tiny\vspace{-3Ex}
\textbf{\examinhaltstart}
\exammodulemathstat
\vspace{-4Ex}
\end{flushright}
\begin{minipage}[t]{0.5\textwidth}
\includegraphics[width = 1.3cm]{/Users/kruppajo/work/GitHub/exam/avatare/Alex.png}
\end{minipage}
\begin{minipage}[t]{0.5\textwidth}
\hfill
\href{https://youtu.be/WZSxntiNF8s}{\includegraphics[width = 2cm]{img/youtube}}
\end{minipage}
\fi
%% --------------------------------------------------------------------

%% --------------------------------------------------------------------
{\tiny\textbf{Stichworte:} Kardaschow-Skala $\bullet$ Dyson-Sphäre $\bullet$ Hohlerde $\bullet$ Entropie $\bullet$ Proton $r_P = 1.7 \times 10e-15$ $\bullet$ Wasserstoff $r_H = 5.3\times 10e-11$}
%% --------------------------------------------------------------------

\ifcollection
\paragraph{Die Dampfnudelerde}
\fi



"'Was für einen Unsinn!"', rufen Sie. Jetzt haben Sie auf Empfehlung von von Alex kostbaren Schlaf prokrastiniert um einem Ernährungswissenschaftler auf YouTube über die Erde als Dampfnudel zu lauschen. Irgendwie passt es dann doch mit der Analogie. Übermüdet müssen Sie darüber nachdenken, warum vor 69 Millionen Jahren die Dinosaurier - so groß sie auch waren - nicht von der Schwerkraft zu Boden gerissen wurden. Hat der Dampfplauderer etwa recht und war die Schwerkraft vor Millionen von Jahren eine andere?  Sind deshalb alle Lebewesen auf der Erde \textit{heutzutage} so viel kleiner, weil die Schwerkraft größer ist als damals? War die Erde kleiner und hatte weniger Masse? Oder ist es nur ein Rechenfehler wie bei der Theorie der Hohlerde von Edmond Halley aus dem 17.–18. Jahrhundert? Müde reiben Sie sich die Augen. So wird es nichts mehr mit dem Schlafen, dann können Sie auch mal etwas rechnen\footnote{Die Quelle der Inspiration
  für die Aufgabe war folgender Artikel:
  \href{https://hpd.de/artikel/erde-dampfnudel-22236}{"Skeptische Anmerkungen --- Die Erde als Dampfnudel" in Der Humanistische Pressedienst}}.  \\

Betrachten wir die Schwerkraft oder Gewichtskraft, die auf Lebewesen damals und heute gewirkt haben soll. Nehmen Sie für die Fallbeschleunigung $g$ der Erde \textit{heutzutage} einen Wert von 9.81m/s$^2$ an. Im Weiteren hat die Erde einen ungefähren Durchmesser von \ensuremath{1.2742\times 10^{4}}km und eine mittlere Dichte $\rho$ von 5.86g/cm$^3$. Das Gewicht von einem heute lebenden asiatischen Elefanten liegt bei 3t bis 5t und das Gewicht von einem Triceratops bei 6t bis 12t.

\begin{enumerate}
\item Welchen Durchmesser müsste die Erde vor 69 Millionen Jahren gehabt haben, wenn Dinosaurier und Elefanten die gleiche Gewichtskraft $\overrightarrow{F_G}$ damals und heute erfahren hätten? \textit{Beantworten Sie die Frage anhand der folgenden Teilaufgaben!}
\begin{enumerate}
\item Berechnen Sie die Fallbeschleunigung von vor 69 Millionen Jahren unter der obigen Annahme gleich wirkender Gewichtskraft $\overrightarrow{F_G}$ auf Elefant und Dinosaurier! \textbf{(1 Punkt)}
\item Berechnen Sie Masse der heutigen Erde! \textbf{(2 Punkte)}
\item Schließen Sie über die Masse auf den Durchmesser der Erde vor 69 Millionen Jahren! \textbf{(2 Punkte)}
\end{enumerate}
\item Beantworten Sie die Eingangsfrage mit 1-2 Antwortsätzen! \textbf{(1 Punkt)}
\end{enumerate}

Die Distanz zwischen Sonne und Erde entspricht 1.05 astronomische Einheiten ($AE$). Die Einheit 1 AE wird mit \ensuremath{1.55\times 10^{8}}km angegeben. Der \textit{massebehaftete} Sonnenwind besteht aus 84\% Wasserstoffkernen mit einer molaren Masse von 1.01g/mol, 11\% Heliumkernen mit 3.92g/mol sowie 5\% weiteren Atomkernen mit  145.31g/mol. Die Teilchendichte bei Eintritt in die Erdatmosphäre liegt zwischen 0.4 bis 100 Teilchen cm$^{-3}$ pro Sekunde mit einer mittleren Teilchendichte von 8cm$^{-3}$ pro Sekunde. \\

\textit{Lösen Sie den folgenden Aufgabenteil mit einer aussagekräftigen Skizze!}

\begin{enumerate}
  \setcounter{enumi}{2}
\item Berechnen Sie die Anzahl an massebehafteten Teilchen des Sonnenwindes, die die gesamte Erde pro Sekunde treffen! \textbf{(2 Punkte)}
%\item Berechnen Sie die Anzahl an massebehafteten Teilchen des Sonnenwindes, die die Sonne pro Sekunde in alle Richtungen aussendet! \textbf{(2 Punkte)}
\item Berechnen Sie die Masse, die die Erde pro Jahr durch die \textit{massebehafteten} Teilchen des Sonnenwind zunimmt! \textbf{(2 Punkte)}
\end{enumerate}

%\blfootnote{\tiny\textbf{Stichworte:} Kardaschow-Skala $\bullet$ Dyson-Sphäre $\bullet$ Hohlerde $\bullet$ Entropie}

% https://de.wikipedia.org/wiki/Kardaschow-Skala

 
\clearpage
% ----------------------------------------------------------------------- 

\section{Aufgabe \hfill (10 Punkte)}

\textit{Geben Sie grundsätzlich Formeln und Rechenweg zur Lösung der Teilaufgaben mit an!} \\[1Ex]
 

 
%% --------------------------------------------------------------------
\ifcollection
\begin{flushright}
\tiny\vspace{-3Ex}
\textbf{\examinhaltstart}
\exammodulemathstat
\vspace{-4Ex}
\end{flushright}
\begin{minipage}[t]{0.5\textwidth}
\includegraphics[width = 1.3cm]{/Users/kruppajo/work/GitHub/exam/avatare/Jonas.png}\hspace{-4mm}\includegraphics[width = 1.3cm]{/Users/kruppajo/work/GitHub/exam/avatare/Mark.png}\hspace{-4mm}\includegraphics[width = 1.3cm]{/Users/kruppajo/work/GitHub/exam/avatare/Tina.png}\hspace{-4mm}\includegraphics[width = 1.3cm]{/Users/kruppajo/work/GitHub/exam/avatare/Yuki.png}
\end{minipage}
\begin{minipage}[t]{0.5\textwidth}
\hfill
\href{https://youtu.be/n451XnhtSh4}{\includegraphics[width = 2cm]{img/youtube}}
\end{minipage}
\fi
%% --------------------------------------------------------------------



\ifcollection
\paragraph{'Entschuldigung, ist das Ihre Feder in meinem Auge?'}
\fi



Man hört schon häufig vieles Geschnatter von höflichen Puten in Mastställen. Enge ist natürlich etwas ungünstig, den dann kommt es zu Picken und Kannibalismus. Denn wenn der Nachbar nervt, dann muss zu Maßnahmen gegriffen werden. Kennt jeder aus einer mittelmäßigen Wohngemeinschaft. Das wollen Yuki, Jonas, Tina und Mark aber als vorsorgliche Puten-Halter:innen nicht\footnote{Die Quelle der Inspiration für die Aufgabe war der folgende wissenschaftliche Artikel: \href{https://www.efsa.europa.eu/en/efsajournal/pub/7788}{EFSA Panel on Animal Health and Welfare, et al. (2023) Welfare of broilers on farm. EFSA Journal 21.2}.}. Gemeinsam sind die Vier in einer Projektgruppe gelandet. Betrachten wir also gemeinsam einmal das Platzangebot (eng. \textit{space allowance}, abk. \textit{SA}) der Puten für vier Tätigkeiten und versuchen die notwendige Fläche zu optimieren. Wie immer gibt es dafür eine mathematische Formel:

\begin{center}
  \begin{tabular}{cc}
    $SA = \sum^n_{i = 1} (A_i \times PB_i)$ & $A_i = \pi \times (r_i + R_i)^2$\\
  \end{tabular}
\end{center}

\vspace{-2Ex}

mit

\begin{itemize}[noitemsep]
\item $SA$ dem benötigten Platzangebot aller aufsummierten Verhalten $i$.
\item $A_i$ dem benötigten Platz für ein Verhalten $i$. 
\item $PB_i$ dem Anteil des Auftretens eines Verhaltens $i$.
\item $r_i$ dem Radius Pute plus dem benötigten Radius für das Verhalten $i$.
\item $R_i$ dem notwendigen Abstand zu den Nachbarn für das Verhalten $i$.    
\item $i$ dem Verhalten: (1) dustbathing, (2) wing/leg stretching, (3)
  drinking/eating und (4) preening.
\end{itemize}

In der folgenden Tabelle 1 sind die Werte für $r_i$, $R_i$ und $PB_i$ für ein spezifisches Verhalten $i$ aus drei wissenschaftlichen Veröffentlichungen dargestellt.

\vspace{-1Ex}

{\small
\begin{knitrout}
\definecolor{shadecolor}{rgb}{0.969, 0.969, 0.969}\color{fgcolor}\begin{table}[!h]
\centering
\begin{tabular}{llll}
\toprule
  & Aldridge et al. (2021) & Baxter et al. (2022) & Jabcobs et al. (2019)\\
\midrule
dustbathing & 34cm; 37cm; 1.2\% & 43cm; 39cm; 8.1\% & 14cm; 35cm; 7.6\%\\
wing/leg stretching & 45cm; 23cm; 4.1\% & 22cm; 41cm; 5.2\% & 21cm; 22cm; 6.3\%\\
drinking/eating & 37cm; 17cm; 3.2\% & 34cm; 40cm; 5.1\% & 12cm; 34cm; 6.9\%\\
preening & 30cm; 35cm; 12.8\% & 31cm; 31cm; 12.8\% & 3cm; 18cm; 18.1\%\\
\bottomrule
\end{tabular}
\end{table}

\end{knitrout}
}

Leider kennen sich die Vier nicht so gut mit der Berechnung aus! Daher brauchen die Vier Ihre Hilfe!

\begin{enumerate}
\item Skizzieren Sie die Werte $r_i$, $R_i$ und $A_i$ für zwei nebeneinander agierende Puten für ein Verhalten $i$. Nutzen Sie hierfür vereinfachte geometrische Formen! \textbf{(2 Punkte)}
\item Erstellen Sie eine zusammenfassende Tabelle mit den mittleren Werten für $r$, $R$ und $PB$ aus der obigen Tabelle 1 für die jeweiligen Verhalten! \textbf{(3 Punkte)}
\item Ergänzen Sie eine Spalte mit dem benötigten Platz $A$ für das jeweilige Verhalten, welches sich aus den mittleren Werten ergibt! \textbf{(1 Punkt)}
\item Berechnen Sie das benötigte Platzangebot $SA$ für alle betrachteten Verhalten! \textbf{(1 Punkt)}
\item Sie entnehmen der Literatur folgende Aussage zur Verteilung der Puten in der Fläche $A$: \textit{"`Assuming, that the animals will optimally and equally distribute in an area $A$, we observe a
    small part, which is not covered. This area is called $\omega$ and is calculated with $\omega = \tfrac{A}{0.9069}$."'} Veranschaulichen Sie die Fläche $\omega$ in einer aussagekräftigen Abbildung!  \textbf{(1 Punkt)}
\item Ein Tier braucht Platz für sich selbst. Berechnen Sie nun die Körperfläche $a$, die ein Tier einnimmt. Welche Annahmen haben Sie für die Berechnung der Körperfläche getroffen? \textbf{(2 Punkte)}
\end{enumerate}



 
\clearpage
% ----------------------------------------------------------------------- 

\section{Aufgabe \hfill (10 Punkte)}

\textit{Geben Sie grundsätzlich Formeln und Rechenweg zur Lösung der Teilaufgaben mit an!} \\[1Ex]
 

 
%% --------------------------------------------------------------------
\ifcollection
\begin{flushright}
\tiny\vspace{-3Ex}
\textbf{\examinhaltstart}
\exammodulemathstat
\vspace{-4Ex}
\end{flushright}
\begin{minipage}[t]{0.5\textwidth}
\includegraphics[width = 1.3cm]{/Users/kruppajo/work/GitHub/exam/avatare/Alex.png}\hspace{-4mm}\includegraphics[width = 1.3cm]{/Users/kruppajo/work/GitHub/exam/avatare/Steffen.png}
\end{minipage}
\begin{minipage}[t]{0.5\textwidth}
\hfill
\href{https://youtu.be/1B53cVFIU7Q}{\includegraphics[width = 2cm]{img/youtube}}
\end{minipage}
\fi
%% --------------------------------------------------------------------



\ifcollection
\paragraph{Nelken von den Molukken}
\fi



Alex und Steffen waren gemeinsam in Berlin und sitzen nun im IC nach Amsterdam um zurück nach Osnabrück zu fahren. 'Weißt du was ich mich frage?', entfährt es Alex ziemlich plötzlich, so dass Steffen die Oreos aus dem Mund fallen. 'Nein, und ehrlich gesagt bin ich auch ziemlich müde...'. Das ist jetzt aber Alex egal, denn er möchte folgende Sachlage diskutieren. Und Alex hat jetzt 3 Stunden Zeit. Plus Verspätung. In der Ausstellung \textit{Europa und das Meer} im Deutschen Historischen Museum in Berlin gab es folgendes Zitat über die Probleme der frühen Hochseeschifffahrt.

\begin{quote}
  >>Ohne ausreichende Zufuhr von Vitamin C stellen sich nach 45 Tagen die ersten Symptome ein; die ersten Toten sind nach 65 Tagen zu beklagen; nach 110 Tagen rafft die Skorbut eine ganze Schiffsbesatzung dahin<<
\end{quote}

Ferdinand Magellan stach im Jahre 1519 in See um eine Passage durch den südamerikanischen Kontinent zu finden. Zu seiner Flotte gehörten fünf Schiffe - das Flaggschiff Trinidad, die San Antonio, die Victoria, die Concepciön und die Santiago - mit einer Besatzung von insgesamt 237 Mann. 

\begin{enumerate}
\item Stellen Sie den Verlauf der Anzahl an Matrosen auf einem Schiff der Flotte in der Form einer Überlebenszeitkurve dar! Beschriften Sie die Achsen entsprechend! \textbf{(2 Punkte)} 
\item Was ist die Besonderheit der Überlebenszeitkurve? Begründen Sie Ihre Antwort! \textbf{(2 Punkte)} 
\item Schätzen Sie die Überlebenswahrscheinlichkeit nach 95 Tagen aus Ihrer Abbildung ab! \textbf{(1 Punkt)} 
\end{enumerate}

Der Chronist an Bord der Trinidad, Antonio Pigafetta, schrieb in seinem Bericht '[...] Um nicht Hungers zu sterben, aßen wir das Leder, mit dem die große Rahe zum Schutz der Taue umwunden war.' Insbesondere die Mannschaft der Concepciön erlitt große Verluste durch die Skrobut bei der Überquerung des Pazifiks, da durch Erkundungsfahrten weniger Zeit blieb, um wilden Sellerie aufzunehmen. Wilder Sellerie enthält $4000\mu g/100g$ Vitamin C. Der Bedarf liegt bei $120mg$ pro Tag für Männer.

\begin{enumerate}
  \setcounter{enumi}{2}
\item Berechnen Sie die notwendige Menge in $kg$ an aufzunehmenden wilden Sellerie auf die Concepciön für die ununterbrochene Fahrt von drei Monate und 22 Tage über den Pazifik! \textbf{(3 Punkte)}
\item Skizzieren Sie die Überlebenszeitkurve für die Concepciön im Vergleich zu der Überlebenszeitkurve der Trinidad! Beschriften Sie die Achsen! \textbf{(2 Punkte)}
\end{enumerate}

 
\clearpage
% ----------------------------------------------------------------------- 

\section{Aufgabe \hfill (10 Punkte)}

\textit{Geben Sie grundsätzlich Formeln und Rechenweg zur Lösung der Teilaufgaben mit an!} \\[1Ex]
 

 
%% --------------------------------------------------------------------
\ifcollection
\begin{flushright}
\tiny\vspace{-3Ex}
\textbf{\examinhaltstart}
\exammodulemathstat
\vspace{-4Ex}
\end{flushright}
\begin{minipage}[t]{0.5\textwidth}
\includegraphics[width = 1.3cm]{/Users/kruppajo/work/GitHub/exam/avatare/Jonas.png}\hspace{-4mm}\includegraphics[width = 1.3cm]{/Users/kruppajo/work/GitHub/exam/avatare/Nilufar.png}
\end{minipage}
\begin{minipage}[t]{0.5\textwidth}
\hfill
\href{https://youtu.be/q-qYK4Chslg}{\includegraphics[width = 2cm]{img/youtube}}
\end{minipage}
\fi
%% --------------------------------------------------------------------



\ifcollection
\paragraph{Event Horizon -- Am Rande des Universums}
\fi



Nilufar ist bei Jonas um gemeinsam \textit{Event Horizon -- Am Rande des Universums} zu streamen. Das war jetzt nicht die beste Idee. Denn Nilufar kann Horror überhaupt nicht ab. Deshalb flüchtet sie sich in Logik um ihre Emotionen zu bändigen. Jonas mampft ungerührt Snickers. Folgenden Gedankengang nutzt Nilufar um dem Film zu entkommen. Die Sonne hat eine aktuelle, angenommene Masse von $\ensuremath{2\times 10^{31}}$kg. Wenn die Sonne nun am Ende ihrer Lebenszeit zu einem schwarzen Loch mit dem Radius von $5000$m kollabiert, wird die Sonne $20$\% der aktuellen Masse verloren haben. Ein Lichtteilchen mit der Masse $m_f$ und der Fluchtgeschwindigkeit $v_f$ will dem schwarzen Loch entkommen. An folgende Formeln erinnert sich Nilufar für die kinetische Energie des Lichtteilchens $E_{kin}$ und der Graviationsenergie des schwarzen Lochs $E_{grav}$\footnote{Die Quelle der Inspiration für die Aufgabe war ein Montagnachtfilm: \href{https://de.wikipedia.org/wiki/Event_Horizon_–_Am_Rande_des_Universums}{Event Horizon – Am Rande des Universums}}.

\begin{center}
  \begin{tabular}{cc}
    $E_{kin} = \cfrac{1}{2}m_fv_f^2$ & $E_{grav} = \cfrac{Gm_sm_f}{r_s}$\\
  \end{tabular}
\end{center}

mit

\begin{itemize}[noitemsep]
\item $m_f$, gleich der Masse [kg] des fliehenden Objektes
\item $m_s$, gleich der Masse [kg] des stationären Objekts
\item $r_s$, gleich dem Radius [m] des stationären Objekts  
\item $G$, gleich der Gravitationskonstante mit $5.974 \cdot 10^{-11} m^3(kg \cdot s^2)^{-1}$ 
\end{itemize}

Im Folgenden wollen wir Nilufar bei der Ablenkung helfen und uns mit der Frage beschäftigen, ob das Lichtteilchen der Gravitation des schwarzen Lochs entkommen kann.

\begin{enumerate}
\item Geben Sie die Formel für die Fluchtgeschwindigkeit $v_f$ an! \textbf{(2 Punkte)}
\item Überprüfen Sie Ihre umgestellte Formel nach $v_f$ anhand der Einheiten! \textbf{(1 Punkt)} 
\item Berechnen Sie die notwendige Fluchtgeschwindigkeit $v_f$ des Lichtteilchens mit den angegebenen Informationen! \textbf{(2 Punkte)}
\item Gehen Sie von einer Lichtgeschwindigkeit von $\ensuremath{2.9\times 10^{8}}m/s$ aus. Kann das Lichtteilchen der Gravitation des schwarzen Lochs entkommen? Begründen Sie Ihre Antwort! \textbf{(2 Punkte)}
\item Stellen Sie den Zusammenhang zwischen dem sich verringernden Radius $r$ des schwarzen Lochs bei gleichbleibender Masse $m_s$ und der notwendigen Fluchtgeschwindigkeit $v_f$ in einer Abbildung dar! \textit{Erstellen Sie dafür eine Datentabelle mit fünf Werten für den Radius $r$!} \textbf{(3 Punkte)}
\end{enumerate}

 
\clearpage
% -----------------------------------------------------------------------

\section{Aufgabe \hfill (10 Punkte)}

\textit{Geben Sie grundsätzlich Formeln und Rechenweg zur Lösung der Teilaufgaben mit an!} \\[1Ex]
 

 
%% --------------------------------------------------------------------
\ifcollection
\begin{flushright}
\tiny\vspace{-3Ex}
\textbf{\examinhaltstart}
\exammodulemathstat
\vspace{-4Ex}
\end{flushright}
\begin{minipage}[t]{0.5\textwidth}
\includegraphics[width = 1.3cm]{/Users/kruppajo/work/GitHub/exam/avatare/Alex.png}\hspace{-4mm}\includegraphics[width = 1.3cm]{/Users/kruppajo/work/GitHub/exam/avatare/Tina.png}
\end{minipage}
\begin{minipage}[t]{0.5\textwidth}
\hfill
\href{https://youtu.be/iCQogS6KhPM}{\includegraphics[width = 2cm]{img/youtube}}
\end{minipage}
\fi
%% --------------------------------------------------------------------

%% --------------------------------------------------------------------
{\tiny\textbf{Stichworte:} Great filter $\bullet$ SETI $\bullet$ WOW-Signal $\bullet$ 5-Sigma $\bullet$ Voyager 1 $\bullet$ Voyager 2}
%% --------------------------------------------------------------------




\ifcollection
\paragraph{Das Fermi Paradoxon}
\fi



Tina und Alex wandern durch den Teuteburgerwald um mal vom Studium runterzukommen. 'Kennst du eigentlich Enrico Fermi?', fragt Tina und fährt ohne die Antwort abzuwarten fort, 'Er war ein berümter Kernphysiker! Enrico Fermi diskutierte 1950 auf dem Weg zum Mittagessen im Los Alamos National Laboratory mit seinen Kollegen angebliche UFO-Sichtungen und fragte schließlich: >>Where is everybody?<<. Warum seien weder Raumschiffe anderer Weltraumbewohner noch andere Spuren extraterrestrischer Technik zu beobachten?'. Alex schaut sie irritiert und interessiert an. Die beiden hat das Problem gepackt. Deshalb wollen Tina und Alex das Paradoxon mal mathematisch untersuchen! Wie lange würde eine außerirdische Zivilisation benötigen um die gesamte Milchstraße zu besuchen, wenn das maximale Reisetempo die Geschwindigkeit der Voyager 1 Sonde wäre?\footnote{Die Quelle der Inspiration für die Aufgabe war folgender Wikipediaeintrag: \href{https://de.wikipedia.org/wiki/Fermi-Paradoxon}{Fermi-Paradoxon}}\\[-1ex]

Die beiden treffen folgende Annahmen. Eine außerirdische Zivilisation schickt $drei$ Voyager 1 ähnliche Sonden mit der Geschwindigkeit von $\ensuremath{6.0523\times 10^{4}}km/h$ los um sich auf den erreichten Planeten selbst zu replizieren. Nach $750$ Jahren ist die Replikation abgeschlossen und wiederum $drei$ Sonden werden ausgesendet. Gehen Sie von $3.57$ Lichtjahren als mittlerer Abstand der Sterne in der Milchstraße aus. Es gibt $\ensuremath{10^{11}}$ Sterne in der Milchstraße. Nehmen Sie eine Lichtgeschwindigkeit von $\ensuremath{2.9\times 10^{8}}m/s$ an.

\begin{enumerate}
\item Skizzieren Sie in einer Abbildung die ersten drei Schritte der Vervielfältigung der Sonden in der Galaxie! Beschriften Sie die Abbildung mit der Dauer und der Anzahl an Sonden für jeden Schritt der Vervielfältigung! \textbf{(4 Punkte)}
\item Berechnen Sie die theoretische Anzahl an Vervielfältigungsschritten die benötigt werden um mit \textit{einem einzigen Vervielfältigungsschritt} die gesamten Sterne der Milchstraße mit Sonden zu besuchen! \textbf{(2 Punkte)}
\item Berechnen Sie die Dauer, die eine außerirdische Zivilisation annährungsweise benötigt um die gesamten Sterne der Milchstraße mit Sonden zu besuchen! \textbf{(2 Punkte)}
\item Bei einem vermutetet Alter der Erde von $\ensuremath{4.3\times 10^{9}}$ Jahren, wie oft war dann eine Sonde einer außerirdischen Zivilisation schon zu Besuch? Korrigieren Sie Ihre Antwort mit dem Wissen, dass sich die Kontinentalplatten einmal alle $\ensuremath{8\times 10^{7}}$ Jahre vollständig im Erdinneren umgewandelt haben! \textbf{(2 Punkte)}
\end{enumerate}


 
\clearpage
% -----------------------------------------------------------------------

\section{Aufgabe \hfill (10 Punkte)}

\textit{Geben Sie grundsätzlich Formeln und Rechenweg zur Lösung der Teilaufgaben mit an!} \\[1Ex]
 

 
%% --------------------------------------------------------------------
\ifcollection
\begin{flushright}
\tiny\vspace{-3Ex}
\textbf{\examinhaltstart}
\exammodulemathstat
\vspace{-4Ex}
\end{flushright}
\begin{minipage}[t]{0.5\textwidth}
\includegraphics[width = 1.3cm]{/Users/kruppajo/work/GitHub/exam/avatare/Yuki.png}
\end{minipage}
\begin{minipage}[t]{0.5\textwidth}
\hfill
\href{https://youtu.be/tDgr6fpkkYA}{\includegraphics[width = 2cm]{img/youtube}}
\end{minipage}
\fi
%% --------------------------------------------------------------------



\ifcollection
\paragraph{Pyramiden bauen}
\fi



Es stehen die niedersächsichen Pyramidentage an! Sie und Yuki sind auf abenteuerlichen Wegen für den Bau der Pyramiden zuständig. Zu allem Überfluss handelt es sich auch noch eine \textit{Reenactment} Veranstaltung. Thema der diesjährigen Pyramidentage sind die Pyramiden von Meroe, die den Königen und Königinnen des historischen Reiches von Kusch in Nubien, dem heutigen Sudan, als Grabstätten dienten. Die Pyramiden in Meroe fallen durch ihren steilen Winkel von 73 Grad im Vergleich zu den ägyptischen Pyramiden mit 52 Grad auf. Die durchschnittliche Seitenlänge der Grundfläche einer Pyramide beträgt 44 Königsellen. Eine Königselle misst 52.6cm.\\

\textit{Lösen Sie diese Aufgabe mit Hilfe einer Skizze der Pyramide. Bezeichnen Sie Seiten und die Winkel der Pyramide entsprechend!}

\begin{enumerate}
\item Bei der Königspyramide von Meroe soll eine Seitenlänge der Grundfläche 44 Königsellen lang sein. Einer der Skla'angestellten hat eine Höhe der Königspyramide von 37.61$m$ berechnet. Dem müssen Sie auf jeden Fall nachgehen. Überprüfen Sie die Berechnung! \textbf{(1 Punkt)}
\item Die Außenflächen der Königspyramide soll begrünt werden. Für die Bepflanzung muss eine 5cm dicke Torfschicht auf die Königspyramide aufgebracht werden. Berechnen Sie die ungefähre Menge an benötigten Torf in $m^3$! \textbf{(2 Punkte)}
\end{enumerate}

Wie in jedem guten \textit{Reenactment} gibt es viel Oberschicht, aber nur 2 Sklaven, die Ihnen und Yuki bei dem Befüllen der Königspyramide mit Schutt zu Seite stehen. Leider haben Ihre Sklaven zu allem Überfluss auch noch chronische Rückenschmerzen entwickelt, als die Sklaven von der anstehenden Aufgabe erfahren haben. Gehen Sie daher von einer Effizienz der Sklaven von 90\% aus. In eine Schubkarre passen 100 Liter.

\begin{enumerate}
  \setcounter{enumi}{2}
\item Wie oft müssen Ihre maladen Sklaven die Rampe mit der Schubkarre zur Spitze der Königspyramide hochfahren um die Pyramide mit Schutt zu füllen? \textbf{(1 Punkt)}
\item Berechnen Sie die Länge der Rampe zur Spitze der Königspyramide mit einem Anstellwinkel von $10^\circ$! \textbf{(2 Punkte)}
\item Wie weit reicht Ihre Rampe vom Fuß der Königspyramide in die niedersächsiche Landschaft?  \textbf{(2 Punkte)}
\end{enumerate}

Bei der Besichtigung der Königspyramide teilt Ihnen der leicht übergewichtige Pharao (Nebenberuf \textit{Mittelständler}) mit, das die Pyramide \underline{zu steil sei} und somit nicht in die niedersächsiche Landschaft passen würde. Sie müssen nochmal ran.

\begin{enumerate}
  \setcounter{enumi}{5}
\item Die Grundfläche der Königspyramide ändert sich nicht. Berechnen Sie die Änderung der Höhe in \underline{Königsellen}, wenn sich der Anstellwinkel der Königspyramide um $8^\circ$ ändert!  \textbf{(2 Punkte)}
\end{enumerate}



% https://de.wikipedia.org/wiki/Rechtwinkliges_Dreieck
% https://www.matheretter.de/wiki/pyramide
% https://de.wikipedia.org/wiki/Knickpyramide 
\clearpage
% -----------------------------------------------------------------------

\section{Aufgabe \hfill (10 Punkte)}

\textit{Geben Sie grundsätzlich Formeln und Rechenweg zur Lösung der Teilaufgaben mit an!} \\[1Ex]
 

 
%% --------------------------------------------------------------------
\ifcollection
\begin{flushright}
\tiny\vspace{-3Ex}
\textbf{\examinhaltstart}
\exammodulemathstat
\vspace{-4Ex}
\end{flushright}
\begin{minipage}[t]{0.5\textwidth}
\includegraphics[width = 1.3cm]{/Users/kruppajo/work/GitHub/exam/avatare/Alex.png}\hspace{-4mm}\includegraphics[width = 1.3cm]{/Users/kruppajo/work/GitHub/exam/avatare/Mark.png}
\end{minipage}
\begin{minipage}[t]{0.5\textwidth}
\hfill
\href{https://youtu.be/3LAq3R0rS14}{\includegraphics[width = 2cm]{img/youtube}}
\end{minipage}
\fi
%% --------------------------------------------------------------------



\ifcollection
\paragraph{Geocaching -- Von Satelliten und Plastikdosen}
\fi



Es ist Wochenende und das Wetter ist \textit{sweet}. Alex und Mark schwingen sich auf ihre Cachermobile um mit 17km/h, geleitet von modernster Satellietentechnologie und einem Supercompter aus dem Jahr 2000 in den Händen, Plastikdosen in der Natur und an sehenswerten Orten zu finden. Alex und Mark wollen diesmal endlich die aufwärts Schwierigkeitschallenge durchführen. Die Reihenfolge der Caches nach Schwierigkeitswertung gibt daher die von den beiden abzufahrenden Orte vor. Die Terrain- und Schwierigkeitswertungen laufen von 1 (leichteste Wertung) bis 5 (schwierigste Wertung) in 0.5 Schritten. Folgende Informationen zu den Orten und den entsprechenden Caches stehen Alex und Mark für die Planung der Route zu Verfügung\footnote{Die Quelle der Inspiration für die Aufgabe war folgende Tätigkeit: \href{https://www.geocaching.com/play}{Geocaching -- Mach mit bei der weltweit größten Schatzsuche.}}.

\begin{center}
  \begin{tabular}{ ccc }
    \toprule
    Ort & Cache & Wertung (S|T|G) \\
    \midrule
    A & GCAFIE9 & 2.5 | 4.5 | Mikro \\
    B & GCDCNX7 & 3.5 | 2.0 | Normal \\ 
    C & GCISRD2 & 4.5 | 4.0 | Normal \\ 
    D & GC9GBBB & 4.0 | 1.0 | Mikro \\ 
    E & GC4MPQ5 & 5.0 | 1.5 | Mikro \\     
 \bottomrule
\end{tabular}
\end{center}

Im Weiteren sind den beiden folgende Informationen zu den Entfernungen der Orte zugänglich. Der Entfernungsvektor $\overrightarrow{AB}$ ist $3$km. Im Weiteren ist Ihnen der Entfernungsvektor $\overrightarrow{CB}$ mit $5.5$km bekannt. Der Entfernungsvektor $\overrightarrow{BE}$ ist das $1.3$-fache des Entfernungsvektor $\overrightarrow{CB}$. Wenn Sie von dem Ort A den Ort C anpeilen, so liegt der Ort B ungefähr $25^\circ$ südlich. Wenn Sie von dem Ort C den Ort B anpeilen, so liegt der Ort D ungefähr $50^\circ$ östlich. Vom Ort B betrachtet, bilden die Orte C und D einen rechten Winkel am Ort B. Der Ort B liegt auf gerader Linie zwischen den Orten C und E. Somit liegt der Ort E südlich von B. Die Strecke zwischen A und E ist nicht passierbar. Sie starten an dem Ort B Ihre Cachertour. \\

Leider sind die beiden sehr schlecht im Navigieren und Entfernungen ausrechnen. Die beiden brauchen Ihre Hilfe!
  
\begin{enumerate}
\item Lösen Sie diese Aufgabe mit Hilfe einer aussagekräftigen Skizze der Orte und Caches. Bezeichnen Sie die Strecken und die Winkel Ihrer Skizze entsprechend! \textbf{(5 Punkte)}
\item Welche Strecke in $km$ legen Sie bei der Bewältigung der aufwärts Schwierigkeitschallenge zurück? \textbf{(2 Punkte)}
\item Gehen Sie von einer zusätzlichen Suchzeit in Stunden für die Caches an den jeweiligen Orten zur reinen Reisezeit mit Ihrem Cachermobil aus. Die Suchzeit in Stunden für \textit{jeden einzelnen} Cache wird durch die Funktion  
  \begin{equation*}
    Suchzeit = 0.2 + 0.15 \cdot Schwierigkeit
  \end{equation*}  
  beschreiben.  Wie lange in Stunden benötigen Sie um die aufwärts Schwierigkeitschallenge zu erfüllen? \textbf{(3 Punkte)}
%\item An der höchsten Schwierigkeit müssen Sie angeln. Ihre Angel ist ausgefahren l_angelm lang. Erreichen Sie einen Cache in der Höhe von h_angelm?  Berechnen Sie dazu Ihre maximale mögliche Angelhöhe! Welche Annahmen mussten Sie treffen um die Aufgabe zu lösen? \textbf{(2 Punkte)} 
\end{enumerate}

 
\clearpage
% -----------------------------------------------------------------------

\section{Aufgabe \hfill (10 Punkte)}

\textit{Geben Sie grundsätzlich Formeln und Rechenweg zur Lösung der Teilaufgaben mit an!} \\[1Ex]
 

 
%% --------------------------------------------------------------------
\ifcollection
\begin{flushright}
\tiny\vspace{-3Ex}
\textbf{\examinhaltstart}
\exammodulemathstat
\vspace{-4Ex}
\end{flushright}
\begin{minipage}[t]{0.5\textwidth}
\includegraphics[width = 1.3cm]{/Users/kruppajo/work/GitHub/exam/avatare/Paula.png}
\end{minipage}
\begin{minipage}[t]{0.5\textwidth}
\hfill
\href{https://youtu.be/4-dSaPMhK9s}{\includegraphics[width = 2cm]{img/youtube}}
\end{minipage}
\fi
%% --------------------------------------------------------------------

%% --------------------------------------------------------------------
{\tiny\textbf{Stichworte:} Brot aus Luft $\bullet$ Walöl $\bullet$ Haber-Bosch-Verfahren $\bullet$ 1. Weltkrieg $\bullet$ 40\% N im menschlichen Körper $\bullet$ Positivist}
%% --------------------------------------------------------------------



\ifcollection
\paragraph{Die atmende Wand und Brot aus Luft}
\fi



Als Kellerkind\footnote{\href{https://www.youtube.com/watch?v=54H0HAJexVI}{Tocotronic - Electric Guitar} als passende Untermalung für diese Aufgabe.} vom Dorf will Paula das Ausmaß der Radonbelastung in ihrem Kellerzimmer bestimmen und lüften daher nicht. Passt schon. Spart dann auch Energie und lüften wird sowieso überschätzt. Während einer Messperiode von 7:00 Uhr bis 17:00 bestimmt sie dreimal automatisch die Radonbelastung in ihrem Kellerraum in $Bq/m^3$. Es ergibt sich folgende Abbildung\footnote{Die Quelle der Inspiration für die Aufgabe war folgender Artikel: \href{https://de.wikipedia.org/wiki/Atmende_Wand}{Atmende Wand}}. Leider helfen die Messwerte Paula überhaupt nicht weiter. Sie müssen also helfen! 

\begin{knitrout}
\definecolor{shadecolor}{rgb}{0.969, 0.969, 0.969}\color{fgcolor}

{\centering \includegraphics[width=\maxwidth]{img/math-10-1} 

}


\end{knitrout}

\vspace{-0.75cm}

\begin{enumerate}
\item Wie lange dauert es in Stunden bis Paula eine kritische Belastung von 320$Bq/m^3$ in dem ungelüfteten Kellerraum erreicht hat? \textbf{(2 Punkte)}
\end{enumerate}

Radon zerfällt mit einer Halbwertszeit von 2.8d zu Polonium. Polonium wiederum zerfällt mit einer Halbwertszeit von 160d zu Blei. Nur Radon und Polonium tragen zur radioaktiven Strahlenbelastung bei.

\begin{enumerate}
  \setcounter{enumi}{1}
\item Wie lange dauert es in Stunden bis die kritische Radonbelastung von
  320$Bq/m^3$ auf unter 80$Bq/m^3$ gefallen ist?
  \textbf{(4 Punkte)}
\end{enumerate}

Folgende Tabelle enthält die Informationen zur Zusammensetzung der normalen Umgebungsluft.

\begin{center}
  \begin{tabular}{ c|c|c|c }
     & Vol-\% & M [g/mol] & ppm \\
    \hline
    Stickstoff & 79.7 & 28.4 &
                                                    \phantom{1000000000000}\strut\\
        \hline
    Sauerstoff & 19.5 & 16.5 &
                                                    \phantom{10000000}\strut\\
        \hline
    Kohlenstoffdioxid & 0.035 & 11.8 & \phantom{10000000}\strut\\     
     \hline
\end{tabular}
\end{center}

\begin{enumerate}
   \setcounter{enumi}{2}
\item Rechnen Sie die Volumenprozente (Vol-\%) der Umgebungsluft in die entsprechenden ppm-Werte um und ergänzen Sie die berechneten ppm-Werte in die Tabelle!  \textbf{(1 Punkt)}
\end{enumerate}

Während Paula ihr etwas pappiges Toastbrot mampfen kommt Paula die Dokumentation über Brot aus Luft in den Sinn. Paula denkt darüber ein wenig nach. Für die Umwandlung von Stickstoff $N_2$ mit Wasserstoff $H_2$ zu Ammoniak $NH_3$ gilt folgende Reaktionsgleichung\footnote{Die Quelle der Inspiration für die Aufgabe war folgender Artikel: \href{https://www.gdch.de/netzwerk-strukturen/fachstrukturen/ag-chemie-und-gesellschaft/projekte-und-veranstaltungen/cartoons/haber-bosch-verfahren-brot-aus-luft.html}{Haber-Bosch-Verfahren – Brot aus Luft}}:

\begin{equation*}
  N_2 + 3H_2 \rightarrow 2NH_3
\end{equation*}  

Ein Mol eines beliebigen Gases hat bei normalen Umweltbedingungen ein Volumen von 22.4 Liter. % Ein Mol enthält $6 \cdot 10^{23}$ Teilchen.

\begin{enumerate}
  \setcounter{enumi}{3}
\item Welche Masse an Ammoniak $2NH_3$ in Kilogramm $kg$ können Sie aus einem Kubikmeter $m^3$ Luft unter normalen Umweltbedingungen gewinnen? \textit{Beachten Sie die Reaktionsgleichung und gehen Sie von einem molaren Gewicht von Wasserstoff $H_2$ mit $2g/mol$ aus.}
  \textbf{(3 Punkte)}
\end{enumerate}

 
\clearpage
% -----------------------------------------------------------------------

\section{Aufgabe \hfill (10 Punkte)}

\textit{Geben Sie grundsätzlich Formeln und Rechenweg zur Lösung der Teilaufgaben mit an!} \\[1Ex]
 

 
%% --------------------------------------------------------------------
\ifcollection
\begin{flushright}
\tiny\vspace{-3Ex}
\textbf{\examinhaltstart}
\exammodulemathstat
\vspace{-4Ex}
\end{flushright}
\begin{minipage}[t]{0.5\textwidth}
\includegraphics[width = 1.3cm]{/Users/kruppajo/work/GitHub/exam/avatare/Steffen.png}
\end{minipage}
\begin{minipage}[t]{0.5\textwidth}
\hfill
\href{https://youtu.be/Bbu6n8MXxQk}{\includegraphics[width = 2cm]{img/youtube}}
\end{minipage}
\fi
%% --------------------------------------------------------------------



\ifcollection
\paragraph{Armee der Finsternis}
\fi



Der Studentenjob von Steffen war nach Ladenschluss bei Penny die Regale einzuräumen. Dabei ist Steffen in der Auslage der Sonderangebote das Necronomicon\footnote{Ein wirklich gefährliches Buch ist: \textit{Du bist genug: Vom Mut, glücklich zu sein} von Fumitake Koga und Ichiro Kishimi} in die Hände gefallen. Nun ist er eine Magierin der Zeichen geworden! Also eigentlich kann Steffen nur Mathe und das dämliche Necronomicon hat ihn in die Vergangenheit geschleudert... aber gut, was tut man nicht alles im Jahr 705 n. Chr. für den neuen Lehnsherren Fürsten Arthur. Steffen baut natürlich einen Schrottkugelturm um sich den
Horden der Finsternis mit genug Schrott erwehren zu können! Steffen stehen zwei mächtige magische Formeln zur Unterstützung zu Verfügung. Leider wird das nicht reichen, deshalb müssen Sie hier auch noch durch Zeit und Raum helfen!

\begin{center}
  \begin{tabular}{cc}
    $E_{kin} = \cfrac{1}{2}\cdot m \cdot v^2$ & $E_{pot} = m \cdot g \cdot h$\\
  \end{tabular}
\end{center}

mit

\begin{itemize}[noitemsep]
\item $m$, gleich der Masse [kg] des Objekts
\item $h$, gleich der Höhe [m] des ruhenden Objekts
\item $v$, gleich der Geschwindigkeit [m/s] des Objekts
\item $g$, gleich der Erdbeschleunigung mit $9.81 \tfrac{m}{s^2}$ 
\end{itemize}

Als erstes müssen Sie die Höhe des zu bauenden Schrottkugelturmes bestimmen. Hierfür ist wichtig zu wissen, dass sich die Blei\textit{tropfen} mit einem Gewicht von $10mg$ zu gleichförmigen Blei\textit{kugeln} bei einer Geschwindigkeit von $14m/s$ bilden.

\begin{enumerate}
\item Wie hoch müssen Sie den Schrottkugelturm bauen lassen, damit sich runde Bleikugeln durch die Fallgeschwindigkeit von $14m/s$ bilden? \textbf{(3 Punkte)}
\end{enumerate}

Ihre erstellten Schrottkugeln sind leider zu gro{\ss} und somit sind zu wenige Schrottkugeln in einer Ladung. Damit können Sie die Armee der Finsternis nicht aufhalten. Die Sachlage müssen Sie einmal mathematisch untersuchen.

\begin{enumerate}
  \setcounter{enumi}{1}
\item Nennen Sie die beiden geometrischen Formen aus denen sich näherungsweise ein Tropfen zusammensetzt! Erstellen Sie eine beschriftete Skizze des Tropfens! \textbf{(2 Punkte)}
\item Sie messen eine Länge des Tropfens von 3.2mm. Die Löcher im Sieb erlauben ein Tropfendurchmesser von 2mm. Welchen Durchmesser in mm haben Ihre produzierten Blei\textit{kugeln}?  \textbf{(3 Punkte)}
\end{enumerate}

Sie haben jetzt die \ensuremath{2.3\times 10^{5}} Bleikugeln zusammen. Blei hat eine Dichte
von $11.23g/cm^3$.

\begin{enumerate}
  \setcounter{enumi}{3}
\item Wie schwer in Kilogramm $kg$ sind die \ensuremath{2.3\times 10^{5}} produzierten
  Bleikugeln, die Sie jetzt auf die Burgmauer transportieren müssen?
  \textbf{(1 Punkt)}
\end{enumerate}

Am Ende müssen Sie noch die Produktion von dem Bleischrott im Turm optimieren.

\begin{enumerate}
  \setcounter{enumi}{4}
\item Wie gro{\ss} in $cm^2$ ist Ihr quadratisches Sieb am oberen Ende des Turms,
  wenn Sie pro Fall ca. 900 Bleikugeln produzieren wollen und
  die Bleikugel im Fall 0.5cm Abstand haben müssen?  \textbf{(1
    Punkt)}
\end{enumerate}
 
\clearpage
% -----------------------------------------------------------------------

\section{Aufgabe \hfill (10 Punkte)}

\textit{Geben Sie grundsätzlich Formeln und Rechenweg zur Lösung der Teilaufgaben mit an!} \\[1Ex]
 

 
%% --------------------------------------------------------------------
\ifcollection
\begin{flushright}
\tiny\vspace{-3Ex}
\textbf{\examinhaltstart}
\exammodulemathstat
\vspace{-4Ex}
\end{flushright}
\begin{minipage}[t]{0.5\textwidth}
\includegraphics[width = 1.3cm]{/Users/kruppajo/work/GitHub/exam/avatare/Jessica.png}
\end{minipage}
\begin{minipage}[t]{0.5\textwidth}
\hfill
\href{https://youtu.be/Mr6eslls4J0}{\includegraphics[width = 2cm]{img/youtube}}
\end{minipage}
\fi
%% --------------------------------------------------------------------



\ifcollection
\paragraph{Armee der Kaninchen}
\fi



Leider hat es bei Jessica mit der Koalakuschelschule in Down Under nicht geklappt. War vielleicht auch nicht \textit{so} die beste Idee... aber dafür hat Jessica eine neue Eingebung! Oder wie es Mike Tyson zugeschrieben wird: >>Ich wurde nie niedergeschlagen, ich war immer am Aufstehen!<<. Daher macht Jessica jetzt einen Großhandel mit Kaninchenfleisch und damit dem teuersten Fleisch in Australien auf. Moment, hopsen hier nicht, seit Thomas Austin im Jahr 1860 ungefähr 30 Kaninchen entlassen hat, Millionen von Kaninchen rum? Wieso ist das Kaninchenfleisch dann so exklusiv? Jessica wird stutzig und frag Sie, dem mal mathematisch nachzugehen!\footnote{Die Quelle der Inspiration für die Aufgabe war der folgendes YouTube Video: \href{https://youtu.be/38fuOr3tdgc?si=Li7NL_FoByML8JtT}{ Incredible Stories -- Why don't they eat wild rabbits in Australia? They have    millions of them! The reason is surprising...}} \\

Forscherinnen fand folgende Sättigungsfunktion für das jährliche Wachstum der gesamten Kaninchenpopulation im westlichen Australien.
\begin{equation*}
  f(t) = \ensuremath{8\times 10^{9}} - \ensuremath{1.4\times 10^{9}} \cdot 1.7^{-0.2 \cdot t + 3.1}
\end{equation*}

\begin{enumerate}
\item Skizzieren Sie die Sättigungsfunktion \textit{annäherungsweise} in einer Abbildung! \textbf{(1 Punkt)}
\item Wie viele Kaninchen können nach der Sättigungsfunktion maximal im westlichen Australien leben? Ergänzen Sie den Wert in Ihrer Abbildung! \textbf{(2 Punkte)}
\item Wie viele Millionen Kaninchen leben nach der Sättigungsfunktion nach 9 Jahren auf dem australischen Kontinent? \textbf{(1 Punkt)}
\end{enumerate}

Um den Kaninchen Einhalt zu gebieten wurde das Myxoma Virus und das Rabbit Haemorrhagic Disease Virus (RHDV) in 12 Kaninchen ausgebracht. Da die Kaninchen keine Maßnahmen gegen die Ausbreitung vornehmen können, verläuft die Ausbreitung mit einem wöchentlichen Wachstumsfakor von 1.4 nach folgender Formel.

\begin{equation*}
  N(t) = N(0) \cdot a^t
\end{equation*}

\begin{enumerate}
  \setcounter{enumi}{2}
\item Wie viele Wochen benötigen die Viren um theoretisch die gesamte Kaninchenpopulation nach 7 Jahren Wachstum zu durchseuchen? \textbf{(1 Punkt)}
\end{enumerate}

Das Myxoma Virus und das RHDV töten 99.7\% der Kaninchenpopulation innerhalb weniger Wochen.

\begin{enumerate}
  \setcounter{enumi}{3}  
\item Wie lange in Jahren dauert es bis eine Kaninchenpopulation nach einer Viruspandemie wieder auf 30\% der gesättigten Kaninchenpopulation angewachsen ist?  \textbf{(2 Punkte)}
\end{enumerate}

Thomas Austin entließ die Kaninchen im äußersten Norden von Australien. Australien hat eine West-Ost-Ausdehnung von 4200km und eine Nord-Süd-Ausdehnung von knapp 3400km. Die Kaninchen breiten sich radial mit einer Geschwindigkeit von 7.8km pro Jahr aus.

\begin{enumerate}
  \setcounter{enumi}{4}
\item Wie lange dauert es in Jahren bis die Kaninchen jeden Ort in Australien erreicht haben? \textit{Lösen Sie die Aufgabe unter der Verwendung einer schematischen Skizze!} \textbf{(2 Punkte)}
\end{enumerate}

Eine jährliche Impfung gegen das Myxoma Virus und das Rabbit Haemorrhagic Disease Virus (RHDV) kosten 13\$ pro Tier und der durchführende Arzt verlangt ca. 45\$ pro Tier.

\begin{enumerate}
  \setcounter{enumi}{5}
\item In Ihrem Stall leben 1000 Mastkaninchen. Mit welchen jährlichen Zusatzkosten für die Impfungen der Kaninchen müssen Sie daher kalkulieren? \textbf{(1 Punkt)}
\end{enumerate}
 
\clearpage
% -----------------------------------------------------------------------

\section{Aufgabe \hfill (12 Punkte)}

\textit{Geben Sie grundsätzlich Formeln und Rechenweg zur Lösung der Teilaufgaben mit an!} \\[1Ex]
 

 
%% --------------------------------------------------------------------
\ifcollection
\begin{flushright}
\tiny\vspace{-3Ex}
\textbf{\examinhaltstart}
\exammodulemathstat
\vspace{-4Ex}
\end{flushright}
\begin{minipage}[t]{0.5\textwidth}
\includegraphics[width = 1.3cm]{/Users/kruppajo/work/GitHub/exam/avatare/Tina.png}
\end{minipage}
\begin{minipage}[t]{0.5\textwidth}
\hfill
\href{https://youtu.be/fiWGgCX-cE4}{\includegraphics[width = 2cm]{img/caution}}
\end{minipage}
\fi
%% --------------------------------------------------------------------






\ifcollection
\paragraph{Uckermark. Unendliche Weiten.}
\fi



Wir schreiben das Jahr 2025. Dies sind die Abenteuer des Esels Frida und Tina. Grünes Gras unter Tinas Füßen und ein strammer Wind im Gesicht, egal wohin sie schaut. Ein schmatzendes Geräusch ertönt unter Tina. Tina sinniert, sollte sie ihre weiten Graslandschaften jetzt schon düngen? Dafür benötigt Tina die \textit{Grünlandtemperatur}! Die Grünlandtemperatur (GLT) ist die Summe aller positiven Tagesmitteltemperaturen seit Jahresbeginn. Ab einer GLT von 200$^\circ C$ kann mit der Stickstoffdüngung begonnen werden. Tina sieht nicht ein, Geld für einen Agrarmetrologen zu bezahlen, wenn auch Sie mitrechnen können. Also rechnen Sie beide mit folgenden Informationen zu Monatsmultiplikatoren des GLT-Wertes: Januar mit $0.7\times$, Februar mit $0.8\times$ und März mit
$1.1\times$. Sie haben noch im letzten Jahr folgende Temperaturen an den entsprechenden Tagen gemessen.

\begin{center}
\begin{tabular}{cc}
  \toprule
  Datum & C$^\circ$ \\
  \midrule
  01. Jan 2024 & 1.6\\
  01. Feb 2024 & 2.4\\
  01. Mrz 2024 & 5.5\\
  01. Apr 2024 & 6.8\\
  \bottomrule
\end{tabular}
\end{center}

\textit{Ignorieren Sie ein eventuelles Schaltjahr in Ihrer Berechnung.}

\begin{enumerate}
\item Erstellen Sie eine Skizze aus den Informationen aus der Temperaturtabelle! Beschriften Sie die Achsen entsprechend! \textit{Gehen Sie von einem linearen Zusammenhang innerhalb der Monate aus.} \textbf{(1 Punkt)}
\item Stellen Sie die linearen Funktionen $f_1(t)$, $f_2(t)$ und $f_3(t)$ aus der obigen Temperaturtabelle auf! \textit{Nehmen Sie zur Vereinfachung den 01. Januar 2024 als y-Achsenabschnitt für $f_1(t)$ an.}  \textbf{(3 Punkte)}
\item Bestimmen Sie die Stammfunktionen $F_1(t)$, $F_2(t)$ und $F_3(t)$ für die linearen Funktionen aus der obigen Temperaturtabelle!  \textbf{(2 Punkte)}
\end{enumerate}

Nachdem Tina die Stammfunktionen vorliegen hat, kann jetzt mit den GLT-Werten als Fläche unter der Kurve gerechnet werden. 

\begin{enumerate}
  \setcounter{enumi}{3} 
\item Am 20. März 2025 hat Tina Geburtstag. Welchen GLT-Wert wird Tina am 20. März 2025 vermutlich erreicht haben? \textbf{(3 Punkte)}
\item Tina liebt Osterglocken.  Osterglocken beginnen ab einer GLT von 78$^\circ C$ zu blühen. An welchem Tag im 1. Quartal des Jahres 2024 war dies der Fall?  \textbf{(3 Punkte)}
\end{enumerate}

%Auf dem Weg zu names apfellantage werden Sie beide auf dem Trecker von einer Gruppe elektrifizierter Rentner abgedrängt. Der Trecker muss wieder aus dem Graben! name_animal und die elektrifizierten Rentner ziehen an zwei, separaten Seilen. Dabei zieht name_animal mit $n_fridN$. Die elektrifizierter Rentner bringen eine Kraft von $n_rentN$ auf.\\

%\textit{Lösen Sie diese Aufgabe mit Hilfe einer aussagekräftigen Skizze der Kraftvektoren. Bezeichnen Sie die Kraftvektoren und die Winkel Ihrer Skizze entsprechend!}

%\begin{enumerate}
%  \setcounter{enumi}{4}  
%\item Im ersten Versuch legen Sie das Seil für name_animal lotrecht über einen Ast oberhalb des Treckers. Die Rentner ziehen in einer geraden Linie über die Böschung hinweg am anderen Seil. Welche Kraft wird aufgebracht?  \textbf{(2 Punkte)}
%\item Im zweiten Versuch ziehen name_animal und die Rentner mit einem $gamma^\circ$ Winkel mit ihrem Seil an dem Trecker. Welche Kraft wird aufgebracht? \textbf{(2 Punkte)}
%\end{enumerate}

 
\clearpage
% -----------------------------------------------------------------------

\section{Aufgabe \hfill (10 Punkte)}

\textit{Geben Sie grundsätzlich Formeln und Rechenweg zur Lösung der Teilaufgaben mit an!} \\[1Ex]
 

 
%% --------------------------------------------------------------------
\ifcollection
\begin{flushright}
\tiny\vspace{-3Ex}
\textbf{\examinhaltstart}
\exammodulemathstat
\vspace{-4Ex}
\end{flushright}
\begin{minipage}[t]{0.5\textwidth}
\includegraphics[width = 1.3cm]{/Users/kruppajo/work/GitHub/exam/avatare/Jessica.png}\hspace{-4mm}\includegraphics[width = 1.3cm]{/Users/kruppajo/work/GitHub/exam/avatare/Jonas.png}\hspace{-4mm}\includegraphics[width = 1.3cm]{/Users/kruppajo/work/GitHub/exam/avatare/Nilufar.png}\hspace{-4mm}\includegraphics[width = 1.3cm]{/Users/kruppajo/work/GitHub/exam/avatare/Paula.png}
\end{minipage}
\begin{minipage}[t]{0.5\textwidth}
\hfill
\href{https://youtu.be/RuzMjwvwT-4}{\includegraphics[width = 2cm]{img/youtube}}
\end{minipage}
\fi
%% --------------------------------------------------------------------



\ifcollection
\paragraph{In der Kartonagenfabrik}
\fi



Jonas, Jessica, Nilufar und Paula sitzen im Bus. Wenn man sich zu spät anmeldet, dann ist die Exkursion nicht so toll. Jonas hatte den Anderen in der Lerngruppe zu spät Bescheid gesagt. 'Was denn, bin ich eure Nanny oder was?!', entfährt es Jonas nachdem die vorwurfsvollen Blicke schon eine Weile auf ihm lasten. Also geht es eben mit Rektor Skinner und Mrs. Krabappel in die Kartonagenfabrik. Wie schon im vorherigen Semester... In der Kartonagenfabrik angekommen erfahren die Vier, dass die Kartons zum Versand von Nägeln nicht hier zusammengebautwerden sondern das sich die Endfertigung in Flint, Michigan befindet. Unter anderem wird dort der berühmte \textit{Doppelt gewellte, 6-mal-gefaltete, 0.5mm, 40-cm-Karton} durch
Falzung hergestellt. Beim letzten Mal war Rektor Skinner die Stimmung zu schlecht und deshalb geht es erst nach Hause, wenn ein paar Aufgaben gelöst sind. Martin gefällt das. An dem Vorrat an Zigaretten von Mrs. Krabappel meinen alle wenig Zuversicht zu erkennen.\\

Jetzt heißt es Kartons optimieren, wenn Sie auch nochmal nach Hause wollen. Warum jetzt \textit{Sie} mit dabei sind, lassen wir mal weg. Der nun zu optimierende, flache Karton hat eine Länge von 40cm und eine Breite von 23cm. Die Kartonagenmaschine in Flint soll dann einen quadratischen Eckenausschnitt der Länge $x$ falzen.

\begin{enumerate}
\item Erstellen Sie eine Skizze des Karton\textit{blatt}rohlings! Beschriften Sie die Skizze mit den entsprechenden Längenangaben \textbf{(1 Punkt)}
\item Berechnen Sie die Falztiefe $x$ für ein maximales Volumen des flachen Kartons! \textbf{(3 Punkte)}
\item Welches Volumen in Liter ergibt sich mit der von Ihnen berechneten Falztiefe $x$?  \textbf{(1 Punkt)}
\item Sie wollen noch einen bündig mit dem Boden abschließenden Deckel für den Karton stanzen lassen. Wie groß ist die Fläche des Kartondeckel\textit{blatt}rohlings in $cm^2$? \textbf{(2 Punkte)}
\end{enumerate}

Rektor Skinner möchte sich gerne wieder in seinem Vorgarten aufhalten und nicht die ganze Zeit von Bart mit Erdnüssen beworfen werden. Deshalb möchte er einen geräumigen Teil seines Vorgartens einzäunen. Ein Teil der Umzäunung bildet seine Vorderhauswand. Wegen Lieferschwierigkeiten stehen Rektor Skinner nur 130m Zaun zu Verfügung. Auch hier sollen Sie mal helfen, sonst fährt der Bus Sie nicht nach Hause. Sie wollen nun die maximale Fläche des abgeschirmten Vorgartens in Abhängigkeit der Seitenlängen bei der Verwendung von 130m Zaun bestimmen!

\begin{enumerate}
  \setcounter{enumi}{4}  
\item  Welche Seitenlängen für den Zaun ergeben sich für die maximale Fläche des abgeschirmten Vorgartens? \textbf{(2 Punkte)}
\item Berechnen Sie die Fläche des abgeschirmten Vorgartens! \textbf{(1 Punkt)}
\end{enumerate}

 
\clearpage
% -----------------------------------------------------------------------

\section{Aufgabe \hfill (10 Punkte)}

\textit{Geben Sie grundsätzlich Formeln und Rechenweg zur Lösung der Teilaufgaben mit an!} \\[1Ex]
 

 
%% --------------------------------------------------------------------
\ifcollection
\begin{flushright}
\tiny\vspace{-3Ex}
\textbf{\examinhaltstart}
\exammodulemathstat
\vspace{-4Ex}
\end{flushright}
\begin{minipage}[t]{0.5\textwidth}
\includegraphics[width = 1.3cm]{/Users/kruppajo/work/GitHub/exam/avatare/Alex.png}\hspace{-4mm}\includegraphics[width = 1.3cm]{/Users/kruppajo/work/GitHub/exam/avatare/Jonas.png}\hspace{-4mm}\includegraphics[width = 1.3cm]{/Users/kruppajo/work/GitHub/exam/avatare/Mark.png}\hspace{-4mm}\includegraphics[width = 1.3cm]{/Users/kruppajo/work/GitHub/exam/avatare/Nilufar.png}
\end{minipage}
\begin{minipage}[t]{0.5\textwidth}
\hfill
\href{https://youtu.be/OhyuH6hzEhY}{\includegraphics[width = 2cm]{img/youtube}}
\end{minipage}
\fi
%% --------------------------------------------------------------------







\ifcollection
\paragraph{Ein Pfund Insekten, bitte!} 
\fi

'Das wird wohl häufiger gehört werden, wenn wir die Menschheit mit Proteinen ausreichend ernähren wollen\footnote{Die Quelle der Inspiration für die Aufgabe war der folgende Artikel aus dem Spiegel: \href{https://www.spiegel.de/ausland/ueberbevoelkerung-acht-milliarden-sind-wir-bald-zu-viele-menschen-auf-der-erde-a-3f20c7bc-3d60-4440-9f52-eb338db207f5}{Acht Milliarden - sind wir bald zu viele Menschen auf der Erde?}}.', merkt Jonas an. Die Lerngruppe um Mark, Alex und Nilufar sind bei Jonas um mal was außergewöhnliches zu essen. Um den Sinn der Nahrungsumstellung zu verdeutlichen, vergleicht Jonas einmal Deutschland mit Nigeria. Nigeria hat eine der am schnellsten wachsenden Bevölkerungen der Welt und wird vermutlich im Jahr 2100 zu den Top 5 der bevölkerungsreichsten Länder zählen. Im Jahr 2020 leben ca. \ensuremath{8.2\times 10^{7}} Menschen in Deutschland und ca. \ensuremath{1.8\times 10^{8}} Menschen in Nigeria. Mit den Informationen wollen Sie und Jonas mit der Überzeugungsarbeit anfangen und dann eine Prognose für den Fleischkonsum im Jahr 2050 zu treffen. \\ 

Im Folgenden ist Abbildung des Fleischkonsums im Jahr 2020 in Deutschland und Nigeria in [kg] einmal dargestellt. 

\begin{knitrout}
\definecolor{shadecolor}{rgb}{0.969, 0.969, 0.969}\color{fgcolor}

{\centering \includegraphics[width=\maxwidth]{img/math-14-a-1} 

}


\end{knitrout}
  
\begin{enumerate}
\item Stellen Sie den Fleischkonsum in Deutschland und Nigeria im Jahr 2020 \textit{pro Kopf} in einer aussagekräftigen Tabelle dar! \textbf{(2 Punkte)}
\item Ergänzen Sie in der Tabelle eine Spalte in der Sie für den Fleischkonsum in Nigeria auf Deutschland normieren, daher ins Verhältnis Nigeria/Deutschland, setzen! \textbf{(1 Punkt)}
\end{enumerate}

In der nächsten Abbildung finden Sie die CO$_2$-Emission in [kg] nach Lebensmittel, die durch die Produktion entsteht, abgebildet.

\begin{knitrout}
\definecolor{shadecolor}{rgb}{0.969, 0.969, 0.969}\color{fgcolor}

{\centering \includegraphics[width=\maxwidth]{img/math-14-b-1} 

}


\end{knitrout}

\begin{enumerate}
  \setcounter{enumi}{2}  
\item Stellen Sie in einer Tabelle die Treibhausgasemissionen an $CO_2$ pro Kopf, die durch den Fleischkonsum in Deutschland und Nigeria im Jahr 2020 entstehen, dar! Ergänzen Sie auch hier das Verhältnis Nigeria zu Deutschland! \textbf{(2 Punkte)}
\end{enumerate}

\newpage

In der folgenden Abbildung sehen Sie die Bevölkerungsentwicklung [Millionen] in Nigeria von 1950 bis ins Jahr 2030 fortgeführt.

\begin{knitrout}
\definecolor{shadecolor}{rgb}{0.969, 0.969, 0.969}\color{fgcolor}

{\centering \includegraphics[width=\maxwidth]{img/math-14-c-1} 

}


\end{knitrout}

\begin{enumerate}
  \setcounter{enumi}{3}  
\item Schätzen Sie graphisch die zu erwartende Bevölkerung [Millionen] in Nigeria im Jahr 2050, die sich anhand der Informationen aus der Abbildung ergibt!
\begin{enumerate}
\item Ohne Berücksichtigung der Covid-19-Pandemie! \textbf{(1 Punkt)}
\item Unter Berücksichtigung der Covid-19-Pandemie! \textbf{(1 Punkt)}
\end{enumerate}
\item Berechnen Sie den geschätzten Fleischkonsum von Nigeria im Jahr 2050 unter der Annahme 70\%-iger Angleichung der Lebensbedingungen zu Deutschland im Jahr 2020! \textbf{(1 Punkt)}
\item Berechnen Sie die prozentuale Steigerung der Treibhausgasemissionen an $CO_2$ in Nigeria im Jahr 2050 im Vergleich zum Jahr 2020, der sich durch den angeglichenen Fleischkonsum ergibt! \textbf{(1 Punkt)}
\item Berechnen Sie die prozentuale Steigerung der Treibhausgasemissionen an $CO_2$ in Nigeria, wenn die gesamte Proteinaufnahme durch Insekten ersetzt würde! \textbf{(1 Punkt)}
\end{enumerate}







 
\clearpage
% -----------------------------------------------------------------------

\section{Aufgabe \hfill (12 Punkte)}

\textit{Geben Sie grundsätzlich Formeln und Rechenweg zur Lösung der Teilaufgaben mit an!} \\[1Ex]
 

 
%% --------------------------------------------------------------------
\ifcollection
\begin{flushright}
\tiny\vspace{-3Ex}
\textbf{\examinhaltstart}
\exammodulemathstat
\vspace{-4Ex}
\end{flushright}
\begin{minipage}[t]{0.5\textwidth}
\includegraphics[width = 1.3cm]{/Users/kruppajo/work/GitHub/exam/avatare/Alex.png}\hspace{-4mm}\includegraphics[width = 1.3cm]{/Users/kruppajo/work/GitHub/exam/avatare/Nilufar.png}
\end{minipage}
\begin{minipage}[t]{0.5\textwidth}
\hfill
\href{https://youtu.be/flRBo1FWQy0}{\includegraphics[width = 2cm]{img/youtube}}
\end{minipage}
\fi
%% --------------------------------------------------------------------

%% --------------------------------------------------------------------
{\tiny\textbf{Stichworte:} Immunsystem -- Muskel vs. Interpol $\bullet$ Inzidenz $\bullet$ Prävalenz}
%% --------------------------------------------------------------------




\ifcollection
\paragraph{Tödliche Seuche AIDS -- Die rätselhafte Krankheit}
\fi



Irritiert legt Alex die historische Ausgabe des Spiegels aus den 80zigern beiseite. Alex und Nilufar sind bei ihrem Hautarzt und wollen einen AIDS-Test machen lassen. Woanders leider keinen Termin gekriegt... Immerhin denken die beiden über Nachwuchs nach und da geht es eben nur durch ungeschützten Sex. Was wissen Alex und Nilufar nun aber über AIDS und dem diagnostischen AIDS-Test, den die beiden nun machen werden? Leider zu wenig. Da brauchen dann Alex und Nilufar mal wieder Ihre Hilfe bei der Interpretation eines diagnostischen Tests!\\

Die Prävalenz von AIDS bei einem Menschen in Europa wird mit 0.1\% angenommen. In 96\% der Fälle ist ein HIV-Test positiv, wenn der Patient erkrankt ist. In 0.5\% der Fälle ist ein HIV-Test positiv, wenn der Patient \textit{nicht} erkrankt ist und somit gesund ist. Sie stutzen. Wie wahrscheinlich ist es denn eigentlich an AIDS erkrankt zu sein $(K^+)$, wenn Sie einen positiven AIDS-Test vorliegen haben $(T^+)$? Gehen Sie für die folgenden Berechnungen von $n = \ensuremath{3\times 10^{4}}$ Patienten mit einem diagnostischen Test für AIDS aus. Sie nehmen sich also einen Kuli und fangen an auf der historischen Ausgabe des Spiegels zu rechnen\footnote{Die Quelle der Inspiration für die Aufgabe war der folgende wissenschaftlicher Artikel: \textit{Binder et al. (2022) Von Baumdiagrammen über Doppelbäume zu Häufigkeitsnetzen -- kognitive überlastung oder didaktische Unterstützung? Journal für Mathematik-Didaktik, 1-33}}.


\begin{enumerate}
\item Welche Wahrscheinlichkeit $Pr$ wollen Sie berechnen? \textbf{(1 Punkt)}
\item Zeichnen Sie einen Häufigkeitsdoppelbaum zur Bestimmung der gesuchten Wahrscheinlichkeit $Pr$! \textbf{(2 Punkte)} \item Beschriften Sie den Häufigkeitsdoppelbaum, mit denen Ihnen bekannten Informationen zu der AIDS Erkrankung und dem AIDS-Test! \textbf{(1 Punkt)}
\item Füllen Sie den Häufigkeitsdoppelbaum mit den sich ergebenden, absoluten Patientenzahlen $n$ aus! \textbf{(2 Punkte)}
\item Berechnen Sie die gesuchte Wahrscheinlichkeit $Pr$! \textbf{(1 Punkt)}
\end{enumerate}

Bei dem folgenden Arztgespräch erfahren Alex und Nilufar, dass beim diagnostischen Testen \textit{True Positives (TP)}, \textit{True Negatives (TN)}, \textit{False Positives (FP)} und \textit{False Negatives (FN)} auftreten. Das verstehen beiden so noch nicht und deshalb stellen Sie für Alex und Nilufar den Zusammenhang in einer 2x2 Kreuztabelle dar.

\begin{enumerate}
  \setcounter{enumi}{5}
\item Tragen Sie \textit{TP}, \textit{TN}, \textit{FP} und \textit{FN} in eine 2x2 Kreuztablle ein. Beschriften Sie die Tabelle entsprechend! \textbf{(1 Punkt)}
\item Berechnen Sie die Sensitivität und Spezifität des diagnostischen Tests für AIDS! Füllen Sie dafür die 2x2 Kreuztabelle mit den Informationen aus dem Häufigkeitsdoppelbaum aus! \textbf{(2 Punkte)}
\item Was beschreibt die Sensitivität und die Spezifität im Bezug auf die Gesunden und Kranken? Stellen Sie beide diagnostische Maßzahlen als Wahrscheinlichkeiten $Pr$ dar! \textbf{(2 Punkte)} 
\end{enumerate}



 
\clearpage
% -----------------------------------------------------------------------

\section{Aufgabe \hfill (12 Punkte)}

\textit{Geben Sie grundsätzlich Formeln und Rechenweg zur Lösung der Teilaufgaben mit an!} \\[1Ex]
 

 
%% --------------------------------------------------------------------
\ifcollection
\begin{flushright}
\tiny\vspace{-3Ex}
\textbf{\examinhaltstart}
\exammodulemathstat
\vspace{-4Ex}
\end{flushright}
\begin{minipage}[t]{0.5\textwidth}
\includegraphics[width = 1.3cm]{/Users/kruppajo/work/GitHub/exam/avatare/Alex.png}\hspace{-4mm}\includegraphics[width = 1.3cm]{/Users/kruppajo/work/GitHub/exam/avatare/Jessica.png}\hspace{-4mm}\includegraphics[width = 1.3cm]{/Users/kruppajo/work/GitHub/exam/avatare/Mark.png}\hspace{-4mm}\includegraphics[width = 1.3cm]{/Users/kruppajo/work/GitHub/exam/avatare/Steffen.png}
\end{minipage}
\begin{minipage}[t]{0.5\textwidth}
\hfill
\href{https://youtu.be/SZqp_vy3rgI}{\includegraphics[width = 2cm]{img/youtube}}
\end{minipage}
\fi
%% --------------------------------------------------------------------



\ifcollection
\paragraph{Network-Marketing oder Schneeballschlacht!}
\fi



Mark, Steffen und Jessica sitzen bei Alex und hören sich etwas über Network-Marketing an. Alex ist jetzt im Network-Marketing tätig. 'Jetzt reicht es. Wir sind eine Lerngruppe und du versuchst uns hier abzuziehen!', poltert Mark und fährt fort, 'Ich erklär dir mal, wie falsch du liegst!'. Eine Möglichkeit, leicht Geld zu verdienen, ist es anderen Menschen für Geld zu versprechen, wie man leicht reich werden kann. Am besten natürlich ohne viel Aufwand und ortsunabhängig. Schnell ein YouTube-Werbevideo gedreht und auf geht es mit unserem Network-Marketing. Aber Moment, wie funktioniert Network-Marketing eigentlich und was hat das alles mit einer Schneeballschlacht zu tun? Mark und Sie wollen hier einmal in die Untiefen des >>passiven Einkommens<< abtauchen und die Lerngruppe vor Schlimmeren bewahren\footnote{Die Quellen der Inspiration für die Aufgabe waren folgendes YouTube Video: \href{https://youtu.be/UOKkZF_qK9M?si=uf4foJVFKfeQMwSw}{Simplicissimus -- Die meistgesuchte Betrügerin der Welt} und der Artikel: \href{https://www.deutschlandfunkkultur.de/netzwerk-marketing-die-illusion-schnell-reich-zu-werden-100.html}{Deutschlandfunk Kultur -- Die Illusion, schnell reich zu werden}}!\\

Das Jahr 2023 war das erfolgreichste Jahr in der Geschichte von Up/Down Systems and Networking (UDSysNet). Das Unternehmen steigerte den Umsatz um rund 18 Prozent von 290 Millionen Euro im Jahr 2022. Doch wie viel kommt bei den Partnern an? Laut UDSysNet habe das Unternehmen \ensuremath{3.8\times 10^{5}} aktive Partner.

\begin{enumerate}
\item Berechnen Sie zuerst den Umsatz der Firma UDSysNet im Jahr 2023! \textbf{(1 Punkt)}
\item Wie viel von dem Umsatz im Jahr 2023 wird im Durchschnitt von jedem aktiven Partner erwirtschaftet? \textbf{(1 Punkt)}
\item Welche \textit{monatlicher} Umsatz ergibt sich dadurch im Durchschnitt für jeden aktiven Partner bei einer direkten Provision von 30\%? \textbf{(1 Punkt)}
\end{enumerate}

Das von Alex zu vermarkende Produkt, hinter dem Alex voll steht, kostet 50EUR pro Einheit im Direktverkauf. Die direkte Provision für die erste Stufe beträgt 30\%. Für die zweite, dritte und vierte Stufe betragen die indirekten Provisionen jeweils 3.25\%, 2.25\% und 1.75\%. Jeder von Alex angeworbener >>Partner<< wirbt wiederum vier Partner für sich selbst an. Pro Monat werden im Schnitt zwei Einheiten vom Produkt verkauft. Alex will nun 4100EUR im Monat \textit{passiv} -- also durch indirekte Provisionen -- erwirtschaften. Kann das klappen? Sie sind zusammen mit Mark skeptisch.

\begin{enumerate}
  \setcounter{enumi}{3}
\item Ergänzen Sie die folgende Tabelle mit den obigen Informationen! \textbf{(2 Punkte)}
\end{enumerate}

\begin{center}
\begin{tabular}{c|c|c|c}
  \toprule
  \textbf{Stufe} & \textbf{Anzahl Partner}  & \textbf{Umsatz/Stufe} & \textbf{Provision}\\
  \midrule
  1 & Sie selber  &  & \\ \midrule
  2 &   &  &  \\ \midrule
  3 &   &  &  \\ \midrule
  4 &   &  &  \\
  \bottomrule
\end{tabular}
\end{center}

\begin{enumerate}
  \setcounter{enumi}{4}
\item Wie viele Partner müssen Sie auf der 2 Stufe anwerben um Ihr passives Einkommen durch indirekte Provision zu erreichen? Wie viele Menschen arbeiten am Ende indirekt für Sie? Stellen Sie den Zusammenhang graphisch dar!  \textbf{(3 Punkte)}
\end{enumerate}

Alex musste zum Einstieg bei UDSysNet Einheiten des Produkts für 1500EUR kaufen. Diese Einheiten kann Alex nur direkt verkaufen. Das ganze Wohnzimmer ist voll davon. Leider musste Alex den Kauf über einen Kredit über 5.1\% p.a. über 48 Monate finanzieren. Sie schütteln den Kopf und klären Alex über Zinsen auf.

\begin{enumerate}
  \setcounter{enumi}{5}
\item Berechnen Sie die Gesamtsumme, die Sie als Kredit abbezahlen müssen! \textbf{(2 Punkte)}
\item Wie viele Einheiten müssen Sie pro Monat verkaufen um die anfallenden Zinsen durch die direkte Provision zu erwirtschaften? \textbf{(1 Punkt)}
\item Wie lange in Monaten benötigen Sie um den Kredit durch die direkte Provision abzubezahlen? \textbf{(1 Punkt)}
\end{enumerate} 
\clearpage
% -----------------------------------------------------------------------

\section{Aufgabe \hfill (10 Punkte)}

\textit{Geben Sie grundsätzlich Formeln und Rechenweg zur Lösung der Teilaufgaben mit an!} \\[1Ex]
 

 
%% --------------------------------------------------------------------
\ifcollection
\begin{flushright}
\tiny\vspace{-3Ex}
\textbf{\examinhaltstart}
\exammodulemathstat
\vspace{-4Ex}
\end{flushright}
\begin{minipage}[t]{0.5\textwidth}
\includegraphics[width = 1.3cm]{/Users/kruppajo/work/GitHub/exam/avatare/Jonas.png}\hspace{-4mm}\includegraphics[width = 1.3cm]{/Users/kruppajo/work/GitHub/exam/avatare/Nilufar.png}\hspace{-4mm}\includegraphics[width = 1.3cm]{/Users/kruppajo/work/GitHub/exam/avatare/Paula.png}\hspace{-4mm}\includegraphics[width = 1.3cm]{/Users/kruppajo/work/GitHub/exam/avatare/Steffen.png}
\end{minipage}
\begin{minipage}[t]{0.5\textwidth}
\hfill
\href{https://youtu.be/8Pb2sKUIMyk}{\includegraphics[width = 2cm]{img/youtube}}
\end{minipage}
\fi
%% --------------------------------------------------------------------




\ifcollection
\paragraph{Höhlen \& Drachen}
\fi



Steffen, Nilufar und Paula sitzen bei Jonas nachdem sich alle begeistert in der Serie \textit{Stranger Thinks} verloren haben. Alle drei wollen jetzt einmal bei Jonas \textit{Höhlen \& Drachen} ausprobieren. Um Geld zu sparen, das Zeug kostet echt, wurde etwas an den Regeln gebastelt. Schnell stellen die Drei fest, dass hier ganz schön viele unterschiedliche Würfel durch die Gegend fliegen. Daher müssen Sie sich jetzt einiges an Fragen klären damit Jonas nicht so alleine ist. \\%[-1ex]

In dem Spiel hat Steffen nun auf einmal 7 achtseitige Würfel (7d8) zum würfeln in der Hand. Wenn Steffen eine 8 würfelt, hat Steffen einen Erfolg.

\begin{enumerate}
\item Berechnen Sie die Wahrscheinlichkeit \textit{genau} 5 Erfolge zu erzielen!  \textbf{(2 Punkte)}
\item Berechnen Sie die Wahrscheinlichkeit keinen Erfolg zu erzielen! \textbf{(1 Punkt)}
\end{enumerate}

Nilufar betrachtet nun aufmerksam die ausufernden Ausrüstungstabellen. Nilufar wird aber geholfen und muss sich jetzt nur zwischen der Axt oder dem Schwert entscheiden.

\begin{enumerate}
  \setcounter{enumi}{2}
\item Würden Sie die Axt mit zwei sechseitigen Würfeln (2d6) als Schaden oder das Schwert mit einem sechseitigen Würfel plus 5 (1d6+5) als Schaden bevorzugen? Begründen Sie Ihre Antwort mathematisch! \textbf{(1 Punkt)}
\end{enumerate}

Jetzt wird es immer wilder, da Nilufar und Paula sich jetzt überlegen müssen, wie wahrscheinlich es ist, dass der Rettungswurf gegen den zaubernden Hexer funktioniert. Nilufar und Paula haben folgende Wahrscheinlichkeiten gegeben. Die Wahrscheinlichkeit für das Ereignis $A$, der Rettungswurf ist erfolgreich, ist $Pr(A) = 0.65$, die Wahrscheinlichkeit für das Ereignis $B$, der Zauberwurf des Hexers ist erfolgreich, ist $Pr(B) = 0.7$. Sie haben aber mitgezählt und festgestellt, dass in $40$ von 100 Fällen der Rettungswurf bei einem erfolgeichen Zauber funktioniert hat.  

\begin{enumerate}
  \setcounter{enumi}{3}
\item Erstellen Sie eine 2x2 Kreuztabelle mit den Ereignissen $A$ und $B$ sowie den Gegenereignissen $\bar{A}$ und $\bar{B}$ mit einen $\Omega = 100$. Beachten Sie hierbei die entsprechenden Wahrscheinlichkeiten für die Ereignisse $A$ und $B$! \textbf{(2 Punkte)}
\item Bestimmen Sie $Pr(A \cap B)$! \textbf{(1 Punkt)}
\item Erstellen Sie ein Baumdiagramm mit den passenden Informationen aus der 2x2 Kreuztabelle! \textbf{(2 Punkte)}
\item Bestimmen Sie Wahrscheinlichkeit $Pr(A|B)$, dass Ihr Rettungswurf gelingt, wenn der Hexer erfolgreich gezaubert hat! \textbf{(1 Punkt)}
\end{enumerate}

  
\clearpage
% -----------------------------------------------------------------------

\section{Aufgabe \hfill (12 Punkte)}

\textit{Geben Sie grundsätzlich Formeln und Rechenweg zur Lösung der Teilaufgaben mit an!} \\[1Ex]
 

 
%% --------------------------------------------------------------------
\ifcollection
\begin{flushright}
\tiny\vspace{-3Ex}
\textbf{\examinhaltstart}
\exammodulemathstat
\vspace{-4Ex}
\end{flushright}
\begin{minipage}[t]{0.5\textwidth}
\includegraphics[width = 1.3cm]{/Users/kruppajo/work/GitHub/exam/avatare/Alex.png}\hspace{-4mm}\includegraphics[width = 1.3cm]{/Users/kruppajo/work/GitHub/exam/avatare/Paula.png}\hspace{-4mm}\includegraphics[width = 1.3cm]{/Users/kruppajo/work/GitHub/exam/avatare/Steffen.png}
\end{minipage}
\begin{minipage}[t]{0.5\textwidth}
\hfill
\href{https://youtu.be/8Pb2sKUIMyk}{\includegraphics[width = 2cm]{img/youtube}}
\end{minipage}
\fi
%% --------------------------------------------------------------------



\ifcollection
\paragraph{Retrocheck im TV}
\fi




Das war zu viel für Alex gestern. Die Lerngruppe mit Steffen und Paula ging viel zu lang. Während er wegdämmert, kommen in ihm seltsame Bilder hoch. 'Und hier ist sie wieder, die Show der fantastischen Preise. Seien Sie mit dabei, wenn es wieder heißt: Der Preis ist heiß!', ertönt es und Alex fragt sich, ob er nicht doch lieber bezahlter Gast bei Barbara Salesch hätten sein sollten. Aber Alex braucht das Geld und jetzt heißt es Spielschows farmen! Erstmal eine Kaffemaschine von Mitropa gewinnen. Ein Kandidat gewinnt die Kaffeemaschine von Mitropa, wenn nicht alle Kandidaten überbieten (eng. \textit{outbid}). Mit Ihnen bilden Steffen und Paula das Team der drei Kandidaten. Alex braucht dringend Ihre Hilfe in seinen Wahnträumen. Sie gehen wie in \textit{Interception} rein!\footnote{\href{https://www.youtube.com/watch?v=3RhxyHMs-w8}{South Park Inception Spoof} -- Wunderbare South Park Folge}

\begin{center}
\begin{tabular}{ccc}
  \toprule
  Name & $P(win)$ & $P(outbid)$\\
  \midrule
  Steffen & 0.3 & 0.11\\
  Paula & 0.3 & 0.076 \\
  \bottomrule
\end{tabular}
\end{center}

\begin{enumerate}
\item Mit welcher Wahrscheinlichkeit gewinnen Sie die Kaffeemaschine von Mitropa, wenn keiner der Kandidaten überbietet? \textbf{(1 Punkt)}
\item Wenn Ihre überbietungswahrscheinlichkeit $P(outbid)$ bei 0.043 liegt, mit welcher Wahrscheinlichkeit gewinnt \textit{keiner} die Kaffeemaschine von Mitropa? \textbf{(1 Punkt)}
\end{enumerate}

Glücksrad für Arme auf der Kirmes! Leider hat es für Maren Gilzer nicht gereicht. Deshalb sind Sie jetzt mit Alex auf der Kirmes und spielen mit Catwoman um das große Geld. Das Glücksrad hat 24 Felder. Sie beide drehen das Glücksrad zweimal. Auf 8 Feldern gewinnen Alex und Sie 4000EUR sonst 1500EUR. Ganz schön viel Geld und ganz schön zwielichtig hier...

\begin{enumerate}
  \setcounter{enumi}{2}  
\item Skizzieren Sie das Glücksrad und ergänzen Sie die Wahrscheinlichkeiten! \textbf{(1 Punkt)}
\item Zeichnen Sie das zugehörige Baumdiagramm für das zweimalige Drehen! Ergänzen Sie die Wahrscheinlichkeiten und die entsprechenden Ereignisse \textbf{(2 Punkte)}
\item Mir welcher Wahrscheinlichkeit gewinnen Sie 5500EUR? \textbf{(1 Punkt)}
\end{enumerate}

Im Fiebertraum von Alex reisen sie beide im Zug nach Köln um bei >>Geh aufs Ganze!<< mitzuspielen. Alex und Sie schaffen es tatsächlich ins Finale und können als Hauptgewinn ein Auto hinter einer der drei Türen gewinnen. Und was braucht man mehr als ein Auto in einem Fiebertraum? 

\begin{enumerate}
  \setcounter{enumi}{5}  
\item Bevor die Show beginnt, wird das Auto hinter eine zufällig bestimmte Tür gestellt. Mit welcher Wahrscheinlichkeit wird jeweils eine der drei Türen ausgewählt? Zeichnen Sie ein Baumdiagramm! \textbf{(1 Punkt)}
\item Mit welcher Wahrscheinlichkeit wählen Sie sofort die Tür mit dem Auto? Erweitere Sie das Baumdiagramm entsprechend! \textbf{(1 Punkt)}
\item Der Moderator öffnet nun eine der nicht gewählten Türen, aber natürlich nicht die mit dem Auto. Mit welcher Wahrscheinlichkeit steht das Auto hinter der anderen Tür? Erweitern Sie das Baumdiagramm entsprechend! \textbf{(2 Punkte)}
\item Lösen Sie nun das >>Ziegenproblem<<! Berechne Sie dazu die Wahrscheinlichkeiten der einzelnen Pfade. Lohnt sich ein Wechsel der anfangs gewählte Tür? Begründen Sie Ihre Antwort mathematisch! \textbf{(2 Punkte)}
\end{enumerate}
 

%https://unterrichten.zum.de/wiki/Laplace-Wahrscheinlichkeit_wiederholen_und_vertiefen/Ziegen

%Buchstabenhäufigkeit im Deutschen

%%https://www.mathelike.de/abiturloesungen-mathematik-bayern/mathematik-abitur-bayern-2020/pruefungsteil-a/stochastik-2/teilaufgabe-a.html

%%https://de.wikipedia.org/wiki/Buchstabenh%C3%A4ufigkeit


 
\clearpage
% -----------------------------------------------------------------------
\part{Angewandte Nutztier- und Pflanzenwissenschaften (M.Sc.)}
% -----------------------------------------------------------------------  

\section{Aufgabe \hfill (6 Punkte)}
Vergleichen Sie die Standardabweichung mit dem Standardfehler und grenzen
Sie die beiden Kennzahlen voneinander ab.



\section{Aufgabe \hfill (8 Punkte)}
Ihnen liegt folgendes Varianzanalysemodell mit der {\"u}blichen Beschreibung
zur Auswertung des Merkmals fett- und eiwei{\ss}korrigierte Milchleistung pro
Kuh und Jahr in kg vor:

\begin{equation*}
  Y_{ijkl} = \mu + Var_i + EKA_j + VarEKA_{ij} + V_k + b(L_{ij} - L) + e_{ijkl}
\end{equation*}

mit

\begin{itemize}
\item $Y_{ijkl}$: l-te Beobachtung
\item $\mu$: Populationsmittel 
\item $Var_i$: fixer Effekt der i-ten Variante (i: Kontrolle,
  Versuchsgruppe 1, Versuchsgruppe 2)
\item $EKA_j$: fixer Effekt der j-ten Erstkalbealtergruppe (j: EKA $\leq$ 25
  Monate, EKA > 25 Monate)
\item $VarEKA_{ij}$: fixer Effekt der Interaktion Variante x
  Erstkalbealtergruppe 
\item $V_k$: zuf{\"a}lliger Effekt des Vaters 
\item $b(L_{ij} - L)$: lineare Kovariable Laktationsnummer 
\item $e_{ijkl}$: zuf{\"a}lliger
  Restfehler
\end{itemize}
Erl{\"a}utern Sie anhand dieses Beispiels die Begriffe fixer Effekt,
Interaktion, zuf{\"a}lliger Effekt und Kovariable und grenzen Sie diese
Begriffe voneinander ab.



\section{Aufgabe \hfill (6 Punkte)}
Wie bestimmen Sie die richtige Stichprobengr{\"o}{\ss}e? Welche Kennzahlen /
statistische Ma{\ss}zahlen ben{\"o}tigen Sie dabei und nennen Sie die
Voraussetzungen.

 
\clearpage
% -----------------------------------------------------------------------
\part{Platzhalter}

\section{Hausarbeit \hfill (10 Punkte)}

%% --------------------------------------------------------------------
\ifcollection
\begin{flushright}
\tiny\vspace{-3Ex}
\textbf{\examinhaltstart}
\exammodulestat $\;\bullet$
\exammodulestatversuch $\;\bullet$
\exammodulelanddaten 
\end{flushright}
\hfill
\href{https://youtu.be/GNoORO7GfAE}{\includegraphics[width = 2cm]{img/caution}}
\fi
%% --------------------------------------------------------------------

\begin{graybox}{}
Teil der Prüfungsleistung in dem Modul ist eine Hausarbeit zur Datenanalyse in R.\\

Diese Aufgabe dient in der Klausur als Platzhalter für die erreichten Punkte in der Hausarbeit.\\

Zum Bestehen des Moduls ist die Abgabe des Berichts \underline{nicht} notwendig.\\

Alle Informationen zu der Hausarbeit sowie dem Stichtag der Abgabe finden Sie auf ILIAS.
\end{graybox}
 
\clearpage
% -----------------------------------------------------------------------
\part{Lösungen}

Die folgenden Lösungen sind absolut ein Entwurf und können Fehler enthalten. Daher immer mit Vorsicht genießen. Im Weiteren ist das Runden so eine Sache. Deshalb wird es vermutlich auch Abweichungen zu deiner händischen Berechnung geben. Damit leben wir dann aber alle.

% ----------------------------------------------------------------------- 
\section*{Statistik}
% ----------------------------------------------------------------------- 

\begin{verbatim}
library(tidyverse)
data_tbl <- read.table(file = pipe("pbpaste"), sep = " ", header = TRUE)
\end{verbatim}

% ----------------------------------------------------------------------- 
\section*{Mathematik}
% ----------------------------------------------------------------------- 
\subsection*{Herodot – der Schimmel aus Ivenack}
\begin{knitrout}
\definecolor{shadecolor}{rgb}{0.969, 0.969, 0.969}\color{fgcolor}\begin{table}[!h]
\centering\begingroup\fontsize{10}{12}\selectfont

\begin{tabular}{ccccc}
\toprule
\textbf{1)} & \textbf{2)} & \textbf{3)} & \textbf{4)} & \textbf{5)}\\
\midrule
$d_{1820} = 419.93cm$ & r = 0 + 3mm * Baumalter & $V_{eff} = 1.94m^3$ & $s_{Eiche} = 89.96cm$ & $s_{Platz} = 74.96cm$\\
\bottomrule
\end{tabular}
\endgroup{}
\end{table}

\end{knitrout}
% ----------------------------------------------------------------------- 
\subsection*{Von Töpfen auf Tischen}
\begin{knitrout}
\definecolor{shadecolor}{rgb}{0.969, 0.969, 0.969}\color{fgcolor}\begin{table}[!h]
\centering\begingroup\fontsize{10}{12}\selectfont

\begin{tabular}{ccccc}
\toprule
\textbf{2)} & \textbf{3a)} & \textbf{3b)} & \textbf{4)} & \textbf{5)}\\
\midrule
$n = 378$ & $A_{Tisch} = 1.52m^2$ & $A_{Topf} = 0.99m^2$ & $V_{Topf} = 237.5cm^3$ & $61.75EUR$\\
\bottomrule
\end{tabular}
\endgroup{}
\end{table}

\end{knitrout}
% ----------------------------------------------------------------------- 
\subsection*{Solar- \& Biogasanlagen}
\begin{knitrout}
\definecolor{shadecolor}{rgb}{0.969, 0.969, 0.969}\color{fgcolor}\begin{table}[!h]
\centering\begingroup\fontsize{10}{12}\selectfont

\begin{tabular}{cccc}
\toprule
\textbf{2)} & \textbf{3)} & \textbf{4)} & \textbf{5)}\\
\midrule
$A_{Dach} = 616.12m^2$ & $\rho = 245kg/m^3$ & $V = 36.73m^3$ & $h = 11.69m$\\
\bottomrule
\end{tabular}
\endgroup{}
\end{table}

\end{knitrout}
% ----------------------------------------------------------------------- 
\subsection*{Aligatorenbirnen und Blaubeere}
\begin{knitrout}
\definecolor{shadecolor}{rgb}{0.969, 0.969, 0.969}\color{fgcolor}\begin{table}[!h]
\centering\begingroup\fontsize{10}{12}\selectfont

\begin{tabular}{cc}
\toprule
\textbf{1)} & \textbf{2)}\\
\midrule
$9.95EUR$ & $n = 90$\\
\bottomrule
\end{tabular}
\endgroup{}
\end{table}

\begin{table}[!h]
\centering\begingroup\fontsize{10}{12}\selectfont

\begin{tabular}{cccc}
\toprule
\textbf{3)} & \textbf{Wasserverbrauch} & \textbf{Mittleres Gewicht} & \textbf{Liter pro Stück}\\
\midrule
Strauchtomate & $0.19l/g$ & $107.5g$ & $20.425l$\\
Salat & $0.13l/g$ & $410g$ & $53.3l$\\
Avocado & $1.1l/g$ & $280g$ & $308l$\\
Blaubeere & $0.82l/g$ & $3.35g$ & $2.747l$\\
\bottomrule
\end{tabular}
\endgroup{}
\end{table}

\begin{table}[!h]
\centering\begingroup\fontsize{10}{12}\selectfont

\begin{tabular}{ccccc}
\toprule
\textbf{4a)} & \textbf{4b)} & \textbf{5a)} & \textbf{5b)} & \textbf{6)}\\
\midrule
$V = 603749600hl$ & $V = 2468290000hl$ & 4$\times$ & 0.67$\times$ & FAO Stat, Our World in Data\\
\bottomrule
\end{tabular}
\endgroup{}
\end{table}

\end{knitrout}
% ----------------------------------------------------------------------- 
\subsection*{Die Dampfnudelerde}
\begin{knitrout}
\definecolor{shadecolor}{rgb}{0.969, 0.969, 0.969}\color{fgcolor}\begin{table}[!h]
\centering\begingroup\fontsize{10}{12}\selectfont

\begin{tabular}{cccccc}
\toprule
\textbf{1)} & \textbf{2)} & \textbf{3)} & \textbf{4)} & \textbf{5)} & \textbf{6)}\\
\midrule
$h = 3.27 
                                            \cdot 11.5 = 37.61m$ & $V_{T} = 27.14m^3$ & $n = 1927$ & $r = 216.59m$ & $b' = 201.8m$ & $66.54Ellen$\\
\bottomrule
\end{tabular}
\endgroup{}
\end{table}

\end{knitrout}
% ----------------------------------------------------------------------- 
\subsection*{Pyramiden bauen}
\begin{knitrout}
\definecolor{shadecolor}{rgb}{0.969, 0.969, 0.969}\color{fgcolor}\begin{table}[!h]
\centering\begingroup\fontsize{10}{12}\selectfont

\begin{tabular}{cccccc}
\toprule
\textbf{1)} & \textbf{2)} & \textbf{3)} & \textbf{4)} & \textbf{5)} & \textbf{6)}\\
\midrule
$h = 3.27 
                                            \cdot 11.5 = 37.61m$ & $V_{T} = 27.14m^3$ & $n = 1927$ & $r = 216.59m$ & $b' = 201.8m$ & $66.54Ellen$\\
\bottomrule
\end{tabular}
\endgroup{}
\end{table}

\end{knitrout}
% ----------------------------------------------------------------------- 
\end{document}
% -----------------------------------------------------------------------


  
