\documentclass[a4paper, 10pt]{scrartcl}\usepackage[]{graphicx}\usepackage[]{xcolor}
% maxwidth is the original width if it is less than linewidth
% otherwise use linewidth (to make sure the graphics do not exceed the margin)
\makeatletter
\def\maxwidth{ %
  \ifdim\Gin@nat@width>\linewidth
    \linewidth
  \else
    \Gin@nat@width
  \fi
}
\makeatother

\definecolor{fgcolor}{rgb}{0.345, 0.345, 0.345}
\newcommand{\hlnum}[1]{\textcolor[rgb]{0.686,0.059,0.569}{#1}}%
\newcommand{\hlstr}[1]{\textcolor[rgb]{0.192,0.494,0.8}{#1}}%
\newcommand{\hlcom}[1]{\textcolor[rgb]{0.678,0.584,0.686}{\textit{#1}}}%
\newcommand{\hlopt}[1]{\textcolor[rgb]{0,0,0}{#1}}%
\newcommand{\hlstd}[1]{\textcolor[rgb]{0.345,0.345,0.345}{#1}}%
\newcommand{\hlkwa}[1]{\textcolor[rgb]{0.161,0.373,0.58}{\textbf{#1}}}%
\newcommand{\hlkwb}[1]{\textcolor[rgb]{0.69,0.353,0.396}{#1}}%
\newcommand{\hlkwc}[1]{\textcolor[rgb]{0.333,0.667,0.333}{#1}}%
\newcommand{\hlkwd}[1]{\textcolor[rgb]{0.737,0.353,0.396}{\textbf{#1}}}%
\let\hlipl\hlkwb

\usepackage{framed}
\makeatletter
\newenvironment{kframe}{%
 \def\at@end@of@kframe{}%
 \ifinner\ifhmode%
  \def\at@end@of@kframe{\end{minipage}}%
  \begin{minipage}{\columnwidth}%
 \fi\fi%
 \def\FrameCommand##1{\hskip\@totalleftmargin \hskip-\fboxsep
 \colorbox{shadecolor}{##1}\hskip-\fboxsep
     % There is no \\@totalrightmargin, so:
     \hskip-\linewidth \hskip-\@totalleftmargin \hskip\columnwidth}%
 \MakeFramed {\advance\hsize-\width
   \@totalleftmargin\z@ \linewidth\hsize
   \@setminipage}}%
 {\par\unskip\endMakeFramed%
 \at@end@of@kframe}
\makeatother

\definecolor{shadecolor}{rgb}{.97, .97, .97}
\definecolor{messagecolor}{rgb}{0, 0, 0}
\definecolor{warningcolor}{rgb}{1, 0, 1}
\definecolor{errorcolor}{rgb}{1, 0, 0}
\newenvironment{knitrout}{}{} % an empty environment to be redefined in TeX

\usepackage{alltt}
\usepackage[ngerman]{babel}
% -----------------------------------------------------------------------

% -----------------------------------------------------------------------
%% ------------------------------------------------------------
%% by J.Kruppa on Friday, February 11, 2022 (11:31)
%% \def\mainDir{\Sexpr{exam_path}}
\def\source{/Users/jokruppa/source/tex}
\usepackage[margin=2cm, includefoot]{geometry}
\setlength{\parindent}{0cm}
\usepackage{booktabs}
\usepackage{amsmath}
\usepackage{scalerel,amssymb}
\usepackage{setspace}
\def\csquare{{\Large $\boxtimes$}}
\def\msquare{{\Large $\square$}}
\usepackage[normalem]{ulem}
\usepackage{array}
\usepackage{xcolor}
\usepackage{float}
\usepackage{currfile}
\usepackage{tikz}
\usepackage[nomessages]{fp}

%% beamer defs
\def\lecture{Klausurfragen der Bio Data Science}

%% exam defs
\def\examtitle{\lecture}
\def\exammodule{
\vspace{-1.75cm}  
\begin{graybox}{}
\vspace{2Ex}
\textbf{\large Name:} \rule[0ex]{16.75em}{.4pt}
\hfill \textnormal{\textit{Nicht bestanden:}} \msquare \\[2.5Ex]
\textbf{\large Vorname:} \rule[0ex]{15em}{.4pt} \\[2.5Ex]
\textbf{\large Matrikelnummer:} \rule[0ex]{10.8em}{.4pt}
\hfill Endnote: \rule[0ex]{7em}{.4pt} 
\end{graybox}
\vspace{3Ex}
\phantom{text}
}
\def\examsemester{Sommersemester \& Wintersemester}
\def\examdate{\today}
%% ------------------------------------------------------------
\definecolor{darkblue}{rgb}{0,0,.5}
\definecolor{darkpurple}{rgb}{0.4117, 0.2, 0.4117}
\definecolor{uni}{rgb}{0,0.3137,0.6078}
\definecolor{gray}{gray}{0.7}

\usepackage{tcolorbox}
\definecolor{logo1}{RGB}{0, 158, 227}
\definecolor{gray5}{RGB}{247, 247, 247}
\definecolor{gray2}{RGB}{102, 102, 102}

\newtcolorbox{graybox}[1]{
  colback=gray5,%%red!5!white,
  colframe=gray2,%%red!75!black,
  fonttitle=\bfseries\Large,
  %%valign=center,
  fontupper=\large,
  before skip=10pt plus 2pt,
  after skip=20pt plus 4pt,
  title=#1}

\newtcolorbox{takehomebox}[1]{
  colback=gray5,%%red!5!white,
  colframe=logo1,%%red!75!black,
  fonttitle=\bfseries\Large,
  %%valign=center,
  fontupper=\large,
  before skip=10pt plus 2pt,
  after skip=10pt plus 2pt,
  title=#1}

\def\Rlogo{\includegraphics[width = 0.5cm]{\string~/Documents/GitHub/exam/img/Rlogo}\;}

\usepackage[scaled=.90]{helvet} 
\usepackage{fancyhdr}
\usepackage{lastpage}
\usepackage{hyperref}
\hypersetup{
    colorlinks=true,       % false: boxed links; true: colored links
    linkcolor=black,          % color of internal links 
    urlcolor=magenta           % color of external links
}
\renewcommand{\familydefault}{\sfdefault}

\title{
\large \exammodule \\[5Ex]
\Huge \examtitle \\[2Ex] 
\Large Hochschule Osnabr{\"u}ck
}
\author{Pr{\"u}fer: Prof. Dr. Jochen Kruppa \\
Fakult{\"a}t f{\"u}r Agrarwissenschaften und Landschaftsarchitektur \\ 
j.kruppa@hs-osnabrueck.de}
\date{Version vom \examdate}

%% ------------------------------------------------------------
%% by J.Kruppa on Tuesday, September 23, 2014 (12:50)
%% Header
\renewcommand{\headrulewidth}{0pt}
\renewcommand{\footrulewidth}{0pt}
\pagestyle{fancy}

\fancyhf{}
\fancyhead[L]{}
\fancyhead[R]{}
\fancyfoot[R]{\thepage}
\fancyfoot[L]{\footnotesize \examtitle}

\fancypagestyle{empty}{
 \fancyhf{}
 \fancyhead[L]{}
 \fancyhead[R]{}
 \fancyfoot[R]{\thepage}
 \fancyfoot[L]{\footnotesize \examtitle}
}

\usepackage{arevtext,arevmath}

\newcommand\Tstrut{\rule{0pt}{2.6ex}}         % = `top' strut
\newcommand\Bstrut{\rule[-0.9ex]{0pt}{0pt}}   % = `bottom' strut
\def\strut{\Tstrut\Bstrut}

% -----------------------------------------------------------------------
\IfFileExists{upquote.sty}{\usepackage{upquote}}{}
\begin{document}
% -----------------------------------------------------------------------

\begin{graybox}{Checkbox für die Version vom \today}
  \Large Die gesamte Klausur beinhaltet aktuell in Summe
  \textbf{116}
  Fragen.\\[1Ex]
  Davon sind \textbf{40} Multiple
  Choice Fragen sowie \textbf{76} Rechen- und
  Textaufgaben.
\end{graybox}
  
% -----------------------------------------------------------------------
\maketitle
\thispagestyle{empty}
\clearpage
% -----------------------------------------------------------------------

\begin{graybox}{Erlaubte Hilfsmittel für die Klausur}
  \vspace{1Ex}
  \begin{itemize}
  \item Normaler Taschenrechner ohne Möglichkeit der Kommunikation mit anderen
    Geräten - also ausdrücklich kein Handy!
  \item Eine DIN A4-Seite als beidseitig, selbstgeschriebene,
    handschriftliche Formelsammlung - keine digitalen Ausdrucke. 
  \end{itemize}
\end{graybox}
\vfill

\begin{graybox}{Ergebnis der Klausur}
  \vspace{1Ex}
  \begin{itemize}
  \item[] \rule[0ex]{3em}{.4pt}\, von 20\, Punkten sind aus dem Multiple
    Choice Teil erreicht.
  \item[] \rule[0ex]{3em}{.4pt}\, von 62 Punkten sind aus dem Rechen- und
    Textteil erreicht. 
  \item[] \rule[0ex]{3em}{.4pt}\, von 82 Punkten in Summe.
  \item[] Es wird folgender Notenschlüssel angewendet.   
  \end{itemize}
  \vspace{1ex}
\begin{center}
  \begin{tabular}[c]{cc}
    \toprule
    \textbf{Punkte}	&	\textbf{Note}	\\
    \midrule
    78.5 - 82.0	&	1,0	\\
    74.5 - 78.0	&	1,3	\\
    70.0 - 74.0	&	1,7	\\
    66.0 - 69.5	&	2,0	\\
    62.0 - 65.5	&	2,3	\\
    58.0 - 61.5	&	2,7	\\
    54.0 - 57.5	&	3,0	\\
    49.5 - 53.5	&	3,3	\\
    45.5 - 49.0	&	3,7	\\
    41.0 - 45.0	&	4,0	\\
    \bottomrule
  \end{tabular}
\end{center}
  \vspace{1ex}
\begin{itemize}
\item[] Es ergibt sich eine Endnote von \rule[0ex]{4em}{.4pt}.
\end{itemize}
  \vspace{1Ex}
\end{graybox}

% -----------------------------------------------------------------------
\newpage
% -----------------------------------------------------------------------

\begin{graybox}{Multiple Choice Aufgaben}
  \begin{itemize}
  \item Pro Multipe Choice Frage ist \emph{genau} eine Antwort richtig.
  \item \textbf{Übertragen Sie Ihre Kreuze in die Tabelle auf
      dieser Seite.}
  \item Es werden nur Antworten berücksichtigt, die in dieser Tabelle
    angekreuzt sind!
  \end{itemize}

\begin{center}
  \large
  \begin{tabular}{|r|c|c|c|c|c||c|}
    \hline
    & \textbf{A} & \textbf{B} & \textbf{C} & \textbf{D} & \textbf{E} & $\boldsymbol{\checkmark}$\strut\\
    \hline
    1 Aufgabe &   &   &   &   &   & \strut\\
    \hline
    2 Aufgabe &   &   &   &   &   & \strut\\
    \hline
    3 Aufgabe &   &   &   &   &   & \strut\\
    \hline
    4 Aufgabe &   &   &   &   &   & \strut\\
    \hline
    5 Aufgabe &   &   &   &   &   & \strut\\
    \hline
    6 Aufgabe &   &   &   &   &   & \strut\\
    \hline
    7 Aufgabe &   &   &   &   &   & \strut\\
    \hline
    8 Aufgabe &   &   &   &   &   & \strut\\
    \hline
    9 Aufgabe &   &   &   &   &   & \strut\\
    \hline
    10 Aufgabe &   &   &   &   &   & \strut\\
    \hline
  \end{tabular}
\end{center}

\begin{itemize}
\item Es sind \rule[0ex]{2em}{.4pt}\, von 20 Punkten erreicht worden.
\end{itemize}
\end{graybox}

\vfill

\begin{graybox}{Rechen- und Textaufgaben}
  \begin{itemize}
  \item Die Tabelle wird vom Dozenten ausgefüllt.
  \end{itemize}
  \begin{center}
    \large
    \begin{tabular}{|l|c|c|c|c|c|c|c|}
      \hline
      \textbf{Aufgabe} & 11 & 12 & 13 & 14 & 15 & 16 & 17 \strut\\
      \hline
      \textbf{Punkte} & 
      \hspace{1Ex}\Large\textcolor{gray!70}{6}\hspace{1Ex}  & 
      \hspace{1Ex}\Large\textcolor{gray!70}{9}\hspace{1Ex}  & 
      \hspace{1Ex}\Large\textcolor{gray!70}{6}\hspace{1Ex}  & 
      \hspace{1Ex}\Large\textcolor{gray!70}{10}\hspace{1Ex}  & 
      \hspace{1Ex}\Large\textcolor{gray!70}{11}\hspace{1Ex}  & 
      \hspace{1Ex}\Large\textcolor{gray!70}{12}\hspace{1Ex}  & 
      \hspace{1Ex}\Large\textcolor{gray!70}{8}\hspace{1Ex} \strut\\
      \hline
  \end{tabular}
\end{center}
\begin{itemize}
\item Es sind \rule[0ex]{2em}{.4pt}\, von 62 Punkten erreicht worden.
\end{itemize}
\end{graybox}

% -----------------------------------------------------------------------
\clearpage
% -----------------------------------------------------------------------
\begin{graybox}{Multiple Choice Aufgaben}
  \begin{itemize}
  \item Es wird nie mehr als vierzig Multiple Choice Fragen geben.
  \item Im Laufe der Zeit werden einzelne Fragen durch andere Fragen \textit{ersetzt}, bitte beachten Sie diesen Sachstand, wenn Sie eine \textit{Wiederholungsklausur} im nächsten Semester schreiben. 
  \end{itemize}
\end{graybox}
    

\section{Aufgabe \hfill (2 Punkte)}



Sie haben folgende unadjustierten p-Werte gegeben: 0.01, 0.34, 0.21 und 0.89. Sie adjustieren die p-Werte nach
Bonferroni. Welche Aussage ist richtig?



\begin{enumerate}
\item [\textbf{A} \msquare] Nach der Bonferroni-Adjustierung ergeben sich die adjustierten p-Werte von 0.0025, 0.085, 0.0525 und 0.2225. Die adjustierten p-Werte werden zu einem $\alpha$-Niveau von 5\% verglichen.
\item [\textbf{B} \msquare] Nach der Bonferroni-Adjustierung ergeben sich die adjustierten p-Werte von 0.0025, 0.085, 0.0525 und 0.2225. Die adjustierten p-Werte werden zu einem $\alpha$-Niveau von 1.25\% verglichen.
\item [\textbf{C} \msquare] Nach der Bonferroni-Adjustierung ergeben sich die adjustierten p-Werte von 0.04, 1, 0.84 und 1. Die adjustierten p-Werte werden zu einem $\alpha$-Niveau von 5\% verglichen.
\item [\textbf{D} \msquare] Nach der Bonferroni-Adjustierung ergeben sich die adjustierten p-Werte von 0.04, 1, 0.84 und 1. Die adjustierten p-Werte werden zu einem $\alpha$-Niveau von 1.25\% verglichen.
\item [\textbf{E} \msquare] Nach der Bonferroni-Adjustierung ergeben sich die adjustierten p-Werte von 0.04, 1.36, 0.84 und 3.56. Die adjustierten p-Werte werden zu einem $\alpha$-Niveau von 5\% verglichen.
\end{enumerate} 

\section{Aufgabe \hfill (2 Punkte)}



Der Datensatz \texttt{PlantGrowth} enth{\"a}lt das Gewicht von Pflanzen, die
unter einer Kontrolle und zwei verschiedenen Behandlungsbedingungen erzielt
wurden. Nach der Berechnung einer einfaktoriellen ANOVA ergibt sich ein
$\eta^2 = 0.22$. Welche Aussage ist richtig?



\begin{enumerate}
\item [\textbf{A} \msquare] Die Berechnung von $\eta^2$ ist ein Wert f{"u}r die Interaktion.
\item [\textbf{B} \msquare] Das $\eta^2$ ist die Korrelation der ANOVA. Mit der Ausnahme, dass 0 der beste Wert ist.
\item [\textbf{C} \msquare] Das $\eta^2$ ist ein Wert f{"u}r die G{"u}te der ANOVA. Je kleiner desto besser. Ein $\eta^2$ von 0 bedeutet ein perfektes Modell mit keiner Abweichung. Die Varianz ist null.
\item [\textbf{D} \msquare] Das $\eta^2$ beschreibt den Anteil der Varianz, der von den Behandlungsbedingungen erkl{"a}rt wird. Das $\eta^2$ ist damit mit dem $R^2$ aus der linearen Regression zu vergleichen.
\item [\textbf{E} \msquare] Das $\eta^2$ beschreibt den Anteil der Varianz, der von den Behandlungsbedingungen nicht erkl{"a}rt wird. Somit der Rest an nicht erkl{"a}rbarer Varianz.
\end{enumerate} 

\section{Aufgabe \hfill (2 Punkte)}

Die folgende Abbildung enth{\"a}lt die Daten aus einer Studie zur
Bewertung der Wirkung von Vitamin C auf das Zahnwachstum bei
Meerschweinchen. Der Versuch wurde an 60 Schweinen durchgef{\"u}hrt, wobei
jedes Tier eine von drei Vitamin-C-Dosen (0.5, 1 und 1.5 mg/Tag) {\"u}ber eine
von zwei Verabreichungsmethoden mit Orangensaft (OJ)  oder
Ascorbins{\"a}ure (VC) erhielt. 



{\centering \includegraphics[width=\maxwidth]{img/mc-anova-02-a-1} 

}




Welche Aussage ist richtig im Bezug auf eine zweifaktorielle ANOVA?



\begin{enumerate}
\item [\textbf{A} \msquare] Keine Interaktion ist zu erwatzen. Die Geraden der Verabreichungsmethode laufen parallel und mit {"a}hnlichen Abst{"a}nden.
\item [\textbf{B} \msquare] Eine leichte Interaktion ist zu erwarten. Die Geraen schneiden sich noch nicht, aber die Abst{"a}nde unterscheiden sich stark.
\item [\textbf{C} \msquare] Eine starke Interaktion ist zu erwarten. Die Geraden schneiden sich und die Abst{"a}nde sind nicht gleichbleibend.
\item [\textbf{D} \msquare] Eine starke Interaktion liegt vor. Die Geraden laufen parallel und schneiden sich nicht.
\item [\textbf{E} \msquare] Keine Interaktion liegt vor. Die Geraden scheiden sich und laufen nicht parallel.
\end{enumerate} 

\section{Aufgabe \hfill (2 Punkte)}

Eine einfaktorielle ANOVA berechnet eine Teststatistik um zu die Nullhypothese abzulehnen. Welche Aussage {\"u}ber die Teststatistik der ANOVA ist richtig?



\begin{enumerate}
\item [\textbf{A} \msquare] Die ANOVA berechnet die F-Statistik indem die MS der Behandlung durch die MS des Fehlers geteilt werden. Wenn die F-Statistik sich der 0 ann{"a}hert kann die Nullhypothese nicht abgelehnt werden.
\item [\textbf{B} \msquare] Die ANOVA berechnt die F-Statistik aus den SS Behandlung geteilt durch die SS Fehler.
\item [\textbf{C} \msquare] Die ANOVA berechnet die F-Statistik indem die MS des Fehlers durch die MS der Behandlung geteilt werden. Wenn die F-Statistik sich der 1 ann{"a}hert kann die Nullhypothese nicht abgelehnt werden.
\item [\textbf{D} \msquare] Die ANOVA berechnet die T-Statistik indem den Mittelwertsunterschied der Gruppen simultan durch die Standardabweichung der Gruppen teilt. Wenn die T-Statistik h{"o}her als 1.96 ist, kann die Nullhypothese abgelehnt werden.
\item [\textbf{E} \msquare] Die ANOVA berechnet die T-Statistik aus der Multiplikation der MS Behandlung mit der MS der Fehler. Wenn die F-Statistik genau 0 ist, kann die Nullhypothese abgelehnt werden.
\end{enumerate} 

\section{Aufgabe \hfill (2 Punkte)}




Sie haben das abstrakte Modell $Y \sim X$ mit $X$ als Faktor mit zwei
Leveln vorliegen. Welche Aussage {\"u}ber $s^2_1 \neq s^2_2$ ist richtig?



\begin{enumerate}
\item [\textbf{A} \msquare] Es liegt Varianzhomogenit{"a}t vor.
\item [\textbf{B} \msquare] Es liegt Varianzhetrogenit{"a}t vor.
\item [\textbf{C} \msquare] Es handelt sich um ein unbalanciertes Design
\item [\textbf{D} \msquare] Es handelt sich um abh{"a}ngige Beobachtungen.
\item [\textbf{E} \msquare] Es handelt sich um ein balanciertes Design.
\end{enumerate} 

\section{Aufgabe \hfill (2 Punkte)}

Die Mindestanzahl an Beobachtungen f{\"u}r eine Zelle der Vierfeldertafel bei
der Nutzung eines Chi-Quadrat-Testes ist...



\begin{enumerate}
\item [\textbf{A} \msquare] 1 Beobachtung
\item [\textbf{B} \msquare] 0 Beobachtungen
\item [\textbf{C} \msquare] 2 Beobachtungen
\item [\textbf{D} \msquare] 5 Beobachtungen
\item [\textbf{E} \msquare] 10 Beobachtungen
\end{enumerate} 

\section{Aufgabe \hfill (2 Punkte)}




Welche Aussage {\"u}ber den Korrelationskoeffizienten nach Spearman
ist richtig?



\begin{enumerate}
\item [\textbf{A} \msquare] Der Korrelationskoeffizienten nach Spearman wird genutzt, wenn das Outcome Y normalverteilt ist. Der Korrelationskoeffizienten liegt zwischen -1 und 1.
\item [\textbf{B} \msquare] Der Korrelationskoeffizienten nach Spearman wird genutzt, wenn das Outcome Y normalverteilt ist. Der Korrelationskoeffizienten liegt zwischen 0 und 1.
\item [\textbf{C} \msquare] Der Korrelationskoeffizienten nach Spearman wird genutzt, wenn das Outcome Y nicht normalverteilt ist. Der Korrelationskoeffizienten liegt zwischen -1 und 1.
\item [\textbf{D} \msquare] Der Korrelationskoeffizienten nach Spearman wird genutzt, wenn der Korrelationskoeffizienten zwischen -1 und 1 liegt. Dann sind die Residuen normalverteilt.
\item [\textbf{E} \msquare] Der Korrelationskoeffizienten nach Spearman wird genutzt, wenn das Outcome Y nicht normalverteilt ist. Der Korrelationskoeffizienten liegt zwischen 0 und 1.
\end{enumerate} 

\section{Aufgabe \hfill (2 Punkte)}




Berechnen Sie den Mittelwert und Standardabweichung von $y$ mit 6, 14, 5, 8 und 17.



\begin{enumerate}
\item [\textbf{A} \msquare] Es ergibt sich 9 +/- 13.75
\item [\textbf{B} \msquare] Es ergibt sich 10 +/- 5.24
\item [\textbf{C} \msquare] Es ergibt sich 10 +/- 27.5
\item [\textbf{D} \msquare] Es ergibt sich 11 +/- 2.62
\item [\textbf{E} \msquare] Es ergibt sich 10 +/- 2.62
\end{enumerate} 

\section{Aufgabe \hfill (2 Punkte)}




Berechnen Sie den Median, das $1^{st}$ Quartile sowie das $3^{rd}$ Quartile von $y$ mit 8, 12, 16, 25, 22, 15, 1, 24, 28, 19 und 51.



\begin{enumerate}
\item [\textbf{A} \msquare] Es ergibt sich 20 +/- 12
\item [\textbf{B} \msquare] Es ergibt sich 19 +/- 25
\item [\textbf{C} \msquare] Es ergibt sich 19 [12, 25]
\item [\textbf{D} \msquare] Es ergibt sich 20 [13, 26]
\item [\textbf{E} \msquare] Es ergibt sich 19 +/- 12
\end{enumerate} 

\section{Aufgabe \hfill (2 Punkte)}

Welche Aussage {\"u}ber Cook's d und Cohen's d ist richtig? 



\begin{enumerate}
\item [\textbf{A} \msquare] Wir nutzen Cook's d um Outlier in den Daten zu finden. Cohen's d findet auch Outlier, ist aber ein veraltetetes Konzept in der Statistik.
\item [\textbf{B} \msquare] Wir nutzen Cook's d um Outlier in den Daten zu finden und Cohen's d um standardisierte Outlier f{"u}r Gruppenvergeliche zu erhalten.
\item [\textbf{C} \msquare] Wir nutzen Cook's d um Outlier in den Daten zu finden und Cohen's d um einen standardisierten Effektsch{"a}tzer f{"u}r Gruppenvergeliche zu erhalten.
\item [\textbf{D} \msquare] Wir nutzen Cohen's d um Outlier in den Daten zu finden und Cook's d um einen standardisierten Effektsch{"a}tzer f{"u}r Gruppenvergeliche zu erhalten.
\item [\textbf{E} \msquare] Wir nutzen Cook's d um Outlier in den Daten zu finden und Cohen's d um einen nicht standardisierten Effektsch{"a}tzer f{"u}r Gruppenvergeliche zu erhalten.
\end{enumerate}

\section{Aufgabe \hfill (2 Punkte)}



Die empfohlene Mindestanzahl an Beobachtungen f{\"u}r ein Histogramm sind...



\begin{enumerate}
\item [\textbf{A} \msquare] mindestens 20 Beobachtungen.
\item [\textbf{B} \msquare] 2-5 Beobachtungen.
\item [\textbf{C} \msquare] 1 Beobachtung.
\item [\textbf{D} \msquare] 5 und mehr Beobachtungen.
\item [\textbf{E} \msquare] 10 Beobachtungen.
\end{enumerate} 

\section{Aufgabe \hfill (2 Punkte)}



Nachdem Sie in einem Experiment die Daten $D$ erhoben haben, berechnen Sie den
Mittelwert und den Median. Der Mittelwert $\bar{y}$ und der Median
$\tilde{y}$ unterscheiden sich. Welche Aussage ist richtig?




\begin{enumerate}
\item [\textbf{A} \msquare] Da sich der Mittelwert und der Median unterscheiden, liegen vermutlich keine Outlier in den Daten vor.
\item [\textbf{B} \msquare] Da sich der Mittelwert und der Median unterscheiden, liegen vermutlich Outlier in den Daten vor. Wir untersuchen den Datensatz nach auff{"a}lligen Beobachtungen.
\item [\textbf{C} \msquare] Da sich der Mittelwert und der Median nicht unterscheiden, liegen vermutlich keine Outlier in den Daten vor. Wir verweden den Datensatz so wie er ist.
\item [\textbf{D} \msquare] Da sich der Mittelwert und der Median unterscheiden, ist der Datensatz nicht zu verwenden. Mittelwert und Median m{"u}ssen gleich sein.
\item [\textbf{E} \msquare] Da sich der Mittelwert und der Median nicht unterscheiden, liegen vermutlich Outlier in den Daten vor.
\end{enumerate} 

\section{Aufgabe \hfill (2 Punkte)}

Nach einer simplen linearen Regression zur Untersuchung vom Einfluss der
$CO_2$-Konzentration [$\mu g$] im Wasser auf das Wachstum von Wasserlinsen
[$kg$] erhalten Sie einen $\beta_1$ Koeffizienten von $0.00001$ und einen
hoch signifikanten $p$-Wert mit $2.3\cdot 10^{-9}$. Warum sehen Sie so einen kleinen
Effekt bei einer so deutlichen Signifikanz? 




\begin{enumerate}
\item [\textbf{A} \msquare] Das Gewicht und die $CO_2$-Konzentration korrelieren sehr stark, deshalb wird der $\beta_1$ Koeffizient sehr klein.
\item [\textbf{B} \msquare] Die Einheit der $CO_2$-Konzentration ist zu klein gew{"a}hlt. Dadurch sehen wir den sehr kleinen $p$-Wert. Der $p$-Wert und die Einheit von der $CO_2$-Konzentration h{"a}ngen zusammen.
\item [\textbf{C} \msquare] Die Fallzahl ist zu hoch angesetzt. Je h{"o}her die Fallzahl ist, desto kleiner ist die Teststatistik und damit ist dann auch der $p$-Wert sehr klein.
\item [\textbf{D} \msquare] Die Einheit der $CO_2$-Konzentration ist zu klein gew{"a}hlt. Die Erh{"o}hung der $CO_2$-Konzentration um 1 f{"u}hrt nur zu einem sehr winzigen Anstieg im Gewicht der Wasserlinsen. Die Einheit muss besser gew{"a}hlt werden.
\item [\textbf{E} \msquare] Die Fallzahl ist zu klein angesetzt. Je kleiner die Fallzahl ist, desto h{"o}her ist die Teststatsitik und damit auch der $p$-Wert kleiner. 
\end{enumerate} 

\section{Aufgabe \hfill (2 Punkte)}

In dem folgenden Histogramm von $n = 200$ Pflanzen ist welche Verteilung
mit welchen korrekten Verteilungsparametern dargestellt?



{\centering \includegraphics[width=\maxwidth]{img/mc-distribution-02-a-1} 

}







\begin{enumerate}
\item [\textbf{A} \msquare] Eine Standardnormalverteilung mit N(0,1).
\item [\textbf{B} \msquare] Es handelt sich um eine Poisson-Verteilung mit Pois(10).
\item [\textbf{C} \msquare] Eine rechtsschiefe, multivariate Normalverteilung.
\item [\textbf{D} \msquare] Es handelt sich um eine Normalverteilung mit N(10, 5).
\item [\textbf{E} \msquare] Es handelt sich um eine Binomial-Verteilung mit Binom(10).
\end{enumerate} 

\section{Aufgabe \hfill (2 Punkte)}



Sie sch{\"a}tzen zwei lineare Regressionsmodelle zur Analyse von
Z{\"a}hldaten. Modell 1 mit einer Possion Verteilung und Modell 2 mit einer
Quasi-Poisson Verteilung. Welche Aussage zu einer gesch{\"a}tzen Overdispersion
von 3.5 ist richtig?




\begin{enumerate}
\item [\textbf{A} \msquare] Bei einer gesch{"a}tzen Overdispersion h{"o}her als 1.5 ist von Overdispersion in den Daten auszugehen. Daher wird die Varianz systematisch untersch{"a}tzt, was zu h{"o}heren p-Werten f{"u}hrt. Daher gibt es weniger signifikante Ergebnisse als es in Wirklichkeit gibt. Daher ist das Modell 1 die bessere Wahl.
\item [\textbf{B} \msquare] Das vergleichen von verschiedenen Modellen muss erst {"u}ber ein AIC Kriterium erfolgen. Die Absch{"a}tzung {"u}ber die Overdispersion ist nicht notwendig. Die Varianzen werden sp{"a}ter in einer ANOVA adjustiert. Die Confounder Adjustierung.
\item [\textbf{C} \msquare] Bei einer gesch{"a}tzen Overdispersion h{"o}her als 1.5 ist von Overdispersion in den Daten auszugehen. Daher wird die Varianz systematisch untersch{"a}tzt, was zu kleineren p-Werten f{"u}hrt. Daher gibt es mehr signifikante Ergebnisse als es in Wirklichkeit gibt. Daher ist das Modell 2 die bessere Wahl.
\item [\textbf{D} \msquare] Bei einer gesch{"a}tzen Overdispersion h{"o}her als 1.5 ist von keiner Overdispersion in den Daten auszugehen. Dennoch sind die p-Werte zu klein, dass diese p-Werte nat{"u}rlich entstanden sein k{"o}nnten. Die p-Werte m{"u}ssen adjustiert werden.
\item [\textbf{E} \msquare] Bei einer gesch{"a}tzen Overdispersion h{"o}her als 1.5 ist von Overdispersion in den Daten auszugehen. Daher wird die Varianz systematisch {"u}bersch{"a}tzt, was zu h{"o}heren p-Werten f{"u}hrt. Daher gibt es mehr signifikante Ergebnisse als es in Wirklichkeit gibt. Daher ist das Modell 1 die bessere Wahl
\end{enumerate}

\section{Aufgabe \hfill (2 Punkte)}

In einem Zuchtexperiment messen wir die Ferkel verschiedener Sauen. Die
Ferkel einer Muttersau sind daher im statistischen Sinne... 



\begin{enumerate}
\item [\textbf{A} \msquare] Untereinander stark korreliert. Die Ferkel sind von einer Mutter und sommit miteinander korreliert. Dies wird in der Statistik jedoch meist nicht modelliert.
\item [\textbf{B} \msquare] Untereinander unabh{"a}ngig. Die Ferkel sind eigenst{"a}ndig und ben{"o}tigen keine zus{"a}tzliche Behandlung.
\item [\textbf{C} \msquare] Untereinander unabh{"a}ngig. Sollten die M{"u}tter verwandt sein, so ist die Varianzstruktur {"a}hnlich und muss modelliert werden.
\item [\textbf{D} \msquare] Untereinander abh{"a}ngig, wenn die M{"u}tter ebenfalls miteinander verwandt sind. Erst die Abh{"a}ngigkeit 2. Grades wird in der Statistik modelliert.
\item [\textbf{E} \msquare] Untereinander abh{"a}ngig. Die Ferkel stammen von einem Muttertier und haben vermutliche eine {"a}hnliche Varianzstruktur.
\end{enumerate}

\section{Aufgabe \hfill (2 Punkte)}




Sie haben das Modell $Y \sim X$ vorliegen und wollen nun ein
pr{"a}diktives Modell rechnen. Welche Aussage ist richtig?



\begin{enumerate}
\item [\textbf{A} \msquare] Ein pr{"a}diktives Modell wird auf einem Trainingsdatensatz trainiert und anschliessend {"u}ber eine explorative Datenanalyse validiert. Signifikanzen {"u}ber $\beta_i$ k{"o}nnen hier nicht festgestellt werden.
\item [\textbf{B} \msquare] Ein pr{"a}diktives Modell basiert auf einem Traingsdatensatz und einem Testdatensatz. Auf dem Trainingsdatensatz wird das Modell trainiert und auf dem Testdatensatz validiert.
\item [\textbf{C} \msquare] Ein pr{"a}diktives Modell m{"o}chte die Zusammenh{"a}nge von X auf Y modellieren. Hierbei geht es um die Effekte von X auf Y. Man sagt, wenn X um 1 ansteigt {"a}ndert sich Y um einen Betrag $\beta$.
\item [\textbf{D} \msquare] Ein pr{"a}diktives Modell ben{"o}tigt mindestens eine Fallzahl von {"u}ber 100 Beobachtungen und darf keine fehlenden Werte beinhalten. Die Varianzkomponenten m{"u}ssen homogen sein.
\item [\textbf{E} \msquare] Ein pr{"a}diktives Modell schliesst grunds{"a}tzlich lineare Modell aus. Es muss ein Graph gefunden werden, der alle Punkte beinhaltet. Erst dann kann das $R^2$ berechnet werden.
\end{enumerate}

\section{Aufgabe \hfill (2 Punkte)}



Sie f{\"u}hren ein Experiment zur Behandlung von Klaueninfektionen bei K{\"u}hen
durch. Bei 6 Tieren finden Sie eine Erkrankung der Klauen vor und
12 Tiere sind gesund. Welche Aussage {\"u}ber den Odds ratio
Effektsch{\"a}tzer ist richtig?



\begin{enumerate}
\item [\textbf{A} \msquare] Es ergibt sich ein Odds ratio von 0.33, da es sich um ein Anteil handelt.
\item [\textbf{B} \msquare] Es ergibt sich ein Odds ratio von 0.33, da es sich um eine Chancenverh{"a}ltnis handelt.
\item [\textbf{C} \msquare] Es ergibt sich ein Odds ratio von 0.5, da es sich um eine Chancenverh{"a}ltnis handelt.
\item [\textbf{D} \msquare] Es ergibt sich ein Odds ratio von 0.5, da es sich um ein Anteil handelt.
\item [\textbf{E} \msquare] Es ergibt sich ein Odds ratio von 2, da es sich um ein Anteil handelt.
\end{enumerate}

\section{Aufgabe \hfill (2 Punkte)}




Welche Aussage {\"u}ber die parametrische Statistik ist richtig?



\begin{enumerate}
\item [\textbf{A} \msquare] Die nicht-parametrische Statistik ist ein Vorg{"a}nger der parametrischen Statistik und wurde wegen dem Mangel an Effektsch{"a}tzern nicht mehr ab 1960 genutzt.
\item [\textbf{B} \msquare] Die parametrische Statistik basiert auf dem Sch{"a}tzen von Parametern aus einer a priori festgelegten Verteilung. Daher gibt es auch direkt zu interpretierenden Effektsch{"a}tzer.
\item [\textbf{C} \msquare] Die parametrische Statistik basiert auf R{"a}ngen. Daher wird jeder Zahl ein Rang zugeteilt. Nur auf den R{"a}ngen wird die Auswertung gerechnet. Daher gibt es auch keinen direkt zu interpretierenden Effektsch{"a}tzer.
\item [\textbf{D} \msquare] Die parametrische Statistik basiert auf R{"a}ngen. Daher gibt es auch direkt zu interpretierenden Effektsch{"a}tzer.
\item [\textbf{E} \msquare] Die parametrische Statistik basiert auf dem Sch{"a}tzen von Parametern aus einer festgelegten Verteilung. Daher gibt es auch direkt zu interpretierenden Effektsch{"a}tzer.
\end{enumerate}

\section{Aufgabe \hfill (2 Punkte)}

Die Randomisierung von Beobachtungen bzw. Samples zu den Versuchseinheiten
ist bedeutend in der Versuchsplanung. Welche der folgenden Aussagen ist richtig?



\begin{enumerate}
\item [\textbf{A} \msquare] Randomisierung erlaubt erst die Varianzen zu sch{"a}tzen. Ohne eine Randomisierung ist die Berechnung von Mittelwerten und Varianzen nicht m{"o}glich.
\item [\textbf{B} \msquare] Randomisierung bringt starke Unstrukturiertheit in das Experiment und erlaubt erst von der Stichprobe auf die Grundgesamtheit zur{"u}ckzuschliessen.
\item [\textbf{C} \msquare] Randomisierung erlaubt erst die Mittelwerte zu sch{"a}tzen. Ohne Randomisierung keine Mittelwerte.
\item [\textbf{D} \msquare] Randomisierung war bis 1952 bedeutend, wurde dann aber in Folge besserer Rechnerleistung nicht mehr verwendet. Aktuelle Statistik nutzt keine Randomisierung mehr.
\item [\textbf{E} \msquare] Randomisierung sorgt f{"u}r Strukturgleichheit und erlaubt erst von der Stichprobe auf die Grundgesamtheit zur{"u}ckzuschliessen.
\end{enumerate}

\section{Aufgabe \hfill (2 Punkte)}

Wenn Sie einen Datensatz erstellen, dann ist es ratsam die Spalten und die
Eintr{\"a}ge in englischer Sprache zu verfassen, wenn Sie sp{\"a}ter die Daten in
\Rlogo auswerten wollen. Welcher folgende Grund ist richtig?



\begin{enumerate}
\item [\textbf{A} \msquare] Alle Funktionen und auch Anwendungen sind in \Rlogo in englischer Sprache. Die Nutzung von deutschen W{"o}rtern ist nicht schick und das ist zu vermeiden.
\item [\textbf{B} \msquare] Die Spracherkennung von \Rlogo ist nicht in der Lage Deutsch zu verstehen.
\item [\textbf{C} \msquare] Es gibt keinen Grund nicht auch deutsche W{"o}rter zu verwenden. Es ist ein Stilmittel.
\item [\textbf{D} \msquare] Programmiersprachen k{"o}nnen nur englische Begriffe verarbeiten. Zus{"a}tzliche Pakete k{"o}nnen zwar geladen werden, aber meist funktionieren diese Pakete nicht richtig. Deutsch ist International nicht bedeutend genug.
\item [\textbf{E} \msquare] Im Allgemeinen haben Programmiersprachen Probleme mit Umlauten und Sonderzeichen, die in der deutschen Sprache vorkommen. Eine Nutzung der englischen Sprache umgeht dieses Problem auf einfache Art.
\end{enumerate}

\section{Aufgabe \hfill (2 Punkte)}

Bei der explorativen Datenanalyse (EDA) in \Rlogo gibt es eine richtige Abfolge von Prozessschritten, auch \textit{Circle of life} genannt. Wie lautet die richtige Reihenfolge f{\"u}r die Erstellung einer EDA?



\begin{enumerate}
\item [\textbf{A} \msquare] Wir transformieren die Spalten {"u}ber \texttt{mutate()} in ein \texttt{tibble} und k{"o}nnen dann {"u}ber \text{ggplot()} uns die Abbildungen erstellen lassen. Dabei beachten wir das wir keine Faktoren in den Daten haben.
\item [\textbf{B} \msquare] Wir lesen die Daten {"u}ber eine generische Funktion \texttt{read()} ein und m{"u}ssen dann die Funktion \texttt{ggplot()} nur noch installieren. Dann haben wir die Abbildungen als \texttt{*.png} vorliegen.
\item [\textbf{C} \msquare] Wir lesen die Daten ein und mutieren die Daten. Dabei ist wichtig, dass wir nicht das Paket \texttt{tidyverse} nutzen, da dieses Paket veraltet ist. {"U}ber die Funktion \texttt{library(tidyverse)} entfernen wir das Paket von der Analyse.
\item [\textbf{D} \msquare] Wir lesen als erstes die Daten {"u}ber \texttt{read\_excel()} ein, transformieren die Spalten {"u}ber \texttt{mutate()} in die richtige Form und k{"o}nnen dann {"u}ber \text{ggplot()} uns die Abbildungen erstellen lassen.
\item [\textbf{E} \msquare] Wir lesen als erstes die Daten {"u}ber \texttt{read\_excel()} ein, transformieren die Spalten {"u}ber \texttt{mutate()} in die richtige Form und k{"o}nnen dann  {"u}ber \text{ggplot()} uns die Abbildungen erstellen lassen. Wichtig ist, dass wir keine Faktoren sondern nur numerische Variablen vorliegen haben.
\end{enumerate}

\section{Aufgabe \hfill (2 Punkte)}

In einer linearen Regression werden die $\epsilon$ oder Residuen
gesch{\"a}tzt. Welcher Verteilung folgen die Residuen bei einer optimalen
Modellierung? 



\begin{enumerate}
\item [\textbf{A} \msquare] Die Residuen folgen einer Poissonverteilung mit Pois(0).
\item [\textbf{B} \msquare] Die Residuen sind normalverteilt mit $\mathcal{N}(0, 1)$.
\item [\textbf{C} \msquare] Die Residuen sind normalverteilt mit $\mathcal{N}(\bar{y}, s^2)$.
\item [\textbf{D} \msquare] Die Residuen sind normalverteilt mit $\mathcal{N}(0, s^2)$.
\item [\textbf{E} \msquare] Die Residuen sind binomialverteilt.
\end{enumerate}

\section{Aufgabe \hfill (2 Punkte)}

Welche Aussage {\"u}ber das \textit{generalisierte lineare Modell (GLM)} ist richtig?  



\begin{enumerate}
\item [\textbf{A} \msquare] Das GLM ist eine Vereinfachung des LM in R. Mit dem GLM lassen polygonale Regressionen rechnen.
\item [\textbf{B} \msquare] Das GLM erlaubt auch weitere Verteilungsfamilien f{"u}r das Y bzw. das Outcome in einer linearen Regression zu w{"a}hlen.
\item [\textbf{C} \msquare] Das GLM erlaubt auch nicht normalverteilte Residuen in der Sch{"a}tzung der Regressionsgrade.
\item [\textbf{D} \msquare] Das GLM ist eine allgemeine Erweiterung der linearen Regression auf die Normalverteilung.
\item [\textbf{E} \msquare] Das GLM ist ein faktisch maschineller Lernalgorithmus, der selstst{"a}ndig die Verteilungsfamilie f{"u}r Y w{"a}hlt.
\end{enumerate}

\section{Aufgabe \hfill (2 Punkte)}

Sie rechnen in eine linearen Regression und erhalten folgenden QQ
Plot. Welche Aussage ist richtig?




{\centering \includegraphics[width=\maxwidth]{img/mc-regression-05-a-1} 

}







\begin{enumerate}
\item [\textbf{A} \msquare] Die Annahme der normalverteilten Residuen ist erf{"u}llt. Die Punkte liegen zum {"u}berwiegenden Teil nicht auf der Geraden und Korrelation ist negativ.
\item [\textbf{B} \msquare] Die Annahme der normalverteilten Residuen ist erf{"u}llt. Die Punkte liegen zum {"u}berwiegenden Teil auf der Geraden.
\item [\textbf{C} \msquare] Die Annahme der normalverteilten Residuen ist erf{"u}llt. Die Punkte liegen zum {"u}berwiegenden Teil nicht auf der Geraden.
\item [\textbf{D} \msquare] Die Annahme der normalverteilten Residuen ist nicht erf{"u}llt. Die Punkte liegen zum {"u}berwiegenden Teil nicht auf der Geraden.
\item [\textbf{E} \msquare] Die Annahme der normalverteilten Residuen ist nicht erf{"u}llt. Die Punkte liegen zum {"u}berwiegenden Teil auf der Geraden.
\end{enumerate}

\section{Aufgabe \hfill (2 Punkte)}

Sie rechnen eine linearen Regression und erhalten folgenden Residual
Plot. Welche Aussage ist richtig?




{\centering \includegraphics[width=\maxwidth]{img/mc-regression-06-a-1} 

}







\begin{enumerate}
\item [\textbf{A} \msquare] Die Annahme der normalverteilten Residuen ist erf{"u}llt. Die Punkte liegen zum {"u}berwiegenden Teil auf der Diagonalen.
\item [\textbf{B} \msquare] Die Annahme der normalverteilten Residuen ist nicht erf{"u}llt. Vereinzelte Punkte liegen oberhalb bzw. unterhalb der Geraden um die 0 Linie weiter entfernt. Ein klares Muster ist zu erkennen.
\item [\textbf{C} \msquare] Die Annahme der normalverteilten Residuen ist erf{"u}llt. Es ist ein Muster zu erkennen.
\item [\textbf{D} \msquare] Die Annahme der normalverteilten Residuen ist erf{"u}llt. Kein Muster ist zu erkennen und keine Outlier zu beobachten.
\item [\textbf{E} \msquare] Die Annahme der normalverteilten Residuen ist nicht erf{"u}llt. Es ist kein Muster zu erkennen.
\end{enumerate}

\section{Aufgabe \hfill (2 Punkte)}

Welche Aussage zum mathematische Ausdruck $Pr(D|H_0)$ ist richtig? 



\begin{enumerate}
\item [\textbf{A} \msquare] Die Wahrscheinlichkeit der Daten unter der Nullhypothese in der Grundgesamtheit.
\item [\textbf{B} \msquare] Die Inverse der Wahrscheinlichkeit unter der die Nullhypothese nicht mehr die Alternativehypothese {"u}berdeckt.
\item [\textbf{C} \msquare] $Pr(D|H_0)$ ist die Wahrscheinlichkeit die Daten D zu beobachten wenn die Nullhypothese wahr ist.
\item [\textbf{D} \msquare] $Pr(D|H_0)$ ist die Wahrscheinlichkeit der Alternativehypothese und somit $1 - Pr(H_A)$
\item [\textbf{E} \msquare] Die Wahrscheinlichkeit f{"u}r die Nullhypothese, wenn die Daten wahr sind.
\end{enumerate}

\section{Aufgabe \hfill (2 Punkte)}

Das Falsifikationsprinzip besagt... 



\begin{enumerate}
\item [\textbf{A} \msquare] ... dass Fehlerterme in statistischen Modellen nicht verifiziert werden k{"o}nnen.
\item [\textbf{B} \msquare] ... dass Annahmen an statistische Modelle meist falsch sind.
\item [\textbf{C} \msquare] ... dass Modelle meist falsch sind und selten richtig.
\item [\textbf{D} \msquare] ... dass ein schlechtes Modell durch ein weniger schlechtes Modell ersetzt wird. Die Wissenschaft lehnt ab und verifiziert nicht.
\item [\textbf{E} \msquare] ... dass in der Wissenschaft immer etwas falsch sein muss. Sonst gebe es keinen Fortschritt.
\end{enumerate}

\section{Aufgabe \hfill (2 Punkte)}

Der Fehler 1. Art oder auch Signifikanzniveau $\alpha$ genannt, liegt bei
5\%. Welcher der folgenden Gr{\"u}nde f{\"u}r diese Festlegeung auf 5\% ist richtig?



\begin{enumerate}
\item [\textbf{A} \msquare] Im Rahmen eines langen Disputs zwischen Neyman und Fischer wurde $\alpha = 5\%$ festgelegt. Leider werden die Randbedingungen und Voraussetzungen an statistsiche Modelle heute immer wieder ignoriert.
\item [\textbf{B} \msquare] Die Festlegung von $\alpha = 5\%$ ist eine Kulturkonstante. Wissenschaftler ben{"o}tigt eine Schwelle f{"u}r eine statistische Testentscheidung, der Wert von $\alpha$ wurde aber historisch mehr zuf{"a}llig gew{"a}hlt.
\item [\textbf{C} \msquare] Auf einer Statistikkonferenz in Genf im Jahre 1942 wurde dieser Cut-Off nach langen Diskussionen festgelegt. Bis heute ist der Cut Off aber umstritten, da wegen dem 2. Weltkrieg viele Wissenschaftler nicht teilnehmen konnten.
\item [\textbf{D} \msquare] Der Wert ergab sich aus einer Auswertung von 1042 wissenschaftlichen Ver{"o}ffentlichungen zwischen 1914 und 1948. Der Wert $5\%$ wurde in $28\%$ der Ver{"o}ffentlichungen genutzt. Daher legte man sich auf diese Zahl fest.
\item [\textbf{E} \msquare] Der Begr{"u}nder der modernen Statistik, R. Fischer, hat die Grenze simuliert und berechnet. Dadurch ergibt sich dieser optimale Cut-Off.
\end{enumerate}

\section{Aufgabe \hfill (2 Punkte)}

Welche Aussage {\"u}ber die Power ist richtig?



\begin{enumerate}
\item [\textbf{A} \msquare] Es gilt $\alpha + \beta = 1$ und somit liegt $\beta$ meist bei 95\%.
\item [\textbf{B} \msquare] Die Power $1-\beta$ wird auf 80\% gesetzt. Damit liegt die Wahrscheinlichkeit f{"u}r die $H_0$ bei 20\%.
\item [\textbf{C} \msquare] Die Power ist nicht in der aktuellen Testthorie mehr vertreten. Wir rechnen nur noch mit dem Fehler 1. Art.
\item [\textbf{D} \msquare] Die Power $1-\beta$ wird auf 80\% gesetzt. Alle statistischen Tests sind so konstruiert, dass die $H_A$ mit 80\% "bewiesen wird".
\item [\textbf{E} \msquare] Die Power beschreibt die Wahrscheinlichkeit die $H_A$ abzulehnen. Wir testen die Power jedoch nicht.
\end{enumerate}

\section{Aufgabe \hfill (2 Punkte)}

Beim statistischen Testen wird \texttt{signal} mit \texttt{noise} zur
Teststatistik T verrechnet. Welche der Formel berechnet korrekt die
Teststatistik T?



\begin{enumerate}
\item [\textbf{A} \msquare] Es gilt $T = signal \cdot noise$
\item [\textbf{B} \msquare] Es gilt $T = \cfrac{signal}{noise^2}$
\item [\textbf{C} \msquare] Es gilt $T = \cfrac{signal}{noise}$
\item [\textbf{D} \msquare] Es gilt $T = (signal \cdot noise)^2$
\item [\textbf{E} \msquare] Es gilt $T = \cfrac{noise}{signal}$
\end{enumerate}

%% ------------------------------------------------------------

\section{Aufgabe \hfill (2 Punkte)}



In der Theorie zur statistischen Testentscheidung kann "`$H_0$ beibehalten obwohl die $H_0$ falsch ist"'
in welche richtige Analogie gesetzt werden?



\begin{enumerate}
\item [\textbf{A} \msquare] In die Analogie eines Rauchmelders: \textit{Alarm without fire}, dem $\alpha$-Fehler.
\item [\textbf{B} \msquare] In die Analogie eines Feuerwehrautos: \textit{Car without noise}.
\item [\textbf{C} \msquare] In die Analogie eines brennenden Hauses ohne Rauchmelder: \textit{House without noise}.
\item [\textbf{D} \msquare] In die Analogie eines Rauchmelders: \textit{Alarm with fire}.
\item [\textbf{E} \msquare] In die Analogie eines Rauchmelders: \textit{Fire without alarm}, dem $\beta$-Fehler.
\end{enumerate}

\section{Aufgabe \hfill (2 Punkte)}



Sie rechnen eine simple logistische Regression. Welche Aussage bestreffend der
Konfidenzintervalle ist f{\"u}r die logistische Regression richtig?



\begin{enumerate}
\item [\textbf{A} \msquare] Wenn die 0 im Konfidenzinterval enthalten ist, kann die Nullhypothese abgelehnt werden.
\item [\textbf{B} \msquare] Wenn die Relevanzschwelle mit enthalten ist, kann die Nullhypothese abgelehnt werden.
\item [\textbf{C} \msquare] Wenn die 1 im Konfidenzinterval enthalten ist, kann die Nullhypothese nicht abgelehnt werden.
\item [\textbf{D} \msquare] Wenn die 0 im Konfidenzinterval enthalten ist, kann die Nullhypothese nicht abgelehnt werden.
\item [\textbf{E} \msquare] Wenn die Konfidenzintervalle den p-Wert der Regression enthalten, kann die Nullhypothese abgelehnt werden.
\end{enumerate}

\section{Aufgabe \hfill (2 Punkte)}

In der Bio Data Science wird h{\"a}ufig mit sehr gro{\ss}en Datens{\"a}tzen
gerechnet. Historisch ergibt sich nun ein Problem bei der Auswertung der
Daten und deren Bewertung hinsichtlich der Signifikanz. Welche Aussage ist richtig?



\begin{enumerate}
\item [\textbf{A} \msquare] Big Data ist ein Problem der parametrischen Statistik. Parameter lassen sich nur auf kleinen Datens{"a}tzen berechnen, da es sich sonst nicht mehr um eine Stichprobe im engen Sinne der Statistik handelt.
\item [\textbf{B} \msquare] Aktuell werden zu grosse Datens{"a}tze f{"u}r die g{"a}nigige Statistik gemessen. Daher wendet man maschinelle Lernverfahren f{"u}r kausale Modelle an. Hier ist die Relevanz gleich Signifikanz.
\item [\textbf{C} \msquare] Aktuell werden immer gr{"o}ssere Datens{"a}tze erhoben. Eine erh{"o}hte Fallzahl f{"u}hrt automatisch auch zu mehr signifikanten Ergebnissen, selbst wenn die eigentlichen Effekte nicht relevant sind.
\item [\textbf{D} \msquare] Aktuell werden immer gr{"o}ssere Datens{"a}tze erhoben. Dadurch wird auch die Varianz immer h{"o}her was automatisch zu mehr signifikanten Ergebnissen f{"u}hrt.
\item [\textbf{E} \msquare] Relevanz und Signifikanz haben nichts miteinander zu tun. Daher gibt es auch keinen Zusammenhang zwischen hoher Fahlzahl (n > 10000) und einem signifikanten Test. Ein Effekt ist immer relevant und somit signifikant.
\end{enumerate}

\section{Aufgabe \hfill (2 Punkte)}

Welche statistische Masszahl erlaubt es \textit{Relevanz} mit
\textit{Signifikanz} zuverbinden? Welche Aussage ist richtig?



\begin{enumerate}
\item [\textbf{A} \msquare] Das OR. Als Chancenverh{"a}ltnis gibt es das Verh{"a}ltnis von Relevanz und Signifikanz wieder.
\item [\textbf{B} \msquare] Der p-Wert. Durch den Vergleich mit $\alpha$ l{"a}sst sich {"u}ber die Signifikanz entscheiden und der $\beta$-Fehler erlaubt {"u}ber die Power eine Einsch{"a}tzung der Relevanz.
\item [\textbf{C} \msquare] Das Konfidenzintervall. Durch die Visualizierung des Konfidenzintervals kann eine Relevanzschwelle vom Anwender definiert werden. Zus{"a}tzlich erlaubt das Konfidenzinterval auch eine Entscheidung {"u}ber die Signifikanz.
\item [\textbf{D} \msquare] Das $\Delta$. Durch die Effektst{"a}rke haben wir einen Wert f{"u}r die Relevanz, die vom Anwender bewertet werden muss. Da $\Delta$ antiproportional zum p-Wert ist, bedeutet auch ein hohes $\Delta$ ein sehr kleinen p-Wert.
\item [\textbf{E} \msquare] Die Teststatistik. Durch den Vergleich von $T_c$ zu $T_k$ ist es m{"o}glich die $H_0$ abzulehnen. Die Relevanz ergibt sich aus der Fl{"a}che rechts vom dem $T_c$-Wert.
\end{enumerate}

\section{Aufgabe \hfill (2 Punkte)}

Welche Aussage über den $p$-Wert und dem Signifikanzniveau $\alpha$ gleich 5\% ist richtig?



\begin{enumerate}
\item [\textbf{A} \msquare] Wir vergleichen mit dem $p$-Wert und dem Signifikanzniveau $\alpha$ absolute Werte auf einem Zahlenstrahl und damit den Unterschied der Teststatistiken, wenn die $H_0$ gilt.
\item [\textbf{B} \msquare] Wir machen eine Aussage über die indivduelle Wahrscheinlichkeit des Eintretens der Nullhypothese $H_0$.
\item [\textbf{C} \msquare] Wir vergleichen mit dem $p$-Wert und dem Signifikanzniveau $\alpha$ Wahrscheinlichkeiten und damit die Flächen unter der Kurve der Teststatistik, wenn die $H_0$ gilt.
\item [\textbf{D} \msquare] Wir vergleichen die Effekte des $p$-Wertes mit den Effekten der Signifiaknzschwelle unter der Annahme der Nullhypothese.
\item [\textbf{E} \msquare] Wir vergleichen mit dem $p$-Wert und dem Signifikanzniveau $\alpha$ Wahrscheinlichkeiten und damit die absoluten Werte auf einem Zahlenstrahl, wenn die $H_0$ gilt.
\end{enumerate}

\section{Aufgabe \hfill (2 Punkte)}

Welche Aussage {\"u}ber den t-Test ist richtig?



\begin{enumerate}
\item [\textbf{A} \msquare] Der t-Test vergleicht die Mittelwerte von zwei Gruppen unter der strikten Annahme von Varianzhomogenit{"a}t. Sollte keine Varianzhomogenit{"a}t vorliegen, so gibt es keine M{"o}glichkeit den t-Test in einer Variante anzuwenden.
\item [\textbf{B} \msquare] Der t-Test testet generell zu einem erh{"o}hten $\alpha$-Niveau von 20\%.
\item [\textbf{C} \msquare] Der t-Test vergleicht die Varianzen von mindestens zwei oder mehr Gruppen
\item [\textbf{D} \msquare] Der t-Test ist ein Vortest der ANOVA und basiert daher auf dem Vergleich von Streuungsparametern
\item [\textbf{E} \msquare] Der t-Test vergleicht die Mittelwerte von zwei Gruppen.
\end{enumerate}

\section{Aufgabe \hfill (2 Punkte)}

Welche Aussage {\"u}ber den Welch t-Test ist richtig?



\begin{enumerate}
\item [\textbf{A} \msquare] Der Welch t-Test vergleicht die Mittelwerte von zwei Gruppen unter der strikten Annahme von Varianzhomogenit{"a}t.
\item [\textbf{B} \msquare] Der Welch t-Test wird angewendet, wenn Varianzheterogenit{"a}t zwischen den beiden zu vergleichenden Gruppen vorliegt.
\item [\textbf{C} \msquare] Der Welch t-Test vergleicht die Varianz von zwei Gruppen.
\item [\textbf{D} \msquare] Der Welch t-Test ist ein Post-hoc Test der ANOVA und basiert daher auf dem Vergleich der Varianz.
\item [\textbf{E} \msquare] Der Welch t-Test ist die veraltete Form des Student t-Test und wird somit nicht mehr verwendet.
\end{enumerate}

\section{Aufgabe \hfill (2 Punkte)}

Nach einem Experiment mit f{\"u}nf Weizensorten ergibt eine ANOVA ($p = 0.041$)
einen signifikanten Unterschied f{\"u}r den Ertrag. Sie f{\"u}hren anschlie{\ss}end die
paarweisen t-Tests f{\"u}r alle Vergleiche der verschiedenen Weizensorten
durch. Nach der Adjustierung f{\"u}r multiples Testen ist kein p-Wert unter der
$\alpha$-Schwelle. Sie schauen sich auch die rohen, unadjustierten p-Werte
an und finden hier als niedrigsten p-Wert $p_{3-2} = 0.053$. Welche Aussage
ist richtig? 



\begin{enumerate}
\item [\textbf{A} \msquare] Es gibt einen Fehler in der Varianzstruktur. Daher kann die ANOVA nicht richtig sein und paarweise t-Tests liefern das richtige Ergebnis.
\item [\textbf{B} \msquare] Die ANOVA testet auf der gesamten Fallzahl. Es w{"a}re besser die ANOVA auf der gleichen Fallzahl wie die einzelnen t-Tests zu rechnen.
\item [\textbf{C} \msquare] Der Fehler liegt in den t-Tests. Wenn eine ANOVA signifikant ist, dann muss zwangsweise auch ein t-Test signifikant sein.
\item [\textbf{D} \msquare] Die adjustierten p-Werte deuten in die richtige Richtung. Zusammen mit den nicht signifikanten rohen p-Werten ist von einem Fehler in der ANOVA auszugehen.
\item [\textbf{E} \msquare] Die ANOVA testet auf der gesamten Fallzahl. Die einzelnen t-Tests immer nur auf einer kleineren Subgruppe. Da mit weniger Fallzahl weniger signifikante Ergebnisse zu erwarten sind, kann eine Diskrepenz zwischen der ANOVA und den paarweisen t-Tests auftreten.
\end{enumerate}

\section{Aufgabe \hfill (2 Punkte)}

Welche Aussage {\"u}ber den gepaarten t-Test f{\"u}r verbundene Stichproben ist richtig?



\begin{enumerate}
\item [\textbf{A} \msquare] Der gepaarte t-Test wird genutzt, wenn die Differenzen der Beobachtungen verbunden sind und wir dadurch die Unabh{"a}{"a}ngigkeit nicht mehr vorliegen haben.
\item [\textbf{B} \msquare] Beim gepaarten t-Test kombinieren wir die Vorteile des Student t-Test f{"u}r Varianzhomogenit{"a}t mit den Vorteilen des Welch t-Test f{"u}r Varianzheterogenit{"a}t. Wir bilden daf{"u}r die Differenz der Einzelbeobachtungen.
\item [\textbf{C} \msquare] Der gepaarte t-Test wird gerechnet, wenn die Beobachtungen nicht unabh{"a}ngig voneinander sind. Wir messen wiederholt an dem gleichen Probanden oder Tier oder Pflanze. Wir bilden die Differenzen um den gepaarten t-Test rechnen zu k{"o}nnen.
\item [\textbf{D} \msquare] Der gepaarte t-Test nutzt die Varianz der Beobachtungen jeweils paarweise und bildet daf{"u}r eine verbundene Stichprobe. Dieser Datensatz $d$ dient dann zur Differenzbildung.
\item [\textbf{E} \msquare] Der gepaarte t-Test wird gerechnet, wenn die Beobachtungen abh{"a}ngig voneinander sind. Wir messen jede Beobachtung nur einmal und berechnen dann die Differenz zu dem Mittel der anderen Beobachtungen.
\end{enumerate}
% -----------------------------------------------------------------------
\clearpage
% -----------------------------------------------------------------------
\begin{graybox}{Deskriptive Statistik \& Explorative Datenanalyse}
Mehr Informationen zu den Aufgaben in den folgenden Kapiteln aus dem Skript Bio Data Science.
  \begin{itemize}
  \item \href{https://jkruppa.github.io/eda-descriptive.html}{Kapitel 15 - Deskriptive Statistik}
  \item \href{https://jkruppa.github.io/eda-ggplot.html}{Kapitel 16 - Visualisierung von Daten}
  \item \href{https://jkruppa.github.io/eda-distribution.html}{Kapitel 18 - Verteilung von Daten}
  \end{itemize}
\end{graybox}
\clearpage
% -----------------------------------------------------------------------

\section{Aufgabe \hfill (10 Punkte)}

\textit{Geben Sie grunds{\"a}tzlich Formeln und Rechenweg zur L{\"o}sung der
  Teilaufgaben mit an!} \\[1Ex]

%% --------------------------------------------------------------------
\hfill\href{https://youtu.be/sBlSc_eJbnw}{\includegraphics[width =
  2cm]{img/youtube}}\\[1Ex]
%% --------------------------------------------------------------------


Sie haben folgende Zahlenreihe $y$ vorliegen
$y = \{20, 20, 18, 14, 19, 21, 17, 23, 19, 21, 16\}$. Berechnen Sie folgende
deskriptive Ma{\ss}zahlen. Geben Sie Formeln und Rechenwege mit an!



\begin{enumerate}
\item Die Range oder Spannweite \textbf{(2 Punkte)}
\item Das 1st Quartile \textbf{(2 Punkte)}
\item Den Median \textbf{(2 Punkte)}
\item Den Interquartileabstand \textbf{(2 Punkte)}
\item Die Standardabweichung \textbf{(2 Punkte)}
\end{enumerate}
 
\clearpage
% -----------------------------------------------------------------------

\section{Aufgabe \hfill (8 Punkte)}

\textit{Geben Sie grunds{\"a}tzlich Formeln und Rechenweg zur L{\"o}sung der
  Teilaufgaben mit an!} \\[1Ex]

%% --------------------------------------------------------------------
\hfill\href{https://youtu.be/oMdtYbDInYE}{\includegraphics[width =
  2cm]{img/youtube}}\\[1Ex]
%% --------------------------------------------------------------------

Sie haben folgende Zahlenreihe $y$ vorliegen
$y = \{15, 14, 19, 17, 12, 16\}$.

\begin{enumerate}
\item Visualisieren Sie den Mittelwert von $y$ in der untenstehenden
  Abbildung! \textbf{(4 Punkte)}
\item Beschriften Sie die $Y$ und $X$-Achse entsprechend! \textbf{(2 Punkte)}
\item F{\"u}r die Berechnung der Varianz wird der Abstand der einzelnen Werte $y_i$
  zum Mittelwert $\bar{y}$ quadriert. Warum muss der Abstand, $y_i -
  \bar{y}$, in der Varianzformel quadriert werden?
  Erkl{\"a}ren Sie den Zusammenhang unter Ber{\"u}cksichtigung der Abbildung!
  \textbf{(2 Punkte)}  
\end{enumerate}



{\centering \includegraphics[width=\maxwidth]{img/desc-01-1} 

}


 
\clearpage
% -----------------------------------------------------------------------

\section{Aufgabe \hfill (7 Punkte)}

\textit{Geben Sie grunds{\"a}tzlich Formeln und Rechenweg zur L{\"o}sung der
  Teilaufgaben mit an!} \\[1Ex]

%% --------------------------------------------------------------------
\hfill\href{https://youtu.be/vXnLttRL_VI}{\includegraphics[width =
  2cm]{img/youtube}}\\[1Ex]
%% --------------------------------------------------------------------

Nach einem Gew{\"a}chshausexperiment mit drei Bew{\"a}sserungstypen ($low$, $mid$
und $high$) ergibt sich die folgende Datentabelle mit dem gemessenen
Frischgewicht (\textit{freshmatter}).

\begin{table}[!h]
\centering
\begin{tabular}{cc}
\toprule
water\_type & freshmatter\\
\midrule
low & 22\\
high & 15\\
mid & 27\\
low & 23\\
mid & 28\\
\addlinespace
low & 23\\
high & 21\\
low & 22\\
high & 19\\
high & 16\\
\addlinespace
mid & 31\\
\bottomrule
\end{tabular}
\end{table}



\begin{enumerate}
\item Zeichnen Sie in \textit{einer} Abbildung die Barplots f{\"u}r die
  Bew{\"a}sserungstypen! Beschriften Sie die Achsen entsprechend!  \textbf{(4
    Punkte)}
\item Beschriften Sie \textit{einen} Barplot mit den g{\"a}ngigen
  statistischen Ma{\ss}zahlen! \textbf{(2 Punkte)}
\item Wenn Sie \textit{keinen Effekt} zwischen der Bew{\"a}sserungstypen
  erwarten w{\"u}rden, wie sehen dann die Barplots aus? \textbf{(1 Punkt)}
\end{enumerate} 
\clearpage
% -----------------------------------------------------------------------

\section{Aufgabe \hfill (6 Punkte)}

\textit{Geben Sie grunds{\"a}tzlich Formeln und Rechenweg zur L{\"o}sung der
  Teilaufgaben mit an!} \\[1Ex]

%% --------------------------------------------------------------------
\hfill\href{https://youtu.be/MiD42k4l5Ag}{\includegraphics[width =
  2cm]{img/youtube}}\\[1Ex]
%% --------------------------------------------------------------------



\begin{enumerate}
\item Skizieren Sie in die unten stehenden, freien Abbildungen die
  Verteilungen, die sich nach der Abbildungs{\"u}berschrift ergeben! \textbf{(4
    Punkte)}
\item Achten Sie auf die entsprechende Skalierung der beiden Verteilungen
  in der ersten Abbildung! \textbf{(2 Punkte)}
\end{enumerate}



{\centering \includegraphics[width=\maxwidth]{img/histogram-01-1} 

}



 
\clearpage
% -----------------------------------------------------------------------

\section{Aufgabe \hfill (6 Punkte)}

\textit{Geben Sie grunds{\"a}tzlich Formeln und Rechenweg zur L{\"o}sung der
  Teilaufgaben mit an!} \\[1Ex]

%% --------------------------------------------------------------------
\hfill\href{https://youtu.be/ZrJhn2wPbq4}{\includegraphics[width =
  2cm]{img/youtube}}\\[1Ex]
%% --------------------------------------------------------------------



\begin{enumerate}
\item Skizieren Sie $3$ Normalverteilungen \textit{in einer
    Abbildung} mit $\bar{y}_1 \neq \bar{y}_2 \neq \bar{y}_3$ und $s_1 = s_2 = s_3$! \textbf{(2 Punkte)}
\item Beschriften Sie die Normalverteilungen mit den entsprechenden
  Parametern! \textbf{(2 Punkte)}
\item Liegt Varianzhomogenit{\"a}t oder Varianzheterogenit{\"a}t vor? Begr{\"u}nden Sie
  Ihre Antwort! \textbf{(2 Punkte)}
\end{enumerate}

 
\clearpage
% -----------------------------------------------------------------------

\section{Aufgabe \hfill (6 Punkte)}

\textit{Geben Sie grunds{\"a}tzlich Formeln und Rechenweg zur L{\"o}sung der
  Teilaufgaben mit an!} \\[1Ex]

%% --------------------------------------------------------------------
\hfill\href{https://youtu.be/aXvxGC4YLqk}{\includegraphics[width =
  2cm]{img/youtube}}\\[1Ex]
%% --------------------------------------------------------------------



Nach einem Experiment z{\"a}hlen Sie folgende Anzahl an L{\"a}sionen auf den
Bl{\"a}ttern von Sonnenblumen nach einer durchgestandenen Infektion. 

\begin{center}
$1, 6, 7, 6, 7, 9, 3, 5, 6, 2, 3, 6, 4, 8$
\end{center}

\begin{enumerate}
\item Zeichen Sie ein Histogramm um die Verteilung der Daten zu visualiseren! (\textbf{3 Punkte})
\item Beschriften Sie die Achsen der Abbildung! (\textbf{2 Punkte})
\item Erg{\"a}nzen Sie die relativen H{\"a}ufigkeiten in der Abbildung! \textbf{(1
    Punkt)}  
\end{enumerate}

 
\clearpage
% -----------------------------------------------------------------------

\section{Aufgabe \hfill (8 Punkte)}

\textit{Geben Sie grunds{\"a}tzlich Formeln und Rechenweg zur L{\"o}sung der
  Teilaufgaben mit an!} \\[1Ex]

%% --------------------------------------------------------------------
\hfill\href{https://youtu.be/ORHSPTCdfeY}{\includegraphics[width =
  2cm]{img/youtube}}\\[1Ex]
%% --------------------------------------------------------------------



Nach einem Experiment z{\"a}hlen Sie folgende Trockengewichte von Sonnenblumen nach einer durchgestandenen Infektion. 

\begin{center}
$11.4, 9.3, 8.6, 7.2, 7.3, 10.1, 10.6, 10.6, 9.9, 6.2, 9.7, 11.2, 8.7, 9.4$
\end{center}

\begin{enumerate}
\item Zeichen Sie ein Histogramm um die Verteilung der Daten zu
  visualiseren! (\textbf{3 Punkte})
 \item Erl{\"a}utern Sie Ihr Vorgehen um ein Histogramm f{\"u}r kontinuierliche
  Daten zu zeichnen!  (\textbf{2 Punkte})
\item Beschriften Sie die Achsen der Abbildung! (\textbf{2 Punkte})
\item Erg{\"a}nzen Sie die relativen H{\"a}ufigkeiten in der Abbildung! \textbf{(1
    Punkt)}  
\end{enumerate}

 
\clearpage
% -----------------------------------------------------------------------

\section{Aufgabe \hfill (9 Punkte)}

\textit{Geben Sie grunds{\"a}tzlich Formeln und Rechenweg zur L{\"o}sung der
  Teilaufgaben mit an!} \\[1Ex]

%% --------------------------------------------------------------------
\hfill\href{https://youtu.be/0xc0jIPeiyw}{\includegraphics[width =
  2cm]{img/youtube}}\\[1Ex]
%% --------------------------------------------------------------------


Nach einem Feldexperiment mit zwei D{\"u}ngestufen (A und B) ergibt sich die
folgende Datentabelle mit dem gemessenen Trockengewicht (\textit{drymatter}). 

\begin{table}[!h]
\centering
\begin{tabular}{cc}
\toprule
trt & drymatter\\
\midrule
A & 21.5\\
B & 17.1\\
A & 15.1\\
A & 16.9\\
B & 13.1\\
\addlinespace
B & 16.3\\
B & 18.7\\
A & 18.3\\
B & 14.7\\
A & 14.9\\
\addlinespace
A & 18.2\\
A & 14.4\\
B & 15.7\\
B & 15.2\\
\bottomrule
\end{tabular}
\end{table}



\begin{enumerate}
\item Zeichnen Sie in \textit{einer} Abbildung die beiden Boxplots f{\"u}r die
  zwei D{\"u}ngestufen A und B! Beschriften Sie die Achsen entsprechend!
  \textbf{(6 Punkte)}
\item Beschriften Sie \textit{einen} der beiden Boxplots mit den g{\"a}ngigen
  statistischen Ma{\ss}zahlen! \textbf{(2 Punkte)}
\item Wenn Sie \textit{keinen Effekt} zwischen den D{\"u}ngestufen erwarten
  w{\"u}rden, wie sehen dann die beiden Boxplots aus? \textbf{(1 Punkt)}
\end{enumerate} 
\clearpage
% -----------------------------------------------------------------------

\section{Aufgabe \hfill (9 Punkte)}

\textit{Geben Sie grunds{\"a}tzlich Formeln und Rechenweg zur L{\"o}sung der
  Teilaufgaben mit an!} \\[1Ex]

%% --------------------------------------------------------------------
\hfill\href{https://youtu.be/VX4Hs81h8_A}{\includegraphics[width =
  2cm]{img/youtube}}\\[1Ex]
%% --------------------------------------------------------------------

Nach einem Feldexperiment mit mehreren D{\"u}ngestufen stellt sich die Frage,
ob die D{\"u}ngestufe \textit{low} im Bezug auf das Trockengewicht
normalverteilt sei. Sie erhalten folgende Datentabelle.

\begin{table}[!h]
\centering
\begin{tabular}{cc}
\toprule
fertilizer & drymatter\\
\midrule
low & 13\\
low & 22\\
low & 24\\
low & 14\\
low & 21\\
\addlinespace
low & 21\\
low & 14\\
low & 23\\
low & 25\\
low & 24\\
\addlinespace
low & 14\\
\bottomrule
\end{tabular}
\end{table}



\begin{enumerate}
\item Zeichnen Sie eine passende Abbildung in der Sie visuell {\"u}berpr{\"u}fen
  k{\"o}nnen, ob eine Normalverteilung des Trockengewichts vorliegt! \textbf{(4
    Punkte)}
\item Beschriften Sie die Achsen und erg{\"a}nzen Sie die statistischen
  Ma{\ss}zahlen. \textbf{(3 Punkte)}
\item Entscheiden Sie, ob eine Normalveteilung vorliegt. Begr{\"u}nden Sie Ihre
  Antwort. \textbf{(2 Punkte)}
\end{enumerate} 
\clearpage
% -----------------------------------------------------------------------

\section{Aufgabe \hfill (4 Punkte)}

\textit{Geben Sie grunds{\"a}tzlich Formeln und Rechenweg zur L{\"o}sung der
  Teilaufgaben mit an!} \\[1Ex]

%% --------------------------------------------------------------------
\hfill\href{https://youtu.be/Op-gjzASH9I}{\includegraphics[width =
  2cm]{img/youtube}}\\[1Ex]
%% --------------------------------------------------------------------



\begin{enumerate}
\item Zeichnen Sie {\"u}ber den untenstehenden Boxplot die entsprechende
  zugeh{\"o}rige Verteilung! \textbf{(2 Punkte)} 
\item Zeichnen Sie unter den untenstehenden Boxplot die entsprechende
  zugeh{\"o}rige Beobachtungen! \textbf{(2 Punkte)}
\end{enumerate}

\vspace*{8cm}

\begin{center}
  \includegraphics[width=10cm]{/Users/kruppajo/Documents/GitHub/exam/question/img/boxplot-03-b.png}
\end{center}



 
\clearpage
% -----------------------------------------------------------------------

\section{Aufgabe \hfill (9 Punkte)}

\textit{Geben Sie grunds{\"a}tzlich Formeln und Rechenweg zur L{\"o}sung der
  Teilaufgaben mit an!} \\[1Ex]

%% --------------------------------------------------------------------
\hfill\href{https://youtu.be/lXI_H6m26HE}{\includegraphics[width =
  2cm]{img/youtube}}\\[1Ex]
%% --------------------------------------------------------------------


In einem Experiment mit zwei D{\"u}ngestufen f{\"u}r den Ertrag von Kichererbsen
ergibt sich folgende Abbildung. 





{\centering \includegraphics[width=\maxwidth]{img/boxplot-4b-1} 

}




\begin{enumerate}
\item Tragen Sie in die untenstehende Tabelle die g{\"a}ngigen Ma{\ss}zahlen des
  Boxplots und die abgesch{\"a}tzen Werte aus den obigen Boxplots ein! \textbf{(4 Punkte)}
\end{enumerate}

\begin{center}
  \large
  \begin{tabular}[c]{c|c|c}
    Statistische Ma{\ss}zahl  & \multicolumn{2}{c}{Abgesch{\"a}tzter Wert}  \strut\\
    & low & high \\
    \hline
    \hspace{2cm} & \hspace{2cm} & \hspace{2cm} \strut\\
    \hline
    \hspace{2cm} & \hspace{2cm} & \hspace{2cm} \strut\\
    \hline
    \hspace{2cm} & \hspace{2cm} & \hspace{2cm} \strut\\
    \hline
    \hspace{2cm} & \hspace{2cm} & \hspace{2cm} \strut\\
    \hline
    \hspace{2cm} & \hspace{2cm} & \hspace{2cm} \strut\\
    \hline
  \end{tabular}
\end{center}


\begin{enumerate}
  \setcounter{enumi}{1}
\item Erg{\"a}nzen Sie den Mittelwert f{\"u}r beide Level des D{\"u}ngers in die
  Abbildung der Boxplots! Begr{\"u}nden Sie Ihre Antwort! \textbf{(2 Punkte)}
\item Erg{\"a}nzen Sie in der untenstehenden Tabelle die $p$-Werte f{\"u}r den
  Shapiro-Wilk-Test auf Normalverteilung und den Levene-Test auf
  Varianzhomogenit{\"a}t. Beachten Sie die unterschiedliche, angenommene
  Fallzahl $n_g$ der beiden Level des D{\"u}ngers! \textbf{(3 Punkte)}
\end{enumerate}

\begin{center}
  \large
  \begin{tabular}[c]{l|c|c}
  Fallzahl  & Shapiro-Wilk-Test & Levene-Test \strut\\ 
    \hline
    \textbf{$n_g = 5$} & \hspace{4cm} & \hspace{4cm} \strut\\
    \hline
    \textbf{$n_g = 20$} & \hspace{4cm} & \hspace{4cm} \strut\\
    \hline
    \textbf{$n_g > 50$} & \hspace{4cm} & \hspace{4cm} \strut\\
    \hline
  \end{tabular}
\end{center} 
\clearpage
% -----------------------------------------------------------------------

\section{Aufgabe \hfill (6 Punkte)}

\textit{Geben Sie grunds{\"a}tzlich Formeln und Rechenweg zur L{\"o}sung der
  Teilaufgaben mit an!} \\[1Ex]

%% --------------------------------------------------------------------
\hfill\href{https://youtu.be/knAziLLQGb0}{\includegraphics[width =
  2cm]{img/youtube}}\\[1Ex]
%% --------------------------------------------------------------------

Nach einer Bonitur von Schnittlauch mit einer Kontrolle und drei Pestiziden (ctrl, pestKill, roundUp, zeroX) ergibt sich die folgende Datentabelle mit den Boniturnoten (\textit{grade}). 

\begin{table}[!h]
\centering
\begin{tabular}{cc}
\toprule
pesticide & grade\\
\midrule
ctrl & 8\\
roundUp & 5\\
pestKill & 3\\
ctrl & 8\\
pestKill & 4\\
\addlinespace
pestKill & 4\\
roundUp & 5\\
zeroX & 1\\
zeroX & 1\\
roundUp & 2\\
\addlinespace
zeroX & 1\\
zeroX & 1\\
ctrl & 8\\
pestKill & 3\\
\bottomrule
\end{tabular}
\end{table}



\begin{enumerate}
\item Zeichnen Sie in \textit{einer} Abbildung die Dotplots f{\"u}r die
  vier Pestizidlevel! Beschriften Sie die Achsen entsprechend!
  \textbf{(4 Punkte)}
\item Erg{\"a}nzen Sie die Dotplots mit der g{\"a}ngigen
  statistischen Ma{\ss}zahl. \textbf{(1 Punkt)}
\item Wenn Sie \textit{keinen Effekt} zwischen den Pestizidlevel erwarten
  w{\"u}rden, wie sehen dann die Dotplots aus? \textbf{(1 Punkt)}
\end{enumerate} 
\clearpage
% -----------------------------------------------------------------------

\section{Aufgabe \hfill (8 Punkte)}

\textit{Geben Sie grunds{\"a}tzlich Formeln und Rechenweg zur L{\"o}sung der
  Teilaufgaben mit an!} \\[1Ex]

%% --------------------------------------------------------------------
\hfill\href{https://youtu.be/t_1KL77mfmg}{\includegraphics[width =
  2cm]{img/youtube}}\\[1Ex]
%% --------------------------------------------------------------------

Nach einem Feldexperiment mit zwei Pestiziden (\textit{RoundUp} und
\textit{OutEx}) ergibt sich die folgende Datentabelle mit dem jeweiligen
beobachteten Infektionsstatus.

\begin{table}[!h]
\centering
\begin{tabular}{cc}
\toprule
pesticide & infected\\
\midrule
OutEx & no\\
OutEx & no\\
OutEx & no\\
OutEx & yes\\
OutEx & yes\\
\addlinespace
OutEx & no\\
RoundUp & yes\\
RoundUp & no\\
OutEx & yes\\
RoundUp & no\\
\addlinespace
OutEx & yes\\
RoundUp & no\\
RoundUp & yes\\
RoundUp & yes\\
RoundUp & yes\\
\addlinespace
RoundUp & no\\
RoundUp & no\\
RoundUp & no\\
OutEx & yes\\
RoundUp & yes\\
\bottomrule
\end{tabular}
\end{table}



\begin{enumerate}
\item Stellen Sie in einer 2x2 Tafel den Zusammenhang zwischen dem
  Pesizid und dem Infektionsstatus dar! \textbf{(4 Punkte)}
\item Zeichnen Sie den zugeh{\"o}rigen Mosaic-Plot. Berechnen Sie das
  Verh{\"a}ltnis pro Spalte! \textbf{(2 Punkte)}
\item Wenn das Pesizid keine Auswirkung auf den Infektionsstatus h{\"a}tte, wie
  sehe dann der Mosaic-Plot aus? \textbf{(2 Punkte)}
\end{enumerate} 
\clearpage
% -----------------------------------------------------------------------

\section{Aufgabe \hfill (10 Punkte)}

\textit{Geben Sie grunds{\"a}tzlich Formeln und Rechenweg zur L{\"o}sung der
  Teilaufgaben mit an!} \\[1Ex]

%% --------------------------------------------------------------------
\hfill\href{https://youtu.be/VAqiUdV4WQ0}{\includegraphics[width =
  2cm]{img/youtube}}\\[1Ex]
%% --------------------------------------------------------------------

In einem Feldexperiment f{\"u}r die Bodendurchl{\"a}ssigkeit wurde der Niederschlag
pro Parzelle sowie der durchschnittliche Ertrag gemessen. Es ergibt sich
folgende Datentabelle. 

\begin{table}[!h]
\centering
\begin{tabular}{cc}
\toprule
water & drymatter\\
\midrule
16 & 13\\
16 & 8\\
12 & 9\\
12 & 21\\
18 & 18\\
\addlinespace
14 & 4\\
20 & 5\\
\bottomrule
\end{tabular}
\end{table}



\begin{enumerate}
\item Erstellen Sie den Scatter-Plot f{\"u}r die Datentabelle. Beschriften Sie
  die Achsen entsprechend! \textbf{(4 Punkte)}
\item Zeichnen Sie eine Gerade durch die Punkte! \textbf{(1 Punkt)}
\item Beschriften Sie die Gerade mit den g{\"a}ngigen statistischen Ma{\ss}zahlen!
  Geben Sie die numerischen Zahlenwerte mit an! \textbf{(3 Punkte)}
\item Wenn kein Effekt von dem Niederschlag auf das Trockengewicht
  vorhanden w{\"a}re, wie w{\"u}rde die Gerade verlaufen und welche Werte w{\"u}rden die
  statistischen Ma{\ss}zahlen annehmen? \textbf{(2 Punkt)}
\end{enumerate} 
\clearpage
% -----------------------------------------------------------------------
\begin{graybox}{Statistisches Testen}
Mehr Informationen zu den Aufgaben in den folgenden Kapiteln aus dem Skript Bio Data Science.
  \begin{itemize}
  \item \href{https://jkruppa.github.io/preface.html#lernziel-3-falsifikationsprinzip}{Kapitel 3 - Falsifikationsprinzip}
  \item \href{https://jkruppa.github.io/stat-tests-basic.html}{Kapitel 19 - Die Testentscheidung}
  \item \href{https://jkruppa.github.io/stat-tests-theorie.html}{Kapitel 20 - Die Testtheorie}
  \end{itemize}
\end{graybox}
\clearpage
% -----------------------------------------------------------------------  

\section{Aufgabe \hfill (6 Punkte)}

%% --------------------------------------------------------------------
\hfill\href{https://youtu.be/1S-FuQisTpE}{\includegraphics[width =
  2cm]{img/youtube}}\\[1Ex]
%% --------------------------------------------------------------------

\begin{enumerate}
\item Erkl{\"a}ren Sie den Zusammenhang zwischen Stichprobe und Grundgesamtheit
  an einem Schaubild! \textbf{(3 Punkte)}
\item Was ist der Unterschied zwischen $\mu$ und $\sigma$ und $\bar{y}$ und
  $s$ im Kontext der Stichprobe und Grundgesamtheit? \textbf{(2 Punkte)}
\item Warum m{\"u}ssen wir {\"u}berhaupt zwischen einer Stichprobe und einer
  Grundgesamtheit unterscheiden? \textbf{(1 Punkt)}
\end{enumerate} 
\clearpage
% -----------------------------------------------------------------------

\section{Aufgabe \hfill (8 Punkte)}

%% --------------------------------------------------------------------
\hfill\href{https://youtu.be/3DfWs9NNrCk}{\includegraphics[width =
  2cm]{img/youtube}}\\[1Ex]
%% --------------------------------------------------------------------




Geben ist folgende 2x2 Kreuztabelle. 

\begin{center}
  \includegraphics[width = 13cm]{/Users/kruppajo/Documents/GitHub/exam/question/img/text-error-cross-table}
\end{center}

\begin{enumerate}
\item Tragen Sie folgende Fachbegriffe korrekt in die 2x2 Kreuztabelle ein! \textbf{(4 Punkte)}
  \begin{itemize}
  \item (Unbekannte) Wahrheit	
  \item H$_0$ wahr
  \item H$_0$ falsch
  \item H$_0$ abgelehnt
  \item H$_0$ beibehalten
  \item Testentscheidung
  \item $\alpha$-Fehler
  \item $\beta$-Fehler
  \item Richtige Entscheidung
  \item 5\%
  \item 20\%
  \end{itemize}
\item In der Analogie des Feuermelders, wie lauetet der $\alpha$-Fehler? \textbf{(1 Punkt)}
\item In der Analogie des Feuermelders, wie lauetet der $\beta$-Fehler? \textbf{(1 Punkt)}
\item Wenn der Feuermelder einmal pro Tag messen w{\"u}rde, wie oft w{\"u}rde der
  Feuermelder mit einem $\alpha$ von 5\% in einem Monat Alarm schlagen?
  Begr{\"u}nden Sie Ihre Antwort! \textbf{(2 Punkte)}
\end{enumerate}



 
\clearpage
% -----------------------------------------------------------------------

\section{Aufgabe \hfill (8 Punkte)}

%% --------------------------------------------------------------------
\hfill\href{https://youtu.be/32JjH1eyuTU}{\includegraphics[width =
  2cm]{img/youtube}}\\[1Ex]
%% --------------------------------------------------------------------



Im folgenden ist eine t-Verteilung abgebildet. Erg{\"a}nzen Sie die Abbildung wie folgt.

\begin{enumerate}
\item Zeichnen Sie das Signifikanzniveau $\alpha$ in die Abbildung! \textbf{(2 Punkte)} 
\item Zeichnen Sie einen signifikant p-Wert in die Abbildung! \textbf{(2 Punkte)} 
\item Erg{\"a}nzen Sie "`$\bar{y}_1 = \bar{y}_2$"'! \textbf{(1 Punkt)} 
\item Erg{\"a}nzen Sie "`$A = 0.95$"'! \textbf{(1 Punkt)}
\item Zeichnen Sie $T_{\alpha=5\%}$ in die Abbildung! \textbf{(1 Punkt)} 
\item Zeichnen Sie $+T_{calc}$ in die Abbildung! \textbf{(1 Punkt)} 
\end{enumerate}



{\centering \includegraphics[width=\maxwidth]{img/statistisches-testen-3-1} 

}


 
\clearpage
% -----------------------------------------------------------------------

\section{Aufgabe \hfill (10 Punkte)}

%% --------------------------------------------------------------------
\hfill\href{https://youtu.be/CN_O4fYPbhs}{\includegraphics[width =
  2cm]{img/youtube}}\\[1Ex]
%% --------------------------------------------------------------------



Sie rechnen einen t-Test f{\"u}r Gruppenvergleiche. Sie sch{\"a}tzen den Unterschied
zwischen dem mittleren Trockengewicht nach D{"u}ngergabe. 

\begin{enumerate}
\item Beschriften Sie die untenstehende Abbildung mit der
  Signifikanzschwelle! Begr{\"u}nden Sie Ihre Antwort! \textbf{(2 Punkte)}
\item Erg{\"a}nzen Sie eine \textit{in den Kontext passende} Relevanzschwelle!
  Begr{\"u}nden Sie Ihre Antwort! \textbf{(2 Punkte)} 
\item Skizieren Sie in die
  untenstehende Abbildung sechs einzelne Konfidenzintervalle (a-f) mit den
  jeweiligen Eigenschaften! \textbf{(6 Punkte)}
  \begin{itemize}
  \item[(a)] Ein nicht signifikantes, nicht relevantes 95\%-Konfidenzintervall 	
  \item[(b)] Ein 95\%-Konfidenzintervall mit h{"o}herer Fallzahl $n$ in der Stichprobe als der Rest der 95\%-Konfidenzintervalle 	
  \item[(c)] Ein signifikantes, relevantes 95\%-Konfidenzintervall 	
  \item[(d)] Ein signifikantes, nicht relevantes 95\%-Konfidenzintervall 
  \item[(e)] Ein signifikantes, relevantes 99\%-Konfidenzintervall.
  \item[(f)] Ein 95\%-Konfidenzintervall mit niedriger Fallzahl $n$ in der Stichprobe als der Rest 95\%-der Konfidenzintervalle
  \end{itemize}
\end{enumerate}

\begin{center}
  \includegraphics[height = 8cm]{/Users/kruppajo/Documents/GitHub/exam/question/img/statistisches-testen-04}
\end{center}


 
\clearpage
% -----------------------------------------------------------------------

\section{Aufgabe \hfill (6 Punkte)}

%% --------------------------------------------------------------------
\hfill\href{https://youtu.be/bc1m7rkXld4}{\includegraphics[width =
  2cm]{img/youtube}}\\[1Ex]
%% --------------------------------------------------------------------

Gegeben ist die vereinfachte Formel f{\"u}r den Zweistichproben t-Test mit der
gepoolten Standardabweichung $s_p$ und gleicher Gruppengr{\"o}sse $n_g$ der
beiden Sample.

\begin{equation*}
  \label{eq:1}
  T = \cfrac{\bar{y}_1 - \bar{y}_2}{s_p \cdot \sqrt{\tfrac{2}{n_g}}}
\end{equation*}

Welche Auswirkung hat die {\"A}nderungen der jeweiligen statistischen Masszahl
auf den T-Wert und damit auf die \textit{vermutliche} Signifikanz? F{\"u}llen
Sie hierzu die untenstehende Tabelle aus! \textbf{(6 Punkte)}

\begin{center}
  \large
  \begin{tabular}[c]{l|l|l|l|l|l|l|l}
    & T Statistik & $Pr(D|H_0)$ & $KI_{1-\alpha}$ & & T Statistik & $Pr(D|H_0)$ & $KI_{1-\alpha}$\strut\\ 
    \hline
    \textbf{$\Delta\; \uparrow$} & \hspace{2cm} & \hspace{2cm}  & \hspace{2cm} & \textbf{
                                                          $\Delta\; \downarrow$} &
                                                                          \hspace{2cm} & \hspace{2cm}  & \hspace{2cm}\strut\\
    \hline
        \textbf{$s\; \uparrow$} & \hspace{2cm} & \hspace{2cm}  & \hspace{2cm} & \textbf{
                                                          $s\; \downarrow$} &
                                                                          \hspace{2cm}
                                                & \hspace{2cm}  & \hspace{2cm}\strut\\
    \hline
        \textbf{$n\; \uparrow$} & \hspace{2cm} & \hspace{2cm}  & \hspace{2cm} & \textbf{
                                                          $n\; \downarrow$} &
                                                                          \hspace{2cm}
                                                & \hspace{2cm}  & \hspace{2cm}\strut\\
    \hline
  \end{tabular}
\end{center}
 
\clearpage
% -----------------------------------------------------------------------

\section{Aufgabe \hfill (8 Punkte)}

%% --------------------------------------------------------------------
\hfill\href{https://youtu.be/gQwvMuZ-Sjs}{\includegraphics[width =
  2cm]{img/youtube}}\\[1Ex]
%% --------------------------------------------------------------------



Sie haben folgende Aussage gegeben.

\begin{center}
  \Large\textbf{Bin ich im Herbst?}
\end{center}

\begin{enumerate}
\item Erkl{\"a}ren Sie den Gedankengang der Testtheorie sowie des Falsifikationsprinzips an der Aussage! \textbf{(4 Punkte)}
\item Erkl{\"a}ren Sie Ihre Entscheidung zu der Aussage! \textbf{(3 Punkte)}
\item Sch{\"a}tzen Sie den p-Wert zu der Aussage ab! \textbf{(1 Punkt)}
\end{enumerate}

 
\clearpage
% -----------------------------------------------------------------------
\begin{graybox}{Der t-Test}
Mehr Informationen zu den Aufgaben in den folgenden Kapiteln aus dem Skript Bio Data Science.
  \begin{itemize}
  \item \href{https://jkruppa.github.io/stat-tests-ttest.html}{Kapitel 22 - Der t-Test}
  \end{itemize}
\end{graybox}
\clearpage
% -----------------------------------------------------------------------

\section{Aufgabe \hfill (12 Punkte)}

\textit{Geben Sie grunds{\"a}tzlich Formeln und Rechenweg zur L{\"o}sung der
  Teilaufgaben mit an!} \\[1Ex]

%% --------------------------------------------------------------------
\hfill\href{https://youtu.be/Cq_rF_z4xOk}{\includegraphics[width =
  2cm]{img/youtube}}\\[1Ex]
%% --------------------------------------------------------------------

Nach einem Experiment mit zwei Pestiziden (\textit{RoundUp} und
\textit{GoneEx}) ergibt sich die folgende Datentabelle mit dem gemessenen
Trockengewicht (\textit{drymatter}) von Weizen.

\begin{table}[!h]
\centering
\begin{tabular}{cc}
\toprule
pesticide & drymatter\\
\midrule
RoundUp & 14\\
RoundUp & 18\\
GoneEx & 13\\
GoneEx & 16\\
RoundUp & 15\\
\addlinespace
GoneEx & 13\\
GoneEx & 14\\
RoundUp & 17\\
GoneEx & 12\\
RoundUp & 15\\
\addlinespace
GoneEx & 15\\
RoundUp & 18\\
RoundUp & 14\\
GoneEx & 13\\
\bottomrule
\end{tabular}
\end{table}



\begin{enumerate}
  \item Formulieren Sie die wissenschaftliche Fragestellung! \textbf{(1 Punkt)}
  \item Formulieren Sie das statistische Hypothesenpaar! \textbf{(2
      Punkte)}
  \item Bestimmen Sie die Teststatistik $T_{calc}$ eines Student t-Tests f{\"u}r den
  Vergleich der beiden Pestizide. Geben Sie den Rechenweg und die Formeln
  mit an! \textbf{(5 Punkte)}
\item Treffen Sie mit $T_{\alpha = 5\%} = 2.04$ und dem berechneten $T_{calc}$ eine Aussage
  zur Nullhypothese! \textbf{(2 Punkte)}
\item Wenn Sie keinen Unterschied zwischen den beiden Pestiziden erwarten
  w{\"u}rden, wie gro{\ss}e w{\"a}re dann die Teststatistik $T_{calc}$? Begr{\"u}nden Sie Ihre
  Antwort! \textbf{(2 Punkte)}
\end{enumerate} 
\clearpage
% -----------------------------------------------------------------------

\section{Aufgabe \hfill (11 Punkte)}

\textit{Geben Sie grunds{\"a}tzlich Formeln und Rechenweg zur L{\"o}sung der
  Teilaufgaben mit an!} \\[1Ex]

%% --------------------------------------------------------------------
\hfill\href{https://youtu.be/QR90zyn0Iz8}{\includegraphics[width =
  2cm]{img/youtube}}\\[1Ex]
%% --------------------------------------------------------------------


Das Gewicht von K{\"u}ken wurde \textit{vor} der Behandlung mit STARTex und 1
Woche \textit{nach} der Behandlung gemessen. Es ergibt sich die folgende
Datentabelle.

\begin{table}[!h]
\centering
\begin{tabular}{ccc}
\toprule
animal\_id & before & after\\
\midrule
1 & 17 & 20\\
2 & 14 & 11\\
3 & 14 & 13\\
4 & 14 & 18\\
5 & 14 & 13\\
\addlinespace
6 & 2 & 15\\
7 & 9 & 14\\
\bottomrule
\end{tabular}
\end{table}



\begin{enumerate}
\item Formulieren Sie die Fragestellung! \textbf{(1 Punkt)}
\item Formulieren Sie das statistische Hypothesenpaar! \textbf{(2
    Punkte)}
\item Bestimmen Sie die Teststatistik $T_{calc}$ eines gepaarten t-Tests f{\"u}r den
  Vergleich der beiden Zeitpunkte. Geben Sie den Rechenweg und die Formeln
  mit an! \textbf{(4 Punkte)}
\item Treffen Sie mit $T_{\alpha = 5\%} = 2.04$ und dem berechneten $T_{calc}$ eine Aussage
  zur Nullhypothese! \textbf{(2 Punkte)}
\item Sch{\"a}tzen Sie den $p$-Werte aus Ihrem berechneten $T_{calc}$ ab! Begr{\"u}nden Sie Ihre Antwort! \textbf{(2
    Punkte)}
\end{enumerate} 
\clearpage
% -----------------------------------------------------------------------

\section{Aufgabe \hfill (10 Punkte)}

%% --------------------------------------------------------------------
\hfill\href{https://youtu.be/exDo7AyHl4Q}{\includegraphics[width =
  2cm]{img/youtube}}\\[1Ex]
%% --------------------------------------------------------------------

Sie erhalten folgende \Rlogo Ausgabe der Funktion \texttt{t.test()}.

\begin{knitrout}
\definecolor{shadecolor}{rgb}{0.969, 0.969, 0.969}\color{fgcolor}\begin{kframe}
\begin{verbatim}
## 
## 	Two Sample t-test
## 
## data:  weight by group
## t = 0.67048, df = 12, p-value = 0.5152
## alternative hypothesis: true difference in means between group high and group low is not equal to 0
## 95 percent confidence interval:
##  -2.249639  4.249639
## sample estimates:
## mean in group ctrl  mean in group low 
##           17.42857           16.42857
\end{verbatim}
\end{kframe}
\end{knitrout}


\begin{enumerate}
  \item Formulieren Sie das statistische Hypothesenpaar! \textbf{(2
Punkte)}
\item Liegt ein signifikanter Unterschied zwischen den Gruppen vor?
  Begr{\"u}nden Sie Ihre Antwort! \textbf{(2 Punkte)}
\item Skizieren Sie eine Abbildung in der Sie $T_{calc}$, $Pr(D|H_0)$, $A=0.95$,
  sowie $T_{\alpha=5\%} = |2.18|$ einzeichnen! \textbf{(4 Punkte)}
\item Beschriften Sie die Abbildung entsprechend! \textbf{(2 Punkte)}  
\end{enumerate} 
\clearpage
% -----------------------------------------------------------------------

\section{Aufgabe \hfill (8 Punkte)}

%% --------------------------------------------------------------------
\hfill\href{https://youtu.be/wJqsNV1hOW8}{\includegraphics[width =
  2cm]{img/youtube}}\\[1Ex]
%% --------------------------------------------------------------------

Sie erhalten folgende \Rlogo Ausgabe der Funktion \texttt{t.test()}.

\begin{knitrout}
\definecolor{shadecolor}{rgb}{0.969, 0.969, 0.969}\color{fgcolor}\begin{kframe}
\begin{verbatim}
## 
## 	Two Sample t-test
## 
## data:  freshmatter by N
## t = 1.6726, df = 12, p-value = 0.1202
## alternative hypothesis: true difference in means between group high and group low is not equal to 0
## 95 percent confidence interval:
##  -0.7868426  5.9868426
## sample estimates:
## mean in group ctrl  mean in group low 
##               17.0               14.4
\end{verbatim}
\end{kframe}
\end{knitrout}


\begin{enumerate}
  \item Formulieren Sie die wissenschaftliche Fragestellung! \textbf{(2
Punkte)}
\item Liegt ein signifikanter Unterschied zwischen den Gruppen vor?
  Begr{\"u}nden Sie Ihre Antwort! \textbf{(2 Punkte)}
\item Skizieren Sie das sich ergebende 95\% Konifidenzintervall! \textbf{(2 Punkte)}
\item Beschriften Sie die Abbildung und
  das 95\% Konfidenzintervall entsprechend! \textbf{(2 Punkte)}  
\end{enumerate} 
\clearpage
% -----------------------------------------------------------------------

\section{Aufgabe \hfill (8 Punkte)}

%% --------------------------------------------------------------------
\hfill\href{https://youtu.be/w62HJlbN28U}{\includegraphics[width =
  2cm]{img/youtube}}\\[1Ex]
%% --------------------------------------------------------------------

Sie erhalten folgende \Rlogo Ausgabe der Funktion \texttt{t.test()}.

\begin{knitrout}
\definecolor{shadecolor}{rgb}{0.969, 0.969, 0.969}\color{fgcolor}\begin{kframe}
\begin{verbatim}
## 
## 	Two Sample t-test
## 
## data:  freshmatter by N
## t = 1.3646, df = 8, p-value = 0.2095
## alternative hypothesis: true difference in means between group high and group low is not equal to 0
## 95 percent confidence interval:
##  -1.241828  4.841828
## sample estimates:
## mean in group ctrl mean in group trt1 
##               18.8               17.0
\end{verbatim}
\end{kframe}
\end{knitrout}


\begin{enumerate}
  \item Formulieren Sie das statistische Hypothesenpaar! \textbf{(2
Punkte)}
\item Liegt ein signifikanter Unterschied zwischen den Gruppen vor?
  Begr{\"u}nden Sie Ihre Antwort! \textbf{(2 Punkte)}
\item Skizieren Sie die sich ergebenden Boxplots! Welche Annahmen an die Daten haben Sie getroffen? Begr{\"u}nden Sie Ihre Antwort! \textbf{(4 Punkte)} 
\end{enumerate}
 
\clearpage
% -----------------------------------------------------------------------

\section{Aufgabe \hfill (8 Punkte)}

%% --------------------------------------------------------------------
\hfill\href{https://youtu.be/kHmfEmU6lrk}{\includegraphics[width =
  2cm]{img/youtube}}\\[1Ex]
%% --------------------------------------------------------------------


Sie erhalten folgende \Rlogo Ausgabe der Funktion \texttt{t.test()}.

\begin{knitrout}
\definecolor{shadecolor}{rgb}{0.969, 0.969, 0.969}\color{fgcolor}\begin{kframe}
\begin{verbatim}
## 
## 	Paired t-test
## 
## data:  freshmatter by N
## t = -2.0167, df = 4, p-value = 0.1139
## alternative hypothesis: true mean difference is not equal to 0
## 95 percent confidence interval:
##  -10.457488   1.657488
## sample estimates:
## mean difference 
##            -4.4
\end{verbatim}
\end{kframe}
\end{knitrout}


\begin{enumerate}
  \item Formulieren Sie das statistische Hypothesenpaar! \textbf{(2
Punkte)}
\item Liegt ein signifikanter Unterschied zwischen den Gruppen vor?
  Begr{\"u}nden Sie Ihre Antwort! \textbf{(2 Punkte)}
\item Skizzieren Sie den sich ergebenden Datensatz mit $n = 4$
  Beobachtungen! Die Daten m{\"u}ssen \textit{nicht} die Mittelwertsdifferenz
  $d$ erf{\"u}llen! \textbf{(2 Punkte)} 
\item Skizieren Sie den sich ergebenden Boxplot der Differenzen! Welche Annahmen an die Daten haben Sie getroffen? Begr{\"u}nden Sie Ihre Antwort! \textbf{(2 Punkte)} 
\end{enumerate}
 
\clearpage
% -----------------------------------------------------------------------
\begin{graybox}{Die ANOVA}
Mehr Informationen zu den Aufgaben in den folgenden Kapiteln aus dem Skript Bio Data Science.
  \begin{itemize}
  \item \href{https://jkruppa.github.io/stat-tests-anova.html}{Kapitel 23 - Die ANOVA}
  \end{itemize}
\end{graybox}
\clearpage
% -----------------------------------------------------------------------

\section{Aufgabe \hfill (8 Punkte)}

%% --------------------------------------------------------------------
\hfill\href{https://youtu.be/Q7xtQJoOmQI}{\includegraphics[width =
  2cm]{img/youtube}}\\[1Ex]
%% --------------------------------------------------------------------

In einem Experiment wurde der Ertrag von Erbsen unter drei verschiedenen
Pestizid-Dosen 0.5 g/l, 1.5 g/l und 2.5 g/l gemessen. Unten stehenden sehen
Sie die Visualisierung des Datensatzes.

\begin{knitrout}
\definecolor{shadecolor}{rgb}{0.969, 0.969, 0.969}\color{fgcolor}

{\centering \includegraphics[width=\maxwidth]{img/anova-01-a-1} 

}


\end{knitrout}

\begin{enumerate}
\item Zeichnen Sie folgende statistischen Masszahlen in die Abildung ein!
  Beschriften Sie die statistischen Ma{\ss}zahlen! \textbf{(6 Punkte)}
  \begin{itemize}
  \item Total (grand) mean: $\beta_0$
  \item Mittelwerte der Dosen: $\bar{y}_{0.5}$, $\bar{y}_{1.5}$ und $\bar{y}_{2.5}$
  \item Effekt der einzelnen Level der Dosen: $\beta_{0.5}$, $\beta_{1.5}$,
    und $\beta_{2.5}$
  \item Residuen oder Fehler: $\epsilon$
  \end{itemize}
\item Liegt ein \textit{vermutlicher} signifikanter Unterschied zwischen
  den Dosisstufen vor? Begr{\"u}nden Sie Ihre Antwort! \textbf{(2 Punkte)}
\end{enumerate}
 
\clearpage
% -----------------------------------------------------------------------

\section{Aufgabe \hfill (13 Punkte)}

%% --------------------------------------------------------------------
\hfill\href{https://youtu.be/49hvImMwVyE}{\includegraphics[width =
  2cm]{img/youtube}}\\[1Ex]
%% --------------------------------------------------------------------


Der Datensatz \texttt{plant\_growth\_tbl} enth{\"a}lt das Gewicht der Kohlk{\"o}pfe
(\textit{weight}), die unter einer Kontrolle und zwei verschiedenen
Behandlungsbedingungen erzielt wurden -- dem Faktor \textit{group} mit den
Faktorstufen \textit{ctrl}, \textit{trt1}, \textit{trt2}.



\begin{enumerate}
\item F{\"u}llen Sie die unterstehende einfaktorielle ANOVA Ergebnistabelle aus
  mit den gegebenen Informationen von \texttt{Df} und \texttt{Sum Sq}!
  \textbf{(4 Punkte)}
\item Sch{\"a}tzen Sie den p-Wert der Tabelle mit der Information von
  $F_{\alpha = 5\%} = 3.44$ ab. Begr{\"u}nden Sie Ihre
  Antwort! \textbf{(2 Punkte)}
\end{enumerate}

\vspace{1Ex}

\begin{center}
  \Large
  \begin{tabular}{l|c|c|c|c|c}
     & \textbf{Df} & \textbf{Sum Sq} & \textbf{Mean Sq} & \textbf{F value} & \textbf{Pr(>F)} \strut\\
    \hline
   \textbf{group}  & 2 & 0.93 &  &  &  \strut\\
    \hline
   \textbf{Residuals}  & 22 & 70.83 &  &  &  \strut\\
  \end{tabular}
\end{center}

\vspace{1Ex}

\begin{enumerate}
  \setcounter{enumi}{2}
\item Was bedeutet ein signifikantes Ergebnis in einer einfaktoriellen
  ANOVA im Bezug auf die m{\"o}glichen Unterschiede zwischen den Gruppen? Beziehen Sie sich auf den obigen Fragetext bei Ihrer Antwort!
  \textbf{(2 Punkte)}
\item Berechnen Sie \textit{einen} Student t-Test mit f{\"u}r den \textit{vermutlich}
  signifikantesten Gruppenvergleich anhand der untenstehenden Tabelle mit
  $T_{\alpha = 5\%} = 2.03$. Begr{\"u}nden Sie Ihre Auswahl! \textbf{(3 Punkte)}
\end{enumerate}

\begin{knitrout}
\definecolor{shadecolor}{rgb}{0.969, 0.969, 0.969}\color{fgcolor}\begin{table}[!h]
\centering
\begin{tabular}{cccc}
\toprule
group & n & mean & sd\\
\midrule
ctrl & 9 & 19.11 & 0.78\\
trt1 & 9 & 19.44 & 0.88\\
trt2 & 7 & 19.57 & 3.15\\
\bottomrule
\end{tabular}
\end{table}

\end{knitrout}

\begin{enumerate}
  \setcounter{enumi}{4}
\item Gegebenen der ANOVA Tabelle war das Ergebnis des t-Tests zu erwarten?
  Begr{\"u}nden Sie Ihre Antwort! \textbf{(2 Punkte)}
\end{enumerate}

 
\clearpage
% -----------------------------------------------------------------------

\section{Aufgabe \hfill (9 Punkte)}

%% --------------------------------------------------------------------
\hfill\href{https://youtu.be/d4CFR2MKX7I}{\includegraphics[width =
  2cm]{img/youtube}}\\[1Ex]
%% --------------------------------------------------------------------

Der Datensatz \textit{crop\_tbl} enth{\"a}lt das Trockengewicht der
Maispflanzen (\textit{drymatter}), die unter drei 
verschiedenen D{\"u}ngerbedingungen erzielt wurden. Die D{\"u}ngerbedingungen sind in dem Faktor
\textit{trt} mit den Faktorstufen \textit{ctrl},  \textit{B} und
 \textit{trt2} codiert. Sie erhalten folgenden Output in \Rlogo.

\begin{knitrout}
\definecolor{shadecolor}{rgb}{0.969, 0.969, 0.969}\color{fgcolor}\begin{kframe}
\begin{verbatim}
## Analysis of Variance Table
## 
## Response: drymatter
##           Df  Sum Sq Mean Sq F value  Pr(>F)
## trt        2  37.841  18.921  3.6463 0.04289
## Residuals 22 114.159   5.189
\end{verbatim}
\end{kframe}
\end{knitrout}

\begin{enumerate}
\item Stellen Sie die statistische $H_0$ und $H_A$ Hypothese f{\"u}r die obige
  einfaktorielle ANOVA auf! \textbf{(2 Punkte)}
\item Interpretieren Sie das Ergebnis der einfaktoriellen ANOVA! \textbf{(2 Punkt)} 
\item Berechen Sie den Effektsch{\"a}tzer $\eta^2$. Was sagt Ihnen der Wert von
  $\eta^2$ aus? \textbf{(2 Punkte)}
\item Skizieren Sie eine Abbildung, der dem obigen Ergebnis der
  einfaktoriellen ANOVA n{\"a}herungsweise entspricht! \textbf{(3 Punkte)}
\end{enumerate}

 
\clearpage
% -----------------------------------------------------------------------

\section{Aufgabe \hfill (6 Punkte)}

%% --------------------------------------------------------------------
\hfill\href{https://youtu.be/zDK2dhgtFt0}{\includegraphics[width =
  2cm]{img/youtube}}\\[1Ex]
%% --------------------------------------------------------------------


Sie haben ein Experiment mit drei Behandlungen (A, B und C) und vier
Bl{\"o}cken (I, II, III und IV) durchgef{\"u}hrt. Insgesamt haben Sie die Wuchsh{\"o}he
von zw{\"o}lf Sonnenblumen bestimmt. Im Folgenden sehen Sie die Wuchsh{\"o}hen in
[cm] aus dem Experiment.


\begin{knitrout}
\definecolor{shadecolor}{rgb}{0.969, 0.969, 0.969}\color{fgcolor}\begin{kframe}
\begin{verbatim}
##  [1] 118 129 131 113 101  97 134 145 138 142 144 136
\end{verbatim}
\end{kframe}
\end{knitrout}

Erstellen Sie vier Zeichnungen des experimentellen Designs und beachten
Sie folgende Angaben zu der Quelle der erkl{\"a}rten Varianz. 

\begin{enumerate}
\item Ordnen Sie die Pflanzen so in den vier Bl{\"o}cken und drei Behandlungen an,
  \begin{enumerate}
  \item[(1)] dass die Bl{\"o}cke \textit{kaum} Varianz erkl{\"a}ren. \textbf{(1 Punkt)}
  \item[(2)] dass die Bl{\"o}cke \textit{viel} Varianz erkl{\"a}ren. \textbf{(1 Punkt)}  
  \item[(3)] dass die Behandlungen \textit{kaum} Varianz erkl{\"a}ren. \textbf{(1 Punkt)}
  \item[(4)] dass die Behandlungen \textit{viel} Varianz erkl{\"a}ren. \textbf{(1 Punkt)}
  \end{enumerate}
\item Wenn Sie ein geplantes Experiment durchf{\"u}hren, wie viel Varianz soll dann von
  den Bl{\"o}cken und den Behandlungen jeweils erkl{\"a}rt werden? Begr{\"u}nden Sie
  Ihre Antwort! \textbf{(2 Punkte)}
\end{enumerate}
 
\clearpage
% -----------------------------------------------------------------------

\section{Aufgabe \hfill (12 Punkte)}

%% --------------------------------------------------------------------
\hfill\href{https://youtu.be/8Pb2sKUIMyk}{\includegraphics[width =
  2cm]{img/youtube}}\\[1Ex]
%% --------------------------------------------------------------------



Der Datensatz \textit{tooth\_tbl} enth{\"a}lt Daten aus einer Studie zur
Bewertung der Wirkung von Vitamin C auf das Zahnwachstum bei
Meerschweinchen. Der Versuch wurde an verschiedenen Schweinen durchgef{\"u}hrt,
wobei jedes Tier eine von 4 Vitamin-C-Dosen \textit{dose}
{\"u}ber eine von 2 Verabreichungsmethoden \textit{supp}
erhielt. Die Zahnl{\"a}nge wurde als normalverteiltes Outcome gemessen.



\begin{enumerate}
\item F{\"u}llen Sie die unterstehende zweifaktorielle ANOVA Ergebnistabelle aus
  mit den gegebenen Informationen von \texttt{Df} und \texttt{Sum Sq}!
  \textbf{(4 Punkte)}
\item Sch{\"a}tzen Sie den p-Wert der Tabelle mit der Information von den
  kritischen F-Werten mit
  $F_{supp} = 4.15$ und
  $F_{dose} = 2.9$ sowie
  $F_{supp:dose} = 2.9$ ab. Begr{\"u}nden Sie Ihre
  Antwort! \textbf{(4 Punkte)}
\end{enumerate}

\vspace{1Ex}

\begin{center}
  \Large
  \begin{tabular}{l|c|c|c|c|c}
     & \textbf{Df} & \textbf{Sum Sq} & \textbf{Mean Sq} & \textbf{F value} & \textbf{Pr(>F)} \strut\\
    \hline
   \textbf{supp}  & 1 & 0.29 &  &  &  \strut\\
    \hline
    \textbf{dose}  & 3 & 253.64 &  &  &  \strut\\
    \hline
    \textbf{supp:dose}  & 3 & 157.11 &  &  &  \strut\\
    \hline
   \textbf{Residuals}  & 32 & 165.15 &  &  &  \strut\\
  \end{tabular}
\end{center}

\vspace{1Ex}

\begin{enumerate}
  \setcounter{enumi}{2}
\item Was bedeutet ein signifikantes Ergebnis in einer zweifaktoriellen
  ANOVA im Bezug auf die m{\"o}glichen Unterschiede zwischen den Gruppen?
  Beziehen Sie sich dabei einmal auf den Faktor \textit{supp} und einmal
  auf den Faktor \textit{dose}! \textbf{(2 Punkte)}
\item Was sagt der Term \textit{supp:dose} aus? Interpretieren Sie das
  Ergebnis des abgesch{\"a}tzten p-Wertes! \textbf{(2 Punkte)}
\end{enumerate}
 
\clearpage
% -----------------------------------------------------------------------

\section{Aufgabe \hfill (8 Punkte)}

%% --------------------------------------------------------------------
\hfill\href{https://youtu.be/rWTyHXXlYjY}{\includegraphics[width =
  2cm]{img/youtube}}\\[1Ex]
%% --------------------------------------------------------------------


Der Datensatz \textit{pig\_gain\_weight\_tbl} enth{\"a}lt Daten aus einer Studie zur Bewertung
der Wirkung vom Vitamin Selen auf das Wachstum bei Mastschweinen. Der
Versuch wurde an 40 Mastschweinen durchgef{\"u}hrt, wobei
jedes Tier eine von drei Selen-Dosen $dose$ (0.5 ng/Tag, 1 ng/Tag und 5 ng/Tag)
{\"u}ber eine von zwei Verabreichungsmethoden $form$ erhielt (Wasser oder
Festnahrung). Sie erhalten folgenden Output in \Rlogo.

\begin{knitrout}
\definecolor{shadecolor}{rgb}{0.969, 0.969, 0.969}\color{fgcolor}\begin{kframe}
\begin{verbatim}
## Analysis of Variance Table
## 
## Response: gain
##           Df Sum Sq Mean Sq F value    Pr(>F)
## dose       2  32.32   16.16  1.2681    0.3053
## form       1 417.41  417.41 32.7544 1.997e-05
## dose:form  2   5.95    2.98  0.2335    0.7941
## Residuals 18 229.39   12.74
\end{verbatim}
\end{kframe}
\end{knitrout}

\begin{enumerate}
\item Stellen Sie die statistische $H_0$ und $H_A$ Hypothese f{\"u}r die obige
  zweifaktorielle ANOVA f{\"u}r den Faktor dose
  auf! \textbf{(2 Punkte)}
\item Interpretieren Sie das Ergebnis der zweifaktoriellen ANOVA. Gehen Sie
  im besonderen auf den Term $dose:form$ ein! \textbf{(2 Punkte)}
\item Zeichnen Sie eine Abbildung, der dem obigen Ergebnis der
  zweifaktoriellen ANOVA n{\"a}herungsweise entspricht! \textbf{(4 Punkte)}
\end{enumerate}
 
\clearpage
% -----------------------------------------------------------------------

\section{Aufgabe \hfill (8 Punkte)}


%% --------------------------------------------------------------------
\hfill\href{https://youtu.be/FjjJXkFJfIY}{\includegraphics[width =
  2cm]{img/youtube}}\\[1Ex]
%% --------------------------------------------------------------------


In der untenstehenden Tabelle ist die Formel f{\"u}r den F-Test aus der ANOVA
und die Formel f{\"u}r den Student t-Test dargestellt. In der ANOVA berechnen
Sie die F-Statistik $F_{calc}$ und in dem Student t-Test die T-Statistik
$T_{calc}$.

\begin{center}
  \begin{tabular}{cc}
    $F_{calc} = \cfrac{MS_{treatment}}{MS_{error}}$ & $T_{calc} = \cfrac{\bar{y}_1 - \bar{y}_2}{s_p \cdot \sqrt{2/n_g}}$\\
  \end{tabular}
\end{center}


\begin{enumerate}
\item Erkl{\"a}ren Sie den konzeptionellen Zusammenhang zwischen der $F_{calc}$
  Statistik und $T_{calc}$ Statistik! \textbf{(2 Punkte)}
\item Visualisieren Sie eine nicht signifikante $F_{calc}$ Statistik sowie
  eine signifikante $F_{calc}$ Statistik anhand von $MS_{treatment}$ und
  $MS_{error}$! Beschriften Sie die Abbildung! \textbf{(2 Punkte)}
\item Erkl{\"a}ren Sie an der Formel des F-Tests sowie an der Abbildung warum
  das Minimum der F-Statistik 0 ist! \textbf{(2 Punkte)}
\item Wenn die F-Statistik 0 ist, spricht dies eher f{\"u}r oder gegen die
  Nullhypothese? Begr{\"u}nden Sie Ihre Antwort! \textbf{(2 Punkte)}
\end{enumerate}

 
\clearpage
% -----------------------------------------------------------------------

\section{Aufgabe \hfill (6 Punkte)}

%% --------------------------------------------------------------------
\hfill\href{https://youtu.be/2qG1Dws0MJo}{\includegraphics[width =
  2cm]{img/youtube}}\\[1Ex]
%% --------------------------------------------------------------------


Sie rechnen eine zweifaktorielle ANOVA und erhalten einen signifikanten
Interaktionseffekt zwischen den beiden Faktoren $f_1$ und $f_2$. Der Faktor
$f_1$ hat drei Level. Der Faktor $f_2$ hat dagegen nur zwei Level.




\begin{enumerate}
\item Visualisieren Sie in zwei getrennten Abbildungen eine
  mittelere und eine schwache Interaktion zwischen
  den Faktoren $f_1$ und $f_2$! \textbf{(2 Punkte)}
\item Erkl{\"a}ren Sie den Unterschied zwischen den beiden St{\"a}rken der Interaktion!
  \textbf{(2 Punkte)}
\item Wenn eine signifikante Interaktion in den Daten vorliegt, wie ist
  dann das weitere Vorgehen bei einem Posthoc-Test?
  \textbf{(2 Punkte)}
\end{enumerate}

 
\clearpage
% -----------------------------------------------------------------------

\section{Aufgabe \hfill (7 Punkte)}

%% --------------------------------------------------------------------
\hfill\href{https://youtu.be/M9Uhm67ndxM}{\includegraphics[width =
  2cm]{img/youtube}}\\[1Ex]
%% --------------------------------------------------------------------




Sie rechnen eine einfaktorielle ANOVA mit einem Faktor $f_1$ mit
f{"u}nf Leveln. Nachdem Sie die einfaktorielle ANOVA gerechnet
haben, erhalten Sie einen p-Wert von $0.078$ und eine F Statistik mit
$F_{calc} = 1.2$. Als Sie sich die Boxplots der Behandlungen anschauen,
stellen Sie fest, dass es eigentlich einen Mittelwertsunterschied zwischen
dem zweiten und ersten Level geben m{\"u}sste. Die
$IQR$-Bereiche {\"u}berlappen sich nicht und die Mediane liegen auch weit vom
globalen Mittel entfernt.


\begin{enumerate}
\item Erkl{\"a}ren Sie die Annahme der Normalverteilung und die Annahme der
  Varianzhomogenit{\"a}t f{\"u}r eine ANOVA an einer passenden Abbildung! \textbf{(2 Punkte)}
\item Visualisieren Sie die Berechnung von $F_{calc}$ am obigen Beispiel!
  \textbf{(2 Punkte)}
\item Erkl{\"a}ren Sie das Ergebnis der obigen einfaktoriellen ANOVA unter der
  Ber{\"u}cksichtigung der Annahmen an eine ANOVA! \textbf{(3 Punkte)}
\end{enumerate}

 
\clearpage
% -----------------------------------------------------------------------
  \begin{graybox}{Der $\mathcal{X}^2$-Test \& Der diagnostische Test}
Mehr Informationen zu den Aufgaben in den folgenden Kapiteln aus dem Skript Bio Data Science.
  \begin{itemize}
  \item \href{https://jkruppa.github.io/stat-tests-chi-test.html}{Kapitel 28 - Der $\mathcal{X}^2$-Test}
  \item \href{https://jkruppa.github.io/stat-tests-diagnostic.html}{Kapitel 29 - Der diagnostische Test}
  \end{itemize}
\end{graybox}
\clearpage
% -----------------------------------------------------------------------

\section{Aufgabe \hfill (10 Punkte)}

\textit{Geben Sie grunds{\"a}tzlich Formeln und Rechenweg zur L{\"o}sung der
  Teilaufgaben mit an!} \\[1Ex]

%% --------------------------------------------------------------------
\hfill\href{https://youtu.be/PVUK0zdkZkk}{\includegraphics[width =
  2cm]{img/youtube}}\\[1Ex]
%% --------------------------------------------------------------------




Nach einem Experiment ergibt sich die folgende 2x2 Datentabelle mit einem
Pestizid (ja/nein), dargestellt in den Zeilen. Im Weiteren mit dem
infizierten Pflanzenstatus (ja/nein) in den Spalten. Insgesamt wurden
$n = 150$ Pflanzen untersucht.
\vspace{5Ex}

\begin{center}
  \Large
  \begin{tabular}{c|c|c|c}
     & \textbf{Erkrankt (ja)} & \textbf{Erkrankt (nein)} &  \strut\\
    \hline
    \textbf{Pestizid (ja)} & 56  & 19  &     \strut\\
    \hline
    \textbf{Pestizid (nein)} & 23  & 52  &      \strut\\
    \hline
     \phantom{100} & \phantom{100}  & \phantom{100}  &  \phantom{100}  \strut\\
  \end{tabular}
\end{center}

\vspace{5Ex}

\begin{enumerate}
\item Erg{\"a}nzen Sie die Tabelle um die Randsummen! \textbf{(1 Punkt)} 
\item Formulieren Sie die Fragestellung! \textbf{(1 Punkt)}
\item Formulieren Sie das Hypothesenpaar! \textbf{(2 Punkte)}
\item Berechnen Sie die Teststatistik eines Chi-Quadrat-Test auf der 2x2
  Tafel. Geben Sie Formeln und Rechenweg mit an! \textbf{(4 Punkte)}
\item Treffen Sie eine Entscheidung im Bezug zu der Nullhypothese gegeben
  einem $\mathcal{X}^2_{\alpha = 5\%} = 3.841$! \textbf{(1 Punkt)}
\item Skizzieren Sie eine 2x2 Tabelle mit
  $n = 34$ Pflanzen in dem \textit{vermutlich}
  die Nullhypothese nicht abgelehnt werden kann! \textbf{(1 Punkt)}
\end{enumerate} 
\clearpage
% -----------------------------------------------------------------------

\section{Aufgabe \hfill (7 Punkte)}

\textit{Geben Sie grunds{\"a}tzlich Formeln und Rechenweg zur L{\"o}sung der
  Teilaufgaben mit an!} \\[1Ex]

%% --------------------------------------------------------------------
\hfill\href{https://youtu.be/jakM7fHyZfU}{\includegraphics[width =
  2cm]{img/youtube}}\\[1Ex]
%% --------------------------------------------------------------------




Gegeben sind folgende Randsummen in einer 2x2 Kreuztabelle aus einem
Experiment mit $n = 123$ Sauen. In dem Experiment wurde gemessen,
ob eine Sau nach einer Behandlung mit einem Medikament (ja/nein)
mehr als 30 Ferkel pro Jahr bekommen konnte (ja/nein).

\vspace{5Ex}

\begin{center}
  \Large
  \begin{tabular}{c|c|c|c}
     & \textbf{>30 Ferkel (ja)} & \textbf{$\leq$30 Ferkel (nein)} &  \strut\\
    \hline
    \textbf{Medikament (ja)} & \phantom{100}  & \phantom{100}  &   53  \strut\\
    \hline
    \textbf{Medikament (nein)} & \phantom{100}  & \phantom{100}  &   70   \strut\\
    \hline
     &  60 &  63 &  123  \strut\\
  \end{tabular}
\end{center}



\vspace{5Ex}

\begin{enumerate}
\item Erg{\"a}nzen Sie die Felder innerhalb der 2x2 Kreuztabelle in dem Sinne,
  dass \textit{kein} signifikanter Effekt zu erwarten w{\"a}re!
  \textbf{(2 Punkte)}
\item Erkl{\"a}ren und Begr{\"u}nden Sie Ihr Vorgehen an der Formel des
  Chi-Quadrat-Tests mit
  \begin{equation*}
  \mathcal{X}^2 = \sum\tfrac{(O - E)^2}{E}.  
  \end{equation*}
  Sie k{\"o}nnen dies an einem Beispiel erkl{\"a}ren! \textbf{(2 Punkte)}
\item Was ist die Mindestanzahl an Beobachtungen je Zelle? Wenn in einer
  der Zellen weniger Beobachtungen auftreten, welchen Test k{\"o}nnen Sie
  anstatt des "`normalen"' Chi-Quadrat-Tests anwenden? \textbf{(2 Punkte)}
\item Warum hat die obige Vierfeldertafel einen Freiheitsgrad von $df=1$?
  \textbf{(1 Punkt)}
\end{enumerate} 
\clearpage
% -----------------------------------------------------------------------

\section{Aufgabe \hfill (11 Punkte)}

%% --------------------------------------------------------------------
\hfill\href{https://youtu.be/VQlNl8hvRII}{\includegraphics[width =
  2cm]{img/youtube}}\\[1Ex]
%% --------------------------------------------------------------------


Die Pr{\"a}valenz von Klauenseuche bei Wollschweinen wird mit
4\% angenommen. In 85\% der F{\"a}lle ist ein Test positiv, wenn das Wollschwein erkrankt
ist. In 7.5\% der F{\"a}lle ist ein Test positiv,
wenn das Wollschwein \textit{nicht} erkrankt ist und somit gesund ist. Sie
werten 1000 Wollschweine mit einem
diagnostischen Test auf Klauenseuche aus.



\begin{enumerate}
\item F{\"u}llen und beschriften Sie den untenstehenden Doppelbaum! Beschriften
  Sie auch die {\"A}ste des Doppelbaumes, mit denen Ihnen bekannten
  Informationen!  \textbf{(8 Punkte)}
\item Berechnen Sie die Wahrscheinlichkeit $Pr(K^+|T^+)$! \textbf{(2 Punkte)}
\item Was sagt Ihnen die Wahrscheinlichkeit $Pr(K^+|T^+)$ aus? \textbf{(1 Punkt)}
\end{enumerate}

\vspace{1cm}

\begin{center}
  \includegraphics[width=17cm]{/Users/kruppajo/Documents/GitHub/exam/question/img/diag-doppelbaum}
\end{center}



 
\clearpage
% -----------------------------------------------------------------------

\section{Aufgabe \hfill (12 Punkte)}


%% --------------------------------------------------------------------
\hfill\href{https://youtu.be/_7s44pbOc00}{\includegraphics[width =
  2cm]{img/youtube}}\\[1Ex]
%% --------------------------------------------------------------------





Folgender diagnostischer Doppelbaum nach der Testung auf Klauenseuche bei
Fleckvieh ist gegeben.

\begin{enumerate}
\item F{\"u}llen und beschriften Sie den untenstehenden Doppelbaum! \textbf{(4
    Punkte)}
\item Berechnen Sie die Wahrscheinlichkeit $Pr(K^+|T^+)$! \textbf{(2 Punkte)}
\item Berechnen Sie die Pr{\"a}valenz f{\"u}r Klauenseuche! \textbf{(2 Punkte)}
\item Berechnen Sie die Sensifit{\"a}t und Spezifit{\"a}t des diagnostischen Tests
  f{\"u}r Klauenseuche! Erstellen Sie daf{\"u}r zun{\"a}chst eine 2x2 Kreuztabelle aus
  dem ausgef{\"u}llten Doppelbaum!
  \textbf{(4 Punkte)}
\end{enumerate}

\vspace{1cm}
 
\begin{tikzpicture}
  \node (image) at (0,0) {
    \includegraphics[width=\textwidth]{/Users/kruppajo/Documents/GitHub/exam/question/img/diag-doppelbaum}
  };
  \node[] at (-4.8,0) {\huge 160};
  \node[] at (-1.7,0) {\huge 60};
  \node[] at (1.7,0) {\huge 560};
  \node[] at (4.75,0) {\huge 1800};
\end{tikzpicture}




 
\clearpage
% -----------------------------------------------------------------------

\section{Aufgabe \hfill (10 Punkte)}

%% --------------------------------------------------------------------
\hfill\href{https://youtu.be/G-_r2KplGTI}{\includegraphics[width =
  2cm]{img/youtube}}\\[1Ex]
%% --------------------------------------------------------------------


Beim diagnostischen Testen erhalten Sie \textit{True Positives (TP)},
\textit{True Negatives (TN)}, \textit{False Positives (FP)} und
\textit{False Negatives (FN)}. Erkl{\"a}ren Sie den Zusammenhang wir folgt.

\begin{enumerate}
\item Tragen Sie \textit{TP}, \textit{TN}, \textit{FP} und \textit{FN} in
  eine 2x2 Kreuztablle ein. Beschriften Sie die Tabelle entsprechend!
  \textbf{(2 Punkte)}
\item Visualisieren Sie \textit{TP}, \textit{TN}, \textit{FP} und
  \textit{FN} in einer Abbildung. Beschriften Sie die Abbildung und die
  Achsen entsprechend! \textbf{(4 Punkte)}
\item Erkl{\"a}ren Sie an einem numerischen Beispiel und der Abbildung die
  Berechnung der Pr{\"a}valenz!  \textbf{(2 Punkte)}
\item Erkl{\"a}ren Sie an einem Schaubild den Unterschied zwischen Inzidenz und
  Pr{\"a}valenz!  \textbf{(2 Punkte)}
\end{enumerate}





 
\clearpage
% -----------------------------------------------------------------------
\begin{graybox}{Simple lineare Regression}
Mehr Informationen zu den Aufgaben in den folgenden Kapiteln aus dem Skript Bio Data Science.
  \begin{itemize}
  \item \href{https://jkruppa.github.io/stat-linear-reg-basic.html}{Kapitel 32 - Simple lineare Regression}
  \item \href{https://jkruppa.github.io/stat-linear-reg-quality.html}{Kapitel 33 - Maßzahlen der Modelgüte}
  \item \href{https://jkruppa.github.io/stat-linear-reg-corr.html}{Kapitel 34 - Korrelation}
  \end{itemize}
\end{graybox}
\clearpage
% -----------------------------------------------------------------------

\section{Aufgabe \hfill (7 Punkte)}

\textit{Geben Sie grunds{\"a}tzlich Formeln und Rechenweg zur L{\"o}sung der
  Teilaufgaben mit an!} \\[1Ex]

%% --------------------------------------------------------------------
\hfill\href{https://youtu.be/2dUJcYK9RgU}{\includegraphics[width =
  2cm]{img/youtube}}\\[1Ex]
%% --------------------------------------------------------------------

In einer Studie zur "`Arbeitssicherheit auf dem Feld"' wurde gemessen wie viele
Stunden auf einem Feld gefahren wurden und wie oft der Fahrer dabei drohte
einzunicken. Es ergab sich folgende Abbildung. 



{\centering \includegraphics[width=\maxwidth]{img/scatter-02-1} 

}




\begin{enumerate}
\item Erstellen Sie die Regressionsgleichung aus der obigen Abbildung in
  der Form $y \sim \beta_0 + \beta_1 \cdot x$! \textbf{(2 Punkte)}
\item Beschriften Sie die Gerade mit den Parametern der linearen
  Regressionsgleichung! \textbf{(2 Punkte)}
\item Liegt ein Zusammenhang zwischen der Anzahl an gefahrenen Runden und
  der M{\"u}digkeit vor? Begr{\"u}nden Sie Ihre Antwort! \textbf{(2 Punkte)}
\item Wenn kein Zusammenhang zu beobachten w{\"a}re, wie w{\"u}rde die Gerade aussehen? \textbf{(1 Punkt)}
\end{enumerate} 
\clearpage
% -----------------------------------------------------------------------

\section{Aufgabe \hfill (10 Punkte)}

\textit{Geben Sie grunds{\"a}tzlich Formeln und Rechenweg zur L{\"o}sung der
  Teilaufgaben mit an!} \\[1Ex]

%% --------------------------------------------------------------------
\hfill\href{https://youtu.be/lJp8rFmMnrs}{\includegraphics[width =
  2cm]{img/youtube}}\\[1Ex]
%% --------------------------------------------------------------------



In einem Stallexperiment mit $n = 40$ Ferkeln wurde der
Gewichtszuwachs unter bestimmten Lichtverh{\"a}ltnissen gemessen. Sie erhalten
den \Rlogo Output der Funktion \texttt{tidy()} einer simplen Gaussian linearen
Regression sieben Wochen nach der ersten Messung.

\begin{table}[!h]
\centering\begingroup\fontsize{14}{16}\selectfont

\begin{tabular}{ccccc}
\toprule
term & estimate & std.error & t statistic & p-value\\
\midrule
(Intercept) & 25.62 & 1.72 &  & \\
light & 0.03 & 0.17 &  & \\
\bottomrule
\end{tabular}
\endgroup{}
\end{table}



\begin{enumerate}
\item Berechnen Sie die t Statistik f{\"u}r \textit{(Intercept)} und
  \textit{light}! \textbf{(2 Punkte)}
\item Sch{\"a}tzen Sie den p-Wert f{\"u}r \textit{(Intercept)} und
  \textit{light} mit $T_{\alpha = 5\%} = 1.96$ ab. Was sagt Ihnen der p-Wert aus?
  Begr{\"u}nden Sie Ihre Antwort! \textbf{(3 Punkte)}
\item Zeichnen Sie die Gerade aus der obigen Tabelle in ein Koordinatenkreuz! \textbf{(1 Punkt)}
\item Beschriften Sie die Abbildung und die Gerade mit den statistischen
  Kenngr{\"o}{\ss}en! \textbf{(2 Punkte)}
\item Formulieren Sie die Regressionsgleichung! \textbf{(2 Punkte)}
\end{enumerate} 
\clearpage
% -----------------------------------------------------------------------

\section{Aufgabe \hfill (6 Punkte)}

%% --------------------------------------------------------------------
\hfill\href{https://youtu.be/tNNzcndrpSk}{\includegraphics[width =
  2cm]{img/youtube}}\\[1Ex]
%% --------------------------------------------------------------------

Sie erhalten folgende R Ausgabe der Funktion lm() nach einem Experiment mit
zwei Behandlungen (A und B) sowie dem Ertragsgewicht von Weizen.

\begin{knitrout}
\definecolor{shadecolor}{rgb}{0.969, 0.969, 0.969}\color{fgcolor}\begin{kframe}
\begin{verbatim}
## 
## Call:
## lm(formula = weight ~ trt, data = data_tbl)
## 
## Residuals:
##    Min     1Q Median     3Q    Max 
## -6.429 -2.107  0.500  2.143  4.571 
## 
## Coefficients:
##             Estimate Std. Error t value Pr(>|t|)
## (Intercept)   20.429      1.185  17.233 8.01e-11
## trtB           4.571      1.581   2.892   0.0118
## 
## Residual standard error: 3.136 on 14 degrees of freedom
## Multiple R-squared:  0.374,	Adjusted R-squared:  0.3293 
## F-statistic: 8.365 on 1 and 14 DF,  p-value: 0.01182
\end{verbatim}
\end{kframe}
\end{knitrout}


\begin{enumerate}
\item Ist die Annahme der Normalverteilung an das Outcome \textit{rsp} erf{\"u}llt?
  Begr{\"u}nden Sie die Antwort! \textbf{(2 Punkte)}
\item Wie gro{\ss} ist der Effekt des \textit{Trt}? Liegt ein signifikanter
  Effekt vor? Begr{\"u}nden Sie Ihre Antwort! \textbf{(2 Punkte)}
\item Schreiben Sie das Ergebnis der \Rlogo Ausgabe in einen Satz nieder, der die
  Information zum Effekt und der Signifikanz enth{\"a}lt! \textbf{(2 Punkte)} 
\end{enumerate}
 
\clearpage
% -----------------------------------------------------------------------

\section{Aufgabe \hfill (8 Punkte)}

%% --------------------------------------------------------------------
\hfill\href{https://youtu.be/C_a8aOMI7GE}{\includegraphics[width =
  2cm]{img/youtube}}\\[1Ex]
%% --------------------------------------------------------------------



\begin{enumerate}
\item Skizieren Sie in die unten stehenden, freien Abbildungen ein kausales
  und ein pr{\"a}diktives Modell mit $n = 9$
  Beobachtungen! \textbf{(4 Punkte)}
\item Beachten Sie bei der Erstellung der Skizze, ob ein Effekt von X
  vorliegt oder nicht! \textbf{(2 Punkte)}
\item Beschriften Sie die Abbildung mit "`Trainingsdaten"' und "`Testdaten"'!  \textbf{(2 Punkte)}
\end{enumerate}



{\centering \includegraphics[width=\maxwidth]{img/modeling-01-1} 

}



 
\clearpage
% -----------------------------------------------------------------------

\section{Aufgabe \hfill (9 Punkte)}

%% --------------------------------------------------------------------
\hfill\href{https://youtu.be/fB6nF4dxodA}{\includegraphics[width =
  2cm]{img/youtube}}\\[1Ex]
%% --------------------------------------------------------------------


Im folgenden sehen Sie drei leere Scatterplots. F{\"u}llen Sie diese
Scatterplots nach folgenden Anweisungen.

\begin{enumerate}
\item Zeichnen Sie f{\"u}r die angegebene $\rho$-Werte eine Gerade in die
  entsprechende Abbildung! \textbf{(3 Punkte)}
\item Zeichnen Sie f{\"u}r die angegebenen $R^2$-Werte die entsprechende
  Punktewolke um die Gerade. \textbf{(3 Punkte)}
\item Sie rechnen ein statistisches Modell. Was sagen Ihnen die $R^2$-Werte
  {\"u}ber das jeweilige Modell? \textbf{(3 Punkte)}
\end{enumerate}




{\centering \includegraphics[width=\maxwidth]{img/correlation-01-1} 

}



 
\clearpage
% -----------------------------------------------------------------------

\section{Aufgabe \hfill (9 Punkte)}

%% --------------------------------------------------------------------
\hfill\href{https://youtu.be/2QJa19ZwLls}{\includegraphics[width =
  2cm]{img/youtube}}\\[1Ex]
%% --------------------------------------------------------------------

Im folgenden sehen Sie vier Scatterplots. Erg{\"a}nzen Sie die {\"U}berschriften
der jeweiligen Scatterplots.


\begin{enumerate}
\item Sch{\"a}tzen Sie die $\rho$-Werte in der entsprechenden
  Abbildung! \textbf{(4 Punkte)}
\item Sch{\"a}tzen Sie die $R^2$-Werte in der entsprechenden
  Punktewolke um die Gerade! \textbf{(4 Punkte)}
\item Sie rechnen ein statistisches Modell. Was sagen Ihnen die $R^2$-Werte
  {\"u}ber das jeweilige Modell? \textbf{(1 Punkt)}
\end{enumerate}




{\centering \includegraphics[width=\maxwidth]{img/correlation-02-1} 

}



 
\clearpage
% -----------------------------------------------------------------------

\section{Aufgabe \hfill (12 Punkte)}

%% --------------------------------------------------------------------
\hfill\href{https://youtu.be/C9skfFRTHhI}{\includegraphics[width =
   2cm]{img/youtube}}\\[1Ex]
%% --------------------------------------------------------------------

Sie erhalten folgende \Rlogo Ausgabe der Funktion \texttt{cor.test()}.

\begin{knitrout}
\definecolor{shadecolor}{rgb}{0.969, 0.969, 0.969}\color{fgcolor}\begin{kframe}
\begin{verbatim}
## 
## 	Kendall's correlation
## 
## data:  weight and fertilizer
## t = -2.7596, df = 8, p-value = 0.02469
## alternative hypothesis: true correlation is not equal to 0
## 95 percent confidence interval:
##  -0.9223983 -0.1226558
## sample estimates:
##        cor 
## -0.6983504
\end{verbatim}
\end{kframe}
\end{knitrout}


\begin{enumerate}
  \item Formulieren Sie die wissenschaftliche Fragestellung! \textbf{(1
Punkt)}
\item Nennen Sie die zwei Eigenschaften des Korrelationskoeffizienten!
  Erkl{\"a}ren Sie \textit{eine} der Eigenschaften an einem Beispiel! \textbf{(3
    Punkte)}
\item Sind die Variablen \texttt{weight and fertilizer} normalverteilt?
  Begr{\"u}nden Sie Ihre Antwort! \textbf{(2 Punkte)}
\item Interpretieren Sie den Korrelationskoefizienten hinsichtlich des
  Effekts und der Signifikanz! Begr{\"u}nden Sie
  Ihre Antwort! \textbf{(3 Punkte)}
\item Visualisieren Sie das 95\% Konfidenzintervall! Beschriften Sie die Abbildung! \textbf{(3 Punkte)} 
\end{enumerate} 
\clearpage
% -----------------------------------------------------------------------

\section{Aufgabe \hfill (6 Punkte)}

%% --------------------------------------------------------------------
\hfill\href{https://youtu.be/EK7JEtdZbnw}{\includegraphics[width =
  2cm]{img/youtube}}\\[1Ex]
%% --------------------------------------------------------------------




\begin{enumerate}
\item Skizieren Sie in die unten stehenden, freien Abbildungen die
  Abbildung, die sich nach der {\"U}berschrift ergibt! \textbf{(4 Punkte)}
\item Beschriften Sie die Achsen entsprechend! \textbf{(2 Punkte)}
\end{enumerate}



{\centering \includegraphics[width=\maxwidth]{img/regression-03-1} 

}



 
\clearpage
% -----------------------------------------------------------------------

\section{Aufgabe \hfill (7 Punkte)}

%% --------------------------------------------------------------------
\hfill\href{https://youtu.be/cYyvOXR4qa8}{\includegraphics[width =
  2cm]{img/youtube}}\\[1Ex]
%% --------------------------------------------------------------------




\begin{enumerate}
\item Skizieren Sie in die unten stehenden, freien Abbildungen die
  Abbildung, die sich nach der {\"U}berschrift ergibt! \textbf{(4 Punkte)}
\item Beschriften Sie die Achsen entsprechend! \textbf{(3 Punkte)}
\end{enumerate}



{\centering \includegraphics[width=\maxwidth]{img/regression-04-1} 

}



 
\clearpage
% -----------------------------------------------------------------------

\section{Aufgabe \hfill (10 Punkte)}

%% --------------------------------------------------------------------
\hfill\href{https://youtu.be/dyQlYV9nOqY}{\includegraphics[width =
  2cm]{img/youtube}}\\[1Ex]
%% --------------------------------------------------------------------

Sie rechnen eine lineare Regression um nach einem Feldexperiment den
Zusammenhang zwischen Trockengewicht kg/m$^2$ (\textit{drymatter}) und
Wassergabe l/m$^2$ (\textit{water}) bei Spargel zu bestimmen. Sie erhalten
folgende Datentabelle.

\begin{knitrout}
\definecolor{shadecolor}{rgb}{0.969, 0.969, 0.969}\color{fgcolor}\begin{table}[!h]
\centering\begingroup\fontsize{12}{14}\selectfont

\begin{tabular}{ccccc}
\toprule
.id & drymatter & water & .fitted & .resid\\
\midrule
1 & 38.0 & 16.2 & 34.6 & \\
2 & 18.2 & 7.6 & 22.0 & \\
3 & 20.5 & 6.2 & 19.9 & \\
4 & 27.8 & 11.9 & 28.3 & \\
5 & 24.7 & 10.6 & 26.3 & \\
\addlinespace
6 & 27.4 & 9.3 & 24.4 & \\
7 & 23.0 & 8.9 & 23.9 & \\
8 & 31.7 & 16.2 & 34.6 & \\
9 & 26.0 & 7.9 & 22.4 & \\
10 & 15.9 & 3.6 & 16.1 & \\
\addlinespace
11 & 29.5 & 11.2 & 27.3 & \\
12 & 33.8 & 16.3 & 34.8 & \\
13 & 20.5 & 7.9 & 22.4 & \\
\bottomrule
\end{tabular}
\endgroup{}
\end{table}

\end{knitrout}

\begin{enumerate}
\item Erg{\"a}nzen Sie die Werte in der Spalte \texttt{.resid} in der obigen
  Tabelle. Geben Sie den Rechenweg und Formel mit an! \textbf{(4 Punkte)}
\item Zeichnen Sie den sich aus der obigen Tabelle ergebenden
  Residualplot. Beschriften Sie die Abbildung! \textbf{(4 Punkte)}
\item Gibt es auff{\"a}llige Werte anhand des Residualplots? Begr{\"u}nden Sie Ihre
  Antwort! \textbf{(2 Punkte)}
\end{enumerate}
 
\clearpage
% -----------------------------------------------------------------------
\begin{graybox}{Multiple lineare Regression}
Mehr Informationen zu den Aufgaben in den folgenden Kapiteln aus dem Skript Bio Data Science.
  \begin{itemize}
  \item \href{https://jkruppa.github.io/stat-modeling-basic.html}{Kapitel 35 - Multiple lineare Regression}
  \item \href{https://jkruppa.github.io/stat-modeling-gaussian.html}{Kapitel 40 - Gaussian Regression}
  \item \href{https://jkruppa.github.io/stat-modeling-poisson.html}{Kapitel 41 - Poisson Regression}
  \item \href{https://jkruppa.github.io/stat-modeling-logistic.html}{Kapitel 43 - Logistische Regression}
  \item \href{https://jkruppa.github.io/stat-modeling-mixed.html}{Kapitel 44 - Lineare gemischte Modelle}
  \end{itemize}
\end{graybox}
\clearpage
% -----------------------------------------------------------------------

\section{Aufgabe \hfill (12 Punkte)}

%% --------------------------------------------------------------------
\hfill\href{https://youtu.be/lHzRgm7hPw0}{\includegraphics[width =
  2cm]{img/youtube}}\\[1Ex]
%% --------------------------------------------------------------------



\begin{enumerate}
\item Zeichen Sie in die drei untenstehenden, leeren Abbilungen die Zeile des
  Regressionskreuzes der Poissonverteilung. W{\"a}hlen Sie die Beschriftung der
  y-Achse sowie der x-Achse entsprechend aus! \textbf{(6 Punkte)}
\item Erg{\"a}nzen Sie die jeweiligen statistischen Methoden zu der Abbildung! \textbf{(2 Punkte)}
\item Welchen Effektsch{\"a}tzer erhalten Sie aus der entsprechend linearen
  Regression bzw. den Gruppenvergleich? Geben Sie ein Beispiel! \textbf{(2 Punkte)}
\item Wenn Sie keinen Effekt erwarten, welchen \textit{Zahlenraum} nimmt dann
  der Effektsch{\"a}tzer ein? Geben Sie ein Beispiel! \textbf{(2 Punkte)}
\end{enumerate}



{\centering \includegraphics[width=\maxwidth]{img/regression-01-1} 

}



 
\clearpage
% -----------------------------------------------------------------------

\section{Aufgabe \hfill (9 Punkte)}

%% --------------------------------------------------------------------
\hfill\href{https://youtu.be/AwQEcQWLFCw}{\includegraphics[width =
  2cm]{img/youtube}}\\[1Ex]
%% --------------------------------------------------------------------



Ein Feldexperiment wurde mit $n = 200$ Pflanzen durchgef{\"u}hrt. Folgende
Einflussvariablen ($x$) wurden erhoben: center, region und P. Als m{\"o}gliche Outcomevariablen stehen Ihnen nun
folgende gemessene Endpunkte zu Verf{\"u}gung: drymatter, yield, count, quality\_score und dead.

\begin{enumerate}
\item W{\"a}hlen Sie ein Outcome was zu der Verteilungsfamilie
  \textit{Gaussian} geh{\"o}rt! \textbf{(1 Punkt)}
\item Schreiben Sie das Modell in der Form $y \sim x$ wie es in \Rlogo in
  der Funktion \texttt{glm()}
  {\"u}blich ist \textit{ohne Interaktionsterm}! \textbf{(3 Punkte)}
\item Schreiben Sie das Modell in der Form $y \sim x$ wie es in \Rlogo
  {\"u}blich ist und erg{\"a}nzen Sie \textit{einen} Interaktionsterm nach Wahl! \textbf{(1 Punkt)} 
\item Zeichen Sie eine \textit{starke}
  Interaktion in die Abbildung unten f{\"u}r den Endpunkt
  \textit{yield}. Erg{\"a}nzen Sie eine aussagekr{\"a}ftige Legende. Wie erkennen
  Sie eine Interaktion? Begr{\"u}nden Sie Ihre Antwort! \textbf{(4 Punkte)}
\end{enumerate}



{\centering \includegraphics[width=\maxwidth]{img/modeling-R-01-1} 

}


 
\clearpage
% -----------------------------------------------------------------------

\section{Aufgabe \hfill (9 Punkte)}

%% --------------------------------------------------------------------
\hfill\href{https://youtu.be/NSMrpAYzOcs}{\includegraphics[width =
  2cm]{img/youtube}}\\[1Ex]
%% --------------------------------------------------------------------



Maispflanzen sollen auf die ertragssteigerende Wirkung von verschiedenen
Einflussfaktoren untersucht werden. Gemessen wurde als Outcome die
Trockenmasse in kg/m$^2$. Daf{\"u}r wurde f{\"u}r jede Maispflanze gemessen wieviel
Wasser (l/m$^2$) die Pflanze erhalten hat oder ob die Pflanze ein
neuartiges Lichtregime (0 = alt, 1 = neu) erhalten hatte. Zus{\"a}tzlich wurde
die Anzahl an Nematoden im Boden bestimmt sowie der Eisen- und
Phosphorgehalt ($\mu$g/kg) des Bodens. Es ergibt sich folgender Auszug aus
den Daten.

\begin{knitrout}
\definecolor{shadecolor}{rgb}{0.969, 0.969, 0.969}\color{fgcolor}\begin{table}[!h]
\centering
\begin{tabular}{lrrrrr}
\toprule
water & light & P & Fe & drymatter & nematodes\\
\midrule
9.04 & 0 & 10.67 & 97.38 & 64.66 & 6\\
10.54 & 0 & 10.93 & 101.00 & 72.87 & 5\\
7.21 & 1 & 11.10 & 99.31 & 65.93 & 11\\
10.61 & 1 & 9.96 & 96.20 & 67.40 & 9\\
\bottomrule
\end{tabular}
\end{table}

\end{knitrout}

Sie rechnen nun eine Gaussian lineare Regression auf den Daten und erhalten
folgenden \Rlogo Output.

{\small
\begin{knitrout}
\definecolor{shadecolor}{rgb}{0.969, 0.969, 0.969}\color{fgcolor}\begin{kframe}
\begin{verbatim}
## 
## Call:
## lm(formula = reformulate(response = "drymatter", termlabels = wanted_vec), 
##     data = data_tbl)
## 
## Residuals:
##     Min      1Q  Median      3Q     Max 
## -5.2297 -1.5144  0.0293  1.6002  7.5267 
## 
## Coefficients:
##              Estimate Std. Error t value Pr(>|t|)
## (Intercept) -3.020177   7.062369  -0.428   0.6698
## light        0.988086   0.493658   2.002   0.0479
## Fe           0.725425   0.070997  10.218   <2e-16
## nematodes   -0.007064   0.067577  -0.105   0.9169
## 
## Residual standard error: 2.311 on 106 degrees of freedom
## Multiple R-squared:    0.5,	Adjusted R-squared:  0.4858 
## F-statistic: 35.33 on 3 and 106 DF,  p-value: 6.558e-16
\end{verbatim}
\end{kframe}
\end{knitrout}
}


\begin{enumerate}
\item Sind die Residuals approximativ Normalverteilt? Begr{\"u}nden Sie Ihre Antwort! \textbf{(3 Punkte)}  
\item Welche der Einflussfaktoren sind signifikant? Begr{\"u}nden Sie Ihre
  Antwort! \textbf{(3 Punkte)}
\item Interpretieren Sie die Spalte \textit{estimate} im Bezug auf den
  Ertrag in Trockenmasse der Maispflanzen! \textbf{(3 Punkte)}
\end{enumerate}
 
\clearpage
% -----------------------------------------------------------------------

\section{Aufgabe \hfill (10 Punkte)}

%% --------------------------------------------------------------------
\hfill\href{https://youtu.be/K_28Ne6ladI}{\includegraphics[width =
  2cm]{img/youtube}}\\[1Ex]
%% --------------------------------------------------------------------




In verschiedenen Fl{\"u}{\ss}en (\textit{stream}) wurde die Anzahl an
Knochenhechten (\textit{longnose}) gez{\"a}hlt. Daneben wurden noch andere
Eigenschaften der entspechenden Fl{\"u}sse gemessen. Es ergibt sich folgender
Auszug aus den Daten. 


\begin{knitrout}
\definecolor{shadecolor}{rgb}{0.969, 0.969, 0.969}\color{fgcolor}\begin{table}[!h]
\centering
\begin{tabular}{lrrrrr}
\toprule
stream & longnose & acerage & maxdepth & so4 & do2\\
\midrule
MUDLICK\_RUN & 8 & 1507 & 51 & 16.30 & 7.4\\
DEEP\_RUN & 5 & 1175 & 57 & 21.68 & 7.7\\
BEAVER\_RUN & 37 & 1722 & 43 & 5.91 & 8.1\\
\bottomrule
\end{tabular}
\end{table}

\end{knitrout}


Sie rechnen nun eine Poisson lineare Regression auf den Daten und erhalten
folgenden \Rlogo Output.

{\small
\begin{knitrout}
\definecolor{shadecolor}{rgb}{0.969, 0.969, 0.969}\color{fgcolor}\begin{kframe}
\begin{verbatim}
## 
## Call:
## glm(formula = reformulate(response = "longnose", termlabels = wanted_vec), 
##     family = quasipoisson, data = data_tbl)
## 
## Deviance Residuals: 
##     Min       1Q   Median       3Q      Max  
## -8.9301  -4.0162  -2.1500  -0.0069  23.8845  
## 
## Coefficients:
##              Estimate Std. Error t value Pr(>|t|)
## (Intercept) 6.017e-01  1.668e+00   0.361   0.7199
## acerage     3.212e-05  1.737e-05   1.849   0.0704
## maxdepth    3.881e-03  5.838e-03   0.665   0.5092
## so4         7.656e-03  2.432e-02   0.315   0.7542
## do2         2.539e-01  1.628e-01   1.560   0.1253
## 
## (Dispersion parameter for quasipoisson family taken to be 49.43073)
## 
##     Null deviance: 1822.9  on 53  degrees of freedom
## Residual deviance: 1510.1  on 49  degrees of freedom
## AIC: NA
## 
## Number of Fisher Scoring iterations: 5
\end{verbatim}
\end{kframe}
\end{knitrout}
}

\begin{enumerate}
\item Erkl{\"a}ren Sie warum eine Quasipoisson-Regression gerechnet wurde! \textbf{(2 Punkte)}
\item Erkl{\"a}ren Sie den Effekt der alternativen Verwendung einer Poisson-Regression auf
  den obigen \Rlogo Output!  \textbf{(2 Punkte)}
\item K{\"o}nnen Sie die \textit{Estimate} der einzelnen Einflussvariablen
  direkt interpretieren? Begr{\"u}nden Sie Ihre Antwort! \textbf{(2 Punkte)}
\item Interpretieren Sie den Effekt von \textit{no3} auf die Anzahl an Knochenhechten! Liegt ein
  signifikanter Effekt vor? Begr{\"u}nden Sie Ihre Antwort! \textbf{(4 Punkte)}
\end{enumerate}
 
\clearpage
% -----------------------------------------------------------------------

\section{Aufgabe \hfill (10 Punkte)}


%% --------------------------------------------------------------------
\hfill\href{https://youtu.be/PVUK0zdkZkk}{\includegraphics[width =
  2cm]{img/youtube}}\\[1Ex]
%% --------------------------------------------------------------------




Auf einer Erdbeerplantage treten unerwartet h{\"a}ufig infizierte
Erdbeerpflanzen auf. In einem Versuch sollen verschiedende Einflussfaktoren
auf den Infektionsstatus betrachtet werden. Daf{\"u}r wurde f{\"u}r jede
Erdbeerpflanze gemessen, wieviel Wasser die Pflanze erhalten hat oder ob
die Pflanze ein neuartiges Lichtregime erhalten hatte. Zus{\"a}tzlich wurde die
Anzahl an Nematoden im Boden bestimmt. Es ergibt sich folgender Auszug aus
den Daten.

\begin{knitrout}
\definecolor{shadecolor}{rgb}{0.969, 0.969, 0.969}\color{fgcolor}\begin{table}[!h]
\centering
\begin{tabular}{lrrr}
\toprule
infected & water & light & nematodes\\
\midrule
1 & 10.55 & 0 & 3\\
1 & 13.75 & 0 & 1\\
0 & 10.02 & 0 & 0\\
1 & 9.43 & 0 & 1\\
\bottomrule
\end{tabular}
\end{table}

\end{knitrout}

Sie rechnen nun eine logistische lineare Regression auf den Daten und erhalten
folgenden \Rlogo Output.

\begin{knitrout}
\definecolor{shadecolor}{rgb}{0.969, 0.969, 0.969}\color{fgcolor}\begin{kframe}
\begin{verbatim}
## # A tibble: 3 x 4
##   term        std.error statistic p.value
##   <chr>           <dbl>     <dbl>   <dbl>
## 1 (Intercept)     0.407     0.635 0.525  
## 2 light           0.477     2.81  0.00499
## 3 nematodes       0.151    -1.30  0.193
\end{verbatim}
\end{kframe}
\end{knitrout}


\begin{enumerate}
\item Die Spalte \textit{estimate} wurde gel{\"o}scht. Berechnen Sie die Werte
  der Spalte \textit{estimate} aus den \Rlogo Output! \textbf{(2 Punkte)}
\item Welche Einflussfaktoren sind protektiv, welche ein Risiko? Berechnen
  Sie hierf{\"u}r zun{\"a}chst das OR aus der Spalte \textit{estimate}! \textbf{(4 Punkte)}
\item Interpretieren Sie die Spalte \textit{estimate} im Bezug auf den
  Infektionsstatus der Erdbeerpflanzen! \textbf{(2 Punkte)}
\item Was ist der Unterschied zwischen einem OR und einem RR? Geben Sie ein
  numerisches Beispiel! \textbf{(2 Punkte)}
\end{enumerate}
 
\clearpage
% -----------------------------------------------------------------------

\section{Aufgabe \hfill (11 Punkte)}

%% --------------------------------------------------------------------
\hfill\href{https://youtu.be/ysai7umvPoA}{\includegraphics[width =
  2cm]{img/youtube}}\\[1Ex]
%% --------------------------------------------------------------------


In einem Experiment zur Steigerung der Milchleistung (\textit{gain}) in $dl/h$ von
K{\"u}hen wurden zwei Arten von Musik in den St{\"a}llen gespielt. Zum einen ruhige
Musik (\textit{calm}) und eher flotte Musik (\textit{pop}). Die Messungen
wurden an jeder Kuh (\textit{subject}) wiederholt durchgef{\"u}hrt. Dar{\"u}ber
hinaus wurden verschiedene St{\"a}lle (\textit{barn}) mit der Musik bespielt.

\begin{knitrout}
\definecolor{shadecolor}{rgb}{0.969, 0.969, 0.969}\color{fgcolor}\begin{kframe}
\begin{verbatim}
## Linear mixed model fit by REML ['lmerMod']
## Formula: gain ~ attitude + (1 | subject) + (1 | barn)
##    Data: data_tbl
## 
## REML criterion at convergence: 793.9
## 
## Scaled residuals: 
##     Min      1Q  Median      3Q     Max 
## -2.0486 -0.5947 -0.0929  0.6177  3.5937 
## 
## Random effects:
##  Groups   Name        Variance Std.Dev.
##  barn     (Intercept)  222.3   14.91   
##  subject  (Intercept) 4093.1   63.98   
##  Residual              648.4   25.46   
## Number of obs: 83, groups:  barn, 7; subject, 6
## 
## Fixed effects:
##             Estimate Std. Error t value
## (Intercept)  203.704     27.007   7.543
## attitudepop  -20.795      5.595  -3.717
## 
## Correlation of Fixed Effects:
##             (Intr)
## attitudepop -0.102
\end{verbatim}
\end{kframe}
\end{knitrout}


\begin{enumerate}
\item Ist die Annahme der Normalverteilung an das Outcome \textit{gain} erf{\"u}llt?
  Begr{\"u}nden Sie Ihre Antwort! \textbf{(2 Punkte)}
\item Wie gro{\ss} ist der Effekt der Musikart \textit{attitude}? Liegt ein signifikanter
  Effekt vor? Sch{\"a}tzen Sie den p-Wert mit einem kritischen t-Wert von $T_k
  = 1.96$ ab. Begr{\"u}nden und visualisieren Sie Ihre Antwort und
  Entscheidung! \textbf{(3 Punkte)}
\item Was ist der Unterschied zwischen einem "`random"' und "`fixed"'
  Effekt. Gehen Sie in der Begr{\"u}ndung Ihrer Antwort auf dieses konkrete
  Beispiel ein! \textbf{(3 Punkte)}
\item Wie gro{\ss} ist die Varianz, die durch die zuf{\"a}lligen Effekte erkl{\"a}rt wird? \textbf{(1 Punkt)}
\item Schreiben Sie das Ergebnis der \Rlogo Ausgabe in einen Satz nieder, der die
  Information zum Effekt und der Signifikanz enth{\"a}lt! \textbf{(2 Punkte)}
\end{enumerate}
 
\clearpage
% -----------------------------------------------------------------------
\begin{graybox}{Nicht parametrische Tests}
Mehr Informationen zu den Aufgaben in den folgenden Kapiteln aus dem Skript Bio Data Science.
  \begin{itemize}
  \item \href{https://jkruppa.github.io/stat-tests-utest.html}{Kapitel 25 - Der Wilcoxon-Mann-Whitney-Test}
  \item \href{https://jkruppa.github.io/stat-tests-kruskal.html}{Kapitel 26 - Der Kruskal-Wallis-Test}
  \end{itemize}
\end{graybox}
\clearpage
% -----------------------------------------------------------------------

\section{Aufgabe \hfill (12 Punkte)}

\textit{Geben Sie grunds{\"a}tzlich Formeln und Rechenweg zur L{\"o}sung der
  Teilaufgaben mit an!} \\[1Ex]

%% --------------------------------------------------------------------
\hfill\href{https://youtu.be/ArHA6MZOEOw}{\includegraphics[width =
  2cm]{img/youtube}} %%youtube
\hspace{2Ex}
%% --------------------------------------------------------------------


Die Anzahl an Nematoden wurde vor und nach einer Behandlung mit einem
bioaktiven D{\"u}nger gez{\"a}hlt. Es ergibt sich folgende Datentabelle.

\begin{table}[!h]
\centering
\begin{tabular}{ccccccc}
\toprule
Vorher & Nachher & Differenz & Vorzeichen & Rang & Positiv Rang & Negativ Rang\\
\midrule
8 & 12 &  &  &  &  & \\
8 & 10 &  &  &  &  & \\
8 & 13 &  &  &  &  & \\
12 & 12 &  &  &  &  & \\
9 & 9 &  &  &  &  & \\
\addlinespace
8 & 16 &  &  &  &  & \\
10 & 10 &  &  &  &  & \\
12 & 13 &  &  &  &  & \\
10 & 13 &  &  &  &  & \\
13 & 14 &  &  &  &  & \\
\addlinespace
7 & 8 &  &  &  &  & \\
9 & 11 &  &  &  &  & \\
9 & 9 &  &  &  &  & \\
10 & 10 &  &  &  &  & \\
10 & 14 &  &  &  &  & \\
\bottomrule
\end{tabular}
\end{table}



\begin{enumerate}
\item Erg{\"a}nzen Sie die obige Tabelle mit den notwendigen Informationen, die
  Sie ben{\"o}tigen um einen Wilcoxon-Vorzeichen-Rang-Test zu rechnen!
  \textbf{(4 Punkte)}
\item Bestimmen Sie die Teststatistik $W$ mit $W = \min(T_{-}; T_{+})$ und
  berechnen Sie den erwarteten Wert $\mu_W = \cfrac{n_{!0} \cdot (n_{!0} + 1)}{4}$!
  \textbf{(2 Punkte)}
\item Berechnen Sie anschlie{\ss}end den $z$-Wert mit $z = \cfrac{W -
    \mu_W}{9.811}$! \textbf{(2 Punkte)}
\item Liegt mit einer Signifikanzschwelle von $z_{\alpha = 5\%} =
  1.96$ ein Unterschied zwischen den beiden Zeitpunkten vor? Begr{\"u}nden Sie
  Ihre Antwort! \textbf{(2 Punkte)} 
\item Berechnen Sie die Effektst{\"a}rke mit $r = |\frac{z}{\sqrt{n}}| $ und
  interpretieren Sie die Effektst{\"a}rke! \textbf{(2 Punkte)} 
\end{enumerate} 
\clearpage
% ----------------------------------------------------------------------- 

\section{Aufgabe \hfill (8 Punkte)}

\textit{Geben Sie grunds{\"a}tzlich Formeln und Rechenweg zur L{\"o}sung der
  Teilaufgaben mit an!} \\[1Ex]

%% --------------------------------------------------------------------
\hfill\href{https://youtu.be/5tiJFxuZcco}{\includegraphics[width =
  2cm]{img/youtube}} %%youtube
\hspace{2Ex}
%% --------------------------------------------------------------------




Nach einer Behandlung mit RootsGoneX wurde die mittelere Anzahl an Wurzeln
an der invasiven Lupine (\textit{Lupinus polyphyllus}) gez{\"a}hlt. Es ergab sich
folgender Datensatz an mittleren Wurzelanzahl.

\begin{knitrout}
\definecolor{shadecolor}{rgb}{0.969, 0.969, 0.969}\color{fgcolor}\begin{table}[!h]
\centering
\begin{tabular}{cc}
\toprule
Treatment & Count\\
\midrule
Kontrolle & 6.2\\
RootsGoneX & 13.3\\
RootsGoneX & 13.9\\
Kontrolle & 5.1\\
Kontrolle & 4.7\\
\addlinespace
RootsGoneX & 9.5\\
RootsGoneX & 10.8\\
Kontrolle & 8.7\\
Kontrolle & 4.5\\
RootsGoneX & 11.3\\
\addlinespace
Kontrolle & 10.1\\
\bottomrule
\end{tabular}
\end{table}

\end{knitrout}

Rechnen Sie einen Mann-Whitney-U-Test auf den obigen Daten.

\begin{enumerate}
\item Bestimmen Sie hierf{\"u}r $U_c$ mit $U_c = n_1n_2 +
  \cfrac{n_1(n_1+1)}{2}-R_1$! \textbf{(4 Punkte)} 
\item Geben Sie eine Aussage {\"u}ber die Signifikanz von $U_c$ durch
  $z = \cfrac{U_c -
    \cfrac{n_1n_2}{2}}{\sqrt{\cfrac{n_1n_2(n_1+n_2+1)}{12}}}$ und dem
  kritischen Wert von $z_{\alpha = 5\%} = 1.96$. Begr{\"u}nden Sie Ihre
  Antwort! \textbf{(2 Punkte)}
\item Berechnen Sie die Effektst{\"a}rke mit $r = |\frac{z}{\sqrt{n}}| $ und
  interpretieren Sie die Effektst{\"a}rke! \textbf{(2 Punkte)} 
\end{enumerate} 
\clearpage
% -----------------------------------------------------------------------  

\section{Aufgabe \hfill (10 Punkte)}

\textit{Geben Sie grunds{\"a}tzlich Formeln und Rechenweg zur L{\"o}sung der
  Teilaufgaben mit an!} \\[1Ex]

%% --------------------------------------------------------------------
\hfill\href{https://youtu.be/gC0SXiIG2wQ}{\includegraphics[width =
  2cm]{img/youtube}} %%youtube
\hspace{2Ex}
%% --------------------------------------------------------------------




Die Anzahl an Bl{\"u}ten der Vanilleplanze pro Box wurde nach der Gabe von
zus{\"a}tzlichen Phosporl{\"o}sung (Kontrolle, Dosis 20 und Dosis 40) bestimmt. Es
ergeben sich folgende nach der Anzahl der Bl{\"u}ten geordnete Daten.

\begin{knitrout}
\definecolor{shadecolor}{rgb}{0.969, 0.969, 0.969}\color{fgcolor}\begin{table}[!h]
\centering
\begin{tabular}{ccccc}
\toprule
Treatment & Count & Rang Kontrolle & Rang Dosis 20 & Rang Dosis 40\\
\midrule
Dosis 40 & 15.0 &  &  & \\
Kontrolle & 5.0 &  &  & \\
Dosis 40 & 12.1 &  &  & \\
Dosis 40 & 11.2 &  &  & \\
Dosis 20 & 10.6 &  &  & \\
\addlinespace
Kontrolle & 4.8 &  &  & \\
Dosis 20 & 15.0 &  &  & \\
Kontrolle & 6.8 &  &  & \\
Dosis 20 & 10.5 &  &  & \\
Dosis 20 & 13.3 &  &  & \\
\addlinespace
Dosis 20 & 16.7 &  &  & \\
Kontrolle & 5.0 &  &  & \\
Dosis 40 & 7.9 &  &  & \\
Dosis 20 & 10.5 &  &  & \\
Dosis 40 & 12.8 &  &  & \\
\addlinespace
Dosis 40 & 9.6 &  &  & \\
Dosis 40 & 10.5 &  &  & \\
Kontrolle & 3.5 &  &  & \\
\bottomrule
\end{tabular}
\end{table}

\end{knitrout}

Rechnen Sie einen Kruskal-Wallis-Test auf den obigen Daten.

\begin{enumerate}
\item Bestimmen Sie hierf{\"u}r $H_c$ mit $H_c =
  \cfrac{12}{n(n+1)}\left(\cfrac{R_1^2}{n_1}+\cfrac{R_2^2}{n_2}
    + \cfrac{R_3^2}{n_3}\right)
  - 3(n+1)$! \textbf{(6 Punkte)} 
\item Geben Sie eine Aussage {\"u}ber die Signifikanz von $H_c$ durch
  den kritischen Wert von $H_{\alpha = 5\%} = 5.99$! \textbf{(1 Punkt)}
\item Wie lautet die statistische Nullhypothese die Sie mit dem Kruskal-Wallis-Test
  {\"u}berpr{\"u}fen? \textbf{(1 Punkt)}
\item Was sagt ein signifikantes Ergebnis des Kruskal-Wallis-Test in Bezug
  auf die einzelnen Gruppenvergleiche aus? \textbf{(1 Punkt)}
\item Nennen Sie das statistische Verfahren, welches Sie als Posthoc Test
  nach einem signifikanten Kruskal-Wallis-Test durchf{\"u}hren w{\"u}rden! \textbf{(1 Punkt)}
\end{enumerate} 
\clearpage
% -----------------------------------------------------------------------
\begin{graybox}{Multiple Gruppenvergleiche}
Mehr Informationen zu den Aufgaben in den folgenden Kapiteln aus dem Skript Bio Data Science.
  \begin{itemize}
  \item \href{https://jkruppa.github.io/stat-tests-theorie.html#sec-statistisches-testen-alpha-adjust}{Kapitel 20.3 Adjustierung für multiple Vergleiche}
  \item \href{https://jkruppa.github.io/stat-tests-posthoc.html#sec-compact-letter}{Kapitel 31.7 Compact letter display}
  \end{itemize}
\end{graybox}
\clearpage
% ----------------------------------------------------------------------- 

\section{Aufgabe \hfill (8 Punkte)}


%% --------------------------------------------------------------------
 \hfill\href{https://youtu.be/hr_jPd1hpKY}{\includegraphics[width =
   2cm]{img/youtube}}\\[1Ex]
%% --------------------------------------------------------------------


In einem Experiment zur Dosiswirkung wurden verschiedene Dosisstufen mit
einer Kontrollgruppe vergleichen. Es wurden vier t-Test f{\"u}r den
Mittelwertsvergleich gerechnet und es ergab sich folgende Tabelle mit den
rohen p-Werten.



\begin{center}
  \Large
  \begin{tabular}{c|c|c|c}
    \textbf{Vergleich} & \textbf{Raw p-val} & \textbf{Adjusted p-val} &
                                                                        \textbf{Reject $\boldsymbol{H_0}$} \strut\\
    \hline
    dose 10 - ctrl  & 0.001 &  &\strut\\
    \hline
    dose 15 - ctrl  & 0.760 & &\strut\\
    \hline
    dose 20 - ctrl  & 0.020 & &\strut\\
    \hline
    dose 40 - ctrl  & 0.340 & &\strut\\
  \end{tabular}
\end{center}

\begin{enumerate}
\item F{\"u}llen Sie die Spalte "`adjustierte p-Werte"' mit den adjustierten
  p-Werten nach Bonferoni aus! \textbf{(4 Punkte)}
\item Entscheiden Sie, ob nach der Adjustierung die Nullhypothese weiter
  abglehnt werden kann. Tragen Sie Ihre Entscheidung in die obige Tabelle
  ein. Begr{\"u}nden Sie Ihre Antwort! \textbf{(2 Punkte)}
\item Erkl{\"a}ren Sie warum die p-Werte bei multiplen Vergleichen
  adjustiert werden m{\"u}ssen! \textbf{(2 Punkte)}
\end{enumerate}

\vspace{1Ex}

 
\clearpage
% ----------------------------------------------------------------------- 

\section{Aufgabe \hfill (9 Punkte)}

%% --------------------------------------------------------------------
 \hfill\href{https://youtu.be/RagTFFKFbFg}{\includegraphics[width =
   2cm]{img/youtube}}\\[1Ex]
%% --------------------------------------------------------------------


In einem Experiment mit vier Dosisstufen (ctrl, low, mid und high) erhalten Sie
folgende Matrix als \Rlogo Ausgabe mit den rohen, unadjustierten $p$-Werten. 



\begin{knitrout}
\definecolor{shadecolor}{rgb}{0.969, 0.969, 0.969}\color{fgcolor}\begin{kframe}
\begin{verbatim}
##           ctrl      high       low       mid
## ctrl 1.0000000 0.0019630 0.0152845 0.0207268
## high 0.0019630 1.0000000 0.0000013 0.3547722
## low  0.0152845 0.0000013 1.0000000 0.0000201
## mid  0.0207268 0.3547722 0.0000201 1.0000000
\end{verbatim}
\end{kframe}
\end{knitrout}

Im Weiteren erhalten Sie folgende Informationen {\"u}ber die Fallzahl $n$, den
Mittelwert $mean$ und die Standardabweichung $sd$ in den jeweiligen Dosisstufen.

\begin{knitrout}
\definecolor{shadecolor}{rgb}{0.969, 0.969, 0.969}\color{fgcolor}\begin{table}[!h]
\centering
\begin{tabular}{cccc}
\toprule
trt & n & mean & sd\\
\midrule
ctrl & 9 & 13.01 & 2.10\\
high & 9 & 9.46 & 1.68\\
low & 9 & 15.70 & 2.59\\
mid & 9 & 10.45 & 2.44\\
\bottomrule
\end{tabular}
\end{table}

\end{knitrout}


\begin{enumerate}
\item Zeichnen Sie in eine Abbildung, die sich ergebenden Barplots! \textbf{(2 Punkte)}
\item Adjustieren Sie die rohen $p$-Werte nach Bonferroni. Begr{\"u}nden Sie Ihre Antwort! \textbf{(3 Punkte)}
\item Erg{\"a}nzen Sie das \textit{Compact letter display (CLD)} zu der
  Abbildung! \textbf{(2 Punkte)}
\item Interpretieren Sie das \textit{Compact letter display (CLD)}! \textbf{(2 Punkte)} 
\end{enumerate}

 
\clearpage
% ----------------------------------------------------------------------- 

\section{Aufgabe \hfill (8 Punkte)}


%% --------------------------------------------------------------------
 \hfill\href{https://youtu.be/xq29O8qDrg0}{\includegraphics[width =
   2cm]{img/youtube}}\\[1Ex]
%% --------------------------------------------------------------------


In einem Experiment mit f{\"u}nf Dosisstufen (A, B, C, D und E) erhalten Sie
folgendes \textit{Compact letter display (CLD)} als \Rlogo Ausgabe aus den rohen, unadjustierten $p$-Werten. 



\begin{knitrout}
\definecolor{shadecolor}{rgb}{0.969, 0.969, 0.969}\color{fgcolor}\begin{kframe}
\begin{verbatim}
##    A    B    C    D    E 
##  "a"  "a"  "b" "ab"  "b"
\end{verbatim}
\end{kframe}
\end{knitrout}

\begin{enumerate}
\item Erstellen Sie eine Matrix mit den paarweisen $p$-Werten, die sich
  n{\"a}herungsweise aus dem \textit{Compact letter display (CLD)} ergeben w{\"u}rde! Begr{\"u}nden Sie Ihre Antwort! \textbf{(3 Punkte)}
\item Zeichnen Sie eine Abbildung, der sich ergebenden Barplots! \textbf{(2 Punkte)}
\item Erg{\"a}nzen Sie das \textit{Compact letter display (CLD)} zu der
  Abbildung! \textbf{(1 Punkt)}
\item Erkl{\"a}ren Sie \textit{einen} Vorteil und \textit{einen} Nachteil des \textit{Compact letter display (CLD)}! \textbf{(2 Punkte)}
\end{enumerate}

 
\clearpage
% -----------------------------------------------------------------------
\begin{graybox}{R Programmierung}
Mehr Informationen zu den Aufgaben in den folgenden Kapiteln aus dem Skript Bio Data Science.
  \begin{itemize}
  \item \href{https://jkruppa.github.io/programing-preface.html}{Kapitel 9ff. Programmieren in R}
  \end{itemize}
In der Klausur zu dem Modul \textbf{Mathematik \& Statistik} wird \textit{eine Aufgabe} aus den folgenden Aufgaben zur R Programmierung ausgewählt. \\

In der Klausur zu dem Modul \textbf{Statistik} wird \textit{eine Aufgabe} aus den folgenden Aufgaben zur R Programmierung ausgewählt.  \\

In der Klausur zu dem Modul \textbf{Angewandte Statistik für Bioverfahrenstechnik} wird \textit{eine Aufgabe} aus den folgenden Aufgaben zur R Programmierung ausgewählt.  \\

In der Klausur zu dem Modul \textbf{Angewandte Statistik und Versuchswesen} wird \textit{eine Aufgabe} aus den folgenden Aufgaben zur R Programmierung ausgewählt.  \\

In der Klausur zu dem Modul \textbf{Biostatistik} wird \textit{eine Aufgabe} aus den folgenden Aufgaben zur R Programmierung ausgewählt.  \\

\end{graybox}
\clearpage
% -----------------------------------------------------------------------  

\section{Aufgabe \hfill (8 Punkte)}

%% --------------------------------------------------------------------
\hfill\href{https://youtu.be/Bo0VOhBhJmA}{\includegraphics[width =
  2cm]{img/youtube}}\\[1Ex]
%% --------------------------------------------------------------------



Bearbeiten Sie folgenden Aufgaben mit Bezug zu \Rlogo! 

\begin{enumerate}
\item Erkl{\"a}ren Sie den Pipe-Operator am Beispiel der Berechnung des Mittelwertes
mit der Funktion \texttt{mean} und den Zahlen 11, 12, 9, 11 und 5!  \textbf{(2 Punkte)} 
\item Erkl{\"a}ren Sie den Unterschied zwischen einer Funktion und einem Objekt
  in R an einem Beispiel! \textbf{(2 Punkte)} 
\item Erkl{\"a}ren Sie den Vorteil der Verwendung der Funktion \texttt{p\_load()} an
einem Beispiel. Was ist das alternative Vorgehen zu der Verwendung der
Funktion? \textbf{(2 Punkte)} 
\item Erkl{\"a}ren Sie die Verwendung des Operators \texttt{::} am Beispiel der
Funktion \texttt{select()} und \texttt{p\_load()}! \textbf{(2 Punkte)} 
\end{enumerate}


 
\clearpage
% -----------------------------------------------------------------------

\section{Aufgabe \hfill (8 Punkte)}

%% --------------------------------------------------------------------
\hfill\href{https://youtu.be/xP9xjcLIbDE}{\includegraphics[width =
  2cm]{img/youtube}}\\[1Ex]
%% --------------------------------------------------------------------




Bearbeiten Sie folgenden Aufgaben mit Bezug zu \Rlogo! 

\begin{enumerate}
  \item Erkl{\"a}ren Sie den Pfeil-Operator am Beispiel eines Zahlenvektors mit der
Funktion \texttt{c()} und den Zahlen 5, 11, 9, 9 und 15! \textbf{(2 Punkte)}
\item Erkl{\"a}ren Sie den Nutzen des R Paketes \texttt{conflicted} am Beispiel der
  Funktion \texttt{select()}! \textbf{(2 Punkte)} 
\item Erkl{\"a}ren Sie den Unterschied zwischen einer \texttt{library} und
  einem \texttt{package} in R an einem Beispiel! \textbf{(2 Punkte)} 
\item Erkl{\"a}ren Sie den Unterschied zwischen \texttt{"mean"}, \texttt{mean}
  und \texttt{mean()}! \textbf{(2 Punkte)} 
\end{enumerate}


 
\clearpage
% -----------------------------------------------------------------------  

\section{Aufgabe \hfill (10 Punkte)}

%% --------------------------------------------------------------------
\hfill\href{https://youtu.be/WIgK_Oj_NW0}{\includegraphics[width =
  2cm]{img/youtube}}
\hspace{2Ex}
\href{https://youtu.be/JCdL7JrZo9o}{\includegraphics[width =
  2cm]{img/youtube_R}}\\[1Ex]
%% --------------------------------------------------------------------


Sie wollen eine explorative Datenananalyse auf dem folgenden, in \Rlogo schon geladenen, Datensatz \texttt{leaf\_tbl} durchf{\"u}hren.



\begin{knitrout}
\definecolor{shadecolor}{rgb}{0.969, 0.969, 0.969}\color{fgcolor}\begin{kframe}
\begin{alltt}
\hlstd{leaf_tbl}
\end{alltt}
\begin{verbatim}
## # A tibble: 10 x 3
##    treatment block  leaf
##        <dbl> <int> <dbl>
##  1         1     1    11
##  2         1     2     9
##  3         1     3     8
##  4         1     4    10
##  5         1     5    11
##  6         2     1    11
##  7         2     2     9
##  8         2     3    12
##  9         2     4    10
## 10         2     5    12
\end{verbatim}
\end{kframe}
\end{knitrout}

\begin{enumerate}
\item Welche \Rlogo Pakete ben{\"o}tigen Sie f{\"u}r die explorative Datenanalyse?
  \textbf{(2 Punkte)} 
\item Skizzieren Sie den \Rlogo Code f{\"u}r die Erstellung eines
  Barplots unter der Verwendung des Pipe-Operators! \textbf{(4 Punkte)}
\item Nehmen Sie an, dass Sie die Funktion \texttt{as\_factor()}
  verwenden. Wozu ben{\"o}tigen Sie die Funktion? Begr{\"u}nden Sie Ihre Antwort!
  \textbf{(2 Punkte)}
\item Erl{\"a}utern Sie Ihr weiteres Vorgehen nachdem Sie eine explorative
  Datenanalyse durchgef{\"u}hrt haben! \textbf{(2 Punkte)}
\end{enumerate}


 
\clearpage
% -----------------------------------------------------------------------  

\section{Aufgabe \hfill (8 Punkte)}

%% --------------------------------------------------------------------
\hfill\href{https://youtu.be/f5fHm_jCHe4}{\includegraphics[width =
  2cm]{img/youtube}}
\hspace{2Ex}
\href{https://youtu.be/_EGebjrOCUQ}{\includegraphics[width =
  2cm]{img/youtube_R}}\\[1Ex]
%% --------------------------------------------------------------------


Sie wollen einen multiplen, paarweisen Gruppenvergleich auf dem folgenden, in \Rlogo schon geladenen, Datensatz \texttt{leaf\_tbl} durchf{\"u}hren.



\begin{knitrout}
\definecolor{shadecolor}{rgb}{0.969, 0.969, 0.969}\color{fgcolor}\begin{kframe}
\begin{alltt}
\hlstd{leaf_tbl}
\end{alltt}
\begin{verbatim}
## # A tibble: 25 x 3
##    treatment block  leaf
##    <fct>     <fct> <dbl>
##  1 1         1        13
##  2 1         2        11
##  3 1         3        10
##  4 1         4        11
##  5 1         5        11
##  6 2         1         7
##  7 2         2         9
##  8 2         3        12
##  9 2         4        12
## 10 2         5         7
## # ... with 15 more rows
\end{verbatim}
\end{kframe}
\end{knitrout}

\begin{enumerate}
\item Welche \Rlogo Pakete ben{\"o}tigen Sie f{\"u}r den multipen Vergleich?
  \textbf{(2 Punkte)} 
\item Skizzieren Sie den \Rlogo Code f{\"u}r die Erstellung einer
  Berechnung eines multiplen Vergleiches unter der Verwendung des
  Pipe-Operators! Nutzen Sie hierf{\"u}r folgende Funktionen in der passenden
  Reihenfolge: \texttt{emmeans()},  \texttt{cld()},
  \texttt{lm()},  \texttt{anova()},  \texttt{ggplot()}!  \textbf{(4 Punkte)}
\item Erkl{\"a}ren Sie den Unterschied zwischen der Funktion
  \texttt{contrast()} und \texttt{cld()}!
  \textbf{(2 Punkte)}
\end{enumerate}


 
\clearpage
% -----------------------------------------------------------------------  

\section{Aufgabe \hfill (9 Punkte)}

%% --------------------------------------------------------------------
\hfill\href{https://youtu.be/Oxa97uqNyCQ}{\includegraphics[width =
  2cm]{img/youtube}}
\hspace{2Ex}
\href{https://youtu.be/ymFfBkWyb8s}{\includegraphics[width =
  2cm]{img/youtube_R}}\\[1Ex]
%% --------------------------------------------------------------------




Sie wollen einen Datensatz aus Excel in \Rlogo laden. In Ihrem Experiment haben Sie
die Behandlungen eins bis vier sowie die Bl{\"o}cke eins bis drei
vorliegen. Sie messen die Outcomes pigs, grade und freshmatter an vier verschiedenen Messterminen.

\begin{enumerate}
\item Welches \Rlogo Paket ben{\"o}tigen Sie f{\"u}r das Einlesen einer Excel Datei?
  \textbf{(1 Punkt)} 
\item Skizzieren Sie den sich ergebenden Datensatz als Datentabelle im
  Long-Format, so dass Sie die Daten erfolgreich \Rlogo laden k{\"o}nnen!
  \textbf{(3 Punkte)}
\item Skizzieren Sie den \Rlogo Code, den Sie ben{\"o}tigen um die Daten aus
  Excel in \Rlogo zu laden! Nutzen Sie hierf{\"u}r die Funktion
  \texttt{pivot\_longer()} aus dem \Rlogo Paket \texttt{tidyr}! \textbf{(3
    Punkte)}
\item Skizzieren Sie die Anwendung der Funktion \texttt{mutate()}!
  Begr{\"u}nden Sie die Anwendung! \textbf{(2 Punkte)}
\end{enumerate}


 
\clearpage
% ----------------------------------------------------------------------- 
\begin{graybox}{Mathematik}
Mehr Informationen zu den Aufgaben in den \href{https://jkruppa.github.io/math/}{Skript Mathematik} und den entsprechenden Kapiteln.\\

In der Klausur zu dem Modul \textbf{Mathematik \& Statistik} werden \textit{zwei bis drei Aufgaben} aus den folgenden Aufgaben zur Mathematik ausgewählt. \\
\end{graybox}
\clearpage
% -----------------------------------------------------------------------  

\section{Aufgabe \hfill (12 Punkte)}

\textit{Geben Sie grunds{\"a}tzlich Formeln und Rechenweg zur L{\"o}sung der
  Teilaufgaben mit an!} \\[1Ex]

%% --------------------------------------------------------------------
\hfill\href{https://youtu.be/Fu8kN0Uj13Y}{\includegraphics[width =
  2cm]{img/youtube}} %%youtube
\hspace{2Ex}
%% --------------------------------------------------------------------

\paragraph{Herodot – der Schimmel aus Ivenack}

W{\"a}hrend der Besetzung Mecklenburgs durch die Franzosen kamen Napoleon die
Geschichten des ber{\"u}hmten Apfelschimmels Herodot aus Ivenack zu
Geh{\"o}r. Herodot lief zwar niemals Rennen, war aber eines der ber{\"u}hmtesten
Pferde dieser Zeit. Napoleon selbst gab den Auftrag, diesen
Schimmel durch die Armee nach Frankreich zu bringen. Der Legende nach
sollen Arbeiter den Schimmel im hohlen Stamm einer 1000-j{\"a}hrigen Eiche aus Ivenack vor
den Franzosen versteckt haben. Doch Herodot verriet sein Versteck durch
lautes Wiehern, woraufhin die franz{\"o}sische Armee den Schimmel
beschlagnahmte und nach Frankreich f{\"u}hrte. \\



\textit{Forschungsfrage: "Konnten die Ivenacker den Apfelschimmel Herodot
  vor dem Zugriff von Napoleon in der 1000-j{\"a}hrigen Eiche verstecken?"} \\

Gehen Sie von einem radialen Wachstum der 1000-j{\"a}hrigen Eiche von
$0.9mm$ pro Jahr aus. Es ist bekannt, dass die Eiche im
Jahr 2022 einen Umfang von $12.5m$ in Brusth{\"o}he hatte.

\begin{enumerate}
\item Wie gro{\ss} war der Durchmesser der Eiche im Jahr $1805$ als
  Herodot in der Eiche versteckt werden sollte?
  \textbf{(3 Punkte)}
\item Skizzieren Sie in einer Abbildung einen linearen Zusammenhang und einen
exponentiellen Zusammenhang f{\"u}r das Wachstum der 1000-j{\"a}hrigen Eiche. Erkl{\"a}ren Sie die
Auswirkungen der Entscheidung f{\"u}r linear oder exponentiell auf Ihre
Berechnungen! \textbf{(2 Punkte)}
\end{enumerate}
 
Herodot hatte eine Schulterh{\"o}he von $180$cm, eine Breite von
$95$cm sowie eine L{\"a}nge von  $250$cm.

\begin{enumerate}
  \setcounter{enumi}{2}
\item Berechnen Sie das effektive Volumen von Herodot in $m^3$, welches
  Herodot in der 1000-j{\"a}hrigen Eiche einnehmen w{\"u}rde! \textbf{(2 Punkte)}
\end{enumerate}

Es wurde berichtet, dass sich Herodot in der 1000-j{\"a}hrigen Eiche
$bequem$ um die eigene Achse drehen konnte.

\begin{enumerate}
  \setcounter{enumi}{3}
\item Berechnen Sie die Dicke der Eichenwand in cm! Verdeutlichen Sie Ihre
  Berechnungen an einer aussagekr{\"a}ftigen Skizze f{\"u}r Pferd und Eiche! \textbf{(3 Punkte)} 
\item Unter einer Dicke der Eichenwand von $10cm$ bricht
  die Eiche zusammen. Beantworten Sie die Forschungsfrage! Begr{\"u}nden Sie
  Ihre Antwort! \textbf{(2 Punkte)} 
\end{enumerate}
 
\clearpage
% ----------------------------------------------------------------------- 

\section{Aufgabe \hfill (10 Punkte)}

\textit{Geben Sie grunds{\"a}tzlich Formeln und Rechenweg zur L{\"o}sung der
  Teilaufgaben mit an!} \\[1Ex]

%% --------------------------------------------------------------------
\hfill\href{https://youtu.be/57B-yYoFSk0}{\includegraphics[width =
  2cm]{img/youtube}} %%youtube
\hspace{2Ex}
%% --------------------------------------------------------------------

\paragraph{Von T{\"o}pfen auf Tischen}



In einem Experiment wollen Sie die Wuchsh{\"o}he von 60
Sonnenblumen bestimmen. Bevor Sie {\"u}berhaupt mit dem Experiment beginnen
k{\"o}nnen, gibt es aber ein paar Absch{\"a}tzungen {\"u}ber die Kosten und den Aufwand
zu treffen. Zum einen m{\"u}ssen Sie die Sonnenblumen einpflanzen und m{\"u}ssen
daf{\"u}r Substrat bestellen. Zum anderen m{\"u}ssen Sie die Sonnenblumen auch
bewegen und in ein Gew{\"a}chshaus platzieren. Die T{\"o}pfe f{\"u}r die Keimung haben
einen Radius von 4cm und eine H{\"o}he von 9cm. Der
Kubikmeterpreis f{\"u}r Torf liegt bei 290 EUR.

\begin{enumerate}
\item Skizzieren Sie den Versuchsplan auf \textit{einem} Tisch im
  Gew{\"a}chshaus! \textbf{(2 Punkte)}
\item Berechnen Sie die ben{\"o}tigten T{\"o}pfe, wenn Sie Randpflanzen mit
  ber{\"u}cksichtigen wollen! \textbf{(1 Punkt)}
\item Welche Fl{\"a}che in $m^2$ gegeben der Anzahl an T{\"o}pfen ben{\"o}tigen Sie im
  Gew{\"a}chshaus am Anfang der Keimungsphase? \textbf{(3 Punkte)}
\item Berechnen Sie die ben{\"o}tigte Menge an Torf und die Kosten f{\"u}r Ihre
  Pflanzung! Gehen Sie von einem Zylinder aus! \textbf{(2 Punkte)}
\item Nach dem Bef{\"u}llen der T{\"o}pfe haben Sie noch Torf {\"u}brig. Erkl{\"a}ren Sie
  den Sachstand! Wie gehen Sie zuk{\"u}nftig vor? \textbf{(2 Punkte)}
\end{enumerate}


 
\clearpage
% ----------------------------------------------------------------------- 

\section{Aufgabe \hfill (10 Punkte)}

\textit{Geben Sie grunds{\"a}tzlich Formeln und Rechenweg zur L{\"o}sung der
  Teilaufgaben mit an!} \\[1Ex]

%% --------------------------------------------------------------------
\hfill\href{https://youtu.be/1B53cVFIU7Q}{\includegraphics[width =
  2cm]{img/youtube}} %%youtube
\hspace{2Ex}
%% --------------------------------------------------------------------

\paragraph{Nelken von den Molukken}



In der Ausstellung "`Europa und das Meer"' im Deutschen Historischen Museum in
Berlin gab es folgendes Zitat {\"u}ber die Probleme der fr{\"u}hen Hochseeschifffahrt.

\begin{quote}
  >>Ohne ausreichende Zufuhr von Vitamin C stellen sich nach 40 Tagen die
  ersten Symptome ein; die ersten Toten sind nach 60 Tagen zu beklagen;
  nach 100 Tagen rafft die Skorbut eine ganze Schiffsbesatzung dahin<<
\end{quote}

Ferdinand Magellan stach im Jahre 1519 in See um eine Passage durch den
s{\"u}damerikanischen Kontinent zu finden. Zu seiner Flotte geh{\"o}rten
f{\"u}nf Schiffe - das Flaggschiff Trinidad, die San Antonio, die Victoria, die
Concepci{\'o}n und die Santiago - mit einer Besatzung von insgesamt
222 Mann. 

\begin{enumerate}
\item Stellen Sie den Verlauf der Skorbuterkrankung auf einem Schiff der
  Flotte dar! Beschriften Sie die Achsen! \textbf{(4 Punkte)} 
\item Sch{\"a}tzen Sie die {\"U}berlebenswahrscheinlichkeit nach 95 Tagen
  aus Ihrer Abbildung ab! \textbf{(1 Punkt)} 
\end{enumerate}


Der Chronist an Bord der Trinidad, Antonio Pigafetta, schrieb in seinem
Bericht "`[...] Um nicht Hungers zu sterben, a{\ss}en wir das Leder, mit dem
die gro{\ss}e Rahe zum Schutz der Taue umwunden war."' Insbesondere die
Mannschaft der Concepci{\'o}n erlitt gro{\ss}e Verluste durch die Skrobut bei der
{\"U}berquerung des Pazifiks, da durch Erkundungsfahrten weniger Zeit blieb, um
wilden Sellerie aufzunehmen. Wilder Sellerie enth{\"a}lt
$6000\mu g/100g$ Vitamin C. Der Bedarf liegt bei
$110mg$ pro Tag f{\"u}r M{\"a}nner.

\begin{enumerate}
  \setcounter{enumi}{2}
\item Berechnen Sie die notwendige Menge an aufzunehmenden wilden Sellerie
  auf die Concepci{\'o}n f{\"u}r die ununterbrochene Fahrt von drei Monate und 22
  Tage {\"u}ber den Pazifik! \textbf{(3 Punkte)}
\item Skizzieren Sie die {\"U}berlebenszeitkurve f{\"u}r die Concepci{\'o}n im
  Vergleich zu der {\"U}berlebenszeitkurve der Trinidad! Beschriften Sie die
  Achsen! \textbf{(2 Punkte)}
\end{enumerate}


 
\clearpage
% -----------------------------------------------------------------------

\section{Aufgabe \hfill (12 Punkte)}

\textit{Geben Sie grunds{\"a}tzlich Formeln und Rechenweg zur L{\"o}sung der
  Teilaufgaben mit an!} \\[1Ex]

%% --------------------------------------------------------------------
\hfill\href{https://youtu.be/q-qYK4Chslg}{\includegraphics[width =
  2cm]{img/youtube}} %%youtube
\hspace{2Ex}
%% --------------------------------------------------------------------

\paragraph{Event Horizon -- Am Rande des Universums}



Die Sonne hat eine aktuelle, angenommene Masse von
$\ensuremath{2\times 10^{31}}$kg. Wenn die Sonne nun am Ende ihrer Lebenszeit zu einem schwarzen Loch mit dem Radius
von $3000$m kollabiert, wird
die Sonne $45$\% der aktuellen Masse verloren haben. Ein
Lichtteilchen mit der Masse $m_f$ und der Fluchtgeschwindigkeit $v_f$ will
dem schwarzen Loch entkommen.  Sie haben folgende Formeln f{\"u}r die
kinetische Energie des Lichtteilchens $E_{kin}$ und der Graviationsenergie des
schwarzen Lochs $E_{grav}$ gegeben.

\begin{center}
  \begin{tabular}{cc}
    $E_{kin} = \cfrac{1}{2}m_fv_f^2$ & $E_{grav} = \cfrac{Gm_sm_f}{r_s}$\\
  \end{tabular}
\end{center}

mit

\begin{itemize}
\item $m_f$, gleich der Masse [kg] des fliehenden Objektes
\item $m_s$, gleich der Masse [kg] des station{\"a}ren Objekts
\item $r_s$, gleich dem Radius [m] des station{\"a}ren Objekts  
\item $G$, gleich der Gravitationskonstante mit $6.674 \cdot 10^{-11}
  \tfrac{m^3}{kg \cdot s^2}$ 
\end{itemize}

Im Folgenden wollen wir uns mit der Frage besch{\"a}ftigen, ob das
Lichtteilchen der Gravitation des schwarzen Lochs entkommen kann.

\begin{enumerate}
\item Geben Sie die Formel f{\"u}r die Fluchtgeschwindigkeit $v_f$ an! 
  \textbf{(2 Punkte)}
\item {\"U}berpr{\"u}fen Sie Ihre umgestellte Formel nach $v_f$ anhand der Einheiten!
  \textbf{(2 Punkte)} 
\item Berechnen Sie die notwendige Fluchtgeschwindigkeit $v_f$ des
  Lichtteilchens mit den angegebenen Informationen! \textbf{(2 Punkte)}
\item Die Lichtgeschwindigkeit ist mit $2.9e+08m/s$ angegeben. Kann das
  Lichtteilchen der Gravitation des schwarzen Lochs entkommen? Begr{\"u}nden
  Sie Ihre Antwort! \textbf{(2 Punkte)}
\item Stellen Sie den Zusammenhang zwischen dem sich verringernden Radius
  $r$ des schwarzen Lochs bei gleichbleibender Masse $m_s$
  und der notwendigen Fluchtgeschwindigkeit $v_f$ in einer Abbildung dar!
  \textbf{(2 Punkte)}
\item Erkl{\"a}ren Sie in diesem Zusammenhang den Begriff
  \textit{Ereignishorizont}! \textbf{(2 Punkte)} 
\end{enumerate}


 
\clearpage
% ----------------------------------------------------------------------- 

\section{Aufgabe \hfill (10 Punkte)}

\textit{Geben Sie grunds{\"a}tzlich Formeln und Rechenweg zur L{\"o}sung der
  Teilaufgaben mit an!} \\[1Ex]

%% --------------------------------------------------------------------
\hfill\href{https://youtu.be/-b4IRu2-EJo}{\includegraphics[width =
  2cm]{img/youtube}} %%youtube
\hspace{2Ex}
%% --------------------------------------------------------------------

\paragraph{Das Fermi Paradoxon}



Der Kernphysiker Enrico Fermi diskutierte 1950 auf dem Weg zum Mittagessen
im Los Alamos National Laboratory mit seinen Kollegen angebliche
UFO-Sichtungen und fragte schlie{\ss}lich: "`Where is everybody?"'. Warum seien
weder Raumschiffe anderer Weltraumbewohner noch andere Spuren
extraterrestrischer Technik zu beobachten? Wie lange w{\"u}rde eine au{\ss}erirdische
Zivilisation ben{\"o}tigen um die gesamte Milchstra{\ss}e zu
besuchen, wenn das maximale Reisetempo die Geschwindigkeit der Voyager 1 Sonde w{\"a}re?\\[-1ex]

Wir treffen folgende Annahmen. Eine au{\ss}erirdische Zivilisation schickt $vier$
Voyager 1 {\"a}hnliche Sonden mit der Geschwindigkeit von $\ensuremath{6.2587\times 10^{4}}km/h$
los um sich auf den erreichten Planeten selbst zu replizieren. Nach
$1000$ Jahren ist die Replikation abgeschlossen und wiederum
$vier$ Sonden werden ausgesendet. Gehen Sie von
$6.23$ Lichtjahren als mittlerer Abstand der Sterne in der
Milchstra{\ss}e aus. Es gibt $\ensuremath{10^{11}}$ Sterne in der Milchstra{\ss}e. Die
Lichtgeschwindigkeit betr{\"a}gt $2.9e+08m/s$.

\begin{enumerate}
\item Skizzieren Sie in einer Abbildung die ersten vier Schritte der
  Vervielf{\"a}ltigung der Sonden in der Galaxie! Beschriften Sie die Abbildung
  mit der Dauer jedes Schrittes! \textbf{(2 Punkte)}
\item Berechnen Sie die Dauer, die eine au{\ss}erirdische Zivilisation
  ben{\"o}tigt, um die ganze Milchstra{\ss}e zu besuchen! \textbf{(3 Punkte)}
\item Bei einem vermutetet Alter der Erde von $\ensuremath{4.3\times 10^{9}}$ Jahren,
  wie oft war dann eine Sonde einer au{\ss}erirdischen Zivilisation schon zu
  Besuch? Korrigieren Sie Ihre Antwort mit dem Wissen, dass sich die
  Kontinentalplatten einmal alle $\ensuremath{9\times 10^{7}}$ Jahre vollst{\"a}ndig im
  Erdinneren umgewandelt haben! \textbf{(2 Punkte)}
\item Mit welcher Wahrscheinlichkeit wurde die Menschheit schon von der
  Au{\ss}erirdischen besucht, wenn wir von einem Zivilisationsbeginn der
  Menschheit von vor $\ensuremath{1.2\times 10^{4}}$ Jahren ausgehen? \textbf{(1 Punkt)}
\item Skizzieren Sie in einer Abbildung den Zusammenhang zwischen Zeit $t$
  und Raum $r$. Erg{\"a}nzen Sie den Geschwindigkeitsvektor $\vec{v_t}$ und
  $\vec{v_r}$ einer ruhenden Sonde, einer mit 50\% Lichtgeschwindigkeit und
  mit 99\% Lichtgeschwindigkeit fliegender Sonde! \textbf{(1 Punkt)}
\item Warum ist die Lichtgeschwindigkeit die maximale m{\"o}gliche Geschwindigkeit?
Begr{\"u}nden Sie Ihre Antwort anhand der Abbildung!  \textbf{(1 Punkt)}
\end{enumerate}



 
\clearpage
% ----------------------------------------------------------------------- 

\section{Aufgabe \hfill (10 Punkte)}

\textit{Geben Sie grunds{\"a}tzlich Formeln und Rechenweg zur L{\"o}sung der
  Teilaufgaben mit an!} \\[1Ex]

%% --------------------------------------------------------------------
\hfill\href{https://youtu.be/8Pb2sKUIMyk}{\includegraphics[width =
  2cm]{img/youtube}} %%youtube
\hspace{2Ex}
%% --------------------------------------------------------------------

\paragraph{H{\"o}hlen \& Drachen}



Nachdem Sie sich begeistert in der Serie \textit{Stranger Thinks} verloren
haben, wollen Sie bei einer Ihrer Freundinnen einmal \textit{H{\"o}hlen \& Drachen}
ausprobieren. Um Geld zu sparen, das Zeug kostet echt, wurde etwas an den
Regeln gebastelt. Schnell stellen Sie fest, dass hier ganz sch{\"o}n viele
unterschiedliche W{\"u}rfel durch die Gegend fliegen. Daher m{\"u}ssen Sie sich
jetzt einiges an Fragen stellen. \\%[-1ex]

In dem Spiel haben Sie nun auf einmal 5 sechseitige W{"u}rfel (5d6) zum w{\"u}rfeln in der Hand. Wenn Sie eine 6 w{\"u}rfeln,
haben Sie einen Erfolg.

\begin{enumerate}
\item Berechnen Sie die Wahrscheinlichkeit \textit{genau}
  4 Erfolge zu erzielen!  \textbf{(2 Punkte)}
\item Berechnen Sie die Wahrscheinlichkeit keinen Erfolg zu erzielen!
  \textbf{(1 Punkt)}
\end{enumerate}

Sie betrachten nun aufmerksam die ausufernden Ausr{\"u}stungstabellen. Ihnen
wird aber geholfen und Sie m{\"u}ssen sich jetzt nur zwischen der Axt oder dem
Schwert entscheiden.

\begin{enumerate}
  \setcounter{enumi}{2}
\item W{\"u}rden Sie die Axt mit zwei sechseitigen W{"u}rfeln (2d6) als Schaden oder
  das Schwert mit einem vierseitigen W{"u}rfel plus 7 (1d4+7) als Schaden bevorzugen?
  Begr{\"u}nden Sie Ihre Antwort mathematisch! \textbf{(1 Punkt)}
\end{enumerate}

Jetzt wird es immer wilder, da Sie sich jetzt {\"u}berlegen m{\"u}ssen, wie
wahrscheinlich es ist, dass Ihr Rettungswurf gegen den zaubernden Hexer
funktioniert. Sie haben folgende Wahrscheinlichkeiten gegeben. Die
Wahrscheinlichkeit f{\"u}r das Ereignis $A$, der Rettungswurf ist erfolgreich,
ist $Pr(A) = 0.6$, die Wahrscheinlichkeit f{\"u}r das Ereignis $B$,
der Zauberwurf des Hexers ist erfolgreich, ist $Pr(B) = 0.9$. Sie
haben mitgez{\"a}hlt und festgestellt, dass in $45$ von 100 F{\"a}llen
Ihr Rettungswurf bei einem erfolgeichen Zauber funktioniert hat.  

\begin{enumerate}
  \setcounter{enumi}{3}
\item Erstellen Sie eine 2x2 Kreuztabelle mit den Ereignissen $A$ und $B$
  sowie den Gegenereignissen $\bar{A}$ und $\bar{B}$ mit einen
  $\Omega = 100$. Beachten Sie hierbei die entsprechenden
  Wahrscheinlichkeiten f{\"u}r die Ereignisse $A$ und $B$! \textbf{(2 Punkte)}
\item Bestimmen Sie $Pr(A \cap B)$! \textbf{(1 Punkt)}
\item Erstellen Sie ein Baumdiagramm mit den passenden Informationen aus der 2x2
  Kreuztabelle! \textbf{(2 Punkte)}
\item Bestimmen Sie Wahrscheinlichkeit $Pr(A|B)$, dass Ihr Rettungswurf gelingt, wenn
  der Hexer erfolgreich gezaubert hat! \textbf{(1 Punkt)}
\end{enumerate}

  
\clearpage
% ----------------------------------------------------------------------- 

\section{Aufgabe \hfill (10 Punkte)}

\textit{Geben Sie grunds{\"a}tzlich Formeln und Rechenweg zur L{\"o}sung der
  Teilaufgaben mit an!} \\[1Ex]

%% --------------------------------------------------------------------
\hfill\href{https://youtu.be/aBxLkdF-c4M}{\includegraphics[width =
  2cm]{img/youtube}} %%youtube
\hspace{2Ex}
%% --------------------------------------------------------------------

\paragraph{Solar- \& Biogasanlagen}



Um die Energiekosten Ihres Betriebes zu senken, wollen Sie eine Solaranlage
auf den Schweinestall montieren lassen. Sie messen Ihren Stall und finden
folgende Ma{\ss}e wieder. Die vordere Seite des Schweinestall hat eine H{\"o}he
$h_v$ von $6m$. Die hintere Seite des Schweinestall hat eine
H{\"o}he $h_b$ von $9m$. Der Schweinestall hat eine Tiefe $t$ von
$13m$ und eine Breite $b$ von $50m$.

\begin{enumerate}
\item Skizzieren Sie den Schweinestall auf dem die Solaranlage montiert
  werden soll! Erg{\"a}nzen Sie die Angaben f{\"u}r die H{\"o}hen $h_v$, $h_b$, die
  Tiefe $t$ und die Breite $b$ des Stalls!  \textbf{(2 Punkte)}
\item Berechnen Sie die Fl{\"a}che der schr{\"a}gen, neuen Solaranlage auf dem
  Schweinestall! \textbf{(3 Punkte)}
\end{enumerate}

Ebenfalls planen Sie eine neue Biogasanlage f{\"u}r Ihren Betrieb. Der neue
Methantank hat einen Radius $r$ von $1m$. Leider gibt es ein
paar bauliche Beschr{\"a}nkungen auf dem Grundst{\"u}ck. Ihr Fundament des
zylindrischen Methantanks kann nur ein Gewicht von maximal
$15t$ aushalten bevor der Tank wegbricht. Sie rechnen eine
Sicherheitstoleranz von $15\%$ ein. In fl{\"u}ssiger Form hat
Methan bei $-80^\circ\text{C}$ eine Dichte von
$240kg/m^3$. Bei $-100^\circ\text{C}$ hat Methan eine Dichte
von $290kg/m^3$. Sie betrieben Ihre Anlage bei
$-95^\circ\text{C}$.

\begin{enumerate}
  \setcounter{enumi}{2}
\item Extrapolieren Sie die effektive Dichte des Methans in Ihrem
  Methantank! Welche Annahme haben Sie getroffen? \textbf{(1 Punkt)}
\item Berechnen Sie wie viel Kubikmeter $m^3$ Sie in den Methantank f{\"u}llen
  k{\"o}nnen, bevor das Fundament nachgibt! \textbf{(2 Punkte)}
\item Berechnen Sie die maximale H{\"o}he $h_{max}$ f{\"u}r den gef{\"u}llten
  Methantank mit dem Radius $r$, bevor das Fundament wegbricht! \textbf{(2
    Punkte)}
\end{enumerate}



 
\clearpage
% ----------------------------------------------------------------------- 

\section{Aufgabe \hfill (10 Punkte)}

\textit{Geben Sie grunds{\"a}tzlich Formeln und Rechenweg zur L{\"o}sung der
  Teilaufgaben mit an!} \\[1Ex]

%% --------------------------------------------------------------------
\hfill\href{https://youtu.be/x_MwWJoEsME}{\includegraphics[width =
  2cm]{img/youtube}} %%youtube
\hspace{2Ex}
%% --------------------------------------------------------------------



Sie haben den Vektor $A = \{12, 14, 3\}$ und den Vektor $B = \{4, 0, 2\}$ gegeben.

\begin{enumerate}
\item Addieren Sie die Vektoren $A$ und $B$! \textbf{(1 Punkt)}
\item Transponieren Sie den Vektor $A$!
  \textbf{(1 Punkt)} 
\item Multiplizieren Sie den Vektor $A$ mit
  $10$! \textbf{(1 Punkt)}
\item Erstellen Sie eine Diagonalmatrix mit $m \times n$ Dimensionen!
  Legen Sie $m$ und $n$ vorher sinnvoll fest! \textbf{(1 Punkt)}
\item {\"U}bersetzen Sie folgendes linearen Gleichungssystem bestehend aus drei
  Gleichungen in die Matrixschreibweise! \textbf{(2 Punkte)}
\begin{knitrout}
\definecolor{shadecolor}{rgb}{0.969, 0.969, 0.969}\color{fgcolor}\begin{kframe}
\begin{verbatim}
## [1] "2x1 + 7x2 + 2x3 = 2"
## [1] "5x1 + 8x2 + 9x3 = 2"
## [1] "2x1 + 8x2 + 2x3 = 2"
\end{verbatim}
\end{kframe}
\end{knitrout}
\item Multiplizieren Sie folgende gegebene Matrix mit dem Vektor $A$!
  \textbf{(2 Punkte)}
\begin{knitrout}
\definecolor{shadecolor}{rgb}{0.969, 0.969, 0.969}\color{fgcolor}\begin{kframe}
\begin{verbatim}
##      [,1] [,2] [,3]
## [1,]    6    8    5
## [2,]    7    6    4
## [3,]    5    7    5
## [4,]    6    6    4
\end{verbatim}
\end{kframe}
\end{knitrout}
\item Skizzieren Sie den entsprechenden \Rlogo Code f{\"u}r die
  Matrixmultiplikation! \textbf{(2 Punkte)}
\end{enumerate}



 
\clearpage
% -----------------------------------------------------------------------

\section{Aufgabe \hfill (10 Punkte)}

\textit{Geben Sie grunds{\"a}tzlich Formeln und Rechenweg zur L{\"o}sung der
  Teilaufgaben mit an!} \\[1Ex]

%% --------------------------------------------------------------------
\hfill\href{https://youtu.be/paXxVmyfDPs}{\includegraphics[width =
  2cm]{img/youtube}} %%youtube
\hspace{2Ex}
%% --------------------------------------------------------------------

\vspace{1Ex}



\begin{enumerate}
\item Erg{\"a}nzen Sie die logischen Operatoren in \Rlogo in die untenstehende Tabelle!
  \textbf{(2 Punkte)}
  \begin{center}
    \begin{tabular}{c|c}
      \textbf{Ausdruck} & \textbf{Operator} \strut\\ \hline
      GROESSER  & \phantom{GROESSER}\strut\\ \hline
      NICHT  & \phantom{NICHT}\strut\\ \hline
      GROESSER GLEICH  & \phantom{GROESSER GLEICH}\strut\\ \hline
      GLEICH  & \phantom{GLEICH}\strut\\ 
    \end{tabular}
  \end{center}  
\item Visualisieren Sie folgende logische Aussagen zu der Menge $A$ und der Menge
  $B$ als Mengendiagramme bzw. Venndiagramme!
  \textbf{(2 Punkte)}
  \begin{itemize}
  \item B ist Teilmenge von A
  \item A tritt nicht ein
  \item Entweder A tritt ein oder B tritt ein oder keins von beiden
  \item A tritt ein, aber B tritt nicht ein
  \end{itemize}
\item Erg{\"a}nzen Sie zu den Venndiagrammen die mathematische Notation!
  \textbf{(2 Punkte)}
\item Erstellen Sie die \texttt{logical} \Rlogo Ausgabe f{\"u}r die
  Suche nach der Zeichenfolge \texttt{AT} in folgenden DNA
  Sequenzen! \textbf{(2 Punkte)}
  \begin{itemize}
  \item TCTCTCTC
  \item ACTGGTAA
  \item ATGTGTAT
  \item AACCAACC
  \end{itemize}
\item Geben ist das \Rlogo Objekt A beinhaltend die Zahl
  8. Erkl{\"a}ren Sie den Unterschied zwischen dem Ausdruck \texttt{A == 8}
  und dem Ausdruck \texttt{A != 8} in \Rlogo! Wie lautet die Ausgabe von \Rlogo
  in beiden F{\"a}llen? \textbf{(2 Punkte)}
\end{enumerate}




  
\clearpage
% -----------------------------------------------------------------------
\end{document}
% -----------------------------------------------------------------------


  

  
