\section{Aufgabe \hfill (12 Punkte)}

\textit{Geben Sie grunds{\"a}tzlich Formeln und Rechenweg zur L{\"o}sung der
  Teilaufgaben mit an!} \\[1Ex]

%% --------------------------------------------------------------------
\hfill\href{https://youtu.be/SZqp_vy3rgI}{\includegraphics[width = 2cm]{img/youtube}} %%youtube
\hspace{2Ex}
%% --------------------------------------------------------------------

\paragraph{Aligatorenbirnen und Blaubeeren}


\begin{knitrout}
\definecolor{shadecolor}{rgb}{0.969, 0.969, 0.969}\color{fgcolor}\begin{kframe}
\begin{alltt}
\hlstd{person} \hlkwb{<-} \hlkwd{c}\hlstd{(}\hlstr{"einer Rentnerin mit Monokel"}\hlstd{,} \hlstr{"einem Studienrat mit Stock"}\hlstd{,}
            \hlstr{"einem Mädchen mit Zöpfen und Zahnspange"}\hlstd{,} \hlstr{"einem Studenten im Karohemd"}\hlstd{) |>}
  \hlkwd{sample}\hlstd{(}\hlnum{1}\hlstd{)}
\hlstd{markt} \hlkwb{<-} \hlkwd{c}\hlstd{(}\hlstr{"Lidl"}\hlstd{,} \hlstr{"Aldi"}\hlstd{,} \hlstr{"Edeka"}\hlstd{,} \hlstr{"Netto"}\hlstd{,} \hlstr{"Rewe"}\hlstd{) |>} \hlkwd{sample}\hlstd{(}\hlnum{1}\hlstd{)}
\end{alltt}
\end{kframe}
\end{knitrout}

"'Sind Sie ein Riesenfautier oder warum kaufen Sie so viele Aligatorenbirnen?"' Irritiert drehen Sie sich um und blicken in das hochrote Gesicht einem Mädchen mit Zöpfen und Zahnspange. "'Wieso?"', entfährt es Ihnen und Sie bereuen sogleich die Frage. Schnell werden Sie zu einem Whiteboard geschliffen und müssen folgenden mathematischen untermauerten Argumenten im Lidl über sich ergehen lassen. Da kommen Sie nicht mehr raus, also können Sie auch gleich mitmachen.


\begin{enumerate}
  \setcounter{enumi}{5}
\item Be
  Provision abzubezahlen? \textbf{(1 Punkt)}
\end{enumerate}
