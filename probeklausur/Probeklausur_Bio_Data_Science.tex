\documentclass[a4paper, 9pt]{scrartcl}\usepackage[]{graphicx}\usepackage[]{xcolor}
% maxwidth is the original width if it is less than linewidth
% otherwise use linewidth (to make sure the graphics do not exceed the margin)
\makeatletter
\def\maxwidth{ %
  \ifdim\Gin@nat@width>\linewidth
    \linewidth
  \else
    \Gin@nat@width
  \fi
}
\makeatother

\definecolor{fgcolor}{rgb}{0.345, 0.345, 0.345}
\newcommand{\hlnum}[1]{\textcolor[rgb]{0.686,0.059,0.569}{#1}}%
\newcommand{\hlsng}[1]{\textcolor[rgb]{0.192,0.494,0.8}{#1}}%
\newcommand{\hlcom}[1]{\textcolor[rgb]{0.678,0.584,0.686}{\textit{#1}}}%
\newcommand{\hlopt}[1]{\textcolor[rgb]{0,0,0}{#1}}%
\newcommand{\hldef}[1]{\textcolor[rgb]{0.345,0.345,0.345}{#1}}%
\newcommand{\hlkwa}[1]{\textcolor[rgb]{0.161,0.373,0.58}{\textbf{#1}}}%
\newcommand{\hlkwb}[1]{\textcolor[rgb]{0.69,0.353,0.396}{#1}}%
\newcommand{\hlkwc}[1]{\textcolor[rgb]{0.333,0.667,0.333}{#1}}%
\newcommand{\hlkwd}[1]{\textcolor[rgb]{0.737,0.353,0.396}{\textbf{#1}}}%
\let\hlipl\hlkwb

\usepackage{framed}
\makeatletter
\newenvironment{kframe}{%
 \def\at@end@of@kframe{}%
 \ifinner\ifhmode%
  \def\at@end@of@kframe{\end{minipage}}%
  \begin{minipage}{\columnwidth}%
 \fi\fi%
 \def\FrameCommand##1{\hskip\@totalleftmargin \hskip-\fboxsep
 \colorbox{shadecolor}{##1}\hskip-\fboxsep
     % There is no \\@totalrightmargin, so:
     \hskip-\linewidth \hskip-\@totalleftmargin \hskip\columnwidth}%
 \MakeFramed {\advance\hsize-\width
   \@totalleftmargin\z@ \linewidth\hsize
   \@setminipage}}%
 {\par\unskip\endMakeFramed%
 \at@end@of@kframe}
\makeatother

\definecolor{shadecolor}{rgb}{.97, .97, .97}
\definecolor{messagecolor}{rgb}{0, 0, 0}
\definecolor{warningcolor}{rgb}{1, 0, 1}
\definecolor{errorcolor}{rgb}{1, 0, 0}
\newenvironment{knitrout}{}{} % an empty environment to be redefined in TeX

\usepackage{alltt}
\usepackage[ngerman]{babel}
% -----------------------------------------------------------------------

% -----------------------------------------------------------------------
%% ------------------------------------------------------------
%% by J.Kruppa on Friday, February 11, 2022 (11:31)
%% \def\mainDir{\Sexpr{exam_path}}
\def\source{/Users/jokruppa/source/tex}
\usepackage[margin=2cm, includefoot]{geometry}
\setlength{\parindent}{0cm}
\usepackage{booktabs}
\usepackage{amsmath}
\usepackage{scalerel,amssymb}
\usepackage{setspace}
\def\csquare{{\Large $\boxtimes$}}
\def\msquare{{\Large $\square$}}
\usepackage[normalem]{ulem}
\usepackage{array}
\usepackage{xcolor}
\usepackage{float}
\usepackage{currfile}
\usepackage{tikz}
\usepackage[nomessages]{fp}

%% beamer defs
\def\lecture{Klausurfragen der Bio Data Science}

%% exam defs
\def\examtitle{\lecture}
\def\exammodule{
\vspace{-1.75cm}  
\begin{graybox}{}
\vspace{2Ex}
\textbf{\large Name:} \rule[0ex]{16.75em}{.4pt}
\hfill \textnormal{\textit{Nicht bestanden:}} \msquare \\[2.5Ex]
\textbf{\large Vorname:} \rule[0ex]{15em}{.4pt} \\[2.5Ex]
\textbf{\large Matrikelnummer:} \rule[0ex]{10.8em}{.4pt}
\hfill Endnote: \rule[0ex]{7em}{.4pt} 
\end{graybox}
\vspace{3Ex}
\phantom{text}
}
\def\examsemester{Sommersemester \& Wintersemester}
\def\examdate{\today}
%% ------------------------------------------------------------
\definecolor{darkblue}{rgb}{0,0,.5}
\definecolor{darkpurple}{rgb}{0.4117, 0.2, 0.4117}
\definecolor{uni}{rgb}{0,0.3137,0.6078}
\definecolor{gray}{gray}{0.7}

\usepackage{tcolorbox}
\definecolor{logo1}{RGB}{0, 158, 227}
\definecolor{gray5}{RGB}{247, 247, 247}
\definecolor{gray2}{RGB}{102, 102, 102}

\newtcolorbox{graybox}[1]{
  colback=gray5,%%red!5!white,
  colframe=gray2,%%red!75!black,
  fonttitle=\bfseries\Large,
  %%valign=center,
  fontupper=\large,
  before skip=10pt plus 2pt,
  after skip=20pt plus 4pt,
  title=#1}

\newtcolorbox{takehomebox}[1]{
  colback=gray5,%%red!5!white,
  colframe=logo1,%%red!75!black,
  fonttitle=\bfseries\Large,
  %%valign=center,
  fontupper=\large,
  before skip=10pt plus 2pt,
  after skip=10pt plus 2pt,
  title=#1}

\def\Rlogo{\includegraphics[width = 0.5cm]{\string~/Documents/GitHub/exam/img/Rlogo}\;}

\usepackage[scaled=.90]{helvet} 
\usepackage{fancyhdr}
\usepackage{lastpage}
\usepackage{hyperref}
\hypersetup{
    colorlinks=true,       % false: boxed links; true: colored links
    linkcolor=black,          % color of internal links 
    urlcolor=magenta           % color of external links
}
\renewcommand{\familydefault}{\sfdefault}

\title{
\large \exammodule \\[5Ex]
\Huge \examtitle \\[2Ex] 
\Large Hochschule Osnabr{\"u}ck
}
\author{Pr{\"u}fer: Prof. Dr. Jochen Kruppa \\
Fakult{\"a}t f{\"u}r Agrarwissenschaften und Landschaftsarchitektur \\ 
j.kruppa@hs-osnabrueck.de}
\date{Version vom \examdate}

%% ------------------------------------------------------------
%% by J.Kruppa on Tuesday, September 23, 2014 (12:50)
%% Header
\renewcommand{\headrulewidth}{0pt}
\renewcommand{\footrulewidth}{0pt}
\pagestyle{fancy}

\fancyhf{}
\fancyhead[L]{}
\fancyhead[R]{}
\fancyfoot[R]{\thepage}
\fancyfoot[L]{\footnotesize \examtitle}

\fancypagestyle{empty}{
 \fancyhf{}
 \fancyhead[L]{}
 \fancyhead[R]{}
 \fancyfoot[R]{\thepage}
 \fancyfoot[L]{\footnotesize \examtitle}
}

\usepackage{arevtext,arevmath}

\newcommand\Tstrut{\rule{0pt}{2.6ex}}         % = `top' strut
\newcommand\Bstrut{\rule[-0.9ex]{0pt}{0pt}}   % = `bottom' strut
\def\strut{\Tstrut\Bstrut}

% -----------------------------------------------------------------------
\IfFileExists{upquote.sty}{\usepackage{upquote}}{}
\begin{document}
% -----------------------------------------------------------------------
\maketitle
\thispagestyle{empty}
\clearpage
% -----------------------------------------------------------------------

\begin{graybox}{Erlaubte Hilfsmittel f{\"u}r die Klausur}
  \vspace{1Ex}
  \begin{itemize}
  \item Normaler Taschenrechner ohne M{\"o}glichkeit der Kommunikation mit anderen
    Ger{\"a}ten - also ausdr{\"u}cklich kein Handy!
  \item Eine DIN A4-Seite als beidseitig, selbstgeschriebene,
    handschriftliche Formelsammlung - keine digitalen Ausdrucke. 
  \item \textbf{You can answer the questions in English without any consequences.}  
  \end{itemize}
\end{graybox}
\vfill

\begin{graybox}{Ergebnis der Klausur}
  \vspace{1Ex}
  \begin{itemize}
  \item[] \rule[0ex]{3em}{.4pt}\, von 20\, Punkten sind aus dem Multiple
    Choice Teil erreicht.
  \item[] \rule[0ex]{3em}{.4pt}\, von 70 Punkten sind aus dem Rechen- und
    Textteil erreicht. 
  \item[] \rule[0ex]{3em}{.4pt}\, von 90 Punkten in Summe.
  \item[] Es wird folgender Notenschl{\"u}ssel angewendet.   
  \end{itemize}
  \vspace{1ex}
\begin{center}
  \begin{tabular}[c]{cc}
    \toprule
    \textbf{Punkte}	&	\textbf{Note}	\\
    \midrule
    86.0 - 90.0	&	1,0	\\
    81.5 - 85.5	&	1,3	\\
    77.0 - 81.0	&	1,7	\\
    72.5 - 76.5	&	2,0	\\
    68.0 - 72.0	&	2,3	\\
    63.5 - 67.5	&	2,7	\\
    59.0 - 63.0	&	3,0	\\
    54.5 - 58.5	&	3,3	\\
    50.0 - 54.0	&	3,7	\\
    45.0 - 49.5	&	4,0	\\
    \bottomrule
  \end{tabular}
\end{center}
  \vspace{1ex}
\begin{itemize}
\item[] Es ergibt sich eine Endnote von \rule[0ex]{4em}{.4pt}.
\end{itemize}
  \vspace{1Ex}
\end{graybox}

% -----------------------------------------------------------------------
\newpage
% -----------------------------------------------------------------------

\begin{graybox}{Multiple Choice Aufgaben}
  \begin{itemize}
  \item Pro Multipe Choice Frage ist \emph{genau} eine Antwort richtig.
  \item \textbf{Übertragen Sie Ihre Kreuze in die Tabelle auf
      dieser Seite.}
  \item Es werden nur Antworten berücksichtigt, die in dieser Tabelle
    angekreuzt sind!
  \end{itemize}

\begin{center}
  \large
  \begin{tabular}{|r|c|c|c|c|c||c|}
    \hline
    & \textbf{A} & \textbf{B} & \textbf{C} & \textbf{D} & \textbf{E} & $\boldsymbol{\checkmark}$\strut\\
    \hline
    1 Aufgabe &   &   &   &   &   & \strut\\
    \hline
    2 Aufgabe &   &   &   &   &   & \strut\\
    \hline
    3 Aufgabe &   &   &   &   &   & \strut\\
    \hline
    4 Aufgabe &   &   &   &   &   & \strut\\
    \hline
    5 Aufgabe &   &   &   &   &   & \strut\\
    \hline
    6 Aufgabe &   &   &   &   &   & \strut\\
    \hline
    7 Aufgabe &   &   &   &   &   & \strut\\
    \hline
    8 Aufgabe &   &   &   &   &   & \strut\\
    \hline
    9 Aufgabe &   &   &   &   &   & \strut\\
    \hline
    10 Aufgabe &   &   &   &   &   & \strut\\
    \hline
  \end{tabular}
\end{center}

\begin{itemize}
\item Es sind \rule[0ex]{2em}{.4pt}\, von 20 Punkten erreicht worden.
\end{itemize}
\end{graybox}

\vfill

\begin{graybox}{Rechen- und Textaufgaben}
  \begin{itemize}
  \item Die Tabelle wird vom Dozenten ausgefüllt.
  \end{itemize}
  \begin{center}
    \large
    \begin{tabular}{|l|c|c|c|c|c|c|c|}
      \hline
      \textbf{Aufgabe} & \textbf{11} & \textbf{12} & \textbf{13} & \textbf{14} & \textbf{15} & \textbf{16} & \textbf{17} \strut\\
      \hline
      \textbf{Punkte} & 
      \hspace{1Ex}\Large\textcolor{gray!70}{9}\hspace{1Ex}  & 
      \hspace{1Ex}\Large\textcolor{gray!70}{9}\hspace{1Ex}  & 
      \hspace{1Ex}\Large\textcolor{gray!70}{12}\hspace{1Ex}  & 
      \hspace{1Ex}\Large\textcolor{gray!70}{10}\hspace{1Ex}  & 
      \hspace{1Ex}\Large\textcolor{gray!70}{10}\hspace{1Ex}  & 
      \hspace{1Ex}\Large\textcolor{gray!70}{10}\hspace{1Ex}  & 
      \hspace{1Ex}\Large\textcolor{gray!70}{10}\hspace{1Ex} \strut\\
      \hline
  \end{tabular}
\end{center}
\begin{itemize}
\item Es sind \rule[0ex]{2em}{.4pt}\, von 70 Punkten erreicht worden.
\end{itemize}
\end{graybox}

% -----------------------------------------------------------------------
\clearpage
% -----------------------------------------------------------------------


\section{Aufgabe \hfill (2 Punkte)}

%% --------------------------------------------------------------------
\ifcollection
\begin{flushright}
\tiny\vspace{-2Ex}
\textbf{\examinhaltstart}
\exammodulemathstat $\;\bullet$
\exammodulestat $\;\bullet$
\exammodulestatbbv 
\vspace{-1Ex}
\end{flushright}
\fi
%% --------------------------------------------------------------------




Um die \uline{Standardabweichung} zu berechnen müssen wir folgende Rechenoperationen durchführen.



\begin{enumerate}
\item [\textbf{A} \msquare] Wir berechnen erst den Mittelwert und dann die quadratischen Abstände zu dem Mittelwert. Diese quadratischen Abstände summieren wir auf und teilen am Ende durch die Fallzahl $(n-1)$.
\item [\textbf{B} \msquare] Als erstes berechnen wir den Mittelwert. Dann bilden wir die Summe der quadratischen Abstände zu dem Mittelwert. Abschließend subtrahieren wir die Fallzahl $(n-1)$.
\item [\textbf{C} \msquare] Den Mittelwert berechen, dann die absoluten Abstände zum Mittelwert aufsummieren. Die Fallzahl $(n-1)$ entsprechend gewichten.
\item [\textbf{D} \msquare] Den Mittelwert berechen, dann die quadratischen Abstände zum Mittelwert aufsummieren und durch die Fallzahl $(n-1)$ teilen, dann die Wurzel ziehen.
\item [\textbf{E} \msquare] Den Median berechen, dann die quadratischen Abstände zum Median aufsummieren, dann die Wurzel ziehen. Am Ende durch die Fallzahl ($n-1$) teilen
\end{enumerate} 

\section{Aufgabe \hfill (2 Punkte)}

%% --------------------------------------------------------------------
\ifcollection
\begin{flushright}
\tiny\vspace{-2Ex}
\textbf{\examinhaltstart}
\exammodulemathstat $\;\bullet$
\exammodulestat $\;\bullet$
\exammodulestatbbv 
\vspace{-1Ex}
\end{flushright}
\fi
%% --------------------------------------------------------------------




Gegeben ist $y$ mit 19, 19, 11, 13 und 18. Berechnen Sie den Mittelwert und Standardabweichung.



\begin{enumerate}
\item [\textbf{A} \msquare] Sie erhalten 16 +/- 1.87
\item [\textbf{B} \msquare] Es berechnet sich 17 +/- 14
\item [\textbf{C} \msquare] Sie erhalten 16 +/- 3.74
\item [\textbf{D} \msquare] Sie erhalten 16 +/- 1.93
\item [\textbf{E} \msquare] Es ergibt sich 15 +/- 7
\end{enumerate} 

\section{Aufgabe \hfill (2 Punkte)}

%% --------------------------------------------------------------------
\ifcollection
\begin{flushright}
\tiny\vspace{-2Ex}
\textbf{\examinhaltstart}
\exammodulemathstat $\;\bullet$
\exammodulestat $\;\bullet$
\exammodulestatbbv $\;\bullet$
\exammodulestatversuch $\;\bullet$
\exammodulebiostat
\vspace{-1Ex}
\end{flushright}
\fi
%% --------------------------------------------------------------------




Ein statistischer Test produziert für einen Gruppenvergleich einen $p$-Wert. Welche Aussage zusammen mit dem Signifikanzniveau $\alpha$ gleich 5\% stimmt?



\begin{enumerate}
\item [\textbf{A} \msquare] Wir vergleichen mit dem $p$-Wert und dem Signifikanzniveau $\alpha$ absolute Werte auf einem Zahlenstrahl und damit den Unterschied der Teststatistiken, wenn die $H_0$ gilt.
\item [\textbf{B} \msquare] Wir machen ein Aussage über die Flächen unter der Kurve der Teststatistik der Hypothese $H_0$ und über die Flächen unter den Kurve der Teststatistik der Hypothese $H_A$. Dabei werden Wahrscheinlichkeiten vergleichen, die durch die Flächen unter der Kurve der beiden Testverteilungen repräsentiert werden.
\item [\textbf{C} \msquare] Wir machen eine Aussage über die indivduelle Wahrscheinlichkeit des Eintretens der Nullhypothese $H_0$. Der $p$-Wert wird mit dem Signifikanzniveau verglichen und bewertet.
\item [\textbf{D} \msquare] Wir machen ein Aussage über die Flächen und der Kurve der Teststatistik, wenn die $H_0$ gilt. Dabei werden Wahrscheinlichkeiten vergleichen, die durch die Flächen unter der Kurve repräsentiert werden.
\item [\textbf{E} \msquare] Wir vergleichen mit dem $p$-Wert und dem Signifikanzniveau $\alpha$ Wahrscheinlichkeiten und damit die absoluten Werte auf einem Zahlenstrahl, wenn die $H_0$ gilt.
\end{enumerate} 

\section{Aufgabe \hfill (2 Punkte)}

%% --------------------------------------------------------------------
\ifcollection
\begin{flushright}
\tiny\vspace{-2Ex}
\textbf{\examinhaltstart}
\exammodulemathstat $\;\bullet$
\exammodulestat $\;\bullet$
\exammodulestatbbv $\;\bullet$
\exammodulestatversuch $\;\bullet$
\exammodulebiostat
\vspace{-1Ex}
\end{flushright}
\fi
%% --------------------------------------------------------------------




Das statistische Testen basiert auf dem Falsifikationsprinzip. Es besagt,



\begin{enumerate}
\item [\textbf{A} \msquare] ... dass in der Wissenschaft immer etwas falsch sein muss. Sonst gebe es keinen Fortschritt.
\item [\textbf{B} \msquare] ... dass ein schlechtes Modell durch ein schlechteres Modell ersetzt wird. Die Wissenschaft lehnt ab und verifiziert nicht.
\item [\textbf{C} \msquare] ... dass ein minderwertes Modell durch ein minderwertiges Modell ersetzt wird. Es gilt das Verifikationsprinzip nach Karl Popper.
\item [\textbf{D} \msquare] ... dass Modelle meist falsch sind und selten richtig.
\item [\textbf{E} \msquare] ... dass ein schlechtes Modell durch das Falsifikationsprinzip durch ein weniger schlechtes Modell ersetzt wird.
\end{enumerate} 

\section{Aufgabe \hfill (2 Punkte)}

%% --------------------------------------------------------------------
\ifcollection
\begin{flushright}
\tiny\vspace{-2Ex}
\textbf{\examinhaltstart}
\exammodulemathstat $\;\bullet$
\exammodulestat $\;\bullet$
\exammodulestatbbv $\;\bullet$
\exammodulestatversuch $\;\bullet$
\exammodulebiostat
\vspace{-1Ex}
\end{flushright}
\fi
%% --------------------------------------------------------------------




Um die Testtheorie besser zu verstehen, mag es manchmal sinnvoll sein ein Beispiel aus dem Alltag zu wählen. Die Ergebnisse der Analyse durch einen statistischen Test können auch in grobe Analogie zur Wettervorhersage gebracht werden. Welche Aussage trifft am ehesten zu?



\begin{enumerate}
\item [\textbf{A} \msquare] In der Analogie der Wahrscheinlichkeit für Regen: ein statistischer Test erlaubt die Wahrscheinlichkeit für ein Ereignis abzuschätzen. Die Stärke des Effektes können wir nicht bestimmen.
\item [\textbf{B} \msquare] In der Analogie der Regenwahrscheinlichkeit in einem bestimmten Gebiet: ein statistischer Test gibt die Wahrscheinlichkeit für ein Ereignis in einem Experiment mit den Daten $D$ wieder und lässt sich kaum verallgemeinern.
\item [\textbf{C} \msquare] In der Analogie der Sonnenscheindauer: Wie lange kann mit einem entsprechenden Effekt gerechnet werden? Die Wahrscheinlichkeit für den Effekt gibt der statistische Test wieder.
\item [\textbf{D} \msquare] In der Analogie des Niederschlags oder Regenmenge: ein statistischer Test gibt die Stärke eines Effektes wieder. Zum Beispiel, wie hoch ist der Mittelwertsunterschied.
\item [\textbf{E} \msquare] In der Analogie der Maximaltemperatur: Was ist der maximale Unterschied zwischen zwei Gruppen. Wir erhalten hier eine Aussage über die Spannweite und den maximalen Effekt.
\end{enumerate} 

\section{Aufgabe \hfill (2 Punkte)}

%% --------------------------------------------------------------------
\ifcollection
\begin{flushright}
\tiny\vspace{-2Ex}
\textbf{\examinhaltstart}
\exammodulemathstat $\;\bullet$
\exammodulestat $\;\bullet$
\exammodulestatbbv 
\vspace{-1Ex}
\end{flushright}
\fi
%% --------------------------------------------------------------------




Nach einem Feldexperiment wollen Sie zwei Gruppen mit einem Welch t-Test vergleichen. Welche Aussage ist auch für den Student t-Test richtig?



\begin{enumerate}
\item [\textbf{A} \msquare] Der t-Test vergleicht die Mittelwerte von zwei Gruppen unter der strikten Annahme von Varianzhomogenität. Sollte keine Varianzhomogenität vorliegen, so gibt es keine Möglichkeit den t-Test in einer Variante anzuwenden.
\item [\textbf{B} \msquare] Der t-Test berechnet die Differenz von zwei Mittelwerten als Effekt und gibt eine Entscheidung, ob sich die beiden Mittelwerte \textit{jeweils} von Null unterscheiden.
\item [\textbf{C} \msquare] Der t-Test vergleicht die Mittelwerte von zwei Gruppen.
\item [\textbf{D} \msquare] Der t-Test testet generell zu einem erhöhten $\alpha$-Niveau von 20\%.
\item [\textbf{E} \msquare] Der t-Test vergleicht zwei oder mehr Gruppen indem die Mittelwerte miteinander verglichen werden.
\end{enumerate} 

\section{Aufgabe \hfill (2 Punkte)}

%% --------------------------------------------------------------------
\ifcollection
\begin{flushright}
\tiny\vspace{-2Ex}
\textbf{\examinhaltstart}
\exammodulemathstat $\;\bullet$
\exammodulestat $\;\bullet$
\exammodulestatbbv $\;\bullet$
\exammodulestatversuch $\;\bullet$
\exammodulebiostat
\vspace{-1Ex}
\end{flushright}
\fi
%% --------------------------------------------------------------------



Roland Fischer entwickelte Anfang des letzten Jahrhunderts als Grundlage für das experimentelle Design in der Statistik die Randomisierung. Warum ist die Randomisierung für die Entscheidung anhand einer statistischen Auswertung so wichtig?



\begin{enumerate}
\item [\textbf{A} \msquare] Randomisierung war bis 1952 bedeutend, wurde dann aber in Folge besserer Rechnerleistung nicht mehr verwendet. Aktuelle Statistik nutzt keine Randomisierung mehr.
\item [\textbf{B} \msquare] Durch eine Randomisierung können wir von Strukturgleichheit zwischen der Stichprobe und der Grundgesamtheit ausgehen.
\item [\textbf{C} \msquare] Durch eine Randomisierung können wir nicht von Strukturgleichheit zwischen der Stichprobe und der Grundgesamtheit ausgehen.
\item [\textbf{D} \msquare] Randomisierung erlaubt erst die Mittelwerte zu schätzen. Ohne Randomisierung keine Mittelwerte. Ohne Mittelwerte keine Varianz und somit auch kein statistischer Test.
\item [\textbf{E} \msquare] Randomisierung erlaubt erst die Varianzen zu schätzen. Ohne eine Randomisierung ist die Berechnung von Mittelwerten und Varianzen nicht möglich. Dadurch lässt sich erst ein Experiment auswerten.
\end{enumerate} 

\section{Aufgabe \hfill (2 Punkte)}

%% --------------------------------------------------------------------
\ifcollection
\begin{flushright}
\tiny\vspace{-2Ex}
\textbf{\examinhaltstart}
\exammodulemathstat $\;\bullet$
\exammodulestat $\;\bullet$
\exammodulestatbbv $\;\bullet$
\exammodulestatversuch $\;\bullet$
\exammodulebiostat
\vspace{-1Ex}
\end{flushright}
\fi
%% --------------------------------------------------------------------




Welche Aussage zum mathematische Ausdruck $Pr(D|H_0)$ ist richtig?



\begin{enumerate}
\item [\textbf{A} \msquare] $Pr(D|H_0)$ ist die Wahrscheinlichkeit nicht die Daten $D$ zu beobachten sondern die Nullhypothese, wenn diese wahr ist.
\item [\textbf{B} \msquare] $Pr(D|H_0)$ stellt die Wahrscheinlichkeit die Daten $D$ und somit die Teststatistik $T_D$ zu beobachten dar, wenn die Nullhypothese wahr ist.
\item [\textbf{C} \msquare] Die Wahrscheinlichkeit für die Nullhypothese, wenn die Daten wahr sind.
\item [\textbf{D} \msquare] Die Inverse der Wahrscheinlichkeit unter der die Nullhypothese nicht mehr die Alternativehypothese überdeckt.
\item [\textbf{E} \msquare] Die Wahrscheinlichkeit der Daten unter der Nullhypothese in der Grundgesamtheit.
\end{enumerate}

\section{Aufgabe \hfill (2 Punkte)}

%% --------------------------------------------------------------------
\ifcollection
\begin{flushright}
\tiny\vspace{-2Ex}
\textbf{\examinhaltstart}
\exammodulemathstat $\;\bullet$
\exammodulestat $\;\bullet$
\exammodulestatbbv 
\vspace{-1Ex}
\end{flushright}
\fi
%% --------------------------------------------------------------------




Berechnen Sie den Median, das $1^{st}$ Quartile sowie das $3^{rd}$ Quartile von $y$ mit 14, 30, 29, 20, 19, 18 und 51.




\begin{enumerate}
\item [\textbf{A} \msquare] Es ergibt sich 26 +/- 18
\item [\textbf{B} \msquare] Sie erhalten 20 [16; 28]
\item [\textbf{C} \msquare] Es berechnet sich 21 [19; 29]
\item [\textbf{D} \msquare] Es ergibt sich 20 +/- 18
\item [\textbf{E} \msquare] Es berechnet sich 20 [18; 30]
\end{enumerate} 

\section{Aufgabe \hfill (2 Punkte)}

%% --------------------------------------------------------------------
\ifcollection
\begin{flushright}
\tiny\vspace{-2Ex}
\textbf{\examinhaltstart}
\exammodulemathstat $\;\bullet$
\exammodulestat $\;\bullet$
\exammodulestatbbv $\;\bullet$
\exammodulestatversuch $\;\bullet$
\exammodulebiostat
\vspace{-1Ex}
\end{flushright}
\fi
%% --------------------------------------------------------------------




Der \uline{Boxplot} stellt folgende statistische Maßzahlen in einer Abbildung dar. Damit gehört der Boxplot zu einem der am meisten genutzten statistischen Verfahren zur Visualisierung von Daten.

 



\begin{enumerate}
\item [\textbf{A} \msquare] Den Mittelwert und die Varianz.
\item [\textbf{B} \msquare] Den Median und die Standardabweichung.
\item [\textbf{C} \msquare] Der Boxplot stellt die Mittelwerte und die Varianz dar.
\item [\textbf{D} \msquare] Der Boxplot stellt die Mittelwerte und die Standardabweichung dar.
\item [\textbf{E} \msquare] Den Median und die Quartile.
\end{enumerate}

% -----------------------------------------------------------------------
\clearpage
% -----------------------------------------------------------------------

\section{Aufgabe \hfill (9 Punkte)}

\textit{Geben Sie grundsätzlich Formeln und Rechenweg zur Lösung der Teilaufgaben mit an!} \\[1Ex]
 

 
%% --------------------------------------------------------------------
\ifcollection
\begin{flushright}
\tiny\vspace{-3Ex}
\textbf{\examinhaltstart}
\exammodulemathstat $\;\bullet$
\exammodulestat $\;\bullet$
\exammodulestatbbv $\;\bullet$
\exammodulestatversuch\\
\exammodulelanddaten $\;\bullet$
\exammodulebiostat
\vspace{-4Ex}
\end{flushright}
\begin{minipage}[t]{0.5\textwidth}
\includegraphics[width = 1.3cm]{/Users/kruppajo/work/GitHub/exam/avatare/Alex.png}
\end{minipage}
\begin{minipage}[t]{0.5\textwidth}
\hfill
\href{https://youtu.be/MyrNZ46i3uY}{\includegraphics[width = 2cm]{img/youtube}}
\end{minipage}
\vspace{-3ex}
\fi
%% --------------------------------------------------------------------



\ifcollection
\paragraph{Zerforschen des Boxplots}
\fi

Boxplots sind bedeutend in der Darstellung von wissenschaftlichen Ergebnissen. Leider hat sich Alex nicht gemerkt, welche statistischen Maßzahlen für einen Boxplot erhoben werden müssen. Besser wäre was anderes gewesen. Am Ende dann doch besser Starcraft. Wunderbar. Eine echte Ablenkung für Alex. Das ist in soweit doof, da nach seinem Betreuer erstmal ein Boxplot nachgebaut werden soll, bevor es mit seinem Projektbericht losgeht. Dann hat er schonmal den \Rlogo Code vorliegen und nachher geht dann alles schneller. Na dann mal los. Alex schafft sich die nötige Stimmung. Alex nickt im Takt von Abba und bemerkt dabei gar nicht was die Katze schon wieder anstellt. In der Behandlung für Kartoffeln werden verschiedene Bewässerungstypen ($low$, $mid$ und $high$) sein. Erfasst wird als Outcome ($Y$) \textit{Trockengewicht}. Alex soll dann \textit{drymatter} in seiner Exceldatei eintragen. Aber nur in passender Atmospäre! Schon dutzende Male gesehen: Alien. Aber immer noch großartig zusammen mit Gummibärchen.



{\centering \includegraphics[width=\maxwidth]{img/boxplot-02-zer-1} 

}




Leider kennt sich Alex mit der Erstellung von Boxplots in \Rlogo nicht aus. Deshalb braucht er bei der Visualisierung Ihre Hilfe!

\begin{enumerate}
\item Formulieren Sie die wissenschaftliche Fragestellung! \textbf{(1 Punkt)}
\item Erstellen Sie eine Tabelle mit den statistischen Maßzahlen aus der obigen Abbildung der drei Boxplots! \textit{Beachten Sie die korrekte Darstellungsform der statistischen Maßzahlen!} \textbf{(2 Punkte)}
\item Beschriften Sie \textit{einen} der Boxplots mit den gängigen statistischen Maßzahlen! \textbf{(2 Punkte)}
\item Erstellen Sie einen beispielhaften Datensatz, aus dem die drei Boxplots \textit{möglicherweise} erstellt wurden, im \Rlogo üblichen Format! \textbf{(2 Punkte)}
\item Kann Alex einen Unterschied zwischen den Behandlungen erwarten? Begründen Sie Ihre Antwort! \textbf{(2 Punkte)}
\end{enumerate} 
\clearpage
% -----------------------------------------------------------------------

\section{Aufgabe \hfill (9 Punkte)}


 
%% --------------------------------------------------------------------
\ifcollection
\begin{flushright}
\tiny
\textbf{\examinhaltstart}
\exammodulestat $\;\bullet$
\exammodulestatbbv $\;\bullet$
\exammodulestatversuch $\;\bullet$
\exammodulebiostat
\vspace{-4Ex}
\end{flushright}
\begin{minipage}[t]{0.5\textwidth}
\includegraphics[width = 1.3cm]{/Users/kruppajo/work/GitHub/exam/avatare/Steffen.png}\hspace{-4mm}\includegraphics[width = 1.3cm]{/Users/kruppajo/work/GitHub/exam/avatare/Yuki.png}
\end{minipage}
\begin{minipage}[t]{0.5\textwidth}
\hfill
\href{https://youtu.be/KxDklbFEEHw}{\includegraphics[width = 2cm]{img/youtube}}
\end{minipage}
\fi
%% --------------------------------------------------------------------



\ifcollection
\paragraph{Grundgesamtheit und experimentelle Stichprobe}
\fi

'Schnapspraline?', räuspert sich das Känguruh und schaukelt in der Wippe. Steffen und Yuki schauen erschrocken auf. 'Ähm, das ist hier eine Klausuraufgabe...', merkt Steffen mit leicht schrägen Blick an. 'Ich mache hier ein Praktikum und schreibe Teile der Aufgaben.', gähnt das Känguruh. 'Also, ich glaube das ist so nicht gedacht. Und das sind die Schnapspralinen für den Geburtstag meiner Oma!', ruft Yuki. 'Pillepalle! Meins, deins, das sind doch alles bürgerliche Kategorien!', entgegnet das Känguruh und liest von einem zerknitterten Stück Papier ab: 'Was ist der Unterschied zwischen dem Einen und dem Anderen. Steht alles unten in den Fragen. Einfach selber lesen...'.

\vspace{1ex}

Leider kennen sich Steffen und Yuki mit der Grundgesamtheit und der Stichprobe überhaupt nicht aus. Daher sind Sie gefragt!

\begin{enumerate}
\item Erklären Sie den Zusammenhang zwischen Stichprobe und Grundgesamtheit an einem Schaubild! Beschriften Sie das Schaubild entsprechend! \textit{Nutzen Sie hierfür als Veranschaulichung ein aussagekräftiges Beispiel!}  \textbf{(3 Punkte)}
\begin{enumerate}
\item Nennen Sie das statistische Verfahren um von einer Grundgesamtheit auf eine Stichprobe zu gelangen!  \textbf{(1 Punkt)}
\item Nennen Sie ein konkretes Beispiel zur Durchführung um von einer Grundgesamtheit auf eine Stichprobe zu gelangen! \textbf{(1 Punkt)}
\item Benennen Sie die statistische Eigenschaft, die zwischen Grundgesamtheit und Stichprobe vorliegen muss! \textbf{(1 Punkt)}
\end{enumerate}
\item Erweitern Sie das Schaubild um die Entstehung von $Pr(D|H_0)$! \textit{Nutzen Sie hierfür als Veranschaulichung ein aussagekräftiges Beispiel!}  \textbf{(3 Punkte)}
\end{enumerate}

 
\clearpage
% -----------------------------------------------------------------------

\section{Aufgabe \hfill (9 Punkte)}

\textit{Geben Sie grundsätzlich Formeln und Rechenweg zur Lösung der Teilaufgaben mit an!} \\[1Ex]
 

 
%% --------------------------------------------------------------------
\ifcollection
\begin{flushright}
\tiny\vspace{-3Ex}
\textbf{\examinhaltstart}
\exammodulemathstat $\;\bullet$
\exammodulestat $\;\bullet$
\exammodulestatbbv 
\vspace{-4Ex}
\end{flushright}
\begin{minipage}[t]{0.5\textwidth}
\includegraphics[width = 1.3cm]{/Users/kruppajo/work/GitHub/exam/avatare/Steffen.png}
\end{minipage}
\begin{minipage}[t]{0.5\textwidth}
\hfill
\href{https://youtu.be/zgpw9GC0plk}{\includegraphics[width = 2cm]{img/youtube}}
\end{minipage}
\vspace{-3ex}
\fi
%% --------------------------------------------------------------------



\ifcollection
\paragraph{Berechnung des Student t-Test \underline{oder} Welch t-Test}
\fi

Der t-Test. Steffen erschaudert. Steffen und die Romantik, eine unendliche Geschichte mit kniffeligen Wendungen. Ein mächtiges Werkzeug ist der t-Test in den Händen desjenigen, der einen normalverteilten Messwert ($Y$) hat. Aber erstmal überhaupt den t-Test rechnen können. Wie sah das Experiment von Steffen überhaupt aus? Steffen schmeißt noch eine Handvoll Oreos in seinen Rachen. Im Hintergrund klirrt leise der Spiegel zum Sound von Taylor Swift. Steffen hat ein Feldexperiment mit Kartoffeln durchgeführt um eine neue technische Versuchsanlage zu testen. Bei dem Pilotexperiment mit sehr geringer Fallzahl $(n_1 = n_2 = 3)$ wurde die Behandlung Genotypen ($AA$ und $BB$) an den Kartoffeln getestet und dabei wurde geschaut, ob der Versuch überhaupt technisch klappen könnte. Gemessen hat Steffen dann als Messwert Proteingehalt [g/kg]. Warum der Versuch im Wendland für seinen Projektbericht stattfinden musste, ist ihm bis heute ein Rätsel. Egal. Gibt es jetzt einen Zusammenhang zwischen der Behandlung und Proteingehalt [g/kg]?

\begin{table}[!h]
\centering
\begin{tabular}{cc}
\toprule
Genotypen & Proteingehalt\\
\midrule
BB & 14.4\\
AA & 21.5\\
BB & 20.8\\
BB & 15.2\\
AA & 19.5\\
\addlinespace
AA & 17.7\\
\bottomrule
\end{tabular}
\end{table}



Leider kennt sich Steffen mit der Berechnung eines Welch t-Tests überhaupt nicht aus. Deshalb braucht er bei der Berechnung Ihre Hilfe!

\begin{enumerate}
  \item Formulieren Sie die wissenschaftliche Fragestellung! \textbf{(1 Punkt)}
  \item Bestimmen Sie die Teststatistik $T_{D}$ eines Welch t-Tests! \textbf{(3 Punkte)}
  \item Treffen Sie mit $T_{\alpha = 5\%} = 2.68$ eine Aussage zur Nullhypothese! Begründen Sie Ihre Antwort! \textbf{(2 Punkte)}
  \item Berechnen Sie den Effekt des Welch t-Tests! \textbf{(1 Punkt)}
  \item Formulieren Sie eine Antwort an Steffen über das Ergebnis Ihrer statistischen Analyse! \textbf{(2 Punkte)}
\end{enumerate} 
\clearpage
% -----------------------------------------------------------------------

\section{Aufgabe \hfill (10 Punkte)}

\textit{Geben Sie grundsätzlich Formeln und Rechenweg zur Lösung der Teilaufgaben mit an!} \\[1Ex]
 

 
%% --------------------------------------------------------------------
\ifcollection
\begin{flushright}
\tiny\vspace{-3Ex}
\textbf{\examinhaltstart}
\exammodulemathstat
\vspace{-4Ex}
\end{flushright}
\begin{minipage}[t]{0.5\textwidth}
\includegraphics[width = 1.3cm]{/Users/kruppajo/work/GitHub/exam/avatare/Jessica.png}\hspace{-4mm}\includegraphics[width = 1.3cm]{/Users/kruppajo/work/GitHub/exam/avatare/Jonas.png}\hspace{-4mm}\includegraphics[width = 1.3cm]{/Users/kruppajo/work/GitHub/exam/avatare/Nilufar.png}\hspace{-4mm}\includegraphics[width = 1.3cm]{/Users/kruppajo/work/GitHub/exam/avatare/Tina.png}
\end{minipage}
\begin{minipage}[t]{0.5\textwidth}
\hfill
\href{https://youtu.be/aBxLkdF-c4M}{\includegraphics[width = 2cm]{img/youtube}}
\end{minipage}
\fi
%% --------------------------------------------------------------------




\ifcollection
\paragraph{Solar- \& Biogasanlagen}
\fi



Tina bringt ein neues, tolles Projekt mit in die Lerngruppe \textit{Sinnlos im Studium} bestehend aus ihr, Nilufar, Jessica sowie Jonas. Um die Energiekosten ihres Betriebes zu senken, will sie eine Solaranlage auf den Rinderstall montieren lassen. Dafür hat sie ihren Stall ausgemessen und findet folgende Maße wieder. Die vordere Seite des Rinderstall hat eine Höhe $h_v$ von $6m$. Die hintere Seite des Rinderstall hat eine Höhe $h_b$ von $9m$. Der Rinderstall hat eine Tiefe $t$ von $13m$ und eine Breite $b$ von $40m$. 'Sag mal Tina, ist das eine Matheaufgabe oder rechnen wir hier gerade für dich kostenlos als menschliche Computer Sachen für deinen Betrieb?', fragt Jessica mit erhobenenen Augenbrauen. Jonas und Nilufar nicken zustimmend.

\vspace{1Ex}

Wenn die Lerngruppe nicht will, dann müssen Sie bei der Planung helfen!

\begin{enumerate}
\item Skizzieren Sie den Rinderstall auf dem die Solaranlage montiert werden soll! Ergänzen Sie die Angaben für die Höhen $h_v$, $h_b$, die Tiefe $t$ und die Breite $b$ des Stalls!  \textbf{(2 Punkte)}
\item Berechnen Sie die Fläche der schrägen, neuen Solaranlage auf dem Rinderstall! \textbf{(3 Punkte)}
\end{enumerate}

Ebenfalls plant Tina eine neue Biogasanlage für ihren Betrieb. Der neue Methantank hat einen Radius $r$ von $1m$. Leider gibt es ein paar bauliche Beschränkungen auf dem Grundstück. Das Fundament des zylindrischen Methantanks kann nur ein Gewicht von maximal $12t$ aushalten bevor der Tank wegbricht. Tina rechnen eine Sicherheitstoleranz von $25\%$ ein beinhaltend das Gewicht des Methantanks. In flüssiger Form hat Methan bei $-80^\circ\text{C}$ eine Dichte von $220kg/m^3$. Bei $-100^\circ\text{C}$ hat Methan eine Dichte von $270kg/m^3$. Tina betreibt ihre Anlage bei $-95^\circ\text{C}$.

\begin{enumerate}
  \setcounter{enumi}{2}
\item Extrapolieren Sie die effektive Dichte des Methans in dem Methantank! Welche Annahme haben Sie getroffen? \textbf{(1 Punkt)}
\item Berechnen Sie wie viel Kubikmeter $m^3$ in den Methantank gefüllen werden können, bevor das Fundament nachgibt! \textbf{(2 Punkte)}
\item Berechnen Sie die maximale Höhe $h_{max}$ in $m$ für den gefüllten Methantank mit dem Radius $r$, bevor das Fundament wegbricht! \textbf{(2 Punkte)}
\end{enumerate}

 
\clearpage
% -----------------------------------------------------------------------

\section{Aufgabe \hfill (10 Punkte)}

\textit{Geben Sie grundsätzlich Formeln und Rechenweg zur Lösung der Teilaufgaben mit an!} \\[1Ex]
 

 
%% --------------------------------------------------------------------
\ifcollection
\begin{flushright}
\tiny\vspace{-3Ex}
\textbf{\examinhaltstart}
\exammodulemathstat
\vspace{-4Ex}
\end{flushright}
\begin{minipage}[t]{0.5\textwidth}
\includegraphics[width = 1.3cm]{/Users/kruppajo/work/GitHub/exam/avatare/Jessica.png}\hspace{-4mm}\includegraphics[width = 1.3cm]{/Users/kruppajo/work/GitHub/exam/avatare/Jonas.png}\hspace{-4mm}\includegraphics[width = 1.3cm]{/Users/kruppajo/work/GitHub/exam/avatare/Steffen.png}\hspace{-4mm}\includegraphics[width = 1.3cm]{/Users/kruppajo/work/GitHub/exam/avatare/Tina.png}
\end{minipage}
\begin{minipage}[t]{0.5\textwidth}
\hfill
\href{https://youtu.be/n451XnhtSh4}{\includegraphics[width = 2cm]{img/youtube}}
\end{minipage}
\fi
%% --------------------------------------------------------------------



\ifcollection
\paragraph{'Entschuldigung, ist das Ihre Feder in meinem Auge?'}
\fi



Man hört schon häufig vieles Geschnatter von höflichen Gänse in Mastställen. Enge ist natürlich etwas ungünstig, den dann kommt es zu Picken und Kannibalismus. Denn wenn der Nachbar nervt, dann muss zu Maßnahmen gegriffen werden. Kennt jeder aus einer mittelmäßigen Wohngemeinschaft. Das wollen Steffen, Jessica, Tina und Jonas aber als vorsorgliche Gänse-Halter:innen nicht\footnote{Die Quelle der Inspiration für die Aufgabe war der folgende wissenschaftliche Artikel: \href{https://www.efsa.europa.eu/en/efsajournal/pub/7788}{EFSA Panel on Animal Health and Welfare, et al. (2023) Welfare of broilers on farm. EFSA Journal 21.2}.}. Gemeinsam sind die Vier in einer Projektgruppe gelandet. Betrachten wir also gemeinsam einmal das Platzangebot (eng. \textit{space allowance}, abk. \textit{SA}) der Gänse für vier Tätigkeiten und versuchen die notwendige Fläche zu optimieren. Wie immer gibt es dafür eine mathematische Formel:

\begin{center}
  \begin{tabular}{cc}
    $SA = \sum^n_{i = 1} (A_i \times PB_i)$ & $A_i = \pi \times (r_i + R_i)^2$\\
  \end{tabular}
\end{center}

\vspace{-2Ex}

mit

\begin{itemize}[noitemsep]
\item $SA$ dem benötigten Platzangebot aller aufsummierten Verhalten $i$.
\item $A_i$ dem benötigten Platz für ein Verhalten $i$. 
\item $PB_i$ dem Anteil des Auftretens eines Verhaltens $i$.
\item $r_i$ dem Radius Gans plus dem benötigten Radius für das Verhalten $i$.
\item $R_i$ dem notwendigen Abstand zu den Nachbarn für das Verhalten $i$.    
\item $i$ dem Verhalten: (1) dustbathing, (2) walking, (3)
  wing/leg stretching und (4) drinking/eating.
\end{itemize}

In der folgenden Tabelle 1 sind die Werte für $r_i$, $R_i$ und $PB_i$ für ein spezifisches Verhalten $i$ aus drei wissenschaftlichen Veröffentlichungen dargestellt.

\vspace{-1Ex}

{\small
\begin{knitrout}
\definecolor{shadecolor}{rgb}{0.969, 0.969, 0.969}\color{fgcolor}\begin{table}[!h]
\centering
\begin{tabular}{llll}
\toprule
  & Aldridge et al. (2021) & Baxter et al. (2022) & Jabcobs et al. (2019)\\
\midrule
dustbathing & 35cm; 31cm; 2.1\% & 31cm; 12cm; 2.1\% & 36cm; 19cm; 4.1\%\\
walking & 22cm; 9cm; 8.1\% & 37cm; 34cm; 2.4\% & 53cm; 29cm; 8.1\%\\
wing/leg stretching & 43cm; 25cm; 3.2\% & 23cm; 32cm; 6.1\% & 49cm; 35cm; 3.2\%\\
drinking/eating & 30cm; 53cm; 16.4\% & 37cm; 22cm; 12.8\% & 14cm; 14cm; 16.4\%\\
\bottomrule
\end{tabular}
\end{table}

\end{knitrout}
}

Leider kennen sich die Vier nicht so gut mit der Berechnung aus! Daher brauchen die Vier Ihre Hilfe!

\begin{enumerate}
\item Skizzieren Sie die Werte $r_i$, $R_i$ und $A_i$ für zwei nebeneinander agierende Gänse für ein Verhalten $i$. Nutzen Sie hierfür vereinfachte geometrische Formen! \textbf{(2 Punkte)}
\item Erstellen Sie eine zusammenfassende Tabelle mit den mittleren Werten für $r$, $R$ und $PB$ aus der obigen Tabelle 1 für die jeweiligen Verhalten! \textbf{(3 Punkte)}
\item Ergänzen Sie eine Spalte mit dem benötigten Platz $A$ für das jeweilige Verhalten, welches sich aus den mittleren Werten ergibt! \textbf{(1 Punkt)}
\item Berechnen Sie das benötigte Platzangebot $SA$ für alle betrachteten Verhalten! \textbf{(1 Punkt)}
\item Sie entnehmen der Literatur folgende Aussage zur Verteilung der Gänse in der Fläche $A$: \textit{"`Assuming, that the animals will optimally and equally distribute in an area $A$, we observe a
    small part, which is not covered. This area is called $\omega$ and is calculated with $\omega = \tfrac{A}{0.9069}$."'} Veranschaulichen Sie die Fläche $\omega$ in einer aussagekräftigen Abbildung!  \textbf{(1 Punkt)}
\item Ein Tier braucht Platz für sich selbst. Berechnen Sie nun die Körperfläche $a$, die ein Tier einnimmt. Welche Annahmen haben Sie für die Berechnung der Körperfläche getroffen? \textbf{(2 Punkte)}
\end{enumerate}



 
\clearpage
% -----------------------------------------------------------------------

\section{Aufgabe \hfill (10 Punkte)}

\textit{Geben Sie grundsätzlich Formeln und Rechenweg zur Lösung der Teilaufgaben mit an!} \\[1Ex]
 

 
%% --------------------------------------------------------------------
\ifcollection
\begin{flushright}
\tiny\vspace{-3Ex}
\textbf{\examinhaltstart}
\exammodulemathstat
\vspace{-4Ex}
\end{flushright}
\begin{minipage}[t]{0.5\textwidth}
\includegraphics[width = 1.3cm]{/Users/kruppajo/work/GitHub/exam/avatare/Alex.png}
\end{minipage}
\begin{minipage}[t]{0.5\textwidth}
\hfill
\href{https://youtu.be/4-dSaPMhK9s}{\includegraphics[width = 2cm]{img/youtube}}
\end{minipage}
\fi
%% --------------------------------------------------------------------

%% --------------------------------------------------------------------
{\tiny\textbf{Stichworte:} Brot aus Luft $\bullet$ Walöl $\bullet$ Haber-Bosch-Verfahren $\bullet$ 1. Weltkrieg $\bullet$ 40\% N im menschlichen Körper $\bullet$ Positivist}
%% --------------------------------------------------------------------



\ifcollection
\paragraph{Die atmende Wand und Brot aus Luft}
\fi



Als Kellerkind\footnote{\href{https://www.youtube.com/watch?v=54H0HAJexVI}{Tocotronic - Electric Guitar} als passende Untermalung für diese Aufgabe.} vom Dorf will Alex das Ausmaß der Radonbelastung in seinem Kellerzimmer bestimmen und lüften daher nicht. Passt schon. Spart dann auch Energie und lüften wird sowieso überschätzt. Während einer Messperiode von 7:00 Uhr bis 19:00 bestimmt er dreimal automatisch die Radonbelastung in seinem Kellerraum in $Bq/m^3$. Es ergibt sich folgende Abbildung\footnote{Die Quelle der Inspiration für die Aufgabe war folgender Artikel: \href{https://de.wikipedia.org/wiki/Atmende_Wand}{Atmende Wand}}. Leider helfen die Messwerte Alex überhaupt nicht weiter. Sie müssen also helfen! 

\begin{knitrout}
\definecolor{shadecolor}{rgb}{0.969, 0.969, 0.969}\color{fgcolor}

{\centering \includegraphics[width=\maxwidth]{img/math-10-1} 

}


\end{knitrout}

\vspace{-0.75cm}

\begin{enumerate}
\item Wie lange dauert es in Stunden bis Alex eine kritische Belastung von 380$Bq/m^3$ in dem ungelüfteten Kellerraum erreicht hat? \textbf{(2 Punkte)}
\end{enumerate}

Radon zerfällt mit einer Halbwertszeit von 2.8d zu Polonium. Polonium wiederum zerfällt mit einer Halbwertszeit von 160d zu Blei. Nur Radon und Polonium tragen zur radioaktiven Strahlenbelastung bei.

\begin{enumerate}
  \setcounter{enumi}{1}
\item Wie lange dauert es in Stunden bis die kritische Radonbelastung von
  380$Bq/m^3$ auf unter 90$Bq/m^3$ gefallen ist?
  \textbf{(4 Punkte)}
\end{enumerate}

Folgende Tabelle enthält die Informationen zur Zusammensetzung der normalen Umgebungsluft.

\begin{center}
  \begin{tabular}{ c|c|c|c }
     & Vol-\% & M [g/mol] & ppm \\
    \hline
    Stickstoff & 78.1 & 28.1 &
                                                    \phantom{1000000000000}\strut\\
        \hline
    Sauerstoff & 19.5 & 16.5 &
                                                    \phantom{10000000}\strut\\
        \hline
    Kohlenstoffdioxid & 0.045 & 12.1 & \phantom{10000000}\strut\\     
     \hline
\end{tabular}
\end{center}

\begin{enumerate}
   \setcounter{enumi}{2}
\item Rechnen Sie die Volumenprozente (Vol-\%) der Umgebungsluft in die entsprechenden ppm-Werte um und ergänzen Sie die berechneten ppm-Werte in die Tabelle!  \textbf{(1 Punkt)}
\end{enumerate}

Während Alex sein etwas pappiges Toastbrot mampfen kommt Alex die Dokumentation über Brot aus Luft in den Sinn. Alex denkt darüber ein wenig nach. Für die Umwandlung von Stickstoff $N_2$ mit Wasserstoff $H_2$ zu Ammoniak $NH_3$ gilt folgende Reaktionsgleichung\footnote{Die Quelle der Inspiration für die Aufgabe war folgender Artikel: \href{https://www.gdch.de/netzwerk-strukturen/fachstrukturen/ag-chemie-und-gesellschaft/projekte-und-veranstaltungen/cartoons/haber-bosch-verfahren-brot-aus-luft.html}{Haber-Bosch-Verfahren – Brot aus Luft}}:

\begin{equation*}
  N_2 + 3H_2 \rightarrow 2NH_3
\end{equation*}  

Ein Mol eines beliebigen Gases hat bei normalen Umweltbedingungen ein Volumen von 22.4 Liter. % Ein Mol enthält $6 \cdot 10^{23}$ Teilchen.

\begin{enumerate}
  \setcounter{enumi}{3}
\item Welche Masse an Ammoniak $2NH_3$ in Kilogramm $kg$ können Sie aus einem Kubikmeter $m^3$ Luft unter normalen Umweltbedingungen gewinnen? \textit{Beachten Sie die Reaktionsgleichung und gehen Sie von einem molaren Gewicht von Wasserstoff $H_2$ mit $2g/mol$ aus.}
  \textbf{(3 Punkte)}
\end{enumerate}

 
\clearpage
% -----------------------------------------------------------------------

\section{Aufgabe \hfill (10 Punkte)}

\textit{Geben Sie grundsätzlich Formeln und Rechenweg zur Lösung der Teilaufgaben mit an!} \\[1Ex]
 

 
%% --------------------------------------------------------------------
\ifcollection
\begin{flushright}
\tiny\vspace{-3Ex}
\textbf{\examinhaltstart}
\exammodulemathstat
\vspace{-4Ex}
\end{flushright}
\begin{minipage}[t]{0.5\textwidth}
\includegraphics[width = 1.3cm]{/Users/kruppajo/work/GitHub/exam/avatare/Jessica.png}\hspace{-4mm}\includegraphics[width = 1.3cm]{/Users/kruppajo/work/GitHub/exam/avatare/Jonas.png}\hspace{-4mm}\includegraphics[width = 1.3cm]{/Users/kruppajo/work/GitHub/exam/avatare/Mark.png}\hspace{-4mm}\includegraphics[width = 1.3cm]{/Users/kruppajo/work/GitHub/exam/avatare/Nilufar.png}
\end{minipage}
\begin{minipage}[t]{0.5\textwidth}
\hfill
\href{https://youtu.be/RuzMjwvwT-4}{\includegraphics[width = 2cm]{img/youtube}}
\end{minipage}
\fi
%% --------------------------------------------------------------------



\ifcollection
\paragraph{In der Kartonagenfabrik}
\fi



Nilufar, Jessica, Jonas und Mark sitzen im Bus. Wenn man sich zu spät anmeldet, dann ist die Exkursion nicht so toll. Nilufar hatte den Anderen in der Lerngruppe zu spät Bescheid gesagt. 'Was denn, bin ich eure Nanny oder was?!', entfährt es Nilufar nachdem die vorwurfsvollen Blicke schon eine Weile auf ihr lasten. Also geht es eben mit Rektor Skinner und Mrs. Krabappel in die Kartonagenfabrik. Wie schon im vorherigen Semester... In der Kartonagenfabrik angekommen erfahren die Vier, dass die Kartons zum Versand von Nägeln nicht hier zusammengebautwerden sondern das sich die Endfertigung in Flint, Michigan befindet. Unter anderem wird dort der berühmte \textit{Doppelt gewellte, 5-mal-gefaltete, 0.8mm, 50-cm-Karton} durch
Falzung hergestellt. Beim letzten Mal war Rektor Skinner die Stimmung zu schlecht und deshalb geht es erst nach Hause, wenn ein paar Aufgaben gelöst sind. Martin gefällt das. An dem Vorrat an Zigaretten von Mrs. Krabappel meinen alle wenig Zuversicht zu erkennen.\\

Jetzt heißt es Kartons optimieren, wenn Sie auch nochmal nach Hause wollen. Warum jetzt \textit{Sie} mit dabei sind, lassen wir mal weg. Der nun zu optimierende, flache Karton hat eine Länge von 50cm und eine Breite von 18cm. Die Kartonagenmaschine in Flint soll dann einen quadratischen Eckenausschnitt der Länge $x$ falzen.

\begin{enumerate}
\item Erstellen Sie eine Skizze des Karton\textit{blatt}rohlings! Beschriften Sie die Skizze mit den entsprechenden Längenangaben \textbf{(1 Punkt)}
\item Berechnen Sie die Falztiefe $x$ für ein maximales Volumen des flachen Kartons! \textbf{(3 Punkte)}
\item Welches Volumen in Liter ergibt sich mit der von Ihnen berechneten Falztiefe $x$?  \textbf{(1 Punkt)}
\item Sie wollen noch einen bündig mit dem Boden abschließenden Deckel für den Karton stanzen lassen. Wie groß ist die Fläche des Kartondeckel\textit{blatt}rohlings in $cm^2$? \textbf{(2 Punkte)}
\end{enumerate}

Rektor Skinner möchte sich gerne wieder in seinem Vorgarten aufhalten und nicht die ganze Zeit von Bart mit Erdnüssen beworfen werden. Deshalb möchte er einen geräumigen Teil seines Vorgartens einzäunen. Ein Teil der Umzäunung bildet seine Vorderhauswand. Wegen Lieferschwierigkeiten stehen Rektor Skinner nur 130m Zaun zu Verfügung. Auch hier sollen Sie mal helfen, sonst fährt der Bus Sie nicht nach Hause. Sie wollen nun die maximale Fläche des abgeschirmten Vorgartens in Abhängigkeit der Seitenlängen bei der Verwendung von 130m Zaun bestimmen!

\begin{enumerate}
  \setcounter{enumi}{4}  
\item  Welche Seitenlängen für den Zaun ergeben sich für die maximale Fläche des abgeschirmten Vorgartens? \textbf{(2 Punkte)}
\item Berechnen Sie die Fläche des abgeschirmten Vorgartens! \textbf{(1 Punkt)}
\end{enumerate}

 
\clearpage
% -----------------------------------------------------------------------
\end{document}
% -----------------------------------------------------------------------


  
