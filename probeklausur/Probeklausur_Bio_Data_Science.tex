\documentclass[a4paper, 9pt]{scrartcl}\usepackage[]{graphicx}\usepackage[]{xcolor}
% maxwidth is the original width if it is less than linewidth
% otherwise use linewidth (to make sure the graphics do not exceed the margin)
\makeatletter
\def\maxwidth{ %
  \ifdim\Gin@nat@width>\linewidth
    \linewidth
  \else
    \Gin@nat@width
  \fi
}
\makeatother

\definecolor{fgcolor}{rgb}{0.345, 0.345, 0.345}
\newcommand{\hlnum}[1]{\textcolor[rgb]{0.686,0.059,0.569}{#1}}%
\newcommand{\hlstr}[1]{\textcolor[rgb]{0.192,0.494,0.8}{#1}}%
\newcommand{\hlcom}[1]{\textcolor[rgb]{0.678,0.584,0.686}{\textit{#1}}}%
\newcommand{\hlopt}[1]{\textcolor[rgb]{0,0,0}{#1}}%
\newcommand{\hlstd}[1]{\textcolor[rgb]{0.345,0.345,0.345}{#1}}%
\newcommand{\hlkwa}[1]{\textcolor[rgb]{0.161,0.373,0.58}{\textbf{#1}}}%
\newcommand{\hlkwb}[1]{\textcolor[rgb]{0.69,0.353,0.396}{#1}}%
\newcommand{\hlkwc}[1]{\textcolor[rgb]{0.333,0.667,0.333}{#1}}%
\newcommand{\hlkwd}[1]{\textcolor[rgb]{0.737,0.353,0.396}{\textbf{#1}}}%
\let\hlipl\hlkwb

\usepackage{framed}
\makeatletter
\newenvironment{kframe}{%
 \def\at@end@of@kframe{}%
 \ifinner\ifhmode%
  \def\at@end@of@kframe{\end{minipage}}%
  \begin{minipage}{\columnwidth}%
 \fi\fi%
 \def\FrameCommand##1{\hskip\@totalleftmargin \hskip-\fboxsep
 \colorbox{shadecolor}{##1}\hskip-\fboxsep
     % There is no \\@totalrightmargin, so:
     \hskip-\linewidth \hskip-\@totalleftmargin \hskip\columnwidth}%
 \MakeFramed {\advance\hsize-\width
   \@totalleftmargin\z@ \linewidth\hsize
   \@setminipage}}%
 {\par\unskip\endMakeFramed%
 \at@end@of@kframe}
\makeatother

\definecolor{shadecolor}{rgb}{.97, .97, .97}
\definecolor{messagecolor}{rgb}{0, 0, 0}
\definecolor{warningcolor}{rgb}{1, 0, 1}
\definecolor{errorcolor}{rgb}{1, 0, 0}
\newenvironment{knitrout}{}{} % an empty environment to be redefined in TeX

\usepackage{alltt}
\usepackage[ngerman]{babel}
% -----------------------------------------------------------------------

% -----------------------------------------------------------------------
%% ------------------------------------------------------------
%% by J.Kruppa on Friday, February 11, 2022 (11:31)
%% \def\mainDir{\Sexpr{exam_path}}
\def\source{/Users/jokruppa/source/tex}
\usepackage[margin=2cm, includefoot]{geometry}
\setlength{\parindent}{0cm}
\usepackage{booktabs}
\usepackage{amsmath}
\usepackage{scalerel,amssymb}
\usepackage{setspace}
\def\csquare{{\Large $\boxtimes$}}
\def\msquare{{\Large $\square$}}
\usepackage[normalem]{ulem}
\usepackage{array}
\usepackage{xcolor}
\usepackage{float}
\usepackage{currfile}
\usepackage{tikz}
\usepackage[nomessages]{fp}

%% beamer defs
\def\lecture{Klausurfragen der Bio Data Science}

%% exam defs
\def\examtitle{\lecture}
\def\exammodule{
\vspace{-1.75cm}  
\begin{graybox}{}
\vspace{2Ex}
\textbf{\large Name:} \rule[0ex]{16.75em}{.4pt}
\hfill \textnormal{\textit{Nicht bestanden:}} \msquare \\[2.5Ex]
\textbf{\large Vorname:} \rule[0ex]{15em}{.4pt} \\[2.5Ex]
\textbf{\large Matrikelnummer:} \rule[0ex]{10.8em}{.4pt}
\hfill Endnote: \rule[0ex]{7em}{.4pt} 
\end{graybox}
\vspace{3Ex}
\phantom{text}
}
\def\examsemester{Sommersemester \& Wintersemester}
\def\examdate{\today}
%% ------------------------------------------------------------
\definecolor{darkblue}{rgb}{0,0,.5}
\definecolor{darkpurple}{rgb}{0.4117, 0.2, 0.4117}
\definecolor{uni}{rgb}{0,0.3137,0.6078}
\definecolor{gray}{gray}{0.7}

\usepackage{tcolorbox}
\definecolor{logo1}{RGB}{0, 158, 227}
\definecolor{gray5}{RGB}{247, 247, 247}
\definecolor{gray2}{RGB}{102, 102, 102}

\newtcolorbox{graybox}[1]{
  colback=gray5,%%red!5!white,
  colframe=gray2,%%red!75!black,
  fonttitle=\bfseries\Large,
  %%valign=center,
  fontupper=\large,
  before skip=10pt plus 2pt,
  after skip=20pt plus 4pt,
  title=#1}

\newtcolorbox{takehomebox}[1]{
  colback=gray5,%%red!5!white,
  colframe=logo1,%%red!75!black,
  fonttitle=\bfseries\Large,
  %%valign=center,
  fontupper=\large,
  before skip=10pt plus 2pt,
  after skip=10pt plus 2pt,
  title=#1}

\def\Rlogo{\includegraphics[width = 0.5cm]{\string~/Documents/GitHub/exam/img/Rlogo}\;}

\usepackage[scaled=.90]{helvet} 
\usepackage{fancyhdr}
\usepackage{lastpage}
\usepackage{hyperref}
\hypersetup{
    colorlinks=true,       % false: boxed links; true: colored links
    linkcolor=black,          % color of internal links 
    urlcolor=magenta           % color of external links
}
\renewcommand{\familydefault}{\sfdefault}

\title{
\large \exammodule \\[5Ex]
\Huge \examtitle \\[2Ex] 
\Large Hochschule Osnabr{\"u}ck
}
\author{Pr{\"u}fer: Prof. Dr. Jochen Kruppa \\
Fakult{\"a}t f{\"u}r Agrarwissenschaften und Landschaftsarchitektur \\ 
j.kruppa@hs-osnabrueck.de}
\date{Version vom \examdate}

%% ------------------------------------------------------------
%% by J.Kruppa on Tuesday, September 23, 2014 (12:50)
%% Header
\renewcommand{\headrulewidth}{0pt}
\renewcommand{\footrulewidth}{0pt}
\pagestyle{fancy}

\fancyhf{}
\fancyhead[L]{}
\fancyhead[R]{}
\fancyfoot[R]{\thepage}
\fancyfoot[L]{\footnotesize \examtitle}

\fancypagestyle{empty}{
 \fancyhf{}
 \fancyhead[L]{}
 \fancyhead[R]{}
 \fancyfoot[R]{\thepage}
 \fancyfoot[L]{\footnotesize \examtitle}
}

\usepackage{arevtext,arevmath}

\newcommand\Tstrut{\rule{0pt}{2.6ex}}         % = `top' strut
\newcommand\Bstrut{\rule[-0.9ex]{0pt}{0pt}}   % = `bottom' strut
\def\strut{\Tstrut\Bstrut}

% -----------------------------------------------------------------------
\IfFileExists{upquote.sty}{\usepackage{upquote}}{}
\begin{document}
% -----------------------------------------------------------------------
\maketitle
\thispagestyle{empty}
\clearpage
% -----------------------------------------------------------------------

\begin{graybox}{Erlaubte Hilfsmittel f{\"u}r die Klausur}
  \vspace{1Ex}
  \begin{itemize}
  \item Normaler Taschenrechner ohne M{\"o}glichkeit der Kommunikation mit anderen
    Ger{\"a}ten - also ausdr{\"u}cklich kein Handy!
  \item Eine DIN A4-Seite als beidseitig, selbstgeschriebene,
    handschriftliche Formelsammlung - keine digitalen Ausdrucke. 
  \item \textbf{You can answer the questions in English without any consequences.}  
  \end{itemize}
\end{graybox}
\vfill

\begin{graybox}{Ergebnis der Klausur}
  \vspace{1Ex}
  \begin{itemize}
  \item[] \rule[0ex]{3em}{.4pt}\, von 20\, Punkten sind aus dem Multiple
    Choice Teil erreicht.
  \item[] \rule[0ex]{3em}{.4pt}\, von 69 Punkten sind aus dem Rechen- und
    Textteil erreicht. 
  \item[] \rule[0ex]{3em}{.4pt}\, von 89 Punkten in Summe.
  \item[] Es wird folgender Notenschl{\"u}ssel angewendet.   
  \end{itemize}
  \vspace{1ex}
\begin{center}
  \begin{tabular}[c]{cc}
    \toprule
    \textbf{Punkte}	&	\textbf{Note}	\\
    \midrule
    85.0 - 89.0	&	1,0	\\
    80.5 - 84.5	&	1,3	\\
    76.5 - 80.0	&	1,7	\\
    72.0 - 76.0	&	2,0	\\
    67.5 - 71.5	&	2,3	\\
    63.0 - 67.0	&	2,7	\\
    58.5 - 62.5	&	3,0	\\
    54.5 - 58.0	&	3,3	\\
    50.0 - 54.0	&	3,7	\\
    44.5 - 49.5	&	4,0	\\
    \bottomrule
  \end{tabular}
\end{center}
  \vspace{1ex}
\begin{itemize}
\item[] Es ergibt sich eine Endnote von \rule[0ex]{4em}{.4pt}.
\end{itemize}
  \vspace{1Ex}
\end{graybox}

% -----------------------------------------------------------------------
\newpage
% -----------------------------------------------------------------------

\begin{graybox}{Multiple Choice Aufgaben}
  \begin{itemize}
  \item Pro Multipe Choice Frage ist \emph{genau} eine Antwort richtig.
  \item \textbf{Übertragen Sie Ihre Kreuze in die Tabelle auf
      dieser Seite.}
  \item Es werden nur Antworten berücksichtigt, die in dieser Tabelle
    angekreuzt sind!
  \end{itemize}

\begin{center}
  \large
  \begin{tabular}{|r|c|c|c|c|c||c|}
    \hline
    & \textbf{A} & \textbf{B} & \textbf{C} & \textbf{D} & \textbf{E} & $\boldsymbol{\checkmark}$\strut\\
    \hline
    1 Aufgabe &   &   &   &   &   & \strut\\
    \hline
    2 Aufgabe &   &   &   &   &   & \strut\\
    \hline
    3 Aufgabe &   &   &   &   &   & \strut\\
    \hline
    4 Aufgabe &   &   &   &   &   & \strut\\
    \hline
    5 Aufgabe &   &   &   &   &   & \strut\\
    \hline
    6 Aufgabe &   &   &   &   &   & \strut\\
    \hline
    7 Aufgabe &   &   &   &   &   & \strut\\
    \hline
    8 Aufgabe &   &   &   &   &   & \strut\\
    \hline
    9 Aufgabe &   &   &   &   &   & \strut\\
    \hline
    10 Aufgabe &   &   &   &   &   & \strut\\
    \hline
  \end{tabular}
\end{center}

\begin{itemize}
\item Es sind \rule[0ex]{2em}{.4pt}\, von 20 Punkten erreicht worden.
\end{itemize}
\end{graybox}

\vfill

\begin{graybox}{Rechen- und Textaufgaben}
  \begin{itemize}
  \item Die Tabelle wird vom Dozenten ausgefüllt.
  \end{itemize}
  \begin{center}
    \large
    \begin{tabular}{|l|c|c|c|c|c|c|c|}
      \hline
      \textbf{Aufgabe} & 11 & 12 & 13 & 14 & 15 & 16 & 17 \strut\\
      \hline
      \textbf{Punkte} & 
      \hspace{1Ex}\Large\textcolor{gray!70}{9}\hspace{1Ex}  & 
      \hspace{1Ex}\Large\textcolor{gray!70}{9}\hspace{1Ex}  & 
      \hspace{1Ex}\Large\textcolor{gray!70}{8}\hspace{1Ex}  & 
      \hspace{1Ex}\Large\textcolor{gray!70}{12}\hspace{1Ex}  & 
      \hspace{1Ex}\Large\textcolor{gray!70}{11}\hspace{1Ex}  & 
      \hspace{1Ex}\Large\textcolor{gray!70}{9}\hspace{1Ex}  & 
      \hspace{1Ex}\Large\textcolor{gray!70}{11}\hspace{1Ex} \strut\\
      \hline
  \end{tabular}
\end{center}
\begin{itemize}
\item Es sind \rule[0ex]{2em}{.4pt}\, von 69 Punkten erreicht worden.
\end{itemize}
\end{graybox}

% -----------------------------------------------------------------------
\clearpage
% -----------------------------------------------------------------------


\section{Aufgabe \hfill (2 Punkte)}

%% --------------------------------------------------------------------
\ifcollection
\begin{flushright}
\tiny\vspace{-2Ex}
\textbf{\examinhaltstart}
\exammodulemathstat $\;\bullet$
\exammodulestat $\;\bullet$
\exammodulestatbbv $\;\bullet$
\exammodulestatversuch $\;\bullet$
\exammodulebiostat
\vspace{-1Ex}
\end{flushright}
\fi
%% --------------------------------------------------------------------




Welche Aussage zum mathematische Ausdruck $Pr(D|H_0)$ ist richtig?



\begin{enumerate}
\item [\textbf{A} \msquare] $Pr(D|H_0)$ ist die Wahrscheinlichkeit der Alternativehypothese und somit $1 - Pr(H_A)$
\item [\textbf{B} \msquare] $Pr(D|H_0)$ stellt die Wahrscheinlichkeit die Teststatistik $T$ zu beobachten dar, wenn die Nullhypothese falsch ist.
\item [\textbf{C} \msquare] $Pr(D|H_0)$ ist die Wahrscheinlichkeit nicht die Daten $D$ zu beobachten sondern die Nullhypothese, wenn diese wahr ist.
\item [\textbf{D} \msquare] Die Inverse der Wahrscheinlichkeit unter der die Nullhypothese nicht mehr die Alternativehypothese überdeckt.
\item [\textbf{E} \msquare] $Pr(D|H_0)$ ist die Wahrscheinlichkeit die Daten $D$ zu beobachten, wenn die Nullhypothese wahr ist.
\end{enumerate} 

\section{Aufgabe \hfill (2 Punkte)}

%% --------------------------------------------------------------------
\ifcollection
\begin{flushright}
\tiny\vspace{-2Ex}
\textbf{\examinhaltstart}
\exammodulemathstat $\;\bullet$
\exammodulestat $\;\bullet$
\exammodulestatbbv $\;\bullet$
\exammodulestatversuch $\;\bullet$
\exammodulebiostat
\vspace{-1Ex}
\end{flushright}
\fi
%% --------------------------------------------------------------------




Die einfaktorielle ANOVA ist ein Standardverfahren in der agrawissenschaftlichen Forschung wenn es um den Vergleich von Behandlungsgruppen geht. Welche der folgenden Aussage zu der Berechnung der Teststatistik der einfaktoriellen ANOVA ist richtig?



\begin{enumerate}
\item [\textbf{A} \msquare] Die ANOVA berechnet die T-Statistik indem den Mittelwertsunterschied der Gruppen simultan durch die Standardabweichung der Gruppen teilt. Wenn die T-Statistik h{"o}her als 1.96 ist, kann die Nullhypothese abgelehnt werden.
\item [\textbf{B} \msquare] Die F-Statistik wird berechnet indem die MS der Behandlung durch die MS des Fehlers geteilt werden. Wenn die F-Statistik sich kaum von der Null unterscheidet kann die Nullhypothese nicht abgelehnt werden.
\item [\textbf{C} \msquare] Die ANOVA berechnt die F-Statistik aus den SS Behandlung geteilt durch die SS Fehler.
\item [\textbf{D} \msquare] Die ANOVA berechnet die F-Statistik indem die MS des Fehlers durch die MS der Behandlung geteilt werden. Wenn die F-Statistik sich der 0 ann{"a}hert kann die Nullhypothese abgelehnt werden.
\item [\textbf{E} \msquare] Die ANOVA berechnet die T-Statistik aus der Multiplikation der MS Behandlung mit der MS der Fehler. Wenn die F-Statistik genau 0 ist, kann die Nullhypothese nicht abgelehnt werden.
\end{enumerate} 

\section{Aufgabe \hfill (2 Punkte)}

%% --------------------------------------------------------------------
\ifcollection
\begin{flushright}
\tiny\vspace{-2Ex}
\textbf{\examinhaltstart}
\exammodulemathstat $\;\bullet$
\exammodulestat $\;\bullet$
\exammodulestatbbv $\;\bullet$
\exammodulestatversuch $\;\bullet$
\exammodulebiostat
\vspace{-1Ex}
\end{flushright}
\fi
%% --------------------------------------------------------------------






Aus einem Feldversuch ergibt sich die Notwendigkeit der Berechnung einer einfaktoriellen ANOVA. Es ergibt sich ein $\eta^2 = 0.12$. Welche Aussage ist richtig?



\begin{enumerate}
\item [\textbf{A} \msquare] Das $\eta^2$ beschreibt den Anteil der globalen Varianz, der von den Umweltbedingungen erklärt wird.
\item [\textbf{B} \msquare] Das $\eta^2$ ist die Korrelation der ANOVA. Mit der Ausnahme, dass $\eta^2 = 0$ der beste Wert ist.
\item [\textbf{C} \msquare] Die Berechnung von $\eta^2$ ist ein Wert für die Interaktion in der einfaktoriellen ANOVA.
\item [\textbf{D} \msquare] Der Anteil der Varianz, der von den Behandlungsbedingungen erklärt wird, wird durch das $\eta^2$ beschrieben.
\item [\textbf{E} \msquare] Das $\eta^2$ beschreibt den Anteil der Varianz, der von den Behandlungsbedingungen nicht erklärt wird. Somit der Rest an nicht erklärbarer Varianz.
\end{enumerate} 

\section{Aufgabe \hfill (2 Punkte)}

%% --------------------------------------------------------------------
\ifcollection
\begin{flushright}
\tiny\vspace{-2Ex}
\textbf{\examinhaltstart}
\exammodulestatversuch $\;\bullet$
\exammodulebiostat
\vspace{-1Ex}
\end{flushright}
\fi
%% --------------------------------------------------------------------




Die Abkürzung \textit{CLD} steht für welches statistische Verfahren? Welche folgende Beschreibung der Interpretation ist korrekt?



\begin{enumerate}
\item [\textbf{A} \msquare] Compact letter display. Das CLD ist umstritten, da es die Gleichheit der Behandlungen durch gleiche Buchstaben darstellt. Dadurch ist das CLD nicht mehr sauber auf einer Linie mit dem statistischen Testen. Wir lehnen die Nullhypothese ab und zeigen keine Gleichheit im statistischen Testen.
\item [\textbf{B} \msquare] Compact letter detection. Gleichheit in den Behandlungen wird durch den gleichen Buchstaben oder Symbol dargestellt.
\item [\textbf{C} \msquare] Compound letter display. Gleichheit in dem Outcomes wird durch den gleichen Buchstaben oder Symbol dargestellt. Teilweise ist die Interpretation des Verbunds (eng. compound) herausfordernd, da wir ja nach dem Unterschied suchen.
\item [\textbf{D} \msquare] Compact letter display. Gleiche Buchstaben bedeuten, dass sich die Behandlungen unterscheiden. Daher ist das CLD sehr unintuitiv. Es wäre besser, wenn gleiche Buchstaben Gleichheit anzeigen würden. Dies ist aber leider in der statistischen Testtheorie nicht möglich.
\item [\textbf{E} \msquare] Compact letter display. Gleiche Buchstaben zeigen Gleichheit in den Behandlungen. Die Interpretation ist deshalb sehr intuitiv und einfach. Darüber hinaus ist damit das CLD auch auf einer Linie mit der Testtheorie, da wir ja auch dort die Gültigkeit der Nullhypothese nachweisen. Wir suchen ja Gleichheit.
\end{enumerate} 

\section{Aufgabe \hfill (2 Punkte)}

%% --------------------------------------------------------------------
\ifcollection
\begin{flushright}
\tiny\vspace{-2Ex}
\textbf{\examinhaltstart}
\exammodulemathstat $\;\bullet$
\exammodulestat $\;\bullet$
\exammodulestatbbv 
\vspace{-1Ex}
\end{flushright}
\fi
%% --------------------------------------------------------------------




In Ihrer Abschlussarbeit rechnen Sie einen Student t-Test. Welche Aussage ist auch für den Welch t-Test richtig?



\begin{enumerate}
\item [\textbf{A} \msquare] Der t-Test vergleicht zwei oder mehr Gruppen indem die Mittelwerte miteinander verglichen werden.
\item [\textbf{B} \msquare] Der t-Test vergleicht die Varianzen von mindestens zwei oder mehr Gruppen
\item [\textbf{C} \msquare] Der t-Test vergleicht die Mittelwerte von zwei Gruppen unter der strikten Annahme von Varianzhomogenität. Sollte keine Varianzhomogenität vorliegen, so gibt es keine Möglichkeit den t-Test in einer Variante anzuwenden.
\item [\textbf{D} \msquare] Der t-Test berechnet die Differenz von zwei Mittelwerten als Effekt und gibt eine Entscheidung, ob sich die beiden Mittelwerte \textit{jeweils} von Null unterscheiden.
\item [\textbf{E} \msquare] Der t-Test vergleicht zwei Gruppen indem die Mittelwerte miteinander verglichen werden.
\end{enumerate} 

\section{Aufgabe \hfill (2 Punkte)}

%% --------------------------------------------------------------------
\ifcollection
\begin{flushright}
\tiny\vspace{-2Ex}
\textbf{\examinhaltstart}
\exammodulemathstat $\;\bullet$
\exammodulestat $\;\bullet$
\exammodulestatbbv 
\vspace{-1Ex}
\end{flushright}
\fi
%% --------------------------------------------------------------------




Die \uline{Standardabweichung} ist eine bedeutende deskriptive Statistik für die Analyse von Daten. Wie müssen Sie vorgehen um die Standardabweichung zu berechnen?



\begin{enumerate}
\item [\textbf{A} \msquare] Den Mittelwert berechnen und die Abstände quadrieren. Die Summe mit der Fallzahl $(n-1)$ multiplizieren.
\item [\textbf{B} \msquare] Als erstes berechnen wir den Mittelwert. Dann bilden wir die Summe der quadratischen Abstände zu dem Mittelwert. Abschließend subtrahieren wir die Fallzahl $(n-1)$.
\item [\textbf{C} \msquare] Als erstes berechnen wir den Mittelwert. Dann bilden wir die Summe der quadratischen Abstände zu dem Mittelwert. Abschließend teilen wir durch die Fallzahl $(n-1)$. Nicht zu vergessen, am Ende dann noch die Wurzel zu ziehen.
\item [\textbf{D} \msquare] Den Median berechen, dann die quadratischen Abstände zum Median aufsummieren, dann die Wurzel ziehen. Am Ende durch die Fallzahl ($n-1$) teilen
\item [\textbf{E} \msquare] Den Mittelwert berechen, dann die quadratischen Abstände zum Mittelwert aufsummieren und durch die Fallzahl $(n-1)$ teilen.
\end{enumerate} 

\section{Aufgabe \hfill (2 Punkte)}

%% --------------------------------------------------------------------
\ifcollection
\begin{flushright}
\tiny\vspace{-2Ex}
\textbf{\examinhaltstart}
\exammodulemathstat $\;\bullet$
\exammodulestat $\;\bullet$
\exammodulestatbbv 
\vspace{-1Ex}
\end{flushright}
\fi
%% --------------------------------------------------------------------




Berechnen Sie den Median, das $1^{st}$ Quartile sowie das $3^{rd}$ Quartile von $y$ mit 20, 10, 12, 3, 27, 26, 16, 33, 5, 23 und 51.




\begin{enumerate}
\item [\textbf{A} \msquare] Es berechnet sich 21 [11; 28]
\item [\textbf{B} \msquare] Sie erhalten 20 [8; 25]
\item [\textbf{C} \msquare] Es ergibt sich 20 +/- 10
\item [\textbf{D} \msquare] Es ergibt sich 20 [10; 27]
\item [\textbf{E} \msquare] Es berechnet sich 21 [11; 26]
\end{enumerate} 

\section{Aufgabe \hfill (2 Punkte)}

%% --------------------------------------------------------------------
\ifcollection
\begin{flushright}
\tiny\vspace{-2Ex}
\textbf{\examinhaltstart}
\exammodulemathstat $\;\bullet$
\exammodulestat $\;\bullet$
\exammodulestat $\;\bullet$
\exammodulestatbbv $\;\bullet$
\exammodulestatversuch $\;\bullet$
\exammodulebiostat
\vspace{-1Ex}
\end{flushright}
\fi
%% --------------------------------------------------------------------




Die folgende Abbildung enthält die Daten aus einer Studie zur Bewertung der Wirkung des Mikronährstoff Sulfit auf den Ertrag in t/ha von Mango im Vergleich zu einer Kontrolle. Der Versuch wurde in 6 Parzellen pro Gruppe durchgeführt. Welche Aussage im Bezug auf eine statistische Auswertung ist richtig?



{\centering \includegraphics[width=\maxwidth]{img/mc-testing-ttest-02-1} 

}







\begin{enumerate}
\item [\textbf{A} \msquare] Die Barplots deuten auf keinen signifikanten Unterschied. Der Effekt liegt vermutlich bei 5 unter einer groben Abschätzung. Wir müssen aber eine ANOVA rechnen um den Effekt wirklich bestimmen zu können.
\item [\textbf{B} \msquare] Es liegt ein signifikanter Unterschied vor. Der Effekt liegt bei 5.
\item [\textbf{C} \msquare] Nach Betrachtung des Barplots liegt kein signifikanter Unterschied vor. Der Effekt liegt bei 5.
\item [\textbf{D} \msquare] Die Barplots deuten auf einen signifikanten Unterschied. Der Effekt liegt vermutlich bei 5. Wir müssen aber einen Posthoc-Test rechnen um den Effekt wirklich bestimmen zu können.
\item [\textbf{E} \msquare] Es liegt ein signifikanter Unterschied vor. Der Effekt liegt bei 0.5.
\end{enumerate}

\section{Aufgabe \hfill (2 Punkte)}

%% --------------------------------------------------------------------
\ifcollection
\begin{flushright}
\tiny\vspace{-2Ex}
\textbf{\examinhaltstart}
\exammodulemathstat $\;\bullet$
\exammodulestat $\;\bullet$
\exammodulestatbbv $\;\bullet$
\exammodulestatversuch $\;\bullet$
\exammodulebiostat
\vspace{-1Ex}
\end{flushright}
\fi
%% --------------------------------------------------------------------




Ein statistischer Test produziert für einen Gruppenvergleich einen $p$-Wert. Welche Aussage zusammen mit dem Signifikanzniveau $\alpha$ gleich 5\% stimmt?



\begin{enumerate}
\item [\textbf{A} \msquare] Wir vergleichen mit dem $p$-Wert und dem Signifikanzniveau $\alpha$ absolute Werte auf einem Zahlenstrahl und damit den Unterschied der Teststatistiken, wenn die $H_0$ gilt.
\item [\textbf{B} \msquare] Wir machen ein Aussage über die Flächen und der Kurve der Teststatistik, wenn die $H_0$ gilt. Dabei werden Wahrscheinlichkeiten vergleichen, die durch die Flächen unter der Kurve repräsentiert werden.
\item [\textbf{C} \msquare] Wir vergleichen die Effekte des $p$-Wertes mit den Effekten der Signifikanzschwelle unter der Annahme der Nullhypothese. Dabei gilt, dass wir die Nullhypothese nur ablehnen können anhand des Falsifikationsprinzips.
\item [\textbf{D} \msquare] Wir vergleichen mit dem $p$-Wert und dem Signifikanzniveau $\alpha$ Wahrscheinlichkeiten und damit die absoluten Werte auf einem Zahlenstrahl, wenn die $H_0$ gilt.
\item [\textbf{E} \msquare] Wir schauen, ob der $p$-Wert größer ist als das Signifikanzniveau $\alpha$ und vergleichen somit Wahrscheinlichkeiten. Die Wahrscheinlichkeiten werden als Flächen unter der Kurve der Teststaistik dargestellt, wenn die $H_A$ gilt.
\end{enumerate} 

\section{Aufgabe \hfill (2 Punkte)}

%% --------------------------------------------------------------------
\ifcollection
\begin{flushright}
\tiny\vspace{-2Ex}
\textbf{\examinhaltstart}
\exammodulemathstat $\;\bullet$
\exammodulestat $\;\bullet$
\exammodulestatbbv 
\vspace{-1Ex}
\end{flushright}
\fi
%% --------------------------------------------------------------------




Berechnen Sie den Mittelwert und Standardabweichung von $y$ mit 12, 14, 19, 15 und 19.



\begin{enumerate}
\item [\textbf{A} \msquare] Es ergibt sich 14.8 +/- 4.85
\item [\textbf{B} \msquare] Sie erhalten 15.8 +/- 1.76
\item [\textbf{C} \msquare] Sie erhalten 15.8 +/- 1.555
\item [\textbf{D} \msquare] Es berechnet sich 15.8 +/- 3.11
\item [\textbf{E} \msquare] Es berechnet sich 16.8 +/- 9.7
\end{enumerate}

% -----------------------------------------------------------------------
\clearpage
% -----------------------------------------------------------------------

\section{Aufgabe \hfill (9 Punkte)}

\textit{Geben Sie grundsätzlich Formeln und Rechenweg zur Lösung der Teilaufgaben mit an!} \\[1Ex]
 

 
%% --------------------------------------------------------------------
\ifcollection
\begin{flushright}
\tiny\vspace{-3Ex}
\textbf{\examinhaltstart}
\exammodulemathstat $\;\bullet$
\exammodulestat $\;\bullet$
\exammodulestatbbv 
\vspace{-4Ex}
\end{flushright}
\begin{minipage}[t]{0.5\textwidth}
\includegraphics[width = 1.3cm]{/Users/kruppajo/work/GitHub/exam/avatare/Alex.png}\hspace{-4mm}\includegraphics[width = 1.3cm]{/Users/kruppajo/work/GitHub/exam/avatare/Yuki.png}
\end{minipage}
\begin{minipage}[t]{0.5\textwidth}
\hfill
\href{https://youtu.be/IhecxMcCENY}{\includegraphics[width = 2cm]{img/youtube}}
\end{minipage}
\fi
%% --------------------------------------------------------------------



\ifcollection
\paragraph{Ergebnistabelle der einfaktoriellen ANOVA}
\fi

Yuki und Alex schauen sich etwas entnervt an. Gemeinsam schreiben die beiden ihre Abschlussarbeit und sollen nun als erstes einmal die Daten mit eine einfaktoriellen ANOVA auswerten damit abgeschätzt werden kann, ob überhaupt signifikante Ergebnisse in den multipen Gruppenvergleichen zu erwarten sind. Da hilft die Katze von Alex auch nur bedingt. Die beiden waren im Emsland um ein Gewächshausexperiment mit Brokkoli durchzuführen. Dabei haben Yuki und Alex den Messwert Trockengewicht [kg/ha] unter der Behandung Düngestufen ($ctrl$, $low$, $mid$ und $high$) ermittelt. Nachher wollen sich beide noch mit dem Hobby Starcraft von Alex beschäftigen. Kennt Yuki noch nicht, klingt aber interessant.



\vspace{1ex}

Leider kennen sich Yuki und Alex mit Berechnung einer einfaktoriellen ANOVA überhaupt nicht aus. Deshalb brauchen beide bei der Erstellung Ihre Hilfe, die Katze reicht als Hilfe nicht aus! 

\begin{enumerate}
  \item Formulieren Sie die wissenschaftliche Fragestellung! \textbf{(1 Punkt)}
  \item Formulieren Sie das statistische Hypothesenpaar! \textbf{(1 Punkt)}
\item Füllen Sie die unterstehende einfaktorielle ANOVA Ergebnistabelle aus! \textbf{(3 Punkte)}
\end{enumerate}

\vspace{1Ex}

\begin{center}
  \Large
  \begin{tabular}{lccccp{3cm}}
\toprule
     & \textbf{Df} & \textbf{Sum Sq} & \textbf{Mean Sq} & \textbf{F value} & \textbf{Pr(>F)} \strut\\
    \midrule
   \textbf{Düngestufen}  & 3 &  &  &  &  \strut\\
   \textbf{error}  & 23 & 897.01 &  &  &  \strut\\
   \textbf{Total}  & 26 & 6063.19 &  &  &  \strut\\
\bottomrule
  \end{tabular}
\end{center}

\vspace{1Ex}

\begin{enumerate}
  \setcounter{enumi}{3}
\item Schätzen Sie den p-Wert der Tabelle mit $F_{\alpha = 5\%} = 3.03$ ab. Begründen Sie Ihre Antwort! \textbf{(2 Punkte)}
\item Berechen Sie den Effektschätzer $\eta^2$. Was sagt Ihnen der Wert von $\eta^2$ aus? \textbf{(2 Punkte)}
\end{enumerate}



 
\clearpage
% -----------------------------------------------------------------------

\section{Aufgabe \hfill (9 Punkte)}


 
%% --------------------------------------------------------------------
\ifcollection
\begin{flushright}
\tiny
\textbf{\examinhaltstart}
\exammodulestat $\;\bullet$
\exammodulestatbbv $\;\bullet$
\exammodulestatversuch $\;\bullet$
\exammodulebiostat
\vspace{-4Ex}
\end{flushright}
\begin{minipage}[t]{0.5\textwidth}
\includegraphics[width = 1.3cm]{/Users/kruppajo/work/GitHub/exam/avatare/Alex.png}\hspace{-4mm}\includegraphics[width = 1.3cm]{/Users/kruppajo/work/GitHub/exam/avatare/Mark.png}
\end{minipage}
\begin{minipage}[t]{0.5\textwidth}
\hfill
\href{https://youtu.be/aHVYuFKTqZs}{\includegraphics[width = 2cm]{img/youtube}}
\end{minipage}
\fi
%% --------------------------------------------------------------------



\ifcollection
\paragraph{Grundgesamtheit und experimentelle Stichprobe}
\fi

An einem schwülem Sommernachmittag sitzen Alex und Mark in einem Eiskaffee und wollen sich auf die Klausur vorbereiten. In fast allen Fragen geht es ja um die Interpretation eines statistischen Tests. Daher wollen die beiden jetzt nochmal nacharbeiten, was die Grundlagen der Stichprobe (eng. \textit{sample}) und der Grundgesamtheit (eng. \textit{population} oder \textit{ground truth}) sind. Alex hat sich Gummibärchen Eisbecher bestellt und Mark bleibt lieber bei einem Marzipankugeln Eis. 'Irre, was die Lebensmittelindustrie alles auf die Beine kriegt', merk Mark an und Alex schüttelt anerkennend den Kopf.

\vspace{1ex}

Leider kennen sich Alex und Mark mit der Grundgesamtheit und der Stuchprobe überhaupt nicht aus. Daher sind Sie gefragt!

\begin{enumerate}
\item Erklären Sie den Zusammenhang zwischen Stichprobe und Grundgesamtheit an einem Schaubild! Beschriften Sie das Schaubild entsprechend! \textit{Nutzen Sie hierfür als Veranschaulichung ein aussagekräftiges Beispiel!}  \textbf{(3 Punkte)}
\begin{enumerate}
\item Nennen Sie das statistische Verfahren um von einer Grundgesamtheit auf eine Stichprobe zu gelangen!  \textbf{(1 Punkt)}
\item Nennen Sie ein konkretes Beispiel zur Durchführung um von einer Grundgesamtheit auf eine Stichprobe zu gelangen! \textbf{(1 Punkt)}
\item Benennen Sie die Eigenschaft, die zwischen Grundgesamtheit und Stichprobe vorliegen muss! \textbf{(1 Punkt)}
\end{enumerate}
\item Erweitern Sie das Schaubild um die Entstehung von $Pr(D|H_0)$! \textit{Nutzen Sie hierfür als Veranschaulichung ein aussagekräftiges Beispiel!}  \textbf{(3 Punkte)}
\end{enumerate}

 
\clearpage
% -----------------------------------------------------------------------

\section{Aufgabe \hfill (8 Punkte)}

\textit{Geben Sie grundsätzlich Formeln und Rechenweg zur Lösung der Teilaufgaben mit an!} \\[1Ex]
 

 
%% --------------------------------------------------------------------
\ifcollection
\begin{flushright}
\tiny\vspace{-3Ex}
\textbf{\examinhaltstart}
\exammodulemathstat $\;\bullet$
\exammodulestat $\;\bullet$
\exammodulestatbbv $\;\bullet$
\exammodulestatversuch\\
\exammodulelanddaten $\;\bullet$
\exammodulebiostat
\vspace{-4Ex}
\end{flushright}
\begin{minipage}[t]{0.5\textwidth}
\includegraphics[width = 1.3cm]{/Users/kruppajo/work/GitHub/exam/avatare/Jessica.png}
\end{minipage}
\begin{minipage}[t]{0.5\textwidth}
\hfill
\href{https://youtu.be/t0WYa_LVc5o}{\includegraphics[width = 2cm]{img/youtube}}
\end{minipage}
\vspace{-3ex}
\fi
%% --------------------------------------------------------------------



\ifcollection
\paragraph{Zerforschen des Barplots}
\fi

Jessica und der Mangel, eine unendliche Geschichte mit kniffeligen Wendungen. Deshalb gilt anschauen, was andere vor einem gemacht haben. Für Jessica ist es eine Möglichkeit schneller ans Ziel zu gelangen. Jessica soll in ihrer Hausarbeit Kartoffeln untersuchen. Die Behandlung in ihrer Hausarbeit werden verschiedene Lichtstufen ($none$, $200lm$ und $600lm$) sein. Erheben wird Jessica als Messwert ($Y$) \textit{Proteingehalt} benannt als \textit{protein} in ihrer Exceldatei. Von ihrer Betreuer erhält sie nun folgende Abbildung von Barplots, die sie erstmal zur Übung nachbauen soll, bevor sie mit dem eigentlichen Versuch beginnt.



{\centering \includegraphics[width=\maxwidth]{img/barplot-02-1} 

}




Leider kennt sich Jessica mit der Erstellung von Barplots in \Rlogo nicht aus. Deshalb braucht sie bei der Visualisierung Ihre Hilfe!

\begin{enumerate}
\item Formulieren Sie die wissenschaftliche Fragestellung! \textbf{(1 Punkt)}
\item Erstellen Sie eine Tabelle mit den statistischen Maßzahlen der drei Barplots! \textit{Beachten Sie die korrekte Darstellungsform der statistischen Maßzahlen!} \textbf{(3 Punkte)}
\item Erstellen Sie einen beispielhaften Datensatz im \Rlogo üblichen Format, aus dem die drei Barplots \textit{möglicherweise} erstellt wurden! \textbf{(2 Punkte)}
\item Kann Jessica einen Unterschied zwischen den Behandlungen erwarten? Begründen Sie Ihre Antwort! \textbf{(2 Punkte)}
\end{enumerate} 
\clearpage
% -----------------------------------------------------------------------

\section{Aufgabe \hfill (9 Punkte)}

\textit{Geben Sie grundsätzlich Formeln und Rechenweg zur Lösung der Teilaufgaben mit an!} \\[1Ex]
 

 
%% --------------------------------------------------------------------
\ifcollection
\begin{flushright}
\tiny\vspace{-3Ex}
\textbf{\examinhaltstart}
\exammodulemathstat $\;\bullet$
\exammodulestat $\;\bullet$
\exammodulestatbbv 
\vspace{-4Ex}
\end{flushright}
\begin{minipage}[t]{0.5\textwidth}
\includegraphics[width = 1.3cm]{/Users/kruppajo/work/GitHub/exam/avatare/Alex.png}
\end{minipage}
\begin{minipage}[t]{0.5\textwidth}
\hfill
\href{https://youtu.be/zgpw9GC0plk}{\includegraphics[width = 2cm]{img/youtube}}
\end{minipage}
\vspace{-3ex}
\fi
%% --------------------------------------------------------------------



\ifcollection
\paragraph{Berechnung des Student t-Test \underline{oder} Welch t-Test}
\fi

Der t-Test. Alex erschaudert. Eine echte Herausforderung für ihn war schon immer die Gefälligkeit gewesen. Ein leidiges Lied. Ein mächtiges Werkzeug ist der t-Test in den Händen desjenigen, der einen normalverteilten Messwert ($Y$) hat. Aber erstmal überhaupt den t-Test rechnen können. Wie sah das Experiment von Alex überhaupt aus? Alex schmeißt noch eine Handvoll Gummibärchen in seinen Rachen. Im Hintergrund klirrt leise der Spiegel zum Sound von Abba. Alex hat einen Leistungssteigerungsversuch mit Lamas durchgeführt um eine neue technische Versuchsanlage zu testen. Bei dem Pilotexperiment mit sehr geringer Fallzahl $(n_1 = n_2 = 3)$ wurde die Behandlung Ernährungszusatz ($ctrl$ und $fedX$) an den Lamas getestet und dabei wurde geschaut, ob der Versuch überhaupt technisch klappen könnte. Gemessen hat Alex dann als Messwert Gewichtszuwachs in der 1LW [\%/kg]. Warum der Versuch im Oldenburger Land für seinen Projektbericht stattfinden musste, ist ihm bis heute ein Rätsel. Egal. Gibt es jetzt einen Zusammenhang zwischen der Behandlung und Gewichtszuwachs in der 1LW [\%/kg]?

\begin{table}[!h]
\centering
\begin{tabular}{cc}
\toprule
Ernährungszusatz & Gewichtszuwachs\\
\midrule
ctrl & 13.0\\
fedX & 25.0\\
ctrl & 11.7\\
ctrl & 15.4\\
fedX & 23.0\\
\addlinespace
fedX & 21.5\\
\bottomrule
\end{tabular}
\end{table}



Leider kennt sich Alex mit der Berechnung eines Student t-Tests überhaupt nicht aus. Deshalb braucht er bei der Berechnung Ihre Hilfe!

\begin{enumerate}
  \item Formulieren Sie die wissenschaftliche Fragestellung! \textbf{(1 Punkt)}
  \item Bestimmen Sie die Teststatistik $T_{D}$ eines Student t-Tests! \textbf{(3 Punkte)}
  \item Treffen Sie mit $T_{\alpha = 5\%} = 1.64$ eine Aussage zur Nullhypothese! Begründen Sie Ihre Antwort! \textbf{(2 Punkte)}
  \item Berechnen Sie den Effekt des Student t-Tests! \textbf{(1 Punkt)}
  \item Formulieren Sie eine Antwort an Alex über das Ergebnis Ihrer statistischen Analyse! \textbf{(2 Punkte)}
\end{enumerate} 
\clearpage
% -----------------------------------------------------------------------

\section{Aufgabe \hfill (11 Punkte)}

\textit{Geben Sie grundsätzlich Formeln und Rechenweg zur Lösung der Teilaufgaben mit an!} \\[1Ex]
 

 
%% --------------------------------------------------------------------
\ifcollection
\begin{flushright}
\tiny\vspace{-3Ex}
\textbf{\examinhaltstart}
\exammodulemathstat $\;\bullet$
\exammodulestat $\;\bullet$
\exammodulestatbbv 
\vspace{-4Ex}
\end{flushright}
\begin{minipage}[t]{0.5\textwidth}
\includegraphics[width = 1.3cm]{/Users/kruppajo/work/GitHub/exam/avatare/Alex.png}\hspace{-4mm}\includegraphics[width = 1.3cm]{/Users/kruppajo/work/GitHub/exam/avatare/Yuki.png}
\end{minipage}
\begin{minipage}[t]{0.5\textwidth}
\hfill
\href{https://youtu.be/wEePzcwwti8}{\includegraphics[width = 2cm]{img/youtube}}
\end{minipage}
\fi
%% --------------------------------------------------------------------



\ifcollection
\paragraph{Visualisierung der einfaktoriellen ANOVA}
\fi

Alex und Yuki schauen sich etwas entnervt an. Gemeinsam schreiben die beiden ihre Abschlussarbeit und sollen nun als erstes einmal die Daten visualisieren damit abgeschätzt werden kann, ob überhaupt signifikante Ergebnisse zu erwarten sind. Die beiden waren in der Uckermark um einen Versuch in einer Klimakammer mit Maiss durchzuführen. Dabei haben Alex und Yuki den Messwert Chlorophyllgehalt (SPAD-502Plus) [SPAD] unter der Behandung Lichtstufen ($none$, $200lm$ und $600lm$) ermittelt. Kennengelernt haben sich die beiden auf einem Konzert von London Grammar. Später wird noch Matrix geguckt. Yuki befürwortet das!

\begin{knitrout}
\definecolor{shadecolor}{rgb}{0.969, 0.969, 0.969}\color{fgcolor}\begin{table}[!h]
\centering
\begin{tabular}{cc}
\toprule
Lichtstufen & Chlorophyllgehalt\\
\midrule
200lm & 41\\
200lm & 39\\
none & 38\\
none & 37\\
600lm & 41\\
\addlinespace
200lm & 40\\
200lm & 40\\
200lm & 40\\
none & 40\\
600lm & 42\\
\addlinespace
600lm & 39\\
600lm & 40\\
600lm & 40\\
200lm & 40\\
none & 40\\
\addlinespace
none & 41\\
none & 40\\
none & 39\\
\bottomrule
\end{tabular}
\end{table}

\end{knitrout}

Leider kennen sich Alex und Yuki mit Darstellung einer einfaktoriellen ANOVA überhaupt nicht aus. 

\begin{enumerate}
\item Erstellen  Sie  eine  Visualisierung  der  Datentabelle! Beschriften  Sie  die  Abbildung! \textbf{(2 Punkte)}
\item Benennen Sie die Visualisierung mit dem korrekten, statistischen Fachbegriff! \textbf{(1 Punkt)}
\item Zeichnen Sie folgende statistischen Maßzahlen passend ein! 
  \begin{itemize}
  \item Den globalen Mittelwert $\beta_0$ \textbf{(1 Punkt)}
  \item Die Mittelwerte der einzelnen Behandlungsstufen \textbf{(1 Punkt)}
  \item Die Mittelwertsdifferenz der einzelnen Behandlungsstufen mit $\beta_{none}$, $\beta_{200lm}$ und $\beta_{600lm}$ \textbf{(1 Punkt)}
  \item Die Residuen oder Fehler mit $\epsilon$ \textbf{(1 Punkt)}
  \end{itemize}
\item Liegt ein \textit{vermutlicher} signifikanter Unterschied vor? Begründen Sie Ihre Antwort! \textbf{(2 Punkte)}
\item Schätzen Sie die Effekte der Behandlungsstufen! \textbf{(2 Punkte)}
\end{enumerate}
 
\clearpage
% -----------------------------------------------------------------------

\section{Aufgabe \hfill (9 Punkte)}

\textit{Geben Sie grundsätzlich Formeln und Rechenweg zur Lösung der Teilaufgaben mit an!} \\[1Ex]
 

 
%% --------------------------------------------------------------------
\ifcollection
\begin{flushright}
\tiny\vspace{-3Ex}
\textbf{\examinhaltstart}
\exammodulemathstat $\;\bullet$
\exammodulestat $\;\bullet$
\exammodulelanddaten $\;\bullet$
\exammodulestatbbv 
\vspace{-4Ex}
\end{flushright}
\begin{minipage}[t]{0.5\textwidth}
\includegraphics[width = 1.3cm]{/Users/kruppajo/work/GitHub/exam/avatare/Mark.png}
\end{minipage}
\begin{minipage}[t]{0.5\textwidth}
\hfill
\href{https://youtu.be/0xc0jIPeiyw}{\includegraphics[width = 2cm]{img/youtube}}
\end{minipage}
\vspace{-3ex}
\fi
%% --------------------------------------------------------------------



\ifcollection
\paragraph{Visualisierung des Boxplots}
\fi

Mark steht vor einem ersten Problem, denn wenn es nach seinem Betreuer geht, soll er in einem einem Feldexperiment Maiss auswertet. Soweit eigentlich alles passend. Besser wäre was anderes gewesen. Geocaching. Ein wunderbares Hobby um sich drin zu verlieren und Abstand zu bekommen. Mark denkt gerne über Geocaching nach. Die Behandlung waren verschiedene Lichtstufen ($none$ und $600lm$). In seiner Exceldatei hat er den Messwert ($Y$) \textit{Trockengewicht} als \textit{drymatter} aufgenommen. Nun soll Mark die Daten eimal als Boxplots in einer Präsentation visualisieren, damit seinem Betreuer wieder klar wird, was er eigentlich nochmal gemacht hat und was für ein Ergbnis in einem statistischen Test zu erwarten wäre. Anhand von Boxplots lässt sich eine Aussage über die Varianzhomogenität über die Behandlungsgruppen treffen. Wäre da nicht noch etwas. Eine echte Herausforderung für ihn war schon immer die Unsicherheit gewesen. Ein leidiges Lied. Aber egal. Mark will später nochmal raus um zu Reiten. Druck ablassen, dass muss er auch.

\begin{table}[!h]
\centering
\begin{tabular}{cc}
\toprule
treatment & drymatter\\
\midrule
600lm & 28.2\\
600lm & 13.1\\
none & 31.5\\
none & 29.2\\
none & 20.0\\
\addlinespace
600lm & 38.7\\
none & 30.1\\
600lm & 24.4\\
none & 31.7\\
600lm & 36.3\\
\addlinespace
none & 34.9\\
600lm & 20.3\\
none & 15.9\\
600lm & 10.8\\
none & 34.1\\
\addlinespace
none & 31.9\\
\bottomrule
\end{tabular}
\end{table}



Leider kennt sich Mark mit der Erstellung von Boxplots nicht aus. Deshalb braucht er bei der Visualisierung Ihre Hilfe!

\begin{enumerate}
\item Zeichnen Sie in \textit{einer} Abbildung die beiden Boxplots für die zwei Behandlungen von Maiss! Beschriften Sie die Achsen entsprechend! \textbf{(5 Punkte)} 
\item Wie ist Ihr Vorgehen, wenn Sie eine \textit{gerade} Anzahl an
  Beobachtungen pro Gruppe haben? \textbf{(1 Punkt)}
\item Beschriften Sie \textit{einen} der beiden Boxplots mit den gängigen
  statistischen Maßzahlen! \textbf{(2 Punkte)}
\item Wenn Sie \underline{keinen} Effekt zwischen den Behandlungen von Maiss erwarten würden, wie sehen dann die beiden Boxplots aus? \textit{Antworten Sie mit einer Skizze der Boxplots!} \textbf{(1 Punkt)}
\end{enumerate} 
\clearpage
% -----------------------------------------------------------------------

\section{Aufgabe \hfill (10 Punkte)}

\textit{Geben Sie grundsätzlich Formeln und Rechenweg zur Lösung der Teilaufgaben mit an!} \\[1Ex]
 

 
%% --------------------------------------------------------------------
\ifcollection
\begin{flushright}
\tiny\vspace{-3Ex}
\textbf{\examinhaltstart}
\exammodulemathstat $\;\bullet$
\exammodulestat $\;\bullet$
\exammodulestatbbv $\;\bullet$
\exammodulestatversuch $\;\bullet$
\exammodulebiostat
\vspace{-4Ex}
\end{flushright}
\begin{minipage}[t]{0.5\textwidth}
\includegraphics[width = 1.3cm]{/Users/kruppajo/work/GitHub/exam/avatare/Nilufar.png}\hspace{-4mm}\includegraphics[width = 1.3cm]{/Users/kruppajo/work/GitHub/exam/avatare/Steffen.png}\hspace{-4mm}\includegraphics[width = 1.3cm]{/Users/kruppajo/work/GitHub/exam/avatare/Tina.png}
\end{minipage}
\begin{minipage}[t]{0.5\textwidth}
\hfill
\href{https://youtu.be/K3LGigegKIg}{\includegraphics[width = 2cm]{img/youtube}}
\end{minipage}
\fi
%% --------------------------------------------------------------------



\ifcollection
\paragraph{Interpretation des t-Tests in \Rlogo - die Teststatistik und der p-Wert}
\fi

'Wir sind uns relativ sicher, dass unser Messwert Fettgehalt [\%/kg] ist!', ruft Steffen wild gestikulierend. Steffen wäre mehr präsent, wenn es die Romantik nicht gäbe. Als würde sowas die Ausgabe von \Rlogo interessieren. Steffen und Tina sind in einem Cafè mit Nilufar um sich Hilfe von ihr in \Rlogo zu holen. Während Nilufar Kirschstreuselkuchen und Takis Blue Heat mampft, versuchen die Steffen und Tina ihren Versuch in der Uckermark mit Schweinen in einem Stallexperiment zu erklären. Nilufar hofft insgeheim, dass die \Rlogo Ausgabe des t-Tests ihr mehr Informationen liefert. Eigentlich würde sie dann doch lieber raus um zu Kicken vielleicht mit Tina?

\begin{knitrout}
\definecolor{shadecolor}{rgb}{0.969, 0.969, 0.969}\color{fgcolor}\begin{kframe}
\begin{verbatim}
## 
## 	Two Sample t-test
## 
## data:  Fettgehalt by Bestandsdichte
## t = -0.74325, df = 15, p-value = 0.4688
## alternative hypothesis: true  is not equal to [condensed]
## 95 percent confidence interval:
##  -14.520079   7.011746
## sample estimates:
## mean in group Verordnung mean in group Gesteigert 
##                 25.31250                 29.06667
\end{verbatim}
\end{kframe}
\end{knitrout}

Helfen Sie Nilufar bei der Interpretation des t-Tests! Sonst geht es auch für Steffen und Tina nicht weiter.
  
\begin{enumerate}
  \item Formulieren Sie die wissenschaftliche Fragestellung! \textbf{(1 Punkt)}
  \item Formulieren Sie das statistische Hypothesenpaar! \textbf{(1 Punkt)}
\item Liegt ein signifikanter Unterschied zwischen den Gruppen vor? Begründen Sie Ihre Antwort! \textbf{(2 Punkte)}
\item Berechnen Sie den Effekt des t-Tests! \textbf{(1 Punkt)}
\item Skizzieren Sie eine Abbildung in der Sie $T_{D}$, $Pr(D|H_0)$, $A=0.95$, sowie $T_{\alpha=5\%} = |2.13|$ einzeichnen! \textbf{(4 Punkte)}
\item Beschriften Sie die Abbildung! \textbf{(1 Punkt)}  
\end{enumerate} 
\clearpage
% -----------------------------------------------------------------------
\end{document}
% -----------------------------------------------------------------------


  
